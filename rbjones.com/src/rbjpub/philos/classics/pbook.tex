% $Id: pbook.tex,v 1.3 2013/03/01 12:26:30 rbj Exp $
% Change the next line before cvs checkin prior to an upload to
% CreateSpace to get up-to-date Id in book. !!

%\usepackage{titlesec}
%\titleformat{\chapter}{\bf\large}{\thechapter}{1em}{}

\tolerance=10000

\usepackage[T1]{fontenc}
\usepackage{textcomp}
\renewcommand{\rmdefault}{ppl}
\linespread{1.04}

% this is to make section headings (which are Aristotle's ``Part''s) raggedright

\makeatletter
    \renewcommand\section{\@startsection{section}{1}{\z@}%
                                      {-3.5ex \@plus -1ex \@minus -.2ex}%
                                      {2.3ex \@plus.2ex}%
                                      {\raggedright\normalfont\large\bfseries}}
\makeatother

\usepackage[greek,english]{babel}

\usepackage{titlesec}

% This gets the chapter section and subsection titles a bit smaller.

\titleformat{\chapter}
  {\raggedright\normalfont\huge\bfseries}{\thechapter}{20pt}{\LARGE}

\titleformat{\section}
  {\raggedright\normalfont\LARGE\bfseries}{\thesection}{18pt}{\LARGE}

\titleformat{\subsection}
  {\raggedright\normalfont\large\bfseries}{\thesubsection}{12pt}{\large}

\usepackage{fancyhdr}
\pagestyle{fancyplain}

% The following is to suppress headers and footers on blank pages.

\makeatletter
\def\cleardoublepage{\clearpage\if@twoside \ifodd\c@page\else
\hbox{}
\vspace*{\fill}
\begin{center}
\end{center}
\vspace{\fill}
\thispagestyle{empty}
\newpage
\if@twocolumn\hbox{}\newpage\fi\fi\fi}
\makeatother

\fancyhfoffset[EL,RO]{0pt}
\newcommand{\bookname}{}
\newcommand{\bookletter}{}
\newcommand{\aref}{}
\newcommand{\RbJsref}{}
\newcommand{\volumename}{}

\renewcommand{\sectionmark}[1]{\markboth{\bookletter\ \ #1}{}}
\renewcommand{\subsectionmark}[1]{\markright{#1}}
\lhead[\fancyplain{}{\thepage}]         {\fancyplain{}{}}
\chead[\fancyplain{}{\slshape\leftmark}]                 {\fancyplain{}{\slshape\rightmark}}
\rhead[\fancyplain{}{}]       {\fancyplain{}{\thepage}}
\lfoot[\fancyplain{}{\aref}]            {\fancyplain{}{}}
\cfoot[\fancyplain{}{}]                 {\fancyplain{}{}}
\rfoot[\fancyplain{}{}]                 {\fancyplain{}{\volumename}}

\renewcommand{\headrulewidth}{0pt}

\renewcommand{\chaptername}{}

%\pagestyle{headings}
\usepackage[twoside,paperwidth=6in,paperheight=9in,hmargin={0.875in,0.5in},vmargin={0.5in,0.5in},includehead,includefoot]{geometry}
\usepackage{tocloft}
\usepackage{tocbibind}

\makeindex
\newcommand{\indexentry}[2]{\item #1 #2}
\newcommand{\ignore}[1]{}

\newcommand{\Avolume}[1]{\chapter{#1}\renewcommand{\volumename}{#1}\setcounter{subsection}{0}}
%\newcommand{\ASbook}[1]{\section{#1}}
%\newcommand{\AMbook}[2]{\section{#2}}

%\newcommand{\Avolume}[1]{}
\newcommand{\ASbook}[2]{\renewcommand{\bookletter}{#1}\renewcommand{\thesection}{Book #1}\renewcommand{\bookname}{#2}\section{#1}}
\newcommand{\AMbook}[2]{\renewcommand{\bookletter}{#1}\renewcommand{\thesection}{Book #1}\renewcommand{\bookname}{#2}\section{#2}}

\newcounter{partcount}

\newcommand{\AMsection}[2]{%
\renewcommand{\thesection}{Section \Roman{section}}%
\setcounter{partcount}{\value{subsection}}\section{#2}\setcounter{subsection}{\value{partcount}}}

\newcommand{\Apart}[1]{\subsection{#1}\label{\RbJsref}}
%\newcommand{\Apart}[1]{\section{#1}}

\newcommand{\RbjAlpha}{A}
\newcommand{\Rbjalpha}{\ensuremath{\alpha}}
\newcommand{\RbjBeta}{B}
\newcommand{\RbjGamma}{\ensuremath{\Gamma}}
\newcommand{\RbjDelta}{\ensuremath{\Delta}}
\newcommand{\RbjEpsilon}{E}
\newcommand{\RbjZeta}{Z}
\newcommand{\RbjTheta}{\ensuremath{\Theta}}
\newcommand{\RbjEta}{H}
\newcommand{\RbjIota}{I}
\newcommand{\RbjKappa}{K}
\newcommand{\RbjLambda}{\ensuremath{\Lambda}}
\newcommand{\RbjMu}{M}
\newcommand{\RbjNu}{N}

\newcommand{\dq}{\texttt{"}}


\title{Physics and Metaphysics}
\author{Aristotle}
\date{\ }

\begin{document}

%\renewcommand{\thechapter}{Book \arabic{chapter}}
%\renewcommand{\thesection}{Part \arabic{section}}
\renewcommand{\thechapter}{Volume \arabic{chapter}}
\renewcommand{\thesection}{Book \Roman{section}}
\renewcommand{\thesubsection}{Part \arabic{subsection}}

\addtolength{\cftchapnumwidth}{3em}
\addtolength{\cftsecnumwidth}{2.5em}
\addtolength{\cftsubsecnumwidth}{0.5em}
\cftsetpnumwidth{2em}

\frontmatter

\begin{titlepage}
\maketitle

\hspace{2in}

\vfill

\begin{centering}

\vspace{0.2in}

Edited by Roger Bishop Jones\\
www.rbjones.com\\

\vspace{0.2in}

ISBN-10: \\
ISBN-13: 

\vspace{0.2in}

{\footnotesize

$ $Revision: 1.3 $~$Date: 2013/03/01 12:26:30 $ $

}%footnotesize

\end{centering}

\thispagestyle{empty}
\end{titlepage}

%\renewcommand{\cfttoctitlefont}{\Large}
%\renewcommand{\cftlottitlefont}{\Large}

{\parskip=0pt\tableofcontents}
%\addcontentsline{toc}{section}{Contents}
%{\parskip=0pt\listoftables}
%\addcontentsline{toc}{section}{List of Tables}
\vfill

\renewcommand{\aref}{Book \arabic{chapter} Part \arabic{section}}

\pagebreak

\chapter*{Preface}
\addcontentsline{toc}{chapter}{Preface}

This bare-bones English language edition of Aristotle's Physics and Metaphysics was prepared for my own use but is now made generally available.

It is based on a translations by R. P. Hardie and R. K. Gaye (of the Physics) and by Sir David Ross (of the Metaphysics), all now in the public domain, which were transformed into a form suitable for typesetting using {\LaTeX}.
In the process, titles for each book and part and an extensive index were added.

\vspace{0.5in} 

Roger Bishop Jones

\vspace{0.05in} 

February 2013

\vfill

\mainmatter

\renewcommand{\aref}{Volume \arabic{chapter} Book \Roman{section} Part \arabic{subsection}}
\renewcommand{\RbJsref}{\arabic{chapter}\Roman{section}\arabic{subsection}}

\input{pi.tex}

\backmatter
%\bibliographystyle{alpha}
%\bibliography{rbj}

%\clearpage
%\listoftables
%\addcontentsline{toc}{section}{List of Tables}

%\clearpage
%\listofposition
%\addcontentsline{toc}{chapter}{List of Positions}

%\twocolumn[
%\section*{Index}\label{index}
%\addcontentsline{toc}{section}{Index}
%]
%{\small

%οὗ καταλύεις

\renewcommand{\aref}{}
\renewcommand{\bookname}{}
\renewcommand{\chaptermark}[1]{}
\renewcommand{\sectionmark}[1]{}

\printindex
%}

\vfil

\end{document}

