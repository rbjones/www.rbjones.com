% $Id: t041a.tex,v 1.4 2011/10/25 09:10:46 rbj Exp $

\documentclass[11pt]{article}
\usepackage{latexsym}
\usepackage{rbj}

\ftlinepenalty=9999
\usepackage{A4}

% the following two modal operators come from the amsfonts package
\def\PrKI{\Diamond}	%Modify printing for \250
\def\PrKJ{\Box}		%Modify printing for \251

\def\PrJA{\|-}		%Modify printing for  (syntactic entailment)
\def\PrJI{\models}	%Modify printing for ˜ (semantic entailment)

\def\ouml{\"o}

\tabstop=0.4in
\newcommand{\ignore}[1]{}

\def\thyref#1{Appendix \ref{#1}}

%\def\ExpName{\mbox{{\sf exp}}}
%\def\Exp#1{\ExpName(#1)}


\title{An Illative Lambda-Calculus}
\makeindex
\usepackage[unicode,pdftex]{hyperref}
\hypersetup{pdfauthor={Roger Bishop Jones}, pdffitwindow=false}
\hypersetup{colorlinks=true, urlcolor=red, citecolor=blue, filecolor=blue, linkcolor=blue}
\author{Roger Bishop Jones}
\date{\ }

\begin{document}
\begin{titlepage}
\maketitle
\begin{abstract}
This is an approach to illative lambda-calculi via construction of an infinitary calculus in a well-founded set theory.
\end{abstract}
\vfill

\begin{centering}
{\footnotesize

Created 2010/09/07

\input{t041i.tex}

\href{http://www.rbjones.com/rbjpub/pp/doc/t041.pdf}
{http://www.rbjones.com/rbjpub/pp/doc/t041.pdf}

\copyright\ Roger Bishop Jones; Licenced under Gnu LGPL

}%footnotesize
\end{centering}

\thispagestyle{empty}
\end{titlepage}

\newpage
\addtocounter{page}{1}
{\parskip=0pt\tableofcontents}

\section{Prelude}

This document is one of a series in which I explore approaches to non-well-founded foundations for mathematics.
The immediately preceding document in this series is \cite{rbjt027} which considers an idea for the construction of non-well-founded interpretations of set theory.
It was my intention in that document ultimately to apply the results in giving a semantics to an illative lambda-calculus.
When I last considered resuming the work in \cite{rbjt027} I concluded that it might be better to go directly to the desired lambda-calculus omitting the non-well-founded set theory, by adapting the methods of that paper to a functional rather than a set theoretic ontology.
This had not been possible in the first place, because I did not have clear enough idea of how it could be done.
The methods I am employing are more easily understood in the first instance through set theory (at least, it was in that context that I came to them).
With the benefit of having explored the techniques in the context of set theory, I now am able to see how they might be applied directly to a lambda-calculus, and in this document I make the attempt.

Discussion of what might become of this document in the future may be found the postscript (Section \ref{POSTSCRIPT}).

In this document, phrases in coloured text are hyperlinks, like on a web page, which will usually get you to another part of this document (the blue parts, the contents list, page numbers in the Index) but sometimes take you (the red bits) somewhere altogether different (if you happen to be online), e.g.: \href{http://rbjones.com/rbjpub/pp/doc/t041.pdf}{the online copy of this document}.

\cite{rbjt000}

\section{Introduction}
