% $Id: t038a.tex,v 1.3 2011/02/12 09:14:19 rbj Exp $

\documentclass[11pt]{article}
\usepackage{latexsym}
\usepackage{ProofPower}
%\usepackage{amsfonts}

\ftlinepenalty=9999
\usepackage{A4}

% the following two modal operators come from the amsfonts package
\def\PrKI{\Diamond}	%Modify printing for \250
\def\PrKJ{\Box}		%Modify printing for \251

\def\PrJA{\|-}		%Modify printing for  (syntactic entailment)
\def\PrJI{\models}	%Modify printing for ˜ (semantic entailment)

\tabstop=0.4in
\newcommand{\ignore}[1]{}

\def\thyref#1{Appendix \ref{#1}}

%\def\ExpName{\mbox{{\sf exp}}}
%\def\Exp#1{\ExpName(#1)}


\title{Higher Order Logic}
\makeindex
\usepackage[unicode,pdftex]{hyperref}
\hypersetup{pdfauthor={Roger Bishop Jones}, pdffitwindow=false}
\hypersetup{colorlinks=true, urlcolor=red, citecolor=blue, filecolor=blue, linkcolor=blue}
\author{Roger Bishop Jones}
\date{\ }

\begin{document}
\begin{titlepage}
\maketitle
\begin{abstract}
At present this document is a (small) mess pot of explorations of how one might go about presentation of Church's Simple Theory of Types and ultimately ProofPower HOL using Standard ML and/or ProofPower HOL.
I am interested both in exposing exactly what Church said, comparing the details of his system with those of ProofPower HOL and discussing the reasons for the differences, but also I am looking for an interesting and digestible way of presenting ProofPower HOL to philosophers or other groups without much IT background.
Its doubtful that all these objectives are compatible, and so far my attempts have not been in the least impressive, but I probably will keep picking at it and may eventually come up with something useful.
\end{abstract}
\vfill

\begin{centering}
{\footnotesize

Created 2010/07/16

Last Change $ $Date: 2011/02/12 09:14:19 $ $

\href{http://www.rbjones.com/rbjpub/pp/doc/t038.pdf}
{http://www.rbjones.com/rbjpub/pp/doc/t038.pdf}

$ $Id: t038a.tex,v 1.3 2011/02/12 09:14:19 rbj Exp $ $

\copyright\ Roger Bishop Jones; Licenced under Gnu LGPL

}%footnotesize
\end{centering}

\thispagestyle{empty}
\end{titlepage}

\newpage
\addtocounter{page}{1}
{\parskip=0pt\tableofcontents}

\section{Prelude}

This document is intended possibly to form a chapter of {\it Analyses of Analysis} \cite{rbjb001}.
My initial purpose in preparing the document is simply to present in detail the logical system which is used throughout that work.

Further discussion of what might become of this document in the future may be found in my postscript (Section \ref{POSTSCRIPT}).

In this document, phrases in coloured text are hyperlinks, like on a web page, which will usually get you to another part of this document (the blue parts, the contents list, page numbers in the Index) but sometimes take you (the red bits) somewhere altogether different (if you happen to be online) like \href{http://rbjones.com/pipermail/hist-analytic_rbjones.com}{the hist-analytic archives}.

\section{Introduction}
