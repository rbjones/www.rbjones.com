% $Id: anan.tex,v 1.3 2012/03/23 15:41:40 rbj Exp $

\chapter{Analyticity and Analysis}\label{AnalyticityAnalysis}

{\it[
Having established the distinction between logical and empirical
truths we now turn to a closer consideration of logical truth. 

There are two main considerations.
One is the scope and relevance of logical truth.
This should include a discussion of its place in analytic philosophy,
in mathematics, empirical science, and engineering. 
A principle aim in this discussion is both to re-emphasize, as Hume
did, the severe limits to what can be established purely by deduction,
and to show that it nevertheless is of the greatest practical
significance. 
The limitations and potential are to be made real by illustrations of
a different character to those of Hume but which are connected with
the various concerns of my own positive philosophical thinking,
theoretical and practical. 
Specifically in relation to philosophy, some discussion of the nature
of philosophical analysis making the distinction between a claim being
analytic and a claim being ``about language'', i.e. between subject
matter and epistemic status. 

I would like to introduce here the notion of an analytic oracle, which
can then be refined in the next chapter to the FAn oracle.

]}%it

We have seen that the \emph{words} ``analytic''\index{analytic} and
``synthetic''\index{synthetic} acquired in the philosophy of Kant a
sense distinct both from their use outside academic philosophy and
from previous philosophical and mathematical usage.

In non-philosophical use the related terms
\emph{analysis}\index{analysis} and \emph{synthesis}\index{synthesis}
have diverse application, generally concerned with \emph{taking apart}
or \emph{putting together} the parts of some complex whole. 
In the special domain of logical proof, the usage in classical Greece
was similar, connoting two \emph{methods of proof}.
An analytic proof proceeded by analysis of the proposition to be
proven, ultimately reducing it to principles which can be known
without proof.
A synthetic proof begins instead with axiomatic principles reasoning
forward until the desired theorem is eventually reached.
In contemporary automation of reason, these different methods, which
we will consider further in a later chapter, are sometimes known as
backwards and forwards proof respectively. 

Kant\index{Kant} introduced a new usage in which analytic and
synthetic are applied to propositions (in the Aristotelian sense),
classifying them along the lines of Hume's fork. 
Kant gave two criteria for this classification, one proof theoretic
(concerning the manner in which such a proposition might be
established), and the second \emph{semantic} concerned with meanings. 
However characterized, analyticity in this sense is closely connected
with deductive reason, and it is our purpose in this chapter to relate
this precise technical concept with the very general notion of
analysis, both in its academic and its more worldly applications. 

\section{Abstract Logical Analysis}

Though we are concerned here with analysis \emph{in general}, there is
one particular kind of analysis which will have special place.
This kind of analysis is closely coupled with the notion of
analyticity through the concept of deductive soundness.

The notion of soundness is applicable both to formal logical systems
which are supplied both with a \emph{semantics} (an account of the
meanings of the sentences of the language) and a formal notion of
derivation or proof, or to particular inferences in any language for
which the semantics is well-defined.
In the former case the system is sound if the inference rules respect
the semantics in such a manner that from true propositions only true
conclusions can be derived.
In the latter case, without reference to any particular deductive
system we may say that an inference is sound if the premises
\emph{entail} the conclusion.
A set of sentences (the premises) entails some other sentence (the
conclusion) if under all circumstances whenever the premises are true
the conclusion will also be true. 

The connection with analyticity is then that all claims about
entailments are true if and only if analytic.
An alternative statement of the connection is that all propositions
derivable from the empty set of premises, the theorems, of a sound
deductive system are analytic.

Because of this connection, Rudolf Carnap and the logical positivist
held that philosophy, which they thought of as an \emph{a priori}
science consisted of analytic truths.
This involves a position on the demarcation of philosophy which we
will not adopt here.
Instead we simply note the scope of this particular \emph{kind} of
analysis, and of the kind of philosophy which employs it.
Philosophy falling within the scope of such methods can now be made as
rigorous and reliable as mathematics by the use of modern formal
languages and computer software which assists in the construction of
formal proofs and automatically checks that the proofs are correct.

Abstract logical analysis consists in the application of deductive
reasoning to analysis of arguments, of concepts, of theories or
doctrines. 
The application to arguments may be considered to be primary, and in
this case the idea is that in any domain in which reason is thought to
be applicable, modern logical methods can be used to improve the
rigour of the reasoning. 
The idea here is much as it was in Plato, it is that the standards of
rigour which are generally found in mathematics and generally lacking
elsewhere, can be made more widely available by appropriate methods. 
In particular, by focussing on the concepts involved, by ensuring that
we have clear definitions of these concepts, we can reason within the
relevant domain rigorously.

Plato, Aristotle, and the many logicians who followed them until very
recent times, failed to realize this ideal of logically rigorous
reasoning beyond mathematics.
The greater difficulty in pinning down non-mathematical concepts may
have been a factor here, and it is not until the $19^{th}$ century
that logic was progressed to the point at which the logic itself could
be used in making precise definitions from which one can reason with
formal rigour. 

\section{}

Other points of controversy important to this project remain,
concerned with the scope, applicability and importance of analytic
truth.
These are important to us here for two distinct kinds of reason.

The theoretical core of Positive Philosophy, \emph{Metaphysical
  Positivism}, is primarily an \emph{analytic} philosophy, and it is
essential in presenting a conception of analytic philosophy to address
some of the reasons why a purely analytic philosophy might be thought
to be of narrow scope and limited value.

The project we are considering is of broader scope, just as were the
projects of Leibniz and Carnap, encompassing not merely an approach to
philosophical analysis, but also important and substantial parts of
mathematics, empirical science, engineering and other activities in
which deductive reasoning might play a significant role.

There is a circle to be squared here.
There is one perspective from which the entire enterprise is without
content, and represents a preoccupation with academic trivia, and
another diametrically opposed perspective in which it is of the
greatest practical significance and may be thought to warrant
substantial and energetic prosecution.

\section{The Scope of Analytic Truth}

I have presented a history of the evolution of the concept of
analyticity and related concepts, and have adopted in Metaphysical
Positivism a conception which is similar to that described by Hume
``relations between ideas''.
A good recent account of this concept may be found in the writings of
Rudolf Carnap, but more importantly analyticity is a characteristic of
the theorems of a large class of modern tools for constructing and
reasoning in formal logical theories.
Further sharpening remains of interest from a philosophical point of
view, and will be discussed later, but for practical purposes, even
its relevance to applications demanding the very highest standards of
rigour, the concept and our technologies for checking when it applies
are sufficient.

It remains a matter of controversy how significant analyticity,
analytic truths, and methods in which they figure prominently are or
might be.

Analytic truths may be thought of as falling into two principal
groups.

The first group consists of true claims in languages whose subject
matter is entirely ``abstract''.
For this we understand abstract entities as mere ideas, about which we
reason by offering a ``definition'' of the relevant domain of entities
and then reasoning logically from the definition to conclusions about
that domain which will be true \emph{by definition}.
This group encompasses the whole of mathematics.
It also encompasses reasoning about abstract models of any domain
whatever, whether or not one considers the model to be
``mathematical'', so long as the model is well-defined and the method
is deductive.

This is a \emph{Platonic} conception of the scope of deductive
knowledge, and can be broadened very broadly in the direction which
Aristotle took, to yield a body of analytic truths which might be
thought to be \emph{about} the material world rather than purely
concerned with abstract entities.
There are several ways in which this can be done in the context of
modern logic without embracing the complexities of Aristotelian
metaphysics.

Carnap's approach is to adopt formal languages whose subject matter is
the material world, to define the semantics of the languages by giving
the truth conditions, and to define analyticity in terms or such a
truth conditional semantics.
The analytic truths are then those sentences in such languages whose
truth conditions are invariably satisfied.

\section{Analytic Philosophy}

The term \emph{analytic philosophy} was first applied in the $20^th$
century to a kind of philosophy which began at the turn of the century
as Bertrand Russell \index{Russell, Bertrand} and G.E.Moore
\index{Moore, G.E.}.
These two men, though sharing the idea that philosophy should be in
some sense analytic, had quite different conceptions of the kind of
analysis involved, and their differences remained significant as
methods of analysis evolved throughout the $20^{th}$ century.

Russell conception of analysis was shaped by the new methods in logic
in the development of which he played an important part.
He believed that failures of rigour in philosophical reasoning
resulted in some cases from imperfections in ordinary language, and
could be avoided if a special ideal language were adopted along the
lines of the Theory of Types which he devised with A.N. Whitehead
\index{Whitehead, A.N.} for the formalization of mathematics in
Principia Mathematica.
He did not advocate or practice the adoption of such formality in
philosophical rather than mathematical reasoning, but did advocate the
adoption of similar logical methods.
An example the kind of method he envisaged is the use of logical
constructions.
By such means Russell advocated a ``scientific'' philosophy in which
logical methods transformed philosophy into a rigorous deductive
discipline progressively advancing in a manner similar to that of
mathematics and science, rather a perpetual sequence of conflicting
theories and a lack of solid progress.
We may recall Plato's prescription here, in which the subject matter
of philosophy and the only place where true knowledge might be
attained is in the world of ideal forms.
Even though Russell was an empiricist, his conception of philosophy is
as a deductive discipline.

On the other hand Moore's conception of analysis concerned natural
languages and consisted in the clarification of
concepts in those languages, yielding illumination consistent with
common sense.
There is a connection here with Socratic method, but neither Socrates
nor Plato conferred such authority upon ordinary language and common
sense.
Socrates sought the true nature of concepts such as justice and
virtue, and though he believed that ordinary men were in some sense
possessed of these true concepts, they nevertheless might not correctly
grasp them until they are induced by a Socratic dialogue to properly
recall this knowledge.

Is it the case that analytic philosophy in the $20^{th}$ century was
narrower in its scope than philosophical tradition from which it grew,
and if so is this a necessary consequence of the conception of
philosophy as analytic, or of some particular ideas of what kind of
analysis was at stake.

Two mid century conceptions of the nature of analytic philosophy
illustrate the issue, those of Rudolf Carnap, exhibited in his
Philosophy of Logical Syntax and the subsequent developments of his
ideas, and those of the ``linguistic philosophy'', exemplified by
J.L.Austin, which prevailed briefly in post war Oxford.

A central thesis of Carnap's philosophy of logical syntax was that
philosophy consists of logical analysis and yields results insofar as
they are established truths, which are themselves analytic.
This apparent narrowing of scope is confirmed by explicit exclusion of
fields such as ethics, not only from the domain of analytic
philosophy, but from scientific discourse altogether (bearing in mind
that this kind of philosophy is itself conceived of as scientific in
character, though not empirical).
The severity of the apparent narrowing here is mitigated by the
predominance in the actual philosophy of Rudolf Carnap of work which
is methodological, including work which ensues in some proposal for
the use of language.
Though philosophy has relinquished any claim to offer factual
enlightenment about the world, it may nevertheless make material
contributions to the advancement of such knowledge by contributions to
the methods of science.

Linguistic philosophy in its most extreme form conceives of philosophy
as the analysis of natural language.


\section{Analyticity in Science}
