% $Id: regress.tex,v 1.6 2015/04/23 09:58:08 rbj Exp $

\chapter{Foundationalisms}\label{Foundationalisms}

\section{Foundations for Knowledge}

How are we to judge claims to scientific knowledge?

In metaphysical philosophy and natural philosophy there have been
ideas on this topic which may be called foundational.
In this chapter we present some ideas along these lines.

Principally this concerns foundations for logical and metaphysical
knowledge, knowledge \emph{a priori}, but I will touch upon
foundational aspects of empirical knowledge.

Before entering into a positive account of foundations I want to say a
few words about the role which such foundations are intended here to
fulfill.

It may be useful to draw an analogy with the use of the term
`foundation' in the construction of buildings.
In the context a foundation provides a base solid enough for the
construction of the desired building, so that the building will
stand firm and will survive the stresses to which it may reasonably be
expected to be subject.

For this a foundation does not need to be \emph{absolutely} solid.

The construction of a foundation does not itself proceed in the same
way as that for the building.
One does not, in order to obtain a solid foundation, seek a yet more
solid foundation on which to build the foundation (though sometimes
bedrock serves this function).
There does not arise in this way, an unsolvable problem of regress in
the foundations of buildings.

There are two reasons


\section{Logical Foundations}

In keeping with the positivist tendency to which it belongs, our
account of metaphysical positivism has been concerned primarily with
underpinning and articulating methods and tools suitable not only for
rigorous philosophical reasoning but for application in science and
engineering.

In Aristotle's conception of first philosophy, utility is regarded with
some scorn, and the insistence of positivists that philosophy should
facilitate positive science leads to the idea that positivism is an
\emph{anti-philosophical} philosophy (which is consistent with seeing
it as continuous with academic and pyrrhonean scepticism).

Metaphysics is the name by which those topics at the apex of
philosophy as conceived by Aristotle is now known, and the name
``metaphysical positivism'' may therefore be read as hinting that the
pragmatic orientation of our positivism does not involve a rejection
of those more remote regions of philosophy whose connection with life
seems most tenuous.

Nevertheless, in metaphysical positivism, locating a place for
metaphysical investigation is not easy.
Two kinds of defect which may be found in metaphysical (and other
controversy) from a positivistic standpoint are meaningless claims and
purely verbal disputes.
It is characteristic of positivist to reject metaphysics as
meaningless, and in metaphysical positivism I retain a concern for
precision and clarity in language, which motivates some of the deeper
concerns which we address here.
However, it is the business of philosophy to address problems whose
articulation is difficult, and that one philosopher does not find a
conjecture or a definition meaningful does not suffice to establish
that it is not.

In keeping with the graduated scepticism in metaphysical positivism
meaningfulness is not taken to be an all or nothing affair.
Languages (or idiolects) may be compared on two related kinds of scale.
First they may be compared according to their expressiveness.
A language A is as expressive as language B if everything which can be
said in language B can also be said in language A.
To compare precision of definiteness of a language we have to consider
languages as having multiple possible meanings or interpretations and
then compare the range of interpretations of two languages.

An easy and fundamental illustration of this kind of comparison may be
found in axiomatic set theory.
If we consider a specific theory. say ZFC\index{ZFC}, the axioms of
the theory provide an implicit definition of the concept of set which
is the subject matter of the theory.
The truth conditions of sentences of ZFC can be made very definite by stipulating
that a sentence is true in ZFC iff it is true in every model of the axioms.
Truth will then correspond to provability, in consequence of the
completeness of first order logic.
It is also reasonable in this domain to take the \emph{meaning} to be
the truth conditions, so that ZFC becomes as definite in its meaning
as first order logic is.
Unfortunately when we look at the intended applications of set theory,
of which the first is to the theory of arithmetic, we find that this
conception of the meaning of set theory is unsatisfactory.
The normal procedure in reasoning about arithmetic in set theory is to
define the natural numbers as some convenient countably infinite set
of representatives.
The most popular has been the scheme for representation of ordinal
numbers under which the natural number zero is represented by the
empty set, and every other natural number is represented by the set of
its predecessors, i.e. the set of all natural numbers which are less
than that number.

Using this definition, together with definitions of the usual
arithmetic operations over these representatives, we can derive the
usual theorems of arithmetic in ZFC.
More true theorems of arithmetic are provable in this way than is
possible in the usual direct axiomatization of arithmetic in first
order logic (known as PA, for Peano Arithmetic).
However, it is known, as a result related to the incompleteness
results proved by Kurt G\"odel\index{Kurt G\"odel} that not all the
truths of arithmetic are provable in this way.

However, because of the completeness of first order logic, all the
statements of arithmetic as expressed in the way indicated in ZFC
which are true under that semantics.
The arithmetic truths which are not provable in ZFC, are, under
that semantics, with that manner of representation of numbers, not
even true.
We have failed to produce an adequate definition of the natural
numbers.

This is not an avoidable defect in the Von Neumann\index{Von Neumann}
representation of ordinals.
It is easy to see that under the proposed semantics for ZFC the truths
of set theory will be recursively enumerable, and therefore any
decidable subset of those truths (the truths of set theory which
happen to correspond to sentences of first order arithmetic) will also
be effectively enumerable, whatever definition of natural number we
start out with.
But it is know that the truths of arithmetic are not recursively
enumerable.
It follows that the concept of natural number is not representable in
ZFC under the given semantics.

Though the semantics is definite, it is not expressive.

We can make the semantics more expressive, at the cost of making it
less definite, in the following way.
The semantics we have been discussing involves acceptance of all
models of ZFC, and its (semantic) incompleteness reflects the
existence of models of ZFC in which the set defined to be the natural
numbers is not what the definition is intended to give.

The definition of the natural numbers is intended to give a set whose
members are all the sets obtainable from the empty set by repeated
application of the successor function (the function s(x) = x+1).
The idea ``obtainable by repeated application'' cannot be directly
expressed in first order logic, so the definition instead is given in
terms of closure under the successor function.
A set is closed under the successor function if for every member of
the set, its successor is also a member.
The natural numbers are then defined as the intersection of all sets
which contain the empty set and are closed under the successor
function.

Unfortunately, if we take an interpretation in which the intended set
of natural numbers does not exist, in which every set which contains
all the natural numbers also contains some other set, then when you
take the intersection you get a set which contains that other set.
This is a model with non-standard natural numbers, and such an
interpretation will not get the truths of arithmetic right.
Because our semantics allows these non-standard models, as well as
models in which arithmetic is standard, statements of arithmetic which
are true in the standard models but which are violated in some
non-standard model will come out under the semantics as false.

To get the semantics on the nose for arithmetic statements we need to
eliminate these non-standard models from the semantics.
We cannot do this by adding another axiom, because the required
constraint on the models is not expressible in first order logic.
But we can add an informal stipulation to the semantics.
We can specify the truth condition for sentences in ZFC as truth in
all models of ZFC with standard natural numbers.

We now have a version of ZFC which is more expressive than the
previous one.
One in which the natural numbers really are definable (though not in
the sense of this term which is used by mathematical logicians) and in
which the sentences of arithmetic have the correct truth values.
Though the semantics of this language are defined in part informally,
the language can now be used to define the semantics of other
languages in a formal way (relatively), whose semantics would not be
definable in first order logic.

In this way we can define variants of the language of first order set
theory which have progressively greater expressive power.
This is done by using informal constraints on the class of intended
interpretations of the theory.
Such informal constraints can be placed in order of strength and the
stronger the constraint is the more expressive the resulting language
will be.

Beyond the constraint to models with standard natural numbers the
following stronger constraints can be applied:

\begin{itemize}
\item Constraining interpretations to be well-founded. 
\item Requiring full power-sets.
\end{itemize}

Constraining interpretations to be well-founded is strictly stronger
than requiring standard natural numbers.
It is as strong because in any model in which the set of natural numbers is
non-standard it is not well-founded.
It is strictly stronger because the same consideration applies to all
limit ordinals (all transfinite numbers).
In the absence of well-foundedness we can have models which give
standard natural numbers but have a non-standard ordinal somewhere
higher in the hierarchy.
Well-founded models not only have standard arithmetic, but they also
have standard ordinals all the way up.
So under this semantics, but not under the previous one, the ordinals
are definable.

Yet greater strength is obtained by requiring the full power set.
All models of ZFC are closed under the formation of power sets.
That means, that for every set in the domain of discourse the
collection of its subsets (in the domain of discourse) is also a set
in the domain of discourse.
However, it is not the case that the power set is the same in every
model, since not all subsets of the set are bound to be in the domain
of discourse.
The power set will include all the subsets which exist, but there may
be subsets which don't exist (in this particular model).
The constraint to full power sets eliminates any model in which there
are missing subsets.

If we define truth in ZFC as truth in every model of ZFC in which the
power set is full, then we get another language which is again
strictly more expressive than the ones we have previously considered.
Why is this?
To understand this it is helpful to address the question why there
exist non-well-founded models of ZFC.

There is an axiom in ZFC which is intended to assert that all sets are
well-founded.
This is called the axiom of regularity.
Its intended effect is to deny that there are any infinite descending
chains in the membership relationship, but this, like the obvious informal
definition of the natural numbers, cannot be directly stated
informally.
The axiom of regularity states instead that every set has a minimal
element.
A minimal element of a set A is a member B of A which contains no member of
A, such that A intersection B is empty.

This definition does ensure well foundedness if there are enough sets
in the domain of discourse, which is the case if we have full power
sets.
Otherwise it does not, and there are non-well founded models of ZFC
despite it having an axiom intended to deny their existence.
Well foundedness is not expressible in first order logic, and so
cannot be fully incorporated into the axioms, which is why the
previous semantics relies on an informal constraint to
well-foundedness.

If instead of requiring well-foundedness we stipulate the semantics in
terms of models with full power sets, then since these are all well
founded the resulting semantics is as expressive as the semantics
based on arbitrary well-founded models.

That the semantics is strictly stronger can be seen from consideration
of cardinal numbers.
Cardinal numbers may be represented in ZFC as \emph{initial ordinals}.
An initial ordinal is an ordinal which has a greater cardinality than
any previous ordinal.
Two sets have the same cardinality if there exists a bijection between
their elements.
Such a bijection pairs up the elements in the two sets in a one-one
manner so that they can be seen to have the same size.
A difficulty with this definition of cardinality arises from the
existence of models in which not all subsets of every set are
present, in which we do not have full power sets.
This is because the non-existence of a bijection might arise not
because the two sets really are of different size, but because the
bijection between their elements is just not in the domain of the
model.
The effect of this is that not all models of ZFC agree about which
ordinals are initial, and consequently they do not agree about
cardinal arithmetic.

Constraining the truth conditions to involve truth only in models with
full power sets eliminates this source of disagreement between
intended models of ZFC about cardinal arithmetic.
Under this semantics but not under any of the previous semantics we
can define the notion of cardinal number, so it gives a strictly more
expressive language.

\section{Empirical Foundations}

