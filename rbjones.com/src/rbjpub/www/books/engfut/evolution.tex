% $Id: evolution.tex,v 1.1 2007/01/24 15:32:53 rbj01 Exp $
\chapter{Evolution and History}\label{Evolution}

Philosophy has traditionally been an armchair science (if a science at all) in which brute facts about our world are used sparingly, and are of the kind which a well educated layperson might be acquainted with or would at least readily comprehend and accept.
Until this last century, a detailed knowledge even of prior philosophy has rarely been prerequisiste.

\section{Kinds of Evolution}

The word ``evolution'' covers a wide variety of phenomena.

The concept was placed center stage by Darwin's theory on the evolution of species.
Modern biology adds to Darwin's key idea of natural selection a detailed scientific model of the mechanisms of inheritance to give a genetic theory of evolution drawing conclusions for example about the kinds of genes which can possibly emerge from this process (e.g. that they must be ``selfish'' \cite{dawkinsSG}).
Extrapolating to an extreme, Daniel Dennett \cite{dennettDDI} talks about the evolution of the universe as a whole, moving the concept of evolution into a context in which neither genetics nor natural selection is applicable.

\section{Evolution and Philosophy}



