% $Id: prelude.tex,v 1.6 2010/04/05 16:05:16 rbj Exp $
\mainmatter
\def\rbjidprelude{$$Id: prelude.tex,v 1.6 2010/04/05 16:05:16 rbj Exp $$}

\chapter*{Prelude}\label{Prelude}
\addcontentsline{toc}{chapter}{Prelude}

This book is concerned with rationality.
It responds to sceptical doubts about the rationality of human reasoning and conduct.

The aim is to examine some areas of perceived irrationality with a view first to understanding its nature and causes, and secondly to considering what future amelioration might be possible.

The academic discipline of \emph{analytic philosophy} is intended to be a rational pursuit in itself, and is often thought to provide various kinds of underpinning for the rational conduct of other academic or practical pursuits.
However, its own standards of rationality have often been questioned.
This happens once again here.

The central critique appears in Part \ref{partIII} and concerns the conduct of analytic philsophers during the 20th Century, focussing on a small number of two significant episodes.
In order to make clear our view of these episodes we first examine two other topics.
First, in Part \ref{partI} we examine certain evidence from the theory of evolution which casts light on how the faculty of reason has evolved, for the sake of insights which might help to explain its modes of ``failure''.
Then, in Part \ref{partII} we supply some philosophical background for the problems which will be examined in Part \ref{partII}.
Finally, Part \ref{partIV} considers whether and how matters might be improved.

\ignore{

For those among us who believe in the possibility of progress, the progression from violence to more moderate means of resolving differences between men must be a hope to be nourished.
There have been many signs of such progress over the past few millenia, but no lack of contra-indications.

When we enter into a reasonable discussion, in the hope of resolving some difference which might otherwise escalate into violence, we need to have confidence that the process we submit to will take due regard of our concerns and interests, and will yield an equitable and fair result.

If this is not the case, then fewer will be motivated to adopt this course, and more lives will be blighted by violence.

It is the purpose of this book to contribute to progress by casting light on some of the ways in which reasonable discourse can fail, and suggesting some ideas that may reduce the likelyhood of such lapses.
Though these considerations are of relevance in all areas of humen endeavour, they are particularly important in \emph{analytic philosophy} of which important aims are the use of reason to illuminate matters of importance, and the presentation of methods of analysis which may be applied widely in the resolution of problems in other disciplines.
In this philosophy should offer a model of rational discourse for all to follow.

It is regrettable therefore to find striking examples of unreason among philosophers having a substantial effect on the direction, conduct and results of philosophical enquiries.

We are concerned here with some startling examples of such pathologies which took place in the 20th century, of which the severity has yet to be accepted, and whose roots causes remain untouched.
Evidence of these lapses forms a part of the account we give in Part III of this book of the course of analytic philosophy in the last century.

To understand how this can be, I have had resort to the theory of evolution, which has cast light for me in this dark corner.
My principle source of enlightenment has been the writings of Howard Bloom, particularly \emph{Global Brain} \cite{bloomBRAIN}, from which the larger part of my evolutionary material is drawn.
In Part \ref{partI} I tell an evolutionary story the purpose of which is to delineate certain elements of human social behiour which seem helpful in understanding the history which follows.
The first part of that history comes in Part \ref{partII} and covers aspects of the history of philosophy and of mathematics which provide the necessary context to understand the particular pathologies in 20th Century philosophy which we then examine in Part \ref{partIII}.
This historical material also underpins the ideas which are presented in Part \ref{partIV} on how matters might conceivably develop from here in.

}


