% $Id: history.tex,v 1.1 2012/01/23 21:40:02 rbj Exp $

\chapter{Historical Themes}

The practical side of Positive philosophy is forward looking, seeking
an understanding of morality, politics and economics relevant to a
future in which the human genome might be evolved \emph{by design}
and in which intelligence is manifest in inanimate or hybrid machines
and networks.

It is desirable in considering such profound changes to connect our
thinking with the history of ideas, so that our projections about
future possibilities are grounded in reflections on past actualities
and ideas.

In this chapter some such connections are sought by tracing the
history of some of the themes in terms of which our future options are
to be analysed.

\ignore{
By the time of Socrates, philosophy had become with at least some of
\emph{the sophists}\index{sophists} relativistic.
The participatory democracy of ancient of Athens placed a premium on
the ability to succeed in public speaking and debate, and created a
niche for those who could teach success in these arts.
The \emph{sophists} were predominantly professionally engaged in
teaching skills in these matters.
The doctrines of some of the most prominent of the sophists were
relativistic, thus accoring to Protagoras, ``man is the measure of all
things'', a doctrine suggesting that there are no absolute truths, but
that all is determined by cultural norms which vary from place to
place and from time to time.
}%ignore

