% $Id: ratio.tex,v 1.1 2008/07/11 18:16:10 rbj Exp $
\chapter{Rationality}

\index{Sceptic}
I am a sceptic first by nature, but also by conviction.
These scepticisms have a pervasive influence on the weltanschauung which underlies this work.

In what do these scepticisms consist?

My natural scepticism is a reluctance to take things on authority.
As I have grown older this has become more severe, and has become particularly severe in relation to academic philosophy.
My concerns about academic philosophy were for a period so severe that they were a source of great puzzlement to me as well as representing a serious impediment to my own philosophical endeavours.
Fortunately I eventually came to believe that I understood why philosophy is as it appears to me to be.
Since this has an important impact on the views I present here, this story will have to be told.

Scepticism is not usually merely a harsher than usual critique of received wisdom.
It has here also the following elements.

Firstly there are some elementary observations about knowledge and the impossibility of absolutely certain knowledge, and also the impossibility of absolutely compelling grounds for any course of action.
The presence of the world ``absolutely'' makes these observations somewhat academic, but they nevertheless form an important epistemological starting point.

Possible subsections:
\begin{enumerate}
\item
\item
\end{enumerate}
