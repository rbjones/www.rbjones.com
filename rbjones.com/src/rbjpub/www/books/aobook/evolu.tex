% $Id: evolu.tex,v 1.1 2008/07/11 18:16:10 rbj Exp $
\chapter{Evolution}

Various different kinds of evolution are considered here, of which the two headline cases are ``Darwinian evolution'' and ``cultural evolution''.
Evolution is used here very generally to speak of processes of incremental change which may result in the creation of order from chaos, and are sometimes considered to defy the second law of thermodynamics.

[{\it
Sources? \cite{BloomBRAIN}
}]
\section{Evolution and Thermodynamics}

\index{Thermodynamics!second law}
The second law of thermodynamics \Wiki{Second\_law\_of\_thermodynamics} asserts that the entropy of any isolated system never decreases.

Though biological evolution is not normally observable in closed systems it may be thought to involve violation of the second law of thermodynamics.
This is because entropy is like disorganisation, and evolution leads to more complex and organised forms of life.

However, the observational evidence for and the practical applications of thermodynamics are consistent with it not being {\it impossible} for entropy to decrease but rather {\it extremely} improbable.
A simple thought experiment may be used to illustrate this.

Consider gas in a closed container, notionally divided into two halves.
If we hypothesise a starting position in which all the molecules of the gas happen to be in one half of the container, then we would expect to see the gas immediately distribute across the whole container, and we would never expect again to see the gas again all in the same half.
However, if you take the state of the gas in the chamber at some subsequent time and reverse the direction of every molecule, keeping the speed the same, then the gas would indeed change from being fully distributed to being confined (momentarily) in one half of the chamber.

[{\it This is no good as it stands. It should refer to Boltzmann and deal with the objections to Boltzmann's conclusion given on wikipedia.
I need to consider how important this is and whether I need to get into it.
}]

\section{Biological Evolution}

The purpose of this thread is to give an account of how the process of biological evolution has itself evolved since the beginning of life on Earth, and to speculate about how it is likely to evolve further in the future.

The stages which we consider are as follows:

\begin{enumerate}
\item chemical evolution
\item autocatalytic sets
\item the cell membrane
\item the nucleus
\item sexual reproduction
\item multicellular organisms
\item central nervous system
\item memory
\item imitation
\item social hierarchy
\end{enumerate}

\subsection{Chemical Evolution}

Let us assume here that life evolved on earth without benefit of organic materials from external sources or divine intervention.

In the process of the evolution of life on Earth there must have been a first self-replicating entity.
Lets assume also that that was a molecule, since we know of self replicating molecules which are simpler than any self replicating collection of molecules.
Of course such a molecule will only be self-replicating in an appropriate medium, which includes the necessary parts for constructing the replica.

If this is how it happened then there was a moment at which there existed a medium (``soup'') in which life could spontaneously appear, and prior to that there will probably have been a history of media which gradually transformed into that special medium in which a self-replicating entity might form.
It is probable that this formation event would have been highly improbable, but that once such an entity is formed, the probability of further similar entities being form immediately increases radically.

The ``evolution'' of the medium up to the point of formation of the self-replicating entity is what we call here ``chemical evolution'' and that subsequently will be biological evolution (or initially a mix of the two).

\subsection{Autocatalytic Sets}


\subsection{The Cell}

\wiki{Prokaryote}

\subsection{The nucleus}

\wiki{Eukaryote}

\subsection{Sexual Reproduction}
