% $Id: anarc02.tex,v 1.1 2008/07/11 18:16:10 rbj Exp $
\section{Coercion and The State}

I'd like to explain here why, in aiming to minimise coercion rather than insisting on the abolition of the state, I still consider this to be a radical anarchism.

I propose to do this by some comparisons.

The purest anarchist theory holds that social disorder is caused by the coercion exercised by the state rather than the latter being necessary to limit the former.
The idea is that if the state is abolished, there will no longer be an reason for antisocial behaviour on the part of individuals.
That may or may not be so, but I'm afraid that I don't believe, however little reason there may be for it, that antisocial behaviour will ever entirely disappear.
Furthermore, I'm not inclined to think that violence on the part of individuals is any more acceptable than violence on the part of the state.
If some violence does persist, and there is no state to deal with its perpetrators, then others will, and violent crime will yield violent retribution and vigilante justice.

In my view we should seek to minimise coercion, and adopt the best institutions for that purpose.
Allowing violent restraint and possible incarceration of those who would otherwise continue to commit greater violence serves to reduce the overall level and is therefore consistent with a full-blooded anarchist position.

In fact, it seems to me unlikely that in the kind of society which I am proposing the government would ever entirely disappear, and that the government would continue to exercise some minimal coercion.
The kind of minimal policing which is desirable to minimise coercion could be undertaken by some non-governmental organisation, allowing for the abolition of the state.
However, it seems to me that there would be no advantage in taking this final step, and the resulting society would suffer no less coercion and would in no substantive sense be more fully anarchic.

In  anarcho-capitalism, it is envisaged that private protection agencies would serve instead of the state for this kind of purpose, and that individuals would purchase protection policies from such agencies.
It seems to me however that this role, if not undertaken by the state would be better done by not-for-profit agencies which would be funded by voluntary contributions and which would supply the service to all without charge.
The difference between this and state which does nothing more is purely nominal.

For many anarchists the state is synonymous with coercion.
A body which used no coercion would be for them, by definition, not a state.
So relative to such anarchists the difference between proposing abolition of the state, and proposing that the state relinquish coercion is purely verbal.
Residual coercion will in either case be justified if it palpably serves to reduce un-sanctioned coercion.
Whether we call the body which exercises this coercion a ``state'' or not is not important.
