\chapter{Introduction}\label{Introduction}

This book is intended as the core, fulrum or lynchpin of a
philosophical perspective which I am calling \emph{positive
  philosophy} because of the positivistic elements which it
incorporates from the philosophies of Hume, Comte and Carnap.

In this introductory chapter I offer a sketch of the whole of which
the book is the central and most fundamental part.

Positive philosophy consists of two parts, which correspond
approximately to the Aristotelian notions of theoretical and
practical philosophy.
This book belongs to the first part, though for the sake of setting
context this chapter embraces the whole.

I have come to feel that philosophical logic and metaphysics are
inextricably intertwined, and are together fundamental to the whole.
Though there have been great advances in the technical study of
formal deductive systems of late, the philosophical aspects seem to me
less impressive, and this work is put forward as more in the spirit of
Aristotle than of modern philosophical logic.

That spirit may partly be captured by the use of the term ``Organon''
for the collected works of Aristotle on logic.
This suggests that logic is to be considered primarily as a tool and
perhaps only second to that as an object of metatheoretical
investigation.
It is the central purpose of this work to present methods for logical
analysis, and hence one might think that the ``Organon'' presented is
central and the ``metaphysics'' incidental.

I don't consider it incidental myself.
Metaphysics, if we understand it along the lines suggested by
Aristotle, is a philosophical holy-of-holies.
In the explanation of a method of logical analysis, fundamental
distinctions must be drawn, and insofar as we consider these to be
fundamental and objective rather than pragmatic and discretionary, we
are concerned with metaphysics.
In addition to its place in giving an account of analytic method,
metaphysics represents a domain of analysis which serves as an
important testbed.
Though positivists have thought analysis an alternative to
metaphysics, analytic methods adequate for the whole of philosophy
must prove themselves on the testing grounds of metaphysics.

Further metaphysical connections comes throught the application of
analytic method to historical exegesis, and the use here and in
supporting publications of comparative analysis of Aristotle's logic
and metaphysics against the present offering as an expository device.

Notwithstanding the efforts of Aristotle, neither Aristotle's Organon
nor any subsequent account of logical methods has secured for
philosophy the rigour which is attributed to logical deduction in the
demonstration of mathematical conjectures.
By contrast with mathematics, philosophical reasoning remains
incapable of reliably establishing truth.

For most of the period between Aristotle and today there have been
solid reasons for this failure, and philosophers have had no
alternative but to make the best of a virtually hopeless situation.
Aristotelian syllogism, though a major achievement in its time, was
adequate neither for philosophy nor mathematics.
Despite this, deductive methods were sucessful and reliable in
mathematics.
Not so for philosophy.

The difference is probably due to the relative simplicity and clarity
of the concepts with which the mathematician deals, but whatever the
cause, extended logical reasoning in philosophy is so well known to
support any conclusion that it has been a mark of philosophical acumen
to be able to argue persuasively on either side of any philosophical
dispute.
