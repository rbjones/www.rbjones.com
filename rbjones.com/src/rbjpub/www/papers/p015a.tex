
\begin{quote}
\emph{
Philosophy, from the earliest times, has made greater claims, and achieved fewer results, than any other branch of learning. Ever since Thales said that all is water, philosophers have been ready with glib assertions about the sum-total of things; and equally glib denials have come from other philosophers ever since Thales was contradicted by Anaximander. I believe that the time has now arrived when this unsatisfactory state of things can be brought to an end.}
\end{quote}
With these words Bertrand Russell opened the first chapter of \emph{Our Knowledge of the External World, as a Field for Scientific Method in Philosophy}\cite{russell1921}.

Russell's presentiment of momentous change was provoked by advances in logic which began toward the end of the $19^{th}$ century, instigating the new discipline of mathematical logic and a period of rapid advancement in both mathematical and philosophical logic which continued throughout the following century, and continues apace.

These advances have been primarily technical advances in a new technical discipline.
What concerns us here is not these technical advances in themselves but the prospect of a transformation in the methods and in the veracity of philosophy which Russell perceived.

Possibly there are philosophers who believe that what Russell anticipated came to pass, though perhaps not quite as he anticipated.
I shall argue here that it did not, but that it still may (though certainly not just as he anticipated).
My presentation will begin with the background, expanding on Russell's concise account of the `unsatisfactory state of things'.
Next I shall describe how it seems to me matters have proceeded for those, particularly Rudolf Carnap, who were persuaded by Russell's prospectus and dedicated their lives to its realisation, to no avail.
From this dispiriting tale I shall then advance to a new prospectus, in which the picture is painted anew, and the question is addressed how it might ultimately be realised.

\chapter{Introduction}

This is an analytic polemic, a perspective on the history, past and future, of a particular conception of philosophical analysis.

My plan is as follows:

\begin{itemize}
\item first I shall locate as swiftly as possible the origin of this conception of philosophical analysis, which will be with Bertrand Russell
\item next I will trace some of the features of the previous two and a half millenia of philosophical development which help to clarify and motivate this conception of analysis
\item I will then sketch principle points in the subsequent history of the idea showing how its realisation has foundered
\item a review will then be necessary, by way of assessing whether the failure was due to irremedial defect in the conception
\item finally, some ideas on how to move forward.
\end{itemize}

In this introduction the whole story will be presented in brief.
In the sequel some parts will be dealt with at greater length.

\section{Philosophy as Logical Analysis}

The twentieth century has been called for philosophy, the age of analysis.
We are not here concerned with philosophical analysis in general, but in a particular kind of analysis which we will call, following Russell, \emph{Logical Analysis}.
Even this latter term has been used often by philosophers, sometimes for kinds of analysis which we will not recognise here under that title.

I don't know that Frege used that particular term (or a German synonym) but in the preface to his \emph{Grundgesetze der Arithmetik} he gives this short characterisation of \emph{scientific method} in mathematics:
 
\begin{quote}
The ideal of strictly scientific method in mathematics which I have tried to realise here, and which perhaps might be named after Euclid, I should like to describe in the following way.

It cannot be required that we should prove everything, because that is impossible; but we can demand that all propositions used without proof should be expressly mentioned as such, so that we can see distinctly what the whole construction rests upon.
	
We should accordingly strive to reduce the number of these fundamental laws as much as possible, by proving everything which can be proved.
	
Furthermore I demand --- and in this I go beyond Euclid --- that all the methods of inference used should be specified in advance.
Otherwise is it impossible to ensure satisfying the first demand.
\end{quote}

Frege's interest was in `scientific' methods for mathematics, and by `scientific' he seems to mean something like `rigorous'.
His interest in philosophy was directed to that end, to providing the necessary underpinning for a return to and advancement beyond Euclidean rigour in mathematics.

Russell had similar ambitions for mathematics, but his interest in philosophy was broader.
After applying the new logical methods to mathematics in \emph{Principia Mathematica}\cite{russell10} he then sought to apply these methods to philosophical analysis.
Russell talks about `scientific philosophy' and in this context the word `scientific' has two important aspects.
The first aspect is that of seeking a theoretical understanding of an objective reality, and much of what Russell says in describing scientific philosophy is an indictment of prior philosophical work as not being of this character, in this a proximate exemplar is the metaphysical philosophy of Bradley, but there is a history of previous philosophical systems which fall under similar indictment.
The second aspect connects better with Frege's sense of scientific.
In the first place this is because unscientific philosophies tend to proceed by deduction to conclusions which cannot be so obtained, but this aspect is prominent in its own right, for Russell's conception of scientific philosophy is one in which philosophy is logic.
\cite{russellPF,russellPLA,russell1921,russellML,russellSMP}

\section{The Origins of Logical Analysis}

This short story falls into three parts, culminating in certain groups of disparite philosophers.

The first part is set in ancient Greece, and runs from Thales to Socrates, Plato and Aristotle.
In this we first see the success of mathematics and the failure of philosophy as a deductive science.
In mathematics deduction proves reliable, mathematical knowledge grows continuously and is relatively infrequently found to be erroneous.



\section{Its Rise and Fall}

\section{Diagnosis and Prognosis}

\chapter{Some Background}

\section{Two Faces of Deduction}

It might be argued that deductive reason begins with the evolution of \emph{descriptive} language, and understanding of \emph{entailment} being part and parcel of the understanding of meaning or descriptive content.
However, the \emph{idea} of deduction, an awareness of it as a method for the advancement of knowledge, begins, to the best of our knowledge, at the same time, in ancient Greece, as western philosophy and mathematics, some two and a half millenia ago.

The reputation of deduction and the high esteem in which it was then held rested on its successful application in the development of mathematics \emph{as a branch of theoretical knowledge} rather than as a body of practically useful technique.
The success of deductive reasoning in mathematics may have encouraged philosophers to apply the method more widely.
Insofar as this constituted a rejection of religious, and other, forms of authority in favour of unprejudiced enquiry this was an important advance, but beyond mathematics deduction was to prove unreliable.


\section{The Scope of Deduction}

\section{Rigour and Formality}


\chapter{The Frege/Russell/Carnap Programme}

\section{}


\appendix

\chapter{Misc}

For two and a half millenia up to the nineteenth century logic has been a province of philosophy.
The twentieth century spawned the mathematical discipline of symbolic logic, and a new conception of philosophy as analysis, in which the role of logic was to be a matter of controversy.

There are several stories about the beginnings of modern analytic philosophy.
The first story was that it began with Russell and Moore in Cambridge around the turn of the $20^{th}$ century, as they moved away from british idealism.
One can see in these two philosphers rather different conceptions of analysis which reverberated down the century and may be seen in the diverse views of subsequent philosophers on the place of formal logic in philosophy and science.

Moore's conception of analysis had little connection with formal logic, and was concerned with the elucidation of philosophically interesting aspects of natural English.
Russell was involved in the developments to formal logic inspired by the logicist thesis that mathematics \emph{is} logic.
His first decade of the century was devoted to obtaining a formal logic (Russell's \emph{Theory of logical types}\cite{russell1908}) and the formal derivation of mathematics (with A.N.Whitehead) in \emph{Principia Mathematica}\cite{russell10}.

\chapter{Quotations}

This appendix is a place where I am holding quotes from important figures prior to finding the right place to locate them in the work.
My idea is to make extensive use of quotes for all but the most recent parts of the story (i.e. in those periods where copyright problems seem unlikely!).

\section{Frege}

From the preface to \emph{Grundgesetze der Arithmetik, Volume I} \cite{frege1893,frege1903}.

\begin{quote}
The ideal of strictly scientific method in mathematics which I have tried to realise here, and which perhaps might be named after Euclid, I should like to describe in the following way.

It cannot be required that we should prove everything, because that is impossible; but we can demand that all propositions used without proof should be expressly mentioned as such, so that we can see distinctly what the whole construction rests upon.
	
We should accordingly strive to reduce the number of these fundamental laws as much as possible, by proving everything which can be proved.
	
Furthermore I demand --- and in this I go beyond Euclid --- that all the methods of inference used should be specified in advance.
Otherwise is it impossible to ensure satisfying the first demand.
\end{quote}

\chapter{Russell}


\begin{quotation}

\end{quotation}
