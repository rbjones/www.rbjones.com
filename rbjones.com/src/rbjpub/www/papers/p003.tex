% $Id: p003.tex,v 1.4 2006/10/21 17:18:21 rbj01 Exp $
% bibref{rbjp003} pdfname{p003} 
\documentclass{rbjk}

\newdisplay{guess}{Conjecture}

\begin{document}                                                                                   
\begin{article}
\begin{opening}  
\title{On how many things there might be}
\runningtitle{On {\it how many} things there {\it might} be}
\author{Roger Bishop \surname{Jones}}
\runningauthor{Roger Bishop Jones}
%\runningtitle{}

\begin{abstract}
An exercise in classical (possibly even pre-historic) skepticism.
Of necessity this exercise has the character of tentative ideas rendered as sketchy notes.
These notes document a retreat in which ideas on the scope for objective ontological knowledge are progressively narrowed.
The last ditch in this process is the question of how many things it is logically possible that there might be.
\end{abstract}
\end{opening}

{\it \tableofcontents}

\section{Introduction}

I am engaged here in skeptical positivistic metaphysical ontology.

The work is skeptical in perhaps the most ancient sense.
It is a search for knowledge (about ontology) which predominantly fails, but no less genuine a search for that.
Much of the paper is concerned with the description of those kinds of ontological knowledge which I find elusive, and my reasons (inconclusive though they may be) for doubting its possibility.

There is I think more than one connection between this work and positivist thought.
I will mention here that my position on abstract ontology is probably as close to that of Rudolf Carnap as is consistent with any positive metaphysical conclusions at all.

\subsection{Some snippets of history}

It is a feature of natural languages, or at least of the English language, that in speaking of something one may appear to assert its existence.
This has given rise to much philosophical discussion and to some interesting philosophical theories.
At the beginning of the twentieth century some of the ontological liberality of informal discourse having found its way into the first attempt, by Frege, to provide formal logical foundations for mathematics, this became a serious problem, made conspicuous to Bertrand Russell by his discovery of the paradox which now bears his name.

It is clear from his writings of the time that Russell believed that the question of what exists was one which should have a definite answer, and that it was the business of philosophers to discover what it is.
Similarly, the question {\it what numbers are} seems presumed, both by Russell and by Frege before him to have a definite answer which could be discovered by philosophers working in the foundations of mathematics.

By the middle of the century metaphysics had fallen out of fashion and Rudolf Carnap, in the twilight of logical positivism, still maintained in dispute with Quine, that absolute ontological questions (what he called {\it external questions}) lacked meaning, and that an ontological question could only be answered in the context of some linguistic framework which gave it sense.
The sense given by such a framework to these ({\it internal}) ontological questions was not subject to constraints beyond those of pragmatics (coherence being one such pragmatic desideratum). 
Quine's positive contribution, ``on what there is'', was not to the ontological question itself, but to that of ``ontological commitment'', explicitly rendered ``what one should be construed as having asserted that there is''.
True enough Quine does in that paper turn to address the ostensive topic of his paper after discussing ontological commitment, but even then he says no more than that we can adopt whatever ontology suits us, telling us not {\it what is} but rather {\it what he considers it reasonable for us to suppose that there is}.

By the end of the twentieth century, the reservations of logical empiricists about metaphysics have long faded, and one may once again see questions such as ``what numbers are'' debated in earnest.

The position which I shall put forward in this essay is not very far removed from those of Carnap and Quine, who it seems to me are rather closer to each other than their apparent disagreement on these matters might lead one to suspect.
I will be more explicit perhaps in saying what questions in relation to ontology appear to have no objective answers, and more industrious perhaps in seeking ontological questions which do have answers.

\subsection{The independence of Metaphysical Ontology}

I should like to mention here that the ontological discussion which follows is not intended to say anything about what is properly or truly stated in ordinary or in scientific discourse.
The kind of philosophy in which I am seeking to engage is one in which truths are expected to transcend the accidents of the development of natural language, or even of the language of the natural sciences.
We find ourselves at the beginning of the 21st century facing the prospect that knowledge may no longer be the exclusive domain of human intelligence, and that questions of ontology, mathematics, science and engineering may be entertained and resolved by fabrications in silicon.

Arguments rooted in past usage of any particular language will therefore be treated with some suspicion in this context.

I shall begin however, by some comments on the historical snippets which have been presented above.

\section{On What There Is}

\subsection{The Ineffability of Reference}

How did Frege and Russell come to definite views on what numbers are?

Since their time various developments have helped to make such definite ontologies seem less plausible.

The model theory of first order logic  makes clear that the {\it identity} of the individuals in the domain of discourse is of no consequence.
For any interpretation of a first order language over some domain of individuals an isomorphic interpretation can be constructed over any other collection of individuals which has the same size.
Such a theory therefore tells us about the individuals, not {\it what} they are, but at best, {\it how many} of them there are.
The main interest lies not in the invividuals themselves but in the predicates and relations over these individuals which appear in the language.

This indifference to identity is exalted by modern pure mathematics almost into a religious principle.
When studying mathematical structures (such as that of the natural numbers), one should abstract away from any particular instance of the structure.
Mathematics is interested in such structures only ``up to isomorphism''.
What the natural numbers are is a question which has no answer, we can understand how to count and compute without know what numbers are.

Half a century after the invention of the digital computer and the explosion of work on formal notations which was stimulated by the need to write enormously complex algorithms without ambiguity or error, it is hard to understand how Frege and Russell could so readily infer from {\it what works} (for arithmetic) to a metaphysical conclusion.
We are now so familiar with the construction of formal notations or with the solution of problems with formal notations that a statement about what numbers {\it are} (if we must have one) will seem, if not entirely arbitrary, then at least, pragmatic rather than absolute.

\subsection{Existence ``up-to-isommorphism''}

Difficulties in knowing, for example, {\it what numbers are}, may be considered peripheral to the question of {\it what exists}.
We might possibly know everything there is to know about what countable collections exist without knowing which of these collections is the one referred to by the phrase {\it the natural number}, which phrase need not of course unambiguously denote any one of them.
A partial solution to this kind of ambiguity might be to speak instead of the existence of ``structures'', ``up-to-isomorphism''.
The question {\it what structures exist} is not liable to the same kind of skeptical attack, but nevertheless leaves us with no epistemological resources.
How could we resolve this question?

\subsection{Ontological Reductionism}

\subsection{Carnap and Quine}

Though they differed in how it should be said, Carnap \cite{Carnap50} and Quine \cite{Quine51} \cite{Quine53} were neither of them able to answer the question.
Each in his own way proposed that ontological questions should be settled pragmatically, neither offered a way to answer (or even to understand) questions about what there is as objective questions which have answers independently of our needs or interests.

Carnap was the more explicit, by declaring ``external'' ontological questions as without meaning.
Quine, having rejected the distinction which Carnap drew between ontological questions external to and internal to a ``linguistic framework'' says that:
\begin{quote}
Our ontology is determined once we have fixed upon the over-all conceptual scheme which is to accomodate science in the broadest sense\ldots
\end{quote}
On the question how that choice of conceptual scheme is to be made Quine offers at best only such help as can be drawn from `the rule of simplicity' with no hint that this will lead us to a true ontology rather than a merely convenient one.

\subsection{Limits of Relativity}

Carnap's position is relativistic, in that he considers ontological question to have answers {\it only in the context of some linguistic framework}, and denies that the ontological questions presupposed by that framework have objective meaning (unless they are themselves internal to some other framework).
This position boils down to the truism that an ontological question can only have a definite answer if it has (been given?) a meaning, and is otherwise meaningless.

His position on external questions is in effect a position on the problem of regress in the foundations of semantics.
He assumes that the semantics of a language must be determined in some other language (or linguistic context), and that the regress which ensues must ultimately terminate in a definition in some language which itself has no defined semantics.
He goes one step further than this by asserting that the external existence question is not merely primitive in some sense, but meaningless.

If we fall back on primitives which are supposed, though lacking in an explicit definition, to be nonetheless meaningful, then we may have a tenable foundational posture.
However, to regress to primitives which are definitely supposed to be without meaning leaves the entire structure undefined.

This problem of semantic regress is however not my present concern.

Carnap's ontological position is a part of a broader logical positivist anti-metaphysical stance, in which all necessity is {\it de dicto} and metaphysics, construed as synthetic {\it a priori} theorising, vanishes.



\section{Absolute and Relative}

\section{Abstract and Concrete}

\section{On What There Might Be}

\section{On How Many Things There Might Be}

\begin{thebibliography}{}

\bibitem[\protect\citeauthoryear{Carnap}{1950}]{Carnap50}
Rudolf Carnap: 1950,
\newblock {Empiricism, Semantics and Ontology},
\newblock {in {\it Meaning and Necessity}, University of Chicago Press, 1956, pp. 205-221}.

\bibitem[\protect\citeauthoryear{MacKensie}{2001}]{MacKensie}
Donald MacKensie: 2001,
\newblock {\it Mechanising Proof - Computing, Risk and Trust},
\newblock {The MIT Press}.

\bibitem[\protect\citeauthoryear{Quine}{1951}]{Quine51}
Quine W.V.: 1951,
\newblock {On Carnap's Views on Ontology},
\newblock {in {\it The Ways of Paradox and other essays}, Harvard University Press, Cambridge, Massachusetts, 1966}.

\bibitem[\protect\citeauthoryear{Quine}{1953}]{Quine53}
Quine W.V.: 1953,
\newblock {On what there is},
\newblock {in {\it From a Logical Point of View}, New York, Harper \& Row, 1963}.

\end{thebibliography}
\end{article}
\end{document}
