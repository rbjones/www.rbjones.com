% pdfname{PolySets}
\documentclass[11pt,a4paper]{article}
\usepackage{isabelle,isabellesym}
\makeindex

\newcommand{\ignore}[1]{}

% this should be the last package used
\usepackage{pdfsetup}

% urls in roman style, theory text in math-similar italics
\urlstyle{rm}
\isabellestyle{it}

\begin{document}

\title{The Theory of PolySets and its Applications}

\author{Roger Bishop Jones}
\date{$ $Date: 2006/12/11 13:56:10 $ $}
\maketitle

\begin{abstract}
A model for a set theory with a universal set and its use in the construction of other interesting and possibly useful foundational ontologies.
\end{abstract}

\vfill
\begin{centering}
{\footnotesize
\href{http://www.rbjones.com/rbjpub/isar/HOL/PolySets.pdf}{http://www.rbjones.com/rbjpub/isar/HOL/PolySets.pdf}\\
\href{http://www.rbjones.com/rbjpub/isar/HOL/PolySets.tgz}{http://www.rbjones.com/rbjpub/isar/HOL/PolySets.tgz}\\
\ \\
$ $Id: root.tex,v 1.4 2006/12/11 13:56:10 rbj01 Exp $ $\\
}%footnotesize
\end{centering}

\newpage

\tableofcontents

\parindent 0pt\parskip 0.5ex
\newpage
\section{Introduction}

The purpose of thie document is to construct a non-well-founded domain of sets with a universal set.
It is hoped that the particular construction employed will yield a collection of sets suitable for use in further constructions yielding domains with elements with slightly more common structure than sets (for example functions or functors).
These domains will then be used as the underlying ontology for a foundation system for formalised mathematics, and more generally for the formal demonstration of analytic truths.
Some of these avenues of further development will be explored.

The special feature of the proposed ontologies by contrast with other already established systems is their anticipated suitability for foundational systems intended for formalisation ``in the large'' i.e. which are designed for modular specification and proof, maximising re-usability.
This is the terminology of software engineering, and is used because of the relevant similarities which can be seen, between large scale formalisation of mathematics and (other analytic domains), and software development.

These non-well-founded foundation systems are not intended to make a difference to the way in which ordinary mathematics is done.
By `ordinary' mathematics, I mean arithmetic, analysis, geometry and any other mathematical study in which the domain of discourse is some definite concrete mathematical structure (often a system of numbers).
Nor is it intended to provide an alternative treatment of cardinal arithmetic, which would continue to be undertaken via the Von Neumann ordinals.

The areas it would impact more substantially are those parts of mathematics which are in some way generic, where a theory is developed without any definite conception of its domain of discourse, expecting the theory to be applicable in many different domains.
The most obvious examples of this kind of mathematics are abstract algebra, universal algebra and category theory.

\subsection{Some History}

The problem addressed is a long standing problem going back to the beginnings of mathematical logic.
It may be helpful to describe briefly some of the historical connections,

Our story begins with the discovery by Russell of the incoherence of Frege's logical system.
This discovery made necessary Russell's work leading to his ``Theory of Types'' in which draconian restrictions were placed through the type theory on what entities could be obtained by abstraction.
It was immediately evident that the restrictions placed were prohibitive and would render the formal logicisation of mathematics unrealisable using the Theory of Types.
Two separate problems were perceived and resolved in Principia Mathematica.
The first which comes from the ``ramifications'' in the Type Theory which made it predicative, was resolved by Russell's adoption of an axiom which made it as if the ramifications had never been included, and lead to the ramifications being omitted from most subsequent type theories.
This does not here concern is.

The second problem was that many widely used mathematical entities appeared no longer as single objects but as complete families of objects with a variety of different types.
Strict formalisation of mathematics would require that theorems about these objects be proved separately for each type with which they are used, even thought the proofs would all, apart from type annotations, be identical.
To deal with this Russell adopted the device of typical ambiguity, an informal understanding that type annotations could be omitted in a proof and the proof then suffice for all results for which it is type-correct.

This essential logical scheme travelled down through the 20th century arriving in about 1985 in the logic HOL devised and implemented in Computer Software for use in the formal verification of digital hardware.
This variant of what was by then known as ``Higher Order Logic'' remained logically similar to Russell's theory of types (without the ramifications).
The two main innovations, of which the impact was pragmatic rather than fundamental, were the translation into a form based on the lambda-calculus and the transfer of the type variables with which Russell's pragmatic notion of ``typical ambiguity'' were made explicit, from the metalanguage into the object language.
The lambda-calclulus, and the variety of Higher Order Logic based on it (known as the Simple Theory of Types), are both due to Church \cite{church36,church40}.

\subsection{Notation and Terminology}

It is in the nature of the subject matter that many different variants of familiar mathematical concepts are used.
In particular, there are a variety of different kinds of set membership, and also various kinds of predication and type assignment.
Not only do we have a confusing profusion of similar but not identical concpts, but we are constrained by the fact that this work is mathematics formalised for processing by computer, and must therefore fit in with the constraints imposed by the software (in this case Isabelle).

Overloading the membership sign, would I fear lead to an unintelligible document, but chosing another symbol each time a new relationship is introduced would not be much better.
Subscripts and superscripts are not much needed for their cusomary purposes and are therefore used exclusively as decorations for symbols which distinguish the variants employed.

I have tried to make their use as systematic as possible, a different letter is associated with each domain and is used as a subscript on operators over that domain.

The letters are as follows:

\begin{centering}
\begin{tabular}{| l | l | l | l |}
\hline
letter & type & domain & description \\
\hline
g & Set & UNIV & a domain of pure well-founded sets \\
r & Set & polysets & intermediate representatives for the polysets \\
q & Set set & ps\_eqc & final representatives for polysets \\
p & pset & UNIV & the polysets themselves (as a new type) \\
\hline
\end{tabular}
\end{centering}

\newpage
\input{PsMisc}
\newpage
\input{Sets.tex}
\newpage
\input{PolySetsC.tex}
\newpage
\input{PolySets}
\newpage
\input{Ilambda}

% optional bibliography
\bibliographystyle{abbrv}
\begin{thebibliography}{99}

\bibitem{debruijn72} N.G. De Bruijn, \emph{Lambda-Calculus Notation with Nameless Dummies, a Tool for Automatic Formula Manipulation with Application to the Church Rosser Theorem}, Indag Math 34,5 381-392 (1972)

\bibitem{church36} Alonzo Church, \emph{An Unsolvable Problem in Elementary Number Theory}, American Journal of Mathematics Vol.58 345-363 (1936)

\bibitem{church40} Alonzo Church, \emph{A formulation of the simple theory of types}, Journal of Symbolic Logic Vol.5 56-68 (1940)

\bibitem{coquand86} Thierry Coquand, \emph{An Analysis of Girard's Paradox}, Proc. Symposium on Logic in Computer Science, Cambridge Mass, (1986) 

\bibitem{jech2002} Thomas Jech, \emph{Set Theory}, The Third Millenium Edition, Springer 2002

\bibitem{milner78} Robin Milner, \emph{A Theory of Type Polymorphism in Programming}, Journal of Computer and System Sciences 17, 348-375 (1978)

\end{thebibliography}

%\twocolumn{\small% pdfname{Membership}
\documentclass[11pt,a4paper]{article}
\usepackage{isabelle,isabellesym}

\newcommand{\ignore}[1]{}

% further packages required for unusual symbols (see also
% isabellesym.sty), use only when needed

%\usepackage{amssymb}
  %for \<leadsto>, \<box>, \<diamond>, \<sqsupset>, \<mho>, \<Join>,
  %\<lhd>, \<lesssim>, \<greatersim>, \<lessapprox>, \<greaterapprox>,
  %\<triangleq>, \<yen>, \<lozenge>

%\usepackage[greek,english]{babel}
  %option greek for \<euro>
  %option english (default language) for \<guillemotleft>, \<guillemotright>

%\usepackage[latin1]{inputenc}
  %for \<onesuperior>, \<onequarter>, \<twosuperior>, \<onehalf>,
  %\<threesuperior>, \<threequarters>, \<degree>

%\usepackage[only,bigsqcap]{stmaryrd}
  %for \<Sqinter>

%\usepackage{eufrak}
  %for \<AA> ... \<ZZ>, \<aa> ... \<zz> (also included in amssymb)

%\usepackage{textcomp}
  %for \<cent>, \<currency>

% this should be the last package used
\usepackage{pdfsetup}

% urls in roman style, theory text in math-similar italics
\urlstyle{rm}
\isabellestyle{it}

\begin{document}

\title{Membership Structures}
\author{Roger Bishop Jones}
\date{$ $Date: 2007/06/11 20:48:54 $ $}
\maketitle

\begin{abstract}
A exploration of a way of doing set theory in HOL without using axioms.
I thought that locales in Isabelle-HOL would make this kind of approach workable, and this document is the result of my investigating this hope.
However, it quickly became apparent that the limitations on the implementation of locales. particularly in relation to the forms of definition which you can use in a locale, were too severe.
Consequently I don't get very far.
I had hoped to do the construction of the PolySets on this kind of set theory, instead of using an axiomatisation of set theory in HOL, but I reverted to the latter course.
\end{abstract}

\vfill
\begin{centering}
{\footnotesize
\href{http://www.rbjones.com/rbjpub/isar/HOL/Membership.pdf}{http://www.rbjones.com/rbjpub/isar/HOL/Membership.pdf}\\
\href{http://www.rbjones.com/rbjpub/isar/HOL/Membership.tgz}{http://www.rbjones.com/rbjpub/isar/HOL/Membership.tgz}\\
\ \\
$ $Id: root.tex,v 1.5 2007/06/11 20:48:54 rbj01 Exp $ $\\
}%footnotesize
\end{centering}

\newpage

\tableofcontents

\parindent 0pt\parskip 0.5ex

\section{Introduction}

This document is concerned with what is usually known as ``set theory''.

The approach adopted is intended to be distinguished from usual treatmente of set theory in the following ways:

\begin{itemize}

\item The subject matter is taken to be {\it membership structures} rather than sets.
A membership structure is, of course, some collection together with a binary relation over that collection, i.e. an interpretation of the first order language of set theory.

\item No specific set of axioms is adopted.
Instead the interrelationships between various properties of membership structures are studied.

\item The distinction is drawn between those properties of membership structures which are expressible as sentences in the language of first order set theory (``first-order'' for short), and those which are not (``higher order'').

\item Particular interest is shown in certain properties which are not first order and in the kinds of membership structure posessing these properties.

\item The informal language of set theory is made more precise by noting that certain properties (notably cardinality) normally thought of as properties of sets, are perhaps better seen as relationships between membership structures and their elements (in what may be called their internal use) unless an external criterion of equipollence is applied (yielding an ``external'' notion of cardinality).
Some theorems of set theory are explicated using this distinction, for example the result that V=L is incompatible with the existence of measurable cardinals.
\end{itemize}

The theory is substantially but not exclusively concerned with well-founded membership structures, in which aspect its motivation is largely philosophical.
In practical terms, the main application of the well-founded structures is to the construction of non-well-founded structures which are to be used elsewhere.

I regard the use of locales in this to have been a failure and the set theory has not been progressed.
I have added material on Boolean Valued Models, motivated by different considerations.


% generated text of all theories
\input{session}

\begin{thebibliography}{99}

\bibitem{jech2002} Thomas Jech, \emph{Set Theory}, The Third Millenium Edition, Springer 2002

\end{thebibliography}

\end{document}

%%% Local Variables:
%%% mode: latex
%%% TeX-master: t
%%% End:
}

\end{document}
%%% Local Variables:
%%% mode: latex
%%% TeX-master: t
%%% End:
