% $Id: internal_audit.tex,v 1.3 2008/07/25 18:47:05 rbj Exp $ bibref{} pdfname{internal_audit} 
\documentclass[numreferences]{rbjk}
%\documentclass[10pt,a4paper.titlepage,openany,twocolumn]{article}
\usepackage{makeidx}
\newcommand{\ignore}[1]{}
\usepackage[unicode,pdftex]{hyperref}
\hypersetup{pdfauthor={Roger Bishop Jones}}
\hypersetup{colorlinks=true, urlcolor=black, citecolor=black, filecolor=black, linkcolor=black}

\crline{no}
\makeindex
\begin{document}                                                                                   
\begin{article}
\begin{opening}  
\title{Notes on Internal Audit}
%\runningtitle{Abstract Ontology}
\author{Roger Bishop \surname{Jones}}
\date{$ $\ $ $}
%\runningauthor{Roger Bishop Jones}

\begin{abstract}
Notes on how Heydour Parish Council internal audit might be brought in line with recommended practice.
\end{abstract}

\end{opening}

\vfill

\begin{centering}
\footnotesize{
Created 2008/07/23

Last Change $ $Date: 2008/07/25 18:47:05 $ $ $ $Revision: 1.3 $ $

}%footnotesize
\end{centering}

%\def\tableofcontents{{\parskip=0pt\@starttoc{toc}}}
\setcounter{tocdepth}{4}
{\parskip-0pt\tableofcontents}

\section{Introduction}

The parish council is obliged to undertake annually an assessment of the effectiveness of its provisions for internal audit.

At present, an assessment is not possible, because the process for internal audit is not documented.

Having conducted the last internal audit, I am in a position to advise the council that the audit was not fully compliant with recommended practice, which I did at the council meeting on 16th July 2008.
At that meeting I was asked to provide some notes on how the council might go about bringing its operations in line with recommended practice.
This document responds to that request, but is confined to consideration of what would be necessary to permit a compliant positive internal audit.

\section{An Outline of Recommended Practice}

The following is based on a first perusal of the practitioners' guide \cite{JPAG08}, and should not be regarded as definitive.
It is confined to matters obviously relevant to internal audit.

At the beginning of the year the council should appoint an internal auditor.
The internal auditor should be given terms of reference, normally drafted by the Responsible Financial Officer (RFO), who is normally (and in our case) the Clerk to the Council.

The internal audit must be conducted on the basis of an assessment of risk.

The council is obliged to undertake an assessment of risk annually, and to put in place a system of ``internal controls'' to manage the identified risks.

The internal audit page on the annual return contains a set of questions relating to the objectives of the internal controls, asking in each case whether these objectives are being met, together with a supplementary declaration affirming the existence of adequate controls for any other areas of risk, and a list of such other areas of risk.
It is not necessary for all areas to be covered by the internal audit every year, but the audit must cover the planned areas for the year, so if the audit is not to be complete there must be a plan to determine the coverage.
A compliant audit will check the operation of the internal controls relating to each internal control objective, so there will need to be some documentation for the internal controls relevant to each area of risk.

I would have thought that there should be a close relationship between the items in the internal audit report and the risks identified in the risk assessment prepared by the council, and that for a council as small as Heydour there will probably be no significant risks not covered by the control objectives listed.

In the practitioners' guide \cite{JPAG08}, Appendix 9 gives guidance on internal audit and includes a table of internal controls and suggested tests for each control.

\section{The Present Situation}

At present, though the council conducts its business in a proper manner, the way it does so is not documented.
The ``internal controls'' required by the Practitioner's Guide are long established but undocumented practice.

\section{Moving into Compliance}

To bring the operation into compliance with the Practitioners Guide reference to Appendix 9 is recommended.
First it desirable to produce a document which prescribes how the internal controls mentioned in this table are to operate.
The presumption is that this would be a brief description of current practice, but the question should be asked whether present practice is auditable, i.e. whether it is possible after the fact for an internal auditor to perform the recommended tests.
If this is not the case, perhaps because some necessary record of action is not kept, then the internal controls as documented should make the necessary provision and the practice should be brought in line.

Once the internal controls are documented it can then be asked whether the set of recommended tests in Appendix 9 should be adopted.
If so this should be stipulated in the internal control documentation of the procedure for internal audit, otherwise a revised table should be prepared incorporating, and justifying, any necessary modifications to the table in Appendix 9. 
In the documentation of internal controls or elsewhere, there should be terms of reference for the internal audit activity which will refer either to table in Appendix 9 of the Practitioners' Guide or to the amended version, as appropriate.

\section{Beyond Internal Audit}

Since this approach to compliance is aimed at enabling a compliant clean internal audit, the approach suggested cannot be expected to realise full compliance in all matters.

Any residual non-compliance after completion of such an exercise should relate to matters which are not thought to be relevant to any significant area of risk identified in the risk assessment, but might possibly be in matters which are legally mandated.

If an initial exercised focused on internal audit compliance were undertaken as suggested, a further review to identify any potential non-compliance with legally mandated aspects of the practitioners guide might then be advisable.
Alternatively these two objectives could be addressed at the same time.


{\raggedright

\bibliographystyle{alpha}
\bibliography{rbj}
} %\raggedright

%\twocolumn[\section{Index}\label{Index}]
%{\small\printindex}

\end{article}
\end{document}
