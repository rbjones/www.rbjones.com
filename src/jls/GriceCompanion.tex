% $Id: GriceCompanion.tex,v 1.10 2013/01/08 16:41:38 rbj Exp $

\documentclass[10pt,titlepage]{book}
\usepackage{makeidx}
\usepackage[unicode]{hyperref}
%\usepackage{parallel}
\pagestyle{headings}
\usepackage[twoside,paperwidth=5.25in,paperheight=8in,hmargin={0.75in,0.5in},vmargin={0.5in,0.5in},includehead,includefoot]{geometry}
\hypersetup{pdfauthor={J.L. Speranza}}
\hypersetup{colorlinks=true, urlcolor=blue, citecolor=blue, filecolor=blue, linkcolor=blue}
%\usepackage{html}
\usepackage{paralist}
\usepackage{relsize}
\newcommand{\ignore}[1]{}
\makeindex

\title{The Grice Companion\\
{\small Notes for the Carnap/Grice Conversation}}
\author{By J.L.~Speranza\\
\\
{\scriptsize Edited by Roger Bishop Jones}}
\date{\ }

\begin{document}
\frontmatter
                               
\begin{titlepage}
\maketitle

\vfill

%\begin{abstract}
%\end{abstract}

\vfill

\begin{centering}

\vfill

\footnotesize{
\ignore{
Started 2010-02

Last Change $ $Date: 2013/01/08 16:41:38 $ $

%\href{http://www.jls.rbjones.com//GriceCompanion.pdf}{http://www.jls.rbjones.com/GriceCompanion.pdf}

Draft $ $Id: GriceCompanion.tex,v 1.10 2013/01/08 16:41:38 rbj Exp $ $
}
\copyright\ J.L.~Speranza;

}%footnotesize


\end{centering}

\end{titlepage}

\setcounter{tocdepth}{4}
{\parskip-0pt\tableofcontents}

\mainmatter

\ignore{Bibliography test:
\cite{grice41}
\cite{grice86}
\cite{grice87}
\cite{grice88}
\cite{grice89}
\cite{grice91}
\cite{grice01}
\cite{grice57}
\cite{speranza89}
\cite{speranza91a}
\cite{speranza91b}
\cite{speranza95}
}%hide

\chapter{Introduction}

J. L. Speranza\index{Speranza} is an Italian born in the pampas who has dedicated his life so far to Gricing this and that.
He holds (in the wall of his room) a doctorate \cite{speranza95} from the University of Buenos Aires on Grice\index{Grice} -- ``Pragmatica Griceana''.
(This, he  regrets, while recognised by Western Educational Services alright -- has been translated as ``Grecian Pragmatics'' -- but he don't (sic) mind.
He has  written extensively, but more recently, spoken extensively on matters Gricean at the Grice\index{Grice} Club which he created just to annoy Dan Sperber\index{Sperber, Dan}.  

\chapter{Grice Biography}

\section{A Chronology}
H. P. Grice, 1913-1988.
Fellow of St. John's, Oxford. Tutorial Fellow in Philosophy, and
University Lecturer, St. John's, Oxford.
FBA 1966.


\begin{itemize}

\item[Thirties]

\item[1938] Negation. Unpublication. Archival material in the Grice
Collection. MSS BANC 90/135c.

\item[Forties]

\item[1941] Personal identity. Mind.
\item[1948] Meaning. repr. in WoW.

\item[Fifties]

\item[Sixties]

\item[1961] Causal theory of perception. Repr. in WoW

\item[1962] Some remarks about the senses, in R. J. Butler, "Analytic
Philosophy", Oxford: Blackwell, repr. in WoW.

\item[1967] The William James Lectures. Manuscript, BANC 90/135c. Repr. WoW,
as Part I, "Logic and Conversation".

\item[1968] Utterer's meaning, sentence meaning and word meaning. Foundations
of Language. Originally from 1967 handwritten mimeo. Repr. WoW as Essay
6.

\item[1969] Utterer's meaning and intentions. Philosophical Review. From
original 1967 manuscript. Repr in WoW as Essay 5.

\item[1969b] Vacuous names. in Davidson/Hintikka, Words and objections:
essays on the work of W. V. O. Quine.


\item[Seventies]

\item[1971] Intention and uncertainty. Proceedings of the British Academy and
Clarendon Press.

\item[1975] Method in philosophical psychology. Proceedings of the American
Philosophical Association. Repr. as Appendix I to Gri1991.

\item[1975] Logic and Conversation. Originally seond William James lecture
\item[1967] in Davidson/Harman, Logic and Grammar. Encino, Calif: Dickinson,
and in Cole/Morgan, Syntax and Semantics, volume 3: Speech acts.
London: Academic Press. Repr. as Essay 2 in WoW.

\item[1978] Further notes on logic and conversation. Originally third William
James Lecture 1967. in P. Cole, Syntax and Semantics, vol 9:
Pragmatics. London: Academic Press. Repr. as Essay 3 in WoW.

\item[Eighties]

\item[1981] Presupposition and conversational implicature. (Dated 1970 in
WoW), in Cole, Radical Pragmatics. Academic Press. Repr. as Essay 17 in
WoW.

\item[1982] Meaning revisited. In N. Smith, "Mutual knowledge" (proceedings
of a symposium at the University of Sussex). London: Academic. Repr. as
Essay 18 in WoW.

\item[1985] Preliminary Valediction. The H. P. Grice Papers, MSS BANC 90/135c.

\item[1986] Actions and events. Pacific Philosophical Quarterly.

\item[1986b] Reply to Richards, in Grandy/Warner. It includes Part I, a
rewrite of his earlier ""Prejudices and predilections which become the
life and opinions of Paul Grice" by Paul Grice", and the "Reply to
Richards" proper -- a commentary on Richard Grandy's and Richard
Warner's introduction to the festchrift.

\item[[posthumous]]

\item[1988] Aristotle on the multiplicity of being. Pacific Philosophical
Quarterly. Posthumous. (published after Grice's passing on Aug. 28,
\item[1988] -- and with a note, "in memory of Paul Grice").

\item[1989] Studies in the Way of Words. Cambridge, Mass.: Harvard University
Press.

\item[Ninetees]

\item[1991] The conception of value. Oxford: Clarendon

\item[Noughties]

\item[2001] Aspects of reason. Oxford: Clarendon.

\end{itemize}

\section{Griceland\index{Griceland}}

Herbert Paul Grice was born in Harborne, Staffs 
in 1913 and died in 1988.
He wrote thus far three books:
\begin{itemize}
\item Studies in the Ways of Words \cite{grice89}
\item The Conception of Value \cite{grice91}
\item Aspects of Reason \cite{grice01}
\end{itemize}
He was educated at Clifton and Corpus Christi, 
and was Tutor in Philosophy for more than forty years at St. John's to end as full professor of philosophy at Berkeley.

\section{Religion and Morality}

During this period (as a student) Carnap\index{Carnap} gradually came to 
disbelieve in God, without being aware of any change in  his beliefs on moral questions. 

Grice's religious inclinations are harder to pin down. Chapman\index{Chapman} in  
fact sounds rather authoritative when she states that "Grice\index{Grice} lost all faith by 
 the age of 19" or something. You can never be so sure. Chapman\index{Chapman} redeems 
herself  by noting the very many religious references, eschatological and 
Biblical, in  Grice's various writings.

While a disbeliever in God, Grice\index{Grice} liked to play God. Borrowing on  
this idea by Carnap\index{Carnap} of pirot\index{pirot}s which karulize elatically, Grice\index{Grice} founds a full  
programme in the vein of the ideal-observer. What we would do, as God, to 
secure  the survival of pirot\index{pirot}s. While not religious in nature, it has a 
religious tone  that is absent in the writings of Carnap\index{Carnap} in any respect.

\section{Influences}

While Carnap\index{Carnap} would write, "The men who had the strongest  influence on my 
philosophical thinking were Frege\index{Frege} and Russell\index{Russell}" in that order%
\footnote{But was the order signficant?
Carnap is very specific about the ways in which he was influenced by these men, and Russell's influence seems to have been more significant in shaping Carnap's philosophical programme.}%
,  Grice\index{Grice} could be 
more irreverent.

A 1913-born man finds the strongest influence in  a 
linguist born in the 1930s: Noam Chomsky\index{Chomsky} and the Vienna Circle\index{Circle!Vienna} refugee: Quine\index{Quine}.  --"
I have to admit" Grice\index{Grice} writes (or words), that what "I admired in them is  
their method -- never their ideas, which I never respected, really". Grice\index{Grice}'s 
 choice is particularly agonistic: in the sense that it provokes sheer 
agony in  us: "for I never ... over the fact that these two genii never agreed 
ON  ANYTHING".
And they lived so close!

\section{Carnap in London}

Carnap\index{Carnap} gave three renowned lectures in London in 1934 \cite{carnap35}.
This  Grice must have been aware of. He was in his early 
20s, then, but I’d think Ayer  would have made the thing known in Oxford.
 
Grice was VERY concerned with what Austin\index{Austin} was doing -- before the Phoney war.

\section{Aristotle\index{Aristotle}}

It really all starts with Aristotle\index{Aristotle}.
Hume is Where the Heart Is, but Kant\index{Kant} is the big influence when we are speaking of the Carnap\index{Carnap}/Grice interface. 
Grice refers to “Kantotle” and “Ariskant” as apt abbreviations of what he has in mind when he thinks of the greatest of them all.
If the early Carnap\index{Carnap} is best seen as a neo-Kantian, Grice’s pedigree can be traced back as early as Oxford Hegelianism.
Ryle, who had been an adherent of  Heidegger\index{Heidegger} (and had reviewed Heiddegger’s Sein und Zeit for Mind in 1929, had sent Ayer to Vienna (Ayer’s Weiner Kreis 
crisis, as it were). On his return, the  ‘enfant-terrible’ as he then was and Grice depicts him as, he soon splits from  Austin\index{Austin}'s playgroup\index{playgroup}.
Everything Grice notes, was brought to a halt with The Phoney War.

\section{The Wars}

Carnap\index{Carnap} fought in the Great War.
Grice\index{Grice} in the 'phoney' one. Grice's father  
was operative during the Great War when Carnap\index{Carnap} fought.
Grice's father was a businessman -- a rather poor one.
He invented a contraception that was good during the Great War but found useless in the aftermath.
In 1939 Grice\index{Grice} joined the Royal Navy and fought the Germans in mid-Atlantic theatre of operation.
He was transferred to Admiralty in London in 1942. 
 
Carnap\index{Carnap} possibly acquired some pessimism after the Great War.
After all, the Germans lost.
This contrasts with the strange optimism felt by the Brits in the post-war period.
Grice\index{Grice} included.
They were grown up, they had been forced to grow up too soon.
Yet they had not been able to develop their philosophical talents.
Now in their thirties, a whole generation had been devastated.
They had no time to lose.
Grice\index{Grice}'s efforts were directed at the formation of a new generation of philosophers at Oxford -- along with his  
current fare of unappointed tuttees.
Strawson\index{Strawson} was Grice\index{Grice}'s favourite student, and the one which gave some sense to his rambling life.
At the senior level, the influence of Austin\index{Austin} was paramount.

\section{After the War}

After the war, Grice becomes more and more influential in Austin\index{Austin}'s  'kindergarten', and eventually gets the greatest credit of them all: he travels to the USA to deliver the William James Lectures, where he manages to pun on Heidegger\index{Heidegger} alla Carnap\index{Carnap} (``Heidegger\index{Heidegger} is the greatest living philosopher, if you can take me seriously''  (\cite{grice89} Essay 1)).
It’s sadly after Carnap\index{Carnap}’s demise that Grice introduces his ``pirotological programme'' in the APA presidential address (Pacific Division) for 1975.
He becomes more metaphysical as gives the Carus Lectures (published as \cite{grice91}).
His postmortem influence grows as his ``Aspects of Reason'' lectures get published in 2001 \cite{grice01}. 

Grice  regards metaphysics as comprising both a general and a special branch.
The general branch comprises two subbranches, metaphysics proper, qua theory of the category, and eschatology, qua theory of transcategorial 
epithets.
The special branch includes cosmology and rational psychology.  

\section{The Oxford Circle\index{Circle!Oxford}}

The  Oxford Circle\index{Circle!Oxford}. The Berkeley Circle.
If Carnap\index{Carnap} was ultimately Vienna Circle\index{Circle!Vienna}, there’s more of a larger geographical spectrum in Grice.
``The Carnap\index{Carnap} Circle'' --  by this we mean the Carnap\index{Carnap} convivial approach to philosophy. 
This is echoed in  the Grice Circle\index{Grice!Circle}.
Grice could only work philosophically in convivial conditions.
Although he never belonged to the early Austin\index{Austin} club, he was a regular, indeed, Austin\index{Austin}'s favourite member of Austin\index{Austin}'s playgroup\index{playgroup}. 
On Austin\index{Austin}'s death, Grice led the playgroup\index{playgroup} until his departure for the States.
What's more important, while in the States he would gather at his ``at-homes'' up the Berkeley Hills.
It's less clear what sort of convivial meeetings Carnap\index{Carnap} held while at Santa Monica. 

Carnap\index{Carnap} observes that the state of Logic teaching in the USA was much better than in Europe.
This is echoed by Grice.
He confesses publicly that part of his  professional reason -- one wonders: couldn't he just write? -- for moving to the States was ``closer contact with logicians''.
Grice is already in 1969 quoting from Boolos, Parsons, Myro\index{Myro} and Mates\index{Mates}.
This is the beginning of his System G.
Putnam would actually cut short Grice's formalistic fancies: ``You are too formal'' he'd complain.
Grice did not fit the brit steroretype of the good humanistic Oxford don of Austinian ilk.
Of course they were wrong.
Grice fit NO stereotype.

\section{System $G_{HP}$}

Grice lists 13 items (\cite{grice89} ii) which Grice thinks are constitutive of a good formal language.
They reproduce almost verbatim Carnap\index{Carnap}’s own ideas of axiomatic languages which he had drawn independently from Euclid and the modern formal logicians like Frege\index{Frege} and Goedel.
System $G_{HP}$ is best illustrated with simple utterances like the one Carnap\index{Carnap} used in ``Introduction to Semantics''.
Grice agrees that lack of sense here is actually a good thing.
For what is the logical  form of, ``Pirots karulize elatically''?
We need a meaning postulate.
An x is a pirot\index{pirot} P iff it karulize K, where K is extensionally defined as a class such that its members are the denotata of all 
the names of pirot\index{pirot}s.
The ‘elatically’ qua adverb should not deviate us from the gist of the logical form.
More important is the need to quantify.
Are all pirot\index{pirot}s meant?
Or is there some scalar implicature at work (``Some pirot\index{pirot}s karulise elatically''; ``some not''.
Is karulising the pirot\index{pirot}’s  raison d’etre.
Is, in the words of Carnap\index{Carnap}’s meaning-and-necessity something that holds for all possible worlds where pirot\index{pirot}s exist?
In other words, indeed Aristotle\index{Aristotle}’s, is karulising an essential property for pirot\index{pirot}s?
Are we to read Carnp's sentence as pirot\index{pirot}s having (``MUST'') to karulis elatically?
These are the  goals.
The means are simple enough in Morrisian terms.
We need first a Syntactics  – where 'proof' is defined, with Gentzen-type rules added for good measure (if we are going to formalize this as (x)Px à Kx, we need introduction and elimination rules for both (x) and à.
Then we need a proper formal Semantics.
This should allow for scope indicator devices.
While pragmatics in Carnapcopia\index{Carnapcopia} is the realm of pragmatism, the landscape of Griceland\index{Griceland} is more formal.
The third element in System G, Pragmatics, is the realm of implicature 

\section{Metaphysics}

Grice, as a Brit, would be more familiar with the views of Waismann, the member of the Vienna Circle\index{Circle!Vienna} who had made it to  England.
Grice lectured on metaphysics for the BBC.
The result is in D. F. Pears, ``The nature of metaphysics'', \cite{grice57}.
This is vintage Grice.
I.e. Grice self-presenting as a metaphysician as 'ambitious' as Kantotle was.
Metaphysics was starting to cease being the term of abuse he felt Ayer had turned it into. 
Metaphysics as a discipline in need of professional defence.

Grice, unlike Carnap\index{Carnap}, was professionally involved in DEFENDING metaphysics.
He delivered annually two courses on Metaphysics.
Usually with G.~Myro\index{Myro}.
Naturally, he felt the defence of the discipline was what was professionally and institutionally required from him, especially after becoming a full professor at Berkeley in 1975.
Grice's student,Sir Peter Strawson\index{Strawson} had become by 1968 the standard for metaphysical theory as understood in England and Oxford.
As Waynflete professor he became more and more interested in neo-Kantian foundations for the discipline. 

\paragraph{How formal can we get}

Myro\index{Myro} was a special influence in Grice's metaphysical thought.
Educated at Oxford in Balliol, he had a strict logical background and inspired in Grice much of what transpired as Grice's System Q, which Myro\index{Myro} later re-baptised System G -- ``in gratitude to Paul Grice for the original idea''.
The syntax of System G makes use of scope devices to allow for pragmatic implicature.

These undertake two forms:

\begin{itemize}
\item[(i)] the use of square brackets to indicate common-ground status.
Grice provides formal rules for the introduction and elimination of square brackets in (\cite{grice89} xviii) 
\item[(ii)] the use of numerical subscripts (in Vacuous Names).
This allows for the scope maximal readings of formulae but allowing as well for a minimal reading upon a simple numerical transformation.  
\end{itemize}

\paragraph{Dogmatism} Grice and Strawson\index{Strawson} had defended the analytic-synthetic 
distinction in "In  defense of a dogma" but Grice grew sceptical as to the success 
of that defense  (of an underdogma, as he later had it). He grew more and 
more pragmatist towards  the viability of the need to postulate the 
distinction. 

\section{Semantics for System G}

Carnap\index{Carnap}'s "meaning postulates" have affinities with Grice's notion  of 'entailment' which he drew from Moore.
Moore, while not a formal logician, is responsible for this coinage, which appealed to Grice, as he would contrast, in his System G, only entailment with 'implicature'. 
There would be no place for 'presupposition' or truth-value gaps in this scheme, as there is in Strawson\index{Strawson}.
The metaphysical implications of Grice's choice of a bivalent standard interpretation of System G are obvious. 
The  Man is the Style.
Grice spoke excellent English.
As Clifton and Corpus Christi educated, he found easily crowds of followers, especially in America, in younger philosophers who had grown tired of their dogmatic empiricist teachers.
Grice brought a breath of fresh air.
This is ironical as seen from the other side of  the 'pond', in that the breath of fresh air can be looked, in a sort of inverted snobbery, as an irreverent reactionary dogmatism!
On the other hand, Carnap\index{Carnap} was perhaps less influential among the younger philosophers.   

\section{Griceland\index{Griceland} and Carnapcopia\index{Carnapcopia} Galore}

It's pretty easy to trace genealogical trees from Grice to the major figures in the Anglo-American analytic philosophy of a decade ago or so.
It is perhaps less easy to do same with Carnap\index{Carnap}.
Important metaphysicians with Gricean influences include G.~Bealer\index{Bealer}, G. Myro\index{Myro} 
in the USA. Strawson\index{Strawson} and  Peacocke in the US. A search for 'metaphysics' 
and Grice retrieves more hits in  Google than it retrieves for 'metaphysics' 
and Carnap\index{Carnap}. The growth, continuing,  of Gricean bibliography is overwhelming. 
Books published in his memory, although  not necessarily from cutting-edge 
philosophers. He was after all, a  philosopher's philosopher. The secondary 
bibliography on Carnap\index{Carnap} is perhaps not  so vast. 
 
\subsection{Dialogue}

Dialogue
Much of the stimulus came from discussions with other  philosophers.
you write on your thing.
 
--
 
Note.

The diagogic. Further to the gladiatorial and the conversational, it is  
worth pointing out that the later Grice\index{Grice} grew less and less tolerant of 
'epagoge'  and more and more embracing of diagoge. The distinction is Aristotelian, 
but  Grice\index{Grice}'s twist reminds one of Carnap\index{Carnap}'s pro-attitude for dialogue as  
stimulating.
 
Grice\index{Grice}'s father had been a musician and so was his younger brother Derek.  
The trios they engaged in in Harborne gave Grice\index{Grice} a rich ... (thing) about the 
 value of cooperation: "Getting together to do philosophy should be like 
getting  to play music".
 
---
The epagoge/diagoge distinction is a basic one for Grice\index{Grice}'s metaphysical  
methodology. If evidence is, as the neo-Kantian he was, all too clearly  
necessary, one would hope however that the BASIS for this or that metaphysical  
claim (or rejection) should rest on its own virtues rather than on the 
success  or failure of having confronted its antithesis.
  
re the Vienna Circle\index{Circle!Vienna} as forum for open discussion.
 
"The Carnap Circle\index{Circle!Carnap}" -- by this we mean the Carnap\index{Carnap} convivial approach to  
philosophy. This is echoed in the Grice Circle\index{Grice!Circle}. Grice could only work  
philosophically in convivial conditions. Although he never belonged to the early  
Austin\index{Austin} club, he was a regular, indeed, Austin\index{Austin}'s favourite member of Austin\index{Austin}'s  
playgroup\index{playgroup}. On Austin\index{Austin}'s death, Grice indeed led the playgroup\index{playgroup} until his 
departure  for the States. What's more important, while in the States he would 
gather at  his "at-homes" up the Berkeley Hills.
 
---- It's less clear what sort of convivial meeetings Carnap\index{Carnap} held while at  
Santa Monica, etc.

(my impression is that he was more isolated [RBJ])

\chapter{Notes on Grice and Others}

\section{Frege\index{Frege}}

The connection between Grice\index{Grice} and Frege\index{Frege} is only recently being  
developed, mainly due to efforts by Beaney and L. Horn. Much of the subtler  
writings by Frege\index{Frege} on 'tone', 'colour' and 'force' can be given proper Gricean  
interpretations.
 
Frege\index{Frege} was of course one of the inspiring models for Carnap\index{Carnap}. For  Grice\index{Grice} 
the Frege\index{Frege} influence came much later. There is one single ref. to Frege\index{Frege} by  
Grice\index{Grice} in his "Prejudices and Predilections" and this only in connection with 
the  idea of the Frege\index{Frege}an 'sense' -- he writes, "in something like a Frege\index{Frege}an 
sense".  Grice\index{Grice} is considering, however, one of his 'metaphysical' routines. 
His Humean  projection is supposed to deliver concepts alla Frege\index{Frege}an senses. 
E.g. the concept  of negation, the concept of value, the concept of -- you 
name it.


Grice  on Frege\index{Frege}. Frege\index{Frege} was of course one of the inspiring models for 
Carnap\index{Carnap}. For  Grice the Frege\index{Frege} influence came much later. There is one single ref. 
to Frege\index{Frege} by  Grice in his "Prejudices and Predilections" and this only in 
connection with the  idea of the Frege\index{Frege}an 'sense' -- he writes, "in something 
like a Frege\index{Frege}an sense".  Grice is considering, however, one of his 
'metaphysical' routines. His Humean  projection is supposed to deliver concepts alla 
Frege\index{Frege}an senses. E.g. the concept  of negation, the concept of value, the 
concept of -- you name  it. 


\section{Hardie}\index{Hardie}

Hardie was the philosophy don at Corpus Christi, Grice's alma mater.  
Corpus had a reputation for the classics, and it was indeed as a classics  
scholar that Grice had come straight from Clifton. He would develop a 
friendship  with Hardie -- who taught him not just to argue but to play golf as 
well.

Corpus had its minus, though.
It was NOT regarded as `in' -- and the fact that it catered for `boys from the provinces' was enough to have Grice never  `hearing' of the Thursday evening group that met at the more prestigious All Souls.
So we can say that Grice between the wars was a lonesome Grice.
He managed to balance it with captaining the football team at Corpus and edit the  college philosophy journal, ``The Pelican''.
And of course he obtained a first cum laude in Lit. Hum. BA which he later turned onto a MA -- which was, as it should  be, his maximal degree.

\section{Hempel\index{Hempel} and Reichenbach\index{Reichenbach}}

Grice cites Hempel\index{Hempel} and Reichenbach\index{Reichenbach} re: 
atomistic  metaphysics in his "actions and events" PPQ, 1986.  
Grice  on Quine\index{Quine}, Grice on Chomsky\index{Chomsky}. While  Carnap would write, "The men who 
had the strongest  influence on my  philosophical thinking were Frege\index{Frege} and 
Russell\index{Russell}" in that order,  Grice could  be more irreverent. A 1913-born man 
finds the strongest influence in  a  linguist born in the 1930s: Noam Chomsky\index{Chomsky} 
and the Vienna Circle\index{Circle!Vienna} refugee:  Quine\index{Quine}.  --"I have to admit" Grice writes (or 
words), that what "I admired  in them is  their method -- never their ideas, 
which I never respected,  really". Grice's choice is particularly agonistic: 
in the sense that it provokes  sheer agony in  us: "for I never ... over 
the fact that these two genii  never agreed ON  ANYTHING". And they lived so 
close!

\section{Kant\index{Kant}}

The  double influence.
The influence of Kant\index{Kant} on Grice was a 
later one. As an Oxonian,  Kant\index{Kant} was not really taken too seriously. In this 
respect, Carnap\index{Carnap}'s education was  more traditionally philosophical (His PhD 
which Grice never attained, was  published in Kant\index{Kant} Studien). Grice first came 
to the proximities of Kant\index{Kant} via  Abbott's translation, and thus he was more 
of a minor Kantian than Carnap\index{Carnap} was,  who could savour Kant\index{Kant} in the 
vernacular! -- In 1966, Sir Peter Strawson\index{Strawson}, Grice's  former student, published his 
"Bounds of Sense" which brought Kant\index{Kant} to the  Oxonian map. Grice will later be 
invited to deliver the Kant\index{Kant} Lectures at  Stanford. But  importantly for the 
present conversation: while it was Kant\index{Kant}'s 'theoretical'  reason that only 
influenced Carnap\index{Carnap}, Kant\index{Kant}'s influence on Grice was just as strong  on the 
theoretical if not MORE in the practical realm. Grice, unlike  Carnap\index{Carnap}, looked for 
the UNITY of reason and justification in all  our attitudes: not just 
doxastic, but notably boulomaic. THis is a strong  contrast with Carnap\index{Carnap}. His 
neo-Kantianism was theoretical in nature: aimed at  epistemological problems 
concerning space/time coordinates as Carnap\index{Carnap}  found had to be 'vamped out' to deal 
with discoveries by Einstein\index{Einstein}, etc.  -- But Carnap\index{Carnap} remained an irrationalist 
in matters of value and  ethics. Grice on the other hand possibly had with 
Kant\index{Kant} the insight of the  categorial imperative: the dark starry night sky 
above us. 

\section{Quine\index{Quine} and Morris\index{Morris}}
 
The connection here is pretty interesting.
While every schoolboy knows that Quine\index{Quine} was THE logical positivist in the USA, Morris\index{Morris} also visited both Vienna and Prague.
 
The connection with Grice here is more indirect.
But typically Gricean rather than triggered, as poor Grice often was, by Strawsonians whims.
Grice had read Stevenson's Ethics and language (1944), which was an offshot of Morris\index{Morris}'s teaching.

In 1948, Grice lectured publicly for the Oxford Philosophical Society  on "Meaning": he is opposing some simplified accounts of things 
he found in  STevenson -- but which he in private lectures at Oxford --  had 
extended to  cover Morris\index{Morris}, and still earlier, Peirce.
 
The project of the unified science which was so Morrisian and  
Carnapian is less easy to detect in Grice. HIs treatment of the ethical views of Stevenson, however, shows his sympathy with a philosophy that is at least ready  and willing to be able to discourse on both matters alethic and practic.

1935 Philosophy and Logical Syntax - the text of  three lectures given in 
London in 1934  

--- where. etc.
 
---
This Grice must have been aware of. He was in his early 20s, then, but I  
would think Ayer would have made the thing known in Oxford. Grice was VERY  
concerned with what Austin\index{Austin} was doing -- before the Phoney war.

1929 Abriss der Logistik - An Introduction to Logic giving special  
attention to the theory of relations and its applications.
 
 
--- Grice expands on the pirot\index{pirot}s that karulise elatically.
 
This can potch and cotch and fed.
 
Fed is a variable for a relation ship. in Carnap\index{Carnap}'s sense.
 
Grice made this public in the Lectures on Language and REality in a  memorable summer symposium in Irvine in 1971. etc.
 
\section{Russell\index{Russell}}
 
Russell\index{Russell} has to be the lingua franca Carnap\index{Carnap}/Grice\index{Grice}.
Russell\index{Russell}'s influence on Carnap\index{Carnap}, which was actually two-way, was invaluable.
Carnap\index{Carnap} was the Russelian par excellence.

The influence of Russell\index{Russell} on Grice\index{Grice} is much more roundabout. 
Grice\index{Grice} got  an interest in Russell\index{Russell}'s modernism, as he calls it, after 
Strawson\index{Strawson} had  challenged it in Intro to logical theory (1952), which was as 
influential in  Oxford as Carnap\index{Carnap}'s Intro to Semantics had been in the USA.

1939 Russell\index{Russell} @ Chicago lecturing on meaning and  truth  

Grice's attitude towards Russell\index{Russell} is ambivalent.
The most provocative Grice could get was in his "Definite descriptions in Russell\index{Russell} and the vernacular", -- 1970.
Grice was ambivalent because Russell\index{Russell} himself was.
He had attacked Grice's student publicly ("Mr. Strawson\index{Strawson} on referring", 19--??. Mind.
While Grice disagreed with Strawson\index{Strawson} over details, he was of course going to align with Strawson\index{Strawson} against Russell\index{Russell} who was giving the Oxford school of ordinary language philosophy some good press that actually worked very well for Grice's  professional life.
So he had to be careful.
His sympathies were for a formalistic approach to langugages alla Russell\index{Russell} and as evidenced in his System G.
But his 'pro-attitude' was institutional and he felt he NEEDED to self-present as an ordinary-language philosopher, even if with very big caveats.
 
1940-41 takes visiting professorship at Harvard  where Russell\index{Russell} was giving 
the William James lectures.  

Take the FL vs NL.
Grice is clear in the ideology behind this.
There's what he calls Modernism and Neo-Modernism.
This is Russell\index{Russell} and the heirs of PM.
This INCLUDES, almost by antonomasia, CARNAP.
But then there's neo-Traditionalism, and earlier, Traditionalism.
By this Grice means Aristotelian logic (made respectable by Lukasiewicz) and Strawson\index{Strawson}'s and indeed Oxonian ordinary-language philosophical logic. 

--  the idea that '\&', 'v', --$>$ -- the connectives in the syntax of FL -- 
do  not correspond to the vernaculars of NL 'and', 'or', 'if'. Vide Carnap\index{Carnap} on 
this  for a formalist (vs informalist) view. Grice came to prefer the  
modernism-traditionalism distinction to his earlier formalim-informalism. The  
important thing here is not so much the labels for these sorts of betes 
noires,  but Grice's own brand: the way he saw or presented himself "in society" 

-- and  what he called the longitudinal history of philosophy: "a foot in 
each camp", he  jokes. But in essence, that's the aptest description of his 
position. For his  System G-- complete with a pragmatics, allows to maintain 
that the alleged  divergences between NL and FL are a matter of 
'implicature' rather than logical  form.  At the end of the Gricean  day, Grice's  
attitude towards Russell\index{Russell} is ambivalent. The most  provocative Grice could  get 
was in his "Definite descriptions in Russell\index{Russell} and the  
vernacular",  -- 1970. Grice was ambivalent because Russell\index{Russell} himself was. He 
had attacked  Grice's student publicly ("Mr. Strawson\index{Strawson} on referring", Mind.  
While Grice  disagreed with Strawson\index{Strawson} over details, he was of course going 
to  align with  Strawson\index{Strawson} against Russell\index{Russell} who was giving the Oxford school of 
ordinary   language philosophy some good press that actually worked very 
well for  Grice's  professional life. So he had to be careful. His sympathies 
were  for a  formalistic approach to languages alla Russell\index{Russell} and as evidenced 
in  his System  G. But his 'pro-attitude' was institutional and he felt he  
NEEDED to  self-present as an ordinary-language philosopher, even if with  
very big  caveats. 

\section{Tarski\index{Tarski}}

"Tarski\index{Tarski} was invited to Vienna in February 1930 and lectured on  
metamathematics, which introduced Carnap\index{Carnap} to the use of formal metalanguages.  
Discussions with Tarski\index{Tarski} and with Gödel helped Carnap\index{Carnap} towards his theory of  logical 
syntax. He disagreed with Tarski\index{Tarski} on the analytic/synthetic dichotomy,  which 
Tarski\index{Tarski} thought a matter of degree. Carnap\index{Carnap} visited Warsaw in November 1930,  
giving lectures to the Warsaw Philosophical Society, talking privately to  
Tarski\index{Tarski}" you wrote.
  
Tarski\index{Tarski}. The influence of Tarski\index{Tarski} on Grice is much more roundoubt. Again, the 
 trigger was his rebel student, Strawson\index{Strawson}. In a famous talk at Bristol,  
much much later revisited by Warnock\index{Warnock} ("Bristol Revisited", in refs.) Strawson\index{Strawson}  
opposed Austin\index{Austin}'s correspondence theory of truth. Strawson\index{Strawson} argued that 'is  
true' was illocutionary in nature. This puzzles historians of philosophy as 
it  would have been natural to to think AUSTIN would have embraced such a 
view.  Yet, Austin\index{Austin}, like Grice, were traditionalists in these. Grice dedicates 
a  whole section of his third William James lecture to a discussion of 
Tarski\index{Tarski}.  Grice ends up endorsing a neo-Tarskian view. The sentence of Tarski\index{Tarski} 
becomes an  utterance in Grice, but the basic Tarski\index{Tarski}an idea of 
'satisfactoriness' is  retained by Grice --. He even goes on to propose some implicatural 
solutions to  formal problems having to do with blind uses of the metalogical 
predicate  'true' in NL: "What the policeman said was true". Much later, when 
 generalising what he now called "alethic satisfactoriness", he introduces  
special apparatus to his System G to deal with satisfactoriness in realms 
other  than the alethic, notably the practical. 

\section{White}\index{White, A.R.}

In 1961 Grice participated in a symposium with A. R. White -- as second  
symposiast -- organised by the Aristotelian Society in Cambridge (Braithwaite  
was the Chair) and which got published in the Proceedings.

\section{Wittgenstein\index{Wittgenstein}}
 
“Not my man”. Wittgenstein\index{Wittgenstein}, or Witters as the more irreverent Grice would have it.
Again, Carnap\index{Carnap}'s contact was literally first hand.
Grice  was a closet Wittgensteinian.
One reads his "Method in philosohical psychology" and finds whoe passages verbatim from Witters without a 
recogntion (well, once).
Grice would often quote from Witters.
He is listed as an  A-philosopher against which Grice reacts in the William James lectures.

First Wittgenstein\index{Wittgenstein}, Last Wittgenstein\index{Wittgenstein}, Middle Wittgenstein\index{Wittgenstein}. The 
refinements of  Witters' philosophy are important for our reconstruction of the Griceland\index{Griceland} of Carnapcopia\index{Carnapcopia}.
--- The first Wittgenstein\index{Wittgenstein} falls squarely in the FL project of Modernism\footnote{Though the Tractatus has only one foot in that camp.}.
The middle Wittgenstein\index{Wittgenstein} is the critical ie. crisis -- Witters.
The later Wittgenstein\index{Wittgenstein} possibly had  a stronger influence on Grice 
than on  Carnap\index{Carnap}.

\chapter{Topics}

\section{Formal v. Natural Languages}
 
Grice\index{Grice} is clear in the ideology behind this.
 
There's what he calls {\it Modernism} and {\it Neo-Modernism}.
 
This is Russell\index{Russell} and the heirs of PM
 
  -- this INCLUDES, almost by antonomasia, CARNAP.
 
 
Then there's
 
  neo-Traditionalism
 
  and earlier, Traditionalism
 
  By this Grice\index{Grice} means Aristotelian logic (made respectable by  Lukasiewicz)
  and Strawson\index{Strawson}'s and indeed Oxonian ordinary-language philosophical  logic.
 
  -- the idea that '\&', 'v', --$>$ -- the connectives in the  syntax of FL
  do not correspond to the vernaculars of NL 'and', 'or', 'if'.
 
  vide Carnap\index{Carnap} on this for a formalist (vs informalist) view.
  

NOTE: Grice\index{Grice} came to prefer the modernism-traditionalism  distinction to his 
earlier formalim-informalism. The important thing here is not  so much the 
labels for these sorts of betes noires, but Grice\index{Grice}'s own brand: the  way he 
saw or presented himself "in society" -- and what he called the  longitudinal 
history of philosophy: "a foot in each camp", he jokes. But in  essence, 
that's the aptest description of his position. For his System G--  complete 
with a pragmatics, allows to maintain that the alleged divergences  between 
NL and FL are a matter of 'implicature' rather than logical form. 
 
\subsection{USA Good for Logic}

 Carnap\index{Carnap} observes that the state of Logic teaching in  the USA was much better than in Europe, 
 
--- this is echoed by Grice.
 
He confesses publicly that part of his professional reason -- one wonders:  
couldn't he just write? -- for moving to the States was "closer contact 
with  logicians".
 
Grice is already in 1969 quoting from Boolos, Parsons, Myro\index{Myro}, Mates\index{Mates}, etc.  
This is the beginning of his System G. Putnam would actually represss Grice's formalistic fancies short: "You are too formal" he'd complain.
Grice did not fit the brit steroretype of the good humanistic Oxford don of the 
Austinian ilk. Of  course they were wrong.
Grice fit NO stereotype.

\section{Metaphysics}

how many angels could dance on the point of a needle." 
 
----
 
Grice cites the same example.
In "Prejudices and Predilections".
 
To consider:
 
-- Note
 
-- Longitudinal unity/latitudinal unity.
Metaphysics is important for Grice as a manifestation of philosophy's latitudinal unity: everything connects and metaphysics is the ground-floor discipline: the theory-theory or first philosophy.
But unlike Carnap\index{Carnap}, Grice was an inveterate historicist\footnote{Believer in the importance of the history of philosophy?}. 
He rejoiced in what he called the longitudinal unity of philosophy which Carnap\index{Carnap} repudiated.
Grice found inspiration in time-honoured philosophies of Socrates, Plato, Aristotle\index{Aristotle}, Descartes, Leibniz, Kant\index{Kant}, (Grice to the) Mill, etc. 
These he called 'the great'.
He was less tolerant with the minors -- among which he provocatively lists Witters!
 
--- etc. 

wiki ref. by you mentions Hempel\index{Hempel}
 
Grice cites Hempel\index{Hempel} and Reichenbach\index{Reichenbach} re: atomistic metaphysics in his ``actions and events'' \cite{grice86}.
To provide quotes by me.
 
\section{Betes Noires}

\subsection{phenomenalism}

It is surprising Grice lists Phenomenalism as a bete noire. Carnap\index{Carnap}'s  
tolerance for phenomenalism was well known. His first hand encounters with  
Goodman couldn't have been but positive. Grice's brand of phenomenalism was of  
an earlier vintage. None of the sophistification of Goodman. Grice's ideas of 
 phenomenalism were either the rather rough notes by Ayer and I. Berlin in 
a  rather influential paper in Mind in the 1930s. In the postwar period, 
Grice  would rely on work by G. A. Paul, "Is there a problem about sense data?" 
and  Austin\index{Austin}'s refutation of Ayer. Etc.

Grice and Carnap\index{Carnap} on physics. Carnap\index{Carnap} on Einstein\index{Einstein}. Grice on Eddington's two  
tables.
 
also:
 
\subsection{inductivism}

Strange Grice does not list this as bete noire. And  
confirmationism (Carnap\index{Carnap}'s reply to Popper's falsificationism). The most  technical 
Grice gets on this is his scattered refs. to Kneale (Ind. and Prob) in  Reply 
to Richards\index{Richards}, and his treatment of Davidsonian's probability operators in  
various publications vis a vis generalisations to desirability operators: 
 
Grice, Probability, Desirability and Mood Operators, 1973.
Grice Aspects of Reason. On Probably, as a sentence modifier. etct
 
this above vis a vis your ref. to Carnap\index{Carnap}/Kemeny
 
 
Bar-Hillel is cited by Chapman\index{Chapman} in connection with a possible influence of  
Carnap\index{Carnap} on Grice. Bar-Hillel had worked with Carnap\index{Carnap} and comes out with this 
idea  that the divergence between FL and NL is in the 'implicature'. He uses  
'implication' and it's the idea of pragmatics as the wastebasket of  
philosophers. Grice on metaphysical excrescences. etc. 
 
\subsection{Diagogism}

The  diagogic. Further to the gladiatorial and the 
conversational, it is  worth  pointing out that the later Grice grew less and less 
tolerant of 'epagoge'   and more and more embracing of diagoge. The distinction is 
Aristotelian,  but  Grice's twist reminds one of Carnap\index{Carnap}'s pro-attitude for 
dialogue  as   stimulating. Grice's  father had been a musician and so was 
his younger brother Derek.  The trios  they engaged in in Harborne gave Grice 
a rich ... (thing) about the value of  cooperation: "Getting together to do 
philosophy should be like getting  to  play music". The epagoge/diagoge 
distinction is a basic one for Grice's  metaphysical  methodology. If evidence 
is, as the neo-Kantian he was, all  too clearly  
necessary, one would hope however that the BASIS for this  or that 
metaphysical  claim (or rejection) should rest on its own virtues  rather than on 
the success  or failure of having confronted its  antithesis. 
Grice  on Kant\index{Kant}. The  double influence.The influence of Kant\index{Kant} on Grice was a 
later one. As an Oxonian,  Kant\index{Kant} was not really taken too seriously. In this 
respect, Carnap\index{Carnap}'s education was  more traditionally philosophical (His PhD 
which Grice never attained, was  published in Kant\index{Kant} Studien). Grice first came 
to the proximities of Kant\index{Kant} via  Abbott's translation, and thus he was more 
of a minor Kantian than Carnap\index{Carnap} was,  who could savour Kant\index{Kant} in the 
vernacular! -- In 1966, Sir Peter Strawson\index{Strawson}, Grice's  former student, published his 
"Bounds of Sense" which brought Kant\index{Kant} to the  Oxonian map. Grice will later be 
invited to deliver the Kant\index{Kant} Lectures at  Stanford. But  importantly for the 
present conversation: while it was Kant\index{Kant}'s 'theoretical'  reason that only 
influenced Carnap\index{Carnap}, Kant\index{Kant}'s influence on Grice was just as strong  on the 
theoretical if not MORE in the practical realm. Grice, unlike  Carnap\index{Carnap}, looked for 
the UNITY of reason and justification in all  our attitudes: not just 
doxastic, but notably boulomaic. THis is a strong  contrast with Carnap\index{Carnap}. His 
neo-Kantianism was theoretical in nature: aimed at  epistemological problems 
concerning space/time coordinates as Carnap\index{Carnap}  found had to be 'vamped out' to deal 
with discoveries by Einstein\index{Einstein}, etc.  -- But Carnap\index{Carnap} remained an irrationalist 
in matters of value and  ethics. Grice on the other hand possibly had with 
Kant\index{Kant} the insight of the  categorial imperative: the dark starry night sky 
above us. 
Grice  on Frege\index{Frege}. Frege\index{Frege} was of course one of the inspiring models for 

\subsection{Carnap\index{Carnap}}

For  Grice the Frege\index{Frege} influence came much later. There is one single ref. 
to Frege\index{Frege} by  Grice in his "Prejudices and Predilections" and this only in 
connection with the  idea of the Frege\index{Frege}an 'sense' -- he writes, "in something 
like a Frege\index{Frege}an sense".  Grice is considering, however, one of his 
'metaphysical' routines. His Humean  projection is supposed to deliver concepts alla 
Frege\index{Frege}an senses. E.g. the concept  of negation, the concept of value, the 
concept of -- you name  it.

\subsection{Secularism}

Grice's religious inclinations are harder to pin down than Carnap\index{Carnap}’s.
Chapman\index{Chapman} in  fact sounds rather authoritative when she states that 
"Grice lost all faith by  the age of 19" or something. You can never be so 
sure. Chapman\index{Chapman} redeems herself  by noting the very many religious references, 
eschatological and Biblical, in  Grice's various writings. While a disbeliever 
in God, Grice liked to play God.  Borrowing on this idea by Carnap\index{Carnap} of 
pirot\index{pirot}s which karulize elatically, Grice  founds a full programme in the vein of 
the ideal-observer. What we would do, as  God, to secure the survival of 
pirot\index{pirot}s. While not religious in nature, it has a  religious tone that is absent 
in the writings of Carnap\index{Carnap} in any  respect. 
Scholasticism  rears its pretty face. how many angels could dance on the 
point of a needle.  Grice cites the same example. In "Prejudices and 
Predilections". To consider:  Longitudinal unity/latitudinal unity. Metaphysics is 
important for Grice as a  manifestation of philosophy's latitudinal unity: 
everything connects and  metaphysics is the ground-floor discipline: the 
theory-theory or first  philosophy. But unlike Carnap\index{Carnap}, Grice was an inveterate 
historicisit. He rejoiced  in what he called the longitudinal unity of 
philosophy which Carnap\index{Carnap} repudiated.  Grice found inspiration in time-honoured 
philosophies of Socrates, Plato,  Aristotle\index{Aristotle}, Descartes, Leibniz, Kant\index{Kant}, (Grice to 
the) Mill, etc. These he called  'the great'. He was less tolerant with the 
minors -- among which he  provocatively lists Witters! 

\subsection{Phenomenalism}

It is surprising Grice lists Phenomenalism as a bete 
noire. Carnap\index{Carnap}'s tolerance  for phenomenalism was well known. His first hand 
encounters with Goodman  couldn't have been but positive. Grice's brand of 
phenomenalism was of an  earlier vintage. None of the sophistification of 
Goodman. Grice's ideas of  phenomenalism were either the rather rough notes by Ayer 
and I. Berlin in a  rather influential paper in Mind in the 1930s. In the 
postwar period, Grice  would rely on work by G. A. Paul, "Is there a problem 
about sense data?" and  Austin\index{Austin}'s refutation of Ayer. Etc. 
Inductivism.  Grice and Carnap\index{Carnap} on physics. Carnap\index{Carnap} on Einstein\index{Einstein}. Grice on 
Eddington's two  tables. Also: 
inductivism.  Strange Grice does not list this as bete noire. And 
confirmationism (Carnap\index{Carnap}'s  reply to Popper's falsificationism). The most technical 
Grice gets on this is  his scattered refs. to Kneale (Ind. and Prob) in Reply 
to Richards\index{Richards}, and his  treatment of Davidsonian's probability operators in 
various publications vis a  vis generalisations to desirability operators:  
Grice, Probability, Desirability and Mood  Operators, 1973. Grice Aspects of 
Reason. On Probably, as a sentence modifier.   

\subsection{Diagogism}

 The  diagogic. Further to the gladiatorial and the 
conversational, it is  worth  pointing out that the later Grice grew less and less 
tolerant of 'epagoge'   and more and more embracing of diagoge. The distinction is 
Aristotelian,  but  Grice's twist reminds one of Carnap\index{Carnap}'s pro-attitude for 
dialogue  as   stimulating. Grice's  father had been a musician and so was 
his younger brother Derek.  The trios  they engaged in in Harborne gave Grice 
a rich ... (thing) about the value of  cooperation: "Getting together to do 
philosophy should be like getting  to  play music". The epagoge/diagoge 
distinction is a basic one for Grice's  metaphysical  methodology. If evidence 
is, as the neo-Kantian he was, all  too clearly  
necessary, one would hope however that the BASIS for this  or that 
metaphysical  claim (or rejection) should rest on its own virtues  rather than on 
the success  or failure of having confronted its  antithesis. 


\subsection{Phenomenalism}

It is surprising Grice lists Phenomenalism as a bete 
noire. Carnap\index{Carnap}'s tolerance  for phenomenalism was well known. His first hand 
encounters with Goodman  couldn't have been but positive. Grice's brand of 
phenomenalism was of an  earlier vintage. None of the sophistification of 
Goodman. Grice's ideas of  phenomenalism were either the rather rough notes by Ayer 
and I. Berlin in a  rather influential paper in Mind in the 1930s. In the 
postwar period, Grice  would rely on work by G. A. Paul, "Is there a problem 
about sense data?" and  Austin\index{Austin}'s refutation of Ayer. Etc. 

\subsection{Inductivism}

Grice and Carnap\index{Carnap} on physics. Carnap\index{Carnap} on Einstein\index{Einstein}. Grice on 
Eddington's two  tables. Also: 
inductivism.  Strange Grice does not list this as bete noire. And 
confirmationism (Carnap\index{Carnap}'s  reply to Popper's falsificationism). The most technical 
Grice gets on this is  his scattered refs. to Kneale (Ind. and Prob) in Reply 
to Richards\index{Richards}, and his  treatment of Davidsonian's probability operators in 
various publications vis a  vis generalisations to desirability operators:  
Grice, Probability, Desirability and Mood  Operators, 1973. Grice Aspects of 
Reason. On Probably, as a sentence modifier.

\subsection{Empiricism}

This is the first bête noire. But Grice fails to mention his 
twin: Rationalism.  There is some mystifying about ‘empeira’, as the Greeks 
used the word. Peira is  ultimately a tribunal. There is nothing to scary 
about having a doctrine based  upon the idea of a tribunal.  

\subsection{Extensionalism}

While Grice lists this, he fails again to mention the 
twin: Intensionalism.  Grice was not necessarily attracted to Intensionalism. So 
his rejection of  Extensionalism is indeed a case of his epagoge, trying to 
refute a thesis rather  than provide positive evidence for its contrary. 
The root ‘tensio’, that is  common to both bêtes noires is an interesting 
one, and related to the deeper  questions about meaning. 


Since we are currently examining this vis a vis Carnap, we speak
of 'Grice morphed onto an intensional isomorphist'. The early Grice was
not. He would say that the utterer who utters,

p --> q

is the same utterer who utters

-p v q

These are intensionally nonisomorphic (perhaps -- but this label is
best applied to predicate calculus). What U means, however, is the
same. As a defender of truth-functionalism, the early Grice is an
extensional isomorphist.

The later Grice finds this a protectionist measure for the commodity of
an explanation that does use 'intensional isomorphism':

Grice then says he'll select "Extensionalism", which he defines as

"a position imbued with the
spirit of Nominalism [another bete noire]
and dear both to those who feel that (b) is no more
informative an answer to the question (a)
than would be (d) as an answer to
(c)."

Scenario I:


a: Why is a pillar box called 'red'?
b: Because it is red.


Scenario II:


c: And why is that person called 'Paul Grice'?

d: Because he is Paul Grice.


Cfr. Geary's daughter:

Geary: Why are pigs dirty?
Daughter: Because they are pigs.

---

The picture of Extensionalism Grice presents is clear enough. It is

"a world of PARTICULARS
as a domain
stocked with tiny pellets ...
distinguish[ed] by the clubs to which they
belong".

He had a thing for clubs. He would define Austin's club as "the club
for those whose members have no class" (or rather 'for those whose
classes have no members')

And cfr. The Grice Club, extensionally and intensionally defined. Cfr.
Jones, "Carnap Corner", next blog.

Grice goes on:

"The potential consequences of the possession
of in fact UNEXEMPLIFIED features [or properties]
would be ... the same."

Grice then turns to a pet topic of his, "Vacuity". He had dedicated his
contribution to the anti-dogmatist of them all, Quine, with an essay
on "Vacuous Names and Descriptions" published 1969 in Hintikka/Davidson
(we need a reprint of that, urgently!). And he knew what he was talking
about. We have discussed this with Roger Bishop Jones elsewhere
('Vacuity' in Hist-Anal)

Vis a vis his critique of Extensionalism (and where is Grice's
diagogism when one wants it?) one may want to

"relieve a certain VACUOUS predicate ... by exploiting the
NON-VACUOUSNESS of other predicates which are constituents in the
definition of the original vacuous predicate."

This is good, because his "Vacuous Names" focuses on, well, names,
rather than predicates or descriptors. Here his approach is more, shall
we say, substantial: connotative, rather than denotative.

Grice exemplifies this with two allegedly vacuous (i.e.
non-extensional) predicates:

1

-- " ... is married to a daughter of an English queen and a pope"


2

-- " ... is a climber on hands and knees of a 29,000 foot mountain."


The second has echoes in "Vacuous Names"

That's Marmaduke Bloggs.

Marmaduke Bloggs is indeed a climber on hands and knees of a 29,000 ft.
mountain. The Merseyside Geographicall Society was so impressed that
they had this cocktail in his honour. But he failed to turn up.

"He is not at the party"

"Who isn't?"

"Marmaduke Bloggs"

"He doesn't exist. He was invented by the journalists".

--- etc. Cfr. Horn on a similar passage by Lewis Carroll on this -- in
his Symbolic Logic.

----

Grice is interested in what makes Marmaduke Bloggs an 'elusive chap, if
ever there was one':

"By appealing to different

"relations" [now, alla Carnap, Abriss]

to the 'primitive' predicates, one can claim is

such distinct relation,

rather than the empty set beloved of extensionalists
which each vacuous predicate is made equivalent to."

But his objection to this move has to do with what he feels an
adhocness in defining the relations as involving again, NON-VACUOUS
predicates.

-- the relevant passage is available as google books --. (p. 70).

A SECOND TACK. (He is looking for harder and harder tacks)

A second way out to the alleged problem involves 'trivial' versus
'non-trivial' explanations.

Recall that for Grice all betes noires trade on the untradeability of
explanations. They want to restrict the realm of explananda. They
regiment our hopes for explanation. (Hume's fork or his is-ought
problem would be similar blockages).

Grice has it in clear enough terms:

"the explanatory opportunities for vacuous predicates depend on their
embodiment in a system".

His caveat here is purely ontological, or shall we say eschatological:

"I conjecture, but cannot demonstrate,

that the only way to secure such a

system would be to confer

SPECIAL ONTOLOGICAL privilege

upon the ENTITIES of PHYSICAL SCIENCE..."

-- But that's Eddington "non-visible" 'table'. And he had a foot on
both camps here, or rather, he knew that, historically, he was and will
forever be seen as a proponent of Austinian Code: the idea that there
is wisdom in folk: the cathedral of laerning is Science but it's also
Common Sense, as expressed in our ordinary ways of talking (ta
legomena).

And he seems to be allowing that sometimes we do engage in talk where
the entities of things OTHER than physical science are relevant too.
Notably stone-age physics. This is possibly a thing of the past now,
but most of the English ways of talking (if grammar is going to be 'a
pretty good guide to logical form') are embued with it, and they would
be just rejected en bloc if only CONTEMPORARY physical science, true
physical science, is deemed articulatory only.

Grice notes at this point:

"It looks AS IF states of affairs in the
... scientific world need, for credibility,
support from the vulgar world of ORDINARY OBSERVATION..." --

Eddington's visible 'table' to which he explicitly refers in his
little quoted, "Actions and Events" (Pacific Philosophical Quarterly,
1986).

And this, he feels would be an 'embellisment' in need of some
justification. In other words, if the real table (of Eddington) is not
made of matter, but of wavicles, why is it that a wavicle be deemed as
a more fundamental entity than, say, 'table as we knew it'?

\subsection{Functionalism}

Again it’s difficult to see what twin bête Grice is having 
in mind. Not  Formalism, but indeed, the standard antonymy is form versus 
function. As per  Aristotle\index{Aristotle}. Thus, think of ‘cabbage’. What is the function 
of a cabbage? To  cabbagise, Grice suggests (2001 – of cabbages and kings). 
Oddly, while Grice  sees functionalism as a bête noire, the Original 
Christian found “Formalist”. We  read from Bunyan: “And as [Christian] was troubled 
thereabout, he espied  two men come tumbling over the wall, on the left 
hand of the narrow way; and  they made up apace to him. The name of the one was 
Formalist”. But what Bunyan  is having in mind of course is mere Dogmatism. 
Something to consider here is  that Grice may have thought that 
Functionalism is Not Enough. Thus, when B.F.~Loar\index{Loar}, for example defines simple things 
like Grice’s conversational maxims as  “empirical generalisations over 
functional states” (Mind and Meaning, Cambridge  University Press) we know that 
that will leave Grice cold. He is going to look  for a deeper explanation 
than that. 

\subsection{Materialism}

When  it comes to this ism, again it’s interesting not just to 
consider the twin, but  the root. After all, Grice was an Aristotelian, and 
matter is all there is for  Aristotle\index{Aristotle}: the hule. So ‘Hylism’ would be a 
better term. At least less hybrid.  By rejecting Hylism, Grice may be again 
saying that it is insufficient rather  than wrong. After all for Aristotle\index{Aristotle}, 
who played with ‘hule’, we need the basic  compound: matter and form: hyle 
and morph. But doesn’t “hylomorphism” sounds as  a scarier bete than Hylism? 

\subsection{Mechanism}

Indeed  one bete noire destroyed by another. Antonym here: 
Telism. I.e. the idea that  final causes are all we need.  The  root of mechanism 
in the Greek idea of mekhane is not as bad after all.   
Naturalism. Twin  bete noire: non-naturalism, as per Moore’s 
non-naturalistic fallacy which Grice  rejects in “Conception of Value”: ‘value’ cannot 
be non-natural, as Moore would  think about this. Grice’s sympathies were for 
constructed entities – and  constructivist approaches to value in 
particular along the lines made familiar  in Oxford by J. L. Mackie\index{Mackie!J.L.} and Philippa Foot\index{Foot!Phillipa} 
(both authors he mainly focuses on  in (1991)). 
Nominalism. Is he  thinking of Realims as the twin bete noire? One thinks 
so. But Grice came only  at a very late stage to become a realist, and with 
caveats. He never was the  closet realist that Davidson\index{Davidson} was (Speranza\index{Speranza} here 
recalls his conversations with  Davidson\index{Davidson} on this on occasion of Speranza\index{Speranza} 
delivering his very first Gricean paper  – in, of all places, Buenos Aires!)(vide 
Speranza\index{Speranza} 1989). Grice was no realist in  that he thought that realism 
needed some transcendental justification: our  beliefs have to be true because 
they would be useless if they weren’t. Similarly  he never abandoned the idea 
that only via sense-data do material objects enter  our cognitive schemes: 
objects threaten and nourish us: sense data don’t. If the  opposition is with 
realia over universalia, I would think Grice was enough of an  occamist 
(would you modify Occam’s razor if you were not?) to throw the baby  with the 
water. 
Phenomenalism. The  funny bit about this that here we do have a case where 
both a bete noire and her  twin (Physicalism) both scare the Christian in 
Grice! Strictly, the perfect  antonym, at least for Kant\index{Kant}, for this would be 
Noumenalism, as per “Philosophy  4”. A caveat however is in order: noumenon 
does not really oppose phainomenon,  at least for the Grecians. Noumenon is 
the realm of thought. So it’s Mentalism  which the proper bete noire would 
be. Incidentally, Mentalism can claim however,  the status of the twin for the 
bete noire of Physicalism,  too. 
Positivism.  However, it may do here to consider  betes noires that one 
usually associates with Positivism, too. Inductivism,  Confirmationism, etc. 
What about Popper’s falsificationism? Is this a twin bete  noire here? It 
seems so. Strictly, the anonym for falsificationism would be  confirmationalism. 
 
\subsection{Physicalism}

The  source for this all “Prejudices and predilections”, now 
safely deposited at BANC  90/35c, was triggered by a request by Grandy\index{Grandy} and 
Warner, so Grice may have been  in a hurry. For indeed, a hasty etymological 
ramble leads you to conclude that  Naturalism and Physicalism are two ugly 
rearing heads of the SAME bete. “phusis”  of the Grecians was the Nature of 
the Romans.   
Reductionism. His  twin, irreductionism, is just as scary. It would seem 
that indeed, analysis is  what bridges C and G. So that an irreducible feature 
is a non-sequitur.  Perhaps the clearest Grice gets on this  is his reply 
to Davidson\index{Davidson} on intending. Davidson\index{Davidson}, who was perhaps Grice’s soul  mate more 
than anyone else while at Berkeley, Grice found reductionist. And this  in 
the sense, that Davidson\index{Davidson} went for the bigger picture, failing to see the  
leaves for the tree. His account of volition, for example, Grice finds obscenely 
 reductionist. Grice has manifestos to the effect: (words). “Not that 
Davidson\index{Davidson}’s  picture is wrong; it’s just too simple: surely when I inspect my 
mind and see  all the volitional things that I associate it with, there’s no 
big core picture  in terms of the desirability operator that Davidson\index{Davidson} is 
claiming is basic”. So  the anti-reductionist Grice is wanting to say that the 
richness of the  phenomenon (or better,  phenomena)  in question has to be 
given proper due before going for the easy way out of a  reductive (even) 
explanation. 
Scepticism. Twin  bete noire? Dogmatism, just as bad. Alla Grandy\index{Grandy} (cited by 
Grice, \cite{grice89}:xix) we can  say that it’s Underdog-matism, with Grice, but yet. 
The idea behind ‘scepsis’  was not that bad: it was a thorough (but not 
perhaps that thorough, for Grice)  examination. His caveats for anti-sceptical 
views however were clear in those  essays Grice cared to reprint in \cite{grice89} 
that bear titles like, “Common sense and  scepticism”. As a autobiographical 
reminder, Speranza\index{Speranza} should here disclose that  indeed he felt very close to 
Grice when he realised that the topic of scepticism  had fascinated Grice, as 
it had fascinated Speranza\index{Speranza}, from a tender age. “Common  sense and scepticism”
 Grice dates as 1946. I.e. the third or fourth piece he  produced. His 
first being “Negation” 1938. Grice’s arguments therein are  directed towards 
Norman Malcom’s assuming the Moorean position of the Ordinary  Languager 
versus Phyrro. 

I enjoyed your three points re: Carnap\index{Carnap}'s thing. Will see what I can say  
about it from G's perspective. (Oddly Grice uses initials a lot: he has M, A, 
R,  U, G., in his Retrospective Epilogue. So surely he would not object to 
C. He  distinguishes between G and G* but I forget what the distinction 
amounts  to.
 
\section{Pirot Talk}

More  on the pirot\index{pirot} talk. Vis a vis Carnap\index{Carnap}’s focus on relations in his 1929  
Abriss der Logistik, Grice expands on the pirot\index{pirot}s that karulise elatically. 
These  can potch and cotch and fed – where Fed is a variable for a relation 
ship. in  Carnap\index{Carnap}'s sense. Grice made this public in the Lectures on Language 
and Reality  in a  memorable summer symposium in Irvine in 1971. etc. 
Can  pirot\index{pirot}s implicate?

Bar-Hillel  is cited by Chapman\index{Chapman} in connection with a 
possible influence of Carnap\index{Carnap} on Grice.  Bar-Hillel had worked with Carnap\index{Carnap}
and comes out with this idea that the  divergence between FL and NL is in the 
'implicature'. He uses 'implication' and  it's the idea of pragmatics as 
the wastebasket of philosophers. Grice on  metaphysical excrescences. etc.  

\section{Carnucopia}

Conceptual  
 Map
                     to
 
CARNUCOPIA and   GRICELAND
 
      (also one page only)
 
 
                       .
                       .
               Aristotle\index{Aristotle}
 
 
          Hume is Where the  Heart Is
 
                  Kant\index{Kant}
 
             (Kantotle, Ariskant)
 
 
                   .
 
Cornucopia
 
neo-Kantianism
 
                                  Oxford Hegelianism
 
                                           Ryle sends Ayer to 
                                               Vienna
 
 Wiener Kries
 
 
                                     Ayer returns from Vienna
                              Splits from Austin\index{Austin}'s playgroup\index{playgroup}
 
                     The War
 
                                     Grice influential
                                      in Austin\index{Austin}'s 'kindergarten'
 
                          
                                      Grice  travels to the USA
                                       to deliver the William James
                                       and puns on Heidegger\index{Heidegger} alla
                                       Carnap\index{Carnap}.
                                    "Heidegger\index{Heidegger} is the greatest living
                                     philosopher, if you can take
                                     me seriously" (\cite{grice89}:i)
 
 
   Carnap\index{Carnap}       
                                     Grice introduces his pirotological
                                      programme in the APA presidential
                                      address (Pacific Division) for 1975.
 
 
      Carnap\index{Carnap} dies
 
                                     Grice gives the Carus Lectures
                                           (published as Grice 1991)
 
                                    Grice's Aspects of Reason lectures
                                            published 2001.
 
 
 
------
 
 
Then one page about 
 
                The Place of Metaphysics in
 
 
CARNUCOPIA                                   GRICELAND
 
 
                                                       ontologia
 
 
                                       generalis                           
specialis
 
                                    Theory of Categories
 
                                                                            
   (a) cosmologia
                                             vs.
 
                                                                            
   (b) psychologia rationalis
                                     Eschatology
 
                    
         
 
-----
 
Then
 
                             one page for
 
 
 
                               Lingua Franca
 
               (Carnap\index{Carnap} and Grice find they can hold
               a conversation in a lingua franca)
 
                         System G-HP
 
                    here I will provide 
                   the 13 items, I think they are
                 which Grice thinks are constitutive
                    of a good formal language
                    (\cite{grice89}:ii -- first two pages)
 
 
 
 We are going to use a simplified semantics for
 
         "Pirots karulize  elatically"
 
---- pirot\index{pirot} P  x is a pirot\index{pirot} iff
    karulize K   K as  a class. = 
    names of pirot\index{pirot}s
    the logical form of adverbs, the elatically of  Carnap\index{Carnap}.
    quantifiers: all pirot\index{pirot}s karulize elatically  understood.
          scalar implicature  of "Some" (Some pirot\index{pirot}s karulise elatically; 
some not)
    etc.
    essential properties.
        pirot\index{pirot}s MUST karulise  elatically
 
 
 
          FL                                          NL
 
 
                  syntactics
 
                 
 
                definition of 'proof'
 
                
              
                 Grentzen-type rules
 
 
           --------------------------------------------
 
                   Semantics
 
 
Carnacopian                        Griceland\index{Griceland}
Pragmatics                       Pragmatics
 
the realm 
of 
pragmatism                                -- the realm of implicature
 
 
 
 
-----
 
The Actual Conversation.
 
For this Jones and Speranza\index{Speranza}
met online and recorded their online
dialogue. The result is as follows
 
 
CARNAP/JONES. Hello.
 
GRICE/SPERANZA. Hello
 
.
.
.
 
 
 
 
 
GRICE-Speranza\index{Speranza}. Whatever
 
CARNAP-Jones. Whatever
 
             --  they part.
 
---
 
----
 
 
----
 
 
For the NOTES
 
 
--- Waismann. Grice, as a Brit, would be more familiar with the views of  
Waismann, the member of the Vienna Circle\index{Circle!Vienna} who had made it to England. 
 
--- Grice lectured on metaphysics for the BBC. The result is in D. F.  
Pears, The nature of metaphysics, 1957. This is vintage Grice. I.e. Grice  
self-presenting as a metaphysician as 'ambitious' as Kantotle was. Metaphysics  
was starting to cease being the term of abuse he felt Ayer had turned  onto.

  
---- Grice, unlike Carnap\index{Carnap}, was professionally involved in DEFENDING  
metaphysics. He delivered annually two courses on Metaphysics. Usually with G.  
Myro\index{Myro}. Naturally, he felt the professional defence of the discipline was what 
was  professionally and institutionally required from him, especially after 
becoming  Full prof. at Berkeley in 1975. 
 
---- Grice's student, Sir Peter Strawson\index{Strawson} had become by 1968 the standard  
for metaphysical theory as understood in England and Oxford. As Waynflete  
professor he became more and more interested in neo-Kantian foundations for 
the  discipline.
 
---- Myro\index{Myro} was a special influence in Grice's metaphysical thought.  
Originated educated at Oxford in Balliol, he had a strict logical background and  
inspired Grice in much of what transpired as Grice's System Q, which Myro\index{Myro} 
later  re-baptised System G -- "in gratitude to Paul Grice for the original 
idea"
 
---- The syntax of System G makes use of scope devices to allow for  
pragmatic implicature. These undertake two forms:
(i) the use of square brackets to indicate common-ground status. Grice  
provides formal rules for the introduction and elimination of square brackets 
in  \cite{grice89}:xviii)
(ii) the use of numerical subscripts (in Vacuous Names). This allows for  
the scope maximal readings of formulae but allowing as well for a minimal  
reading upon a simple numerical transformation. 
 
----
 
Grice and Strawson\index{Strawson} had defended the analytic-syntetic distinction in "In  
defense of a dogma" but Grice grew sceptical as to the success of that 
defense  (of an underdogma, as he later had it). He grew more and more pragmatist 
towards  the viability of the need to postulate the distinction.
 
---
 
Carnap\index{Carnap}'s "meaning postulates" have affinities with Grice's notion of  
'entailment' which he drew from Moore. Moore, while not a formal logician, is  
responsible for this coinage, which appealed Grice, as he would contrast, in 
his  System G, only entailment with 'implicature'. There would be no place 
for  'presupposition' or truth-value gaps in this scheme, as there is in 
Strawson\index{Strawson}.  The metaphysical implications of Grice's choice of a bivalent standard 
 interpretation of System G are obvious.
 
Grice spoke excellent English. As Clifton and Corpus Christi educated, he  
found easily crowds of followers, especially in America, in younger 
philosophers  who had grown tired of their dogmatic empiricist teachers. Grice 
brought a  breath of fresh air. This is ironical as seen from the other side of 
the 'pond',  in that the breath of fresh air can be looked, in a sort of 
inverted snobbery,  as an irreverent reactionary dogmatism!  On the other hand, 
Carnap\index{Carnap} was  perhaps less influential among the younger philosophers. 
 
It's pretty easy to trace genealogical trees from Grice to the major  
figures in the Anglo-American analytic philosophy of a decade ago or so. It is  
perhaps less easy to do same with Carnap\index{Carnap}.
 
Important metaphysicians with Gricean influences include G. Bealer\index{Bealer}, G. Myro\index{Myro} 
 in the USA. Strawson\index{Strawson} and Peacocke in the US.
 
---- The growth, continuing, of Gricean bibliography is overwhelming. Books 
 published in his memory, although not necessarily from cutting-edge  
philosophers. He was after all, a philosopher's philosopher. The secondary  
bibliography on Carnap\index{Carnap} is perhaps not so vast.

\section{Dialogue}

I.e. your gladiatorial thing as thesis.
With a reference to Aristotle\index{Aristotle} 'epagoge' which I think will look cute in Greek letters.
 
Then the second paragraph is your conversation thing -- with Aristotle\index{Aristotle}'s  
"diagoge" in Greek letters. which will look cute.
 
The synthesis is: sort of what you say about this being an "imaginary"  
conversation along these lines. To sound good literary, we can drop the Landor  
reference in the references:
 
   Landor, Imaginary Conversations.
 
 The epagoge/diagoge distinction used by Grice bears on this. It may be best 
 provided some formalisation. Let 'c1' be claim c as put forward by 
philosopher C  (Carnap\index{Carnap}). Let c2 be claim as put forward by philosopher G (Grice)
 
                  C                         G
 
                  c1                        c2
 
In the epagoge model, c2 only attains sense vis a vis c1. G's claim to fame 
 is seen as C's claim to infame, and vice versa. The epagoge works indeed  
gladiatorially: the success of c2 is in the defeat of c1 and vice versa. It 
is a  zero-sum game, where game is loosely understood as such. More like a 
mediaeval  joust, if you ask us.
 
In the diagoge model, we need to add pieces of evidence, e1 and e2. So we  
get
 
                   C                          G
 
  e1  ---->    c1                          c2  <---- e2
 
 The success of each claim does depend on the strength and virtue of  their 
own corresponding backings. But we feel we need a synthesis to the  
epagoge-diagoge dialectic then. For there are issues regarding the  
incommensurability of the respective pieces of evidence and the topicalisation  issue (are 
c1 and c2 about the same thing -- or is Grice changing the  {\it subject}). 
Last but not least, there is the question as to to what extent this  is just 
"imaginary". After all, we are building a bridge: looking for some sort  of 
'actual' conversation, and it seems that we still have C with his c1 and e1  
on one hand and G with his c2 and e2 on the other. So what gives? We propose  
then a sort of criss-crossing. Where we add e1' : i.e. evidence derived  
from c1 as it supports or fails to support c2. And we add e2', i.e.  evidence 
as derived from c2 which supports or fails to support c1.  Only when we 
reach this level of bridging can we say that G is  conversing with C and vice 
versa:
 
                    C                   G
 
  e1                                                e2
      .                                           .
        .                                     .
           .                               .
                c1                    c2
                   .                      .
                       .               .
                        e1'   e2'
 
 
Bunyan,  John. (1678). The pilgrim's progress from this world,  to that 
which is to come, delivered under  the similitude of a dream, wherein is 
discovered, the manner of his setting out,  his dangerous journey; and safe 
arrival at the desired countrey. London:  Nash.

[As early as] 1946, Bar-Hillel [was discussing] the sense of 'imply'  
identified by Moore, proposing to describe it as 'pragmatical' (p. 334).  He 
identifies himself as a supporter of 'logical empiricism' (he  quotes 
approvingly a comment from Carnap\index{Carnap} to the effect that  natural languages are TOO 
COMPLEX and MESSY to be the focus of rigorous  scientific enquiry) and his 
article ['Analysis of "correct"  language'. Mind 55 328-30) is aimed explicitly at 
REJECTING philosophy of  the 'analytic method' ... However, he suggests 
that by using sentences that are  'MEANINGLESS' to logical empiricists, such as 
the sentences of METAPHYSICS or  [worse, JLS] aesthetics [never mind 
'ethics'. JLS], 'one may nevertheless  imply [empahsis Bar-Hillel. JLS] 
sentences which are PERFECTLY MEANINGFUL,  according to the same criteria, and are 
perhaps even true and highly important'  (p.338). -- cited by Chapman\index{Chapman} in her 
book on Grice.

\section{technical}

While Carnap\index{Carnap} has these as applying to 'rejection', in a more  
charitable light we can see the labels as applying to approaches. It seems  
plausible to entertain the idea that it takes a metaphysical stand to reject  
another. So what's the technical side to this. G would surely oppose a  
characterisation of metaphysics as the realm of 'synthetic' truths. If anything  
of value, metaphysics has to transcend that realm. In the way that 
Nietzsche  said (we think) that morality was beyond evil and good. The 
analytic-synthetic  distinction must be one of the first offshots of our metaphysical 
thinking, so  it cannot be presupposed by it! Now C's way out here, the 
distinction between  'necessary' and 'analytic' would be one that may perhaps 
appeal G. We mean, he  was one for splitting anyday (never lump). Assuming a 
retreat to 'analytic'  would be viscious here (or vascuous, if you want), we 
are left with 'necessary'.  Now, this operator Grice found increasingly 
complex. C's idea that it deals with  the denotatum rather than the denotans is 
one which would have appealed G.  Echoes of ratio essendi come to mind. Grice 
was hoping (recall Hopeful is  Christian's soul mate in his pilgrimage to 
the Celestial City) that philosophy  (or metaphysics, specifically) could 
provide a backing for ratio as apply to  esse not just {\it cognoscere}. While 
C's would have had the knee jerk reaction,  "Scholastic!" this need not be so. 
The idea is that while 'must' (the token of  'necessity' as it were) 
applies to various realms but it's not for that reason  'aequivocal'. It is rather 
{\it aequi-}vocal: i.e. the same vox for various items.  There is ontological 
necessity, there is cognitive necessity, there is logical  necessity 
('analyticity' in C's jargon). So what gives. Grice would go on to  define 
metaphysics as that part of the discipline of the philosopher, perhaps  qua 
eschatologist, which defines the axioms for our understanding of 'must'.  When and 
how are we ready to postulate an item as {\it deontic}. Deontic is the  
adjective that would have appealed Grice at this point. It's the deon of the  

\subsection{Greeks}

The idea that some things are, some others must be. The  internal-external 
distinction C draws at this point would have sounded to G "a  mere 
Hartism". Hart had distinguished between internal and external readings of  things 
(notably ascriptions of right: "Carnap\index{Carnap} is right" "Carnap\index{Carnap} is wrong". On an  
internal reading, we assume he is wrong. On an external reading we assume 
that  someone assumes he is wrong. Grice discusses this feature of 'deontic' 
in  connection with Nixon being appointed the Professor of Moral philosophy 
at  Oxford! (Grice 2001). The levels of internal, external, and 
middle-of-the-way  readings are formidably complex as Grice was wont to say. But in any 
case, the  offshot is that the 'deontic' operator need not be 
self-referential: i.e. it's  not like our grasp of the meaning of 'deontic' involves our 
acceptance of  'deontic' as deontic at a higher level. Ultimately Grice would 
have appealed to  a mere reiteration of symbols. Some operators are not 
deontic. Notably the  boulomaic operators, the volitional operators. What we 
decide or deem that  we'll do. This is mere volitional. But volitional 
predicates have the ability to  go recursive. We decide to decide. And we decide 
to decide to decide. When this  iteration is given free reign, we find that 
we have cashed the deontic operator  out of the boulomaic operator. This type 
of transcategorial epithets would  thus define the 'necessity' which we 
associate with this or that metaphysical  scheme. In G's case, his hidden 
agenda is not so hidden: he wants to license the  metaphysics (or physics as he 
would sometime say) underlying English! (The sun  rises from the East -- will 
we still be using 'rise' knowing that that is NOT  what it does? Isn't this 
getting involved in stone-age physics? Whatever. But  that is not a serious 
issue. More serious is the inability of ordinary English  speakers to go 
beyond the solid 'table'. The solid 'table' is what we mean by  table. If it 
turns out that Eddington is right (as he most likely is) and the  real table 
is a bunch of wavicles, we may still want to keep using the  metaphysical 
scheme that we inherited from our ancestors -- cavemen no doubt --  out of 
respect for them, and because if we are not that dumb we know how to  
translate one metaphysical scheme onto another!
 
\subsection{intuitive approach}

There is something to be said for Carnap\index{Carnap}'s gut rejection for metaphysics.
His claim to fame is actually his laughing at Heidegger\index{Heidegger}'s Nothing Noths.
So we could consider this in more detail.
German:  Nicht nichtet.
This was thought serious enough by Ryle when he cared to review Heidegger\index{Heidegger} for Mind.
But what does ``Nicht nichtet'' amount to?
At this point, Bar-Hillel's throwing onto the same wastebasket `metaphysics' and `ethics' or aesthetics, won't do.
Carnap\index{Carnap} is a much more serious philosopher than early Ayer's caricature of the statements of ethics as ``ouch'' and ``pooh!''.
But there is more to consider here.
In this work we cannot hope to cover all realms of statements, so we better focus on allegedly metaphysical statement or pseudostatements (schein- is the lovely suffix for Carnap\index{Carnap} here) of metaphysics.
So what's wrong with ``Nothing noths''.
Carnap\index{Carnap} thinks this breaks a rule of grammar.
And it does!
Heidegger\index{Heidegger} MEANT it as a breach of a rule of grammar.
Heidegger\index{Heidegger} kept saying these things, to the point that, no C, but G, could laugh at him when he said, ``Heidegger\index{Heidegger} is the greatest living philosopher'' (in \cite{grice89}:i).
So what was Heidegger\index{Heidegger} up to.
We believe he was PLAYING with the rules of grammar.
He is into ``not''.
"Not" is a trick of a word.
He had an intuitive, or gutty feel for language as play.
So out of ``not'' he coins the noun, ``nought'', ``Nicht''.
(Nought is a complex word in English, involving ne- and aught.
I.e. Not Something -- No-Thing.
Noths.
But what about the verb, `noths'.
Well, it does  seem like a thing Nothingness would do.
To do nothing.
To who?
G was fascinated by the grammar of verbal constructions.
He played with things like `tigers  tigerise' (1991:x).
So he wouldn't have objected, upon proper understanding of what rules we are 'flouting' here -- with a sprinkling of Strawson\index{Strawson}'s ``Subject and predicate in logic and grammar'' 
for good measure -- to utter, ``Nichts nichtet''.
It's like, if you can't beat them, join them.
 
\subsection{radical approach}

G would have enjoyed that.
Indeed Bradley\index{Bradley} for G, and Heidegger\index{Heidegger} for C are what we may call arch- or ur-metaphysicians.
Recall G's WoW: ``nobody since the demise of the influence of Bradley\index{Bradley} was even remotely inclined to believe that'', where we don't even need to care {\it what the claim Was}.
Some absurd extravaganza.
This is amusing, because we do know that both C and G did care for their ancestors.
C for Kant\index{Kant}'s and Neo-Kant's extravaganzas, G for Bradley\index{Bradley} (eg. on 'negation' or 'deixis').
So it was more of a pose, that, we can say, get the headlines.
The various readings of C's specifics in his "Ontological" essay would have amused G.
Indeed Quine\index{Quine} could get over the top about what 'there is' -- a seminal work that both C and Q were very aware of.
But C's radicalism was perhaps more `intuitive' -- if we can lump these two labels here -- than G's.
G would have examined what we mean by `real' or `really' when C's claim, on the serious reading of his ``Ontology'' paper, that  this or that `does not really exist'.
As opposed to th'other `does really exist'.
After all, why, G would have it, should we give such a hoot to a mere `trouser-word' such as `real' is?
Surely our radical opposition to a campaign as serious as metaphysics is presented to be (by metaphysicians, no doubt) should rest on something more substantial than that.
The topic of the evidence gets us closer to the nail we need to hit. 
This is back to some of Grice's betes noires  (Phenomenalism, indeed) -- but again, a close examination at how language works  does suggest a neat way out. 
For we claim the denotata of our terms to go beyond  the evidence we may have (or lack) to `assert' the `warrantibility' of our  claims.
``The cat is on the mat''.
No intension here makes a strict reference as to how we get to KNOW that.
It's here where an examination of the `simpler ways' in which `pirot\index{pirot}s karulise elatically' will, we hope, eventually land us on the Celestial City.
And then, wouldn't we find it boring enough that we are going to scream alla Heidegger\index{Heidegger}!
(``Out! Get me Out of Here!'').

This quote from WoW:then
 
"(Mrs. Jack\index{Jack})," Grice says,

"also reproves me for "reductionism," in terms which suggest that  whatever 
account or ANALYSIS of meaning is to be offered, it should not be one  
which is 'reductionist,' which might or might not be equivalent to a demand that 
 a PROPER analysis should not be a PROPER REDUCTIVE ANALYSIS. But what KIND 
of  analysis is to be provided? What I think we cannot agree to allow her 
to do is  to pursue the goal of giving a LAX REDUCTIVE ANALYSIS of meaning , 
that is, a  reductive analysis which is UNHAMPERED by the contraints which  
characteristically attach to reductive analysis, like the avoidance of  
circularity; a goal, to which, to my mind several of my opponents have in fact  
addressed themsleves ((In this connection, I should perhaps observe that 
though  MY EARLY ENDEAVOURS in the theory of meaning were attempts to provide a 
 REDUCTIVE analysis, I HAVE NEVER (I THINK) espoused {\it reductionism}, which 
to my  mind involves the idea that semantic concepts are unsatisfactory or 
even  UNINTELLIGIBLE, unless they can be provided with interpretations in 
terms of  some predetermined, privileged, and favoured array of concepts; in 
this sense of  "reductionism" a felt ad hoc need for reductive analysis does 
NOT have to rest  on a REDUCTIONIST foundation. Reductive analysis might be 
called for to get away  from unclarity not to get to some predesignated 
clarifiers)). I shall for the  moment assume that the demand that I face is for a 
form of REDUCTIVE analysis  which is less grievously flawed than the one 
which I in fact offered; and I  shall reserve until later considerations of 
the idea that what is needed is NOT  any kind of reductive analysis but rather 
some other mode of explication of the  concept of meaning" (WoW:351).

\subsection{Clarification}
 
In this respect, it is clarificatory that, for Grice, as for Carnap\index{Carnap},  
psychological concepts should be introduced as theoretical terms, rather than  
as ones based on observation. The locus classicus here is Grice's use of  
Ramsification to introduce T terms in terms of O terms (in Grice 1991). 

We like the idea  that indeed Grice is perhaps slightly hasty in dealing 
with all the betes noires  at one fell swoop as it were, or at the one blading 
of the sword. 
 
The fact that all  of the betes noires end in -ism is perhaps telling. This 
is a Greek suffix,  -ismos which of course Grice would NOT reject in other 
collocations: his beloved  Aristotelian 'syllogismos' for example. So we 
have to be  careful.
 
Grice seems to  consider that besides this common mark -- they all end in 
-ism, they may also  all be seen as the offsprings of "Minimalism" -- his 
rejection of desserted, or  made-dessert landscapes (the rosebushes and 
cherry-trees in the springtime). We  shall have to get back there.
 
\section{Beyond the Pirot talk}

CHAPTER  FOUR: BEYOND THE PIROT TALK.
As  we have seen (3.2.2.1 – i.e. our reactions to extensionalism) there are 
broader  issues here. To what extent will an intensionalist NOT feel 
betrayed by an  intensionalist. Some of the most formidable passages in WoW:RE 
deal with this.  He is here concerned with what we may see as an ‘intensional’
, i.e.  non-truth-functional context. But we’ll need to elaborate on that. 
On p. 374 of  WoW:RE he writes:

“A truth-functional conception of COMPLEX propositions  offers prospects, 
perhaps, for the rational construction of at least part of the  realm of 
propositions, even though the fact that many complex propositions SEEM  PLAINLY 
to be NON-truth-funtctional ensures that many problems  remain” 
---  the naivete of Grice is formidable here. For he is saying that 
something is  plainy or SEEMS plainly thus and thus. By what evidence. It seems 
that whatever  evidence he has to SAY that is of a different kind from the 
evidence he is  supporting as providing a ‘rational construction’. Etc. I would 
think that  whatever Carnap\index{Carnap} had as an intensional context would fit the 
bill here in not  being truth-functional. One is less sure that G would have 
been happy with a  mere extensional treatment of modal propositions.  Etc.
 
\subsection{Un-Carnapian Grice}

"The un-Carnapian character of my
constructivism would perhaps  be
evidenced by my idea that to insist
with respect to each [pirotic]  stage
in metaphysical develoment
upon the need for
THEORETICAL  JUSTIFICATION
might carry with it the thought
that to omit such a  stage
would be to fail to do justice
to some legitimate
metaphysical  demand"
    (Gr91:76)
 
>  ``The un-Carnapian character of my constructivism''
 

he is thinking that Carnap\index{Carnap} was such a pragmatist that he was into 'theoretical justifications' being practical?

would  perhaps  be:
\begin{quote}
evidenced by my idea that to insist
 with  respect to each [pirotic]  stage
 in metaphysical  development
\end{quote}
 
this in the context of what you were asking
before about the objectivity of
value judgements. Grice is introducing
'pirot\index{pirot}s' or creatures the last
stage of which will endow them
with a capacity to project
values onto the world
(and thus turn them objective
via construction).
 
\begin{quote} 
`` upon the need for
 THEORETICAL   JUSTIFICATION
 might carry with it the thought
 that to omit  such a  stage
 would be to fail to do justice
 to some  legitimate
 metaphysical  demand"
\end{quote}

  --- yes here it is where he gets complex. For he uses
  too many neg. constructions,
   omit, fail to do... etc.
  omit and fail are so negative that one wonders...
    he is saying that not to omit is not to fail
            to  include is to succeed
 So he is saying that by including value-thus construed
      he IS fulfilling a demand.
 
--- I think the carnapian would say
  that if there is no theoretical justification
 one would not be succeeding.

Anyway, we will provide such a detailed exegesis that the reader will  omit 
to fail to understand it, for she will!


\chapter{The City of Eternal Truth}

\section{}

And there it lay: the City of Eternal Truth. Carnap and Grice were in
awe, as they approached it with veneration. As fellow pilgrims we can
now report:

The early history of the City of Eternal Truth lies hidden in the mists
of time. The City reached her present form under the patronage of
Kantotle. In an aerial view we can see the total surface area of the
temple covers 13 hectares or 35 acres, each dedicated to a
philosophical speciality -- and placing it among the largest Cities in
the whole of Philosophy.

The City is the result of some urban planning.
It is designed with 5 concentric 'monads', or circumambulatory temple
courtyards. Each of these is associated with one of the Five Elements
-- which are ultimately one, of course.

The innermost 'monad' is not visible.
It lies within the sanctum with the golden roof, and can only be
entered by the Universal Maxim.

The architecture and the rituals of
this City reflect its history and doctrine. Where we now find this
beautiful and ancient City was once an impenetrable Forest of Dogmatic
Trees, which is a kind of mangrove.

This Forest gave The City her first
and most ancient name, "Woody". Within this sprawling forest was a
lotus pond, and at the southern bank of this pond existed a Cunning of
Reason. A Cunning of Reason is a representation of Kantotle -- which
unites both the concepts of Form as well as of Formless in itself. In
modern terms this formless-form might be called an abstraction. What
Carnap calls an ``Intension''.
Intension means ‘self existent’ [only different], signifying that the
Fregean Sense [like the Natural Number]
was not made by human beings, but came into existence by itself, from
what Grice calls ``Nature''.

To this lotus pond in the "Woody" forest
came two saints, named Carnap and Grice. They came from very different
backgrounds and from very different directions. But they came for the
same reason: to witness Kantotle’s Cosmic Dance.

It had been foretold to them that if they would 'elucidate'
the ``Cunning of Reason'' on the bank of the lotus pond in the forest,
Kantotle would come to perform His Dance for them. Eventually this
great event took place. Kantotle (in his guise as Plathegel) came to
perform His Dance on a Saturday morning, when the moon was in the
asterism Ryle, during Hilary, long before the Devil of Scientism era.
Kantotle's dance is called the Dance of Bliss. The two saints achieved
liberation, and on their special request Kantotle (in his guise as
Plathegel) promised to perform His Dance for all time at that place.
For the full narration of the myth the reader is referred to chapter IV
of the present Conversation (again -- and again). The story of the
origin of the worship of Kantotle in the City of Eternal Truth is told
in the Logische Aufbau der Welt.

The Sacred History of the City of Eternal Truth, which is part of
the ``Principia Mathematica'', one of the 18 great vademecums or
collections of mythology. From one of the saints, Carnap, which
means ``Slept in a Vehicle'', The City of Eternal Truth received her
second name, Pirotgrad, meaning ‘City of the Pirot’. Its third name,
Griceland, refers to the philosophy and doctrine of the temple, as
narrated by Grice's arch-enemy: Carnap, in his third
re-incarnation. ``Gri-'' means consciousness or wisdom. ``-ce''
signifies ``ether'' in Pirotese but in Russell it means `hall'.
Carnap-Corner-in-Griceland unifies the two aspects of the one and only
Kantotelian doctrine. Meaning thus both ``Hall of Wisdom'', as well as
the place of the Ether of Consciousness.

\subsection{Inside The City}

The edifice which now includes within its sanctum this Cunning of Reason
form of Kantotle, situated on the southern bank of the sacred pound, is
called ``Bosanquet''.
This term means ‘place of origin’ or ‘root place’
-- an exaggeration, seeing that old Grice saw him as a 'minor
figure'. ``Bosanquet'' can be found in the third courtyard, within the
temple proper.

Facing east, it is a conventional temple with a sanctum containing the
cunning of reason, and aa hall in front of the sanctum. In this hall we
find the images of our two saints, Carnap and Grice. How the images got
there BEFORE THEM is a great mystery. They stand with their hands
folded, worshipping.
A sanctum placed at an angle to the
Cunning-of-Reason shrine, facing south, houses the consort of Kantotle,
the goddess Aletheia. On the western wall of the shrine we find a
relief sculptured of the Wishing Tree of Paradise (Eschatology). This
shrine achieved its present form probably under the middle and later
stages of the Vienna Circle. The main edifices of the temple are the
five Halls. At the centre of the temple is situated the sanctum
sanctorum or Holy Of Holiest. This means the ‘Hall of Wisdom’. It is
the main shrine where Kantotle accompanied by his consort Aletheia (the
Unveiled One) performs His Cosmic Dance, the Dance of Bliss. The World
-- or "Nature" -- is the embodiment of the colossal human form. The
City of Eternal Truth is the centre of this form, the place of the
heart, where Kantotle performs the Cosmic Dance. The City is laid out
as a labyrinth. For this reason the devotees may approach the
central shrine from two sides. One is called Extension. A narrower path
is called Intension. As blood flows to and from the heart. The 16
stupas topping the golden roof represent the sixteen strands of The
Fabric. They also asymbolize the sixteen Strands -- or goddesses. The
roof of this hall is made of 21.600 tiles, representing
inhalations and exhalations of Pirots.
The links and side joints symbolize the connecting veins -- of the
pirots, of course. The five main steps at the entrance to the shrine
stand between the devotees and the image of Kantotle, covered in
silver. They are the five seed words or syllables. By chanting these
syllables:

\begin{center}
KAN -- TO --- TLE
\end{center}

the devotee can cross the ocean of bondage and attain to the Lord. The
granite plinth of the shrine is called Oxonianism -- because it does
duty for Vienna in providing a support for Kantotle (in his Russell
re-incarnation). On all special occasions worship is performed to this
plinth. The name, Hall of Consciousness or Hall of Wisdom, refers to
the quality of wisdom which pervades the atmosphere, bestowed upon the
worshippers by the Dance of the Lord. His boon is the experience of the
Cosmic Dance. A unique feature is that the structure of the actual
stage is made of wood, which has so far not been botanically classified
but is nevertheless real. It is rectangular in form and here Kantotle
is worshipped in his three aspects: as Form or Image, as Formless-Form
-- the crystal Cunning of Reason -- and as formless. From the platform
opposite one can see the image of the Dancing Kantotle, situated in the
middle of the stage. Kantotle is facing south, unlike most other
simpler Philosophers. This signifies he is the Conqueror of Dogma,
dispelling the fear of death for Humanity. The Crystal Cunning of
Reason is Kantotle as Formless-Form. It was formed from the essence of
the crescent moon in Kantotle's matted hair, for the purpose of
peripatetic worship.

This is taken from its keeping place at the feet of the thing six times
a day, and holy ablution is performed to him in the hall. Immediately
to the proper right of this is the ‘mystery’ of Analyticity. Here,
behind a silk curtain which is black
on the outside and red on the inside, is the Treasure of Meaning
Postulates, in the form of Predicate Calculus. An abstract geometrical
design, on which the deity is invoked. Behind the curtain, before it,
hang a few strands of golden fig leaves. This signifies the act of
creation -- or Pirotology. One moment nothing exists, the next instant
the All has been brought into existence.

At regular timings the curtain is removed to allow the devotees to
worship the Ether which is the vehicle of the Absolute and
Consciousness. The hall houses one more unique form of Kantotle. This
is the Organon, the Ruby Lord: a replica of Kantotle in ruby form. This
appeared out of the fire of the sacrifice in response to the devotion
of the Modernists. Every Saturday, as part of the 10.00 o’clock morning
ritual, after the Recitation of the Crystal Cunning of Reason is also
performed to the Ruby Kantotle. As conclusion of this ceremony the Ruby
Thing is placed on the edge of the Swimming Pool and an Implicature is
offered. This is the burning of camphor on a special plate which is
shown both in front and behind the Ruby Thing. This brings out the
special quality of translucence of this, creating a mystical spectacle
for the onlookers. Nobody knows when the worship of Kantotle was
established here, or when the City of Eternal Truth was build. The
original wooden structure is doubtless, and ironically,the oldest
structure in the temple complex,
as the shrine of Plathegel is a later construction under the
Neo-Kantians. The City has no features, really, that could help to date
it and it might just as well be eternal, after all.

It is unique and no other structure is known like it anywhere else in
Philosophy. Analysis by the Leibnizian ifinitesimal method would be
unreliable because it is known to have been regularly renovated during
the centuries. But the origins of the City of Eternal Truth lie back in
prehistoric times. According to the mythology the City was first
constructed by a Philosopher King nicknamed "Thales". This Philosopher
King was healed of leprosy by bathing in the sacred pond in the "Woody"
forest and witnessed the Cosmic Dance. The first gilding of the roof of
the temple and the instituting of the formal worship are all attributed
to this King Philosopher.

The first historical references can be found in Jowett's translation of
the Plato Dialogues, especially in the Timaeus. Here Aletheia, the
six-faced Daughter of Yocasta and Socrates, is described as worshipping
his parents in Athens, before going to do battle with a demon called
Physicalism. This text can be dated to the fourth century BCE. The City
of Eternal Truth is also prominently mentioned in "De Consolatione
Philosophiae", an important religious and philosophical text in ancient
Latin, dating from the beginning of the Christian era. A few centuries
later the temple and its Lord are often mentioned by members of the
Vienna Circle, but only derogatorily and, especially by Schlick. The
first historical persons to claim having gilded the roof of the temple
are Baumgarten and his 'cousin', Kant. By this time the temple had
already become important. The place where Students were crowned, and
where they came to worship and receive counsel. How the gilding of the
roof was done is a knowledge that was sadly lost with time. But it is
without doubt one of the great technical achievements of ancient times.
Immediately in front of the temple is the golden hall.

Its roof is made of copper, although Kanaka means gold. This is the gold of spiritual
treasure: to experience Kantotle's dance from so near. In this hall are
most of the Saturday morning rituals of worship performed. The Early
Morning rituals. The rituals with lamps and ritual objects. And the
Ruby Thing. The pilgrims can enter certain areas of the hall for
worship at specified hours. It is a controversy whether this was
originally constructed together with the older hall, or some time
later. This is the shrine in the form of a chariot, pulled by two stone
horses.
One represents Practical Reason, the other Theoretical Reason. It is
situated opposite the old hall, in the third courtyard. It is the place
of the dance contest between Kantotle and Plathegel.

Kantotle conquered Alethei, who would not calm down after she destroyed
a powerful demon -- Reductionism -- by lifting his right leg straight up towards the
sky. This dance is called ``Gentzen''. Then and there Aletheia suddenly
remembered who she really was, the peaceful consort of Kantotle, and
she was able to leave her furious mood and returned to her peaceful
self. This scene is depicted in the sanctum inside. We see Kantotle
performing his dance, with his leg
lifted straight above his head, and Aletheia calmed down in one corner,
both accompanied by Carnap and Grice playing the Fiddles, the
instruments which are used to accompany the dance.

The chariot form commemorates Kantotle as the Destroyer of the Three
Demon Cities. Several divine powers joined together to create this
chariot. Thus the sun and moon became the wheels, and the Two Reasons
the two horses. After destroying the Three Cities he descended from his
chariot, having landed opposite, and ascended into the City to commence
His Dance. From this, it is also called opposite hall. This opposite
(or subcontrary) hall has several distinguishing features aside from
its shape and its function. Its columns are unique to the chariot hall.
They are square, and circular at the same time, and although carved
from the hardest granite they are covered with exquisite miniature
reliefs, depicting dancers, musicians and all kinds of philosophical
figures. One other feature sets this edifice apart from any other hall
within the temple complex and from all other temple halls in
Philosophy. This is mysteriously connected to the Sphinx -- she of
Riddles fame.

Just under the floor surface of the raised platform which
is the Body is a belt surrounding the whole city.
Here we see lions and sphinxes alternating in pairs, girdling it. Also
the pillars of the two pavilions on the western side of the hall are
supported by four sphinxes which function as caryatids. It is
considered by tradition the second oldest building in the complex,
without any real indication of its age. It is reported in inscriptions
as having been renovated by the St. Bonaventura
in the thirteenth century. The hall can be found in the third
courtyard. The festival deities are kept during the year, and worship
is performed for them
on Saturday mornings. This is done inside the hall, and is open to the
public. The age and history of this hall is also hidden in the mists of
time. There is some evidence the hall was once used not just as a dance
hall, but as an "Music Hall" by visiting Philosophers and Comedians of
the different governing dynasties of the Oxford and the Sorbonne during
the several phases of history. No other information is available, alas.
To the right is the Thousand Pillar Hall in the second courtyard. It is
the architectural representation of the Crown. Which is the seventh
spiritual energy point in the astral body. Kantotle and the goddess
Aletheia, his consort, dance here in the mornings of the 9th and 10th
Saturday of the Chariot Festival. About this too, we have very little
historical information. It is first mentioned as the place where the
Flemish philosopher, Descartes, premiered his great Song-and-Dance
routine (``The Cogito'') on the lives of the 63 saints -- or Malignant
Demon --, before Voltaire. Its base is encircled by reliefs of dancers
and musicians, as it were, participating in a procession.
The most imposing feature of the temple, which can be seen soaring
above the plain from miles away, are the four temple gateways, located
in the second wall of enclosure at the four cardinal points.

These are:

\begin{centering}
\begin{itemize}
\item[The South]
\item[The West]
\item[The East]
\item[and]
\item[The North]
\end{itemize}
\end{centering}

They are considered among the earliest examples of such structures and
are in their present form dated to at least the 12th and 13th century.
Both Carnapians and Griceians disagree about the dates of the
individual columns, or about which one was build first. Some consider
the West as oldest, some the East. In between the sculptures decorating
the inside of the West Corridor we find a musician (Grice?) playing a
standing double drum. This could point to an early date for this -- or
a later one, if one allows for Syncopation and the Jazz Age. On the
outside of the granite bases are found sculptures of many important as
well as less well known deities in niches in a particular order. The
inside walls of passages through all the four corridors are decorated
with the 108 dance movements of Kantotle's Peripatetic Dance, from the
Organon, the world’s most ancient treatise on dance, drama and theatre
-- and logic. Besides here, these movements -- of which the first is
called ``Barbara'' -- are depicted in only four other temples -- but they
circulate widely on the Net. The four corridors, together with the
golden dome of the central shrine are the five towers which represent
the five faces of Kantotle, with the Smiley symbolizing the masterful
face.

In the innermost courtyard, at a right angle with the Golden Thing, we
find the shrine of Aletheia. Reclining on the Cosmic Snake, she is in
the state of consciousness, enjoying the vision of Kantotle’s dance.
The coexistence of the worship of both Kantotle and Aletheia within one
temple is unique. The worship was established in the earliest times and
was originally performed by the ``Minstrels'' themselves. In the later
medieval period, with a shifting political situation and hyperinflation
under pressure of Capitalist invasions, there was possibly a
discontinuation of the worship for a long period, after which it
re-instated by the Wittgenstein, of the Tractatus Logico-Philosophicus.
The worship of Aletheia has since then been in the hands of Carnapians
and Griceians mainly, and was no longer performed by the "Minstrels"
proper.

Within the inner courtyard, to the east, we find a small shrine which
houses the bracelets of both the Creator god, of the Handy Trinity, and
Home, a deified saint. The presence of the Creator-God (a deity almost
never worshipped) establishes the worship of all three deities of the
Handy Trinity with-in the one complex. The temple of goddess Aletheia,
consort of Kantotle, is situated on the west side of the Water Tank. A
flight of steps leads down into its courtyard. The goddess is
worshipped here as the energy and power of wisdom. On the frontal
portion of the pillared hall, on the ceiling of the right and left
wings, the finest eye-capturing fresco paintings of approximately a
thousand years old, illustrate the Sacred Deeds of Kantotle. The
galleries surrounding the temple are decorated with a procession of
dancers, musicians, and philosophers, sculptured in relief. This temple
was possibly built in the 11th century, if not earlier.

This holds The Sacred Twater Place or Tank. It is famous for healing
the ancient king Buridan of his skin disease. His skin became golden
after which he was called ``The Ass''. In this Tank we find a stone
representation of the the Element Twater, which as Putnam has showed,
is not H20, but XYZ. In the dry season it becomes visible as the water
level in the tank is reduced. The Twater temple is dedicated to
Abelard, the second son of Kantotle and Aletheia. This shrine is also
shaped as a chariot, pulled by horses and elephants. This temple was
according to tradition build by a king of the dynasty from Cambridge,
which superceded the rule of the the Oxonians in the fifteenth century.
His name was Testa Bianca (``White Head''), and the temple is named after
him. In the middle of the 19th century this temple was renovated by the
Victorians with the support of Dutch merchants, who had a trading post
in nearby Porto Nuovo. According to an inscription on copper plates
they donated a share of their profit for this purpose, but we do not
know what they did with the rest.

\chapter{Conceptual Maps}

\section{Carnap - Grice}

\pagebreak

\begin{centering}
\begin{tabular}{p{3cm} p{2.5cm} p{3cm}}

\hline\\

CARNAP&\hrule&GRICE\\

\hline\\

\hline\\

&MORRIS&\\
&defines pragmatics&\\

gets defined by Carnap as the study of utterance, assertion, and belief
& \hrule
& is used by Grice vis a vis 'pragmatic rules'\\

\hline\\
\hline\\
Carnap dies\\
\hline\\
&& In 1975, Grice re-introduces the pirots.\\
&&
And to extend the scope of
analytic philosophy, tackles
practical reason (Locke Lectures
1979, Oxford) and value (Carus
Lectures, 1983)\\
\hline\\
&&Grice dies in 1988\\
\hline\\
\end{tabular}
\end{centering}

%\begin{Parallel}{}{}
%\ParallelLText{gets defined by Carnap as the study of utterance, assertion, and belief}
%\ParallelRText{is used by Grice vis a vis `pragmatic rules'}
%\end{Parallel}

\backmatter

%\chapter*{Glossary}\label{glossary}
%\addcontentsline{toc}{chapter}{Glossary}
%
%\begin{description}
%\item[]
%\end{description}

\addcontentsline{toc}{chapter}{Bibliography}
\bibliographystyle{alpha}
\bibliography{rbj}

\addcontentsline{toc}{chapter}{Index}\label{index}
\twocolumn[]
{\small\printindex}

\end{document}
