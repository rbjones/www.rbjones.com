
\subsection*{Contents Summary} 
\addcontentsline{toc}{subsection}{\hspace{0in}{\it Contents Summary}}

\renewcommand{\aref}{\arefB}

\begin{description}
\item[Ch. 1.] Homonyms, synonyms, and derivatives. 

\item[Ch. 2.] Presence and Predication.

\begin{description}
\item[(1)] Simple and composite expressions. 
\item[(2)] Things
\begin{description}
\item[(a)] predicable of a subject.
\item[(b)] present in a subject,
\item[(c)] both predicable of, and present in, a subject,
\item[(d)] neither predicable of, nor present in, a subject.
\end{description}
\end{description}

\item[Ch. 3.] Subclasses and predicability
\begin{description}
\item[(1)] That which is predicable of the predicate is predicable of the subject. 
\item[(2)] The differentiae of species in one genus are not the same as those in another, unless one genus is included in the other. 
\end{description}

\item[Ch. 4.] The eight categories of the objects of thought. 

\item[Ch. 5.] Substance. 
\begin{description}
\item[(1)] Primary and secondary substance. 
\item[(2)] Difference in the relation subsisting between essential and accidental attributes and their subject.
\item[(3)] All that which is not primary substance is either an essential or an accidental attribute of primary substance.
\item[(4)] Of secondary substances, species are more truly substance than genera. 
\item[(5)] All species, which are not genera, are substance in the same degree, and all primary substances are substance in the same degree. 
\item[(6)] Nothing except species and genera is secondary substance. 
\item[(7)] The relation of primary substance to secondary substance and to all other predicates is the same as that of secondary substance to all other predicates. 
\item[(8)] Substance is never an accidental attribute. 
\item[(9)] The differentiae of species are not accidental attributes. 
\item[(10)] Species, genus, and differentiae, as predicates, are univocal with their subject. 
\item[(11)] Primary substance is individual; secondary substance is the 
qualification of that which is individual. 
\item[(12)] No substance has a contrary. 
\item[(13)] No substance can be what it is in varying degrees. 
\item[(14)] The particular mark of substance is that contrary qualities can be predicated of it. 
\item[(15)] Contrary qualities cannot be predicated of anything other than substances, not even of propositions and judgements. 
\end{description}

\item[Ch. 6.] Quantity:
\begin{description}
\item[(1)] Discrete and continuous quantity. 
\item[(2)] Division of quantities, i.e. number, the spoken word, the line, the surface, the solid, time, place, into these two classes. 
\item[(3)] The parts of some quantities have a relative position, those of others have not. Division of quantities into these two classes. 
\item[(4)] Quantitative terms are applied to things other than quantity, in view of their relation to one of the aforesaid quantities. 
\item[(5)] Quantities have no contraries. 
\item[(6)] Terms such as great and small are relative, not quantitative, and moreover cannot be contrary to each other. 
\item[(7)] That which is most reasonably supposed to contain a contrary is space. 
\item[(8)] No quantity can be what it is in varying degrees. 
\item[(9)] The peculiar mark of quantity is that equality and inequality can be predicated of it. 
\end{description}

\item[Ch. 7.] Relation. 
\begin{description}
\item[(1)] First definition of relatives. 
\item[(2)] Some relatives have contraries. 
\item[(3)] Some relatives are what they are in varying degrees. 
\item[(4)] A relative term has always its correlative, and the two are inter dependent. 
\item[(5)] The correlative is only clear when the relative is given its proper name, and in some cases words must be coined for this purpose. 
\item[(6)] Most relatives come into existence simultaneously ; but the objects of knowledge and perception are prior to knowledge and perception. 
\item[(7)] No primary substance or part of a primary substance is relative. 
\item[(8)] Revised definition of relatives, excluding secondary substances. 
\item[(9)] It is impossible to know that a thing is relative, unless we know that to which it is relative. 
\end{description}

\item[Ch. 8.] Quality.
\begin{description}
\item[(1)] Definition of qualities. 
\item[(2)] Different kinds of quality:
\begin{description}
\item[(a)] habits and dispositions; 
\item[(b)] capacities;
\item[(c)] affective qualities.

[Distinction between affective qualities and affections.] 
\item[(d)] shape, {\&}c. [Rarity, density, {\&}c., are not qualities.]
\end{description}
\item[(3)] Adjectives are generally formed derivatively from the names of the corresponding qualities. 
\item[(4)] Most qualities have contraries. 
\item[(5)] If of two contraries one is a quality, the other is also a quality. 
\item[(6)] A quality can in most cases be what it is in varying degrees, and subjects can possess most qualities in varying degrees.
Qualities of shape are an exception to this rule. 
\item[(7)] The peculiar mark of quality is that likeness and unlikeness is 
predicable of things in respect of it. 
\item[(8)] Habits and dispositions as genera are relative; as individual, qualitative. 
\end{description}

\item[Ch. 9.] Action and affection and the other categories described. 

\item[Ch. 10.] Four classes of opposites.
\begin{description}
\item[(a)] Correlatives. 
\item[(b)] Contraries.

[Some contraries have an intermediate, and some have not.] 
\item[(c)] Positives and privatives. 

The terms expressing possession and privation are not the positive 
and privative, though the former are opposed each to each in the same 
sense as the latter. 

Similarly the facts which form the basis of an affirmation or a denial 
are opposed each to each in the same sense as the affirmation and 
denial themselves. 

Positives and privatives are not opposed in the sense in which 
correlatives are opposed. 

Positives and privatives are not opposed in the same sense in which 
contraries are opposed. 

For:
\begin{description}
\item[(i)] they are not of the class which has no intermediate, nor of 
the class which has intermediates. 
\item[(ii)] There can be no change from one state (privation) to its 
opposite.
\end{description}
\item[(d)] Affirmation and negation.

These are distinguished from other 
contraries by the fact that one is always false and the other 
true.
[Opposite affirmations seem to possess this mark, but they do not.] 
\end{description}

\item[Ch. 11.] Contraries further discussed. 

Evil is generally the contrary of good, but sometimes two evils are contrary. 

When one contrary exists, the other need not exist. 

Contrary attributes are applicable within the same species or genus. 

Contraries must themselves be within the same genus, or within 
opposite genera, or be themselves genera. 

\item[Ch. 12.] The word prior is applicable: 
\begin{description}
\item[(a)] to that which is previous in time; 
\item[(b)] to that on which something else depends, but which is not itself dependent on it;
\item[(c)] to that which is prior in arrangement; 
\item[(d)] to that which is better or more honourable; 
\item[(e)] to that one of two interdependent things which is the cause of the other. 
\end{description}

\item[Ch. 13.] The word simultaneous is used: 
\begin{description}
\item[(a)] of those things which come into being at the same time;
\item[(b)] of those things which are interdependent, but neither of which is the cause of the other.
\item[(c)] of the different species of the same genus. 
\end{description}

\item[Ch. 14.] Motion is of six kinds. 

Alteration is distinct from other kinds of motion. 

Definition of the contrary of motion and of the various kinds of 
motion. 

\item[Ch. 15.] The meanings of the term to have . 
\end{description}

\renewcommand{\aref}{\arefC}
