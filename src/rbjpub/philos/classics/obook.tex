% $Id: obook.tex,v 1.19 2013/07/26 16:26:32 rbj Exp $
% Change the next line before cvs checkin prior to an upload to
% CreateSpace to get up-to-date Id in book. !!!
\setmainfont{Palatino}

\newcommand\textgreek[1]{{\fontspec{cardo104s.ttf}#1}}
\newcommand\smallcaps[1]{{\fontspec{cabin}\textsc{#1}}}
\newcommand\scshapeR[1]{{\fontspec{Palatino}\textsc{#1}}}

%\defaultfontfeatures{Scale=MatchLowercase} % scale fonts to have same x-heights

\emergencystretch 3em
\hbadness=2000
\vbadness=10000

%this is to keep lists out of the margins

\usepackage{enumitem}

%\usepackage{textcomp}

%\renewcommand{\rmdefault}{ppl}

%\linespread{1.04}

% this is to make section headings (which are Aristotle's ``Books''s) raggedright
\makeatletter
    \renewcommand\section{\@startsection{section}{1}{\z@}%
                                      {-3.5ex \@plus -1ex \@minus -.2ex}%
                                      {2.3ex \@plus.2ex}%
                                      {\raggedright\normalfont\large\bfseries}}
\makeatother

%\usepackage[mathletters]{ucs}
%\usepackage[utf8x]{inputenc}
%\usepackage[greek.ancient,english,showlanguages]{babel}
%\usepackage[english,showlanguages]{babel}

\usepackage{titlesec}

% This gets the chapter section and subsection titles a bit smaller.

\titleformat{\chapter}
  {\raggedright\normalfont\huge\bfseries}{\thechapter}{20pt}{\LARGE}

\titleformat{\section}
  {\raggedright\normalfont\LARGE\bfseries}{\thesection}{18pt}{\LARGE}

\titleformat{\subsection}
  {\raggedright\normalfont\large\bfseries}{\thesubsection}{12pt}{\large}

\usepackage{fancyhdr}
\pagestyle{fancyplain}

% The following is to suppress headers and footers on blank pages.

\makeatletter
\def\cleardoublepage{\clearpage\if@twoside \ifodd\c@page\else
\hbox{}
\vspace*{\fill}
\begin{center}
\end{center}
\vspace{\fill}
\thispagestyle{empty}
\newpage
\if@twocolumn\hbox{}\newpage\fi\fi\fi}
\makeatother

\fancyhfoffset[EL,RO]{0pt}
\newcommand{\arefA}{}
\newcommand{\arefB}{\RbJvolAbbrev}
\newcommand{\arefC}{\RbJvolAbbrev\ \Roman{section}\hspace{0.6em}Ch. \arabic{subsection}\hspace{0.6em}\RbJbkcom}
\newcommand{\aref}{\arefA}

\newcommand{\RbJsref}{}
\newcommand{\RbJvolAbbrev}{}
\newcommand{\RbJbkcom}{}
\newcommand{\volumename}{}

\renewcommand{\sectionmark}[1]{\markboth{#1}{}}
\renewcommand{\subsectionmark}[1]{\markright{#1}}
\lhead[\fancyplain{}{\thepage}]         {\fancyplain{}{}}
\chead[\fancyplain{}{\slshape\leftmark}]                 {\fancyplain{}{\slshape\rightmark}}
\rhead[\fancyplain{}{}]       {\fancyplain{}{\thepage}}
\lfoot[\fancyplain{}{\aref}]            {\fancyplain{}{}}
\cfoot[\fancyplain{}{}]                 {\fancyplain{}{}}
\rfoot[\fancyplain{}{}]                 {\fancyplain{}{\volumename}}

\renewcommand{\headrulewidth}{0pt}

\usepackage[twoside,paperwidth=6in,paperheight=9in,hmargin={0.875in,0.5in},vmargin={0.5in,0.5in},includehead,includefoot]{geometry}
\usepackage{tocloft}
\usepackage{tocbibind}

\makeindex

\newcommand{\indexentry}[2]{\item #1 #2}
\newcommand{\ignore}[1]{}

\newcommand{\Avolume}[2]{\chapter{#2}\renewcommand{\RbJvolAbbrev}{\textit{#1}}\renewcommand{\volumename}{#2}\setcounter{subsection}{0}}
\newcommand{\ASbook}[1]{\section{#1}}
\newcommand{\AMbook}[2]{\section{#2}}
\newcounter{partcount}
\newcommand{\AMsection}[2]{%
\renewcommand{\thesection}{Section \Roman{section}}%
\setcounter{partcount}{\value{subsection}}\section{#2}\setcounter{subsection}{\value{partcount}}}

\newcommand{\Apart}[1]{\subsection{#1}\label{\RbJsref}}

\newcommand{\RbjAlpha}{A}
\newcommand{\Rbjalpha}{\ensuremath{\alpha}}
\newcommand{\RbjBeta}{B}
\newcommand{\RbjGamma}{\ensuremath{\Gamma}}
\newcommand{\RbjDelta}{\ensuremath{\Delta}}
\newcommand{\RbjEpsilon}{E}
\newcommand{\RbjZeta}{Z}
\newcommand{\RbjTheta}{\ensuremath{\Theta}}
\newcommand{\RbjEta}{H}
\newcommand{\RbjIota}{I}
\newcommand{\RbjKappa}{K}
\newcommand{\RbjLambda}{\ensuremath{\Lambda}}
\newcommand{\RbjMu}{M}
\newcommand{\RbjNu}{N}

\newcommand{\RbJldq}{``}
\newcommand{\RbJrdq}{''}
\newcommand{\RbJdq}{\textquotedbl}
\newcommand{\sq}{\texttt{'}}
\newcommand{\rr}{\raggedright}
\newcommand{\tn}{\tabularnewline}

\title{The Organon}
\author{Aristotle}
\date{\ }

\begin{document}

\renewcommand{\chaptername}{}

\renewcommand{\thechapter}{Volume \arabic{chapter}}
\renewcommand{\thesection}{Book \Roman{section}}
\renewcommand{\thesubsection}{Ch. \arabic{subsection}}

\addtolength{\cftchapnumwidth}{4em}
\addtolength{\cftsecnumwidth}{2.5em}
\addtolength{\cftsubsecnumwidth}{0.5em}
\cftsetpnumwidth{2em}

\frontmatter

%These are used for compounding some Greek characters
\newcommand\txo[3]{%
  \leavevmode
  \vbox{\offinterlineskip
    \halign{%
      \hfil##\hfil\cr % center
      \strut#1\cr
      \noalign{\kern#3}
      \strut#2\cr
    }%
  }%
}
\newcommand\ucd{\txo{\kern.5em ̑}{ὑ}{-1em}}
\newcommand\icd{\txo{\kern.5em ̑}{ἰ}{-1em}}


\begin{titlepage}

\maketitle

\hspace{2in}

\vfill

\begin{centering}

\vspace{0.1in}
Edited by {\it Roger Bishop Jones}\\
www.rbjones.com\\
\vspace{0.2in}
ISBN-10: 1478305622\\
ISBN-13: 978-1478305620\\
\vspace{0.2in}

{\footnotesize

\input{gitdescribe.tex}

}%footnotesize

\end{centering}

\thispagestyle{empty}
\end{titlepage}

{\parskip=0pt\tableofcontents}

\vfill

\pagebreak

\chapter*{Preface}
\addcontentsline{toc}{chapter}{Preface}

The works of Aristotle on logic have traditionally been referred to collectively as ``The Organon''. 
This bare-bones English language edition of those works was prepared for my own use when I sought but failed to find elsewhere an edition convenient for my purposes.
It is now made generally available.
 
The translations from the original Greek sources were undertaken, during the first half of the \ensuremath{20^{th}} Century, under the general editorship of Sir David Ross by:

\vspace{0.2in}

\begin{centering}
\begin{tabular}{l l}
E. M. Edghill & (Categories and On Interpretation)\\
A. J. Jenkinson & (Prior Analytics)\\
G. R. G. Mure & (Posterior Analytics)\\
W. A. Pickard-Cambridge & (Topics and On Sophistical Refutations)
\end{tabular}
\end{centering}

\vspace{0.2in}

The texts from which this edition was prepared were published online at the MIT classics archive, without any clear indication of their origin.
It seems likely that they were obtained by digital scanning and optical character recognition of the original Oxford University Press edition, followed by a (probably manual) cleanup which resulted in deleting materials which do not readily convert in this way, in fact, all but the translated text.
These materials lost in the digital copy published by the MIT archive include translators footnotes, contents listings/summaries, and the Bekker page, column and line numbers (which are marginalia in the original).
None of these omitted materials are Aristotle's, they were all provided by the translators or editors of the OUP edition, so the MIT text does include a complete translation of Aristotle's text.

I first used the MIT texts in 1996 to produce an online HTML (hypertext) edition in which I sought to facilitate navigation by breaking the text into smaller sections and providing book and chapter/part headings, thus redoing a small amount of the work undertaken by the original editors.
It was not until 2012 that I felt the need for a hard copy of The Organon, and at that point I modified the scripts used to produce the HTML edition to generate {\LaTeX} source from which PDF files suitable for use in Print-On-Demand publishing through CreateSpace.com could be obtained.
At this time I also wrote scripts to assist in producing an index.
This edition was made available in September 2012.

At that time I had not seen the original OUP edition, and was not aware of the various materials present in that edition but omitted from the MIT texts.
It was not until early in 2013 that I realised that digital scans of the original editions were available at archive.org, and was able to appreciate the full merits of the original editions.
I then undertook some improvements to bring this edition closer to the merits of the original.
These consist in the addition of Bekker page numbers and column letters, and incorporation of some of the contents summaries provided in the OUP edition (see the beginning of \emph{Categories} and \emph{On Interpretation}).
Improvements were also made to the {\LaTeX} generated contents listings, and to the information provided in the page footers.
The smallest numbered divisions of the work (which have no name in the Greek originals) were called ``parts'' in the MIT texts and hence in the first version of this edition, but were called ``chapters'' in the OUP edition of these translations, though they are generally much shorter than chapters in modern works.
I have now reverted to the terminology of the OUP edition.

Page headers and footers have been designed to aid the reader in finding his way around the book.
Bekker page numbers are now shown in the text and in the left page footer.
Though it would have been possible to put the Bekker numbers in the margins, this would have narrowed the text and lengthened the volume, leading to a higher price.
The left page footer now contains material intended to facilitate locating the points in the text referred to in commentaries and papers on Aristotle's works.

\vspace{0.15in}

Roger Bishop Jones,
May 2013


\vspace{0.15in}


Aided by a savage but informative review on Amazon I have been able to undertake some corrections of transcription and typographical errors.
Improvements in the software available have also enabled me to render more precisely those few words in the ancient Greek language which were included in the original translations (in ``on Sophistical Refutations'' Chapters 4, 14 and 21).

The typesetting is now undertaken by Lua\TeX{} enabling the use of UNICODE characters in the source and the cardio104s TrueType font, providing better support than previously available for typesetting ancient rather than contemporary Greek.
In this, as in all other aspects of the edition, it has been my aim to reproduce faithfully the original translation, not to improve on it.

\vspace{0.15in}

Roger Bishop Jones,
October 2019

\vfill

\mainmatter

\renewcommand{\aref}{\arefC}
\renewcommand{\RbJsref}{\arabic{chapter}\Roman{section}\arabic{subsection}}

\input{oi.tex}

\backmatter

%\bibliographystyle{alpha}
%\bibliography{rbj}

%\clearpage
%\listoftables
%\addcontentsline{toc}{section}{List of Tables}

%\clearpage
%\listofposition
%\addcontentsline{toc}{chapter}{List of Positions}

%\addcontentsline{toc}{part}{Index}

\renewcommand{\aref}{}
\renewcommand{\RbJsref}{}
\renewcommand{\volumename}{}
\renewcommand{\chaptermark}[1]{}
\renewcommand{\sectionmark}[1]{}

\printindex

\vfil

\end{document}

