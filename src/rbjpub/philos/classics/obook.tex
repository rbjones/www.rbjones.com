% $Id: obook.tex,v 1.2 2012/07/04 05:13:33 rbj Exp $

%\usepackage{titlesec}
%\titleformat{\chapter}{\bf\large}{\thechapter}{1em}{}
\makeatletter
    \renewcommand\section{\@startsection{section}{1}{\z@}%
                                      {-3.5ex \@plus -1ex \@minus -.2ex}%
                                      {2.3ex \@plus.2ex}%
                                      {\normalfont\large\bfseries}}
\makeatother

\usepackage{fancyhdr}
\pagestyle{fancyplain}
\newcommand{\tstamp}{\today}
\newcommand{\aref}{Vol \arabic{part} Book \Roman{chapter} Part \arabic{section}}
\renewcommand{\chaptermark}[1]{\markboth{#1}{}}
\renewcommand{\sectionmark}[1]{\markright{#1}}
\lhead[\fancyplain{}{\thepage}]         {\fancyplain{}{\nouppercase\rightmark}}
\chead[\fancyplain{}{}]                 {\fancyplain{}{}}
\rhead[\fancyplain{}{\nouppercase\leftmark}]       {\fancyplain{}{\thepage}}
\lfoot[\fancyplain{}{\aref}]            {\fancyplain{}{}}
\cfoot[\fancyplain{}{}]                 {\fancyplain{}{}}
\rfoot[\fancyplain{}{}]                 {\fancyplain{}{\aref}}
\renewcommand{\headrulewidth}{0pt}

\renewcommand{\partname}{}
\renewcommand{\chaptername}{}
%\renewcommand{\partmark}[1]{\markboth{\partname\ \thepart.\ #1}{}}
%\renewcommand{\chaptermark}[1]{\markboth{\chaptername\ \thechapter.\ #1}{}}

%\pagestyle{headings}
\usepackage[twoside,paperwidth=6in,paperheight=9in,hmargin={0.875in,0.5in},vmargin={0.5in,0.5in},includehead,includefoot]{geometry}
\usepackage{tocloft}
\usepackage{tocbibind}

\makeindex
%\usepackage{idxlayout}

\newcommand{\indexentry}[2]{\item #1 #2}
\newcommand{\ignore}[1]{}
\newcommand{\Avolume}[1]{\part{#1}}
\newcommand{\ASbook}[1]{\chapter{#1}}
\newcommand{\AMbook}[1]{\chapter{#1}}
\newcommand{\Apart}[1]{\section{#1}}
\newcommand{\dq}{\texttt{"}}
\newcommand{\sq}{\texttt{'}}

\title{The Organon}
\author{Aristotle}
\date{\ }

\begin{document}

\renewcommand{\thepart}{Volume \arabic{part}}
\renewcommand{\thechapter}{Book \Roman{chapter}}
\renewcommand{\thesection}{Part \arabic{section}:}

\addtolength{\cftpartnumwidth}{2em}
\addtolength{\cftchapnumwidth}{4em}
\addtolength{\cftsecnumwidth}{2em}

\frontmatter

\begin{titlepage}
\maketitle

\ 
\\

\vfill

\begin{centering}

{\footnotesize

%Started: 9th February 2008

%Last Change $ $Date: 2012/07/04 05:13:33 $ $

%\href{http://www.rbjones.com/rbjpub/www/books/ftd/ftdbook.pdf}{www.rbjones.com/rbjpub/www/books/ftd/ftdbook.pdf}

%Draft $ $Id: obook.tex,v 1.2 2012/07/04 05:13:33 rbj Exp $ $

%\copyright\ Roger Bishop Jones;

}%footnotesize

\end{centering}

% {\dq}a{\dq} {\sq}b{\sq}

\thispagestyle{empty}
\end{titlepage}


%\renewcommand{\cfttoctitlefont}{\Large}
%\renewcommand{\cftlottitlefont}{\Large}
{\parskip=0pt\tableofcontents}
%\addcontentsline{toc}{section}{Contents}
%{\parskip=0pt\listoftables}
%\addcontentsline{toc}{section}{List of Tables}


\pagebreak

\chapter*{Preface}
\addcontentsline{toc}{chapter}{Preface}

This English language edition of Aristotle's Organon began as hypertext in HTML, with
the main purpose of helping me to find my way around the work.
For this reason, I decided to give titles to the large number of
anonymous books and parts which comprise the Organon.

I used the language PERL to automate the transcription into HTML of publicly
available texts of the translations undertaken under the editorship of
W.~D.~Ross at the University of Oxford in the late nineteenth century.
Because of the very large number of parts, my idea of adding titles to
all the parts to facilitate navigation was only partly implemented and
moved forward very slowly.

I subsequently felt the need for a hard copy, and found that the
works of Organon did not seem to be available in a single volume, and
even in two volumes the available editions seemed very poor.
I had been aware for some time of the Print on Demand publishing
facility provided by the Amazon company CreateSpace, and I felt that
the production of a better edition could be achieved with a relatively
modest outlay of time (at no cost).

This is the result.
It is obtained primarily by modification of my Perl scripts to generate input
for the \LaTeX\ typesetting system.
In addition I filled in all the subject headers, and generated an
extended index.

The translation from the original Greek sources is that previously undertaken under the editorship of W.~D.~Ross by:

\vspace{0.2in}

\begin{centering}
\begin{tabular}{l l}
E. M. Edghill & (Categories and On Interpretation)\\
A. J. Jenkinson & (Prior Analytics)\\
G. R. G. Mure & (Posterior Analytics)\\
W. A. Pickard-Cambridge & (Topics and On Sophistical Refutations)
\end{tabular}
\end{centering}

\vspace{0.2in}

My own small contribution is to make the whole available nicely typeset
by \LaTeX\ together with some additional navigational aids,
consisting of a detailed contents list and an extensive index.

Roger Bishop Jones

July 2012

\mainmatter

\input{oi.tex}

\backmatter

%\bibliographystyle{alpha}
%\bibliography{rbj}

%\clearpage
%\listoftables
%\addcontentsline{toc}{section}{List of Tables}

%\clearpage
%\listofposition
%\addcontentsline{toc}{chapter}{List of Positions}

%\addcontentsline{toc}{part}{Index}
\printindex

\vfil

\end{document}

