
\subsection*{Contents Summary}
\addcontentsline{toc}{subsection}{\hspace{0.0in}{\it Contents Summary}}

\renewcommand{\aref}{\arefB}

\begin{description}
\item[Ch. 1.]
\begin{description}[align=parleft]
 \item[(1)] The spoken word is a symbol of thought. 

\item[(2)] Isolated thoughts or expressions are neither true nor false. 

\item[(3)] Truth and falsehood are only attributable to certain
combinations of thoughts or of words. 
\end{description}

\item[Ch. 2.]
\begin{description}
 \item[(1)] Definition of a noun. 

\item[(2)] Simple and composite nouns. 

\item[(3)] Indefinite nouns. 

\item[(4)] Cases of a noun. 
\end{description}

\item[Ch. 3.]
\begin{description}
 \item[(1)] Definition of a verb. 

\item[(2)] Indefinite verbs. 

\item[(3)] Tenses of a verb. 

\item[(4)] Verbal nouns and adjectives. 
\end{description}

\item[Ch. 4.]
Definition of a sentence. 

\item[Ch. 5.]
Simple and compound propositions. 

\item[Ch. 6.]
Contradictory propositions. 

\item[Ch. 7.]
\begin{description}
\item[(1)] Universal, indefinite, and particular affirmations and 
denials. 

\item[(2)] Contrary as opposed to contradictory propositions. 

\item[(3)] In contrary propositions, of which the subject is universal 
or particular, the truth of the one proposition implies the 
falsity of the other, but this is not the case in indefinite 
propositions. 
\end{description}

\item[Ch. 8.]
Definition of single propositions. 

\item[Ch. 9.] Propositions which refer to present or past time must be 
either true or false : propositions which refer to future time must 
be either true or false, but it is not determined which must be true 
and which false. 

\item[Ch. 10.]
\begin{description}
 \item[(1)] Diagrammatic arrangement of pairs of affirmations and 
denials,
\begin{description}
\item[(a)] without the complement] of the verb to be,
\item[(b)] with the complement of the verb to be,
\item[(c)] with an indefinite noun for subject. 
\end{description}

\item[(2)] The right position of the negative. 

\item[(3)] Contraries can never both be true, but subcontraries may both be true. 

\item[(4)] In particular propositions, if the affirmative is false, the contrary 
is true; in universal propositions, if the affirmative is false, the 
contradictory is true. 

\item[(5)] Propositions consisting of an indefinite noun and an indefinite 
verb are not denials. 

\item[(6)] The relation to other propositions of those which have an indefinite noun as subject. 

\item[(7)] The transposition of nouns and verbs makes no difference to the 
sense of the proposition. 

\end{description}

\item[Ch. 11.]

\begin{description}
 \item[(1)] Some seemingly simple propositions are really compound. 

\item[(2)] Similarly some dialectical questions are really compound. 

\item[(3)] The nature of a dialectical question. 

\item[(4)] When two simple propositions having the same subject are true, 
it is not necessarily the case that the proposition resulting from 
the combination of the predicates is true. 

\item[(5)] A plurality of predicates which individually belong to the same 
subject can only be combined to form a simple proposition when 
they are essentially predicable of the subject, and when one is 
not implicit in another. 

\item[(6)] A compound predicate cannot be resolved into simple predicates 
when the compound predicate has within it a contradiction in 
terms, or when one of the predicates is used in a secondary sense. 
\end{description}

\item[Ch. 12.]
\begin{description}
 \item[(1)] Propositions concerning possibility, impossibility, contin 
gency, and necessity. 
\item[(2)] Determination of the proper contradictories of such propositions. 
\end{description}

\item[Ch. 13.]
\begin{description}
 \item[(1)] Scheme to show the relation subsisting between such 
propositions. 

\item[(2)] Illogical character of this scheme proved. 

\item[(3)] Revised scheme. 

\item[(4)] That which is said to be possible may be

\begin{description}
\item[(a)] always actual, 
\item[(b)] sometimes actual and sometimes not,
\item[(c)] never actual. 
\end{description}

\end{description}

\item[Ch. 14.] Discussion as to whether a contrary affirmation or a denial 
is the proper [contrary of an affirmation, either universal or 
particular. 

\end{description}

\renewcommand{\aref}{\arefC}
