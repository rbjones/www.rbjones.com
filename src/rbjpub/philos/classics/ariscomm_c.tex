% $Id: ariscomm_c.tex,v 1.2 2012/09/10 19:02:03 rbj Exp $
% Change the next line before cvs checkin prior to an upload to
% CreateSpace to get up-to-date Id in book. !

%\usepackage{titlesec}
%\titleformat{\chapter}{\bf\large}{\thechapter}{1em}{}

\tolerance=10000

\usepackage[toc,nonumberlist,style=altlist]{glossaries}
\usepackage[mathletters]{ucs}
\usepackage[utf8x]{inputenc}
%\usepackage[iso-8859-7]{inputenc}
%\usepackage[greek,english]{babel}
%\usepackage{teubner}

\usepackage{fancyhdr}
\pagestyle{fancyplain}

% The following is to supress headers and footers on blank pages.

\makeatletter
\def\cleardoublepage{\clearpage\if@twoside \ifodd\c@page\else
\hbox{}
\vspace*{\fill}
\begin{center}\end{center}
\vspace{\fill}
\thispagestyle{empty}
\newpage
\if@twocolumn\hbox{}\newpage\fi\fi\fi}
\makeatother

\fancyhfoffset[EL,RO]{0pt}
\lhead[\fancyplain{}{\thepage}]         {\fancyplain{}{}}
\chead[\fancyplain{}{\slshape\leftmark}]                 {\fancyplain{}{\slshape\rightmark}}
\rhead[\fancyplain{}{}]       {\fancyplain{}{\thepage}}
\lfoot[\fancyplain{}{}]       {\fancyplain{}{}}
\cfoot[\fancyplain{}{}]       {\fancyplain{}{}}
\rfoot[\fancyplain{}{}]       {\fancyplain{}{}}

\renewcommand{\headrulewidth}{0pt}

\usepackage[twoside,paperwidth=5.25in,paperheight=8in,hmargin={0.75in,0.5in},vmargin={0.5in,0.5in},includehead,includefoot]{geometry}
\usepackage{tocloft}
\usepackage{tocbibind}

\newglossary{rg}{rgs}{rgo}{RBJ's Glossary}
\newglossary{aristotle}{ags}{ago}{Aristotelian Glossary}
\makeglossaries
\makeindex

\newcommand{\RbjAlpha}{A}
\newcommand{\Rbjalpha}{\ensuremath{\alpha}}
\newcommand{\RbjBeta}{B}
\newcommand{\RbjGamma}{\ensuremath{\Gamma}}
\newcommand{\RbjDelta}{\ensuremath{\Delta}}
\newcommand{\RbjEpsilon}{E}
\newcommand{\RbjZeta}{Z}
\newcommand{\RbjTheta}{\ensuremath{\Theta}}
\newcommand{\RbjEta}{H}
\newcommand{\RbjIota}{I}
\newcommand{\RbjKappa}{K}
\newcommand{\RbjLambda}{\ensuremath{\Lambda}}
\newcommand{\RbjMu}{M}
\newcommand{\RbjNu}{N}


\newcommand{\indexentry}[2]{\item #1 #2}

\newcommand{\ignore}[1]{}

\title{Aristotelian Commentaries}
\author{Roger Bishop Jones}
\date{\ }

\loadglsentries[rg]{arisrbj_gloss}
\loadglsentries[aristotle]{aristotle_glossary}

\begin{document}

\frontmatter

\begin{titlepage}
\maketitle

\hspace{2in}

\vfill

\begin{centering}

\vfill

\vspace{0.1in}
by Roger Bishop Jones\\
%\vspace{0.05in}
www.rbjones.com\\
\vspace{0.2in}
Published through CreateSpace\\
%\vspace{0.05in}
www.createspace.com\\
\vspace{0.2in}
ISBN-10: \\
ISBN-13: \\
\vspace{0.2in}

{\footnotesize

First edition. $ $Revision: 1.2 $~$Date: 2012/09/10 19:02:03 $ $

\copyright\ Roger Bishop Jones;

}%footnotesize

\end{centering}

\thispagestyle{empty}
\end{titlepage}

{\parskip=0pt\tableofcontents}

\vfill

\pagebreak

\chapter*{Preface}
\addcontentsline{toc}{chapter}{Preface}

This commentary is a part of a larger project in which my own philosophical system is tested and methodologically evaluated in application to the analysis of some of the history of ideas which have contributed to it.

Because of the specific purposes of this commentary in relation to \emph{positive philosophy} it is unlikely ever to be suitable for someome whose main interest is in Aristotle's philosophy.

For those who are interested in \emph{positive philosophy} and the place of these commentaries in its exposition, the best place to start is with the introductory chapters to \emph{Organon and Metaphysic}\cite{rbjbOrgMetap}.

\mainmatter

\chapter{Introduction}

These notes on two of the most fundamental parts of Aristotle's philosophy are written as a part of the elaboration and exposition of my own ``\emph{positive}'' philosophy.
My study of Aristotle is concurrent with the exposition of corresponding parts of my own philosophy in the projected volume \emph{Organon and Metaphysic}\cite{rbjbOrgMetap}.
The study of Aristotle is intended to support that exposition in two principle ways.

Firstly it is intended to enrich the exposition by comparison with the Aristotelian approach to the various problems addressed, including of course the problem of what problems are fundamental to theoretical philosophy.

\emph{Organon and Metaphysic} is substantially engaged in articulating a conception of analytic method, and the second role of these notes in relation to that enterprise is as exemplifying that method.
The commentaries are the informal preliminaries of that method, in which formal models form an important part (but are to be presented elsewhere).

The detailed analysis and formal modelling are not intended to cover the whole of Aristotle's Organon and Metaphysics, but will focus on a small number of areas which are important in understanding the relationship between Aristotle's thought and \emph{positive philosophy}, and which are likely to provide fruitful exemplars of the proposed methods of analysis.

My notes begin prior to the selection of these focal points, they are a part of my process of engaging with Aristotle.
This is not of course a process which takes place in vacuo.
The notes come prior to that detailed exposition of positive philosophy to which they contribute, but are written in the context of a conception of positive philosophy which has already matured gently over many decades.


\chapter{The Metaphysics}

\section{Book \RbjAlpha}

Though we know this subject under the heading ``metaphysics'', this name appears only after Aristotle, who called the subject matter ``first philosophy''.
This is Aristotle's account of ``first philosophy'', the subject matter of the volume which has become known as ``The Metaphysics''.

First of all Aristotle considers the question ``what is wisdom?''.
He takes wisdom to be a kind of knowledge, and distinguishes it from other kinds of knowledge, and concludes that wisdom, which is the subject matter of first philosophy, is concerned with original causes or first principles.

In relating Aristotle's metaphysics to a contemporary position such as the theoretical core of positive philosophy, a comparison both with Aristotle's ideas about what wisdom is and his ideas about what it therefore should be concerned with.
We might then concur with the former, but diverge in some ways from the latter.

\section{Book \Rbjalpha}

\section{Book \RbjBeta}

This contains Aristotle's list of 14 key problems for metaphysics.

A matter of interest here is whether these are comprehensible from the perspective of positive philosophy, as well defined problems.

\section{Book \RbjGamma}

This book begins with the assertion that there exists a single science addressing the kinds of subject matters which fall under ``first philosophy''.
There follow materials about the law of contradiction and the law of excluded middle.

\section{Book \RbjDelta}

This is Aristotle's glossary, a condensation of which appears below.

\backmatter

%\clearpage
%\listoftables
%\addcontentsline{toc}{section}{List of Tables}

%\clearpage
%\listofposition
%\addcontentsline{toc}{chapter}{List of Positions}

%\addcontentsline{toc}{part}{Index}
\gls{definition}

\glsaddall
\renewcommand{\glossarypreamble}{
This is a glossary covering both the language of Aristotle and also some terms used in these commentaries which either do not appear in the translations of Aristotle or are sometimes used in the commentary with a distinct sense.
The glossary entries here will be my own best explanation of the term, insofar as I have understood it or as I am using it, not necessarily the explanation or definition given by Aristotle.
}
\printglossary[type=rg]

\renewcommand{\glossarypreamble}{
This is a glossary of Aristotelian terms explained using the words of Aristotle, usually in literal quotations from the Organon or the Metaphysics.
Part references are supplied in the form: VolBookPart, where Vol is the volume number as an arabic numeral, Book is the Book number as an uppercase Roman numeral, and Part is the part number as an arabic numeral.
The volume numbers are as follows:
\begin{enumerate}
\item Categories
\item On Interpretation
\item Prior Analytics
\item Posterior Analytics
\item Topics
\item Sophistical Refutations
\item Metaphysics
\end{enumerate}
}

\printglossary[type=aristotle]


\bibliographystyle{alpha}
\bibliography{rbj}


\printindex

\vfil

\end{document}

