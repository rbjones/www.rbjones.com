\glossentry{}{}{}
\glossentry{accident}{5I5}{something which, though neither a definition nor a property nor a genus yet belongs to the thing (something which may possibly either belong or not belong to any one and the self-same thing)}
\glossentry{}{}{}
\glossentry{affirmation}{1I10}{By `affirmation' we mean an affirmative proposition.}
\glossentry{affirmation}{2I10}{An affirmation is the statement of a fact with regard to a subject.}
\glossentry{}{}{}
\glossentry{argument}{}{}
\glossentry{}{}{}
\glossentry{assertoric}{}{}
\glossentry{}{}{}
\glossentry{atomic}{PTA 1 15}{I call 'atomic' connexions or disconnexions which involve no intermediate term.}
\glossentry{}{}{}
\glossentry{category}{}{}
\glossentry{}{}{}
\glossentry{connexion}{}{}
\glossentry{}{}{}
\glossentry{definition}{POA II10}{(a) an indemonstrable statement of essential nature, or (b) a syllogism of essential nature differing from demonstration in grammatical form, or (c) the conclusion of a demonstration giving essential nature.}
\glossentry{definition}{5I5}{a phrase signifying a thing's essence}
\glossentry{}{}{}
\glossentry{definitory}{5I5}{everything that falls under the same branch of
inquiry as definitions}
\glossentry{}{}{}
\glossentry{denial}{1I10}{By `denial' a `negative proposition'}
\glossentry{}{}{}
\glossentry{dialectical}{5I10}{a proposition or problem which is open to debate, a subject of enquiry}
\glossentry{}{}{}
\glossentry{disconnexion}{}{}
\glossentry{}{}{}
\glossentry{figure}{}{}
\glossentry{}{}{}
\glossentry{genus}{5I5}{what is predicated in the category of essence of a number of things exhibiting differences in kind}
\glossentry{}{}{}
\glossentry{hypothetical}{}{}
\glossentry{}{}{}
\glossentry{individual}{}{}
\glossentry{}{}{}
\glossentry{induction}{}{}
\glossentry{}{}{}
\glossentry{organon}{tool}{}
\glossentry{}{}{}
\glossentry{particular}{}{}
\glossentry{}{}{}
\glossentry{premiss}{}{}
\glossentry{}{}{}
\glossentry{property}{5I5}{a predicate which does not indicate the essence of a thing, but yet belongs to that thing alone, and is predicated convertibly of it}
\glossentry{}{}{}
\glossentry{proposition}{}{}
\glossentry{}{}{}
\glossentry{reasoning}{}{}
\glossentry{}{}{}
\glossentry{species}{}{}
\glossentry{}{}{}
\glossentry{syllogism}{}{}
\glossentry{}{}{}
\glossentry{theses}{5I11}{}
\glossentry{}{}{}
\glossentry{universal}{}{}
\glossentry{}{}{}
\glossentry{}{}{}
