=TEX
\begin{slide}{}
\begin{center}
{\bf Formal Methods\\
in\\
ICL\\
Secure Systems}

\ \\
\ \\
\ \\

\small{R.B.~Jones \\
\ \\
International Computers Limited,\\
Eskdale Road,\\
Winnersh,\\
Wokingham,\\
Berks RG11 5TT\\
\ \\
tel: 0734 693131 x6536,\\
fax: 0734 697636\\
email: uucp: rbj@win.icl.co.uk}
\end{center}
\end{slide}

\begin{slide}{}

\begin{center}
{\bf FORMAL METHODS}

\ \\

applying\\
\ \\
LOGIC\\
\ \\
and\\
\ \\
MATHEMATICS\\
\ \\
to\\
\ \\
HIGH ASSURANCE\\
SYSTEMS ENGINEERING

\ \\

{\bf ASSURANCE through PROOF}
\end{center}
\end{slide}

\begin{slide}{}

\begin{center}

\ \\

{\bf special\\
REQUIREMENTS}\\
-\\
HIGH ASSURANCE

\ \\

{\bf special\\
METHODS}\\
-\\
based on\\
FORMAL SPECIFICATION and PROOF

\ \\

{\bf special\\
TECHNOLOGY}\\
-\\
NEW `LOGIC LANGUAGES'\\
NEW `LOGIC ENVIRONMENT'

\end{center}
\end{slide}


\begin{slide}{}
\begin{center}
{\bf Business Motivators}
\end{center}

\begin{itemize}
\item
Quality

\item
Secure Systems Market

\item
Safety Critical Systems Market

\item
Professional Services Market supporting above

\item
Certification Schemes for Secure (and Safety Critical) Systems
\end{itemize}

\end{slide}

\begin{slide}{}

\begin{center}
{\bf Application Areas}
\end{center}


\begin{itemize}
\item
Security

\begin{itemize}
\item
Non-disclosure of information
\item
Authentication of users
\item
Integrity
\end{itemize}

\item
Safety Critical Systems

\begin{itemize}
\item
Aerospace
\item
Medical
\item
Industrial Plant
\item
Nuclear plant
\item
Public transport
\end{itemize}

\end{itemize}
\end{slide}


\begin{slide}{}
\begin{center}
{\bf Security Certification Schemes}


US Department of Defence
Trusted Computer System
Evaluation Criteria
``Orange Book''

CESG Computer Security Memorandum No.3
``UK Systems Security Confidence Levels''

Harmonised Criteria
Information Technology Security
Evaluation Criteria
``ITSEC''

{\bf Safety Critical}

UK MOD
interim defence standards 00-55/56
\end{center}
\end{slide}

\begin{slide}{}

\begin{center}
{\bf {Applications In ICL Secure Systems}}
\end{center}

\begin{itemize}
\item
Use of VDM on VDAP compiler
\item
Z on various Design Study and Security Modelling contracts for HMG
\item
HOL/Z on CESG/ICL OWR project
\item
ICL HOL and Z proof tool
\item
Z for security modelling on major contracts
\item
VDM/Z used in various bids. 
\end{itemize}

\end{slide}

\begin{slide}{}
\begin{center}
{\bf The CESG/ICL OWR Project}

\begin{itemize}
\item
A design-and-implement contract placed with ICL by CESG (Communications and Electronics Security Group, GCHQ)
\item
Design and make pre-production models of a ONE WAY REGULATOR to the highest achievable standards of assurance.
\item
Hardware only solution
\item
Full formal verification of detailed design against formal security policy.
\item
Product certified as ``exceeding the requirements of UKL6''
\end{itemize}
\end{center}
\end{slide}

\begin{slide}{}

\begin{center}

{\bf { PROOF}
themes

\ 

focus formality


automate proof}

\ 

\end{center}


{\bf { METHODs}} supporting selective application of formality to best effect in combination with other methods.



{\bf { TOOLs}} proving least cost proof development support integrated with support for other methods.

\end{slide}


\begin{slide}{}
\begin{center}
{\bf FORMAL DEVELOPMENT
for CRITICAL COMPONENTS}

\ 

REQUIREMENTS
supplemented by
HAZARD ANALYSIS

IDENTIFY CRITICAL COMPONENTS

Use FORMAL approach
for CRITICAL COMPONENTS

Use STRUCTURED approach
for NON-CRITICAL COMPONENTS

\end{center}
\end{slide}


\begin{slide}{}
\begin{center}
{\bf  FORMAL DEVELOPMENT
for CRITICAL COMPONENTS
-
FOR and AGAINST}

FOR

formal treatment focused
on critical components

AGAINST

critical components may be
incorrectly identified

specifications of critical
components may be incorrect

\end{center}

\end{slide}

\begin{slide}{}
\begin{center}
{\bf  FORMAL DEVELOPMENT
for CRITICAL REQUIREMENTS}


formalise critical requirements\\
on SYSTEM

formally model architecture

formalise critical requirements\
on SUBSYSTEMS

verify architecture

repeat through structured design process

implement and verify
using pre/post conditions

\end{center}
\end{slide}

\begin{slide}{}
\begin{center}
{\bf FORMAL DEVELOPMENT
of CRITICAL REQUIREMENTS
-
FOR and AGAINST}

FOR

formal treatment focused\\
on critical requirements

identification of critical components\\
formally verified

requirements on critical components\\
formally verified


AGAINST

lack of literature on techniques
\end{center}

\end{slide}

\begin{slide}{}

\begin{center}

{\bf PROCESSING of FORMAL (Z) SPECIFICATIONS}

SYNTAX CHECKING

TYPE CHECKING\\
..................\\
CONSISTENCY PROOFS

SEMANTIC WELL-FORMEDNESS PROOFS

PRE-CONDITION SIMPLIFICATION

REFINEMENT VERIFICATION\\
...................\\
PROOF of CRITICAL PROPERTIES

CODE/HARDWARE VERIFICATON

\end{center}


\end{slide}

\begin{slide}{}

\begin{center}

{\bf  REQUIREMENTS for PROOF TOOLS}

\ 

\ 

SOUNDNESS/INTEGRITY

\ 

PRODUCTIVITY

\ 

ADAPTABILITY/EXTENDIBILITY

\end{center}

\end{slide}

\begin{slide}{}

\begin{center}

{\bf  ICL PROOF TOOL}

\ 

well established\\
proof support paradigm\\
(LCF paradigm)


modern functional metalanguage\\
(ML)


primary object language HOL\\
(Higher Order Logic)


support for multiple\\
`secondary' object languages\\
(e.g. Z)

\end{center}

\end{slide}


\begin{slide}{}

\begin{center}
{\bf THE LCF PARADIGM}
\end{center}
\begin{itemize}
\item
implement proof checker using
a TYPED FUNCTIONAL programming LANGUAGE
as META-LANGUAGE (e.g. SML)
\item
abstract data type of THEOREMS
GUARANTEED SOUND by the type checker
(assuming the logic is well defined)
\item
META-LANGUAGE is AVAILABLE TO the USER
for programming proofs and other customisation,
WITHOUT risk of COMPROMISING the CHECKER.
\end{itemize}
\end{slide}


\begin{slide}{}

\begin{center}
{\bf BENEFITS of the LCF PARADIGM}
\end{center}

\begin{itemize}
\item
HIGH ASSURANCE of SOUNDNESS
\item
EASY to CUSTOMISE and EXTEND
\item
COMPLETE USER CONTROL
of PROOF STRATEGY
\item
RERUNNABLE PROOF SCRIPTS
\item
LEG WORK convertible to HEAD WORK\\
by PROGRAMMING in META-LANGUAGE
\end{itemize}
\end{slide}


\begin{slide}{}

\begin{center}

{\bf ICL PROOF TOOL\\
-\\
CRITICAL REQUIREMENT}\\
all theorems are true\\
if\\
all extensions are conservative

{\bf  ARCHITECTURE}

\end{center}

code for management of theory database and checking of proofs separated out and protected using abstract data type

code for all other functions (e.g. syntax checking, type inference, proof heuristics) written in Standard ML, but non-critical.

System user extensible.
\end{slide}

\begin{slide}{}

\begin{center}
{\bf Current Status}
\end{center}

\begin{itemize}
\item
Under development in government supported FST project

\item
Prototype ICL HOL system developed and used on project since MAY 1990.
\item
Prototype Z proof support on HOL prototype.
\item
`Product quality' ICL HOL well progressed.
\item
Collaboration with PVL to provide integrated support for development through to SPARC (Ada subset) implementation.
\end{itemize}

\end{slide}




