% $Id: t050a.tex,v 1.1 2013/01/03 17:12:44 rbj Exp $

\documentclass[11pt]{article}
\usepackage{latexsym}
\usepackage{rbj}

\ftlinepenalty=9999
\usepackage{A4}

% the following two modal operators come from the amsfonts package
\def\PrKI{\Diamond}	%Modify printing for \250
\def\PrKJ{\Box}		%Modify printing for \251

\def\PrJA{\|-}		%Modify printing for  (syntactic entailment)
\def\PrJI{\models}	%Modify printing for ˜ (semantic entailment)

\def\ouml{\"o}

\tabstop=0.4in
\newcommand{\ignore}[1]{}

\def\thyref#1{Appendix \ref{#1}}

%\def\ExpName{\mbox{{\sf exp}}}
%\def\Exp#1{\ExpName(#1)}


\title{Illative Combinatory Logic}
\makeindex
\usepackage[unicode,pdftex]{hyperref}
\hypersetup{pdfauthor={Roger Bishop Jones}, pdffitwindow=false}
\hypersetup{colorlinks=true, urlcolor=red, citecolor=blue, filecolor=blue, linkcolor=blue}
\author{Roger Bishop Jones}
\date{\ }

\begin{document}
\begin{titlepage}
\maketitle
\begin{abstract}
Another approach to illative combinatory logic, based this time on the hol4 example on pure combinatory logic.
Mainly an attempt to understand why that example is so much simpler than my own efforts.
\end{abstract}
\vfill

\begin{centering}
{\footnotesize

Created 2012/12/18

\input{t050i.tex}

\href{http://www.rbjones.com/rbjpub/pp/doc/t050.pdf}
{http://www.rbjones.com/rbjpub/pp/doc/t050.pdf}

\copyright\ Roger Bishop Jones; Licenced under Gnu LGPL

}%footnotesize
\end{centering}

\thispagestyle{empty}
\end{titlepage}

\newpage
\addtocounter{page}{1}
{\parskip=0pt\tableofcontents}

\section{Prelude}

Every now and then I wonder why the things I try to do with ProofPower are so tortuously lengthy and slow.
Too rarely I look at what is happening elsewhere and wonder whether I can learn anything useful from them.

This is part of such an enterprise.

I am engaged in work on an infinitary illative combinatory logic, but a large proportion of what I want to do with it does not depend on its infinitary nature but could be done with a finitary illative combinatory logic.
So it might be worth progressing without the infinitary aspect, to test out the ideas which can be progressed with finite combinators.

In hol4 there is a very simple example of a Church Rosser proof for the combinatory logic which I imagine would be a good starting point for such an enterprise.
I am therefore looking at this material both in ProofPower and in HOL4, trying to replicate the HOL4 proof in ProofPower and to extend both proofs to cover illative combinatory logics.

Discussion of what might become of this document in the future may be found the postscript (Section \ref{POSTSCRIPT}).

In this document, phrases in coloured text are hyperlinks, like on a web page, which will usually get you to another part of this document (the blue parts, the contents list, page numbers in the Index) but sometimes take you (the red bits) somewhere altogether different (if you happen to be online), e.g.: \href{http://rbjones.com/rbjpub/pp/doc/t050.pdf}{the online copy of this document}.

\cite{rbjt000}

\section{Introduction}
