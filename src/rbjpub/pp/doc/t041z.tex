% $Id: t041z.tex,v 1.1 2010/09/13 11:07:50 rbj Exp $

\section{Postscript}\label{POSTSCRIPT}

The present state of the document represents my first more or less complete essay at the syntax of the proposed infinitary calculus.
I am banking this version (in cvs) because I now perceive some fundamental difficulties in my conception of how this would work, and will therefore have to make a fairly radical revision to the proposed syntax.

The problem lies in my conception of infinitary abstraction and application, which confuses the scope of the bound variables.
I am now inclined to revert to finitary abstraction and application, but add functions as small graphs, by contrast with abstraction which may be thought of as functions as rules.

This is a bit of a drag while I am hand cranking the abstract syntax, since it pushes up the number of constructors to five, but I can't think of a better way at present.

I therefore expect the next version of this document to constitute a complete rework for a different abstract syntax.

\appendix

\vfill

\include{lambda.th}

\pagebreak

\section*{Bibliography}\label{BIBLIOGRAPHY}
\addcontentsline{toc}{section}{Bibliography}

{\def\section*#1{\ignore{#1}}
\raggedright
\bibliographystyle{rbjfmu}
\bibliography{rbj,fmu}
} %\raggedright

{\twocolumn[\section*{Index}\label{INDEX}]
\addcontentsline{toc}{section}{Index}
{\small\printindex}}

\end{document}
