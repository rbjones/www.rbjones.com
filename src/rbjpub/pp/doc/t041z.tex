% $Id: t041z.tex,v 1.2 2010/11/09 13:32:40 rbj Exp $

\section{Postscript}\label{POSTSCRIPT}

The present state of the document is the result of several attempts at a suitable abstract syntax for the infinitary system.

The principle difficulty has naturally been in relation to the infinitary aspects of the system.
The present document presents probably the simplest approach to this, in which the syntactically infinitary element is in function application.

In the last iteration, as a result of a (probably misguided) flush of confidence, I decided that the system would integrate better into {\Product} if it were polymorphic in some type of urelments.
This necessitated producing a version of the underlying set theory with urelements.

This yields the simplest abstract syntax of all the variants I have so far considered, and I am sufficiently happy with it to proceed to the semantics.

\appendix

\vfill

\include{icomb.th}

\pagebreak

\section*{Bibliography}\label{BIBLIOGRAPHY}
\addcontentsline{toc}{section}{Bibliography}

{\def\section*#1{\ignore{#1}}
\raggedright
\bibliographystyle{rbjfmu}
\bibliography{rbj,fmu}
} %\raggedright

{\twocolumn[\section*{Index}\label{INDEX}]
\addcontentsline{toc}{section}{Index}
{\small\printindex}}

\end{document}
