\def\rbjidbAABtex{$$Id: b001.tex,v 1.13 2011/01/05 11:02:31 rbj Exp $$}

\documentclass[11pt]{book}
\usepackage{latexsym}
\usepackage{ProofPower}
\usepackage{amssymb}
\ftlinepenalty=9999

%\usepackage{A4} see next two lines instead.
\advance\parskip by 0.5\baselineskip
\parindent 0pt  % was 20pt, in `lplain.tex', and 15pt in `art10.sty'.

% the following two modal operators come from the amsfonts package
\def\PrKI{\Diamond}	%Modify printing for �
\def\PrJI{\models}	%Modify printing for �

\def\PrJA{\Vdash}	%Modify printing for � (syntactic entailment)
\def\PrJI{\vDash}	%Modify printing for � (semantic entailment)

\def\thyref#1{listing of theory {\it #1}\cite{rbjb001}}
\def\ExpName{\mbox{{\sf exp}}}
\def\Exp#1{\ExpName(#1)}

\tabstop=0.4in
\newcommand{\ignore}[1]{}

\title{Analyses of Analysis - Part I - Exegetical Analysis}
\makeindex
\usepackage[unicode,pdftex]{hyperref}
\pagestyle{headings}
%\usepackage[a4paper,totalwidth={6.75in},totalheight={10.5in},includehead,includefoot]{geometry}
\usepackage{amazon2}
\author{Roger Bishop Jones}
\date{\ }

\begin{document}
\begin{titlepage}
\maketitle

\vfill

\begin{centering}
{\footnotesize

Created 2009/06/18

Last Change $ $Date: 2011/01/05 11:02:31 $ $

$ $Id: b001.tex,v 1.13 2011/01/05 11:02:31 rbj Exp $ $

\copyright\ Roger Bishop Jones

}%footnotesize
\end{centering}

\thispagestyle{empty}
\end{titlepage}

\newpage
\addtocounter{page}{1}
{\parskip=0pt\tableofcontents}

\chapter{Introduction}

\section{Methods}

The analyses in this volume of aspects of the work of various important philosophers are undertaken using modern formal methods with the help of supporting information technology.
Information about these tools and methods is presented in Volume II of this work, which is concerned with synthetic rather than exegetical analysis.

\section{Themes}

It is my aim to trace the history and origin of some of the ideas which underpin formal analysis of the kinds addressed in Volume II.
It is to be expected that as the work progresses my ideas about what the important ideas are will evolve, and this account of the principle ideas will be expanded and refined.

In the first place the following short list will suffice:

\begin{itemize}
\item The distinction between logical and factual truths.
My use of the definite article here reflects my prejudice that there is one important distinction which has and continues to evolve corresponding to Hume's distinction between <i>relations between ideas</i> and <i>matters of fact</i>.
The various related dichotomies include:
\begin{itemize}
\item Necessity and Contingency 
\item Essential and Accidental predication
\item Demonstrative truth
\item a priori and a posteriori knowledge
\item the analytic/synthetic distinction
\item Logical verses empirical truth.
\item The notions of tautology, truth functions and truth conditions
\end{itemize}
\item The evolution of formality.
\end{itemize}

\chapter{Plato and Aristotle}

\include{t028m}

\chapter{Leibniz}

\include{t032m}

\chapter{Hume and Kant}

\chapter{Frege}

\chapter{Russell and Wittgenstein}

\include{t030m}

\chapter{Grice}

\include{t037m}

\chapter{Conclusions}

\pagebreak

%\section*{Bibliography}
\label{Bibliography}
\addcontentsline{toc}{chapter}{Bibliography}

{\def\section*#1{\ignore{#1}}
\raggedright
\bibliographystyle{rbjfmu}
\bibliography{rbj,fmu}
} %\raggedright

\vfil

\pagebreak
\label{Index}
\addcontentsline{toc}{chapter}{Index}
{\twocolumn[\section*{Index}]
{\small\printindex}}

\vfill

{\tiny
\begin{centering}
\begin{tabular}{l}
\rbjidbAABtex\\ %t001
%\rbjidtACJdoc\\ %t029
\rbjidtACIdoc\\ %t028
\rbjidtADCdoc\\ %t032
\rbjidtADAdoc\\ %t030
\end{tabular}
\end{centering}}

\end{document}
