% $Id: t046a.tex,v 1.7 2011/04/25 19:51:22 rbj Exp $

\documentclass[11pt]{article}
\usepackage{latexsym}
\usepackage{rbj}

\ftlinepenalty=9999
\usepackage{A4}

% the following two modal operators come from the amsfonts package
\def\PrKI{\Diamond}	%Modify printing for \250
\def\PrKJ{\Box}		%Modify printing for \251

\def\PrJA{\|-}		%Modify printing for  (syntactic entailment)
\def\PrJI{\models}	%Modify printing for ˜ (semantic entailment)
\def\PrJO{\prec}	%Modify printing for \236 (semantic entailment)

\def\PrIO{\MMM{\notin}}	%Modify printing for \216 (not member of)

\tabstop=0.4in
\newcommand{\ignore}[1]{}

\def\thyref#1{Appendix \ref{#1}}

%\def\ExpName{\mbox{{\sf exp}}}
%\def\Exp#1{\ExpName(#1)}

\title{Iterative Foundational Ontologies}
\makeindex
\usepackage[unicode,pdftex]{hyperref}
\hypersetup{pdfauthor={Roger Bishop Jones}, pdffitwindow=false, pdfkeywords=RogerBishopJones}
\hypersetup{colorlinks=true, urlcolor=red, citecolor=blue, filecolor=blue, linkcolor=blue}
\author{Roger Bishop Jones}
\date{\ }

\begin{document}
\begin{titlepage}
\maketitle
\begin{abstract}
A broad discussion of ontologies for mathematics and abstract semantics.
\end{abstract}
\vfill

\begin{centering}
{\footnotesize

Created 2011/02/24

\input{t046i.tex}

\href{http://www.rbjones.com/rbjpub/pp/doc/t046.pdf}
{http://www.rbjones.com/rbjpub/pp/doc/t046.pdf}

\copyright\ Roger Bishop Jones; Licenced under Gnu LGPL

}%footnotesize
\end{centering}

\thispagestyle{empty}
\end{titlepage}

\newpage
\addtocounter{page}{1}
{\parskip=0pt\tableofcontents}

\section{Prelude}

This is one of a series of documents in which philosophically motivated technical issues are explored making use of an interactive proof tool for a higher order logic.


This document began as an exploration of a draft of \emph{Pluralities and Sets} by {\O}ystein Linnebo \cite{linneboPS}.
Once the bare bones were in place I had to decide where to go from there, and I recalled a paper by Thomas Forster in which he extends the iterative conception of set to embrace set theories with a universal set \cite{forsterTICS}, and decided that iterative ontology would provide a theme under which a broad exposure of my own foundational programme might be presented.

Discussion of what might become of this document in the future may be found the postscript (Section \ref{POSTSCRIPT}).

In this document, phrases in coloured text are hyperlinks, like on a web page, which will usually get you to another part of this document (the blue parts, the contents list, page numbers in the Index) but sometimes take you (the red bits) somewhere altogether different (if you happen to be online), e.g.: \href{http://rbjones.com/rbjpub/pp/doc/t046.pdf}{the online copy of this document}.

\section{Changes}

\subsection{Recent Changes}

\paragraph{Version 1.7}

The scope of the document has been considerably extended and the title has therefore been changed.
The discussion of Linnebo's \emph{Pluralities and Sets} is now one section.

\paragraph{Version 1.6}

Cosmetic changes, corrections to text, rationalisation of use of subscripts.

\paragraph{Version 1.5}

Two changes have been made to the specifications, both of which represent corrections to my reading of Linnebo's intentions, and proofs reworked as necessary.

The first is that the modal version of plural quantification has been made sensitive to the stage, so that plural quantifiers range only over plurals of sets individuated at that stage (previously they ranged over plurals of sets formed at any stage).

The second is to the requirement on stages, which was previously stronger than Linnebo's in asserting closure under union%
\footnote{Which was not so much a misreading of Linnebo, as a careless mistranlation.}.
This is now weakened to directedness.

I have also, \emph{en passant}, tried to ensure that all plural quantifications bind a double letter variable.

\subsection{Changes Under Consideration}

It seems to me that some further requirements should be placed on the stages, but I see nothing in Linnebo's paper to underwrite them.

Firstly that the stages should exhaust the sets.
Secondly that ``ThisStage'' (the default stage in which judgements are to be understood) should be the first stage, and should be empty.

\subsection{Issues}

Immediately following from the above considerations about stages is the question whether of necessity there should be any sets at all (prior to assuming COLLAPSE or some other ontological principle).

Linnebo talks of plurals as being necessarily non-empty, but does not reflect this in his plural comprehension scheme.
I have therefore allowed them to be empty.

If I insisted on their being non-empty, then a contradiction could be derived from plural comprehension alone (without COLLAPSE).
If on the other hand, we insist on them being non-empty and restrict plural comprehension accordingly, then it looks as if we are not then able to prove that there are any sets (unless this is made a part of the conception of stages), and cannot even derive a contradiction from the combination of plural comprehension and COLLAPSE.

See also Section \ref{POSTSCRIPT}.

\section{Introduction}
