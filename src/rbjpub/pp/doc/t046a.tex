% $Id: t046a.tex,v 1.4 2011/03/28 21:18:50 rbj Exp $

\documentclass[11pt]{article}
\usepackage{latexsym}
\usepackage{rbj}

\ftlinepenalty=9999
\usepackage{A4}

% the following two modal operators come from the amsfonts package
\def\PrKI{\Diamond}	%Modify printing for \250
\def\PrKJ{\Box}		%Modify printing for \251

\def\PrJA{\|-}		%Modify printing for  (syntactic entailment)
\def\PrJI{\models}	%Modify printing for ˜ (semantic entailment)
\def\PrJO{\prec}	%Modify printing for \236 (semantic entailment)

\def\PrIO{\MMM{\notin}}	%Modify printing for \216 (not member of)

\tabstop=0.4in
\newcommand{\ignore}[1]{}

\def\thyref#1{Appendix \ref{#1}}

%\def\ExpName{\mbox{{\sf exp}}}
%\def\Exp#1{\ExpName(#1)}

\title{Pluralities and Sets}
\makeindex
\usepackage[unicode,pdftex]{hyperref}
\hypersetup{pdfauthor={Roger Bishop Jones}, pdffitwindow=false, pdfkeywords=RogerBishopJones}
\hypersetup{colorlinks=true, urlcolor=red, citecolor=blue, filecolor=blue, linkcolor=blue}
\author{Roger Bishop Jones}
\date{\ }

\begin{document}
\begin{titlepage}
\maketitle
\begin{abstract}
This is a formal exploration with ProofPower of a topic addressed by {\O}ystein Linnebo in a paper under the same title.
\end{abstract}
\vfill

\begin{centering}
{\footnotesize

Created 2011/02/24

\input{t041i.tex}

\href{http://www.rbjones.com/rbjpub/pp/doc/t046.pdf}
{http://www.rbjones.com/rbjpub/pp/doc/t046.pdf}

\copyright\ Roger Bishop Jones; Licenced under Gnu LGPL

}%footnotesize
\end{centering}

\thispagestyle{empty}
\end{titlepage}

\newpage
\addtocounter{page}{1}
{\parskip=0pt\tableofcontents}

\section{Prelude}

This is one of a series of documents in which philosophically motivated technical issues are explored making use of an interactive proof tool for a higher order logic.

The subject matter in this case is a paper by {\O}ystein Linnebo entitled \emph{Pluralities and Sets}\cite{linneboPS}.

Discussion of what might become of this document in the future may be found the postscript (Section \ref{POSTSCRIPT}).

In this document, phrases in coloured text are hyperlinks, like on a web page, which will usually get you to another part of this document (the blue parts, the contents list, page numbers in the Index) but sometimes take you (the red bits) somewhere altogether different (if you happen to be online), e.g.: \href{http://rbjones.com/rbjpub/pp/doc/t046.pdf}{the online copy of this document}.

\section{Introduction}
