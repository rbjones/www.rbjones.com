% $Id: t046z.tex,v 1.4 2011/04/25 19:51:22 rbj Exp $

\section{Postscript}\label{POSTSCRIPT}

I have now worked through the main body of Linnebo's paper, and the second of his two appendices.

The first apendix is different in character, obtaining a result stated in terms of plural set theory for which an equivalent theorem is uncontroversial in ZFC, I shall not attempt to formalise this proof.

The second appendix gives more details of and results in the modal/plural set theory.
Much of this is metatheoretic.
It would be possible to treat this more fully within the bounds of the syntax-free methods I have adopted, but this would be significantly more complex and time consuming than any of the other material so far undertaken, for relatively little further benefit.

Beyond these technical details there is then the question of the philosophical significance both of Linnebo's paper and of this kind of formal treatment.

There are then some problems in relation to the semantics of set theory not addressed by Linnebo which might benefit from the modal approach, and I may possibly add further material on these matters.

\pagebreak

\appendix

\section{Theory Listings}

{
\let\Section\subsection
\let\Subsection\subsubsection
\def\subsection#1{\Subsection*{#1}}

\def\section#1{\Section{#1}\label{t046a}\index{t046a}}
% $Id: t046a.tex,v 1.4 2011/03/28 21:18:50 rbj Exp $

\documentclass[11pt]{article}
\usepackage{latexsym}
\usepackage{rbj}

\ftlinepenalty=9999
\usepackage{A4}

% the following two modal operators come from the amsfonts package
\def\PrKI{\Diamond}	%Modify printing for \250
\def\PrKJ{\Box}		%Modify printing for \251

\def\PrJA{\|-}		%Modify printing for  (syntactic entailment)
\def\PrJI{\models}	%Modify printing for ˜ (semantic entailment)
\def\PrJO{\prec}	%Modify printing for \236 (semantic entailment)

\def\PrIO{\MMM{\notin}}	%Modify printing for \216 (not member of)

\tabstop=0.4in
\newcommand{\ignore}[1]{}

\def\thyref#1{Appendix \ref{#1}}

%\def\ExpName{\mbox{{\sf exp}}}
%\def\Exp#1{\ExpName(#1)}

\title{Pluralities and Sets}
\makeindex
\usepackage[unicode,pdftex]{hyperref}
\hypersetup{pdfauthor={Roger Bishop Jones}, pdffitwindow=false, pdfkeywords=RogerBishopJones}
\hypersetup{colorlinks=true, urlcolor=red, citecolor=blue, filecolor=blue, linkcolor=blue}
\author{Roger Bishop Jones}
\date{\ }

\begin{document}
\begin{titlepage}
\maketitle
\begin{abstract}
This is a formal exploration with ProofPower of a topic addressed by {\O}ystein Linnebo in a paper under the same title.
\end{abstract}
\vfill

\begin{centering}
{\footnotesize

Created 2011/02/24

\input{t041i.tex}

\href{http://www.rbjones.com/rbjpub/pp/doc/t046.pdf}
{http://www.rbjones.com/rbjpub/pp/doc/t046.pdf}

\copyright\ Roger Bishop Jones; Licenced under Gnu LGPL

}%footnotesize
\end{centering}

\thispagestyle{empty}
\end{titlepage}

\newpage
\addtocounter{page}{1}
{\parskip=0pt\tableofcontents}

\section{Prelude}

This is one of a series of documents in which philosophically motivated technical issues are explored making use of an interactive proof tool for a higher order logic.

The subject matter in this case is a paper by {\O}ystein Linnebo entitled \emph{Pluralities and Sets}\cite{linneboPS}.

Discussion of what might become of this document in the future may be found the postscript (Section \ref{POSTSCRIPT}).

In this document, phrases in coloured text are hyperlinks, like on a web page, which will usually get you to another part of this document (the blue parts, the contents list, page numbers in the Index) but sometimes take you (the red bits) somewhere altogether different (if you happen to be online), e.g.: \href{http://rbjones.com/rbjpub/pp/doc/t046.pdf}{the online copy of this document}.

\section{Introduction}

\pagebreak
\def\section#1{\Section{#1}\label{t046b}\index{t046b}}
\input{t046b.th}
}  %\let

\pagebreak
\subsection{Proof Statistics}

The following table shows the number of times each primitive inference rule was invoked during the proofs of the theorems listed above.

\begin{centering}
\hfill
{\underscoreoff
%\def\statentry#1#2{{#1} && {#2}\\\hline}
%\def\stattotal#1{\\\hline\\&& {#1} \\\hline}
\begin{tabular}{| l | l |}
\hline
\input{t046.stats.tex}
\end{tabular}
}%underscoreoff
\hfill
\end{centering}

The proofs could probably have been done with fewer primitive inference, but there is little incentive to seek shorter proofs.
In these proofs, on average each instruction to the theorem prover results in about 500 primitive inferences, the average number of proof steps at the user interface is less than 3 per theorem (in these theories, in which there are no non-trivial results).
\footnote{The statistic are generated and included in the document automatically and will therefore be correct for the current version of the document, the following comment is not, and might get out of date.}

\pagebreak

\section*{Bibliography}\label{BIBLIOGRAPHY}
\addcontentsline{toc}{section}{Bibliography}

{\def\section*#1{\ignore{#1}}
\raggedright
\bibliographystyle{rbjfmu}
\bibliography{rbj,fmu}
} %\raggedright

{\twocolumn[\section*{Index}\label{INDEX}]
\addcontentsline{toc}{section}{Index}
{\small\printindex}}

\end{document}
