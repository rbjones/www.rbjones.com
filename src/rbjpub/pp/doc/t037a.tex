% $Id: t037a.tex,v 1.4 2010/08/08 15:50:44 rbj Exp $


\documentclass[11pt]{article}
\usepackage{latexsym}
\usepackage{ProofPower}
%\usepackage{amsfonts}

\ftlinepenalty=9999
\usepackage{A4}

% the following two modal operators come from the amsfonts package
\def\PrKI{\Diamond}	%Modify printing for \250
\def\PrKJ{\Box}		%Modify printing for \251

\def\PrJA{\|-}		%Modify printing for  (syntactic entailment)
\def\PrJI{\models}	%Modify printing for ˜ (semantic entailment)

\tabstop=0.4in
\newcommand{\ignore}[1]{}

\def\thyref#1{Appendix \ref{#1}}

%\def\ExpName{\mbox{{\sf exp}}}
%\def\Exp#1{\ExpName(#1)}


\title{Grice on Vacuous Names}
\makeindex
\usepackage[unicode,pdftex]{hyperref}
\hypersetup{pdfauthor={Roger Bishop Jones}, pdffitwindow=false}
\hypersetup{colorlinks=true, urlcolor=red, citecolor=blue, filecolor=blue, linkcolor=blue}
\author{Roger Bishop Jones}
\date{\ }

\begin{document}
\begin{titlepage}
\maketitle
\begin{abstract}
Formal analysis (using Higher Order Logic with ProofPower) and commentary on Grice's system Q (G, $G_{HP}$) first presented in his paper \emph{Vacuous Names}.
\end{abstract}
\vfill

\begin{centering}
{\footnotesize

Created 2010/07/07

Last Change $ $Date: 2010/08/08 15:50:44 $ $

\href{http://www.rbjones.com/rbjpub/pp/doc/t037.pdf}
{http://www.rbjones.com/rbjpub/pp/doc/t037.pdf}

$ $Id: t037a.tex,v 1.4 2010/08/08 15:50:44 rbj Exp $ $

\copyright\ Roger Bishop Jones; Licenced under Gnu LGPL

}%footnotesize
\end{centering}

\thispagestyle{empty}
\end{titlepage}

\newpage
\addtocounter{page}{1}
{\parskip=0pt\tableofcontents}

\section{Prelude}

This document is intended possibly to form a chapter of {\it Analyses of Analysis} \cite{rbjb003}.

My initial purpose in preparing the document is simply to investigate and analyse Grice's ``System Q'' (since renamed ``System G'' by Myro and ``System $G_{HP}$'' by Speranza) using a decent proof tool.

Further discussion of what might become of this document in the future may be found in my postscript (Section \ref{POSTSCRIPT}).

In this document, phrases in coloured text are hyperlinks, like on a web page, which will usually get you to another part of this document (the blue parts, the contents list, page numbers in the Index) but sometimes take you (the red bits) somewhere altogether different (if you happen to be online) like \href{http://rbjones.com/pipermail/hist-analytic_rbjones.com}{the hist-analytic archives}.

For description of the formal languages, methods and tools used in or in producing this document see: \cite{rbjt029}.

\section{Introduction}
