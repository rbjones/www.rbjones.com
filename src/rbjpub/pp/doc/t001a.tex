% $Id: t001a.tex,v 1.1 2009/08/05 06:47:11 rbj Exp $
\documentclass[11pt,a4paper]{article}
\usepackage{latexsym}
\usepackage{ProofPower}
\usepackage{amsfonts}
\ftlinepenalty=9999
\usepackage{A4}

% the following two modal operators come from the amsfonts package
\def\PrKI{\Diamond}	%Modify printing for �
\def\PrKJ{\Box}		%Modify printing for �

\def\PrJA{\|-}		%Modify printing for � (syntactic entailment)
\def\PrJI{\models}	%Modify printing for � (semantic entailment)

\def\ExpName{\mbox{{\sf exp}}}
\def\Exp#1{\ExpName(#1)}

\tabstop=0.4in
\newcommand{\ignore}[1]{}

\title{Mtaphysical Positivism}
\makeindex
\usepackage[unicode,pdftex]{hyperref}
\hypersetup{pdfauthor={Roger Bishop Jones}, pdffitwindow=false}
\hypersetup{colorlinks=true, urlcolor=red, citecolor=blue, filecolor=blue, linkcolor=blue}
\author{Roger Bishop Jones}
\date{\ }

\begin{document}
\begin{titlepage}
\maketitle
\begin{abstract}
Formal models of aspects of Metaphysical Positivism
\end{abstract}
\vfill

\begin{centering}
{\footnotesize

Created 2004/07/15

Last Change $ $Date: 2009/08/05 06:47:11 $ $

\href{http://www.rbjones.com/rbjpub/pp/doc/t001.pdf}
{http://www.rbjones.com/rbjpub/pp/doc/t001.pdf}

$ $Id: t001a.tex,v 1.1 2009/08/05 06:47:11 rbj Exp $ $

\copyright\ Roger Bishop Jones; Licenced under Gnu LGPL

}%footnotesize
\end{centering}

\thispagestyle{empty}
\end{titlepage}

\newpage
\addtocounter{page}{1}
{\parskip=0pt\tableofcontents}

\newpage

\section{Prelude}

This document is intended as the penultimate chapter of the second part of ``An Analytic History of Philosophical Logic'' \cite{rbjb003}.
It is made available as a separate document prior to publication of the whole and serves also as a guide to the other parts of the work in progress.

At this time the work as a whole is inteded ultimately to fall into three parts:

\begin{enumerate}
\item To the Tractatus
\item The Twentieth Century
\item Theory Listings
\end{enumerate}

The sections on Kant and Leibniz will eventually be removed to a different chapter (or discarded).
The material on semantics belongs here, but whether the existing models will survive remains to be seen.

\section{Introduction}



