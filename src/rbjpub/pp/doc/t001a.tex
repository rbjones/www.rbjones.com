% $Id: t001a.tex,v 1.6 2010/08/08 15:50:44 rbj Exp $

\documentclass[11pt,a4paper]{article}
\usepackage{latexsym}
\usepackage{ProofPower}
%\usepackage{amsfonts}
\ftlinepenalty=9999
\usepackage{A4}

% the following two modal operators come from the amsfonts package
\def\PrKI{\Diamond}	%Modify printing for �
\def\PrKJ{\Box}		%Modify printing for �

\def\PrJA{\|-}		%Modify printing for � (syntactic entailment)
\def\PrJI{\models}	%Modify printing for � (semantic entailment)

\def\thyref#1{Appendix \ref{#1}}
%\def\ExpName{\mbox{{\sf exp}}}
%\def\Exp#1{\ExpName(#1)}

\tabstop=0.4in
\newcommand{\ignore}[1]{}

\title{Metaphysical Positivism}
\makeindex
\usepackage[unicode,pdftex]{hyperref}
\hypersetup{pdfauthor={Roger Bishop Jones}, pdffitwindow=false}
\hypersetup{colorlinks=true, urlcolor=red, citecolor=blue, filecolor=blue, linkcolor=blue}
\author{Roger Bishop Jones}
\date{\ }

\begin{document}
\begin{titlepage}
\maketitle
\begin{abstract}
Formal models of aspects of Metaphysical Positivism
\end{abstract}
\vfill

\begin{centering}
{\footnotesize

Created 2004/07/15

Last Change $ $Date: 2010/08/08 15:50:44 $ $

\href{http://www.rbjones.com/rbjpub/pp/doc/t001.pdf}
{http://www.rbjones.com/rbjpub/pp/doc/t001.pdf}

$ $Id: t001a.tex,v 1.6 2010/08/08 15:50:44 rbj Exp $ $

\copyright\ Roger Bishop Jones; Licenced under Gnu LGPL

}%footnotesize
\end{centering}

\thispagestyle{empty}
\end{titlepage}

\newpage
\addtocounter{page}{1}
{\parskip=0pt\tableofcontents}

\newpage

\section{Prelude}

This document is intended as the penultimate chapter of the second part of ``An Analytic History of Philosophical Analysis'' \cite{rbjb003}.
It is made available as a separate document prior to publication of the whole and serves also as a guide to the other parts of the work in progress.

For an overview of the work as a whole see the first chapter of the first part \cite{rbjt029}.

The present content of this document was not written for the purpose of inclusion in the book, but is some material broadly in the scope of the intended chapter which is expected to be replaced in due course by models more carefully engineered for the intended purpose.
 
\section{Introduction}

Metaphysical Positivism is a philosophical system among whose principle objectives is that of articulating a method which suffices to render rigorous, deductive reasoning in philosophy and in all other areas of knowledge in which deductive methods may be applicable.

This chapter is intended to provide formal models which underpin the proposed methods and their applications, and constitutes an application of the method to its own exposition.

A central fulcrum, around which the exposition is expected to turn, is the concept of ``logical truth'' in a broad sense, which we also call ``analyticity'' and ``logical Necessity''.
These concepts take central place in an account of the analytic method, because the results of a sound deduction can be expressed as a logical truth.
(Some prefer to take entailment as the fundamental concept, which idea I will neither dispute nor adopt.)

At present I envisage that this chapter will be concerned primarily with giving precise meanings to certain funcamental concepts through models of certain kinds of language.
