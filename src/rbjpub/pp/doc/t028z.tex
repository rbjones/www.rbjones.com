% $Id: t028z.tex,v 1.6 2010/12/15 07:01:59 rbj Exp $

\section{Postscript}\label{POSTSCRIPT}

These are the kinds of thing which {\it might} happen in future issues of this document:

\begin{itemize}
\item If I get constructive feedback (pointing out errors counts, telling me its a waste of time doesn't, explaining bits of Aristotle I'm getting wrong would be great) then I will probably do something about it.

\item There is a good chance I will get to know something about what Aristotle really said and make improvements arising from this.
In that case I would like to put some hyperlinks into my online hypertext of Aristotle's relevant works, connecting the specific features of the model with the passages in Aristotle which they reflect.

\item I am interested in the methods, and I may spend more time trying to describe them in a way which might be intelligible to philosophers.

\item If this ever becomes a good approximation to Aristotle's position there would then be some philosophical analysis of the metaphysics (by contrast with a purely logical analysis), and somewhere there are points to be made about the kind of analysis which I am aiming for.
\end{itemize}

There are three stages I envisage in the process of getting philosophical insights from this kind of work:

\begin{description}
\item[formal modelling]
I now have a series of models, the last one looks as if it might be useful.

\item[model verification]
The models need checking against Aristotle's writings.

\item[formal analysis]
Further formal work is needed to come to an understanding of these models.

\item[philosophical analysis]
After the models are formulated and verified, and formal analysis has deepened our undertanding of these models, we may then be in a position to obtain some interesting philosophical insights.

\end{description}

\appendix

\vfill

\section{Glossary}




\section{Theory Listings}
{
\let\Section\subsection
\let\Subsection\subsubsection
\def\subsection#1{\Subsection*{#1}}
\def\section#1{\Section{#1}\label{ariscat}}
\include{ariscat.th}
\def\section#1{\Section{#1}\label{syllog1}}
\include{syllog1.th}
\def\section#1{\Section{#1}\label{syllog2}}
\include{syllog2.th}
\def\section#1{\Section{#1}\label{syllog3}}
\include{syllog3.th}
\def\section#1{\Section{#1}\label{modsyllog}}
\include{modsyllog.th}
\def\section#1{\Section{#1}\label{syllmetap}}
\include{syllmetap.th}
\def\section#1{\Section{#1}\label{gccon}}
\include{gccon.th}
}  %\let

\pagebreak

\section*{Bibliography}\label{BIBLIOGRAPHY}
\addcontentsline{toc}{section}{Bibliography}

{\def\section*#1{\ignore{#1}}
\raggedright
\bibliographystyle{rbjfmu}
\bibliography{rbj,fmu}
} %\raggedright

{\twocolumn[\section*{Index}\label{INDEX}]
\addcontentsline{toc}{section}{Index}
{\small\printindex}}

\end{document}
