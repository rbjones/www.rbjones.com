% $Id: t048a.tex,v 1.2 2012/04/25 18:44:36 rbj Exp $

\documentclass[11pt]{article}
\usepackage{latexsym}
\usepackage{rbj}

\ftlinepenalty=9999
\usepackage{A4}

% the following two modal operators come from the amsfonts package
\def\PrKI{\Diamond}	%Modify printing for \250
\def\PrKJ{\Box}		%Modify printing for \251

\def\PrJA{\|-}		%Modify printing for  (syntactic entailment)
\def\PrJI{\models}	%Modify printing for ˜ (semantic entailment)
\def\PrJO{\prec}	%Modify printing for \236 (semantic entailment)

\def\PrIO{\MMM{\notin}}	%Modify printing for \216 (not member of)

\tabstop=0.4in
\newcommand{\ignore}[1]{}

\def\thyref#1{Appendix \ref{#1}}

%\def\ExpName{\mbox{{\sf exp}}}
%\def\Exp#1{\ExpName(#1)}

\title{Iterative Foundational Ontologies}
\makeindex
\usepackage[unicode]{hyperref}
\hypersetup{pdfauthor={Roger Bishop Jones}, pdffitwindow=false, pdfkeywords=RogerBishopJones}
\hypersetup{colorlinks=true, urlcolor=red, citecolor=blue, filecolor=blue, linkcolor=blue}
\author{Roger Bishop Jones}
\date{\ }

\begin{document}
\begin{titlepage}
\maketitle
\begin{abstract}
A broad discussion of ontologies for mathematics and abstract semantics.
\end{abstract}
\vfill

\begin{centering}
{\footnotesize

Created 2011/02/24

\input{t048i.tex}

\href{http://www.rbjones.com/rbjpub/pp/doc/t048.pdf}
{http://www.rbjones.com/rbjpub/pp/doc/t048.pdf}

\copyright\ Roger Bishop Jones; Licenced under Gnu LGPL

}%footnotesize
\end{centering}

\thispagestyle{empty}
\end{titlepage}

\newpage
\addtocounter{page}{1}
{\parskip=0pt\tableofcontents}

\section{Prelude}

This is one of a series of documents in which philosophically motivated technical issues are explored making use of an interactive proof tool for a higher order logic.


This document began as an exploration of a draft of \emph{Pluralities and Sets} by {\O}ystein Linnebo \cite{linneboPS}.
Once the bare bones were in place I had to decide where to go from there, and I recalled a paper by Thomas Forster in which he extends the iterative conception of set to embrace set theories with a universal set \cite{forsterTICS}, and decided that iterative ontology would provide a theme under which a broad exposure of my own foundational programme might be presented.

Discussion of what might become of this document in the future may be found the postscript (Section \ref{POSTSCRIPT}).

In this document, phrases in coloured text are hyperlinks, like on a web page, which will usually get you to another part of this document (the blue parts, the contents list, page numbers in the Index) but sometimes take you (the red bits) somewhere altogether different (if you happen to be online), e.g.: \href{http://rbjones.com/rbjpub/pp/doc/t048.pdf}{the online copy of this document}.

\section{Changes}

\subsection{Recent Changes}

This document began as an extension to \emph{Pluralities and Sets}\cite{rbjt046}, but I then realised that it was a bad idea to combine these two very different approaches to abstract ontology in the same document and so made it into a new document.

\subsection{Changes Under Consideration}


\subsection{Issues}


See also Section \ref{POSTSCRIPT}.

\section{Introduction}
