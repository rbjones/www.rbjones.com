% $Id: t048z.tex,v 1.2 2012/04/25 18:44:36 rbj Exp $

\section{Postscript}\label{POSTSCRIPT}

This document has not yet really got off the ground, so I don't have much to say about it as yet.

\pagebreak

\appendix

\section{The Meta-Language}

The formal analysis is conducted in a single logical system, a Higher Order Logic based on Church's formulation of the Simple Theory of Types\cite{church40}.
The formal treatment is prepared with the assistance of an interactive theorem proving tool.
This tool assists by syntax checking and type-checking specifications, by confirming that these specifications are conservative over the initial logical system, by facilitating the construction of detailed formal proofs and mechanically checking their correctness.
The tool also prepares listings of the resulting theories, which may be found in appendices \ref{t046a} and \ref{t046b} and facilitates the preparation of documents including the formal materials.

There is some additional complexity in undertaking strictly formal work in this manner, which is not entirely eliminated by the use of software support.
Feasibility depends on careful choice of methods (and problems) to keep complexity within bounds.
In the kind of exploratory investigation at hand, one simplification is to avoid reasoning about syntax.
This may be done by constructing interpretations of the target systems and reasoning about these interpretations in HOL.
In the resulting theorems the syntax of HOL (which is in some degree extendable) is used to express claims which correspond to the rules, axioms and theorems of the logical system or systems under investigation.
When proven they give good grounds for belief in the soundness of the logic under consideration, even though we have avoided formal treatment of its syntax.

This approach to reasoning in some logic of interest using a tool supporting a different logical system, is sometimes called \emph{shallow embedding} (by contrast with \emph{deep embedding} in which both syntax and semantics and the relationship between them are formally treated, supporting full formalisation of the metatheory).
An extended discussion of these methods is not within our present scope, but I will try to include a certain amount of further explanation as the document proceeds in the hope of making the technical detail as intelligible as practicable.

\subsection{Dependencies}

For a complete understanding of the details of the formal materials in this document it would strictly be necessary to refer to the definition of the language in which the specifications are written (\ProductHOL) and to the listings of the various theories in the context of which these theories have been developed.
I hope that the material will be intelligible to a reasonable degree without studying all this material, many readers will already be familiar with Church's formulation of the Simple Theory of Types \cite{church40} and will be familiar with the meanings of the usual logical connectives in that context.

For the full detail the following documents may be consulted.

\begin{enumerate}
\item Church's formulation of STT: \cite{church40}.
\item The {\ProductHOL} language: informal description \cite{ds/fmu/ied/usr005}
\item The {\ProductHOL} language: formal specification \cite{ds/fmu/ied/spc001}
\item {\ProductHOL} theory listings: \cite{lemma1/hol/usr029} or \href{http://rbjones.com/rbjpub/pp/pptheories.html}{in HTML at RBJones.com}%
\footnote{http://rbjones.com/rbjpub/pp/pptheories.html}.
\item Other theories at RBJones.com: rbjmisc \cite{rbjt006}, t045 \cite{rbjt045}.
\item Complete documentation for {\Product} can be obtained from \href{http://www.lemma-one.com/ProofPower/doc/doc.html}{the {\Product} web pages}%
\footnote{http://www.lemma-one.com/ProofPower/doc/doc.html}.
\end{enumerate}

\section{Theory Listings}

{
\let\Section\subsection
\let\Subsection\subsubsection
\def\subsection#1{\Subsection*{#1}}

\def\section#1{\Section{#1}\label{t046a}\index{t046a}}
% $Id: t048a.tex,v 1.1 2011/05/16 21:40:17 rbj Exp $

\documentclass[11pt]{article}
\usepackage{latexsym}
\usepackage{rbj}

\ftlinepenalty=9999
\usepackage{A4}

% the following two modal operators come from the amsfonts package
\def\PrKI{\Diamond}	%Modify printing for \250
\def\PrKJ{\Box}		%Modify printing for \251

\def\PrJA{\|-}		%Modify printing for  (syntactic entailment)
\def\PrJI{\models}	%Modify printing for ˜ (semantic entailment)
\def\PrJO{\prec}	%Modify printing for \236 (semantic entailment)

\def\PrIO{\MMM{\notin}}	%Modify printing for \216 (not member of)

\tabstop=0.4in
\newcommand{\ignore}[1]{}

\def\thyref#1{Appendix \ref{#1}}

%\def\ExpName{\mbox{{\sf exp}}}
%\def\Exp#1{\ExpName(#1)}

\title{Iterative Foundational Ontologies}
\makeindex
\usepackage[unicode,pdftex]{hyperref}
\hypersetup{pdfauthor={Roger Bishop Jones}, pdffitwindow=false, pdfkeywords=RogerBishopJones}
\hypersetup{colorlinks=true, urlcolor=red, citecolor=blue, filecolor=blue, linkcolor=blue}
\author{Roger Bishop Jones}
\date{\ }

\begin{document}
\begin{titlepage}
\maketitle
\begin{abstract}
A broad discussion of ontologies for mathematics and abstract semantics.
\end{abstract}
\vfill

\begin{centering}
{\footnotesize

Created 2011/02/24

\input{t046i.tex}

\href{http://www.rbjones.com/rbjpub/pp/doc/t048.pdf}
{http://www.rbjones.com/rbjpub/pp/doc/t048.pdf}

\copyright\ Roger Bishop Jones; Licenced under Gnu LGPL

}%footnotesize
\end{centering}

\thispagestyle{empty}
\end{titlepage}

\newpage
\addtocounter{page}{1}
{\parskip=0pt\tableofcontents}

\section{Prelude}

This is one of a series of documents in which philosophically motivated technical issues are explored making use of an interactive proof tool for a higher order logic.


This document began as an exploration of a draft of \emph{Pluralities and Sets} by {\O}ystein Linnebo \cite{linneboPS}.
Once the bare bones were in place I had to decide where to go from there, and I recalled a paper by Thomas Forster in which he extends the iterative conception of set to embrace set theories with a universal set \cite{forsterTICS}, and decided that iterative ontology would provide a theme under which a broad exposure of my own foundational programme might be presented.

Discussion of what might become of this document in the future may be found the postscript (Section \ref{POSTSCRIPT}).

In this document, phrases in coloured text are hyperlinks, like on a web page, which will usually get you to another part of this document (the blue parts, the contents list, page numbers in the Index) but sometimes take you (the red bits) somewhere altogether different (if you happen to be online), e.g.: \href{http://rbjones.com/rbjpub/pp/doc/t046.pdf}{the online copy of this document}.

\section{Changes}

\subsection{Recent Changes}

This document began as an extension to \emph{Pluralities and Sets}\cite{rbjt046}, but I then realised that it was a bad idea to combine these two very different approaches to abstract ontology in the same document and so made it into a new document.

\subsection{Changes Under Consideration}


\subsection{Issues}


See also Section \ref{POSTSCRIPT}.

\section{Introduction}


%\pagebreak
%\def\section#1{\Section{#1}\label{t046b}\index{t046b}}
%\input{t048b.th}
}  %\let

\pagebreak
\subsection{Proof Statistics}

The following table shows the number of times each primitive inference rule was invoked during the proofs of the theorems listed above.

\begin{centering}
\hfill
{\underscoreoff
%\def\statentry#1#2{{#1} && {#2}\\\hline}
%\def\stattotal#1{\\\hline\\&& {#1} \\\hline}
\begin{tabular}{| l | l |}
\hline
\input{t048.stats.tex}
\end{tabular}
}%underscoreoff
\hfill
\end{centering}

The proofs could probably have been done with fewer primitive inference, but there is little incentive to seek shorter proofs.
In these proofs, on average each instruction to the theorem prover results in about 500 primitive inferences, the average number of proof steps at the user interface is less than 3 per theorem (in these theories, in which there are no non-trivial results).
\footnote{The statistic are generated and included in the document automatically and will therefore be correct for the current version of the document, the following comment is not, and might get out of date.}

\pagebreak

\section*{Bibliography}\label{BIBLIOGRAPHY}
\addcontentsline{toc}{section}{Bibliography}

{\def\section*#1{\ignore{#1}}
\raggedright
\bibliographystyle{rbjfmu}
\bibliography{rbj,fmu}
} %\raggedright


{\twocolumn[\section*{Index}\label{INDEX}]
\addcontentsline{toc}{section}{Index}
{\small\printindex}}

\end{document}
