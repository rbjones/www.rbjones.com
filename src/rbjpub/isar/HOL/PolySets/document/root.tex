% pdfname{PolySets}
\documentclass[11pt,a4paper]{article}
\usepackage{isabelle,isabellesym}

\newcommand{\ignore}[1]{}

% this should be the last package used
\usepackage{pdfsetup}

% urls in roman style, theory text in math-similar italics
\urlstyle{rm}
\isabellestyle{it}

\begin{document}

\title{The Theory of PolySets and its Applications}
\author{Roger Bishop Jones}
\date{$ $Date: 2006/11/17 14:57:38 $ $}
\maketitle

\begin{abstract}
A model for a set theory with a universal set and its use in the construction of other interesting and possibly useful foundational ontologies.
\end{abstract}

\vfill
\begin{centering}
{\footnotesize
\href{http://www.rbjones.com/rbjpub/isar/HOL/PolySets.pdf}{http://www.rbjones.com/rbjpub/isar/HOL/PolySets.pdf}\\
\href{http://www.rbjones.com/rbjpub/isar/HOL/PolySets.tgz}{http://www.rbjones.com/rbjpub/isar/HOL/PolySets.tgz}\\
\ \\
$ $Id: root.tex,v 1.1 2006/11/17 14:57:38 rbj01 Exp $ $\\
}%footnotesize
\end{centering}

\newpage

\tableofcontents

\parindent 0pt\parskip 0.5ex
\newpage
\section{Introduction}

The purpose of thie document is to construct a non-well-founded domain of sets with a universal set.
It is hoped that the particular construction employed will yield a collection of sets suitable for use further constructions yielding domains with elements with slightly more common structure than sets (for example functions or functors).
These domains will then be used as the underlying ontology for a foundation system for formalised mathematics, and more generally for the formal demonstration of analytic truths.

The special feature of the proposed ontologies by contrast with other already established systems is their anticipated suitability for foundational systems intended for formalisation ``in the large'' i.e. which are designed for modular specification and proof maximising re-usability.
This is the terminology of software engineering, and is used because of the relevant similarities which can be seen between large scale formalisation of mathematics and other analytic domains and software development.

These non-well-founded foundation systems are not intended to make a difference to the way in which ordinary mathematics is done.
By `ordinary' mathematics, I mean arithmetic, analysis, geometry and any other mathematical study in which the domain of discourse is some definite concrete mathematical structure (often a system of numbers).
It is specifically not intended to provide an alternative treatment of cardinal arithmetic, which would continue to be undertaken via the Von Neumann ordinals.

The areas it would impact more substantially are those parts of mathematics which are in some way generic, where a theory is developed without any definite conception of the objects, expecting the theory to be applicable in many different domains.
The most obvious examples of this kind of mathematics are abstract algebra, universal algebra and category theory.

\subsection{Notation and Terminology}

It is in the nature of the subject matter that many different variants of familiar mathematical concepts are used.
In particular, there are a variety of different kinds of set membership, and also various kinds of predication and type assignment.

Not only do we have a confusing profusion of similar but not identical concpts, but we are constrained by the fact that this work is mathematics formalised for processing by computer, and must therefore fit in with the constraints imposed by the software (in this case Isabelle).

Overloading the membership sign, would I fear lead to an unintelligible document, but chosing another symbol each time a new relationship is introduced would not be much better.
Subscripts and superscripts are not much needed for their cusomary purposes and are therefore used exclusively as decorations for symbols which distinguish the variants employed.

I have tried to make their use as systematic as possible, a different letter is associated with each domain and is used as a subscript on operators over that domain.

The letters are as follows:

\begin{centering}
\begin{tabular}{| l | l | l | l |}
\hline
letter & type & domain & description \\
\hline
g & Set & & a domain of pure well-founded sets \\
r & Set & polysets & intermediate representatives for the polysets \\
q & Set set & & final representatives for polysets \\
p & pset & & the polysets themselves (as a new type) \\
\hline
\end{tabular}
\end{centering}

\newpage
\input{Sets.tex}
\newpage
\input{PolySetsC.tex}
\newpage
\input{PolySets}

% optional bibliography
%\bibliographystyle{abbrv}
%\bibliography{root}

\end{document}

%%% Local Variables:
%%% mode: latex
%%% TeX-master: t
%%% End:
