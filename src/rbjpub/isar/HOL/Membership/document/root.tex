% pdfname{Membership}
\documentclass[11pt,a4paper]{article}
\usepackage{isabelle,isabellesym}

\newcommand{\ignore}[1]{}

% further packages required for unusual symbols (see also
% isabellesym.sty), use only when needed

%\usepackage{amssymb}
  %for \<leadsto>, \<box>, \<diamond>, \<sqsupset>, \<mho>, \<Join>,
  %\<lhd>, \<lesssim>, \<greatersim>, \<lessapprox>, \<greaterapprox>,
  %\<triangleq>, \<yen>, \<lozenge>

%\usepackage[greek,english]{babel}
  %option greek for \<euro>
  %option english (default language) for \<guillemotleft>, \<guillemotright>

%\usepackage[latin1]{inputenc}
  %for \<onesuperior>, \<onequarter>, \<twosuperior>, \<onehalf>,
  %\<threesuperior>, \<threequarters>, \<degree>

%\usepackage[only,bigsqcap]{stmaryrd}
  %for \<Sqinter>

%\usepackage{eufrak}
  %for \<AA> ... \<ZZ>, \<aa> ... \<zz> (also included in amssymb)

%\usepackage{textcomp}
  %for \<cent>, \<currency>

% this should be the last package used
\usepackage{pdfsetup}

% urls in roman style, theory text in math-similar italics
\urlstyle{rm}
\isabellestyle{it}

\begin{document}

\title{Membership Structures}
\author{Roger Bishop Jones}
\date{$ $Date: 2006/11/13 07:12:37 $ $}
\maketitle

\begin{abstract}
A way of doing set theory in HOL without using axioms.
\end{abstract}

\vfill
\begin{centering}
{\footnotesize
\href{http://www.rbjones.com/rbjpub/isar/HOL/Membership.pdf}{http://www.rbjones.com/rbjpub/isar/HOL/Membership.pdf}\\
\href{http://www.rbjones.com/rbjpub/isar/HOL/Membership.tgz}{http://www.rbjones.com/rbjpub/isar/HOL/Membership.tgz}\\
\ \\
$ $Id: root.tex,v 1.3 2006/11/13 07:12:37 rbj01 Exp $ $\\
}%footnotesize
\end{centering}

\newpage

\tableofcontents

\parindent 0pt\parskip 0.5ex

\section{Introduction}

This document is concerned with what is usually known as ``set theory''.

The approach adopted is intended to be distinguished from usual treatmente of set theory in the following ways:

\begin{itemize}

\item The subject matter is taken to be {\it membership structures} rather than sets.
A membership structure is, of course, some collection together with a binary relation over that collection, i.e. an interpretation of the first order language of set theory.

\item No specific set of axioms is adopted.
Instead the interrelationships between various properties of membership structures are studied.

\item The distinction is drawn between those properties of membership structures which are expressible as sentences in the language of first order set theory (``first-order'' for short), and those which are not (``higher order'').

\item Particular interest is shown in certain properties which are not first order and in the kinds of membership structure posessing these properties.

\item The informal language of set theory is made more precise by noting that certain properties (notably cardinality) normally thought of as properties of sets, are perhaps better seen as relationships between membership structures and their elements (in what may be called their internal use) unless an external criterion of equipollence is applied (yielding an ``external'' notion of cardinality).
Some theorems of set theory are explicated using this distinction, for example the result that V=L is incompatible with the existence of measurable cardinals.
\end{itemize}

The theory is substantially but not exclusively concerned with well-founded membership structures, in which aspect its motivation is largely philosophical.
In practical terms, the main application of the well-founded structures is to the construction of non-well-founded structures which are to be used elsewhere.


% generated text of all theories
\input{session}

% optional bibliography
%\bibliographystyle{abbrv}
%\bibliography{root}

\end{document}

%%% Local Variables:
%%% mode: latex
%%% TeX-master: t
%%% End:
