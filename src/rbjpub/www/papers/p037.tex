% $Id: p037.tex $
% bibref{rbjp037} pdfname{p037}
\documentclass[10pt,titlepage]{book}
\usepackage{makeidx}
\newcommand{\ignore}[1]{}
\usepackage{graphicx}
\usepackage[unicode]{hyperref}
\pagestyle{plain}
\usepackage[paperwidth=5.25in,paperheight=8in,hmargin={0.75in,0.5in},vmargin={0.5in,0.5in},includehead,includefoot]{geometry}
\hypersetup{pdfauthor={Roger Bishop Jones}}
\hypersetup{pdftitle={An Epistemological Synthesis}}
\hypersetup{colorlinks=true, urlcolor=red, citecolor=blue, filecolor=blue, linkcolor=blue}
%\usepackage{html}
\usepackage{paralist}
\usepackage{relsize}
\usepackage{verbatim}
\usepackage{enumerate}
\usepackage{longtable}
\usepackage{url}
\newcommand{\hreg}[2]{\href{#1}{#2}\footnote{\url{#1}}}
\makeindex

\title{\bf An Epistemological Synthesis}
\author{Roger~Bishop~Jones}
\date{\small 2023-12-01}


\begin{document}

%\begin{abstract}
% 
% 
% 
%\end{abstract}
                             
\begin{titlepage}
\maketitle

%\vfill

%\begin{centering}

%{\footnotesize
%copyright\ Roger~Bishop~Jones;
%}%footnotesize

%\end{centering}

\end{titlepage}

\ \

\ignore{
\begin{centering}
{}
\end{centering}
}%ignore

\setcounter{tocdepth}{2}
{\parskip-0pt\tableofcontents}

%\listoffigures

\hfill
\ 
\pagebreak

%\addcontentsline{toc}{section}{Index}\label{index}
%{\twocolumn[]
%{\small\printindex}}

%\vfill

%\tiny{
%Started 2023/12/01


%\href{http://www.rbjones.com/rbjpub/www/papers/p032.pdf}{http://www.rbjones.com/rbjpub/www/papers/p037.pdf}

%}%tiny

\section*{Preface}

\chapter{Introduction}

I have styled this monograph as \emph{epistemology} intending to bend and stretch that concept beyond breaking point.
The monograph falls into two parts, the first historical and the second futuristic, exemplifying two distinct kinds of epistemology which I have labelled \emph{natural} and \emph{synthetic} respectively.

The stretch to which our context extends is 5 billion years, which we may think of as 4.5 billion, from the earliest point at which the pre-biotic evolution of life on Earth might be considered to have begun, to some time in the future when the intelligent (but not necessarily organic) progeny of life on earth has proliferated 10,000 light years through the Milky way.

The breadth of scope is chosen to stretch the concepts involved.
The whole is story is told through an evolutionary lens.
It is a story of the evolution of knowledge in which the epistemological synthesis is a projection into the future on the basis of an understanding of the past presented as natural epistemology.
In this story, evolution itself is viewed as an evolving continuous accretion of knowledge, and the concept of knowledge is stretched to encompass the fruits of evolution in every stage.

In interpreting the pre-biotic evolution leading to first life (abiogenesis) as gathering knowledge, the Gettier conception of knowledge as \emph{justified true belief} is decisively abandoned, along with any other anthropocentric conception of knowledge (including the conception of knowledge as any kind of \emph{belief}, and hence Quine's \emph{epistemology naturalised}).
We may also entertain the possibility that knowledge, in some future where the majority of our intelligent progeny (not necessarily biological) are too far removed (physically) to even communicate with earth, will be advanced and applied in ways which have lost most if not all trace of human origin or influence.

Bard rewrite:
\begin{quote}

"I've titled this monograph as epistemology, but I'm going to stretch and challenge that concept in ways you might not expect. The book is divided into two parts: the first historical and the second futuristic, each representing two different kinds of epistemology that I've called natural and synthetic.

This journey takes us 5 billion years, from the earliest point of prebiotic life on Earth to a future where intelligent (but not necessarily organic) life from Earth has spread 10,000 light-years across the Milky Way. While some of what I'll say here is purely speculative, the core of my synthesis is based on the understanding of a relatively brief period in the past, just a few thousand years around the present day.

This broad scope is intentional, designed to push the boundaries of our understanding. The story is told through an evolutionary lens, charting the evolution of knowledge itself. My synthesis is a projection into the future based on our understanding of the past, a natural epistemology. Evolution itself is viewed as an ongoing accumulation of knowledge, with our concept of knowledge encompassing all the fruits of evolution, from the earliest stages to the most advanced."

\end{quote}

******

In this introductory discussion I aim to clarify the nature and purpose of these two approaches to epistemology.

One reason for the bend and stretch is that my rationale for \emph{inventing} an episteme, which is what my \emph{synthetic epistemology} tries to do, is that knowledge, epistemes and epistemology have all been evolving for a long time.
We should therefore expect their futures to differ from their history, at the same time as being shaped by that history and the exigencies of the present.

Looking at the past evolution of these concepts and the phenomena they address must be an empirical study, and is therefore characterised here as \emph{natural epistemology}.
In making a philosophical contribution to a scientific subject matter careful attention to terminology is appropriate, but that attention should be oriented toward establishing those concepts which serve best to describe the relevant phenomena, not to the analysis of apparently relevant contemporary ordinary discourse.
The merits of Gettier analysis of knowledge, as \emph{justified true belief} are cast in a different light when thinking in evolutionary terms.

In those evolutionary terms it is more productive to consider the psychological phenomenon of human knowledge as a part of an evolutionary development which goes back far early than humanity, and embraces the role of memory in in animal species, if not also the record of organic structures maintained in the genomes of biological species, and even the existence of DNA prior to abiogenesis, the genesis of life from the inanimate.

\section{Varieties of Epistemology}

Epistemology, as a branch of philosophy, is about two and a half millenia old, and has been predominantly anthropocentric.
Some epistemological concerns are centred around the analysis of relevant concepts, a prominent example being debate around the Gettier analysis of knowledge as \emph{justified true belief}.
This may be thought doubly anthropocentric, not only concerned with the meaning of ordinary language, but also in regarding knowledge as a kind of psychological phenomenon.

The epistemological synthesis I undertake here is intended to be suitable for a world in which artefacts, possibly mainly or wholly inorganic, dominate the gathering storage and exploitation of knowledge.
I therefore seek a conception of knowledge, and an appropriate epistemology which can make sense of that kind of future.

In seeking such an epistemology I acknowledge that there is evolution in the kinds of knowledge which exist, and in the epistemes which govern what is considered knowledge, in epistemology and even in evolution itself.
In anticipating or proposing a future episteme, I engage with the evolution and recognise its character as a step forward in an evolutionary process.

\part{Natural Epistemology}

The principal aims of this part are:

\begin{itemize}
\item To sketch the evolution of knowledge and proto-knowledge to provide a foundation for the proposed epistemological synthesis.
\item To identify those features in the history which are most important for understanding and underpinning the synthesis.
\end{itemize}

The earlier parts of the story, 

\chapter{Preliminaries}

\section{The Evolution of Evolution}

\begin{itemize}
\item Pre-biotic evolution
\item Asexual biology
\item Sexual selection
\item Pre-lingual culture
\item Oral culture
\item Cultural evolution and cultural selection
\item Written culture
\end{itemize}

\chapter{Pre-historic Epistemology}

\chapter{}

\part{Synthetic Epistemology}


\appendix


\chapter{Introduction}

The purpose of this monograph is to put forward an \emph{episteme}, which is a constructed to be suitable for adoption in a particular future which I envisage.
I have adopted the word `episteme' with a broadly similar sense to that of  Michel Foucault \cite{foucault1966order}.
  Foucault's work is set in a contemporary context and addresses the kind of social structures which in different times and societies determine what is accepted as knowledge.
  This monograph is concerned with the evolution of knowledge over a very broad timescale, let us say, for about 5 billion years from about four billion years ago, for the vast bulk of which period there were no human societies.
  The purpose of this scope is to arrive at a conception of knowledge which can be extended forward to a time in which, notwithstanding the possible continued existence of human societies, the majority of scientific and technological knowledge is gathered and exploited by inorganic intelligent self-propagating and proliferating intelligent systems.

  The history is presented as an evolutionary process, but the concept of evolution as articulated by Charles Darwin \cite{darwin-oos} too narrowly scoped and will therefore be generalised for our purposes, as it has been often very informally, by many others.
  Ultimately, as our story approaches the present day, it will give way to a prognostic synthesis.
  Our focus is on the evolution of knowledge, or those \emph{episteme}'s which characterise knowledge, but it does not attempt to be exhaustive, only those developments which inform the synthesis will be presented.

\chapter{Introduction}

The purpose of this book is to present an \emph{epistemological synthesis}.
The synthesis is a description of a systematic way in which certain kinds of knowledge can be represented and organised together with an account of the case for adopting that kind of representation.

The proposal rests upon philosophical advances which can reasonably be said to have begun with the invention by the German philosopher Gottlob Frege of his ``Begriffsschrift'' or \emph{concept notation} \cite{frege79,heijenoort67}, which was the most profound advance in logic and mathematics since the philosophy of Aristotle two and a half millenia earlier.

Frege's concept notation was motivated in part by his desire to refute the views of the Emanual Kant about the epistemological status of the truths of Mathematics, but also can be seen as natural conclusion to a progression of developments in the foundations of mathematics which had been under way during the 19th Century.
Though a part of the progression of fundamental research in mathematics, with the begriffsschrift Frege stepped into a more general setting, and offered a conception of language and logic which transcended subject boundaries and aspired to universality.

\section{Varieties of Epistemology}

Epistemology is ``the theory of knowledge''.
It has traditionally been a branch of philosophy and has a foundational status as addressing issues which may be thought of as preconditions of any systematic search for knowledge, and hence as in some sense as prior to science.
This is nowhere more conspicuous than in the philosophy of Ren\'{e} Descartes who in his ``Meditations on First Philosophy'' \cite{descartes2013meditations} doubted all but that doubting made his existence indubitable (the \emph{cogito}).

It is my purpose here to propose ways in which knowledge can be characterised, represented and exploited in world which may soon be upon us where intelligence is no longer a purely human prerogative, but is manifest in technological artefacts.
These ideas represent a continuation of certain lines of advance in what might reasonably be termed, the evolution of knowledge and epistemology, and their exposition demands some account of that evolutionary history.
Notwithstanding the billions of years of history which has played a  role in the genesis of the ideas, some of which is essential to a clear account of their nature, the core ideas are foundational in character, and exhibit a kind of ``First Philosophy'' which makes no pretence of emerging \emph{ex nihilo}.
Later I will present some illustrations of how foundational research works in disciplines other than philosophy, and how work on foundations can progressively improve confidence and precision without aspiring to the absolutes which philosophers may consider essential to the enterprise. 

The principle ideas which inspire this work constitute what I am calling \emph{synthetic epistemology} because they are creations of the human intellect rather than discoveries (though many related discoveries are essential to understanding their scope and significance).
In this respect synthetic epistemology is unusual; philosophers rarely conceive themselves as making things up.

Though the influence of Descartes on Western Philosophy has been profound, the idea of epistemology as part of a ``First Philosophy'' which is prerequisite for rigorous science has been challenged, notably by W.V.O.Quine, who adopted the term ``Epistemology Naturalised'' for a conception of epistemology ``continuous with empirical science'', and rejected the aspirations of ``First Philosophy'' as exemplified in the philosophy of Descartes.

Neither of the two varieties of epistemology discussed here corresponds closely to Descartes' first philosophy or to Quine's epistemology naturalised, though there are some aspects in common.

Synthetic epistemology is philosophical, but is adjacent to the most abstract aspects of some approaches to the design of intelligent artefacts, addressing full on the problem of knowledge representation.
It is presented as a further stage in the evolution of knowledge and epistemology which I construe as a process which extends over past four billion years.
For that reason, the proposal, its presentation and rationale, depend heavily upon the prior account I offer of that preceding evolutionary history.
Despite this close connection with the engineering of artificial intelligence, it is is not offered as contributing to that end, but rather as a way of managing explicit knowledge which is made feasible and productive only by the capabilities which we expect to see in intelligent machinery.

That history is presented as ``natural epistemology'', which is not offered as philosophy.
It is I hope reasonably well grounded in a lay understanding of relevant science, until very late in the story when my professional knowledge of aspects of the relevant developments comes into play.  
It's similarity with Quine's \emph{epistemology naturalized} lies primarily in its empirical nature.

\chapter{Relevant Contemporary Research}

Though in Mathematics and Philosophy interest in formalisation has been primarily meta-theoretic, in the former case with a mathematical flavour and in the latter case often proving models for aspects of natural languages, in Computer Science there has been continuing interest in using formal mathematics for the design and verification of computer systems, especially where there is a particular need for correctness or critical properties to be established.

In this chapter I present a lightweight sketch of the scope of related research, and some commentary on the principal alternatives to the logical systems which form the basis for the epistemological synthesis to follow.

\section{Some Conferences}

\subsection{Logic in Computer Science}

\subsection{Conferences on Intelligent Computer Mathematics}

\addcontentsline{toc}{section}{Bibliography}
\bibliographystyle{rbjfmu}
\bibliography{rbj2}


\addcontentsline{toc}{section}{Index}\label{index}
{\twocolumn[]
{\small\printindex}}

%\vfill

%\tiny{
%Started 2023-08-01


%\href{http://www.rbjones.com/rbjpub/www/papers/p037.pdf}{http://www.rbjones.com/rbjpub/www/papers/p037.pdf}

%}%tiny

\end{document}

% LocalWords:
