% $Id: p037.tex $
% bibref{rbjp037} pdfname{p037}
\documentclass[10pt,titlepage]{article}
\usepackage{makeidx}
\newcommand{\ignore}[1]{}
\usepackage{graphicx}
\usepackage[unicode]{hyperref}
\pagestyle{plain}
\usepackage[paperwidth=5.25in,paperheight=8in,hmargin={0.75in,0.5in},vmargin={0.5in,0.5in},includehead,includefoot]{geometry}
\hypersetup{pdfauthor={Roger Bishop Jones}}
\hypersetup{pdftitle={Evolution and Logic}}
\hypersetup{colorlinks=true, urlcolor=red, citecolor=blue, filecolor=blue, linkcolor=blue}
%\usepackage{html}
\usepackage{paralist}
\usepackage{relsize}
\usepackage{verbatim}
\usepackage{enumerate}
\usepackage{longtable}
\usepackage{url}
\newcommand{\hreg}[2]{\href{#1}{#2}\footnote{\url{#1}}}
\makeindex

\title{\bf\LARGE Evolution\\ and\\ Logic}
\author{Roger~Bishop~Jones}
\date{\small 2023-08-01}


\begin{document}

%\begin{abstract}
%
% Evolution is the mechanism which turns chaos into order,
% not so much by design as by the accident that chance will eventually create complex structures which thrive.
% It has many forms, some very different to what we call Darwinian evolution, and by consideration of how those forms of evolution have themselves evolved we can speculate about the future of evolution and of intelligence.
%My own speculations on this topic say something about epistemology and logic.
%\end{abstract}
                               
\begin{titlepage}
\maketitle

%\vfill

%\begin{centering}

%{\footnotesize
%copyright\ Roger~Bishop~Jones;
%}%footnotesize

%\end{centering}

\end{titlepage}

\ \

\ignore{
\begin{centering}
{}
\end{centering}
}%ignore

\setcounter{tocdepth}{2}
{\parskip-0pt\tableofcontents}

%\listoffigures

\pagebreak

\section*{Preface}
\phantomsection

\addcontentsline{toc}{section}{Preface}

This preface is really immaterial, but there might be some point in saying \emph{why} its immaterial.
Here's Bertrand Russell more than a century ago on a theme that remains current.

\begin{quote}
  \emph{That Man is the product of causes which had no prevision of the end they were achieving; that his origin, his growth, his hopes and fears, his loves and his beliefs, are but the outcome of accidental collocations of atoms; that no fire, no heroism, no intensity of thought and feeling, can preserve an individual life beyond the grave; that all the labours of the ages, all the devotion, all the inspiration, all the noonday brightness of human genius, are destined to extinction in the vast death of the solar system, and that the whole temple of Man's achievement must inevitably be buried beneath the débris of a universe in ruins — all these things, if not quite beyond dispute, are yet so nearly certain, that no philosophy which rejects them can hope to stand.}

Bertrand Russell, Mysticism and Logic \cite{russell17}
\end{quote}

Much has changed, in the world and in philosophy, since Russell penned those words, but majority opinion is probably still with Russell's sentiment (if not his way of putting it), and its not even controversial, as he says (in other words) resistance is futile.

I was reminded of Russell's words just as I resolved on this essay, and the contrast between that point of view and the perspective intended for the essay seemed a good way to highlight that perspective.

Russell talks of the death of the solar system, and of a universe in ruins.
The former would be a consequence of the Sun progressing through the observed life cycle of similar stars, which I don't doubt.
It will not be happening soon, not even in evolutionary timescales, so it doesn't seem wholly irrational to hope that humanity or its progeny will already have ventured to havens beyond our solar system before it happens.
I have something to say about how that might happen.

The idea of a universe in ruins has no support which I know of other than the second law of thermodynamics.
I have no problem with the practical applications of thermodynamics, which as far as I am aware mainly talks about entropy changes in closed systems.

I have struggled to understand the concept in the context of the second law, and struggled to comprehend the evidence offered for the second law, and I have neither understood nor believed.
So this essay is written as a speculation about how evolution has and will continue to progress, by someone who has no expectation that this will ever come to an end (and is agnostic about whether it ever had a beginning).

I might add a more comprehensive scepticism about the plausibility about the science of the very greatest extremes.
I doubt that scientists will ever realise perfect models of the microscopic structure of the universe, or about the gross structure of the universe, or about what happens close to the supposed singularities in relativistic models.
However far we have been able to peer with our best instruments, there may yet be something beyond which defies our expectations.

As to the significance of these beliefs to this essay, it is small and mainly psychological.
The timescales in which the second law might be expected to yield heat death are beyond those in which my speculations are credible even to me.
So far as I know, if I did believe the second law of thermodynamics, I would still think it ``academic'' and write the same essay.
Nevertheless I reject Russell's perspective.
Russell has made important contributions in some of his other works to the ideas presented here, and has been for me and many others over the last century, an inspirational figure.
But on this I demur.

\section{Introduction}

There is at the centre of this essay a very simple idea which is not mine.
I believe its importance is much greater than has been recognised, and this essay is intended argue the case for that belief.
More important than the belief itself is its consequences for the way in which artificial intelligence (and synthetic ecology) is approached and progressed.

Having said that, the ideas in the essay are not intended to be a direct contribution to those fields, in which I have no expertise.

The central idea is very simple, and apart perhaps from the details of presentation and the conception of significance, is not original.
A naked presentation could not convey the importance which I attach to it, and this document is primarily intended to provide a backdrop against which it's importance can more clearly be perceived.

That backdrop is evolutionary.
It is a story of the many kinds of evolution which have brought us to this moment in time at this point in space, and a speculation about some of the evolutionary processes which will lead us forward and the product of those very different evolutions.

My own reading of certain aspects of evolution provides a basis for the speculations to follow about how evolution will itself evolve in the future and how that will shape the growing sphere of influence which homo sapiens has in the universe.
From those speculations, some conclusions are drawn about the representation of knowledge and the automation of reasoning which lead to a a sketch of how those requirements might best be satisfied which form the kernel of this essay.

\subsection{Evolutionary Themes}

It is evolution which has shaped life on earth.
But no single kind of evolution can explain it all, and the purpose of this essay is best served by a varied diet.

If we are to consider the future trajectory of evolution, it will be necessary to identify some of the characteristic which those different kinds of evolution have in common, as features most likely to be preserved into the future, as well as to consider those aspects of contemporary evolutionary processes which are most closely aligned to context which will not be preserved into our futures.

Homo sapiens is soon to become and interplanetary species, perhaps then interstellar.
This is such a profound transformation in context that it must surely have impact not only on what evolves, but on how evolution works, both in the minutiae of the reproductive processes, the ways in which variation occurs and the selective pressures which guide the process.

So, what is evolution?
Some kinds of evolutionary thinking go back as far as the ancient Greeks, but modern conceptions of evolution date back to Darwin, who offered evolution as an explanation of the origin of species.
As science has progressed the dominant conception of evolution has also advanced, the development most conspicuously punctuated by the ``modern synthesis'' of which the most important aspect, though by no means the whole, was the incorporation of Mendelian genetics.

For our purposes its best to start with a simple model, since we seek an idea of evolution which will have broad scope.

Darwinian evolution may be characterised in three points:

\begin{itemize}
\item It concerns the evolution of species, which are populations of interbreeding organisms, and it is the species which are said to evolve.
\item As the population of the species reproduce, the offspring share many of the characteristics of the parent or parents, but they are not identical.
  Variation is thereby introduced.
\item Different variations may impact on reproductive success.
  This also happens in selective breeding by farmers, but in nature no deliberate selection is involved, and the differential success in reproduction is attributed to ``natural selection''.
\end{itemize}

It is generally emphasised that the variation which occurs is random and that the selection process is ``natural''.
These emphases are necessary because the intention is to assert that evolution, even of the most complex organisms, can work without any intelligent intervention.
Neither in creating appropriate variations, nor in weeding out those variations which are not advantageous.

Notwithstanding that point, even though intelligent intervention may be inessential, it may still be reasonable to consider a progress evolutionary even if some intelligence is involved in selection (as is arguably the case in sexually reproducing species) or even in variation (which genetic engineering makes possible).
For the sake of talking about a broad swath of different evolutionary processes, I therefore offer a broader conception of evolution which encompasses Darwinian evolution.

One of the challenges which I address in this reformulation is to obtain a conception of evolution which includes the processes which took place in the ``primordial soup'' leading to the first living organisms, and also to include the continuation of evolution when we have artificial intelligence incorporated into possibly inorganic systems which are capable as a whole of self-reproduction, but capable of undertaking design modifications in the process to optimise effectiveness.
Such systems will still exhibit progress which reflects the differential success of the different design decisions, and we can expect the best design decisions to become dominant among the resulting populations.

The single shared feature which I can see here is change to a population arising from differential replication of its members, which process is unlikely to progress indefinitely unless the replication is not invariably perfect.

\section{How We Got Here}



\section{Some Places We Are Going}

\section{Knowledge Representation and Deductive Closure}

\phantomsection
\addcontentsline{toc}{section}{Bibliography}
\bibliographystyle{rbjfmu}
\bibliography{rbj2}


\addcontentsline{toc}{section}{Index}\label{index}
{\twocolumn[]
{\small\printindex}}

%\vfill

%\tiny{
%Started 2023-08-01


%\href{http://www.rbjones.com/rbjpub/www/papers/p037.pdf}{http://www.rbjones.com/rbjpub/www/papers/p037.pdf}

%}%tiny

\end{document}

% LocalWords:
