% $Id: p037.tex $
% bibref{rbjp037} pdfname{p037}
\documentclass[10pt,titlepage]{article}
\usepackage{makeidx}
\newcommand{\ignore}[1]{}
\usepackage{graphicx}
\usepackage[unicode]{hyperref}
\pagestyle{plain}
\usepackage[paperwidth=5.25in,paperheight=8in,hmargin={0.75in,0.5in},vmargin={0.5in,0.5in},includehead,includefoot]{geometry}
\hypersetup{pdfauthor={Roger Bishop Jones}}
\hypersetup{pdftitle={Philosophical Foundations for Liberal Democracy}}
\hypersetup{colorlinks=true, urlcolor=red, citecolor=blue, filecolor=blue, linkcolor=blue}
%\usepackage{html}
\usepackage{paralist}
\usepackage{relsize}
\usepackage{verbatim}
\usepackage{enumerate}
\usepackage{longtable}
\usepackage{url}
\newcommand{\hreg}[2]{\href{#1}{#2}\footnote{\url{#1}}}
\makeindex

\title{\bf\LARGE Philosophical Foundations\\ for\\ Liberal Democracy}
\author{Roger~Bishop~Jones}
\date{\small 2020-12-23}


\begin{document}

%\begin{abstract}
%
% When society proceeds in a largely rational manner, what the average punter understands of the philosophical principles which underpin the social order may not be crucial.
% When things go awry, not so good.
% It might be a good idea to have a handbook which is as broadly intelligible as possible.
% Professional philosophers are rarely good at stripping down the complexity which proves their metal and laying bare key structures.
% This is me having a go at that.
% 
%\end{abstract}
                               
\begin{titlepage}
\maketitle

%\vfill

%\begin{centering}

%{\footnotesize
%copyright\ Roger~Bishop~Jones;
%}%footnotesize

%\end{centering}

\end{titlepage}

\ \

\ignore{
\begin{centering}
{}
\end{centering}
}%ignore

\setcounter{tocdepth}{2}
{\parskip-0pt\tableofcontents}

%\listoffigures

\pagebreak

\section*{Preface}
\phantomsection

\addcontentsline{toc}{section}{Preface}


\section{Introduction}

\ignore{

``Critical Social Justice''(CSJ) is an ideology which rejects rationality and the epistemological norms associated with ``Western Civilisation''.
These and other supposedly Western norms are rejected by post-colonial theory (an element of CSJ) as part of the systematic oppression by the colonial West of the colonised East.
The doctrines of Western Civilisation (and all other ``meta-narratives'') lack any objective truth and serve only to establish and perpetuate the power of the oppressors and the subjugation of the oppressed.

Those who see difficulties in this point of view, and think perhaps that there may be some redeeming features in the cultural heritage which seems to have abetted rising prosperity around the globe, will find dialogue with the proponents of CSJ unfruitful.
Reasoned dialogue is impossible with those who are unwilling to accept any rational ground rules.

For that reason this essay is not an attempt at dialogue.
It is a laying out, and an advocacy for, those elements of ``Western Culture'' which seem to me most fundamental to the advances which humanity has made over the last three millenia.
The essay addresses those who are not yet under the spell of CSJ ideology, and perhaps some who are already sufficient in doubt about their authodoxy that they will momentarily entertain an alternative viewpoint.
It is especially intended for those whose familiarity with the most fundamental parts of that Western Philosophy rejected by Critical Theory may be limited, who might not have thought these matters important, or who have been educated by schools and universities now intent on looking elsewhere.

My use here of the idea of {\it Western} Philosophy or Culture is perhaps anomalous.
Most of what I speak of is part of our common global heritage, it does not now and never did belong to one hemisphere of our planet.
Nevertheless this term is used, and I don't claim to be giving a broader story.
}%ignore

Where reason prevails liberal democracy may flourish.
Homo sapiens, possibly the only species capable of reason, is, however, also liable to tribal ideology in which belief in absurdities is the litmus test of ideologic purity.

In default of discourse, force may prevail.
Democracy is the choice to resolve differences at the political level without resort to force, and to provide a means (ultimately backed by a state mononopoly over the use of force) for the resolution of differences between individuals and groups within the state, on the basis of a code of law.

Not so long ago, national, let alone global, media were in the hands of tiny elites, and paid little attention to the opinions of the man in the street, which they could hope to shape.
In that context, ignorance of epistemological fundamentals and rational norms on the part of common folk might not have been so significant.
Today, we have reason for greater concern that ``narratives'' may be pressed, public opinion may be swayed, and governments unseated, by activist rhetoric (from anywhere on the political spectrum) in the face of contrary evidence.

In my primary and secondary education, I have no recollection of any attempt to educate me in how to assess the credibility of such claims.
Any ability to do so was acquired by osmosis rather than instruction.

In the contemporary concern over ``fake news'' the suggestion seems to be that one should judge the credibility of the source rather than probe the evidence.
Those sources most likely to be supposed authoritative, mainstream media, have become increasingly partisan, to the point where one may reasonably doubt whether they even recognise a distinction between facts and opinion.

In the new proliferation of sources, it might be helpful to have clear articulation of fundamental principles belonging to epistemology and the philosophy of language, logic and science together with exposure of the most common and egregious ways in which those principles may be flouted.
It is a challenge to see such ideas presented for a broad audience, perhaps together with more detailed and nuanced expansions for those willing to dig a little deeper.

In this essay I'm mainly interested in putting together a story, rather than in curating a presentation which presumes no prior philosophical understanding (though that is where I think I ought to be heading).

It is a discussion of \emph{foundations} and is therefore certain to be ultimately circular, but I hope, nonetheless informative.
This particular circularity involves philosophy of language, of logic, and epistemology, and the concept of \emph{rationality} with which I will begin.

\section{Rationality}

Like most words in the English language which have not been made precise by mathematics or science, the word \emph{rationality}\index{rationality|textbf} has diverse usage which thoroughly obscures the distinction between what is part of the \emph{meaning} of the concept and what is part of our beliefs about what is rational.

For my present purposes I propose to offer a definition of the concept, which is intended to clarify its use in this essay, not to say anything about how it may have previously been used.

My usage is primarily \emph{instrumental} and \emph{normative}.
A course of action may be said to have \emph{instrumental rationality}\index{rationality!instrumental|textbf}  if it is reasonable to suppose that it will realise the purpose for which it was undertaken.
A belief has \emph{epistemic rationality}\index{rationality!epistemic|textbf}, if the belief is held on good grounds, evidence which shows the belief to be most probably true.
We may consider epistemic rationality to be instrumentally rational, insofar as holding true belief will enable the adoption of effective ways of realising our purposes.

\section{Some Kinds of Knowledge}

I will mostly be concerned with aspects of epistemology which are confined to communicable knowledge, but its necessary first to make that distinction.

To that end I suggest that knowledge\index{knowledge} can usefully be considered as falling into three principle types:

\begin{itemize}
\item knowledge by acquaintance (been there, seen that)
\item knowing \emph{how} (done it)
\item knowing \emph{that} (got the tea-shirt?)
\end{itemize}

Knowledge \emph{by acquaintance}\index{knowledge!by acquaintance|textbf} is that familiarity which comes from direct experience, having been there, having seen it, perhaps something you heard, or even knowing how something \emph{feels}.
Maybe some phenomenon you have witnessed, a process you have observed, a ritual or a dance.

Knowing \emph{how}\index{knowing!how|textbf} will usually be a skill acquired by watching and doing, perhaps something which does not really require any \emph{skill} but just a knowledge of what to do acquired by seeing just the once, or trying and discovering.
Sometimes, even in non-human primates, it may be a skill properly mastered only over an extended period of time (years even).

Knowledge by acquaintance and knowing how are both found across a broad swathe of the biosphere, sometimes as innate knowledge buried in genes, sometimes passed from parent to child, or among peers, by example and mimicry, sometimes discovered.

Knowledge \emph{that}\index{knowing!that|textbf} is special and will be the main focus of our attention in this essay.
It appears only when we have descriptive language, in which \emph{propositions}, the content of indicative sentences, can be expressed.

A distinctive feature of this kind of knowledge is that it can be transferred in the absence of the circumstances to which it relates, passed from generation to generation or across large physical distances.
It makes possible an evolving oral culture propagated through story telling and singing.
In a period of unusually volatile climate change, geographical knowledge of where subsistence could be had in different climatic conditions would be valuable, social status would attach to the skills involved and the evolution of the physical and mental infrastructure for linguistic excellence would be accelerated by sexual selection on that basis.
Such conditions did occur in the 600 thousand year period starting just 800 thousand years ago.

During that period typical size of homo sapiens brains grew by about 50\% and then plateaued with the emergence of anatomically modern homo sapiens.
The linguistic abilities marked the beginning of culture and its evolution, an evolution which accelerated throughout the following 200,000 years, in steps which corresponded often to advances in our technology for preserving, replicating and communicating bodies of knowledge.


\section{Language}

\section{Logic}

\section{Metaphysics}

\section{Science}

\section{Skepticisms}

\section{Epistemologies}
\ignore{

\section{Epistemology}

Epistemology is the theory of knowledge.

We can begin with the following classification of kinds of knowledge:

\begin{itemize}
\item knowlege \emph{by acquaintance} 
\item knowing \emph{how}
\item knowing \emph{that}
\end{itemize}

Of which the first two are pre-lingual and are posessed by many species which have no capability for language, but the last is knowing the truth of some proposition, i.e. of that which is expressed by a sentence in a language.

We are concerned here only with \emph{knowing that}, and we find that the different ways of knowing (I shall use the notion of \emph{epistemic status} for this) are related to the meaning of a sentence.

There are important (if controversial) connections between distinctions in epistemology (epistemic status), distinctions in language (meaning or truth conditions), certain metaphysical distinctions (necessity, contingency) and the levels of confidence one can realise (certain knowledge v. speculative opinion).
I will run through these as best I can in short order.

Here is a table:

\begin{table}[h!]
\centering
 
\begin{tabular}{p{1cm} | p{1cm} p{1cm} p{1cm} p{1cm}}
  subject & abstract/math & concrete/science & moral values & personal values \\

  truth conditions & analytic & synthetic & ? & ? \\
  justification & deductive reason &  observation/experiment & god/conscience/reason? & preference/introspection \\
  modal & necessary &  contingent & \\
  confidence & certain & hypothetical &  good & \\
  objectivity &  objective & objective & objective/relative & subjective\\
  \end{tabular}
\end{table}

\section{Language}

The study of language includes two important elements:

\begin{itemize}
\item Pragmatics

  This concerns the ways in which language is used.
  
\item Semantics

  This concerns the \emph{meaning} of indicative sentences.
  
\end{itemize}

An important element of semantics is \emph{truth conditions}, these tell you under what possible circumstances each sentence is true.

}%ignore



\ignore{

\appendix

\section{CSJ in brief}

From Pluckrose\cite{pluckrose-evolution}.

\begin{enumerate}[i)]
\item  racism exists today in both traditional and modern forms
\item  racism is an institutionalized multi-layered multi-level system that distributes unequal power and resources between white people and people of color, as socially identified, and disproportionately benefits White's
\item  all members of society are socialized to participate in the system of racism albeit in various social locations
\item all white people benefit from racism regardless of their intentions
\item no one chose to be socialized into racism so no one is bad, but no one is neutral so not to act against racism is to support racism
\item racism must be continually identified analyzed and challenged no one is ever done
\item the question is not did racism take place but how did racism manifest in that situation
\item the racial status quo is uncomfortable for most White's therefore anything that maintains white comfort is suspect
\item the racially oppressed have a more intimate insight via experiential knowledge into the system of race than their racial oppressors but they're not bad
  \item however white professors will be seen as having more legitimacy thus positionality must be intentionally engaged (means you must always mention your race gender and sexuality and how it impacts on what you're saying)
\item resistance is a predictable reaction to anti-racist education and must be explicitly and strategically addressed
\end{enumerate}

\section{Postmodern Precursors}

From Pluckrose\cite{pluckrose-evolution}.


The imperative then, of postmodern approaches, is to study the discourses of society, to find the Foucian power-knowledge, invert the Derridian binaries and empower the Lyotardian mini-narratives.

This yields the following ``plan'':
     \begin{enumerate}[i)]
     \item there is no way of obtaining objective truth, everything is culturally constructed
     \item society is dominated by systems of power and privilege that people just accept as common sense
     \item these vary from culture to culture and subculture to subculture
     \item none of them is right or superior to any other
     \item the categories that we use to understand things, like fact and fiction, reason and emotion, science and art and male and female, are false
     \item they operate in the service of power need to be examined, broken down and complicated
     \item language is immensely powerful and it is used to construct oppressive social realities, therefore it must be regarded with suspicion and scrutinized to find the discourses of power
     \item the intention of the speaker is no more authoritative than the interpretation of the hearer
     \item the idea of the autonomous individual is a myth, the individual is also a construct of culture programmed by his or her place in relation to power
     \item the idea of a universal human nature is also a myth, it is constructed by what
powerful forces deemed to be the right way to be, therefore it is white Western masculine and heterosexual.
     \end{enumerate}


\phantomsection
\addcontentsline{toc}{section}{Bibliography}
\bibliographystyle{rbjfmu}
\bibliography{rbj}

}%ignore

\addcontentsline{toc}{section}{Index}\label{index}
{\twocolumn[]
{\small\printindex}}

%\vfill

%\tiny{
%Started 2020-07-06


%\href{http://www.rbjones.com/rbjpub/www/papers/p037.pdf}{http://www.rbjones.com/rbjpub/www/papers/p037.pdf}

%}%tiny

\end{document}

% LocalWords:
