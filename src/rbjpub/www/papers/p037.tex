% $Id: p037.tex $
% bibref{rbjp037} pdfname{p037}
\documentclass[10pt,titlepage]{article}
\usepackage{makeidx}
\newcommand{\ignore}[1]{}
\usepackage{graphicx}
\usepackage[unicode]{hyperref}
\pagestyle{plain}
\usepackage[paperwidth=5.25in,paperheight=8in,hmargin={0.75in,0.5in},vmargin={0.5in,0.5in},includehead,includefoot]{geometry}
\hypersetup{pdfauthor={Roger Bishop Jones}}
\hypersetup{pdftitle={A Rational Response to Critical Social Justice}}
\hypersetup{colorlinks=true, urlcolor=red, citecolor=blue, filecolor=blue, linkcolor=blue}
%\usepackage{html}
\usepackage{paralist}
\usepackage{relsize}
\usepackage{verbatim}
\usepackage{enumerate}
\usepackage{longtable}
\usepackage{url}
\newcommand{\hreg}[2]{\href{#1}{#2}\footnote{\url{#1}}}
\makeindex

\title{\bf\LARGE A\\ Rational Response\\ to\\ Critical Social Justice\\}
\author{Roger~Bishop~Jones}
\date{\small 2020/12/23}


\begin{document}

%\begin{abstract}
% An account of the fundamental principles and certain elements in the development of Western Philosophy oriented towards repudiation of Critical Social Justice and its ideological ancestry.
% 
%\end{abstract}
                               
\begin{titlepage}
\maketitle

%\vfill

%\begin{centering}

%{\footnotesize
%copyright\ Roger~Bishop~Jones;
%}%footnotesize

%\end{centering}

\end{titlepage}

\ \

\ignore{
\begin{centering}
{}
\end{centering}
}%ignore

\setcounter{tocdepth}{2}
{\parskip-0pt\tableofcontents}

%\listoffigures

\pagebreak

\section*{Preface}
\phantomsection

\addcontentsline{toc}{section}{Preface}


\section{Introduction}

``Critical Social Justice''(CSJ) is an ideology which rejects rationality and the epistemological norms associated with ``Western Civilisation''.
These and other supposedly Western norms are rejected by post-colonial theory (an element of CSJ) as part of the systematic oppression by the colonial West of the colonised East.
The doctrines of Western Civilisation (and all other ``meta-narratives'') lack any objective truth and serve only to establish and perpetuate the power of the oppressors and the subjugation of the oppressed.

Those who see difficulties in this point of view, and think perhaps that there may be some redeeming features in the cultural heritage which seems to have abetted rising prosperity around the globe, will find dialogue with the proponents of CSJ unfruitful.
Reasoned dialogue is impossible with those who are unwilling to accept any rational ground rules.

For that reason this essay is not an attempt at dialogue.
It is a laying out, and an advocacy for, those elements of ``Western Culture'' which seem to me most fundamental to the advances which humanity has made over the last three millenia.
The essay addresses those who are not yet under the spell of CSJ ideology, and perhaps some who are already sufficient in doubt about their authodoxy that they will momentarily entertain an alternative viewpoint.
It is especially intended for those whose familiarity with the most fundamental parts of that Western Philosophy rejected by Critical Theory may be limited, who might not have thought these matters important, or who have been educated by schools and universities now intent on looking elsewhere.

My use here of the idea of {\it Western} Philosophy or Culture is perhaps anomalous.
Most of what I speak of is part of our common global heritage, it does not now and never did belong to one hemisphere of our planet.
Nevertheless this term is used, and I don't claim to be giving a broader story.
Something special began in Ionia around 600 BC (it seems) which left a trail of

\section{CSJ in brief}

\begin{enumerate}[i)]
\item  racism exists today in both traditional and modern forms
\item  racism is an institutionalized multi-layered multi-level system that distributes unequal power and resources between white people and people of color, as socially identified, and disproportionately benefits White's
\item  all members of society are socialized to participate in the system of racism albeit in various social locations
\item all white people benefit from racism regardless of their intentions
\item no one chose to be socialized into racism so no one is bad, but no one is neutral so not to act against racism is to support racism
\item racism must be continually identified analyzed and challenged no one is ever done
\item the question is not did racism take place but how did racism manifest in that situation
\item the racial status quo is uncomfortable for most White's therefore anything that maintains white comfort is suspect
\item the racially oppressed have a more intimate insight via experiential knowledge into the system of race than their racial oppressors but they're not bad
  \item however white professors will be seen as having more legitimacy thus positionality must be intentionally engaged (means you must always mention your race gender and sexuality and how it impacts on what you're saying)
\item resistance is a predictable reaction to anti-racist education and must be explicitly and strategically addressed
\end{enumerate}

\section{Postmodern Precursors}

The imperative then, of postmodern approaches, is to study the discourses of society, to find the Foucian power-knowledge, invert the Derridian binaries and empower the Lyotardian mini-narratives.

This yields the following ``plan'':
     \begin{enumerate}[i)]
     \item there is no way of obtaining objective truth, everything is culturally constructed
     \item society is dominated by systems of power and privilege that people just accept as common sense
     \item these vary from culture to culture and subculture to subculture
     \item none of them is right or superior to any other
     \item the categories that we use to understand things, like fact and fiction, reason and emotion, science and art and male and female, are false
     \item they operate in the service of power need to be examined, broken down and complicated
     \item language is immensely powerful and it is used to construct oppressive social realities, therefore it must be regarded with suspicion and scrutinized to find the discourses of power
     \item the intention of the speaker is no more authoritative than the interpretation of the hearer
     \item the idea of the autonomous individual is a myth, the individual is also a construct of culture programmed by his or her place in relation to power
     \item the idea of a universal human nature is also a myth, it is constructed by what
powerful forces deemed to be the right way to be, therefore it is white Western masculine and heterosexual.
     \end{enumerate}

\section{Rationality}

\section{Language and Logic}




\phantomsection
\addcontentsline{toc}{section}{Bibliography}
\bibliographystyle{rbjfmu}
\bibliography{rbj}

%\addcontentsline{toc}{section}{Index}\label{index}
%{\twocolumn[]
%{\small\printindex}}

%\vfill

%\tiny{
%Started 2020/07/06


%\href{http://www.rbjones.com/rbjpub/www/papers/p037.pdf}{http://www.rbjones.com/rbjpub/www/papers/p037.pdf}

%}%tiny

\end{document}

% LocalWords:
