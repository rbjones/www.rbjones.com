% $Id: p040.tex $
% bibref{rbjp040} pdfname{p040}
\documentclass[10pt,titlepage]{article}
\usepackage{makeidx}
\newcommand{\ignore}[1]{}
\usepackage{graphicx}
\usepackage[unicode]{hyperref}
\pagestyle{plain}
\usepackage[paperwidth=5.25in,paperheight=8in,hmargin={0.75in,0.5in},vmargin={0.5in,0.5in},includehead,includefoot]{geometry}
\hypersetup{pdfauthor={Roger Bishop Jones}}
\hypersetup{pdftitle={The Systemic Incoherence and Ethical Bancruptcy of Critical Theory}}
\hypersetup{colorlinks=true, urlcolor=red, citecolor=blue, filecolor=blue, linkcolor=blue}
%\usepackage{html}
\usepackage{paralist}
\usepackage{relsize}
\usepackage{verbatim}
\usepackage{enumerate}
\usepackage{longtable}
\usepackage{url}
\newcommand{\hreg}[2]{\href{#1}{#2}\footnote{\url{#1}}}
\makeindex

\title{\bf{\LARGE Epistemology and Revolution}}
\author{Roger~Bishop~Jones}
\date{\small 2023-02-23}

\begin{document}

%\begin{abstract}
% An attempt to sew together into a coherent epistemological narrative my evolving perspective on the ideological currents of our time.
% 
%\end{abstract}
                               
\begin{titlepage}
\maketitle

%\vfill

%\begin{centering}

%{\footnotesize
%copyright\ Roger~Bishop~Jones;
%}%footnotesize

%\end{centering}

\end{titlepage}

\ \

\ignore{
\begin{centering}
{}
\end{centering}
}%ignore

\setcounter{tocdepth}{2}
{\parskip-0pt\tableofcontents}

%\listoffigures

\pagebreak

\section*{Preface}
\phantomsection

\addcontentsline{toc}{section}{Preface}

An attempt to put together an intelligible presentation of my evolving understanding of the intellectual maelstrom which is Planet Earth as it is transformed by near instant, near universally accessible, global communications.

There are ``hyperlinks'' in the PDF version of this monograph which either link to another point in the document  (if coloured blue) or to an internet resource  (if coloured red) giving direct access to the materials referred to (e.g. a Youtube video) if the document is read using some internet connected device.
Important links also appear explicitly in the bibiography.

\section{Introduction}

The conceit (we may call it) of enlightenment thought (as presented by some intellectuals) that truth could be judged by individuals for themselves rather than being dispensed by authority, seems to have been an historical anomaly, and the liberal democracies which followed, embracing both those enlightenment ideas and values and much of the `romanticism' which became more prominent in the nineteenth century, are now at risk of succumbing beneath the power of ideological authoritarianism.


\ignore{

``Critical Theory'' was the term adopted by Horkheimer \cite{horkheimer-trad, horkheimer-crit} for the work of the Frankfurt school after he took over the directorship of the school in 1930.
When capitalised the term usually refers to the work of the Frankfurt school, but the term is also used for a broader range of more recent theories eventually coming together under the umbrella term ``Critical Social Justice''.

The resulting ``theories'' are shaped in part by a thoroughgoing rejection of the idea of rationality as it was understood in the enlightenment, and are therefore, by design, radically incoherent and rationally irredeemable.
The departure from what Horkheim called ``traditional'' theory and the rational standards which it advocated did not take place in an abrupt transition, but in a number of innovations spanning two Centuries, prominent among which are contributions from Kant, Hegel, Marx, the Frankfurt School, Postmodern philosophy and a variety of subsequent critical theorists.

I shall take David Hume as a solid representative of ``traditional theory'' in the Enlightenment ($18^{th}$C), and will consider the subsequent divergence step by step from that traditional perspective.

To begin here, I simply enumerate the steps which I will be considering:

\begin{enumerate}
\item Humean positivism
\item Kant's Critique
\item Hegelian Dialectical Logic
\item Marx's Dialectical Materialism
\item Horkheimer's Critical Theory
  \begin{enumerate}
  \item The conflation of philosophy science and activism
  \item The conflation of disadvantage with enslavement
  \item Totalitarian democracy
  \end{enumerate}
  \item False consciousness
\item Marcuse's Repressive Tolerance
\item Postmodern Selective Radical Scepticism
  \begin{enumerate}
  \item Meaning
  \item Truth
  \item Power
    \end{enumerate}
\item Applied critical theory
\end{enumerate}

}%ignore

\section{On Human Nature}

\section{Hume}

Hume has been posthumously nominated the first positivist, that tendency in philosophy named by August Comte, which was particularly mentioned by Max Horkheimer when speaking of the ``traditional theory'' against which Critical Theory was to be contrasted \cite{horkheimer-trad}.
Hume became famous for his scepticism, a term provoked by his sharp understanding of the limits of deductive reason and his belief that even empirical facts cannot by themselves warrant moral conclusions.
The two distinctions, between verbal and factual, and between factual and moral, both sometimes referred to as ``Hume's fork'', are the first clear delineation of categories which had been important but elusive throughout the prior history of Western philosophy.
No sooner were they enunciated than they began to be unravelled.

Two points to note here.
First, the making of the distinctions does not entail that academics should confine themselves to one realm, even though there are important differences in methods across the realms which may corral enquiry.
Even if originality may be confined to one realm, there is applicability of some realms to others.
Thus, physicists need mathematics, use mathematics, but need not be innovators in mathematics and may not have the kind of competence required for that.
Engineers need physics (and its mathematics) ...
More on this when we come to Critical Theory.

\section{Rousseau}

Rousseau gets a mention here in relation to two subthemes.
While Hume is perhaps atypical as an enlightenment philosopher because his scepticism led him to doubt that reason could conclusively establish either empirical science or value judgements (or belief in God, hence underpinning counter enlightenment religious beliefs founded in faith rather than reason), Rousseau is atypical in completely different ways, and may be perhaps the enlightenment philosopher most easily seen as preparing the way for post-enlightenment romanticism.\cite{berlinRR}


He may therefore be seen as placed at the head of the division sometimes perceived between \emph{continental philosophy} and the more anglo-saxon analytic tradition.
Rousseau is the philosopher most closely associated with the French revolution, a champion of freedom and democracy, who nevertheless corrupted language with the doctrine that freedom consisted in acting according to `the general will'.
This equivocation was later elaborated for the totalitarian regimes of the $20^{th}$ Century, before being denounced by Orwell \cite{orwell-1984} only to be taken to yet greater extremes in the elaboration of Critical Theory in the $21^{th}$.


\section{Kant}

It is Kant who first raises objections to Hume's simple scheme.
To do this he introduces new terminology, usually translated into English as \emph{analytic} and \emph{synthetic}.
The words were not new, but the Kants usage was.
There was for example, in ancient Greece a distinction between analytic and synthetic proof, the former proceeding by analysis of the proposition to be proven, the latter by synthesis of that proposition from others.
More recently, Leibniz had used the concept of analytic, but his conception of the analytic was impacted by his theology, which attributed infinite analytic power to God and which extended the analytic beyond anything which a human might be able to establish.

After Kant the concept of analytic became a \emph{semantic} notion, but in Kant there is a dual characterisation with one of those seemingly more proof theoretic.
The characterisation which we are inclined to count semantic was that an analytic proposition is one in which the subject \emph{is contained in} the object, taking the containment to be containment of concept rather than of extension.
The `proof theoretic' characterisation was that a proposition is analytic if it is derivable from the law of contradiction.
In either case, synthetic propositions are those true propositions which are not analytic.

Using this mew terminology, Kant was able to differ from Hume by asserting that certain kinds of knowledge consist of synthetic \emph{a priori} propositions.
That is, rephrasing in more Humean language, that the scope of \emph{a priori} knowledge extends beyond Hume's relations between ideas.
This result is obtained in part by taking a more restrictive idea of the scope of `relations between ideas', notably by considering mathematics to be synthetic (thought still \emph{a priori}, and in part by taking a more liberal idea of what is \emph{a priori}, in this case by admitting the categorical imperative.

In this story, which begins by observing how the tight constraints upon what can be known with certainty according to Hume are progressively eroded to make way for the dogmatic totalitarian ideologies of the $20^{th}$ Century and the woke revolution of the $21^{st}$, Kant makes the first 

\section{Hegel}

My knowledge of Hegel is superficial.
I mention him here to point out some salient aspects of his dialectical logic.



\phantomsection
\addcontentsline{toc}{section}{Bibliography}
\bibliographystyle{rbjfmu}
\bibliography{rbj}

%\addcontentsline{toc}{section}{Index}\label{index}
%{\twocolumn[]
%{\small\printindex}}

%\vfill

%\tiny{
%Started 2021-02-19


%\href{http://www.rbjones.com/rbjpub/www/papers/p040.pdf}{http://www.rbjones.com/rbjpub/www/papers/p040.pdf}

%}%tiny

\end{document}

% LocalWords:
