% $Id: p044.tex $fi
% bibref{rbjp044} pdfname{p044}
\documentclass[10pt,titlepage]{article}
\usepackage{makeidx}
\newcommand{\ignore}[1]{}
\usepackage{graphicx}
\usepackage[unicode]{hyperref}
\pagestyle{plain}
\usepackage[paperwidth=5.25in,paperheight=8in,hmargin={0.75in,0.5in},vmargin={0.5in,0.5in},includehead,includefoot]{geometry}
\hypersetup{pdfauthor={Roger Bishop Jones}}
\hypersetup{pdftitle={Why and How Twitter}}
\hypersetup{colorlinks=true, urlcolor=red, citecolor=blue, filecolor=blue, linkcolor=blue}
%\usepackage{html}
\usepackage{paralist}
\usepackage{relsize}
\usepackage{verbatim}
\usepackage{enumerate}
\usepackage{longtable}
\usepackage{url}
\newcommand{\hreg}[2]{\href{#1}{#2}\footnote{\url{#1}}}
\makeindex

\title{\LARGE\bf Why and How Twitter}
\author{Roger~Bishop~Jones}
\date{\small 2021/12/13}


\begin{document}

%\begin{abstract}
% An attempt to put together a coherent position on the shifting language around sex and gender,
% and possibly a more general discussion of how to counter the divisive "identity marxism".
%\end{abstract}
                               
\begin{titlepage}
\maketitle

%\vfill

%\begin{centering}

%{\footnotesize
%copyright\ Roger~Bishop~Jones;
%}%footnotesize

%\end{centering}

\end{titlepage}

\ \

\ignore{
\begin{centering}
{}
\end{centering}
}%ignore

\setcounter{tocdepth}{2}
{\parskip-0pt\tableofcontents}

%\listoffigures

\pagebreak

\section*{Preface}

\addcontentsline{toc}{section}{Preface}

This short note is my attempt to aticulate some thoughts about what I might be doing as a user of twitter, and how I think it might be done.

I live in a democracy.
The principle way in which I formally contribute to that is by voting in general elections.
For most of my life (possibly all of it) I have lived in constituencies which are not marginal.
Under our first-past-the-post system, that means that my vote is not going to make a difference to the outcome, though it will contribute to certain ways in which the election may be described (e.g. when the percentage vote for each part is mentioned).

I read twitter;  I do a certain amount of responding in the usual ways.
I have, at this moment, 57 followers, though often I will be responding to people who have many more.
Hence the impact of what I do on twitter is microscopic (or zero).
Nevertheless, it is probably greater than my impact on the outcome of democratic elections that I exert by casting my vote.
Nevertheless, I almost always cast my vote, and if I don't that is because I have chosen to abstain (which I did in the Brexit referendum, and in one or two recent general elections).

This document is not written because I think I make a difference, but just in case.
I think we should all think about what is right and stand up for it in whatever ways we can, however insignificant.

\footnote{There may be ``hyperlinks'' in the PDF version of this monograph which either link to another point in the document  (if coloured blue) or to an internet resource  (if coloured red) giving direct access to the materials referred to (e.g. a Youtube video) if the document is read using some internet connected device.
Important links also appear explicitly in the bibiography.}

\section{Introduction}

On twitter, and beyond, by reference to internet resources and published books, almost every imaginable opinion on virtually every topic which could be of interest is presented.
There is no topic on which there will not be many voices far more knowledgable than mine,

The expert voices are outnumbered manyfold by contributions more vociferous and less well informed, but even posession of ``the facts'' (if they can be settled) does not necessary come with sound judgement.
One might hope that in open discussion the wheat could be sorted from the chaff, but the levels of polarisation we are seeing make compromise, convergence and consensus across the divides seem unlikely.

I  have throughout my life considered myself politically left of center, willing to vote tactically in the hope of unseating conservatives.
Now, as an old man, I'll concede that the center have moved to my left, and many younger voters would call me ``far-right'' or even deride me a fascist.

Of the many important and controversial issues which impress themselves upon me, the one which concerns me most is the poor standard of the discourse.

From this self-perceived lack of epistemic distinction and a desire to put my shoulder gently to the wheel I have started these notes on method.


\section{Some Principles}

Here are some ideas on how to go about it:

\begin{itemize}
\item Speak clearly
  
\item Speak truthfully
  
\item Try to understand others
  
\item Resist linguistic drift and manipulation
  
\item Notice equivocation

\item Distinguish: logic, fact and value
\end{itemize}

A longer catalogue, with illustrations, of ways of distorting meaning and truth is a good idea.
These are liberally interlaced with neologisms, which are not necessarily bad, and often essential to illumination.
James Lindsay \cite{lindsay-translations} has been compiling a dictionaries of wokeish so that people can be aware of the distortions, Peter Boghossian is engaged in a similar project.


\phantomsection
\addcontentsline{toc}{section}{Bibliography}
\bibliographystyle{rbjfmu}
\bibliography{rbj}

%\addcontentsline{toc}{section}{Index}\label{index}
%{\twocolumn[]
%{\small\printindex}}

%\vfill

\tiny{
Started 2021/12/14


\href{http://www.rbjones.com/rbjpub/www/papers/p044.pdf}{http://www.rbjones.com/rbjpub/www/papers/p044.pdf}

}%tiny

\end{document}

% LocalWords:
