% $Id: p045.tex $fi
% bibref{rbjp045} pdfname{p045}
\documentclass[10pt,titlepage]{article}
\usepackage{makeidx}
\newcommand{\ignore}[1]{}
\usepackage{graphicx}
\usepackage[unicode]{hyperref}
\pagestyle{plain}
\usepackage[paperwidth=5.25in,paperheight=8in,hmargin={0.75in,0.5in},vmargin={0.5in,0.5in},includehead,includefoot]{geometry}
\hypersetup{pdfauthor={Roger Bishop Jones}}
\hypersetup{pdftitle={Progressive Resistance}}
\hypersetup{colorlinks=true, urlcolor=red, citecolor=blue, filecolor=blue, linkcolor=blue}
%\usepackage{html}
\usepackage{paralist}
\usepackage{relsize}
\usepackage{verbatim}
\usepackage{enumerate}
\usepackage{longtable}
\usepackage{url}
\newcommand{\hreg}[2]{\href{#1}{#2}\footnote{\url{#1}}}
\makeindex

\title{\LARGE\bf Progressive Resistance}
\author{Roger~Bishop~Jones}
\date{\small 2022/01/04}


\begin{document}

%\begin{abstract}
% A positive reconstruction of Critical Theory.
%\end{abstract}
                               
\begin{titlepage}
\maketitle

%\vfill

%\begin{centering}

%{\footnotesize
%copyright\ Roger~Bishop~Jones;
%}%footnotesize

%\end{centering}

\end{titlepage}

\ \

\ignore{
\begin{centering}
{}
\end{centering}
}%ignore

\setcounter{tocdepth}{2}
{\parskip-0pt\tableofcontents}

%\listoffigures

\pagebreak

\section*{Preface}

\addcontentsline{toc}{section}{Preface}

Though I have had an account on Twitter for more than a decade, it is only in the last couple of years that is has engaged my attention.
Though social media in general, and twitter in particular, come in for a lot of flack, its clear that they are not all bad.
Through twitter I have become aware of a many things of which I most probably would otherwise have remained blissfully ignorant.

One of those is the migration of US citizens who thought of themselves as ``liberals'' out of the democratic party, sometimes into a central no-mans-land, sometimes as far as the now trump-tainted republicans.
One might suspect that this is symptomatic of that decline in idealism which often accompanies advancing years, but that is certainly not how they perceive it.
The complaint often is, that they have just the same ideals as they had before, but that the democratic party has radicalised and left them adrift.
The doctrines and policies which have been particularly effective in provoking the migration are those dubbed ``identity politics'', the praxis of applied critical Marxism.
They are perceived to have created a new radical strain in the democratic party, the dominance of which was not apparent until after the election of the moderate Joe Biden.
Similar effects in many other English speaking countries.

These ideas and policies leave former progressives fighting a rearguard action, suddenly seeking to conserve valued features of our culture and body politic instead of continuing to press for new policies which address their defects.

That people seeking progress should feel it necessary to devote their energy to preventing what they see as undesirable change is regrettable.
A large proportion of misdirected zeal is progressive in intent.
It would be better if it were possible, to offer an alternative and better way of achieving the advances radicals seek rather than trying to stop them in their tracks.
That would involve getting them to think harder about exactly what progress they are seeking.

This essay is an attempt to articulate how that might be done.

I am not a scholar, not expert in any topic (though better in some than others) and least of all a historian.
Nevertheless, my concern about the direction and impact of modern activists has made me enquire into the history, in the hope that understanding might follow, and my ideas here about how we might make real progress are given context by reference to the history as I understand it.
It will not be difficult for most people to pick holes in my understanding of the history.
The ideas nevertheless are offered for consideration in their own right, and are not original or complex (though the composite may be).
The historical references are primarily to illuminate rather than justify the choices I present.



\footnote{There may be ``hyperlinks'' in the PDF version of this monograph which either link to another point in the document  (if coloured blue) or to an internet resource  (if coloured red) giving direct access to the materials referred to (e.g. a Youtube video) if the document is read using some internet connected device.
Important links also appear explicitly in the bibiography.}

\section{Introduction}

The ``progressive resistance'' we envisage here is at once resistance to some aspects of the dominant progressive movements of the day not from a conservative spirit but from a perspective which recognises and values the progress that has already been made and seeks to build on it while excising its least desirable tendencies and effects.

If pressed to name one feature of some ``progressive'' actors it is the doctrine, which we associate among others with Herbert Marcuse, that the present system must be completely destroyed as a prelude to, ... to what?
This attitude, is not progressive but revolutionary.
It is the contentment of people with the system which is the principle problem, and which mitigates against any movement for radical change.
This satisfaction is decried as ``false consciousness'', a phrase whose purpose is to legitimise the radical agenda, \emph{divide} and \emph{destroy}.

\begin{itemize}
\item Hegel
\item Marx
\item Horkheimer's Critical Theory
\item Marcuse's Critical Theory
\item Postmodernisms
\item Applied Critical Theories
\item Praxis
\end{itemize}

\section{Hegel}

My knowledge of Hegel is based on slender foundations, which don't include actually reading what he wrote.
I don't include this section in the belief that I know Hegel better than anyone else, but rather by way of explanation of how I have arrived at the ideas which I put forward in this essay.

I did once have Hegel's logic on my bookshelf, but my bullshit detector went apoplectic when I tried to read it, and it went to charity (or probably to pulp) the last time we downsized.
The delusions which I will here recount are therefore inspired by secondary sources, of which the most important are Sabine \cite{sabine63} and Singer interviewed by Magee \cite{magee-singer}.
Long before consulting them I had read Popper's excellent diatribe \cite{popper-ose}, which undoubtedly contributed to my negative opinion of Hegel, but not so far as I am aware to the detail below.

Since the invention of axiomatic geometry, deductive reasoning has had an allure to those who want their opinions to be considered definitive.
Unfortunately, almost no significant doctrines outside of mathematics fall within the scope of deductive logic.
For a good dose of realism about what does and does not fall within that scope, Hume's \emph{Enquiry} \cite{humeECHU} is worth a look, where the relevant claims are said to express ``relations between ideas'' and are clearly distinguished from ``matters of fact'' which are beyond the reach of pure reason.
That dichotomy became known as \emph{hume's fork}, and was supplemented by another Humean dictum, that you cannot derive an \emph{ought} from an \emph{is}\footnote{distinguishing logical and empirical truths from moral imperatives and other value judgements}, also sometimes known as Hume's fork.

Both before and after Hume these distinction have been misunderstood, or simply ignored, by those who want their views to be seen as proven, and for the sake of certainty seek the imprimateur of deductive logical proof.
To be fair, our language is not so precise as to make these distinctions watertight, but they nevertheless have merit in separating claims which it is prudent to separate because the ways in which the claims can be verified differ in fundamental ways.
\emph{Progressive resistance} promotes that kind of precision in language which supports the relevant epistemic differences.
In general, the progressive impulse is better served by noting imperfections, whether in language or in society, and seeking to illiminate or ameliorate them, by contrast with the revolutionary mindset, which seizes and eggagerates inevitable weaknesses to progress the \emph{divide and destroy} path to utopia.

Hegel, having spent most of his life as a historian, was naturally inclined to think history important, and to think that his method of reasoning about its progress unimpeachable.
In record breaking overreach he claimed of his dialectical method that it had the force of logic, that it enabled the prediction of what history would yield, and that not only di the predicted results have the abolute necessity which is commonly associated with deductive reasoning, but that they also constituted a moral imperative.

It is probable that Hegel did not envisage this doctrine fuelling the hubris of revolutionary zealots, but this was not the first or the last time that a philosopher would find his writings headings underpinning developments he would not condone.


\section{Marx}

Philosophically, I gather, Marx whose method is called ``Dialectical Materialism'', is only a small step from Hegel's Dialectic.
Whereas Hegel was an idealist, an a ``right-Hegelian'' and the dialectic he invisaged was of the national culture or spirit evolving dialectically toward some perfect state, Marx was a materialist and the evolution at stake was economic, of the means of production.

This is the most fundamental point on which compromise seems infeasible.
Either you seek to destroy the system or to reform or advance it, there is no between.
in addition to the incompatibility of these two as ends, the means whereby these distinct ends are approached are likely to be highly contrasted, not least because the revolutionary impulse regards incremental progress as counterproductive because of its potential to make the revolution less essential or urgent.
Also, revolution is destructive, incrememental progress is constructive.

Soon, as we transition into Critical Theory, the prospect of violent overthrow of the system loses credibility, and the strategy for revolution becomes cultural rather than military.
The distinction remains, stark, between a strategy predicated on complete destruction of the existing system, and one of progressive evolution, building on strengths and addressing weaknesses.

This is the distinction we sloganise as that between:

\begin{minipage}[t]{0.8\linewidth}

\begin{centering}
\vspace{0.1in}
       {\bf DIVIDE and DESTROY}
       
\vspace{0.1in}
and

\vspace{0.1in}
       {\bf PRESERVE and PROGRESS}
       
\vspace{0.1in}
\        
\end{centering}
\end{minipage}


It is the latter strategy which underpins \emph{Progressive Resistance}, in which we seek to resist destruction of the culture and institutions which have brought us prosperity and humanity so that we can continue the progress which has brought them this far.
That name is chosen to emphasise that the resistence against the revolutionary impulse is not a purely conservative or reactionary force, it can be as solidly rooted in a commitment to progress.
Of course, that is not how a revolutionary would speak of these things, if he were not already convinced that our culture and institutions were fundamentally flawed then he would make that pretence to fuel the insurrection.

From this stage in history I nevertheless take up some elements.
To the extent that some groups of people are unfairly exploited economically, that is something which we would wish to address, and history has surely shown that in economic terms, progress is more likely by incremental than by revoutionary means.
However, 


\section{Critical Theory}

Critical Theory is a derivative of Marxism developed at the Institute for Social Research, initially in Frankfurt.
The innovations in Critical Theory are generally held to respond to the failure of the Marxist prediction of the collapse of Capitalism as a result of a revolution of the proletariat.

In re-thinking Marxism as Critical Theory, primarily at first under the leadership of Horkheimer, we see a mix of ideas which can comfortably fit into a progressive agenda, with some which seem exclusively revolutionary.

The former can be selectively taken into a modern progressive program, the latter do not belong there.

\subsection{Some Philosophy}

First of all let us consider the philosophical and methodological elements.

There is doubtless more from Hegel/Marx here than I am aware of us.
These are the points which have influenced my conception of \emph{progressive resistance}.

The phrase ``Critical Theory'', which seems to be a replacement for ``Marxism'', is contrasted not with Marxism but with ``Classical Theory'' which Horkheimer seems to have a more definite conception of than most philosophers, as perhaps a not quite so extreme version of logical positivism, which was probably just past its peak as Horkheimer took the lead at the Institute of Social Studies.
Nevertheless the content of the proposal has some merit.

Sociology may possibly have been the science least well suited to the model of science offered by physics, which provided an ideal on which philosophies of science are oriented.
Unfortunately, people are unpredictable, in general, but malleable under social influence, both through the imposition of culture in the development of the young, and through the impact of social context on individual behaviour.
During the enlightenment the successes of the hard sciences were to readily supposed replicable in the social sciences, and eventually some alternative conception of the sciences would have to prevail.
One such alternative is the Hegelian dialectic, in which it is historical forces which shape the present and future of society, a model inherited by dialectical materialism.

However, the predictions of Marx had failed to materialise.
A revolution did take place in Russia, but it was not a proletarian revolution.
The anticipated revolution in Germany did not materialise.
Some further development of the ideas was surely necessary (though some Marixsts would disagree, including Herbert Marcuse\footnote{
 In 1977 Marcuse denied that Marxism had been shown to be fallacious, when questioned about this by Brian Magee \cite{magee-marcuse}}, perhaps the most important figure following Horkheimer in Critical Theory).


\subsection{The Aim of Emancipation}

The Marxist goal might be thought of as the emancipation of the proletariat from exploitation by capitalists.



\subsection{Democracy and Liberty}



\phantomsection
\addcontentsline{toc}{section}{Bibliography}
\bibliographystyle{rbjfmu}
\bibliography{rbj}

%\addcontentsline{toc}{section}{Index}\label{index}
%{\twocolumn[]
%{\small\printindex}}

%\vfill

\tiny{
Started 2022/01/04


\href{http://www.rbjones.com/rbjpub/www/papers/p045.pdf}{http://www.rbjones.com/rbjpub/www/papers/p045.pdf}

}%tiny

\end{document}

% LocalWords:
