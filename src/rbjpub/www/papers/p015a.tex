I sustain to myself the impression that my philosophical thinking is, though in some ways chaotic, neverthless structured and motivated.
It is my aim here to expose that structure and those motivations, and to place in the structure an outline of those philosophical beliefs and attitudes which seem to me most important.

\section{Scope and Divisions}

I do not attach great importance to the demarcation of philosophy from other kinds of knowledge, the attempt to effect clean disciplinary boundaries is even more to the detriment of philosophy than it is of other disciplines.
In the popular mind philosophy is kin to religion, and is concerned with big and important problems such as ``the meaning of life''.
I concur with this expectation of philosophy.

For me, philosophy begins with the existential dilemma, what to do, and all other philosophy is subordinate to that.
Nevertheless, the main bulk by far of my philosophical efforts are directed to more theoretical problems.

The distinction I find helpful, even though it may not be possible to make it precise, that between practical and theoretical philosophy.
In all branches of knowledge there are unlimited possibilities for the furtherance of our knowledge, not all equally deserving of our time.
Which of these to pursue, for those whose interest is in the pursuit, is a part of the existential dilemma.
For some this seems to be a simply a matter of what they find more interesting, for many professionals it will be determined in the interest of career.
For me it is matter of finding the areas in which there is best hope of my talents, such as they are, yielding some significant `practical' benefit (not necessarily for myself).

