For two and a half millenia up to the nineteenth century logic has been a province of philosophy.
The twentieth century spawned the mathematical discipline of symbolic logic, and a new conception of philosophy as analysis, in which the role of logic was to be a matter of controversy.

There are several stories about the beginnings of modern analytic philosophy.
The first story was that it began with Russell and Moore in Cambridge around the turn of the $20^{th}$ century, as they moved away from british idealism.
One can see in these two philosphers rather different conceptions of analysis which reverberated down the century and may be seen in the diverse views of subsequent philosophers on the place of formal logic in philosophy and science.

Moore's conception of analysis had little connection with formal logic, and was concerned with the elucidation of philosophically interesting aspects of natural English.
Russell was involved in the developments to formal logic inspired by the logicist thesis that mathematics \emph{is} logic.
His first decade of the century was devoted to obtaining a formal logic (Russell's \emph{Theory of logical types}\cite{russell1908}) and the formal derivation of mathematics (with A.N.Whitehead) in \emph{Principia Mathematica}\cite{russell10}.
