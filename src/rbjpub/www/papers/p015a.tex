
\begin{quote}
\emph{
Philosophy, from the earliest times, has made greater claims, and achieved fewer results, than any other branch of learning. Ever since Thales said that all is water, philosophers have been ready with glib assertions about the sum-total of things; and equally glib denials have come from other philosophers ever since Thales was contradicted by Anaximander. I believe that the time has now arrived when this unsatisfactory state of things can be brought to an end.}
\end{quote}
With these words Bertrand Russell opened the first chapter of \emph{Our Knowledge of the External World, as a Field for Scientific Method in Philosophy}\cite{russell1921}.

Russell's presentiment of momentous change was provoked by advances in logic on a great scale which began toward the end of the $19^{th}$ century, instigating the new discipline of mathematical logic and a period of rapid advancement in both mathematical and philosophical logic which continued throughout the following century, and continues apace.

These advances have been primarily technical advances in a new technical discipline.
What concerns us here is not these technical advances in themselves but the prospect of a transformation in the methods and in the veracity of philosophy which Russell perceived.

Possibly there are philosophers who believe that what Russell anticipated came to pass, though perhaps not quite as he anticipated.
I shall argue here that it did not, but that it still may (though certainly not just as he anticipated).
My presentation will begin with the background, expanding on Russell's concise account of the `unsatisfactory state of things'.
Next I shall describe how it seems to me matters have proceeded for those, particularly Rudolf Carnap, who were persuaded by Russell's prospectus and dedicated their lives to its realisation, to no avail.
From this dispiriting tale I shall then advance to a new prospectus, in which the picture is painted anew, and the question is addressed how it might ultimately be realised.

\section{Some Background}

\subsection{Two Faces of Deduction}

It might be argued that deductive reason begins with the evolution of \emph{descriptive} language, but the idea of deduction, an awareness of it as a method for the advancement of knowledge, begins, to the best of our knowledge, at the same time in ancient Greece as western philosophy and mathematics.
The reputation of deduction and the high esteem in which it was then held rested on its successful application in the development of mathematics as a branch of theoretical knowledge rather than as a body of practically beneficial techniques.
Its success in mathematics may have encouraged philosophers to apply the method more widely.
Insofar as this constituted a rejection of religious and other forms of authority in favour of unprejudiced enquiry this was an important advance, but beyond mathematics it was to prove entirely unreliable.

\subsection{The Scope of Deduction}

\subsection{Rigour and Formality}


\section{The Frege/Russell/Carnap Programme}

\section{}


\appendix{Misc}

For two and a half millenia up to the nineteenth century logic has been a province of philosophy.
The twentieth century spawned the mathematical discipline of symbolic logic, and a new conception of philosophy as analysis, in which the role of logic was to be a matter of controversy.

There are several stories about the beginnings of modern analytic philosophy.
The first story was that it began with Russell and Moore in Cambridge around the turn of the $20^{th}$ century, as they moved away from british idealism.
One can see in these two philosphers rather different conceptions of analysis which reverberated down the century and may be seen in the diverse views of subsequent philosophers on the place of formal logic in philosophy and science.

Moore's conception of analysis had little connection with formal logic, and was concerned with the elucidation of philosophically interesting aspects of natural English.
Russell was involved in the developments to formal logic inspired by the logicist thesis that mathematics \emph{is} logic.
His first decade of the century was devoted to obtaining a formal logic (Russell's \emph{Theory of logical types}\cite{russell1908}) and the formal derivation of mathematics (with A.N.Whitehead) in \emph{Principia Mathematica}\cite{russell10}.
