% $Id: p007.tex,v 1.3 2006/09/11 10:11:46 rbj01 Exp $
\documentclass{rbjk}

\begin{document}                                                                                   
\begin{article}
\begin{opening}  
\title{The Formalisation of Physics}
\author{Roger Bishop \surname{Jones}}
\runningauthor{Roger Bishop Jones}

\begin{abstract}

\end{abstract}
\end{opening}

%\def\tableofcontents{{\parskip=0pt\@starttoc{toc}}}
\setcounter{tocdepth}{4}
{\parskip-0pt\tableofcontents}

\section{Introduction}

In this essay I propose to discuss the formalisation of physics.

The purpose of the discussion is epistemological.
It falls into three mains parts.
The first part describes a method of formalisation and a rationale for that method.
We then, in a positivistic vein, consider how the formalisation of a scientific theory may go beyond the empirical content of the theory, from the physics into what may be called {\it metaphysics}.
This leads us into consideration the epistemology and methodology of metaphysics as facilitated by the formalisation of physics.
Finally, the same formal method is considered as a basis for {\it the automation of reason} in a sense similar to that envisaged by Leibniz in his {\it calculus ratiocinator}.

\section{The Formalisation of Physics}

I propose to explain the proposed method of formalisation in stages as follows:

\begin{itemize}
\item nomologico-deductive science
\item science as modelling
\item mathematical models
\item the formalisation of mathematics
\item formal abstract models
\end{itemize}

\subsection{Nomologico-Deductive Science}

The method is first of all {\it nomologico-deductive}, by which is meant that the science consists in the formulation of general laws in language sufficiently precise that particular conclusions can be obtained by deduction from the laws, in conjuction with particular facts known about the intended application.

Possibly a part of the  {\it nomologico-deductive} method, but in any case a part of the method proposed here, is the limitation of science to a descriptive role, the description being of (aspects of) how the world behaves.
Insofar as this kind of science provides {\it explanation} rather than mere {\it description} it does so by reduction to more fundamental theories which are themselves descriptive.
At bottom, the most fundamental theories are taken to be purely descriptive without benefit of any ``explanatory'' further reduction.

\subsection{Science as Modelling}

This limitation to descriptive remit is made more definite here by the conception of science as engaged in the construction of {\it models} of reality.
The point of talking about models rather than theories is twofold.
The principle point is simply to avoid commitment to all the detail of the model.
The assertion that a model is faithful to reality is a more modest claim than the assertion that a theory is {\it true}.
Exactly how modest is not entirely clear, its depends upon some supplementary explanation of the relationship between the model and reality.

One, relatively extreme, way to take this relationship (between a model and the system it models) is phenomenalistically.
A model may for example be said to model the behaviour of the system rather than its structure.
Taken in this was the notion of science as engaged in modelling is a positivistic.
A central purpose of positivistic conceptions of science is to separate true (or positive) science from pseudo-science or metaphysics.
Without the dogmatically negative attitude towards metaphysics the modelling perspective can play a similar role in separating physics from metaphysics for us here.

In a purely scientific conception of modelling, we may identify various aspects of scientific models which may be thought of as convenient ways of achieving the required systematisation of the phenomena, but which do not necessarily themselves correspond to features of reality.
One might set aside all the apparent ontological aspects of a model apart from those directly relating to phenomena (which will often not feature in a model at all) as modelling fictions rather than reflecting features of reality.

\subsection{Mathematical Models}

Reality is complex, and faithful models of reality will inevitably share in this complexity.
The construction of complex and precise models depends upon mathematics.
There is no mystery in the relationship between mathematics and the world, it is not a miracle that the world corresponds to some mathematical structure.
Whatever structure the world has, that structure would be mathematics, which encompasses the science of all abtract models. 
That any concrete universe has a structure similar to that of some abstract model is also unsurprising.

\subsection{The Formalisation of Mathematics}

\subsection{Formal Abstract Models}

\section{Analytic Metaphysics}

\subsection{Conceptions of Metaphysics}

\begin{itemize}
\item[Aristotelian]

It is with Aristotle that metaphysics, so called, begins (I understand the name comes from the placement of his work on this topic after his work on physics).
The name originates in Aristotle, but is not his name, which was {\it first philosophy}, and is concerned with {\it first causes}.
Aristotle's conception of a cause is therefore of significance for his view of what was to be called metaphysics.
It is of interest here that Aristotle is explicitly looking for the kind of cause the knowledge of which is {\it wisdom}, his conception of philosophy is similar to what we might describe as the contemporary positive popular stereotype of the philosopher as wise man.
Aristotle considers wisdom to consist in knowledge of certain kinds of cause, which are therefore the subject matter of metaphysics.

Aristotle identifies four different kinds of cause as follows:

\begin{itemize}
\item the essence
\item the matter or substratum
\item the source of the change
\item the end or purpose of the change
\end{itemize}

\item[Hume]

The contrast between Aristotle and Hume is enormous.
With Hume, metaphysics is conceived in less flattering terms.
It has by now become associated with idle and fruitless speculation.

Hume divides indicative sentences into three groups, relations between ideas (a broadly conceived notion of logical statement), matters of fact (empirical or contingent propositions), and value judgements.
Metaphysics is now understood as falling outside this classification, as misconceived nonsense.

\item[Kant]

Partly in response to Hume's condemnation of metaphysics, in the course of refuting Hume, Kant came up with a more definite characterisation of the kind of metaphysics which he supported.
This characterisation he achieved by splitting Hume's (though it was not original in Hume) division of non-evaluative indicative sentences into two kinds.
In Hume the distinction between relations between ideas and matters of fact is both semantic and epistemological.
In Kant the semantic element becomes the analytic/synthetic distinction, the epistemological element is the {\it a priori/ a posteriori} distinction.
Metaphysics is then characterised as synthetic/ {\it a priori} knowledge.

\item[Logical Positivist]

Logical positivism is Humean positivism after the logical revolution.
The revolution in logic which occurred in the latter half of the nineteenth century was aimed primarily at the logicisation of mathematics.
Mathematics was perhaps the most concrete and valuable body of knowledge which was conceived by Kant as synthetic/a priori.
Logical positivism incorporated a logicist philosophy of mathematics, this eliminating the most important prolem for a positivist rejection of metaphysics.

\item[Kripkean]

Kripke reinvented metaphysics in the latter half of the 20th century.
This he did by arguing first that names are rigid designators, and considering this not to be a matter of semantics.
The imposition of a non-semantic constraint on truth conditions made it possible that there are necessary propositions which are not analytic, and hence made synthetic/ a priori knowledge respectable once more.
However the contrasts in motivation between Kripke and Aristotle is stark.
Aristotle was motivated by the desire for wisdom which he construed as knowledge about reality.
Kripke's metaphysics is the result of curiosity about language, not about the world.

\item[Analytic Metaphysics]

The kind of metaphysics under consideration here is {\it analytic} metaphysics.
\end{itemize}

Metaphysics is concerned with the true nature of reality.
This includes the nature of space and time, and the question of what things exist, physical or other.

In the twentieth century the structure of space and time, and other matters which might previously have been thought to belong to metaphysics, have been drawn into the scope of fundamental physics by the development of the theories of relativity and quantum electro-dynamics.

In the case of metaphysics, by contrast with physics, particularly when construed as mathematical modellin we are specifically concerned with what really exists, rather than with what may be convenient in modelling how things appear to be.
This means for example, that though a physicist can reasonably offer or accept a diversity of models for different aspects of reality which are logically incompatible when taken literally.

\section{The Automation of Reason}



%{\raggedright
%\bibliographystyle{klunamed}
%\bibliography{rbj,fmu}
%} %\raggedright

\end{article}
\end{document}
