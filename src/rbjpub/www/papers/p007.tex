% $Id: p007.tex,v 1.1 2005/04/09 14:00:53 rbj Exp $
\documentclass{rbjk}

\begin{document}                                                                                   
\begin{article}
\begin{opening}  
\title{Philosophical Themes}
\runningtitle{X-Logic}
\author{Roger Bishop \surname{Jones}}
\runningauthor{Roger Bishop Jones}

\begin{abstract}
An informal account of a formal calculus of assurance, its rationale and application.
\end{abstract}
\end{opening}

%\def\tableofcontents{{\parskip=0pt\@starttoc{toc}}}
\setcounter{tocdepth}{4}
{\parskip-0pt\tableofcontents}

\section{Introduction}

X-Logic is the most central formal aspect of the philosophy of metaphysical positivism.

\section{Background and Rationale}

X-Logic is a formal response to certain skeptical and positivistic ideas which are embraced in metaphysical positivism.
To give a good account of it I propose here to describe these ideas.

\subsection{Scepticism}

\subsubsection{Pyrrhonean Scepticism}

Elements of scepticism are evident throughout the history of philosophy.
Its most sustained, radical and systematic manifestation has been in the period which begins with Pyrrhus of Ellis and runs through to the comprehensive exposition of scepticism by Sextus Empiricus.

Within this period there is a diversity of sceptical standpoints.
I'm going to try here to pick out the key elements of those standpoints which feed into metaphysical positivism.

\paragraph{Sceptic as Enquirer}

A sceptic in the original Greek sense of this word was an enquirer or investigator.
Someone who seeks the truth.
One cannot know the truth if encumbered by falsehoods, and it therefore becomes one who seeks the truth first to expose the fallacious.
The first pitfall of a positive sceptic is therefore to become perpetually enmeshed in this preliminary clearing of the tables.

The first point of importance for the role of scepticism in metaphysical positivism is that scepticism in its contemporary sense is adopted for the purpose of furthering a true understanding of the world.

\paragraph{Scepticism as Resignation}

Sextus Empiricus give us an account of the purpose of scepticism which is in stark contrast this view of the sceptic as an enquirer.
He tells us that the role of scepticism is to induce in its adherent ``an untroubled and tranquil condition of soul''.
In the social turmoil of the period of Greek philosophy such tranquillity was a prized possession.
The search for truth has given way to a philosophy whose primary aims are first to make any question seem as likely false as true, and then to an quiet acceptance of this ``equipollence'' rather than an energetic and disturbing attempt to resolve the question.

This is just one of several ways in which skepticism can go astray.
Positivism in general, and metaphysical positivism in particular, attempts to use sometimes quite radical sceptical arguments without falling into these pitfalls, within the context of a positive program for seeking and applying knowledge.

\paragraph{Some Kinds of Scepticism}

Sextus Empiricus divides philosophers into three kinds:

\begin{itemize}
\item[dogmatists] who believe they have discovered the truth
\item[academics] who believe that the truth cannot be discovered
\item[sceptics] who go on searching
\end{itemize}

He and we are of the latter kind (though Sextus goes on to contradict himself with the equipollence and tranquility objective).

But here in the academics we have the second danger for a sceptic, which is to slip into a dogmatic (and incoherent) assertion that nothing can be known.
There is an interesting (though unimportant) problem here concerning what is the most radically negative scepticism which can <i>coherently</i> be maintained, but this is not an objective of metaphysical positivism.




\section{Desiderata for X-Logic}

\section{X-Logic}

\section{Application of X-Logic}


%{\raggedright
%\bibliographystyle{klunamed}
%\bibliography{rbj,fmu}
%} %\raggedright

\end{article}
\end{document}
