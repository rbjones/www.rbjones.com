% $Id: p041.tex $fi
% bibref{rbjp041} pdfname{p041}
\documentclass[10pt,titlepage]{article}
\usepackage{makeidx}
\newcommand{\ignore}[1]{}
\usepackage{graphicx}
\usepackage[unicode]{hyperref}
\pagestyle{plain}
\usepackage[paperwidth=5.25in,paperheight=8in,hmargin={0.75in,0.5in},vmargin={0.5in,0.5in},includehead,includefoot]{geometry}
\hypersetup{pdfauthor={Roger Bishop Jones}}
\hypersetup{pdftitle={The Mind Body Problem and Consciousness}}
\hypersetup{colorlinks=true, urlcolor=red, citecolor=blue, filecolor=blue, linkcolor=blue}
%\usepackage{html}
\usepackage{paralist}
\usepackage{relsize}
\usepackage{verbatim}
\usepackage{enumerate}
\usepackage{longtable}
\usepackage{url}
\newcommand{\hreg}[2]{\href{#1}{#2}\footnote{\url{#1}}}
\makeindex

\title{\LARGE\bf The Mind Body Problem and Consciousness}
\author{Roger~Bishop~Jones}
\date{\small 2021/05/21}


\begin{document}

%\begin{abstract}
% A discussion on the nature of consciousness, prefaced by broader considerations about ontology and metaphysics, in which I conclude in favour of the possibility that consciousness is no big deal (though important).
%\end{abstract}
                               
\begin{titlepage}
\maketitle

%\vfill

%\begin{centering}

%{\footnotesize
%copyright\ Roger~Bishop~Jones;
%}%footnotesize

%\end{centering}

\end{titlepage}

\ \

\ignore{
\begin{centering}
{}
\end{centering}
}%ignore

\setcounter{tocdepth}{2}
{\parskip-0pt\tableofcontents}

%\listoffigures

\pagebreak

\section*{Preface}

\addcontentsline{toc}{section}{Preface}

\footnote{There may be ``hyperlinks'' in the PDF version of this monograph which either link to another point in the document  (if coloured blue) or to an internet resource  (if coloured red) giving direct access to the materials referred to (e.g. a Youtube video) if the document is read using some internet connected device.
Important links also appear explicitly in the bibiography.}

\section{Introduction}

I have some ideas about the nature of consciousness.
I have no qualifications which might persuade anyone that my ideas about consciousness should be taken seriously.
I don't pretend they are `scientific', I am not a scientist.
Nor am I a professional philosopher, though I spend a lot of time thinking about and even attempting to write philosophy.

The origin of these ideas is in my own experience, and to a limited extent in anecdotal evidence from others.

You might think that before writing about consciousness I would pay some attention to what others have said about it.
I confess I have not.
From time to time I make an attempt.
But invariably I have found what people say when they talk about consciousness so unconvincing that I bail out in a really quite ridiculously short space of time.
I just did it a few minutes ago.
I saw on my twitter feed that Lex Fridman had talked to Sam Harris about consciousness, and though before launching into this essay that I would listen to what they had to say.
I gave up less than 4 minutes after Sam started talking.
So you may now be assured that I am thoroughly ignorant of what others have said about consciousness.

\section{Metaphysics and Science}

My discussion of conscousness below is instrumental and may be thought materialistic.
Insofar as ``materialism'' is the denial of any existent other than matter, it is not, even though may mention no other.
Of course, to offer an explanation of consciousness which fails to mention ``mind'' I may bolster the case for materialism, but that is not my intent.
On these matters I am in fact closely aligned with the now thoroughly discredited logical positists, or at least, with Rudolf Carnap the only logical positivist with whose views on these matters I am reasonably aligned.

\section{Lapses of Consciousness}

Here are some observations, partly about my own experience, partly about things which are pretty generally accepted, which I have taken to bear upon the problem.

I once had a fairly serious road traffic accident, in which the main damage was to my head, and resulted in various fractures and concussion.
Of course, I had no memory of the accident.
It occurred round about midnight, and I woke up in the hospital the next day not knowing what had happened, except that I knew I was driving when last conscious.
Though I didn't learn it at the time, sometime later I came to know a medical professional who happened to be around when I was brought in to the hospital, and told me that I had made a great fuss.

Was I conscious when they brought me in, and then forgot about it, or was I just acting out some unconscious mental distress?

\section{Partial Consciousness}

Consciousness is not all or nothing.
When we are not actually unconscious, our consciousness is selective.
We are conscious of only those aspects of our experience to which we are \emph{paying attention}, and arguably not all of that.
We can perhaps be `paying attention' to things which we are not {\it consciously} considering until they actually happen.
A mother may be continually on alert for signs that her child is distressed even when her conscious attention is focussed on some other matter.

I drive, and have a pretty reasonable safety record.
I don't often drive while having a conversation with passengers (or on the phone), and that's something I don't do in a wholly satisfactory way.
In that case, my driving doesn't seem to be any less safe, but my navigation can go completely awry.

More commonly, I do drive and think about other things.
In this case neither the safety nor the navigation seems to be compromised, but my memory of these is compromised.
I have driven on ``autopilot'', and I have little or no memory of the details.
It is tempting to say that it way my unconscious mind which did the driving while my conscious mind was thinking about something else.

There is other anecdotal evidence suggesting that the unconscious mind is no less capable of demanding intellectual feats than a conscious mind.
Famous mathematicians have reported that, having begun in earnest to crack a difficult mathematical problem, the solution appears to them in a dream, or after waking up from a nights sleep.
The anecdotes suggest that once a train of thought has been set in motion by the conscious mind, it can be carried through to conclusion sub-consciously.

The tendency of these anecdotes, is to suggest that the differences between conscious and unconscious thinking are perhaps not very substantial.
But the whole point of this essay is to put forward a suggestion about what the key difference is between these two.
Before doing that I will talk about something which has little to do with consciousness, and then put foward an analogy.

\section{Software Development}

\section{Musical Performance}

\section{Levels of Learning}

\section{Evolution of Memory}

\cite{murray2019evolutionary}

\phantomsection
\addcontentsline{toc}{section}{Bibliography}
\bibliographystyle{rbjfmu}
\bibliography{rbj}

%\addcontentsline{toc}{section}{Index}\label{index}
%{\twocolumn[]
%{\small\printindex}}

%\vfill

%\tiny{
%Started 2021/05/21


%\href{http://www.rbjones.com/rbjpub/www/papers/p041.pdf}{http://www.rbjones.com/rbjpub/www/papers/p041.pdf}

%}%tiny

\end{document}

% LocalWords:
