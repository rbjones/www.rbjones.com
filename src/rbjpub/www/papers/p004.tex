% $Id: p004.tex,v 1.2 2005/04/09 14:00:53 rbj Exp $
\documentclass{rbjk}

\begin{document}                                                                                   
\begin{article}
\begin{opening}  
\title{ProofPower}
\runningtitle{ProofPower}
\author{Roger Bishop \surname{Jones}}
\runningauthor{Roger Bishop Jones}
%\runningtitle{}

\begin{abstract}
A description of the specification and proof tool ProofPower.
\end{abstract}
\end{opening}

\tableofcontents

\section{Introduction}

ProofPower is a tool supporting the application of formal mathematical
modelling using the specification languages HOL, Z and others.
It began as a commercial software product developed by International
Computers Limited as part of a collaborative research and development
project partly funded by the Information Engineering Directorate of the
Department of Trade and Industry.
The project ran for three years from 1990 through 1992, but the
development has continued since then under a variety of other funding
arrangements.
Much of the later work has been funded by the Royal Signals Researach
Establishment and various sucessors (now QinetiQ).
The software is now owned by Lemma1 limited.

Originally ProofPower was aimed at supporting the application of
formal methods to the development of highly secure computer systems,
but latterly most developments have been oriented toward safety
critical applications, though none of its features is specific to
these application domains.

The design and implementation of ProofPower drew on an impressive
pedigree of achievement in the fields of Mathematical Logic, Language
Design and Automation of Reasoning, and the characteristics of the
tool may best be understood in the first instance through an account
of its pedigree.

\section{Pedigree}

ProofPower is a proof assistant for Higher Order Logic which supports
other formal notations, some by semantic embedding, some by other
methods.
Other languages for which some actual support is available include Z,
SPARK and the QinetiQ compliance notation, but the design of
ProofPower is sympathetic to extension to support additional notations
or languages via semantic embedding and supports mixed language working.

\subsection{Higher Order Logic}

The primitive logic supported by ProofPower is a variant of Higher
Order Logic.
This logic is a direct descendant of the Theory of Types\cite{russell08} which
Bertrand Russell devised for the formal derivation of mathematics
which he undertook jointly with A.N.Whitehead and published as
Principia Mathematica between 1910 and 1913\cite{whitehead10}.
Russell's original type theory was designed to provide a logical
system sufficient for the derivation of mathematics by conservative
extension, with a particular concern to avoid the antinomies which had
recently been found problematic in a philosophically satisfactory
manner.

Russell had sought to impose restrictions for which a good rationale
could be offered, rather avoid the contradictions by arbitrary
constraints for which no convincing rationale could be offered.
Unfortunately his rationale, based on proscription of ``vicious
circularity'' lead to an unacceptably weak (predicative) type theory
and he was forced to rescue it with an appalling hack called the axiom
of reducibility.
Ramsey later observed that an equivalent effect could be obtained in a
simpler manner by dropping the so-called ramifications in Russell's
theory after which the difficult to swallow axiom of reducibility was
no longer needed.
This gave the simple theory of types, which is now called higher order
logic, or more precisely, $\omega$-order logic, indicating that it has
all finite orders but no infinite orders.

The simple theory of types was then given an elegant reformulation
using the typed lambda calculus by Alonzo Church\cite{church40}.

The final stage in reaching the particular variant of Higher Order
Logic supported by ProofPower was accomplished by Mike Gordon at
Cambridge University, when he adapted the language for use in a proof
assistant derived from Cambridge LCF.
The adaptations which tool place at this stage were partly the kind of
adaptations which are appropriate to a logic when it crossed from
being an object of study by mathematicians to one applied by computer
scientists, and partly logically insignificant changes which made
exploitation of the code base already implemented for Scott's Logic
for Computable Functions.
Most of the former were also inherited from LCF.

Taking these in turn, the features which are arguably essential for
practical applications included:

\begin{description}
\item[polymorphism]
In Church\cite{church40} type variables appear only in the
meta-language.
In practice it is highly desirable to have them in the object
language.
This involves a small change to the grammer of types, and a new rule
for type instantiation.
\item[type constructors]
Church had two primitive types (individuals and propositions) and one
type constructor (function space constructor).
HOL has the same primitives but allows additional type constructors to
be introduced, type constants are treated as 0-ary constructors, and
provision is made for introducing type constructors of any arity by
conservative means.
\item[constants]
Several mechanisms are introduced in HOL for introducing new constants
by means which are guaranteed conservative.
The set of primitive constants in HOL (equality, implication, choice)
differs from that in STT (negation, disjunction, universal
quantification, description).
This naturally affects the axioms of the logic, but it is easy to see
that the resulting theories are the same (the axioms of HOL are
provable in STT and vice versa).

\end{description}

\subsection{The Z Specification Language}

The Z specification language was developed at the University of
Oxford.
The language is based on set theory, in particular on the
axiomatisation of set theory due to Zermelo \cite{zermelo08}, which
was subsequently enhanced to yield the theory ZFC.

The enhancements to Zermelo set theory en-route to the Z specification
language include:
\begin{enumerate}
\item an explicit axiom of regularity asserting that all sets are
  well-founded.
\item sharpenening of the notion of definite property used in
  forming new sets by separation (this sharpening arises from
  formlisation, zermelo's original set theory is not a formal theory)
\item the addition of a type-system together with a kind of
  polymorphism in the form of set-generic specifications
  (i.e. specifications which are parameterised by arbitrary sets)
\item the introduction of labelled products and the elaborate use of
  these labelled products for modelling relations (which can represent
  the behaviour of computer systems), and which serve as abbreviations
  for complete signatures (collections of variables together with
  constraints on those variables) and predicates (properties of signatures).
\end{enumerate}

The Z specification language introduces a rich syntax for a typed set
theory which makes it much more suitable for use as a specification
language in the development of computer systems.

\subsection{Implementation}

The implementation follows a novel paradigm introduced in the
Edinburgh LCF system and subsequently adopted by several direct
descendents of that system and also by other implementations of proof
assistants round the world.
This is called the {\it LCF paradigm}.

The implementation is in the language Standard ML (SML), which is a
modern version of the language ML first devised for the Edinburgh LCF
system (there are several other modern variants of ML).
This language is intended not only for use in implementing the proof
assistant, but also as a language for interaction between the tool and
its users.

\subsection{The LCF Paradigm}

The main features of the LCF paradigm were first used in the Edinburgh
LCF system, a second attempt, lead by Robin Milner, at implementing a
proof assistant for Dana Scott's Logic for Computable Functions.
The first attempt was Stanford LCF, and the main characteristics of
LCF were therefore arrived at by Milner in the light of experience
with the implementation in LISP, and application, of a proof assistant
for LCF.

Already in Stanford LCF (1972), the idea of using an (impure) functional
language for interaction with a proof assistant was exploited.
The first step in developing this idea for Edinburgh LCF was to use a
typed (impure) functional language and to use an abstract data type to
guarantee that any computation of a theorem would yeild a result
derivable in the logic.
The experience of Stanford LCF, which constructed and retained actual
proofs was that these became very large, but were there only to be
checked.
The use of an abstract data type permitted the checking to be built
in, and made the retention of the proof superfluous, so the LCF
paradigm effectively treats a computation as a proof.

The replacement of LISP as an implementation language by a typed
functional language created obvious difficulties in relation to the
implementation of general list processing facilities.
The power of LISP in part arose from the power and flexibility of
general list processing functions in a type-free context.
Typed programming languages were either poor in support for generic
facilities (languages such as COBOL, FORTRAN, even Algol), or had very
complex type system such as that for Algol68.
The polymorphism adopted in ML was a startlingly simple way of adding
types while retaining the ability to write list processing functions
which could operate over all types of list.
It created an effective compromised between the complexity
of then modern languages such as Algol68 and the simple but crude
power of LISP.
It also also permitted another starting compromise, types and type
checking without type declarations.
Functions could be defined without stating their type, and their most
general polymorphic type could then be computed automatically.

\subsection{Standard ML}

Subsequent to the design of the original ML for the Edinburgh LCF
system considerable development to typed functional programming
languages took place, including for example the introduction of 
pattern matching function definitions.

There was by this time a significant community of users of ML, as well
as several variants of the language, which wanted to benefit from the
state of the art in functional programming without entirely leaving ML
behind.

An international effort was initiated to re-design ML an create a new
standard for impure typed functional programming.

\section{Functionality}

What does ProofPower do?

It provides:
\begin{itemize}
\item Document preparation facilities (using \LaTeX) for \LaTeX
  documents containing near wysiwyg formal materials.
\item Syntax and type checking of specificatons in HOL and Z.
\item Management of a theory database in which the details of formal
  specifications and the results proved about them are stored.
\item Facilities for computing theorems in specific logical contexts
  (positions in the theory heirarchy), by means which reliably check
  formal derivability of the theorems in the relevant context.
\end{itemize}

With the supplementary DAZ facility the capability of ProofPower is
extended to support of SPARK and the QinetiQ compliance notation which
supports verified refinement of Z specifications into SPARK.

A tool, CLAWZ, is also available which translated models in Simulink
(a graphical modelling tool associated with the Matlab mathematical
software).

ProofPower is essentially a tool which assists in the construction and
checking of formal proofs in Higher Order Logic, and in various other
languages which can conveniently be interpreted in Higher Order Logic.

The idea, following Russell's conception of the foundations of
mathematics, is to utilise a ``logic'' sufficiently strong that the
concepts of mathematics, or of application domains which are suitable
to be modelled mathematically, are introduced by definition (or,
slightly more liberally, by conservative extension).

Once these concepts have been introduced the tool will support the
construction and checking of mathematical proofs of propositions
involving them.
The mathematics required for modelling information systems using these
languages is not trivial, and the specifications of the systems (which
consist of large numbers of often complex definitions) can be large.
All these must be logically in context before a proof can be
undertaken, and the proof is invariably primarily a work of human
ingenuity, in which the more trivial labour, and the careful checking
for correctness, are undertaken by ProofPower.

For this kind of work special document preparation facilities are
required which can cope with the exotic notations employed in such a
way that the formal specification can not only be printed, but can
also be processed by ProofPower to establish a context in which proofs
can be conducted.

It is essential in practice that the definitions assembled together
for the specification and proof can be structured suitably, and for
this purpose ProofPower maintains a theory heirarchy.
A theory is a bit like a module of specification, and a dependency
heirachy exists between the theories in which a theory A which makes
use of definitions introduced in theory B is a descendent of B, and
A an ancestor of B.

Theories include or have associate with them the following kinds of information:

\begin{description}
\item[Definitions]
of new types, type constructors or constants.
All definitions must be saved in a theory before they can be used.
\item[Theorems]
some of theorems which have been proven in the context of the definitions in
the theory (and its ancestors).
The storage of theorems in theories is entirely at the discretion of
the user.
Theorems are in any case a special type of value which includes
contextual information which determines the scope in which the theorem
is derivable and can be used for further proofs.
\item[Proof Contexts]
which are bundles of resources for theorem proving and are associated
with a particular theory,
Typically they will contain theorems from the theory and its
ancestors which are likely to be useful in proving further results,
and conversions which are functions in ML capable of aspects of proof
automation going beyond what can be captured in a theorem, e.g. which
can determine and prove the result of an arithmetic computation and
return a theorem which captures that result in an equation
(e.g. $\vdash 2+2=4$) suitable for simplifying a proof goal by
rewriting. 
\end{description}

\section{ProofPower's Innovations}

International Computers Limited had a need for a proof tool supporting
formal reasoning about specifications written in the Z specification
language.
There was at that time no established logic for the language, though
the semantics for the main features of the language had been defined
by Spivey in his doctoral dissertation \cite{spivey88}.

ICL already had acquired experience of reasoning about specification
in Z by manual transcription into HOL using the Cambridge HOL proof
assistant.
In principle it seemed probable that a properly engineered semantic
embedding of Z into HOL would yield an effective proof tool.
It was thought however, that to base a commercial product on software
which was the product of ongoing academic research would not be a good
idea.
Changes to Cambridge HOL would not reflect the requirements for
stability for a commercial software product.
In addition ICL had an interest in acquiring skills in the development
of proof technology, and a re-engineering of HOL was one way to
achieve that.

The proposed re-engineering was therefore not intended to be an
innovative undertaking.
It was inteded to yield software products which were suitable for
certification permitting their use in the most demanding high security
developments.
However, the need to support proof in Z imposed some novely in the
development, and most of the novel features in the core ProofPower
system were introduced, either to make it credentials for high
assurance work more conspicuous or to make support for Z possible.

Among these novelties are:

\begin{description}
\item[Generic Multilingual Quoting]
The Cambridge HOL system has facilities for parsing and pretty
printing the object language HOL in which proofs are conducted.
ProofPower has some generic support for embedding arbitrary languages
into HOL, allowing quotations to be tagged with a language identifier,
so that the underlying HOL terms could be entered or printed using
various different concrete syntaxes.
A single term in HOL can be presented as a hybrid quotation involving
multiple languages.
\item[Context Sensitive Proof Facilities]
It proved essential for smooth proof in Z that even the most basic
proof facilities could be made sensitive to the principle language in
which the proof was being conducted.
For example, the same universal quantifier is used both in pure HOL
and also in Z embeeded into HOL, but the best way to handle the
quantifier in a proof depends upon what language is in use.
One reason for this is that though any proof step which is legal in
HOL will also be legal for embedded Z, (such as quantifier
elimination) some sound proof steps will take the user from a subgoal
which is in Z (i.e. a HOL term which is the image under the embedding
of some Z term or formula), to a term which is good HOL, but not in
the image of the embedding, taking the proof out of the Z language.

Though introduced to permit proofs in Z to stay in Z, the context
sensitive facilities (which were made sensitive to things called
``proof contexts'') had wider applicability.
The enable all aspects of the systems proof capabilities to be
enhanced for support of the various theories developed.


\end{description}


{\raggedright
\bibliographystyle{klunamed}
\bibliography{rbj,fmu}
} %\raggedright


\end{article}
\end{document}
