% $Id: p014.tex,v 1.2 2009/05/04 21:22:18 rbj Exp $ bibref{rbjp014} pdfname{p014} 
\documentclass[numreferences]{rbjk}
\usepackage{makeidx}
\newcommand{\ignore}[1]{}
\usepackage[unicode,pdftex]{hyperref}
\hypersetup{pdfauthor={Roger Bishop Jones}}
\hypersetup{colorlinks=true, urlcolor=black, citecolor=black, filecolor=black, linkcolor=black}

\def\indexed#1{#1\index{#1}}
\def\wiki#1{#1\index{#1} \footnote{\href{http://wikipedia.org/wiki/#1}{http://wikpedia.org/wiki/#1}}}
\def\Wiki#1{\footnote{\href{http://wikipedia.org/wiki/#1}{http://wikpedia.org/wiki/#1}}}
%\newtheorem{def}{Definition}
%\newtheorem{conj}{Conjecture}

\makeindex
\begin{document}                                                                                   
\begin{article}
\begin{opening}  
\title{Anarchism and Tao}
%\runningtitle{Tao Eviscerated}
\author{Roger Bishop \surname{Jones}}
\date{$ $\ $ $}
%\runningauthor{Roger Bishop Jones}

\begin{abstract}
A discussion of the relationship between these two different systems of ideas leading to a synthesis.
\end{abstract}

\end{opening}

\vfill

\begin{centering}
\footnotesize{
Created 2008/07/23

Last Change $ $Date: 2009/05/04 21:22:18 $ $

\href{http://www.rbjones.com/rbjpub/www/papers/p014.pdf}{http://www.rbjones.com/rbjpub/www/papers/p014.pdf}

$ $Id: p014.tex,v 1.2 2009/05/04 21:22:18 rbj Exp $ $\\

}%footnotesize
\end{centering}

\newpage
%\def\tableofcontents{{\parskip=0pt\@starttoc{toc}}}
\setcounter{tocdepth}{4}
{\parskip-0pt\tableofcontents}

\section{Introduction}

On reading about the ancient Chinese philosophy (written in English as) {\it Tao} or {\it Dao} I found myself sympathetic to what I took to be some of its core elements.
The rest was variously incomprehensible or disagreeable.

The purpose of this essay is to sketch the parts I like and to say a few words about why I don't like the other bits.

This is not a work of scholarship.
I have read little about the {\it Tao} my principle sources are:

\begin{itemize}
\item The Tao is Silent\cite{smullyan77}, by Raymond Smullyan (a professional mathematical logician)
\item Tao Te Ching, Lao-Tzu (in various translations)
\item five articles by Ted Kardash at {\tt jadedragon.com} covering \indexed{\it te}, \indexed{\it yin-yang}, \indexed{\it wu-wei} and \indexed{\it the sage}. 
\end{itemize}

I do here what is rarely done.
I take a well established ideology, to which one is expected to refer deferentially for wisdom, and I treat it like a high school essay, poke it around, offer an off-the-cuff critique and a sketch of how some of the key ideas can be extracted and improved upon.
The elements of {\it The Tao} which react against such treatment are among those which I deprecate, many of which are by-products of institutionalisation, which may itself be essential to the propagation of the ideology.

To make matters worse, its may be worth observing that of the above sources the most substantial (in length) is that by \index{Smullyan, Raymond} Smullyan, who as a professional logician living and working in the USA will be regarded I would guess by most Tao-ists as a very poor source of authentic wisdom on Tao.
However I too am a western mathematical logician (inter alia, though not professional) and so I find Smullyan relatively easy to swallow.

\section{Institutionalisation}

\index{institutionalisation}
The idea of the {\it meme} is useful here, because it draws out how the survival and proliferation of ideas or ideologies depends not exclusively on the intrinsic merits of the ideas, but also on whether the actions to which the ideas provoke their hosts are conducive to the propagation of the ideas.

To chose a simple but pertinent example, a belief system which causes its adherents:
\begin{itemize}
\item to carefully indoctrinate their children with the ideas from a very early age, and
\item to refrain from birth control
\end{itemize}
has in these features done much to ensure its survival, however little other merit there may be in the ideas.

An important survival strategy for an idealogy is to underpin an institution whose purpose is to propagate the ideology.
Such an institution provides for a class of people (a priesthood perhaps), whose life is largely or wholly dedicated to spreading the word and doing (or causing others to do) whatever else the ideology advocates.
The survival value of institutionalisation is so great that whatever the initial ideas, a belief system which has survived for any decent period of time (a millennium, say) is likely to have mutated and evolved to support the institutions which have based themselves around it. 
Some of the changes which arise in this way may be inconsistent with and amount to outright corruptions of the original ideology.
Some kinds of feature are not conducive to institutionalisation, and are therefore unlikely to appear in mature systems.

Examples of such features include {\it simplicity} and {\it transparency}.
An ideology which is simple and transparent may possibly survive on its own merits, but it is unlikely to become institutionalised, and will be continually undermined by institutions which claim to have complex and difficult approaches to the same matters.

I am not myself familiar with the institutional context which surrounds the Tao.
Perhaps there have been Monasteries in which Tao Master's have passed on their wisdom to disciples, or perhaps the lack of such institutions has lead to the incorporation of the Tao into Zen Buddhism resulting in a more elaborate ideology in which important parts of the Tao are more difficult to locate, but which benefits from a stronger institutional context.

\section{De-Institutionalisation}

The advertised ``evisceration'' consists primarily in stripping out features of which the principle merits seem to be institutional, and some moderation of features which might have been perverted for similar reasons.

For this purpose I will describe the features which it seems to me may have arisen in such ways.

These include:

\begin{itemize}
\item Metaphysical and Mystical elements
\item Excess complexity and obscurity
\item The status of the Tao Master or sage
\end{itemize}

By adopting such radical excisions I am seeking to separate out that part of the Tao which provides useful advice for us all on how to live our lives.

Of course, Tao devotees will tell us that these are of the essence and that nothing of value remains.
You must judge for yourself.

\subsection{Metaphysical and Mystical Elements}

By the metaphysical I mean talking of ``The Tao'' as if it were some thing, with which we are to seek harmony and unity.
Writings on the Tao are ambivalent on whether ``The Tao'' does exist, so this is a point of obscurity.
It is consistent with the writings I am acquainted with to suppose that when Lao-Tzu speaks of the Tao as if it does exist, he is simply using this as a form of words convenient for getting his message across, not to be taken literally.

Thus it may be that the important advice is to do those things which are supposed lead to harmony with The Tao, which will have its desirable effects on your life whether or not ``The Tao'' really does exist.

When I propose then, that the metaphysical elements be discarded, I am not suggesting that all those things said by reference to metaphysical entities should be discarded, but rather that they should be re-expressed without the metaphysics.
I have not attempted this, and it may be that the metaphysics really does make it easier to deliver the message, in which case it would be better, rather than stripping out the metaphysics simply to add a codicil or prefix disavowing the literal interpretation.

How would this differ from the status quo?
Well it seems to be presented as a real issue worthy of some discussion by {\it Smullyan}, and I am inclined to doubt that it is.

That I discuss it here indicates that it seems to be possibly a by-product of institutionalisation.
The role of the metaphysical in religious institutions is to prove an unimpeachable source of true wisdom which may not be readily approached by the novice.
To obtain this unimpeachable wisdom the novice is expected to sit at the feet of the Master (though the relationship may be more complex in Tao or Zen).
The requirement that access to the source be mediated by professionals provides them with their {\it raison d'\^{e}tre} as professionals and justifies their place in an institution which gives them a living.

As to the mysticism, I'm afraid I am even less well qualified to speak, since I am almost blind to it.
Insofar as I understand the mystical side, I believe this to be principally the idea that one should seek enlightenment through unity with the Tao.
This sounds to me like a step beyond using {\it The Tao} as a metaphysical entity to express some ideas which might otherwise be hard to get across.
This unity or communion is I suspect, not just a question of behaving in certain ways, but also of feeling in certain ways, or experiencing a certain state of mind.
I am not sympathetic to this, which is not to say that I consider feelings and states of mind as unimportant, but that I consider feelings or states of mind about or in relation to this supposed metaphysical entity, especially when elevated to be of supreme importance, seem to me to be a bad thing.
I think we should work out for ourselves what is important and that the important things are things in this world, like the people we love, or the improvement of the world they live in, and that our most important feelings and mental states should bear upon these things which we find to be most important.
Mysticism seems to me (in my ignorance) something intended to divert people from addressing real world problems or objectives in such a way as to make a difference, and hence of maintaining the status quo.
Religion is the opium of the masses it is said, keeps them content with their lot.  

\subsection{Excess Complexity and Obscurity}

Tao and Zen literature is full of tales of Masters responding to acolytes by obscure physical gestures.
I have no idea what these are supposed to mean.
I think that is the intention.

The idea is that you spend many years studying under a master, and that somehow or other eventually you find enlightenment.
The fact that so much of the learning experience comes not by words of wisdom from the master (though doubtless that plays a role too) but in much more subtle and obscure ways, underpins the institution.
You are not supposed to believe that you can achieve enlightenment by leading a life with the benefit of some helpful advice which might be found in a book.

Complexity and obscurity makes the Master essential, and makes it worthwhile for some others to devote their life to seeking enlightenment at the feet of the Master.

However, it seems to me that at the core of Tao are some ideas which may be helpful to us all, ones appreciation of these might develop through an extended period (a lifetime even) but the starting point is not so hard to sketch out, and not even so very hard to work out for yourself.
I believe that we should each chose our own life, and that most of us will not want to be religious professionals, and that those of us who do not are just as (if not more) likely to do well (in the fullest sense, but in terms we chose for ourselves) as those who do.
Complexity and obscurity is there for the benefit of the professionals and should be excised.

\subsection{The Status of The Tao Master or Sage}

I have already touched a little upon this, you may have anticipated that the exalted status of the Master serves the purpose of the institution by justifying his professional status and enabling him to work full time for the institution.  

There are other aspects of this which I dislike.

The basic idea that those who excel (in any field) are worth paying attention to is unproblematic (though I'm pretty bad myself at learning from my betters).
My own inclination is to treat no-one with undue reverence, the fact that someone does very well in one area does not mean he will do well in other areas.
One comes to understand peoples strengths and weaknesses, and to understand in what matters they can be taken on trust, and we must make our own judgements about whose view should be given greater weight in matters where our own competence does not suffice.

Masters are not supposed to be considered in this way.
We have not the competence to pick and chose when to take them seriously, this is not what a disciple does. 

\section{Elements of The Eviscerated Tao}

On the positive side I now say a few words about what remains.
This is a crude re-casting, with some discussion and qualification, of the main elements of {\it Tao}.
I follow the topics used by Kardash in his account, but first I 

\subsection{Personal Anarchism}

In my youth I sought guiding principles for my own life, and settled for my most fundamental principle on something which I then chose to call {\it personal anarchism} and which has something in common with the notion of {\it Wu-Wei}, and perhaps also with other aspects of Tao.
The connection more generally between anarchism and Tao has been recognised by anarchists, who view the Tao as a precursor.
At the time when I formulated my own ``personal anarchism'' I was acquainted with some of the anarchist literature, and probably also with Zen Buddhism (which incorporates aspects of Tao), but not specifically with the Tao, acquaintance with which dates some 30 years later.

This is what I had before coming across the Tao, and for me the Tao is some ideas which make sense to me in relation to this personal philosophy, and which to an extent which is not clear to me, elaborate and advance that philosophy.
I remain this personal anarchist, and this personal anarchism when viewed through the language of Tao is what I here misdescribe as ``the Tao eviscerated'', so the evisceration is not just the removal of things I don't like or don't understand, but also the interpretation and possibly augmentation from a different perspective.

\subsection{The Tao}

Though I don't care for the metaphysical or mystical aspects of Tao, I don't believe these can simply be discarded without loss.
Sometimes talk of the Tao is a bit like talking about the Universe, and I am inclined to replace it with talk about our environment, that part of the universe with which we have a meaningful relationship.
When a Tao Sage speaks of ``connectedness, harmony and balance, union with the Tao'' I am inclined to think in terms of awareness of context, of the way our behaviour impacts upon it. and the idea of working with and through rather than against and despite.

I don't see how this can amount, in the eviscerated Tao to as important and central an idea as ``the Tao'' is in the real Tao.

\subsection{Te}

``Te'' means ``virtue'' and stands in the Tao for that kind of virtue which is suggested for the individual.

I'm happy to have something like this centre stage, but my conception of this virtue is not wholly aligned with what I find in the Tao.

I think that our well being depends upon our having a strong inner self, a sense of identity, of self-worth, and even of special place (in the scheme of things) which keeps us on a steady path in the  face of life's vicissitudes.

In my scheme this comes from the habit of self trust.
If you trust to your instincts and feelings, do what comes naturally (after due consideration), and time and time again this does not cause the universe to collapse in on you, then you create a context in which your own self confidence can grow.

Now Lao-Tze advocates {\it compassion}\index{compassion}, {\it humility}\index{humility} and {\it frugality}\index{frugality}.
Which I have a hard time making sense of in this context, but let me work on it a bit.

Now the central principle in personal anarchism may be said to be {\it self-trust} rather than {\it self-discipline}.
This is a kind of {\it self-love}, and it engenders inner harmony, a kind of unity.
Rather than choosing what must be done, and then trying to force oneself to do it, one allows oneself to do what feels right at the time.
If you are going to do that, then the deliberative part of the mind, which might otherwise be putting the plans into place, must operate in a different way.
It can still do most of what should be involved in the planning activity, i.e. in guestimating the consequences of various possible courses of action, and performing some kind of evaluation on the possible outcomes.
Even without the final stage of deciding between the possible courses this is a useful, possibly essential, actually hard to avoid, activity, and, even without the definite decision it will influence what we are likely to do ``spontaneously'' (in our sense of that word, which does not connote ``thoughtless'' or ``uninformed'').

In talking this way we are talking as if the self were a community, as if different parts of our psyche related to each other in ways analogous to the ways in which individuals in society relate to each other.
This {\it personal anarchism} is the point of departure for a more general anarchistic principle which informs the relationship between individuals and groups of individuals.
This anarchism is based on building trust in others, and may be seen as a kind of compassion.

\subsection{Wu-Wei}

The concept of {\it Wu-Wei}\index{wu-wei} connects with my own personal anarchism, which share an appreciation of a kind of spontaneity informed by and contributing to an inner unity and a harmonious relationship with the outer world.

{\it Wu-wei} has several different aspects.

\subsection{Yin-Yang}


\subsection{The Sage}



%{\raggedright
%\bibliographystyle{klunum}
%\bibliography{rbjk}
%} %\raggedright

\twocolumn[\section{Index}\label{Index}]
{\small\printindex}

\end{article}
\end{document}
