% $Id: p001.tex,v 1.2 2004/04/26 15:38:04 rbj Exp $
\documentclass{rbjk}

\newdisplay{guess}{Conjecture}
\newdisplay{prop}{Proposition}
\newcommand{\ignore}[1]{}

\begin{document}                                                                                   
\begin{article}
\begin{opening}  
\title{Semantic Foundations for Deductive Methods}
\subtitle{an essay in the delineation of analyticity}
\runningtitle{Semantic Foundations for Deductive Methods}
\author{Roger Bishop \surname{Jones}}
\runningauthor{Roger Jones}
%\runningtitle{}

\begin{abstract}

The scope of deductive reason is considered, in principle and in practice.
This is done in three stages.
First the relation between deductive reason and analytic truth is discussed.
Next a connection is made between analyticity and set theoretic truth, suggesting both that analytic truths can be read as truths of set theory in various syntactic clothes and that the truths of set theory are themselves analytic.
The semantics of set theory is then considered, with a view to making as precise as possible the extension of the concept of set theoretic truth, thereby ircumscribing the scope of deductive reason.

Alongside these theoretical considerations, the pragmatics of application of deduction, particularly in securing confidence in the development of information systems, are born in mind.
These considerations provide motivation for foundational studies and a moderating context in which foundational skepticism is evaluated.

\end{abstract}
\end{opening}

\section{Introduction}

It is my purpose in this paper both to consider the scope of deductive methods, and to further the precise determination in its broadest sense of what constitutes a sound deduction.
Matters of both principle and of practice are considered, and motivation is supported by a lightweight survey of some relevant developments in philosophy, logic, and the application of formal methods in science and engineering.

\ignore{

\cite{MacKensie}

\subsection{Some Background}


\subsubsection{David Hume}

Let us begin with some words from David Hume, who wrote:

\begin{quote}
 ALL the objects of human reason or enquiry may naturally be divided into two kinds, to wit, Relations of Ideas, and Matters of Fact. Of the first kind are the sciences of Geometry, Algebra, and Arithmetic; and in short, every affirmation which is either intuitively or demonstratively certain.

\ldots

 Matters of fact, which are the second objects of human reason, are not ascertained in the same manner; nor is our evidence of their truth, however great, of a like nature with the foregoing.

{\it An Enquiry Concerning Human Understanding}, Section IV Part I
\end{quote}

This distinction informs Hume's skepticism, through the details of which we see the precision with which Hume is able to draw his line, between the necessary and the contingent, the analytic and the synthetic.

\subsubsection{Gottfried Leibniz}

Before Hume, Leibniz had distinguished between truths of reason and truths of fact.

Leibniz is of interest to us here, not for the precision with which he drew this line, but for his views on the scope and utility formality and in the possibility of mechanisation.

Leibniz sought a {\it lingua characteristica}, a universal language supported by a {\it calculus ratiocinator} which permitted all questions to be answered by computation.
He also recognised that to perform these computations rapidly and reliably calculating machines were desirable.

That Leibniz distinguished between truths of reason and truths of fact might lead us to expect that he would not expect a mechanisation of deductive logic to yield a universal calculus.
Leibniz did consider that for omniscient deities, by contrast with mere mortals, all truths, including matters of fact, are necessary, and so fall within the scope of a suitable calculus.
It is hard to see however, how he could expect any human to arrive at the calculus or construct a universal calculating machine.

Unreal though these speculations may now appear, these considerations are relevant today to an understanding of the scope of deductive reason and the potential applications of computing machinery.

\subsubsection{The Development of Modern Logic}

Leibniz's ideas were ahead of his time, the formal logic which he had at his disposal was not adequate even for the derivation of mathematics let alone for his calculus ratiocinator.
During the 19th century formal logic was developed to the point at which formal derivation of mathematics became feasible, culminating in the publication of {\it Principia Mathematica} in which Russell and Whitehead formally derived large parts of mathematics using Russell's {\it Theory of Types}.

Published in the same year as Russell's logic \cite{Russell08}

\subsubsection{Russell and Carnap}

\subsection{Digital Computers and Issues of Trust}

\subsection{Skepticism about Foundations}

`Foundationalism' in mathematics is often the target of skeptical arguments.
A typical observation often thought to vitiate the adoption of foundational systems is that when scrutinised these system often appear to be more tenuous than the mathematics for which they offer foundations.

Two points are offered here in response to such skepticism.

The first is that this betrays a misconception of the nature of foundations in general.
A foundation is to provide a solid base on which something else may be constructed.
It suffices that it provides a solid base, even if it is not itself impeccably supported.
Foundations come at the bottom.
Of all else we may ask, on what is it based, and expect to be answered.
On a foundation we build in the expectation of support, that the structure stands is sufficient vindication of the foundations.

The point may be illustrated by reference to the use of piles to provide a foundation for buildings in boggy land.
Here a solid base is provided where there is nothing solid to rest upon.
We may build on water, using a foundation which is rigid and buoyant but which itself rests on nothing solid.

In defining the semantics of languages there is of course a problem of foundational regress.
We may doubt the value of giving the semantics of a language using that same language, but if we use some other language then we are in need of an account of the semantics of that language.
Despite these difficulties it is in practice possible to define the semantics of formal languages with very high levels of precision, and skeptical arguments should not be allowed to persuade us that this  is not a worthwhile enterprise.
More specific responses to this problem of semantic regress are considered in relation to set theory below.

}

\section{Deduction and Analyticity}

\ignore{
I propose lightly to sketch in this section a connection between the scope of deductive reasoning and the concept of analyticity.

The discussion relates exclusively to the relation between deductive reasoning in formal deductive systems which are sound relative to an appropriate semantics, and the concept of analyticity as applied to sentences in formal languages with similar semantics.

First let me mention some generally accepted desiderata for formal deductive systems.
Except in the (relatively rare) case where a deductive system is considered purely as a formal calculus, and is intended to have no interpretation, a formal deductive system when constructed is given a semantics and is proven to be consistent with the semantics.
If the deductive system is complete with respect to that semantics then this would also be proven.
Whether or not these proofs are undertaken, it is a generally accepted desiderata for deductive systems that they are sound.

I will expand upon these notions shortly, but first I state the elementary connection which is here alleged.
The conclusions of proofs in sound formal deductive systems are the formal analytic truths (i.e. the set of sentences in formal languages which are analytic).
Thus the concept of analyticity precisely circumscribes the scope of sound deductive reasoning, at least in the realm of formal languages.

Now to make these concepts a little more precise we need to speak of semantics.
A {\it truth conditional semantics} for a formal language L is function which gives the truth conditions for sentences in the language.
i.e. given a sentence, and some other parameters on which the truth of sentences may depend but which may vary from one language to the next, the function yields the truth value of the sentence.
The term sentence here is used in a more general sense than that in first order logic, where it refers to a closed formula.
In this context exactly what a sentence is depends upon the particular language, but it will always be the syntactic category of those things in the language which have truth values.

The function which constitutes the semantics will be defined over a definite domain of sentences and parameters, and the parameters may be thought of as constituting the subject matter of the language.
For a language intended to talk about the material world, the parameters will include a possible world.
For a language intended to talk about groups, one of the parameters will be the group in question.
Languages whose sentences may contain free variables will expect parameters which determine the values of the free variables, and which are permitted to range over the domain of discourse.

A sentence is formally analytic if the truth function yields ``true'' when applied to that sentence, {\it whatever the value of the other parameters} provided only that the truth function is defined for those parameter values.

A formal deductive system is sound if there is a truth conditional semantics given for the language of the deductive system, and the following conditions hold:

\begin{enumerate}
\item The axioms are all true under the semantics, i.e. are formally analytic
\item The inference rules of the system preserve truth under the truth conditional semantics, i.e. for every set of parameter values for which the premises of the rule are true under the semantics, then the conclusion of the rule is also true.
\end{enumerate}

An elementary inductive argument leads us to conclude that all sentences which are provable in sound formal deductive systems are formally analytic.
For the other direction, any analytic sentence is derivable in the formal system which has that sentence as its sole axiom, and has no inference rules.

}

It is normal practice among those who devise formal deductive system to validate those systems by proving them sound.
The effect of this (subject to some caveats which we will address later) is to ensure a connection between the things which are provable in these systems and the concept of analyticity.

To make this connection conspicuous I will give definitions of relevant concepts here from which the alleged elementary connection is readily demonstrable.

I am concerned here exclusively with languages which have a well defined syntax and semantics, and I assume that the semantics for such a language (called an object language) has been defined in some other language (called the meta-language), and that the meta-language is equipped with a deductive system in which the consequences of the semantic definition can be derived.

I will use the word {\it statement} here simply to mean an ordered pair of which the first element is a sentence and the second a semantics for a language in that sentence occurs.

A statement will be called {\it demonstrative} if the sentence is derivable in a deductive system which is sound with respect to the semantics.

A deductive system is a set of sentences and a derivability relation, which is a relation between sets of sentences (the premises of an inference) and single sentences (the conclusion of an inference).

A semantics is an assignment of meaning to the sentences of some language which is of interest here only insofar as it yields information about truth conditions for sentences in the language.
In particular, for the weak definition of soundness which follows, the only part of the semantics which is of interest is which sentences of the language are true

A deductive system is sound with respect to a semantics if it preserves truth under the semantics, i.e. if whenever all the premises of a derivation are true under the semantics then so is the conclusion.

A sentence in some well-defined object language is {\it analytic} if its truth can be proved in the metalanguage from the semantics of the object language.

By an elementary induction on the length of proofs it follows that all the theorems of sound deductive systems are analytic.
Every analytic sentence is demonstrative since it is provable in the sound deductive system which has just one inference rule whose conclusion is that sentence.

Hence:
\prop{The concepts {\it analytic} and {\it demonstrative} (as defined) are coextensive.}

We note that in particular under these definitions, the theorems of set theory (say ZFC) are demonstrative and analytic, and hence the theorems of mathematics in general.

\subsection{Reasoning about the Contingent World}

Firstly it should break down where a formal calculus is devised for reasoning about the real world.
For in this case we would expect to use axioms which are not known to be true on the basis of the semantics of the language alone (i.e. which are not analytic), and to soundly derive conclusions whose truth is contingent.
Even in this case the metatheoretic methodology has some merit but must properly be described in different terms.
One important purpose served by the method is to establish the consistency of the logical system.
By showing that all the derivable theorems are {\it true} it is demonstrated that no contradiction is derivable in the system.
For this purpose it is desirable to {\it concoct} a semantics relative to which the system is sound, even if this semantics distorts the meaning of the sentences, for example, by restricting the domain of discourse.
Thus a deductive system which incorporates physical laws may be evaluated as if those physical laws were incorporated into the meaning of the language.
A second reason for gerrymandering semantics is in connection with completeness.
In this case a domain of discourse may be chosen which is broader than might otherwise have been expected, in order to obtain a completeness result for the deductive system.
An example of this is in the non-standard semantics of higher order logics, in which a broader class of interpretations is admitted, fewer inferences are sound, in particular eliminating any which do no correspond to formal derivations.

For deductive systems intended for deriving theorems about the real world the deductive status of the theorems may be considered in two ways, and the choice between these two is made in the semantics of the language.
Either the semantics is {\it on the nail} and contains no more than one would expect, some of the axioms are contingent and hence not true under the semantics, and consequently the deductive system is not sound and the results not demonstrative.
There is however a recoverable demonstrable result, which is obtained in the logical system with the contingent axioms removed, by taking required instances of these principles as hypotheses.
A second approach to applicability of deduction in applied theories is to treat the theory as providing an abstract model of the relevant aspect of reality.
An abstract semantics is supplied relative to which the deductive system, including the contingent axioms (not contingent under the abstract semantics), is sound.
The theorems of the system are then all true under the semantics, but their relevance to the real world is moot.
The correspondence between the abstract world of which the language speaks and the real world is then contingent, as is the intepretation of the theorems of the system in the real world under such a correspondence.

In this we see an echo of Leibniz's special treatment of God.
In his case truths of fact are necessary because of his special knowledge.
In our case claims, ostensibly about the real world, become necessary truths when the relevant features of the world are incorporated in the semantics of the language.
However, their necessity comes at the expense of their content, they no longer tell us about the world, except by contingent hypothesis.

\subsection{Incompleteness}

A second point of possible divergence between the scope of deduction and the concept of analyticity arises from the incompletability of formal deductive systems.

In his discussion of the concept of logical consequence \cite{Tarski36} Tarski argues from the validity of w-inference in the domain of natural numbers to the conclusion that logical consequence cannot be given a syntactic characterisation (i.e. there is no recursive deductive system in which logical consequence corresponds with derivability).
This commits Tarski to a notion of consequence broader than that of first order consequence, though Tarski stops short of accepting so broad a conception as to coincide with analyticity.

Though there is no deductive system which encompasses all sound derivations, there is no sound derivation which is countenanced by no deductive system.
Soundness is in this sense a least upper bound on the derivations encompassed by respectable deductive systems.
Furthermore, as we will argue later, there are formal deductive systems which are good pragmatic approximations to the full extent of sound derivation.

It is a {\it fait accompli} that the concepts of logical truth and logical consequence are now almost unversally construed in a much narrower way that that of demonstrable truth in Hume, a term no longer in much currency.
For a full appreciation of the scope of formal methods, the broader concept is the one of interest, which I have argued here corresponds well to the notion of sound derivation.
Whether or not the connection is now established we will now articulate a definition of analyticity.

\section{Analyticity and Set Theory}

Let us now proceed from the following definition of analyticity:

\begin{quote}
A sentence in a language with a given semantics is analytic if the semantics uniquely determines the truth value of the sentence.
\end{quote}

For each language L, the semantics of the language determines a property of sentences {\it analytic in L}.
{\it analytic} as opposed to {\it analytic in L} must therefore be a relationship between languages and sentences.
A precise definition of the relationship will depend upon a precise conception of what a language is.

For the purposes of defining analyticity, a {\it truth conditional} semantics will suffice, and we will therefore consider the definition of analyticity as a relationship bewteen {\it truth conditional semantics} and sentences.

What is a {\it truth conditional semantics}?
Tarski's `T' schema comes to mind, this however, is a way of writing down a truth conditional semantics in some suitable metalanguage.
For our present purposes it is helpful to enquire what kind of thing the `T' schema defines. 
A truth conditional semantics defines the conditions under which a sentence is true.
More generally it defines the truth value of sentences (where there may possibly be more than two truth values).
We may think of these truth conditions as a function from some domain of discourse (for the object language) to some domain of truth values.
Neither the domain of discourse nor the set of truth values need be the same in the object language as they are for the meta-language.
For a full account of the semantics of the object language one would expect the ontology of the meta-language to encompass that of the object language.
However, for the purpose of determining which sentences are analytic the identity of the elements in the subject matter of the object language is not important.
For these purposes a domain of discourse which has the same cardinality as the intended domain of discourse but consists entirely of abstract objects would suffice.





\section{The Semantics of Set Theory}

\section{Some Pragmatics of Application}

\begin{thebibliography}{}

\bibitem[\protect\citeauthoryear{MacKensie}{2001}]{MacKensie}
Donald MacKensie: 2001,
\newblock {\it Mechanising Proof - Computing, Risk and Trust},
\newblock {The MIT Press}.

\bibitem[\protect\citeauthoryear{Russell}{1908}]{Russell08}
Bertrand Russell: 1908,
\newblock {Mathematical logic as based on the theory of types},
\newblock {in van Heijenoort (ed.) {\it From Frege to Godel - A source book in mathematical logic 1879-1931}, Harvard University Press, Cambridge Massachusetts, 1967, pp. 150-182}.

\bibitem[\protect\citeauthoryear{Tarski}{1908}]{Tarski36}
Alfred Tarski: 1936,
\newblock {On the concept of logical consequence},
\newblock {in John Corocoran (ed.) {\it Logic, Semantics, Meta-mathematics}, Oxford University Press, 1956, pp. 409-420}.

\end{thebibliography}
\end{article}
\end{document}
