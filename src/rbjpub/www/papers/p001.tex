% $Id: p001.tex,v 1.3 2005/01/26 20:53:38 rbj Exp $
\documentclass[numreferences]{rbjk}

\newdisplay{guess}{Conjecture}
\newdisplay{prop}{Proposition}
\newcommand{\ignore}[1]{}

\begin{document}                                                                                   
\begin{article}
\begin{opening}  
\title{Semantic Foundations for Deductive Methods}
\subtitle{delineating the scope of deductive reason}
\runningtitle{Semantic Foundations for Deductive Methods}
\author{Roger Bishop \surname{Jones}}
\runningauthor{Roger Jones}
%\runningtitle{}

\begin{abstract}

The scope of deductive reason is considered.
First a connection is discussed between the scope of sound deductive inference and the notion of set theoretic truth via the concepts {\it demonstrative} and {\it analytic}.
Then the problem of determining the meaning of set theory and the extension of set theoretic truth is addressed.
\end{abstract}
\end{opening}

\section{Introduction}

My primary purpose is to consider certain problems in the semantics of set theory.
The motivation for this lies however in the notion of deductive reasoning.
Set theory is of importance not merely as branch of mathematics, nor even as a foundation for mathematics, but as a foundation for abstract semantics as a whole, and thereby as an aid in the circumscription of the outer limits of the scope of deductive reasoning.

The early parts of the essay provide some considerations supporting this conception of the importance of set theory.
The main body of the essay then considers ways in which the semantics of set theory and the extension of set theoretic truth can be made definite.

Though people do engage in deduction in natural languages, such languages will not be considered here.
My primary reason for omitting specific consideration of natural languages is that I know no way of including them in the scope of this discussion which would be sufficiently precise and straightforward for present purposes.
However, any natural language which has a well defined semantics (even if incomplete) and a deductive system sound with respect to that semantics (even if these exist but are not known) will fall within the scope of the discussion.

It is suggested that insofar as the delineation of the scope of deductive reason is concerned, this can be done without loss of generality by consideration exclusively of formal deductive systems.

\section{Demonstrative and Analytic Truth}

My aim here is to connect normal practice in establishing the soundness of formal deductive systems with the philosophical concept of analyticity.

This is done via an argument that the concepts {\it demonstrative} and {\it analytic}, suitably defined, are coextensive, the concept demonstrative being defined for this purpose as {\it derivable in a sound deductive system}.

\subsection{Demonstrative Truth}

\subsubsection{Prior Use}

In Aristotle the terms demonstrative and dialectical are used to distinguish two kinds of premis to syllogistic proofs and to distinguish proofs having these kinds of premises.

A demonstrative premise is one ``obtained throught the first principles of its science''.
A dialectical premise is one adopted for argumentative purposes.
A demonstrative premise must be true, whereas a dialectival premise need not be.
 
Furthermore, he says ``Demonstrative knowledge must rest on necessary basic truths; for the object of scientific knowledge cannot be other than it is.'', and that ``demonstrative truth must be knowledge of a necessary nexus''.

For Aristotle then, demonstrative truths are necessary, because reached by syllogistic reasoning from necessary premises.

In Locke the term demonstrative is reserved for the conclusions of proofs, premises are described as {\it intitive}:

\begin{quote}
For if we will reflect on our own ways of thinking, we will find, that sometimes the mind perceives the agreement or disagreement of two ideas immediately by themselves, without the intervention of any other: and this I think we may call intuitive knowledge.
\end{quote}

The wording here is more suggestive of analyticity than of necessity.

Demonstrative knowledge is now defined again via the notion of intuition rather than by reference to the syllogism:

\begin{quote}
Now, in every step reason makes in demonstrative knowledge, there is an intuitive knowledge of that agreement or disagreement it seeks with the next intermediate idea which it uses as a proof...
\end{quote}

and the notion of ``intuitive'' remains here as strong as analytic or necessary.

\subsubsection{Definition}

The standards of modern logic allow what may be thought of as essentially the same concept to be rendered with a greater precision.
Now, the achievement of the highest levels of certainty, which Locke attributes to intuitive and demonstrative knowledge, are associated with the the theorems of formal deductive systems.
The role played the Locke's concept {\it intuitive} in giving us the necessary confidence and assurance is suplanted in modern logic by a proof of soundness, conducted about a formal object language in some suitable metalanguage.

It is proposed here to use the term {\it demonstrative} to mean {\it derivable in a sound deductive system} considering this to be a relation between sentences and semantics, a semantics being a complete account of the truth conditions for a language to which the sentence in question belongs.

It is normal practice among those who devise formal deductive systems to validate those systems by proving them sound.
The effect of this (subject to some caveats which we will address later) is to ensure a connection between the things which are provable in these systems and the concept of analyticity.

To make this connection conspicuous I will give definitions of relevant concepts here from which the alleged elementary connection is readily seen.

I am concerned here exclusively with languages which have a well defined syntax and semantics.
In order to make the desired connection it is convenient to think of the semantics as some kind of abstract entity, in fact, a function.

I will use the word {\it statement} here to mean an ordered pair of which the first element is a sentence and the second the semantics for a language in which that sentence is well-formed.

A statement will be called {\it demonstrative} if the sentence is derivable in a deductive system which is sound with respect to the semantics.

A {\it deductive system} is a set of sentences (the well formed sentences of the language of the deductive system) and an immediate-derivability relation, which is a relation between sets of sentences (the premises of an inference) and single sentences (the conclusion of an inference), all well formed sentences of the language.
The {\it derivability} relation of a deductive system is the transitive closure of its immediate derivability relation.

A {\it semantics} is an assignment of meaning to the sentences of some language which is of interest here only insofar as it yields information about truth conditions for sentences in the language.
We will model this as a function which assigns to each well formed sentence a meaning, the meaning being a set of {\it circumstances} under which that statement is true (these circumstances may be said to {\it satisfy} the sentence).
The nature of a circumstance will vary from one language to the next, but might typically be a possible world and an assignment to free variables of entities in that possible world.

A deductive system is {\it sound} with respect to a semantics if it preserves satisfaction under the semantics, i.e. if all circumstances which satisfy all the premises of a derivation under the semantics also satisfy the conclusion.

\subsection{Analytic Truth}

\subsubsection{Prior Use}

\subsubsection{Definition}

The notion of analyticity likewise will be considered a relationship between sentences and semantics, a sentence being analytic (or an analytic truth) if its truth can be established from the semantics of the language alone, i.e. if the truth conditions show the sentence to be invariably true.

The term {\it analytic} will be used throughout as {\it analytic truth}.

A sentence in some well defined object language is {\it analytic} if it is satisfied under all circumstances.

\subsection{Demonstrative and Analytic are coextensive}

By an elementary induction on the length of proofs it follows that all the theorems of sound deductive systems are analytic.
Every analytic sentence is demonstrative since it is provable in the sound deductive system which has just one inference rule whose conclusion is that sentence.

Hence:
\prop{The concepts {\it analytic} and {\it demonstrative} (as defined) are coextensive.}

We note that in particular under these definitions, the theorems of set theory (say ZFC) are demonstrative and analytic, and hence the theorems of mathematics in general.

\ignore{

\subsection{Reasoning about the Contingent World}

Firstly it should break down where a formal calculus is devised for reasoning about the real world.
For in this case we would expect to use axioms which are not known to be true on the basis of the semantics of the language alone (i.e. which are not analytic), and to soundly derive conclusions whose truth is contingent.
Even in this case the metatheoretic methodology has some merit but must properly be described in different terms.
One important purpose served by the method is to establish the consistency of the logical system.
By showing that all the derivable theorems are {\it true} it is demonstrated that no contradiction is derivable in the system.
For this purpose it is desirable to {\it concoct} a semantics relative to which the system is sound, even if this semantics distorts the meaning of the sentences, for example, by restricting the domain of discourse.
Thus a deductive system which incorporates physical laws may be evaluated as if those physical laws were incorporated into the meaning of the language.
A second reason for gerrymandering semantics is in connection with completeness.
In this case a domain of discourse may be chosen which is broader than might otherwise have been expected, in order to obtain a completeness result for the deductive system.
An example of this is in the non-standard semantics of higher order logics, in which a broader class of interpretations is admitted, fewer inferences are sound, in particular eliminating any which do no correspond to formal derivations.

For deductive systems intended for deriving theorems about the real world the deductive status of the theorems may be considered in two ways, and the choice between these two is made in the semantics of the language.
Either the semantics is {\it on the nail} and contains no more than one would expect, some of the axioms are contingent and hence not true under the semantics, and consequently the deductive system is not sound and the results not demonstrative.
There is however a recoverable demonstrable result, which is obtained in the logical system with the contingent axioms removed, by taking required instances of these principles as hypotheses.
A second approach to applicability of deduction in applied theories is to treat the theory as providing an abstract model of the relevant aspect of reality.
An abstract semantics is supplied relative to which the deductive system, including the contingent axioms (not contingent under the abstract semantics), is sound.
The theorems of the system are then all true under the semantics, but their relevance to the real world is moot.
The correspondence between the abstract world of which the language speaks and the real world is then contingent, as is the intepretation of the theorems of the system in the real world under such a correspondence.

In this we see an echo of Leibniz's special treatment of God.
In his case truths of fact are necessary because of his special knowledge.
In our case claims, ostensibly about the real world, become necessary truths when the relevant features of the world are incorporated in the semantics of the language.
However, their necessity comes at the expense of their content, they no longer tell us about the world, except by contingent hypothesis.

\subsection{Incompleteness}

A second point of possible divergence between the scope of deduction and the concept of analyticity arises from the incompletability of formal deductive systems.

In his discussion of the concept of logical consequence \cite{Tarski36} Tarski argues from the validity of w-inference in the domain of natural numbers to the conclusion that logical consequence cannot be given a syntactic characterisation (i.e. there is no recursive deductive system in which logical consequence corresponds with derivability).
This commits Tarski to a notion of consequence broader than that of first order consequence, though Tarski stops short of accepting so broad a conception as to coincide with analyticity.

Though there is no deductive system which encompasses all sound derivations, there is no sound derivation which is countenanced by no deductive system.
Soundness is in this sense a least upper bound on the derivations encompassed by respectable deductive systems.
Furthermore, as we will argue later, there are formal deductive systems which are good pragmatic approximations to the full extent of sound derivation.

It is a {\it fait accompli} that the concepts of logical truth and logical consequence are now almost unversally construed in a much narrower way that that of demonstrable truth in Hume, a term no longer in much currency.
For a full appreciation of the scope of formal methods, the broader concept is the one of interest, which I have argued here corresponds well to the notion of sound derivation.
Whether or not the connection is now established we will now articulate a definition of analyticity.

}

\section{Analyticity and Set Theory}

Our next observation is that analyticity is reducible to, i.e. definable in terms of, set theoretic truth.

This is not an easily demonstrable claim.
One reason for difficulty is that the universality of set theoretic truth makes that notion itself difficult to define.
This matter will shortly be addressed, but in this preliminary attempt to justify interest in the concept via its connection with demonstrative truth no definition is available.

The justification of the claim is factored into two part.
First it is alleged that for the purposes of determining the extension of analytic truth, abstract semantics suffices.
Then the universality of set theory for abstract semantics is argued.

\subsection{Abstract Semantics suffices for determination of Analytic Truth}



\subsection{The Universality of Set Theory for Abstract Semantics}


\section{The Semantics of Set Theory}

%\begin{thebibliography}{}
%\bibitem[\protect\citeauthoryear{MacKensie}{2001}]{MacKensie}
Donald MacKensie: 2001,
\newblock {\it Mechanising Proof - Computing, Risk and Trust},
\newblock {The MIT Press}.

\bibitem[\protect\citeauthoryear{Russell}{1908}]{Russell08}
Bertrand Russell: 1908,
\newblock {Mathematical logic as based on the theory of types},
\newblock {in van Heijenoort (ed.) {\it From Frege to Godel - A source book in mathematical logic 1879-1931}, Harvard University Press, Cambridge Massachusetts, 1967, pp. 150-182}.

\bibitem[\protect\citeauthoryear{Tarski}{1908}]{Tarski36}
Alfred Tarski: 1936,
\newblock {On the concept of logical consequence},
\newblock {in John Corocoran (ed.) {\it Logic, Semantics, Meta-mathematics}, Oxford University Press, 1956, pp. 409-420}.
%\end{thebibliography}

%{\raggedright
%\bibliographystyle{klunum}
%\bibliography{rbj}
%} %\raggedright

\end{article}
\end{document}




