% $Id: p001.tex,v 1.9 2012/06/29 21:42:53 rbj Exp $
% bibref{rbjp001} pdfname{p001} 
\documentclass[numreferences]{rbjk}

\usepackage[unicode]{hyperref}
\hypersetup{pdfauthor={Roger Bishop Jones}}
\hypersetup{pdftitle={Semantic Foundations for Deductive Methods}}
\hypersetup{colorlinks=true, urlcolor=red, citecolor=blue, filecolor=blue, linkcolor=blue}

\newdisplay{guess}{Conjecture}
\newdisplay{prop}{Proposition}
\newcommand{\ignore}[1]{}

\begin{document}                                                                                   
\begin{article}
\begin{opening}  
\title{Semantic Foundations for Deductive Methods}
\subtitle{delineating the scope of deductive reason}
\runningtitle{Semantic Foundations for Deductive Methods}
\author{Roger Bishop \surname{Jones}}
\runningauthor{Roger Jones}
%\runningtitle{}

\begin{abstract}
The scope of deductive reason is considered.
First a connection is discussed between the scope of sound deductive inference and the notion of set theoretic truth via the concepts {\it demonstrative} and {\it analytic}.
Then the problem of determining the meaning of set theory and the extension of set theoretic truth is addressed.
\end{abstract}
\end{opening}

\section{Introduction}

I propose to consider here certain problems in the foundations of abstract semantics.
The motivation for this lies in the notions of deductive reasoning, of analyticity and entailment.
Set theory is of importance not merely as branch of mathematics, nor even as a foundation for mathematics, but as a foundation for abstract semantics, and thereby as an aid in delimiting the scope of deductive reasoning.

The early parts of the essay provide some considerations supporting this conception of the importance of set theory.
The main body of the essay then considers ways in which the semantics of set theory and the extension of set theoretic truth can be made definite.

Though people do engage in deduction in natural languages, such languages will not be considered here.
My primary reason for omitting specific consideration of natural languages is that I know no way of including them in the scope of this discussion which would be sufficiently precise and straightforward for present purposes.
However, any natural language which has a well defined semantics (even if incomplete) and a deductive system sound with respect to that semantics (even if these exist but are not known) will fall within the scope of the discussion.

It is suggested that insofar as the delineation of the scope of deductive reason is concerned, this can be done without loss of generality by consideration exclusively of formal deductive systems.

\section{Demonstrative and Analytic Truth}

My aim here is to connect normal practice in establishing the soundness of formal deductive systems with the philosophical concept of analyticity.

This is done via an argument that the concepts {\it demonstrative} and {\it analytic}, suitably defined, are coextensive, the concept demonstrative being defined for this purpose as {\it derivable in a sound deductive system}.

\subsection{Demonstrative Truth}

\subsubsection{Prior Use}

In Aristotle the terms demonstrative and dialectical are used to distinguish two kinds of premis to syllogistic proofs and to distinguish proofs having these kinds of premises.

A demonstrative premise is one ``obtained through the first principles of its science''.
A dialectical premise is one adopted for argumentative purposes.
A demonstrative premise must be true, whereas a dialectival premise need not be.
 
Furthermore, he says ``Demonstrative knowledge must rest on necessary basic truths; for the object of scientific knowledge cannot be other than it is.'', and that ``demonstrative truth must be knowledge of a necessary nexus''.

For Aristotle then, demonstrative truths are necessary, because reached by syllogistic reasoning from necessary premises.

In Locke the term demonstrative is reserved for the conclusions of proofs, premises are described as {\it intuitive}:

\begin{quote}
For if we will reflect on our own ways of thinking, we will find, that sometimes the mind perceives the agreement or disagreement of two ideas immediately by themselves, without the intervention of any other: and this I think we may call intuitive knowledge.
\end{quote}

The wording here is more suggestive of analyticity than of necessity.

Demonstrative knowledge is now defined again via the notion of intuition rather than by reference to the syllogism:

\begin{quote}
Now, in every step reason makes in demonstrative knowledge, there is an intuitive knowledge of that agreement or disagreement it seeks with the next intermediate idea which it uses as a proof...
\end{quote}

and the notion of ``intuitive'' remains here as strong as analytic or necessary.

\subsubsection{Definition}

The standards of modern logic allow what may be thought of as essentially the same concept to be rendered with a greater precision.
Now, the achievement of the highest levels of certainty, which Locke attributes to intuitive and demonstrative knowledge, are associated with the the theorems of formal deductive systems.
The role played by Locke's concept {\it intuitive} in giving us the necessary confidence and assurance is suplanted in modern logic by a proof of soundness, conducted about a formal object language in some suitable metalanguage.

It is proposed here to use the term {\it demonstrative} to mean {\it derivable, from the empty set of premises, in a sound deductive system} considering this to be a relation between sentences and semantics, a semantics being an account of the truth conditions for a language to which the sentence in question belongs.

It is normal practice among those who devise formal deductive systems to validate those systems by proving them sound.
The effect of this (subject to some caveats which we will address later) is to ensure a connection between the things which are provable in these systems and the concept of analyticity.

To make this connection conspicuous I will give definitions of relevant concepts here from which the alleged elementary connection is readily seen.

I am concerned here exclusively with languages which have a well defined syntax and semantics.
In order to make the desired connection it is convenient to think of the semantics as some kind of abstract entity, in fact, a function.

I will use the word {\it statement} here to mean an ordered pair of which the first element is a sentence and the second the semantics for a language in which that sentence is well-formed.

A statement will be called {\it demonstrative} if the sentence is derivable in a deductive system which is sound with respect to the semantics.

A {\it deductive system} is a set of sentences (the well formed sentences of the language of the deductive system) and an immediate-derivability relation, which is a relation between sets of sentences (the premises of an inference) and single sentences (the conclusion of an inference), all well formed sentences of the language.
The {\it derivability} relation of a deductive system is the transitive closure of its immediate derivability relation.

A {\it semantics} is an assignment of meaning to the sentences of some language which is of interest here only insofar as it yields information about truth conditions for sentences in the language.
We will model this as a function which assigns to each well formed sentence a meaning, the meaning being a set of {\it circumstances} under which that statement is true (these circumstances may be said to {\it satisfy} the sentence).
The nature of a circumstance will vary from one language to the next, but might typically be a possible world and an assignment to free variables of entities in that possible world.

A deductive system is {\it sound} with respect to a semantics if it preserves satisfaction under the semantics, i.e. if all circumstances which satisfy all the premises of a derivation under the semantics also satisfy the conclusion.

\subsection{Analytic Truth}

\subsubsection{Prior Use}

\subsubsection{Definition}

The notion of analyticity likewise will be considered a relationship between sentences and semantics, a sentence being analytic (or an analytic truth) if its truth can be established from the semantics of the language alone, i.e. if the truth conditions show the sentence to be invariably true.

The term {\it analytic} will be used throughout as {\it analytic truth}.

A sentence in some well defined object language is {\it analytic} if it is satisfied under all circumstances.

\subsection{Demonstrative and Analytic are coextensive}

By an elementary induction on the length of proofs it follows that all the theorems of sound deductive systems are analytic.
Every analytic sentence is demonstrative since it is provable in the sound deductive system which has just one inference rule whose conclusion is that sentence.

Hence:
\prop{The concepts {\it analytic} and {\it demonstrative} (as defined) are coextensive.}

We note that in particular under these definitions, the theorems of set theory (say ZFC) are demonstrative and analytic, and hence the theorems of mathematics in general.

\ignore{

\subsection{Reasoning about the Contingent World}

Firstly it should break down where a formal calculus is devised for reasoning about the real world.
For in this case we would expect to use axioms which are not known to be true on the basis of the semantics of the language alone (i.e. which are not analytic), and to soundly derive conclusions whose truth is contingent.
Even in this case the metatheoretic methodology has some merit but must properly be described in different terms.
One important purpose served by the method is to establish the consistency of the logical system.
By showing that all the derivable theorems are {\it true} it is demonstrated that no contradiction is derivable in the system.
For this purpose it is desirable to {\it concoct} a semantics relative to which the system is sound, even if this semantics distorts the meaning of the sentences, for example, by restricting the domain of discourse.
Thus a deductive system which incorporates physical laws may be evaluated as if those physical laws were incorporated into the meaning of the language.
A second reason for gerrymandering semantics is in connection with completeness.
In this case a domain of discourse may be chosen which is broader than might otherwise have been expected, in order to obtain a completeness result for the deductive system.
An example of this is in the non-standard semantics of higher order logics, in which a broader class of interpretations is admitted, fewer inferences are sound, in particular eliminating any which do no correspond to formal derivations.

For deductive systems intended for deriving theorems about the real world the deductive status of the theorems may be considered in two ways, and the choice between these two is made in the semantics of the language.
Either the semantics is {\it on the nail} and contains no more than one would expect, some of the axioms are contingent and hence not true under the semantics, and consequently the deductive system is not sound and the results not demonstrative.
There is however a recoverable demonstrable result, which is obtained in the logical system with the contingent axioms removed, by taking required instances of these principles as hypotheses.
A second approach to applicability of deduction in applied theories is to treat the theory as providing an abstract model of the relevant aspect of reality.
An abstract semantics is supplied relative to which the deductive system, including the contingent axioms (not contingent under the abstract semantics), is sound.
The theorems of the system are then all true under the semantics, but their relevance to the real world is moot.
The correspondence between the abstract world of which the language speaks and the real world is then contingent, as is the intepretation of the theorems of the system in the real world under such a correspondence.

In this we see an echo of Leibniz's special treatment of God.
In his case truths of fact are necessary because of his special knowledge.
In our case claims, ostensibly about the real world, become necessary truths when the relevant features of the world are incorporated in the semantics of the language.
However, their necessity comes at the expense of their content, they no longer tell us about the world, except by contingent hypothesis.

\subsection{Incompleteness}

A second point of possible divergence between the scope of deduction and the concept of analyticity arises from the incompletability of formal deductive systems.

In his discussion of the concept of logical consequence \cite{Tarski36} Tarski argues from the validity of w-inference in the domain of natural numbers to the conclusion that logical consequence cannot be given a syntactic characterisation (i.e. there is no recursive deductive system in which logical consequence corresponds with derivability).
This commits Tarski to a notion of consequence broader than that of first order consequence, though Tarski stops short of accepting so broad a conception as to coincide with analyticity.

Though there is no deductive system which encompasses all sound derivations, there is no sound derivation which is countenanced by no deductive system.
Soundness is in this sense a least upper bound on the derivations encompassed by respectable deductive systems.
Furthermore, as we will argue later, there are formal deductive systems which are good pragmatic approximations to the full extent of sound derivation.

It is a {\it fait accompli} that the concepts of logical truth and logical consequence are now almost unversally construed in a much narrower way that that of demonstrable truth in Hume, a term no longer in much currency.
For a full appreciation of the scope of formal methods, the broader concept is the one of interest, which I have argued here corresponds well to the notion of sound derivation.
Whether or not the connection is now established we will now articulate a definition of analyticity.

}

\section{Analyticity and Set Theory}

Our next observation is that analyticity is reducible to, i.e. definable in terms of, set theoretic truth.

This is not an easily demonstrable claim.
One reason for difficulty is that the universality of set theoretic truth makes that notion itself difficult to define.
This matter will shortly be addressed, but in this preliminary attempt to justify interest in the concept via its connection with demonstrative truth no definition is available.

The justification of the claim is factored into two part.
First it is alleged that for the purposes of determining the extension of analytic truth, abstract semantics suffices.
Then the universality of set theory for abstract semantics is argued.

\subsection{Abstract Semantics suffices for determination of Analytic Truth}



\subsection{The Universality of Set Theory for Abstract Semantics}

\section{The Semantics of Set Theory}

For the time being I propose to use this section for sundry discussions of set theory.

As well as there being various kinds of set theory (e.g. first order, second order, well-founded, non-well-founded, with or without a universal set), there are various different kinds of thing which might be offered as a semantics.

I am here concerned with set theory as a foundation for abstract semantics.
Furthermore, my concern is with a foundation whose role is to provide a good response to the problem of semantic regress, rather than a foundation which is intended to provide a pragmatically convenient general context in which to undertake a formal development of some substantial body of demonstrative knowledge.

In meeting the latter need, which I hope to consider more fully in due course, non-well-founded set theories might possibly have a contribution to make.
But for the former, well-founded set theory suffices.

When considering the semantics of well-founded set theory, we must first consider what set theoretic syntax is to be given meaning by the exercise.


\subsection{The Iterative Conception of Set}

``The iterative conception of set'' is the name given to a particular {\it explication} of the concept of a well-founded set (though not often presented as an account of a particular kind of set).
It is generally held to have been articulated in the first instance in a paper by Zermelo dated about 1930



\subsection{Defining Truth Predicates}


\appendix

\section{Extracts from FOM discussions}

\subsection{V does not exist}

\paragraph{The Proof}

Fri Oct 7 03:52:27 EDT 2005

\begin{verbatim}

Both A.P.Hazen and Aatu Koskensilta have responded
to an argument on my part (though not mine) to the
effect that the standard interpretation of V in
NBG is incoherent.

Though I argued that calling V a class rather than
a set would not escape the argument, Hazen felt that
if V really were a different kind of thing:

  "they are the (extensionalizations of) meanings
   of predicates of our set-theoretic language, and they 
   exist only by being definable."

then my argument would fail.

Koskensilta's response I didn't entirely understand,
but seemed to be directed toward justifying
quantification over classes, whereas my objection
was not to quantification over classes.  It was to
the possibility of one particular class, V, being
what it is supposed to be.

I provide below a new presentation of the argument
which I think makes the argument more general and
precise, and clarifies the character of the result.

The argument, as now presented is an argument about
the concept "pure well-founded set" (which is what
I take the iterative conception of set to be describing).
It is to the effect that this concept does not have
a "definite" extension.
The meaning of "definite" here is not crucial to the
argument.  In classical set theory as described in
the iterative conception of set "definite" means
something very weak (much weaker than the notion
of "definite property" used in defining separation).
It just means something like that the predicate or
membership relation is boolean.


I offer the following definition:

Defn:	A "pure well-founded set" is any definite
	collection of pure well-founded sets.

>From which I allege follows:

Lemma:	Pure well-founded sets are pure and well-founded
	(in the usual sense of these terms).

My thesis is:

Theorem:
	The extension of the concept "pure well-founded set"
	is not definite.

Proof:  By reductio.  Assume that it is definite and conclude
	that it both is and is not heteronymous.

Since the argument is about the concept of set itself, any
object which purports to have a definite extension which
coincides with that concept, however different that object
may be from a set, must be tainted with the incoherence
of supposing that the concept set has a definite extension.

For anyone who finds this argument too tenuously connected
to the iterative conception of set, it can be reduced
to something closer to that account via a similar argument
to the effect that the extension of the concept ordinal
(which corresponds of course to the stages in the iterative
conception) cannot have a definite extension, and hence
that the conception cannot describe a definite collection
of sets. 

\end{verbatim}

\paragraph{The Elaboration}

Wed Oct 12 04:29:05 EDT 2005

\begin{verbatim}

On Saturday 08 October 2005  6:33 pm, Richard Heck wrote:
> >Both A.P.Hazen and Aatu Koskensilta have responded to an
> > argument on my part (though not mine) to the effect that the
> > standard interpretation of V in NBG is incoherent.
> >
> >Though I argued that calling V a class rather than a set
> > would not escape the argument, Hazen felt that if V really
> > were a different kind of thing:
> >
> >  "they are the (extensionalizations of) meanings
> >   of predicates of our set-theoretic language, and they
> >   exist only by being definable."
> >
> >then my argument would fail.
>
> Allen's language here is somewhat colorful, but I took his
> point to rest upon the observation that quantification over
> classes NBG can be understood as substitutional. Perhaps there
> is a problem here I'm not remembering, one that is connected
> with the presence of parameters in the comprehension axioms,
> but I don't think so. In any event, much the same point could
> be made in a different way: NBG can be interpreted in ZF(C)
> plus a weak truth-theory, one in which the truth-predicate is
> not allowed to figure in instances of schemata. If you think
> of classes that way, then I think it's clear enough what
> Allen's flourishes mean,

However, as I pointed out in my message, my argument is
independent of the nature of V, speaking only to its intended
extension, and has nothing to say about quantification over
classes (though I could easily offer relevent corollories).

I did not argue that NBG cannot be interpreted.

> and
>
> there is no conflict between NBG and the definition:
> >Defn:	A "pure well-founded set" is any definite collection of
> > pure well-founded sets.
>
> which I take to be equivalent to Boolos's insistence that
> set-theory is supposed to be about /all/ collections.

Well it certainly is not intended to be equivalent to it.

First of all, I don't see how a definition can be equivalcnt
to an "insistance"!

If I were to take this alleged insistance as a definition
then I guess it would read "a set is any collection".

The difference between this and my own definition, which
I will paraphrase for comparison as "a set is any
definite collection of sets", seems to me very considerable.

My definition contains so much information that it runs
very close to inconsistency (its a reductio absurdum
on the possibility that the iterative conception
could be completed).
The one you attribute to Boolos contains so little information
that it runs close to vacuity.

My definition is a well-founded recursive definition.
It is a definition by transfinite induction, and should
be understood as involving the tacit codicil: nothing
is a set unless its sethood is entailed by the definition.

>From the definition we are can derive a principle
of transfinite induction asserting that sets have every
"hereditary" property, where, in this context, a property
is hereditary iff it is posessed by set whenever it is
posessed by all its members.

Using this induction principle we can then prove that:

1.  All sets are pure.
2.  All sets are well-founded.
and hence
3.  All sets are "heteronymous"
	(i.e. do not contain themselves) 

None of these conclusions flows from the insistance
which you attribute to Boolos.

More controversially perhaps, it is plain from my
definition that:

4.  All definite collections of sets are sets.

and hence that there are no proper classes, unless
something containing things other than sets or lacking
a definite extension might be said to be a class.

Boolos's alleged insistance, would have the additional
disadvantage that, taken out of context but with some
knowledge of Boolos's metaphysics, we might reasonably
interpret it as referring to all "actual" collections,
where the meaning of "actual" if any, can only be
discovered by probing Boolos's metaphysical intuitions.

By contrast, my definition may be understood as a
definition, not of all the sets which "really exist"
but as a definition of all the sets which might possibly
exist, of which the sets intuited by Boolos are
an infinitesimally small part.

A final but important difference between the "definitions"
is the occurence in mine of the concept "definite".
Without this the argument yields a contradiction without
consideration of classes, suggesting that the concept of
"set" is incoherent.
With it, it appears to demonstrate that there must be some
characteristic of the extesions which yield sets which
is not shared by the extensions which yield classes.
In my view it is best to read "definite" as a feature
implicit in the first order formalisation of set theory,
viz: that for any set s and any putative member x either
x is in s or x is not in s.
This is of course, an instance of excluded middle.
For a theory to emcompass collections which are not definite
in this sense one would have to represent membership by
something more complicated than a relation.
Possibly this motivates attempts to interpret
classes as rules or formulae.   However this won't
help if the rule or formula or whatever, is still
supposed to have a definite extension.

Since NBG is a first order language of set theory
which includes "classes" such as V, this kind of
"definiteness" of extension is possessed both by
the sets and the classes, and the argument shows
that the supposition that the extension of V is
all the sets encompassed by the iterative conception
(rather than all the sets in some other
interpretation of NBG) is incoherent.

I guess that, even with this additional explanation
you will not be convinced by this argument, and in
that case I would be interested to know where you
find the argument to be faulty.

For my part, coming across this particular definition
of "set", (even though its an obvious definition of
pure well-founded set and seems, obviously, to say the
same thing as the iterative conception),
has made a significant change to my beliefs
about classes and about what kinds of accounts of
the semantics for set theory are coherent.

I used to be suspicious about V, doubting whether
the iterative conception of set could coherently
be considered completeable.
But I knew of no argument for or against which I
considered wholly convincing.
I now believe not only that the intended interpretation
of NBG is incoherent, but also that formal set
theories which do not mention classes cannot coherently
be considered to be interpreted in the complete domain
described by the iterative conception of set.

Of course, these are philosophical matters, so
I don't imagine that these arguments are conclusive.

\end{verbatim}

%\begin{thebibliography}{}
%\bibitem[\protect\citeauthoryear{MacKensie}{2001}]{MacKensie}
Donald MacKensie: 2001,
\newblock {\it Mechanising Proof - Computing, Risk and Trust},
\newblock {The MIT Press}.

\bibitem[\protect\citeauthoryear{Russell}{1908}]{Russell08}
Bertrand Russell: 1908,
\newblock {Mathematical logic as based on the theory of types},
\newblock {in van Heijenoort (ed.) {\it From Frege to Godel - A source book in mathematical logic 1879-1931}, Harvard University Press, Cambridge Massachusetts, 1967, pp. 150-182}.

\bibitem[\protect\citeauthoryear{Tarski}{1908}]{Tarski36}
Alfred Tarski: 1936,
\newblock {On the concept of logical consequence},
\newblock {in John Corocoran (ed.) {\it Logic, Semantics, Meta-mathematics}, Oxford University Press, 1956, pp. 409-420}.
%\end{thebibliography}

%{\raggedright
%\bibliographystyle{klunum}
%\bibliography{rbj}
%} %\raggedright

\end{article}
\end{document}




