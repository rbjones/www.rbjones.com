% $Id: p033.tex $fi
% bibref{rbjp032} pdfname{p033}
\documentclass[10pt,titlepage]{book}
\usepackage{makeidx}
\newcommand{\ignore}[1]{}
\usepackage{graphicx}
\usepackage[unicode]{hyperref}
\pagestyle{plain}
\usepackage[paperwidth=5.25in,paperheight=8in,hmargin={0.75in,0.5in},vmargin={0.5in,0.5in},includehead,includefoot]{geometry}
\hypersetup{pdfauthor={Roger Bishop Jones}}
\hypersetup{pdftitle={Failing Democracies and How to Fix Them}}
\hypersetup{colorlinks=true, urlcolor=red, citecolor=blue, filecolor=blue, linkcolor=blue}
%\usepackage{html}
\usepackage{paralist}
\usepackage{relsize}
\usepackage{verbatim}
\usepackage{enumerate}
\usepackage{longtable}
\usepackage{url}
\newcommand{\hreg}[2]{\href{#1}{#2}\footnote{\url{#1}}}
\makeindex

\title{\LARGE\bf Failing Democracies \\and \\How to Fix Them}
\author{Roger~Bishop~Jones}
\date{\small 2020/04/03}


\begin{document}
\frontmatter

%\begin{abstract}
% We seem at this moment in history to be unusually well endowed with tales of how our democracies are failing.
% Against these there is "push back".
% To push back you need to spot what is going on, to describe it clearly, and to understand and articulate why it is pathological.
% There are many recent examples of this, some of which are mentioned in this essay.
%
% Thinking philosophically about these phenomena may provide more compelling support for core values which are threatened against
% the broadest range of subversive strategies and tactics.
% Here we take democracy as the fundamental value and seek to analyse a broad range of contemporary erosions and consider what kinds of defence might be mounted against them, though the principle defence the light of scrutiny, on the thesis that once we see clearly the threats and the values which they threated, their strength will be undermined.
%\end{abstract}
                               
\begin{titlepage}
\maketitle

%\vfill

%\begin{centering}

%{\footnotesize
%copyright\ Roger~Bishop~Jones;
%}%footnotesize

%\end{centering}

\end{titlepage}

\ \

\ignore{
\begin{centering}
{}
\end{centering}
}%ignore

\setcounter{tocdepth}{2}
{\parskip-0pt\tableofcontents}

%\listoffigures

\mainmatter

\pagebreak

\section*{Preface}

\addcontentsline{toc}{section}{Preface}

There are ``hyperlinks'' in the PDF version of this monograph which either link to another point in the document  (if coloured blue) or to an internet resource  (if coloured red) giving direct access to the materials referred to (e.g. a Youtube video) if the document is read using some internet connected device.
Important links also appear explicitly in the bibiography.

\chapter{Introduction}


I don't really have much idea how to do this, but I think quite a lot of digging is in order, and I will start by making notes here on what others have uncovered.

\chapter{Some People and their Books}

\section{Michelle Alexander}

The New Jim Crow: Mass Incarceration
in the Age of Colorblindness \cite{alexander-tnjc}

\section{Ayaan Hirsi Ali}

\subsection{The Challenge of Dawa}
- 
Political Islam as
Ideology and Movement
and How to Counter It \cite{ali-dawa}

\begin{small}
\begin{tabular}{l l}
Executive Summary & 1\\
Summary of Policy Recommendations&5\\
Introduction&9\\
Part I The Constitution of Political Islam&23\\
Part II Dawa: Much More than a “Call to Islam”&35\\
Part III Confronting Ideology to Win the War&51\\
Conclusion&61\\
Detailed Policy Recommendations&65\\
Appendix A: Eight Types of Threat from Radical Islam&79\\
Appendix B: Shay’s Three Joint Pillars of Dawa and Jihad&73\\
Appendix C: Mares’ Five-Step Model of Political Islam’s
Expansion&75\\
Appendix D: Charities and the Terrorist Money Trail&77\\
Glossary&79\\
\end{tabular}
\end{small}

\begin{quote}
{\it
The aim of da’wah and jihaad is not to shed blood, take wealth,
or enslave women and children; these things happen incidentally but are not the aim. This only takes place when the
disbelievers (non-Muslims) refrain from accepting the truth
and persist in disbelief and refuse to be subdued and pay the
jizya (tax levied on free non-Muslims living under Muslim
rule) when it is requested from them. In this case, Allah has
prescribed the Muslims to kill them, take their wealth as booty
and enslave their women and children . . . this religion (Islam)
. . . is superior to every law and system. . . . The truth has been
spread through the correct Islamic da’wah, which in turn has
been aided and supported by jihaad whenever anyone stood in
its way. . . . It was jihaad and da’wah together which helped to
open the doors to victories.
}
\end{quote}
—Saudi Grand Mufti Ibn Baz, 1998

\begin{quote}
{\it
A 2008 survey of more than nine thousand European Muslims
by the Science Center Berlin reported strong belief in a return
to traditional Islam. In the words of the study’s author, Ruud
Koopmans, “almost 60 percent agree that Muslims should return
to the roots of Islam, 75 percent think there is only one interpretation of the Quran possible to which every Muslim should stick,
and 65 percent say that religious rules are more important to
them than the laws of the country in which they live.” More than
half (54 percent) of European Muslims surveyed also believe
that the West is out to destroy Muslim culture.
}
\end{quote}

\section{Peter Boghossian}

We have here a book and a supporting talk.
Talk first.

\subsection{The Way Forward}

\href{https://www.youtube.com/watch?v=LiymUd9FjHA}{The Way Forward}

There is a prelude which is about critical social justice.

\begin{itemize}
\item[9:58] Start by listening: you have to understand what your target is talking about.
  Don't talk over him, concede if there is a clash. (go, no you go)
  Say ``I'm not sure I understand'' rather than complaining of unclarity.
  Say ``I hear you'' and mean it.
  Echo snips.
  Attempt to re-express targets position clearly.
\item[14:29] ``Epistemology'': ask how they know it?
\item[16:28] Scales: how confident are you in that belief? (1:10 or whatever)
  Ask again after (pre-test/post-test).
  Ask ``how much of ...'' (e.g. of a patriarchy).
\item[19:24] Disconfirmation: under what conditions could you be wrong?
  (what evidence would convince you otherwise).
  If can't be wrong, then ``oh, not based on evidence?''.
  You need to now from them what evidence would change mind, what would count for youis not at issue.
\item[23:00] Ask a question to facilitate doubt.
  e.g. if gender studies and biology disagree, who would you be likely to believe mkore?
\item[24:10] Don't provoke defensiveness: Don't say ``but'', it denies everything that has gone before, and sets target into defensive posture.
  Say ``Yes and ...''.
\item[25:35] Build bridges: make golden bridges to save face.
\item[26.35] Don't apologise: unless you are actually sorry. Not if someone is offended.  AS long as criticising ideas rather than persons.
\item[27:35] Be sincere: demonstrate parahesia, don't equivocate or sugarcoat.
  Speaking in unclear language is a type of self-deception.
  \item[29] Be willing to revise your beliefs.  Be willing to say ``I don't know''.
\end{itemize}

Then he runs over it again.

\subsection{How to Have Impossible Conversations: A Very Practical Guide}

\cite{boghossian-manual,boghossian-conversations}

\subsubsection{Where they are coming from:}

\begin{quote}
  We know because we’ve had countless conversations with zealots, criminals, religious fanatics, and extremists of all stripes.
  Peter did his doctoral research in the Oregon State Prison System conversing with offenders about some of life’s most difficult questions, and then built upon those techniques in thousands of hours of conversations with religious hardliners.
  James developed the ideas for his books and articles by engaging in extended conversations with people who hold radically different views about politics, morality, and religion.
  This book is the culmination of our extensive research and a lifetime of experience in conversing with people who profess to be unshakable in their beliefs.
\end{quote}

\subsubsection{They offer:}

\begin{quote}
thirty-six techniques drawn from the best, most effective research on applied epistemology, hostage and professional negotiations, cult exiting, subdisciplines of psychology, and more.
\end{quote}

\subsubsection{Partial Contents:}

\begin{tabular}{l | p{9cm}}
one & When Conversations Seem Impossible \\\hline
two & The Seven Fundamentals of Good Conversations \\\hline
\#1 & GOALS Why are you engaged in this conversation? \\
\#2 & PARTNERSHIPS Be partners, not adversaries \\
\#3 & RAPPORT Develop and maintain a good connection \\
\#4 & LISTEN Listen more, talk less \\
\#5 & SHOOT THE MESSENGER Don’t deliver your truth \\
\#6 & INTENTIONS \\
& People have better intentions than you think \\
\#7 & WALK AWAY\\\hline
three & Beginner Level: Nine Ways to Start Changing Minds \\\hline
\#1 & MODELING Model the behavior you want to see in others \\
\#2 & WORDS Define terms up front \\
\#3 & ASK QUESTIONS Focus on a specific question \\
\#4 & ACKNOWLEDGE EXTREMISTS\\
&Point out bad things people on your side do \\
\#5 & NAVIGATING SOCIAL MEDIA \\
&Do not vent on social media \\
\#6 & DON’T BLAME, DO DISCUSS CONTRIBUTIONS \\
&Shift from blame to contribution \\
\#7 & FOCUS ON EPISTEMOLOGY \\
& Figure out how people know what they claim to know \\
\#8 & LEARN Learn what makes someone close-minded \\
\#9 & WHAT NOT TO DO (REVERSE APPLICATIONS)\\
& A list of fundamental and basic conversational mistakes \\\hline
four & Intermediate Level: Seven Ways to Improve Your Interventions \\\hline
five & Five Advanced Skills for Contentious Conversations \\\hline
six & Six Expert Skills to Engage the Close-Minded \\\hline
seven & Master Level: Two Keys to Conversing with Ideologues \\
\end{tabular}

\section{Ta-Nehisi Coates}

\cite{coatestnh-bwm,coatestnh-wweyip}

\section{Christopher Hitchens}

\subsection{Why Orwell Matters}

\cite{hitchens-wom}

\section{James Lindsay}

\cite{pluckrose-cynical,lindsay-everybody,pluckrose-cynical}

\section{Heather Mac Donald}

\cite{macdonald-bbi, macdonald-woc, macdonald-tdd}

\section{Herbert Marcuse}

\cite{marcuse-magee}

\section{Charles Murray}

\cite{murrayc-tbc,murrayc-hd}

\section{Douglas Murray}

\cite{murrayd-vi,murrayd-sde,murrayd-tmc}

\section{George Orwell}

\subsection{1984}

\cite{orwell-1984}

\subsection{Animal Farm}

\cite{orwell-af}

\section{Steven Pinker}

\cite{pinker-tbs,pinker-angels,pinker-en}

\section{Helen Pluckrose}

\subsection{The Evolution of Postmodern Thought}

This following is extracted from the talk \cite{pluckrose-evolution}.

A timeline:
\begin{itemize}
\item[5:41] late 1960s, 66-70: beginning of era of post-modernity.
  Loss of confidence in ``modernity'' which stood for scientific and social progress and objective truth.
  Both revolutionary socialism and liberalism were modernist.
  Loss of credibility in Marxism and liberalism.
  The void was filled by ``post-modernism''.
   Jean Baudrillard, Giles Deluge, Felix Guattari.
   Descriptive, despairing, aimless.
 \item[11:20]Post-structuralist and deconstructionist thinkers:

   \begin{itemize}

   \item[11:30] Jean Francois Lyotard: ``The Postmodern Condition'' (1979); scepticism towards meta-narratives (christianity, marxism, science).
     Language of science inseparable from the language of power and government.
     Instead of meta-narratives, we need lots of mini-narratives, moral and factual relativism.

   \item[12:50] Jacques Derrida: sceptical about the possibility of conveying meaning by language, ``words only refer to other words so meaning indefinitely deferred'', but can be used to express comparisons such as ``men superior to women''.  So he advocated inverting these, to expose and challenge them.  Possibly justifying inverted oppression to redress the balance.

   \item[14:12] Michel Foucault: episteme, power-knowledge, discourses, biopower.
     Knowledge as cultural construct.  We decide what is true and what is known through categories and narratives created and enforced culturally (an episteme).
     Those in power set the episteme, this is power-knowledge.
   \end{itemize}
   The imperative then, of postmodern approaches, is to study the discourses of society, to find the Foucian power-knowledge, invert the Derridian binaries and empower the Lyotardian mini-narratives.
   \item[17:57] This yields the following ``plan'':
     \begin{enumerate}[i)]
     \item there is no way of obtaining objective truth, everything is culturally constructed
     \item society is dominated by systems of power and privilege that people just accept as common sense
     \item these vary from culture to culture and subculture to subculture
     \item none of them is right or superior to any other
     \item the categories that we use to understand things, like fact and fiction, reason and emotion, science and art and male and female, are false
     \item they operate in the service of power need to be examined, broken down and complicated
     \item language is immensely powerful and it is used to construct oppressive social realities, therefore it must be regarded with suspicion and scrutinized to find the discourses of power
     \item the intention of the speaker is no more authoritative than the interpretation of the hearer
     \item the idea of the autonomous individual is a myth, the individual is also a construct of culture programmed by his or her place in relation to power
     \item the idea of a universal human nature is also a myth, it is constructed by what
powerful forces deemed to be the right way to be, therefore it is white Western masculine and heterosexual.
     \end{enumerate}
     These are the ideas from post-modernism which have survived and now appear in the social justice movement.
     Thus ends the ``High deconstructive'' phase of post-modernism.
   \item[19:36] Late 1980s: new generation of leftist academics, legal support for various oppressions disappears leaving only attitudes to be addressed, to which post-modernist ideas can be applied.
   \item[21:06] Post colonialism: Offshoot of post-modernism headed by Foucauldian Edward Said who argued ``The west constructed the East as its inferior in order to construct itself in noble terms'' and that previous colonies should now reconstruct the East for themselves.
     Spivak and Bhabha followed but were more Derridian, and hence incomprehensible.
   \item[21:41] 1989: In ``critical legal studies'' and ``critical race theory'' Kimberlé Crenshaw began developing her concept of intersectionality, which she described as contemporary politics linked to postmodern theory.
     She accepted the cultural constructivism of post-modernism in relation to the concepts of race and gender but believed in the objective reality of oppressive cultural constructs around race and gender.She thought liberalism inadequate despite evidence of its success.
     Liberalism was too universal and an intense focus on identity politics was needed.
     Mary Poovey Adopted a similar stance in relation to feminism, reconciling postmodernist deconstruction with he objective reality of the category of women.
     She advocated a ``toolbox'' approach using post-modern techniques when helpful and not otherwise.
     Judith Butler belief in  objective categories was the problem.
     Queer theory purest form of post-modernism currently in existence.
     
     \item[24] It was now objectively true that social reality was culturally constructed by specific systems of power.

\item white privilege - Peggy Macintosh
\item white complicity - Barbara Applebaum
\item white fragility - Robin DiAngelo
\item re-ified postmodernism - The Creed:
  \begin{enumerate}[i)]
\item  racism exists today in both traditional and modern forms
\item  racism is an institutionalized multi-layered multi-level system that distributes unequal power and resources between white people and people of color, as socially identified, and disproportionately benefits White's
\item  all members of society are socialized to participate in the system of racism albeit in various social locations
\item all white people benefit from racism regardless of their intentions
\item no one chose to be socialized into racism so no one is bad, but no one is neutral so not to act against racism is to support racism
\item racism must be continually identified analyzed and challenged no one is ever done
\item the question is not did racism take place but how did racism manifest in that situation
\item the racial status quo is uncomfortable for most White's therefore anything that maintains white comfort is suspect
\item the racially oppressed have a more intimate insight via experiential knowledge into the system of race than their racial oppressors but they're not bad
  \item however white professors will be seen as having more legitimacy thus positionality must be intentionally engaged (means you must always mention your race gender and sexuality and how it impacts on what you're saying)
\item resistance is a predictable reaction to anti-racist education and must be explicitly and strategically addressed
\end{enumerate}

\end{itemize}

\subsection{Cynical Theories}

How Activist Scholarship Made Everything about Race, Gender and Sexuality - and Why this Harms Everybody \cite{pluckrose-cynical}.

An overall structure somethings like this:

\begin{enumerate}

\item Postmodernism (1966-1989)

   - Foundational postmodern principles held that objective knowledge is impossible, that knowledge is a construct of power, and that society is made up of systems of power and privilege that need to be deconstructed.

\item Applied postmodernism (80s, 90s)

  \begin{itemize}
  \item postcolonial Theory
  \item queer Theory
  \item critical race Theory
  \item intersectional feminism
  \item disability studies
  \item fat studies
  \end{itemize}

\item since 2010, concretized in the combined intersectional Social Justice scholarship and activism
\end{enumerate}


\subsubsection{Postmodernism}
A revolution in knowledge and power.

Two principles of postmodernism:

\begin{enumerate}
\item Knowledge Principle

  Radical scepticism about objective truth and knowledge, commitment to cultural constructivism.
  
\item Political Principle

  Society is formed of systems of power and hierarchies which determine what can be known and how.
  \end{enumerate}

Four Themes:

\begin{enumerate}
\item blurring of boundaries
\item power of language
\item cultural relativism
\item loss of the individual and the universal
\end{enumerate}

These are core elements of postmodern ``Theory'' which have persisted throughout the evolution of postmodernism and its applications from its initial ``hopelessness'' to its recent strident activism.

\subsubsection{Postcolonial Theory}

\subsubsection{Queer Theory}

\subsubsection{Critical Race Theory}

\subsubsection{Feminism and Gender Studies}

\subsubsection{Disability and Fat Studies}

\subsubsection{Social Justice Scholarship}

\subsubsection{Social Justice in Action}

\subsubsection{An Alternative to the Ideology of Social Justice}

\section{Karl Popper}

\subsection{The Open Society and its Enemies}

\cite{popperOSE1,popperOSE2}

\subsection{The Poverty of Historicism}

\cite{popperPOH}

\section{Marc Sidwell}

\subsubsection{The Long March}

\cite{sidwell-long}

This strategy, named after the long march securing the victory of Communist Party of China, was first enunciated by the German student activist Rudi Dutchke, supported by Herbert Marcuse, and was a replacement for the failed revolutionary expectations of Marxism.

The idea is that student activists should join the various key professions, become competent and progress to positions of power so that these institutions can be subverted from within.
Educational institutions were a first priority, since they educated the new blood which would go into the other professions and institutions, and teacher training of particular interest since activist teachers could then begin the work with younger and more impressionable subjects.

This strategy has proved very successful in propagating critical theories through key social institutions in the UK and in other English-speaking developed nations (USA, Canada, Australia) such as education, the media, and the arts, and thence through quangos and into industry via Human Resource departments.
In the USA the preponderance of strong left wing politics in tertiary education, particularly in social sciences and humanities, is sufficient to silence many of the few conservatives who survive in that context on any matter which is politically sensitive.
The reach of critical theory leaves little out of its scope, now making its final assault on the supposedly objective and non-partisan STEM curriculum.
The United Kingdom follows in the wake, even issues conspicuously local to the USA readily crossing the atlantic.

\begin{itemize}
\item[CHAPTER ONE] Gramsci’s Ghost 1
\item[CHAPTER TWO] Meet the Blob 11
\item[CHAPTER THREE] The ‘Culture Industry’ Industry 21
\item[CHAPTER FOUR] Wolves in Sheep’s Clothing 31
\item[CHAPTER FIVE] Mao, Marx and Marcuse 40
\item[CHAPTER SIX] The Thatcher Revolution 51
\item[CHAPTER SEVEN] From Political Class to Identity Politics 65
\item[CHAPTER EIGHT] Failing Upwards 77
\item[CHAPTER NINE] The Art of Cultural Resistance 89
\item[CHAPTER TEN] Downstream of Politics 99
\end{itemize}

\section{Debra Soh}

\cite{soh-end}

\section{Roger Scruton}

\cite{scruton85,scruton15}

\section{Thomas Sowell}

\cite{sowell-barbarians}

\chapter{Some Issues}

\section{Social Justice and Identity Politics}

These are underpinned by a variety of ``critical theories'' which are in turn underpinned by the ``Critical Theory'' of the Frankfurt School and Postmodern Philosophy.

\subsection{pre-history}

\subsubsection{Hegel and Marx}

{\bf\emph{Notes from Magee and Singer}}.

Magee\cite{magee-singer} summary:

\begin{enumerate}
\item understanding reality = understanding a process (of perpertual change)
\item what is changing - ``geist'' (mind/spirit)
\item why changing? - because in a state of alienation
\item what is the process of change? - the dialectical process
\item where is it going? - politically: to organic society; philosophically: to absolute knowledge
\end{enumerate}

Note that (in this sketch) Magee does not mention Hegel's notion of freedom, and does not give particular prominence to the alleged logical necessity in the dialectic. 

Left and Right Hegelians (young and old)

Right (conservative) took Hegel to be describing the Prussian state and therefore that no radical changes were required.

Left (radical) took Hegel to be concerned with overcoming the conflict between reason and desire or between morality and self interest, a very large undertaking, hence requiring revolution (though Hegel did not advocate it).

Marx was a left Hegelian and carried over 1 and 3-5 above, but eliminates ``geist'' as the subject of the process in favour of matter, hence ``dialectical materialism''.

{\bf\emph{Notes from Sabine}}.

Sabine \cite{sabine63} seems to me to penetrate deeper in his first explanation of Hegel's logic, along the following lines.

Hegel's logic is intended to overcome the constraints advertised (e.g. by Hume) on analytic (deductive) logic and implicit in the tripartite division between truths of reason, empirical truths and value judgements.
Hegel offers in his dialectic a kind of reason which tells us about how history progresses and gives us a logical justification for matters including morals and religion.

Sabine mentions ``three vaguely similar generalisations'':

\begin{enumerate}
\item universal human progress (inherited from the enlightenment, Turgo, Condorcet)
  
\item logically necessary historical development (of national cultures)
  
\item Darwin's theory of ``organic evolution''
\end{enumerate}

The failure to distinguish these very different notions of progress (of which (2) is Hegel's historicism) caused great confusion.
Sabine considers (3) to be irrelevant to (1) and (2), and considers (1) to be revolutionary in tendency while (2) in Hegel's conception is conservative (but is later transformed by Marx into a revolutionary theory).

The main distinctive feature of (2) is that it is held to be a matter of \emph{logic} rather than of \emph{empirical causation}.
Hegel, and later Marx, regarded (1) (and (3)?) as ``philosophically superficial'' \emph{because} they are empirical theories.
Hegel's purpose was to demonstrate the logically necessary stages by which human reason approximates the absolute.

\subsubsection{Nietzsche}

Its not clear to me that this is important to critical theory, but it is a line of thinking leading to totalitarian politics.

Notes from Magee and Stern \cite{magee-stern}.
The main traditions in Western Civilisation which Nietzsce attacked were:

\begin{enumerate}
  
\item Christian morality

Dismisses christian virtues such as ``turning the other cheek'', compassion, humility.
  
\item Secular morality

Dismisses the generalised moral codes which occurs in secular moral theory.

\item Herd morality

The heroic individual should be a law unto himself.
  
\item Some traditions from ancient Greece (Socrates)
  
\end{enumerate}


\subsection{roots}

The three pillars which support Social Justice ideologies (collectively ``critical social justice'' of which ``critical race theory'' is perhaps the most controversial) are:

\begin{itemize}
\item Critical Theory
\item Postmodern Philosophy
\item The Long March through the Institutions
\end{itemize}

The first two are philosophical, the last is an activist strategy.

\subsubsection{Critical Theory}

``Critical Theory'' as capitalised refers to the theories of the Frankfurt School, the later more activist descendents (e.g. critical race theory) lose the capitals.

The Frankfurt School was established as the Institute for Social Research by Felix Weil in Frankfurt with an endowment from his father Herman Weil.
At first a group of orthodox Marxists, it developed from 1930 under the leadership of Max Horkheimer its own distinctive ``Critical Theory'' of Society \cite{horkheimer-trad,horkheimer-crit}.


Some aspects of Critical Theory:

\begin{itemize}
\item Integration of Theoretical, Practical and Normative

  Critical Theory must explain what is wrong with current social reality, identify the actors to change it, and provide both clear norms for criticism and achievable practical goals for social transformation.
  
\item Emancipation of the individual
  
\item Democratisation of society

  “all conditions of social life that are controllable by human beings depend on real consensus” in a rational society (Horkheimer \cite{horkheimer-crit})

  \item Intolerance
\end{itemize}

\subsubsection{Postmodern Philosophy}

\section{Other Activist Foci}

\section{Freedom of Speech}

\section{Due Process}

\section{Conflicts of Interest}

\section{The Nature of Democracy and The Risk of Subversion}

\phantomsection
\addcontentsline{toc}{section}{Bibliography}
\bibliographystyle{rbjfmu}
\bibliography{rbj}

%\addcontentsline{toc}{section}{Index}\label{index}
%{\twocolumn[]
%{\small\printindex}}

%\vfill

%\tiny{
%Started 2020/01/17


%\href{http://www.rbjones.com/rbjpub/www/papers/p032.pdf}{http://www.rbjones.com/rbjpub/www/papers/p033.pdf}

%}%tiny

\end{document}

% LocalWords:
