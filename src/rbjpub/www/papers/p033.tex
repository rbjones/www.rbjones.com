% $Id: p033.tex $fi
% bibref{rbjp032} pdfname{p033}
\documentclass[10pt,titlepage]{book}
\usepackage{makeidx}
\newcommand{\ignore}[1]{}
\usepackage{graphicx}
\usepackage[unicode]{hyperref}
\pagestyle{plain}
\usepackage[paperwidth=5.25in,paperheight=8in,hmargin={0.75in,0.5in},vmargin={0.5in,0.5in},includehead,includefoot]{geometry}
\hypersetup{pdfauthor={Roger Bishop Jones}}
\hypersetup{pdftitle={Failing Democracies and How to Fix Them}}
\hypersetup{colorlinks=true, urlcolor=red, citecolor=blue, filecolor=blue, linkcolor=blue}
%\usepackage{html}
\usepackage{paralist}
\usepackage{relsize}
\usepackage{verbatim}
\usepackage{enumerate}
\usepackage{longtable}
\usepackage{url}
\newcommand{\hreg}[2]{\href{#1}{#2}\footnote{\url{#1}}}
\makeindex

\title{\LARGE\bf Failing Democracies \\and \\How to Fix Them}
\author{Roger~Bishop~Jones}
\date{\small 2022/02/09}


\begin{document}
\frontmatter

%\begin{abstract}
%Maybe intended as an essay, but turned out to be a collection of references and notes on some of them.
%I left more or less the original intent in this abstract.
%
% We seem at this moment in history to be unusually well endowed with tales of how our democracies are failing.
% Against these there is "push back".
% To push back you need to spot what is going on, to describe it clearly, to understand and articulate why it is pathological, but also to propose and promote an effective remedy.
% 
% Thinking philosophically about these phenomena may provide more compelling support, for the core values which are threatened, against
% a range of subversive strategies and tactics.
% Here we take democracy as the fundamental value and seek to analyse contemporary erosions and consider what kinds of defence might be mounted.
%An important defence is the light of scrutiny, on the (doubtful) thesis that once we see clearly the threats and the values which they threaten, their strength will be undermined.
%\end{abstract}

\begin{titlepage}
\maketitle

%\vfill

%\begin{centering}

%{\footnotesize
%copyright\ Roger~Bishop~Jones;
%}%footnotesize

%\end{centering}

\end{titlepage}

\ \

\ignore{
\begin{centering}
{}
\end{centering}
}%ignore

\setcounter{tocdepth}{2}
{\parskip-0pt\tableofcontents}

%\listoffigures

\mainmatter

\pagebreak

\section*{Preface}

\addcontentsline{toc}{section}{Preface}

There are ``hyperlinks'' in the PDF version of this monograph which either link to another point in the document  (if coloured blue) or to an internet resource  (if coloured red) giving direct access to the materials referred to (e.g. a Youtube video) if the document is read using some internet connected device.
Important links also appear explicitly in the bibiography.

\chapter{Introduction}

I don't really have much idea how to do this, but I think quite a lot of digging is in order, and I will start by making notes here on what others have uncovered.

\chapter{Some People and their Books}

\section{Akala}

\subsection{Native\cite{akala-native}}

Race and Class in the Ruins of Empire.

\section{Michelle Alexander}

\subsection{The New Jim Crow\cite{alexander-tnjc}}
Mass Incarceration in the Age of Colorblindness.

\section{Bettina Arndt \cite{arndt-home}}

\section{Ayaan Hirsi Ali}

\subsection{The Challenge of Dawa\cite{ali-dawa}}

Political Islam as
Ideology and Movement
and How to Counter It 

\begin{small}
\begin{tabular}{l l}
Executive Summary & 1\\
Summary of Policy Recommendations&5\\
Introduction&9\\
Part I The Constitution of Political Islam&23\\
Part II Dawa: Much More than a “Call to Islam”&35\\
Part III Confronting Ideology to Win the War&51\\
Conclusion&61\\
Detailed Policy Recommendations&65\\
Appendix A: Eight Types of Threat from Radical Islam&79\\
Appendix B: Shay’s Three Joint Pillars of Dawa and Jihad&73\\
Appendix C: Mares’ Five-Step Model of Political Islam’s
Expansion&75\\
Appendix D: Charities and the Terrorist Money Trail&77\\
Glossary&79\\
\end{tabular}
\end{small}

\begin{quote}
{\it
The aim of da’wah and jihaad is not to shed blood, take wealth,
or enslave women and children; these things happen incidentally but are not the aim. This only takes place when the
disbelievers (non-Muslims) refrain from accepting the truth
and persist in disbelief and refuse to be subdued and pay the
jizya (tax levied on free non-Muslims living under Muslim
rule) when it is requested from them. In this case, Allah has
prescribed the Muslims to kill them, take their wealth as booty
and enslave their women and children . . . this religion (Islam)
. . . is superior to every law and system. . . . The truth has been
spread through the correct Islamic da’wah, which in turn has
been aided and supported by jihaad whenever anyone stood in
its way. . . . It was jihaad and da’wah together which helped to
open the doors to victories.
}
\end{quote}
—Saudi Grand Mufti Ibn Baz, 1998

\begin{quote}
{\it
A 2008 survey of more than nine thousand European Muslims
by the Science Center Berlin reported strong belief in a return
to traditional Islam. In the words of the study’s author, Ruud
Koopmans, “almost 60 percent agree that Muslims should return
to the roots of Islam, 75 percent think there is only one interpretation of the Quran possible to which every Muslim should stick,
and 65 percent say that religious rules are more important to
them than the laws of the country in which they live.” More than
half (54 percent) of European Muslims surveyed also believe
that the West is out to destroy Muslim culture.
}
\end{quote}

\section{Peter Boghossian}\label{Boghossian}

We have here a book and a supporting talk.
Talk first.

\subsection{\href{https://www.youtube.com/watch?v=LiymUd9FjHA}{The Way Forward}}

There is a prelude which is about critical social justice.

\begin{itemize}
\item[9:58] Start by listening: you have to understand what your target is talking about.
  Don't talk over him, concede if there is a clash. (go, no you go)
  Say ``I'm not sure I understand'' rather than complaining of unclarity.
  Say ``I hear you'' and mean it.
  Echo snips.
  Attempt to re-express targets position clearly.
\item[14:29] ``Epistemology'': ask how they know it?
\item[16:28] Scales: how confident are you in that belief? (1:10 or whatever)
  Ask again after (pre-test/post-test).
  Ask ``how much of ...'' (e.g. of a patriarchy).
\item[19:24] Disconfirmation: under what conditions could you be wrong?
  (what evidence would convince you otherwise).
  If can't be wrong, then ``oh, not based on evidence?''.
  You need to now from them what evidence would change mind, what would count for youis not at issue.
\item[23:00] Ask a question to facilitate doubt.
  e.g. if gender studies and biology disagree, who would you be likely to believe mkore?
\item[24:10] Don't provoke defensiveness: Don't say ``but'', it denies everything that has gone before, and sets target into defensive posture.
  Say ``Yes and ...''.
\item[25:35] Build bridges: make golden bridges to save face.
\item[26.35] Don't apologise: unless you are actually sorry. Not if someone is offended.  AS long as criticising ideas rather than persons.
\item[27:35] Be sincere: demonstrate parahesia, don't equivocate or sugarcoat.
  Speaking in unclear language is a type of self-deception.
  \item[29] Be willing to revise your beliefs.  Be willing to say ``I don't know''.
\end{itemize}

Then he runs over it again.

\subsection{How to Have Impossible Conversations: A Very Practical Guide\cite{boghossian-conversations}}

\cite{boghossian-manual}

\subsubsection{Where they are coming from:}

\begin{quote}
  We know because we’ve had countless conversations with zealots, criminals, religious fanatics, and extremists of all stripes.
  Peter did his doctoral research in the Oregon State Prison System conversing with offenders about some of life’s most difficult questions, and then built upon those techniques in thousands of hours of conversations with religious hardliners.
  James developed the ideas for his books and articles by engaging in extended conversations with people who hold radically different views about politics, morality, and religion.
  This book is the culmination of our extensive research and a lifetime of experience in conversing with people who profess to be unshakable in their beliefs.
\end{quote}

\subsubsection{They offer:}

\begin{quote}
thirty-six techniques drawn from the best, most effective research on applied epistemology, hostage and professional negotiations, cult exiting, subdisciplines of psychology, and more.
\end{quote}

\subsubsection{Partial Contents:}

\begin{tabular}{l | p{9cm}}
one & When Conversations Seem Impossible \\\hline
two & The Seven Fundamentals of Good Conversations \\\hline
\#1 & GOALS Why are you engaged in this conversation? \\
\#2 & PARTNERSHIPS Be partners, not adversaries \\
\#3 & RAPPORT Develop and maintain a good connection \\
\#4 & LISTEN Listen more, talk less \\
\#5 & SHOOT THE MESSENGER Don’t deliver your truth \\
\#6 & INTENTIONS \\
& People have better intentions than you think \\
\#7 & WALK AWAY\\\hline
three & Beginner Level: Nine Ways to Start Changing Minds \\\hline
\#1 & MODELING Model the behavior you want to see in others \\
\#2 & WORDS Define terms up front \\
\#3 & ASK QUESTIONS Focus on a specific question \\
\#4 & ACKNOWLEDGE EXTREMISTS\\
&Point out bad things people on your side do \\
\#5 & NAVIGATING SOCIAL MEDIA \\
&Do not vent on social media \\
\#6 & DON’T BLAME, DO DISCUSS CONTRIBUTIONS \\
&Shift from blame to contribution \\
\#7 & FOCUS ON EPISTEMOLOGY \\
& Figure out how people know what they claim to know \\
\#8 & LEARN Learn what makes someone close-minded \\
\#9 & WHAT NOT TO DO (REVERSE APPLICATIONS)\\
& A list of fundamental and basic conversational mistakes \\\hline
four & Intermediate Level: Seven Ways to Improve Your Interventions \\\hline
five & Five Advanced Skills for Contentious Conversations \\\hline
six & Six Expert Skills to Engage the Close-Minded \\\hline
seven & Master Level: Two Keys to Conversing with Ideologues \\
\end{tabular}

\section{Ta-Nehisi Coates}

\subsection{\cite{coatestnh-bwm,coatestnh-wweyip}}

\section{Combahee River Collective}\label{Combahee}

\subsection{Statement (1977) \cite{combahee-statement}}

\section{Kimberlé Crenshaw}\label{KC}

\subsection{Critical Race Theory: The Key Writings that Formed the Movement \cite{crenshaw1995critical}}

\section{Paulo Friere}\label{Friere}

\subsection{The Politics of Education \cite{friere-poled}}

Culture, Power and Liberation

\section{Henry Giroux}\label{Giroux}

\subsection{On Critical Pedagogy \cite{giroux-critped}}

\pagebreak

\section{Isaac Gottesman}\label{Gottesman}

\subsection{The Critical Turn in Education \cite{gottesman-criturn}}

From Marxist Critique to Poststructuralist Feminism to Critical Theories of Race (Critical Social Thought).

Critical Pedagogy aims to transform the education system so that it teaches in a way consistent with the Critical Theory principle that academic studies or research should be fully integrated with social activism.
Thus pedagogy should aim to transform the education system, and should aim to make the system effective in the emancipation of oppressed groups of every kind.
It should do this by teaching students to do just that, from the very beginnings of their education and through into adulthood, children should be educated as critical social justice activists.

This book provides a historical account of the development and the course of Critical Pedagogy in its long March Through the (education) Institutions (see Sidwell for the broader picture of that revolutionary strategy, section \ref{Sidwell}) up to 2016.

\subsubsection{Series Editor's Introduction}

The series is the ``The Critical Social Thought Series'', the series editor is Michael W. Apple.
Bear in mind here that the ``Critical'' adjective here is a reference to the post-Marxist ``Critical Theory'' initiated at the Frankfurt School and continuously developed through to the present day in applied critical theories and critical social justice activism.

His introduction is clear about the nature of the enterprise (Critical Education Theory).

He begins with an anecdote (omitted here) which leads him to infer:

\begin{itemize}
\item that Critical Education has been incorporated into official programs in education throughout the world,

  and that
  
\item in the process, it may have lost connection with its political roots. That ``rather than politicizing the academic, it academicizes the political''.
\end{itemize}

(I find that difficult to believe, given the nature of the content, but I can quite believe that it may not have been inspiring political activism quite as vigorous as hoped.)
It is notable that awareness of the effects of Critical Theory in the various dimensions of ``identity politics'' seems to have ballooned in the period after publication of this book, so he may have written in the calm before the storm.

Apple gives a dense and detailed characterisation of critical education along the following lines:

\begin{itemize}
\item First, critical education aims to ``expose'' how power and inequality appears and is challenged in education.
\item Beyond that (in its most ``robust'' form) it includes a reconstruction of:
  \begin{itemize}
  \item the purpose of education
  \item its methods
  \item what is taught
  \item who should be allowed to teach
  \end{itemize}
\item and should undertake ``fundamental transformations'' of epistemological and ideological assumptions, and
  \item commit to social transformation (to secure social justice).
\end{itemize}

Apple talks about his book ``Can Education Change Society?'' in which he details tasks for critical education scholars to undertake, and enumerates them along the following lines (grossly oversimplified!):

\begin{enumerate}

\item Connect educational policy with relations of exploitation and domination.
  
\item Point to possible spaces for action to effect ``counter hegemonic'' change.
  
\item Broaden the idea of ``research'' to assist those challenging relations of unequal power.
  
\item Assist communities in engaging in dialogues which further decisions made in the interests of the oppressed.
  
\item Keep the traditions of radical and progressive work alive.
  
\item Ask who this is for, and how and in what form it is to be provided to them.
  Develop the skills to deliver to diverse audiences.
  
\item Act in concert with progressive social movements and against ``rightist'' assumptions and policies.
  
\item Act as a deeply committed mentor.

\item Use your privilege to open spaces to those who do not have a voice.
  
\end{enumerate}

In his final paragraphs Apple (inter alia) mentions Gottesman's account of the progress of critical education theory from a critique of the relationship between education and class dynamics to address also race, gender, sexuality and their intersections using both structural and poststructural approaches, embracing class theories, poststructural feminist approaches and critical race theory.

\subsubsection{(Authors) Introduction}

Gottesman begins by quoting Paul Buhle from his ``Marxism in the United States''\cite{buhle-histmarx}:

\begin{quotation}
  ``To the question `Where did all the sixties radicals go?', the most accurate answer would be: neither to religous cults nor yuppiedom, but to the classroom.'' (p.263)
  \end{quotation}
and identifies as an ``intellectual anchor'' what he calls ``the critical marxist tradition'' (aka Western Marxism, neo-Marxism, cultural Marxism).

He then identifies the turn to ``critical Marxist'' thought as the defining moment in the previous 40 years of educational scholarship.
Interesting perhaps that he does not explicitly make the connection with the Critical Theory of the Frankfurt School, though that does seem to be the fountainhead.
He identifies the following social justice ideas as coming from the critical turn:

\begin{itemize}
\item hegemony
\item ideology
\item consciousness
\item praxis
  and
  \item the word 'critical'
  \end{itemize}

and talks about critical educational scholarship as progressing beyond that tradition by embracing intellectual and political traditions addressing:

\begin{itemize}
\item culture and identity
\item gender and sexuality
\item race and ethnicity
\item constructions and ability
\item ecological crisis
  and
\item their intersections  
\end{itemize}

He does not appear to think of these as being part of ``Marxism''.
But they are all part of the trajectory of the ``critical Marxism'' which started with Critical Theory in the Frankfurt school of the 1930s and progressed through Postmoderism to various applied critical theories addressing the above 'tradition', eventually combined, through the theory of intersectionality, into Critical Social Justice, the expansion of Marxism from the economic domain to every conceivable dimension of disadvantage interpreted as oppression.%
\footnote{See the list of applied postmodernism theories in the notes on ``Cynical Theories'' in section \ref{CT}, which corresponds fairly closely to the above list.}

Critical education seems to have tracked the main developments in applied critical Marxism over the last 40+ years with the mission of turning the whole of education into an indoctrination program for social justice activists.

He also notes and celebrates the methodological impact of the critical turn on educational scholarship, but expresses some concerns about over-use of the term ``critical'' in education, and about a dilution of standards and a lack of that cohesion and unity necessary in a movement for radical social change.

Gottesman advocates that educational scholarship should aim not merely for academic theories nor even simply for radical transformation to education, but for radical social change facilitated by a critical turn in the practice of education.
This involves nuanced and sophisticated social theory and strategic political advocacy,  which includes a push for a feminist, anti-racist, democratic-socialist society, which can forcefully push against ideologies that support entrenched patriarchy, white supremacy and capitalism.

\paragraph{Historically Informed Criticism}

The book is intended as ``historically informed criticism'' in the history of ideas, which requires the historian to consider the enduring nature of specific ideas so as to draw out their contemporary significance, i.e. to:

\begin{enumerate}
\item show how critical educational ideas emerged and developed in particular contexts,

  and
  
\item illuminate their enduring nature and show how they provide insight into present day educational and social struggles.
  
\end{enumerate}

Gottesman expresses particular interest in the role of intellectual and political traditions in the development of radical ideas.

\paragraph{The Details}

The book is not intended as a comprehensive history of critical education theory.
It aims to present the key developments in the field, particularly in relation to the role of schooling, and education more broadly, in radical social change.
It is also concerned to place these in the context of intellectual and political traditions beyond academia.

Each chapter presents a historical stage in the development, with reference to the primary texts of the period (of which are recommended concurrent reading), and provides contemporary analysis.

\begin{itemize}
\item [Chapter 1 -]
  Paulo Friere \emph{Pedagogy of the Opporessed} (1970), (according to Lindsay, the third most-cited work in the social sciences and humanities in the history of the world).
  Contributed to Marxist political theory that the process of education must be at the centre of radical movement building.
\item [Chapter 2 -]
  Samual Bowles and Herbert Gintis - \emph{Schooling in Capitalist America} (1976).
  Pushing against capitalism and fostering the building of mass social movements.
\item [Chapter 3 - 70s]
  Michael Apple - \emph{Ideology and Curriculum} (1979).  Drawing on Gramsci's work on ideology and hegemony.
  Intent to examine and eliminate dominant and alienating practices of schooling.
\item [Chapter 4 - late 70s and 80s]
  Henry Giroux - critical pedagogy (construed narrowly by Gottesman).
\item [Chapter 5 - \~1990]
  Feminist ideas, situated knowledge, standpoint epistemology influenced by postmodernism and poststructuralism.
  Elisabeth Elseworth (critique of critical pedagogy), Kathleen Weiler (critique of Paulo Friere), Patti Lather (research methodology).
\item [Chapter 6 - late 1990s]
  Critical theories of race. (CRT, et.al.)
\end{itemize}

\pagebreak

\section{Antonio Gramsci}

\begin{quotation}“Socialism is precisely the religion that must overwhelm Christianity. … In the new order, Socialism will triumph by first capturing the culture via infiltration of schools, universities, churches and the media by transforming the consciousness of society.”
  \end{quotation}
~ Antonio Gramsci, Sotto La Mole 1916-1920.

\subsection{Prison Notebooks \cite{gramsci-notes}}

The idea of cultural hegemony, and the seeds of the long march through the institutions.

\section{Christopher Hitchens}

\subsection{Why Orwell Matters\cite{hitchens-wom}}

\section{Max Horkheimer}

Page numbers shown below in square brackets are from: \cite{horkheimer-crit}.

\subsection{Traditional and Critical Theory}

In this essay \cite{horkheimer-trad, horkheimer-crit} dated 1937 (when Horkheimer and the Frankfurt School would have been at Columbia University in New York), Horkheimer begins with an account of what ``Theory'' is.
The account is remarkably consonant with Aristotle's conception of demonstrative science, but it is Descartes' \emph{Discourse on Method}\cite{descartesDOM} to which Horkheimer refers and which he quotes.

Theory is a body of propositions, some ``primary principles'' others derived from them, which are consonant with ``the actual facts''.
Various elaborations upon this are touched upon.
The increasingly dominant role of mathematics in such scientific theories is noted.

This conception of theory pre-eminent in the natural sciences, is also practiced in the social sciences, and the difference between social science conducted primarily by empirical investigation and that of (German) social scientists who analyse basic concepts and formulate fundamental principles as an ``armchair'' activity does not consist in distinct conceptions of ``theory''.
The differences between the social scientists who prefer an empirical approach and those who prefer a theoretical approach (T\"{o}nnies, Durkheim, Weber) is further discussed.

From here Horkheimer introduces the significance of historical and social context in the formulation of theories.
The Social Scientists ..
\begin{quote}
.. believe they are acting according to personal
determinations, whereas in fact even in their most complicated
calculations they but exemplify the working of an incalculable
social mechanism. [p197]
\end{quote}

\begin{quote}
  [Critical Theory] never aims simply at an increase of knowledge as such. Its goal is man's emancipation from slavery. [p246]
\end{quote}

\section{David Horowitz}

\cite{horowitz-paau,horowitz-bbal}

\section{Jonathan Israel \cite{israel2011revolution,israel2013democratic,israel2006enlightenment,israel2002radical}.}

A historian specialising in The Enlightenment:
\cite{israel2011revolution,israel2013democratic,israel2006enlightenment,israel2002radical}.

Israel's work may be thought of as the historical side of a subject which has been called by the philosopher Isaiah Berlin (section \ref{IsaiahBerlin) ``the history of ideas''}.
Whereas Berlin looks at the history of The Enlightenment and the subsequent period of romanticism from a philosophical perspective and distils as best he can certain central threads which a philosopher might grapple with, Israel revel's in the diversity and exposes conventional oversimplified accounts as omitting important parts of the mosaic, but nevertheless considers himself to be engaged in `history of ideas'.

His most distinctive and controversial contribution to the historical scholarship seems to have been his separation of enlightenment thought into two major and quite distinct currents, the moderate enlightenment which had hitherto perhaps had the attention and was embodied by Locke and Voltaire (among many others), and a radical enlightenment born in the relative freedom of the low countries in the seventeenth Century with a rationalistic philosophical spine running through Descartes, Spinoza, Leibniz and Wolf.
It is Spinoza who seems to have been the most influential, Spinozism securing a prime place amount the literatures printed in Holland for distribution throughout Europe despite persistent supression throughout.


\section{Helen Joyce}

\subsection{Trans \cite{joyce2021}}
Here follow the first few words of each chapter quoted from the book.
With apologies to the author if this goes beyond ``fair usage''.
There is available on Amazon a much more substantial free sample, but that just runs from the start, and I thought a few words from each chapter would give a fuller initial impression.

\subsubsection{Introduction}

This is a book about an idea, one that seems simple but has far-reaching consequences. The idea is that people should count as men or women according to how they feel and what they declare, instead of their biology. It’s called gender self-identification, and it is the central tenet of a fast-developing belief system which sees everyone as possessing a gender identity that may or may not match the body in which it is housed. When there is a mismatch, the person is ‘transgender’ – trans for short – and it is the identity, not the body, that should determine how everyone else sees and treats them. The origins of this belief system date back almost a century, to when doctors first sought to give physical form to the yearnings of a handful of people who longed to change sex.
...

\subsubsection{1 The Danish Girls}

A brief history of transsexuality

It began with stockings. Gerda’s sitter, the actress Anna Larssen, had telephoned to say she was running late for her portrait. Why not use Gerda’s husband Einar, Anna suggested teasingly, as a substitute? After all, his legs were as good as Anna’s. ‘The most perfect ladies’ model!’ cried Gerda, when she saw Einar transformed into . . . whom? ‘What do you say to Lili?’ asked Anna, when she finally joined: ‘A particularly lovely, musical name.’ Whether this is truth or later mythmaking is impossible to tell. But certainly Einar Wegener – an artist born in 1882 and trained in Copenhagen, and the Danish girl of the eponymous 2015 film starring Eddie Redmayne – dated the birth of Lili Elbe (the surname was inspired by the river) to that ‘extravagant joke’. For years afterwards, Einar brought her out for portraits and parties. Hardly anyone knew that Gerda’s sultry, sloe-eyed model was her cross-dressing husband.
...

\subsubsection{2 Sissy Boys and the Woman Inside}

Why some men want to be women, and why some people don’t want you to know

‘He got in the girls’ line instead of the boys’ line at the drinking fountain . . . He was playing with dolls, playing dress-up . . . he loves jewelry . . . his favourite characters are Cinderella [and] Snow White . . . he talks like a girl, sometimes walks like a girl, acts like a girl . . . he’s standing in front of the mirror and he took his penis and he folded it under, and he said, “Look, Mommy, I’m a girl.” ’

These words come from parents’ descriptions of their sons in a landmark fifteen-year study that began in the 1960s by Richard Green, an American doctor and lawyer who spent much of his career in gender medicine. At the time, and for many years after, no gender clinic saw children as patients, but Green wanted to answer a question that had intrigued doctors ever since they became aware of men who said that they were really women: had they always been that way? Were there little boys who insisted they were really girls, and if so, did they grow up to be transsexuals? Or, in the phrase that was now starting to be used, was ‘gender identity’ formed in early childhood, or perhaps even innate?
...

\subsubsection{3 My Name is Neo}

Gender-identity ideology 101

‘You’re here because you know something. What you know you can’t explain, but you feel it. You’ve felt it your entire life, that there’s something wrong with the world. You don’t know what it is, but it’s there, like a splinter in your mind, driving you mad.’ With these words the mysterious Morpheus tells Thomas Anderson, aka Neo, the hero of The Matrix, that his life is a sham. The film, released in 1999, has been interpreted in many ways, including as religious allegory, a vision of an online future and an expression of teenage alienation. But many trans people regard it as expressing their experiences. Some gender therapists even prescribe it as viewing for their clients’ families. It was written, produced and directed by the Wachowski siblings, both of whom were born male and came out as transwomen after its release. In 2020, Lilly, the younger and second to transition, confirmed that it was a ‘trans metaphor’. This chapter will use the film to explain gender-identity ideology. Its characters represent the figures that stalk transactivists’ discourse, from transphobes to detransitioners, and its premise and plot illuminate their worldview.
...

\subsubsection{4 Child, Interrupted}

The catastrophic consequences of an adult ideology for gender-dysphoric minors

In this chapter, I pick up the story of gender-dysphoric children started in chapter 2, and look at what the theory of innate gender identity means for them. When Richard Green and others first studied them in the 1970s and 1980s, no clinician believed their feelings actually made them members of the opposite sex, or dreamed of treating them with drugs or surgery. They simply sought to predict how those children would feel as adults. And every study pointed to the same conclusion: they were pretty likely to grow up gay, and very unlikely to still identify as or want to be members of the opposite sex. There was no such thing as a ‘trans child’ in the sense of one who could be identified as certain, or even highly likely, to grow up to be a trans adult. But these facts contradict gender-identity ideology. So nowadays, they are ignored. The identity claims of gender-dysphoric children are taken at face value and even the possibility of desistance is denied. Paradoxically, an ideology that holds that physiologically normal males can be every bit as much women as people born female, and vice versa, is used to justify children being put on a path to surgery and sterility.
...

\subsubsection{5 Miss Gendering}

Why teenage girls are identifying out of the prospect of womanhood

Keira Bell not only forced clinicians to rethink their treatment of gender-dysphoric minors; she also provided an example of how that group is changing. Until the past decade, hardly any teenage girls sought treatment for gender dysphoria; now, they predominate in clinics around the world. British figures are typical. In 1989, when the Tavistock clinic opened, there were two referrals, both young boys. By 2020, there were 2,378 referrals, almost three-quarters of them girls, and most of those teenagers. Their treatment according to the gender-affirmative model is compounding a medical scandal. Moreover, as anyone familiar with schools in liberal towns and cities will know, many more girls are identifying out of their sex without ever coming to the attention of gender doctors. Some identify as boys; others as non-binary, gender-fluid, demi-boys or suchlike. They ask to be referred to as ‘he/him’ or ‘they/them’, or by novel pronouns such as ‘xie/xir’ – in other words, as anything but female. This chapter looks at what is driving girls to abandon their sex. The story has three strands: female sexuality, modern feminism and, finally, something this group is particularly prone to – social contagion.
...

\subsubsection{6 Back in the Box}

How gender-identity ideology harms all children

‘For as long as I can remember, my favourite colour has been pink,’ starts the children’s book I Am Jazz, by Jazz Jennings. Jazz’s favourite activities are mostly girly: dancing, singing, backflips, drawing, swimming, putting on make-up and pretending to be a pop star. Jazz’s parents are so puzzled that they consult a doctor, who explains that they are mistaken in thinking they have a son: Jazz ‘has a girl brain but a boy body. This is called transgender.’ I Am Jazz and similar books are widely recommended by many activist groups for reading in schools. The stereotypes they promote teach children ideas about what is proper for boys and girls that feminists had thought consigned to the dustbin of history. This is just one of the ways in which gender-identity ideology harms all children, not merely those who end up identifying out of their sex.
...

\subsubsection{7 She Who Must Not Be Named}

How gender-identity ideology erases women

The body-denialism at the heart of gender-identity ideology is harmful for all humans, since we are in fact embodied creatures. But it is especially harmful for women, since female bodies impose costs and make demands in ways that male ones don’t. It is female bodies that bear almost all the burden of reproduction, and ignoring that fact doesn’t change it; it merely muddles thinking about how to arrange society to accommodate reproduction while ensuring that women can live full, self-actualised lives. And it is also especially tempting for women, because throughout history women have been objectified and reduced to their bodies, with men as subject, occupying the realms of mind and intellect. ‘The body has been made so problematic for women that it has often seemed easier to shrug it off and travel as a disembodied spirit,’ wrote Adrienne Rich in 1976 in Of Woman Born, her book about motherhood.
...

\subsubsection{8 We Just Need to Pee}

Why female-only spaces matter so much for women

In 2018, Shelah Poyer, a beautician in Vancouver, started to earn extra money by seeing clients at home. She accepted only women as clients, for reasons of safety and because some services, like Brazilian waxing, cannot be performed on men. So when she received a message on Facebook Marketplace from Jonathan Yaniv, whose profile picture was as male as the name, she replied: Not for men, sorry! ‘I’m a woman,’ Yaniv replied. ‘I transitioned last year.’ Poyer was nonplussed. Truth be told, she didn’t feel any better about having a male who identified as a woman in her home than about any other male. And she didn’t know what she was being asked to do. ‘I wanted to ask about surgery, but how to ask without being offensive?’ she says. Waxing male genitalia is a specialised service (testicles have thinner and looser skin than vulvas, and treating them similarly would cause tearing). She stopped responding, but the messages kept coming, and then phone calls to her salon. Now she was spooked.
...

\subsubsection{9 Folding Like Deckchairs}

How gender self-identification threatens to destroy women’s sports

In 2019, the BBC published a heart-warming piece on its website about Kelly Morgan, a rugby player who – as the title of the piece had it – ‘play[s] with a smile on my face’. Anyone who read on would have learnt some more striking facts. Morgan, who plays for Port Harlequin Ladies Club in Wales, had broken the coach’s ankle during a game of touch rugby – though he seemed remarkably sanguine about it, quipping that Morgan would be a ‘good, good player for the next few years, as long as we can stop her injuring players in training’. The risks Morgan posed to other players were a matter of humour to the club’s captain, too. She laughingly recalled Morgan folding a player on an opposing team ‘like a deckchair’. The reason Morgan – born Nicholas Gareth Morgan, and a fixture on East Wales boys’ teams as a teenager until being injured – could play for a women’s team was that World Rugby, like most sporting authorities, had followed the lead of the International Olympic Committee (IOC) in allowing males to compete as women once they had suppressed their testosterone for a year.
...

\subsubsection{10 Regardless of Sex}

The American Left’s embrace of gender self-identification

‘It passed, it passed! I’m ecstatic,’ says a teenager in a Queen t-shirt, with asymmetrical pink hair. A burly, middle-aged man in a suit – a spokesman for the ACLU – hurries over for a hug. The camera cuts to another teenager, whose eyes are shiny with tears. ‘I feel uncomfortable that my privacy is being invaded,’ she says. ‘As I am a swimmer, I do change multiple times, naked, in front of the other students in the locker room. I understand that the board has an obligation to all students, but I was hoping that they would go about this in a different way that would also accommodate students such as myself.’ The scene is a school in suburban Chicago in late 2019. The board of School District 211 has just voted to allow trans students unrestricted access to private facilities corresponding to their stated identities, rather than their sex. Transgirl Nova Maday – the teenager with the pink hair – recently graduated, having fought to be allowed to use the female toilets and changing rooms since starting to identify as a girl at age fifteen. The other teenager, Julia Burca, is on the swim team. She has just watched the board, which fought to maintain single-sex spaces for several years, give in.
...

\subsubsection{11 Behind the Scenes}

Transactivism’s long march through the institutions

For a movement that is supposedly about the latest oppressed minority gaining full human rights, transactivism has progressed remarkably far and fast. Usually, civil-rights movements start by winning hearts and minds, and that takes time. You might think it a good thing that such delay seems a thing of the past. In fact, this is a major indication that transactivism is not a civil-rights movement at all. Consider the movements commonly claimed as its forerunners. Three decades passed between the AIDS crisis that re-galvanised the gay-liberation movement and the Supreme Court’s ruling that legalised same-sex marriage across the US. In 1996 the share of Americans supporting gay marriage was barely a quarter. It did not become a majority until 2011. Similarly, the movements to enfranchise women and to end segregation in the American South had to be built from the ground up. Campaigners gave speeches and held rallies to raise awareness and win supporters. Solid majorities had to favour the social and legal shifts these groups demanded before politicians and judges implemented them. What same-sex marriage, women’s franchise and the end of segregation all have in common is that they extend the rights of a privileged group to everyone.
...

\subsubsection{12 Through the Looking Glass}

How transactivism is chipping away at civil society

As human-rights organisations have embraced gender-identity ideology, they have adopted policies that harm the most disadvantaged, all the while spouting the language of intersectionality. And they have abandoned the constituencies they were founded to fight for. In the US, this trend is so pronounced that anyone on the Left who opposes gender self-identification has to seek allies on the Right – and be dismissed as a sell-out – or accept not being heard at all. Established women’s groups are the most obvious culprits. If they had stood up for women’s right to single-sex spaces and services, gender self-ID could never have made such inroads. Instead, as they adopted a postmodern, ‘woke’ style of feminism, they abandoned the women who needed them most. It is remarkable that in the US – the only developed country lacking paid maternity leave and universal health care, and where women’s reproductive rights are under constant attack – so-called feminists have prioritised the demands of transwomen, that is, of males. Poor women and girls of colour, who are more likely to attend state schools, to need homeless and rape-crisis shelters, or to fall victim to the war on drugs and end up in prison, depended on feminists to stand up for single-sex spaces.
...

\subsubsection{13 They Can’t Fire Us All}

How British women are starting to fight back

In 1925, John Scopes, a teacher in Tennessee, was charged with the misdemeanour of teaching the theory of evolution. William Jennings Bryan, a three-time presidential candidate, made the case for the prosecution. Scopes lost, and teaching evolution remained banned in Tennessee for another forty years. The ‘Scopes monkey trial’ became a byword for science denialism and the battle between faith and reason. Fast-forward almost a century to see the modern-day equivalent play out in an employment tribunal in London, in late 2019. Like the Scopes trial, it was a depressing reminder of the power of a state-backed belief system to compel citizens’ actions and speech, and to punish those who dissent. It was also the start of a fightback against gender-identity ideology that is gathering pace in the UK, and inspiring similar movements around the world. The protagonist is Maya Forstater, who has lost her job in the London office of the Center for Global Development, a think-tank headquartered in Washington. It objected to her stating a scientific fact as incontrovertible as evolution – that in humans, male and female are distinct, immutable categories – and adding that she regarded acknowledging this fact as essential to protecting women’s rights.
...

\subsubsection{Conclusion}

Trans Rights Are Human Rights

Where do we go from here? One of the jobs of social scientists and political analysts is to predict how public opinion on a controversial issue might evolve. To do this, they sometimes build a model grouping people with similar attitudes, and consider the forces or trends that lead to movement between those groups. To think about how attitudes towards gender self-identification might develop, let’s start with five groups. Four already have some opinion on the issue: enthusiasts; ‘lukewarms’ who know little about it but are positively disposed; those who are equally ignorant but suspicious; and committed opponents. And finally, probably the largest group comprises those who are oblivious to the issue. Here’s how an enthusiast might interpret the current situation and expect things to unfold. The committed opponents are simply bigots. Silencing them will make it less likely that others join them and more likely that trans people will come out, thereby increasing societal acceptance. The people who started in the suspicious or oblivious groups will gradually become lukewarms; the lukewarms will become enthusiasts. Opposition will become ever more stigmatised, and eventually gender self-ID will gain widespread support.


\section{Ernesto Laclau}

\subsection{Hegemony And Socialist Strategy: Towards A Radical Democratic Politics \cite{laclau-hegsoc}}

\section{James Lindsay}

James Lindsay, with a doctorate in mathematics, veered into the social sciences by participating in the `grievance studies hoaxes', which he undertook with Helen Pluckrose (Section \ref{Pluckrose}) and Peter Boghossian.
This followed the early hoax paper which resulted in `The Sokal Affair' \cite{sokal1998} in exposing the academic standards of a selection of supposedly scholarly peer eviewed journals, showing that often the primary criterion for publication was the inclusion of the right buzzwords, but not rigourous evidence based research.

Lindsay and Pluckrose then looked into the history of the ideas behind these disciplines, tracing them back through to the French Postmodern philosophers.
This trail, from postmodernism through academia and on to identity oriented social activism was documented in 'Cynical Theories' \cite{pluckrose-cynical}.
Helen Pluckrose then seems to have focussed her energy on an organisation helping people to cope with adverse impacts on their lives of the activism most conspicuously seen in Critical Race Theory and Gender Ideology, while Linsday kept on digging.

Lindsay clearly thought that stopping with Postmodern Philosophy was leaving out some of the most important contributors to these ideas, and devoted himself to much more comprehensive research into the origins and to exposing the present day manifestations and their origins, in the `critical theory' of the Frankfurt School, and back through Marx and Hegel even as far as Rousseau.
This trail was published by Lindsay in his own web resource, `New Discourses' as videos, essays and a dictionary/encyclopaedia of the relevant terminology.
Along the way publishing books on a couple of aspects of the phenomenon: \emph{Race Marxism} \cite{lindsay-racemarx} and \emph{The Marxification of Education} \cite{lindsay-marxedu},

See also: \emph{Everybody is Wrong About God} \cite{lindsay-everybody},
with Boghossian \emph{How to have Difficult Conversations} \cite{boghossian-conversations}
with Pincourt \emph{Counter Wokecraft} \cite{pincourt-cw}
and, with Pluckrose (section \ref{Pluckrose}),
and \emph{Social (in)justice} \cite{pluckrose-socinj})

\subsection{The Marxification of Education \cite{lindsay-marxedu}}


\subsection{Race Marxism \cite{lindsay-racemarx}}

There is materal on the New Discourses podcasts \cite{lindsay-discourses} about this book, so some of what I say below might possibly derive from that rather than the book.
Just on the basis of his describing how he came to this I have an idea why CRT can be seen as the whole bag rather than just one element of the broader assault which is critical social justice.

\subsubsection{Introduction}

This book was draften in preparation for a set of talks that Lindsay undertook, and reflects his finally concluding that critical race theory \emph{is} Race Marxism.

There is quite a bit of extended history, which is of interest since the further back we go the more likely that history is of the whole of critical social justice.

Whereas his first book in this area, with Pluckrose \cite{pluckrose-cynical}, traces back critical social justice to French postmodern philosophy (not much more than 50 years), this book takes two steps futher. first back to Marx and then to Rousseau.

\subsubsection{Defining Critical Race Theory}

\subsubsection{What Critical Race Theory Believes}

\subsubsection{The Proximate Ideological Origins of Critical Race Theory}

\subsubsection{The Deep Ideological Origins of Critical Race Theory}

\subsubsection{Critical Race Praxis - How Critical Race Theory Operates}

\subsubsection{What Can We Do About Critical Race Theory}

\subsubsection{Conclusion}

\subsection{Does Lindsay ``strawman'' his adversaries?}

\ignore{

This has been alleged.
While not by any means constituting an authority on Lindsay, the accusation strikes me as so far off the mark that I wanted to respond with a short twitter thread.

This section is to play with that idea, it might not get anywhere, its probably too hard to do it in a small number of words.

}%ignore

\section{Heather Mac Donald \cite{macdonald-bbi, macdonald-woc, macdonald-tdd}}

Heather Mac Donald has written against some of the controversial trends of the 21st Century, notably `The Burden of Bad Ideas: How Modern Intellectuals Misshape Our Society' \cite{macdonald-bbi}, `The War on Cops: : How the New Attack
on Law and Order Makes Everyone Less Safe.' \cite{macdonald-woc}, and `The Diversity Delusion: How Race and
Gender Pandering Corrupt the University and Undermine Our
Culture.' \cite{macdonald-tdd}.

\section{Herbert Marcuse}

See Marxists Internet Archive\cite{marcuse-mia}.

\subsection{Eros and Civilisation (1956) \cite{marcuse1956eros}}

\subsection{One Dimensional Man (1964) \cite{marcuse-one-dim}}

Possibly his most influential work.

\subsubsection{Introduction: The Paralysis of Criticism: Society without Opposition}
\subsubsection{Part 1 – One-Dimensional Society}
\paragraph{1 – The New Forms of Control}
\paragraph{2 – The Closing of the Political Universe}
\paragraph{3 – The Conquest of the Unhappy Consciousness: Repressive Desublimation}
\paragraph{4 – The Closing of the Universe of Discourse}
\subsubsection{Part 2 – One-Dimensional Thought}
\paragraph{5 – Negative Thinking: The Defeated Logic of Protest}
\paragraph{6 – From Negative to Positive Thinking: The Logic of Domination}
\paragraph{7 – The Triumph of Positive Thinking: One-Dimensional Philosophy}
\subsubsection{Part 3 – The Chance of the Alternatives}
\paragraph{8 – The Historical Commitment of Philosophy}
\paragraph{9 – The Catastrophe of Liberation}
\paragraph{10 – Conclusion}
\subsubsection{Notes}


\subsection{Repressive Tolerance (1965)\cite{marcuse-repressive}}

An elaborate inversion of Popper's ``Paradox of Intolerance''.

e.g.:

\begin{quotation}
"The whole post-fascist period is one of clear and present danger. Consequently, true pacification requires the withdrawal of tolerance before the deed, at the stage of communication in word, print, and picture. Such extreme suspension of the right of free speech and free assembly is indeed justified only if the whole of society is in extreme danger. I maintain that our society is in such an emergency situation, and that it has become the normal state of affairs,"
\end{quotation}

  
\subsection{Essay on Liberation (1969)\cite{marcuse-liberation}}

\subsubsection{Preface and Introduction}

Marcuse begins by setting the context as a confrontation between the economic and military force of corporate capitalism which exhibits global dominance and opposing forces of the socialist orbit.
In this context the development of socialism is deflected by aspirations to values and wealth exemplified in the American standard of living.

This is now changing, not as a different route to socialism, but as different goals and values for socialists.


\subsubsection{A Biological Foundation for Socialism?}
\subsubsection{The New Sensibility}
\subsubsection{Subverting Forces – in Transition}
\subsubsection{Solidarity}

\subsection{Magee Interview \cite{marcuse-magee}}

A 1977 interview of Marcuse by Brian Magee, just two years before his death.
When I first watched this, mostly oblivious to his historical role despite having read \emph{One Dimensional Man} many years before, he seemed moderate and reasonable.

Factors demanding reconstruction of Marxism:

\begin{itemize}
\item The rise of Fascism
\item The concept of socialism itself.
  In Marx's conception of socialism full-time alienated labor would no longer be the measure of wealth and value.
  The idea of a socialist society as one in which life does not involve alienated labour has disappeared.
\item Magee suggestions:
  \begin{enumerate}
  \item insufficiently libertarian
    \item didn't take sufficient account of the individual
    \end{enumerate}
\item Marcuse response:
  \begin{enumerate}
  \item The organised working class at least no longer has nothing to lose but its chains but a lot more.
    \item ``The consciousness of
      the dependent population changed.
It was one of the more striking phenomena to see to what extent the ruling power
structure could manipulate man and control not only the consciousness
but also the subconscious and unconscious of the individuals.
Therefore my friends at the Frankfurt School considered psychology one of the main
branches of knowledge that had to be integrated with Marxian theory.''
    \end{enumerate}
\end{itemize}

Magee asks about Marcuse's attempts to integrate Marxist and Freudian theory, which Magee thinks incompatible.
Marcuse disagrees!

Magee enumerates predictive failures of Marxism:

\begin{itemize}
\item the failure to
  predict the future success of capitalism
  \item
the anti-libertarian element in Marxism
\item the absence of any theory or attitude to
  the individual
  \item  entirely new theories like
Freudianism which came on the scene
after Marx and therefore couldn't have
been accommodating by Marx in his
outlook
\end{itemize}

and asks why he [Marcuse] remained a Marxist.

Answer: ``Because I do not believe that the theory as such has been falsified''.

Catalogue of the decisive
concepts of Marx which have been
corroborated in the development of
capitalism :
\begin{itemize}
  \item the concentration of economic power
    \item the fusion of economic and      political power
      \item the increasing     intervention of the state into the economy
      \item the increasing difficulties in stemming the tension
      \item the decline in the rate of profit
      \item the need for engaging in a neo-imperialism in order to create markets and possibilities of an large accumulation of capital
\end {itemize}

What were the positive contributions of Marxism:
\begin{itemize}
\item[27:31] a prediction of fascism long before it actually happened
  \item[27:44] the interdisciplinary
approach to the great social and
political problems of the time, cutting
across the academic division of labor
\item[28:17]
the attempt to answer the question what actually has gone wrong in Western
civilization, that at the very height of technical progress we see at the same
time the opposite as far as human progress is concerned
\item[28:56] Especially Horkheimer, but also the others, went back into, not only
a social but also intellectual history, and tried to define the interplay
between progressive and repressive categories throughout the intellectual
history of the West.
Especially in the Enlightenment for example,
which is usually considered as one of the most progressive phases in history.
And the Frankfurt School pointed out to what extent this
apparently perfectly clear progressiveness, this liberating tendency,
was at the same time tied up with regressive and repressive tendencies. 
\end{itemize}

\paragraph{Marxists' Politics of Disillusionment}

Magee asked [29:42]:

\begin{quotation}
This picture that you paint, of a group of Marxists almost obsessed
with the question what has gone wrong, suggests to me politics of
disillusionment.
I mean there seems to be an aura about it of disappointed hopes.
Disappointment with a Marxist theory, disappointment perhaps,
even with the working class itself, for failing to be
an effective instrument of revolution.
Was there something disappointed or disillusioned or pessimistic at the
center of your approach in those days?
\end{quotation}

Marcuse responded [30:13]:

\begin{quotation}
If a disappointment means as you were
formulated disappointment with a working
class I would decidedly reject it.
None of us has a right to blame the working
class for what it is doing or what it is
not doing so this kind of disappointment
certainly not.
There was indeed another disappointment
and that seems to me a very objective attitude.
I mentioned it before, namely that the incredible
social wealth that had been assembled in
Western civilization and mainly as the
achievement of capitalism was
increasingly used for destroying rather
than constructing a more decent and
humane society.
If you call that disappointment, yes, but I think it's
very justified and objective.
\end{quotation}

Magee [31:06]:

\begin{quotation}
And you saw your central task as being an
investigation of the reasons as to why
that ..\emph{[exactly]}..
how had it come about?
So the essential enterprise of the Frankfurt School was a critical one.
\end{quotation}

Marcuse [31:20]:

\begin{quotation}
Definitely.
Therefore the term Critical Theory today for the writings of the Frankfurt School.
\end{quotation}

\ignore{
Magee [31:28]:
  \begin{quotation}
One thing that the members of the
Frankfurt School exhibited very
considerable concern with from the
beginning was the aesthetics, and this I
think differentiates it from most other
philosophies certainly from most other
political philosophies.
\end{quotation}
}%ignore


Magee [34:41]:

\begin{quotation}

this is a field in which thinkers in the tradition of
the Frankfurt School like yourself are now doing fresh and original work.
What other areas do you think this school of philosophy,
this tradition of philosophy,
is going to have to concern itself with in the immediate future?
\end{quotation}

Marcuse [34:59]:

\begin{quotation}
Well I can in this respect only talk of myself and I would say that
far more attention should be paid to the women's liberation movement.
I see in the women's liberation movement today a very strong radical potential.
Now I would have to give a lecture in order to explain and why I do that.
Unfortunately I cannot.

Let me at least try to say it in two sentences.
All domination in recorded history up to [to]day was patriarchal domination.
So if we should indeed live to see, not only equality of the woman before the law,
whatever it is, but the deployment of what is called the specific feminine
qualities throughout the society, for example non-violence, receptivity,
tenderness, this would indeed be, or perhaps could be the beginning of a
qualitatively different society, the very antithesis to male domination with its
violent and brutal character.

No I'm myself perfectly conscious of the fact that these so-called specific
feminine qualities are socially conditioned and \emph{[I say there are people
who would regard it as sexist]} to say all right now I don't care.
There are socially conditioned but to a great extent they are available
they are there so why not use them the way they are regardless of the question
as to their origin.
\end{quotation}

Note that this interview took place in the same year as the Combahee River Collective Statement \cite{combahee-statement}, see section \ref{Combahee}.

\section{Peter Marshall}

\subsection{Demanding the Impossible - A History of Anarchism \cite{marshallHA}}

\section{Charles Murray \cite{murrayc-tbc,murrayc-hd}}

\section{Douglas Murray}

\cite{murrayd-vi,murrayd-sde,murrayd-tmc}

\section{George Orwell}

\subsection{1984}

\cite{orwell-1984}

\subsection{Animal Farm \cite{orwell-af}}

\section{Camille Paglia}

\subsection{Free Women, Free Men: Sex, Gender, Feminism \cite{paglia-fw}}

\section{Steven Pinker}

Stephen Pinker is a cognitive psychologist, psycholinguist and popular science author.
Though he cannot in good faith be described as politically radical, he is certainly well to the right of the modern North American academic norm, and is relatively immune to the delusions of the two political extremes.
It is of necessity in his specialities that he would come into conflict with the left, since the radical left typically oppose the use of evolutionary methods in cognitive science (and socibiology generally), so some of his earliest books directly address polical ideas which conflict with science, and pretty much everything he writes is either critical of or offering alternatives to left wing thought.
He is nevertheless, equally critical of the extreme right, though there is less material on that side because their ideology is not so critical of the Pinker's specialities,

Pinker is included here primarily for his positive ideas about where we should be going, most conspicuous in ``Enlightenment Now'' and ``Rationality''.
Though there is a great deal of criticism of ideological positions which conflict with good science, he is does not dig deeper into the ideologies, their roots or history.

His writings, have a scientific character, but clearly a genuine progressive intent, though not progressive in the current leftist sense (which is often revolutionary and/or regressive rather than progressive).
On the matter of democracy their tendency, when not purely scientific, is contrary to  many of the main trends which I would regard as anti-democratic and in support more generally (and quite explictly) pf ``enlightenment values'', the ideals of reason, science, humanism, and progress.

The following is the briefest taste of some of his popular works in historical sequence as the relate to the concerns of these notes.

My present tendency is to see epistemology, and hence rationality, and related matters everywhere, and I shall draw from that perspective.
But first a mention of an academic paper.

\subsection{Natural Language and Natural Selection \cite{pinker-nlns}}

Early in Pinker's work and in the history of rationality the evolution of language has something to say to us about how little of present day cultural issues bear upon the most fundamental epistemological issues, as an antidote to the cynicism about power-epistemes engendered by Foucault and subsequently weaponised by the cultural revolution.
Pinker, with Bloom wrote a paper on the evolution of language.

The main poimt of the paper is to argue for a Darwinian explanation for the evolution of language, against Gould and others who were skeptical of that standard account.
One particular wrinkle was the following paragraph.

\begin{quote}
  {\it
  Furthermore, in a group of communicators competing for attention and sympathies there is a premium on
the ability to engage, interest, and persuade listeners. This in turn encourages the development of
discourse and rhetorical skills and the pragmatically-relevant grammatical devices that support them.
Symons' (1979) observation that tribal chiefs are often both gifted orators and highly polygynous is a
splendid prod to any imagination that cannot conceive of how linguistic skills could make a Darwinian
difference.}
\end{quote}

Which suggests that sexual selection may have played a significant role, and is of particular interest to me because I have speculated that sexual selection not only was significant in the development of language, but also in the later stages of the expansion of the cerebral cortex at the conclusion of which we have anatomically modern homo sapiens.
That this results from the evolutionary positive feeback loop effected by sexual selection offers a possible explanation for the intelligence overshoot which has equipped homo sapiens to undertake astonishing feats of intellectual virtuosity which could not possibly have been selected for at the time.
Once {\it intelligence} became sexually desirable we were set to get as much of it as could be had without significant evolutionary disadvantage, whether or not there was advantage beyond securing sexual partners.

The following partial list of Pinker's books contains those which seem to me (generally on scant evidence) most relevant to the purpose of this document, and are accompanied by very brief notes on why I think that the case.
The general thrust is that Pinker writes in support of liberal democracy, has an awareness of some of the challenges it faces, and constructively looks for and provides material support for a future brodly within that political landscape.

\subsection{The Blank Slate: The Modern Denial of Human Nature \cite{pinker-tbs}}

He begin's with an attack on the idea of mind as \emph{tabula rasa}, entering in to the world as a clean slate and acquiring both knowledge and character only in the light of experience.

This is closely related to the nature v. nurture debate, which Pinker believes is thought by most people to have settled on the answer ``both''.
With quotes from three books which reflect that common belief, but which proved to be highly controversial just because they admitted the possibility that genetics could be a significant factor in the inheritance of certain human characteristics, Pinker argues that ideology frequently trumps reality in discussion of these questions, and aims ``to explore why the extreme position (that culture is everything) is so often seen as moderate, and the moderate position is seen as extreme''.

The effects of the ideological attachment to a blank slate theory do bot just affect tge public reception of science, but also the kind of science conducted, since scientists in the field are likely to start from a dfferent point and with different research objectives than they would if their starting point was more open or nuanced.

The world has moved a long way since this book was written, in the direction opposite to that which Pinker would have wished, and Pinker's message may be even more important now than it was then.
However, Pinker's writing moved on as well, and several subsequent books by Pinker have shown the evolution of his conception of where our difficulties lie and what kind of thinking might be effective against the adverse trends.

\subsection{The Language Instinct: How the Mind Creates Language\cite{pinker-tli}}

Here Pinker is progressing the view of Chomsky that certain linguistic abilities are innate rather than learned, by supplying a lot of detail on how language works and how the human brain learns or creates particular languages.
Humsn beings learn language not by the use of some general intellectual capability, but by the use of neural machinery which has evolved specifically to enable linguistic skills.

\subsection{How the Mind Works \cite{pinker-tbs}}

This is a more general perspective on the workings of the human brain informed by considering how it evolved.

\subsection{The Better Nature of our Angels: AHistory of Violence and Humanity \cite{pinker-angels}}

A study of violence in human beings over the history of homo sapiens.

\begin{quote}
``Wasn't the twentieth century the most violent in history? In his extraordinary, epic book Steven Pinker shows us that this is wrong, telling the story of humanity in a completely new and unfamiliar way. From why cities make us safer to how books bring about peace, Pinker weaves together history, philosophy and science to examine why we are less likely to die at another's hand than ever before, how it happened and what it tells us about our very natures.
``
\end{quote}
\subsection{Enlightenment Now: The Case for Reason, Science, Humanism and Progress \cite{pinker-en}}


\begin{quote}
``Is modernity really failing? Or have we failed to appreciate progress and the ideals that make it possible?

If you follow the headlines, the world in the 21st century appears to be sinking into chaos, hatred, and irrationality. Yet Steven Pinker shows that this is an illusion - a symptom of historical amnesia and statistical fallacies. If you follow the trendlines rather than the headlines, you discover that our lives have become longer, healthier, safer, happier, more peaceful, more stimulating and more prosperous - not just in the West, but worldwide. Such progress is no accident: it's the gift of a coherent and inspiring value system that many of us embrace without even realizing it. These are the values of the Enlightenment: of reason, science, humanism and progress.

The challenges we face today are formidable, including inequality, climate change, Artificial Intelligence and nuclear weapons. But the way to deal with them is not to sink into despair or try to lurch back to a mythical idyllic past; it's to treat them as problems we can solve, as we have solved other problems in the past. In making the case for an Enlightenment newly recharged for the 21st century, Pinker shows how we can use our faculties of reason and sympathy to solve the problems that inevitably come with being products of evolution in an indifferent universe. We will never have a perfect world, but - defying the chorus of fatalism and reaction - we can continue to make it a better one.
``
\end{quote}

\subsection{Rationality: What is it, Why it Seems Scarce, Why it Matters \cite{pinker-rat}}

\begin{quote}
``In the twenty-first century, humanity is reaching new heights of scientific understanding - and at the same time appears to be losing its mind. How can a species that discovered vaccines for Covid-19 in less than a year produce so much fake news, quack cures and conspiracy theorizing?

In Rationality, Pinker rejects the cynical cliché that humans are simply an irrational species - cavemen out of time fatally cursed with biases, fallacies and illusions. After all, we discovered the laws of nature, lengthened and enriched our lives and set the benchmarks for rationality itself. Instead, he explains, we think in ways that suit the low-tech contexts in which we spend most of our lives, but fail to take advantage of the powerful tools of reasoning we have built up over millennia: logic, critical thinking, probability, causal inference, and decision-making under uncertainty. These tools are not a standard part of our educational curricula, and have never been presented clearly and entertainingly in a single book - until now.

Rationality matters. It leads to better choices in our lives and in the public sphere, and is the ultimate driver of social justice and moral progress. Brimming with insight and humour, Rationality will enlighten, inspire and empower.
``
\end{quote}

\section{Helen Pluckrose}\label{Pluckrose}

\subsection{The Evolution of Postmodern Thought}\label{EPT}

This following is extracted from the talk \cite{pluckrose-evolution}.

A timeline:
\begin{itemize}
\item[5:41] late 1960s, 66-70: beginning of era of post-modernity.
  Loss of confidence in ``modernity'' which stood for scientific and social progress and objective truth.
  Both revolutionary socialism and liberalism were modernist.
  Loss of credibility in Marxism and liberalism.
  The void was filled by ``post-modernism''.
   Jean Baudrillard, Giles Deluge, Felix Guattari.
   Descriptive, despairing, aimless.
 \item[11:20]Post-structuralist and deconstructionist thinkers:

   \begin{itemize}

   \item[11:30] Jean Francois Lyotard: ``The Postmodern Condition'' (1979); scepticism towards meta-narratives (christianity, marxism, science).
     Language of science inseparable from the language of power and government.
     Instead of meta-narratives, we need lots of mini-narratives, moral and factual relativism.

   \item[12:50] Jacques Derrida: sceptical about the possibility of conveying meaning by language, ``words only refer to other words so meaning indefinitely deferred'', but can be used to express comparisons such as ``men superior to women''.  So he advocated inverting these, to expose and challenge them.  Possibly justifying inverted oppression to redress the balance.

   \item[14:12] Michel Foucault: episteme, power-knowledge, discourses, biopower.
     Knowledge as cultural construct.  We decide what is true and what is known through categories and narratives created and enforced culturally (an episteme).
     Those in power set the episteme, this is power-knowledge.
   \end{itemize}
   The imperative then, of postmodern approaches, is to study the discourses of society, to find the Foucian power-knowledge, invert the Derridian binaries and empower the Lyotardian mini-narratives.
   \item[17:57] This yields the following ``plan'':
     \begin{enumerate}[i)]
     \item there is no way of obtaining objective truth, everything is culturally constructed
     \item society is dominated by systems of power and privilege that people just accept as common sense
     \item these vary from culture to culture and subculture to subculture
     \item none of them is right or superior to any other
     \item the categories that we use to understand things, like fact and fiction, reason and emotion, science and art and male and female, are false
     \item they operate in the service of power need to be examined, broken down and complicated
     \item language is immensely powerful and it is used to construct oppressive social realities, therefore it must be regarded with suspicion and scrutinized to find the discourses of power
     \item the intention of the speaker is no more authoritative than the interpretation of the hearer
     \item the idea of the autonomous individual is a myth, the individual is also a construct of culture programmed by his or her place in relation to power
     \item the idea of a universal human nature is also a myth, it is constructed by what
powerful forces deemed to be the right way to be, therefore it is white Western masculine and heterosexual.
     \end{enumerate}
     These are the ideas from post-modernism which have survived and now appear in the social justice movement.
     Thus ends the ``High deconstructive'' phase of post-modernism.
   \item[19:36] Late 1980s: new generation of leftist academics, legal support for various oppressions disappears leaving only attitudes to be addressed, to which post-modernist ideas can be applied.
   \item[21:06] Post colonialism: Offshoot of post-modernism headed by Foucauldian Edward Said who argued ``The west constructed the East as its inferior in order to construct itself in noble terms'' and that previous colonies should now reconstruct the East for themselves.
     Spivak and Bhabha followed but were more Derridian, and hence incomprehensible.
   \item[21:41] 1989: In ``critical legal studies'' and ``critical race theory'' Kimberlé Crenshaw began developing her concept of intersectionality, which she described as contemporary politics linked to postmodern theory.
     She accepted the cultural constructivism of post-modernism in relation to the concepts of race and gender but believed in the objective reality of oppressive cultural constructs around race and gender.She thought liberalism inadequate despite evidence of its success.
     Liberalism was too universal and an intense focus on identity politics was needed.
     Mary Poovey Adopted a similar stance in relation to feminism, reconciling postmodernist deconstruction with he objective reality of the category of women.
     She advocated a ``toolbox'' approach using post-modern techniques when helpful and not otherwise.
     Judith Butler belief in  objective categories was the problem.
     Queer theory purest form of post-modernism currently in existence.
     
     \item[24] It was now objectively true that social reality was culturally constructed by specific systems of power.

\item white privilege - Peggy Macintosh
\item white complicity - Barbara Applebaum
\item white fragility - Robin DiAngelo
\item re-ified postmodernism - The Creed:
  \begin{enumerate}[i)]
\item  racism exists today in both traditional and modern forms
\item  racism is an institutionalized multi-layered multi-level system that distributes unequal power and resources between white people and people of color, as socially identified, and disproportionately benefits White's
\item  all members of society are socialized to participate in the system of racism albeit in various social locations
\item all white people benefit from racism regardless of their intentions
\item no one chose to be socialized into racism so no one is bad, but no one is neutral so not to act against racism is to support racism
\item racism must be continually identified analyzed and challenged no one is ever done
\item the question is not did racism take place but how did racism manifest in that situation
\item the racial status quo is uncomfortable for most White's therefore anything that maintains white comfort is suspect
\item the racially oppressed have a more intimate insight via experiential knowledge into the system of race than their racial oppressors but they're not bad
  \item however white professors will be seen as having more legitimacy thus positionality must be intentionally engaged (means you must always mention your race gender and sexuality and how it impacts on what you're saying)
\item resistance is a predictable reaction to anti-racist education and must be explicitly and strategically addressed
\end{enumerate}

\end{itemize}

\subsection{Cynical Theories  \cite{pluckrose-cynical}}\label{CT}

How Activist Scholarship Made Everything about Race, Gender and Sexuality - and Why this Harms Everybody.

An overall structure something like this:

\begin{enumerate}

\item Postmodernism (1966-1989)

   - Foundational postmodern principles held that objective knowledge is impossible, that knowledge is a construct of power, and that society is made up of systems of power and privilege that need to be deconstructed.

\item Applied postmodernism (80s, 90s)

  \begin{itemize}
  \item postcolonial theory
  \item queer theory
  \item critical race theory
  \item intersectional feminism
  \item disability studies
  \item fat studies
  \end{itemize}

\item since 2010, concretized in the combined intersectional Social Justice scholarship and activism
\end{enumerate}


\subsubsection{Postmodernism}
A revolution in knowledge and power.

Two principles of postmodernism:

\begin{enumerate}
\item Knowledge Principle

  Radical scepticism about objective truth and knowledge, commitment to cultural constructivism.
  
\item Political Principle

  Society is formed of systems of power and hierarchies which determine what can be known and how.
  \end{enumerate}

Four Themes:

\begin{enumerate}
\item blurring of boundaries
\item power of language
\item cultural relativism
\item loss of the individual and the universal
\end{enumerate}

These are core elements of postmodern ``Theory'' which have persisted throughout the evolution of postmodernism and its applications from its initial ``hopelessness'' to its recent strident activism.

\subsubsection{Postcolonial Theory}

Aims to deconstruct the West in order to effect \emph{decolonisation}, the systematic undoing of colonialism.
Prominent in this undertaking are the postmodern knowledge principle (rejection of objective truth) and the postmodern political principle (systems of power and priviledge determining what can be known).

\subsubsection{Queer Theory}

\subsubsection{Critical Race Theory}

\subsubsection{Feminism and Gender Studies}

\subsubsection{Disability and Fat Studies}

\subsubsection{Social Justice Scholarship}

\subsubsection{Social Justice in Action}

\subsubsection{An Alternative to the Ideology of Social Justice}

\subsection{Social (In)Justice \cite{pluckrose-socinj}}\label{SI}

``A reader-friendly remix of Cynical Theories'', see section \ref{CT}.

\section{Karl Popper}

\subsection{The Open Society and its Enemies}

\cite{popperOSE1,popperOSE2}

\subsection{The Poverty of Historicism}

\cite{popperPOH}

\section{Roger Scruton}

\cite{scruton85,scruton15}


\section{Marc Sidwell}\label{Sidwell}

\subsubsection{The Long March \cite{sidwell-long}}

This strategy, named after the long march securing the victory of Communist Party of China, was first enunciated by the German student activist Rudi Dutchke, supported by Herbert Marcuse, and was a replacement for the failed revolutionary expectations of Marxism.

The idea is that student activists should join the various key professions, become competent and progress to positions of power so that these institutions can be subverted from within.
Educational institutions were a first priority, since they educated the new blood which would go into the other professions and institutions, and teacher training of particular interest since activist teachers could then begin the work with younger and more impressionable subjects.

This strategy has proved very successful in propagating critical theories through key social institutions in the UK and in other English-speaking developed nations (USA, Canada, Australia) such as education, the media, and the arts, and thence through quangos and into industry via Human Resource departments.
Other key institutions such as the police and the judiciary have also been effectively targetted through their training professionals and through the bodies writing guidance for police officers, judges and court officials.
In the USA the preponderance of strong left wing politics in tertiary education, particularly in social sciences and humanities, is sufficient to silence many of the few conservatives who survive in that context on any matter which is politically sensitive.
The reach of critical theory leaves little out of its scope, now making ground in the supposedly objective and non-partisan STEM curriculum including mathematics.
The United Kingdom follows in the wake, even issues conspicuously local to the USA readily crossing the atlantic.

\begin{itemize}
\item[CHAPTER ONE] Gramsci’s Ghost 1
\item[CHAPTER TWO] Meet the Blob 11
\item[CHAPTER THREE] The ‘Culture Industry’ Industry 21
\item[CHAPTER FOUR] Wolves in Sheep’s Clothing 31
\item[CHAPTER FIVE] Mao, Marx and Marcuse 40
\item[CHAPTER SIX] The Thatcher Revolution 51
\item[CHAPTER SEVEN] From Political Class to Identity Politics 65
\item[CHAPTER EIGHT] Failing Upwards 77
\item[CHAPTER NINE] The Art of Cultural Resistance 89
\item[CHAPTER TEN] Downstream of Politics 99
\end{itemize}

\section{Peter Singer}

Probably doesn't belong here, but he appears because I am thinking of Ethics as authoritarian and sharing with metaphysics an important \emph{a priori} core, thus epistemologially suspect.
This is not an essential element of ethics but one in which utilitarians are suspects because of their attempt to provide a total (or almost total) moral ordering over our choices, and hence to leave little or no discretion to the moral individual.

Anyway, he came up while we were discussing \emph{equality}, leading into \emph{equity} and at first by my viewing a very short video \cite{singer-eqvid}, after which I went through the free sample of his book on

He is also relevant as a scholar of Hegel and Marx, and the interview with Magee on that topic \cite{magee-singer} is instructive, see Section \ref{Sin-Heg}.

\section{Debra Soh}

\subsection{The End of Gender: Debunking the Myths about
Sex and Identity in Our Society \cite{soh-end}}

\section{Thomas Sowell}

\cite{sowell-barbarians}

\section{Michael J. Thompson}

Editor of the book series: \emph{Political Philosophy and Public Purpose}.

\subsection{The Palgrave Handbook of Critical Theory \cite{thompson-palcrit}}

\subsubsection{Introduction: What is Critical Theory?}

\begin{quotation}
``Critical theory is, then, a radically different form of knowledge from mainstream theory and social science...''
\end{quotation}

But:

\begin{quotation}
``Hence, strands of thought such as feminism, deconstruction, and postcolonialism, among others, have been crowded under the banner of critical theory. But to do this is to commit an error about what critical theory — indeed, about what critique — actually is.''
\end{quotation}

So we see here, the detachment of ``Critical Theory'', that departure from ``traditional theory'' initially undertaken by the Frankfurt School, from what Pluckrose and Lindsay \cite{pluckrose-cynical} (section \ref{CT}) called ``applied critical theories'' (without capitals), diversely addressing the individual forms of disadvantage (and supposed oppression) which were to be re-united by Crenshaw's (section \ref{KC}) notion of intersectionality under the rubric ``critical social justice''.

... and to make the breach between Critical Theory and these applied ``critical theories'' clearer:

\begin{quotation}
``Despite what many have surmised, critical theory was always preoccupied with the normative validity of human progress, by the need to defend the political and cultural values of the Enlightenment and to expand the sphere of human emancipation through reasoned, rational consciousness``
\end{quotation}

though Thompson is not adopting the use of capitals to distinguish the original of which he here speaks, his desire to absolve it entirely of the taint of postmodernism seems stark, but seems to understate theextent to which critical theory is contrasted by Horkheimer with ``traditional theory'' and the aspects of enlightenment rationality which are thus rejected (for example, the Humean distinctions between logical and empirical truths and the indepence of moral and hence political judgements from them).

\paragraph{The Concept of Critique}

\paragraph{The Origins of Critical Thepry}

\paragraph{The Theories of the Frankfurt School}

\paragraph{The Communicative–Pragmatic Turn}

\paragraph{Why Critical Theory Persists and the Purpose of this Book}

\section{Miscellany}

\subsection{A compendium of Free Black Thought \cite{free-black-thought}}.

\chapter{Some Issues}

\section{The Epistemology of Marxist Scolarship}

\subsection{Preliminaries}

I want to explore here the aspects of post-Marxist thought which set it apart apart from what some Marxists (and many others) consider to be ``traditional'' methods.

I'm not expert in either side of this comparison, but nor can I present myself as without prior conception, I lean hevily towards the tradition from which Marxism has progresively departed.

\subsection{Traditional Theory}


\section{Social Justice and Identity Politics}

These are underpinned by a variety of ``critical theories'' which are in turn underpinned by the ``Critical Theory'' of the Frankfurt School and Postmodern Philosophy.

\subsection{pre-history}

\subsubsection{Hegel and Marx}

{\bf\emph{Notes from Magee and Singer}}.\label{Sin-Heg}

Magee\cite{magee-singer} summary:

\begin{enumerate}
\item understanding reality = understanding a process (of perpertual change)
\item what is changing - ``geist'' (mind/spirit)
\item why changing? - because in a state of alienation
\item what is the process of change? - the dialectical process
\item where is it going? - politically: to organic society; philosophically: to absolute knowledge
\end{enumerate}

Note that (in this sketch) Magee does not mention Hegel's notion of freedom, and does not give particular prominence to the alleged logical necessity in the dialectic. 

Left and Right Hegelians (young and old)

Right (conservative) took Hegel to be describing the Prussian state and therefore that no radical changes were required.

Left (radical) took Hegel to be concerned with overcoming the conflict between reason and desire or between morality and self interest, a very large undertaking, hence requiring revolution (though Hegel did not advocate it).

Marx was a left Hegelian and carried over 1 and 3-5 above, but eliminates ``geist'' as the subject of the process in favour of matter, hence ``dialectical materialism''.

{\bf\emph{Notes from Sabine}}.\label{Sab-Heg}

Sabine \cite{sabine63} seems to me to penetrate deeper in his first explanation of Hegel's logic, along the following lines.

Hegel's logic is intended to overcome the constraints advertised (e.g. by Hume) on analytic (deductive) logic and implicit in the tripartite division between truths of reason, empirical truths and value judgements.
Hegel offers in his dialectic a kind of reason which tells us about how history progresses and gives us a logical justification for matters including morals and religion.

Sabine mentions ``three vaguely similar generalisations'':

\begin{enumerate}
\item universal human progress (inherited from the enlightenment, Turgo, Condorcet)
  
\item logically necessary historical development (of national cultures)
  
\item Darwin's theory of ``organic evolution''
\end{enumerate}

The failure to distinguish these very different notions of progress (of which (2) is Hegel's historicism) caused great confusion.
Sabine considers (3) to be irrelevant to (1) and (2), and considers (1) to be revolutionary in tendency while (2) in Hegel's conception is conservative (but is later transformed by Marx into a revolutionary theory).

The main distinctive feature of (2) is that it is held to be a matter of \emph{logic} rather than of \emph{empirical causation}.
Hegel, and later Marx, regarded (1) (and (3)?) as ``philosophically superficial'' \emph{because} they are empirical theories.
Hegel's purpose was to demonstrate the logically necessary stages by which human reason approximates the absolute.

\subsubsection{Nietzsche}

Its not clear to me that this is important to critical theory, but it is a line of thinking leading to totalitarian politics.

Notes from Magee and Stern \cite{magee-stern}.
The main traditions in Western Civilisation which Nietzsce attacked were:

\begin{enumerate}
  
\item Christian morality

Dismisses christian virtues such as ``turning the other cheek'', compassion, humility.
  
\item Secular morality

Dismisses the generalised moral codes which occurs in secular moral theory.

\item Herd morality

The heroic individual should be a law unto himself.
  
\item Some traditions from ancient Greece (Socrates)
  
\end{enumerate}


\subsection{roots}

The three pillars which support Social Justice ideologies (collectively ``critical social justice'' of which ``critical race theory'' is perhaps the most controversial) are:

\begin{itemize}
\item Critical Theory
\item Postmodern Philosophy
\item The Long March through the Institutions
\end{itemize}

The first two are philosophical, the last is an activist strategy.

\subsubsection{Critical Theory}

``Critical Theory'' as capitalised refers to the theories of the Frankfurt School, the later more activist descendents (e.g. critical race theory) lose the capitals.

The Frankfurt School was established as the Institute for Social Research by Felix Weil in Frankfurt with an endowment from his father Herman Weil.
At first a group of orthodox Marxists, it developed from 1930 under the leadership of Max Horkheimer its own distinctive ``Critical Theory'' of Society \cite{horkheimer-trad,horkheimer-crit}.


Some aspects of Critical Theory:

\begin{itemize}
\item Integration of Theoretical, Practical and Normative

  Critical Theory must explain what is wrong with current social reality, identify the actors to change it, and provide both clear norms for criticism and achievable practical goals for social transformation.
  
\item Emancipation of the individual
  
\item Democratisation of society

  “all conditions of social life that are controllable by human beings depend on real consensus” in a rational society (Horkheimer \cite{horkheimer-crit})

  \item Intolerance
\end{itemize}

\subsubsection{Postmodern Philosophy}

See Pluckrose and Lindsay sections \ref{EPT} and \ref{CT} 

\subsubsection{The Long March}\label{LongMarch}

See Marc Sidwell, Section \ref{Sidwell}

\subsection{Gender}

\subsection{Race}

\section{Critical Pedagogy}

See: Gottesman section \ref{Gottesman} and Friere section \ref{Friere}, Henry Giroux \ref{Giroux}.


\section{Freedom of Speech}

\section{Due Process}

\section{Conflicts of Interest}

\section{The Nature of Democracy and The Risk of Subversion}

\section{Pushback}

Here some notes about those who are trying to reverse some of the  trends we have been considering.

First a few words about the phrase \emph{anti-democratic}, which concerns us here because that's the purpose of this document, even though we might debate whether all of the things noted here really are anti-democratic, or whether that's their most important disadvantage.

Many, perhaps most, allegations of anti-democratic or authoritarian are made by the political left-of-centre, about the political right-of centre, sometimes associated with an accusation of fascism.

For present purposes I am considering counter-reactions to doctrines which may be considered anti-democratic because they involve deceit, sometimes in the form of denial of scientifically well established facts, and/or systematic attempts to prevent the articulation not just of particular values or preferences, but also of objective factual knowledge on which they are based.

A healthy democracy depends upon openness and honesty, in default of which confidence in the democratic process will wither and alternative means of resolving differences will be incentivised.
For this reason the line of thought coming down into social activism from Hegel and Marx through Critical Theory, Postmodern Philosophy and the various applied critical theories can be considered compromised and my attention is here directed to some of the pushback against these lines of thought.

Notwithstanding their heritage, it is the continued practice of manipulation of language and self-serving unjustifiable alternative epistemologies which I would like push nack against, but the most substantial pushback will inevitably be provoked by the socal consequences rather than the philosophical mis-steps which contributed.

\subsection{Gender Critical Feminism}

I'm not convinced that it is correct to include ``feminism'' in the title here, since the sources which I will mention here are primarily concerned with exposing various facts the denial of which is facilitating what many consider an assault on ``womens rights'', but are also alleged to underpin practices which are damaging to minors, both psychologically and physically.


\section{Philosophical Responses}

Though no clean distinction is likely to be possible, I try here to distinguish possible responses to the perceived problems as:

\begin{itemize}
\item Activist -
  in which realising effects by whatever means seem most effective is intended.
\item Political -
  in which change by persuasion, campaigning for change through democractic process is the aim
\item Philosophical -
  in which responses to fundamental philosophical innovations and countered by articulating philosophical perspectives or systems
\end{itemize}

In this section I consider exlusively the latter, philosophical responses.

In post-Marxist thought (including Critical Theory and Postmodern Philosophy) there is progressive distancing from fundamental aspects of philosophy, including philosophy of language, epistemology and logic.
It is in these areas that I look for philosophical responses.

A return to enlightenment values has been suggested as a way forward, notably by Steven Pinker \cite{pinker-en}, and his contribution is worthwhile, but I seek here to look forward rather than backward.


\phantomsection
\addcontentsline{toc}{section}{Bibliography}
\bibliographystyle{rbjfmu}
\bibliography{rbj2}

%\addcontentsline{toc}{section}{Index}\label{index}
%{\twocolumn[]
%{\small\printindex}}

%\vfill

%\tiny{
%Started 2020/01/17


%\href{http://www.rbjones.com/rbjpub/www/papers/p032.pdf}{http://www.rbjones.com/rbjpub/www/papers/p033.pdf}

%}%tiny

\end{document}

% LocalWords:
