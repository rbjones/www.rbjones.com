% $Id: p033.tex $fi
% bibref{rbjp032} pdfname{p033}
\documentclass[10pt,titlepage]{book}
\usepackage{makeidx}
\newcommand{\ignore}[1]{}
\usepackage{graphicx}
\usepackage[unicode]{hyperref}
\pagestyle{plain}
\usepackage[paperwidth=5.25in,paperheight=8in,hmargin={0.75in,0.5in},vmargin={0.5in,0.5in},includehead,includefoot]{geometry}
\hypersetup{pdfauthor={Roger Bishop Jones}}
\hypersetup{pdftitle={Failing Democracies and How to Fix Them}}
\hypersetup{colorlinks=true, urlcolor=red, citecolor=blue, filecolor=blue, linkcolor=blue}
%\usepackage{html}
\usepackage{paralist}
\usepackage{relsize}
\usepackage{verbatim}
\usepackage{enumerate}
\usepackage{longtable}
\usepackage{url}
\newcommand{\hreg}[2]{\href{#1}{#2}\footnote{\url{#1}}}
\makeindex

\title{\LARGE\bf Failing Democracies \\and \\How to Fix Them}
\author{Roger~Bishop~Jones}
\date{\small 2020/04/03}


\begin{document}
\frontmatter

%\begin{abstract}
%Maybe intended as an essay, but turned out to be a collection of references and notes on some of them.
%I left more or less the original intent in this abstract.
%
% We seem at this moment in history to be unusually well endowed with tales of how our democracies are failing.
% Against these there is "push back".
% To push back you need to spot what is going on, to describe it clearly, to understand and articulate why it is pathological, but also to propose and promote an effective remedy.
% 
% Thinking philosophically about these phenomena may provide more compelling support, for the core values which are threatened, against
% a range of subversive strategies and tactics.
% Here we take democracy as the fundamental value and seek to analyse contemporary erosions and consider what kinds of defence might be mounted.
%An important defence is the light of scrutiny, on the (doubtful) thesis that once we see clearly the threats and the values which they threaten, their strength will be undermined.
%\end{abstract}

\begin{titlepage}
\maketitle

%\vfill

%\begin{centering}

%{\footnotesize
%copyright\ Roger~Bishop~Jones;
%}%footnotesize

%\end{centering}

\end{titlepage}

\ \

\ignore{
\begin{centering}
{}
\end{centering}
}%ignore

\setcounter{tocdepth}{2}
{\parskip-0pt\tableofcontents}

%\listoffigures

\mainmatter

\pagebreak

\section*{Preface}

\addcontentsline{toc}{section}{Preface}

There are ``hyperlinks'' in the PDF version of this monograph which either link to another point in the document  (if coloured blue) or to an internet resource  (if coloured red) giving direct access to the materials referred to (e.g. a Youtube video) if the document is read using some internet connected device.
Important links also appear explicitly in the bibiography.

\chapter{Introduction}


I don't really have much idea how to do this, but I think quite a lot of digging is in order, and I will start by making notes here on what others have uncovered.

\chapter{Some People and their Books}

\section{Akala}

\subsection{Native\cite{akala-native}}

Race and Class in the Ruins of Empire.

\section{Michelle Alexander}

\subsection{The New Jim Crow\cite{alexander-tnjc}}
Mass Incarceration in the Age of Colorblindness. 

\section{Ayaan Hirsi Ali}

\subsection{The Challenge of Dawa\cite{ali-dawa}}

Political Islam as
Ideology and Movement
and How to Counter It 

\begin{small}
\begin{tabular}{l l}
Executive Summary & 1\\
Summary of Policy Recommendations&5\\
Introduction&9\\
Part I The Constitution of Political Islam&23\\
Part II Dawa: Much More than a “Call to Islam”&35\\
Part III Confronting Ideology to Win the War&51\\
Conclusion&61\\
Detailed Policy Recommendations&65\\
Appendix A: Eight Types of Threat from Radical Islam&79\\
Appendix B: Shay’s Three Joint Pillars of Dawa and Jihad&73\\
Appendix C: Mares’ Five-Step Model of Political Islam’s
Expansion&75\\
Appendix D: Charities and the Terrorist Money Trail&77\\
Glossary&79\\
\end{tabular}
\end{small}

\begin{quote}
{\it
The aim of da’wah and jihaad is not to shed blood, take wealth,
or enslave women and children; these things happen incidentally but are not the aim. This only takes place when the
disbelievers (non-Muslims) refrain from accepting the truth
and persist in disbelief and refuse to be subdued and pay the
jizya (tax levied on free non-Muslims living under Muslim
rule) when it is requested from them. In this case, Allah has
prescribed the Muslims to kill them, take their wealth as booty
and enslave their women and children . . . this religion (Islam)
. . . is superior to every law and system. . . . The truth has been
spread through the correct Islamic da’wah, which in turn has
been aided and supported by jihaad whenever anyone stood in
its way. . . . It was jihaad and da’wah together which helped to
open the doors to victories.
}
\end{quote}
—Saudi Grand Mufti Ibn Baz, 1998

\begin{quote}
{\it
A 2008 survey of more than nine thousand European Muslims
by the Science Center Berlin reported strong belief in a return
to traditional Islam. In the words of the study’s author, Ruud
Koopmans, “almost 60 percent agree that Muslims should return
to the roots of Islam, 75 percent think there is only one interpretation of the Quran possible to which every Muslim should stick,
and 65 percent say that religious rules are more important to
them than the laws of the country in which they live.” More than
half (54 percent) of European Muslims surveyed also believe
that the West is out to destroy Muslim culture.
}
\end{quote}

\section{Peter Boghossian}

We have here a book and a supporting talk.
Talk first.

\subsection{\href{https://www.youtube.com/watch?v=LiymUd9FjHA}{The Way Forward}}

There is a prelude which is about critical social justice.

\begin{itemize}
\item[9:58] Start by listening: you have to understand what your target is talking about.
  Don't talk over him, concede if there is a clash. (go, no you go)
  Say ``I'm not sure I understand'' rather than complaining of unclarity.
  Say ``I hear you'' and mean it.
  Echo snips.
  Attempt to re-express targets position clearly.
\item[14:29] ``Epistemology'': ask how they know it?
\item[16:28] Scales: how confident are you in that belief? (1:10 or whatever)
  Ask again after (pre-test/post-test).
  Ask ``how much of ...'' (e.g. of a patriarchy).
\item[19:24] Disconfirmation: under what conditions could you be wrong?
  (what evidence would convince you otherwise).
  If can't be wrong, then ``oh, not based on evidence?''.
  You need to now from them what evidence would change mind, what would count for youis not at issue.
\item[23:00] Ask a question to facilitate doubt.
  e.g. if gender studies and biology disagree, who would you be likely to believe mkore?
\item[24:10] Don't provoke defensiveness: Don't say ``but'', it denies everything that has gone before, and sets target into defensive posture.
  Say ``Yes and ...''.
\item[25:35] Build bridges: make golden bridges to save face.
\item[26.35] Don't apologise: unless you are actually sorry. Not if someone is offended.  AS long as criticising ideas rather than persons.
\item[27:35] Be sincere: demonstrate parahesia, don't equivocate or sugarcoat.
  Speaking in unclear language is a type of self-deception.
  \item[29] Be willing to revise your beliefs.  Be willing to say ``I don't know''.
\end{itemize}

Then he runs over it again.

\subsection{How to Have Impossible Conversations: A Very Practical Guide\cite{boghossian-conversations}}

\cite{boghossian-manual}

\subsubsection{Where they are coming from:}

\begin{quote}
  We know because we’ve had countless conversations with zealots, criminals, religious fanatics, and extremists of all stripes.
  Peter did his doctoral research in the Oregon State Prison System conversing with offenders about some of life’s most difficult questions, and then built upon those techniques in thousands of hours of conversations with religious hardliners.
  James developed the ideas for his books and articles by engaging in extended conversations with people who hold radically different views about politics, morality, and religion.
  This book is the culmination of our extensive research and a lifetime of experience in conversing with people who profess to be unshakable in their beliefs.
\end{quote}

\subsubsection{They offer:}

\begin{quote}
thirty-six techniques drawn from the best, most effective research on applied epistemology, hostage and professional negotiations, cult exiting, subdisciplines of psychology, and more.
\end{quote}

\subsubsection{Partial Contents:}

\begin{tabular}{l | p{9cm}}
one & When Conversations Seem Impossible \\\hline
two & The Seven Fundamentals of Good Conversations \\\hline
\#1 & GOALS Why are you engaged in this conversation? \\
\#2 & PARTNERSHIPS Be partners, not adversaries \\
\#3 & RAPPORT Develop and maintain a good connection \\
\#4 & LISTEN Listen more, talk less \\
\#5 & SHOOT THE MESSENGER Don’t deliver your truth \\
\#6 & INTENTIONS \\
& People have better intentions than you think \\
\#7 & WALK AWAY\\\hline
three & Beginner Level: Nine Ways to Start Changing Minds \\\hline
\#1 & MODELING Model the behavior you want to see in others \\
\#2 & WORDS Define terms up front \\
\#3 & ASK QUESTIONS Focus on a specific question \\
\#4 & ACKNOWLEDGE EXTREMISTS\\
&Point out bad things people on your side do \\
\#5 & NAVIGATING SOCIAL MEDIA \\
&Do not vent on social media \\
\#6 & DON’T BLAME, DO DISCUSS CONTRIBUTIONS \\
&Shift from blame to contribution \\
\#7 & FOCUS ON EPISTEMOLOGY \\
& Figure out how people know what they claim to know \\
\#8 & LEARN Learn what makes someone close-minded \\
\#9 & WHAT NOT TO DO (REVERSE APPLICATIONS)\\
& A list of fundamental and basic conversational mistakes \\\hline
four & Intermediate Level: Seven Ways to Improve Your Interventions \\\hline
five & Five Advanced Skills for Contentious Conversations \\\hline
six & Six Expert Skills to Engage the Close-Minded \\\hline
seven & Master Level: Two Keys to Conversing with Ideologues \\
\end{tabular}

\section{Ta-Nehisi Coates}

\subsection{\cite{coatestnh-bwm,coatestnh-wweyip}}

\section{Combahee River Collective}\label{Combahee}

\subsection{Statement (1977) \cite{combahee-statement}}

\section{Paulo Friere}\label{Friere}

\subsection{The Politics of Education \cite{friere-poled}}\label{Friere}

Culture, Power and Liberation

\section{Henry Giroux}\label{Giroux}

\subsection{On Critical Pedagogy \cite{giroux-critped}}

\section{Isaac Gottesman}\label{Gottesman}

\subsection{The Critical Turn in Education \cite{gottesman-criturn}}

From Marxist Critique to Poststructuralist Feminism to Critical Theories of Race (Critical Social Thought).



Critical Pedagogy aims to transform the education system so that it teaches in a way consistent with the Critical Theory principle that academic studies or research should be fully integrated with social activism.
Thus pedagogy should aim to transform the education system, and should aim to make the system effective in the emancipation of oppressed groups of every kind.
It should do this by teaching students to do just that, from the very beginnings of their education and through into adulthood, children should be educated as critical social justice activists.

This book provides a historical account of the development and the course of Critical Pedagogy in its long March Through the Institutions (see Sidwell section \ref{Sidwell})) up to 2016.
According to Michael W. Apple the series editor  “the critical turn” in education had by then been integrated into the formal corpus of official programs in education throughout the world, but, perhaps had taken insurgent knowledge and turned it into simply one more academic area that needs to be studied for examinations, thereby severing its connections to its political roots.
Rather than politicizing the academic, it had perhaps academicized the political.

\subsubsection{Series Editor's Introduction}

The series is the ``The Critical Social Thought Series'', the series editor is Michael W. Apple.
Bear in mind here that the ``Critical'' adjective here is a reference to the post-Marxist ``Critical Theory'' initiated at the Frankfurt School and continuously developed through to the present day in applied critical theories and critical social justice activism.

His introduction is at first clear about the nature of the enterprise (Critical Education Theory), but later becomes more esoteric.

He begins with an anecdote which leads him to infer:

\begin{itemize}
\item that Critical Education has been incorporated into official programs in education throughout the world,

  and
\item that in the process it may have lost connection with its political roots, that ``rather than politicizing the academic, it academicizes the political''.
\end{itemize}

I find that difficult to believe, given the nature of the content, but I can quite believe that it may not have been inspiring political activism quite as vigorous as hoped.
It is notable that awareness of the effects of Critical Theory in the various dimemsioms of ``identity politics'' seems to have ballooned in the period after publication of this book, so he may have written in the calm before the storm.

\subsubsection{Introduction}


\section{Antonio Gramsci}

\subsection{Prison Notebooks \cite{gramsci-notes}}

The idea of cultural hegemony, and the seeds of the long march through the institutions.

\section{Christopher Hitchens}

\subsection{Why Orwell Matters\cite{hitchens-wom}}

\section{Max Horkheimer}

Page numbers shown below in square brackets are from: \cite{horkheimer-crit}.

\subsection{Traditional and Critical Theory}

In this essay \cite{horkheimer-trad, horkheimer-crit} dated 1937 (when Horkheimer and the Frankfurt School would have been at Columbia University in New York), Horkheimer begins with an account of what ``Theory'' is.
The account is remarkably consonant with Aristotle's conception of demonstrative science, but it is Descartes' \emph{Discourse on Method}\cite{descartesDOM} to which Horkheimer refers and which he quotes.

Theory is a body of propositions, some ``primary principles'' others derived from them, which are consonant with ``the actual facts''.
Various elaborations upon this are touched upon.
The increasingly dominant role of mathematics in such scientific theories is noted.

This conception of theory pre-eminent in the natural sciences, is also practiced in the social sciences, and the difference between social science conducted primarily by empirical investigation and that of (German) social scientists who analyse basic concepts and formulate fundamental principles as an ``armchair'' activity does not consist in distinct conceptions of ``theory''.
The differences between the social scientists who prefer an empirical approach and those who prefer a theoretical approach (T\"{o}nnies, Durkheim, Weber) is further discussed.

From here Horkheimer introduces the significance of historical and social context in the formulation of theories.
The Social Scientists ..
\begin{quote}
.. believe they are acting according to personal
determinations, whereas in fact even in their most complicated
calculations they but exemplify the working of an incalculable
social mechanism. [p197]
\end{quote}

\begin{quote}
  [Critical Theory] never aims simply at an increase of knowledge as such. Its goal is man's emancipation from slavery. [p246]
\end{quote}


\section{David Horowitz}

\cite{horowitz-paau,horowitz-bbal}

\section{James Lindsay}

\cite{pluckrose-cynical,lindsay-everybody,pluckrose-cynical}

\section{Heather Mac Donald}

\cite{macdonald-bbi, macdonald-woc, macdonald-tdd}

\section{Herbert Marcuse}

See Marxists Internet Archive\cite{marcuse-mia}.

\subsection{One Dimensional Man (1964) \cite{marcuse-one-dim}}

Possibly his most influential work.

\subsubsection{Introduction: The Paralysis of Criticism: Society without Opposition}
\subsubsection{Part 1 – One-Dimensional Society}
\paragraph{1 – The New Forms of Control}
\paragraph{2 – The Closing of the Political Universe}
\paragraph{3 – The Conquest of the Unhappy Consciousness: Repressive Desublimation}
\paragraph{4 – The Closing of the Universe of Discourse}
\subsubsection{Part 2 – One-Dimensional Thought}
\paragraph{5 – Negative Thinking: The Defeated Logic of Protest}
\paragraph{6 – From Negative to Positive Thinking: The Logic of Domination}
\paragraph{7 – The Triumph of Positive Thinking: One-Dimensional Philosophy}
\subsubsection{Part 3 – The Chance of the Alternatives}
\paragraph{8 – The Historical Commitment of Philosophy}
\paragraph{9 – The Catastrophe of Liberation}
\paragraph{10 – Conclusion}
\subsubsection{Notes}


\subsection{Repressive Tolerance (1965)\cite{marcuse-repressive}}

An elaborate inversion of Poppers ``Paradox of Intolerance''.

e.g.:

\begin{quotation}
"The whole post-fascist period is one of clear and present danger. Consequently, true pacification requires the withdrawal of tolerance before the deed, at the stage of communication in word, print, and picture. Such extreme suspension of the right of free speech and free assembly is indeed justified only if the whole of society is in extreme danger. I maintain that our society is in such an emergency situation, and that it has become the normal state of affairs,"
\end{quotation}

  
\subsection{Essay on Liberation (1969)\cite{marcuse-liberation}}

\subsubsection{Preface and Introduction}

Marcuse begins by setting the context as a confrontation between the economic and military force corporate capitalism which exhibits global dominance and opposing forces of the socialist orbit.
In this context the development of socialism is deflected by aspirations to values and wealth exemplified in the American standard of living.

This is now changing, not as a different route to socialism, but as different goals and values for socialists.


\subsubsection{A Biological Foundation for Socialism?}
\subsubsection{The New Sensibility}
\subsubsection{Subverting Forces – in Transition}
\subsubsection{Solidarity}

\subsection{Magee Interview \cite{marcuse-magee}}

A 1977 interview of Marcuse by Brian Magee, just two years before his death.
When I first watched this, mostly oblivious to his historical role despite having read \emph{One Dimensional Man} many years before, he seemed moderate and reasonable.

Factors demanding reconstruction of Marxism:

\begin{itemize}
\item The rise of Fascism
\item The concept of socialism itself.
  In Marx's conception of socialism full-time alienated labor would no longer be the measure of wealth and value.
  The idea of a socialist society as one in which life does not involve alienated labour has disappeared.
\item Magee suggestions:
  \begin{enumerate}
  \item insufficiently libertarian
    \item didn't take sufficient account of the individual
    \end{enumerate}
\item Marcuse response:
  \begin{enumerate}
  \item The organised working class at least no longer has nothing to lose but its chains but a lot more.
    \item ``The consciousness of
      the dependent population changed.
It was one of the more striking phenomena to see to what extent the ruling power
structure could manipulate man and control not only the consciousness
but also the subconscious and unconscious of the individuals.
Therefore my friends at the Frankfurt School considered psychology one of the main
branches of knowledge that had to be integrated with Marxian theory.''
    \end{enumerate}
\end{itemize}

Magee asks about Marcuse's attempts to integrate Marxist and Freudian theory, which Magee thinks incompatible.
Marcuse disagrees!

Magee enumerates predictive failures of Marxism:

\begin{itemize}
\item the failure to
  predict the future success of capitalism
  \item
the anti-libertarian element in Marxism
\item the absence of any theory or attitude to
  the individual
  \item  entirely new theories like
Freudianism which came on the scene
after Marx and therefore couldn't have
been accommodating by Marx in his
outlook
\end{itemize}

and asks why he [Marcuse] remained a Marxist.

Answer: ``Because I do not believe that the theory as such has been falsified''.

Catalogue of the decisive
concepts of Marx which have been
corroborated in the development of
capitalism :
\begin{itemize}
  \item the concentration of economic power
    \item the fusion of economic and      political power
      \item the increasing     intervention of the state into the economy
      \item the increasing difficulties in stemming the tension
      \item the decline in the rate of profit
      \item the need for engaging in a neo-imperialism in order to create markets and possibilities of an large accumulation of capital
\end {itemize}

What were the positive contributions of Marxism:
\begin{itemize}
\item[27:31] a prediction of fascism long before it actually happened
  \item[27:44] the interdisciplinary
approach to the great social and
political problems of the time, cutting
across the academic division of labor
\item[28:17]
the attempt to answer the question what actually has gone wrong in Western
civilization, that at the very height of technical progress we see at the same
time the opposite as far as human progress is concerned
\item[28:56] Especially Horkheimer, but also the others, went back into, not only
a social but also intellectual history, and tried to define the interplay
between progressive and repressive categories throughout the intellectual
history of the West.
Especially in the Enlightenment for example,
which is usually considered as one of the most progressive phases in history.
And the Frankfurt School pointed out to what extent this
apparently perfectly clear progressiveness, this liberating tendency,
was at the same time tied up with regressive and repressive tendencies. 
\end{itemize}

\paragraph{Marxists' Politics of Disillusionment}

Magee asked [29:42]:

\begin{quotation}
This picture that you paint, of a group of Marxists almost obsessed
with the question what has gone wrong, suggests to me politics of
disillusionment.
I mean there seems to be an aura about it of disappointed hopes.
Disappointment with a Marxist theory, disappointment perhaps,
even with the working class itself, for failing to be
an effective instrument of revolution.
Was there something disappointed or disillusioned or pessimistic at the
center of your approach in those days?
\end{quotation}

Marcuse responded [30:13]:

\begin{quotation}
If a disappointment means as you were
formulated disappointment with a working
class I would decidedly reject it.
None of us has a right to blame the working
class for what it is doing or what it is
not doing so this kind of disappointment
certainly not.
There was indeed another disappointment
and that seems to me a very objective attitude.
I mentioned it before, namely that the incredible
social wealth that had been assembled in
Western civilization and mainly as the
achievement of capitalism was
increasingly used for destroying rather
than constructing a more decent and
humane society.
If you call that disappointment, yes, but I think it's
very justified and objective.
\end{quotation}

Magee [31:06]:

\begin{quotation}
And you saw your central task as being an
investigation of the reasons as to why
that ..\emph{[exactly]}..
how had it come about?
So the essential enterprise of the Frankfurt School was a critical one.
\end{quotation}

Marcuse [31:20]:

\begin{quotation}
Definitely.
Therefore the term Critical Theory today for the writings of the Frankfurt School.
\end{quotation}

\ignore{
Magee [31:28]:
  \begin{quotation}
One thing that the members of the
Frankfurt School exhibited very
considerable concern with from the
beginning was the aesthetics, and this I
think differentiates it from most other
philosophies certainly from most other
political philosophies.
\end{quotation}
}%ignore


Magee [34:41]:

\begin{quotation}

this is a field in which thinkers in the tradition of
the Frankfurt School like yourself are now doing fresh and original work.
What other areas do you think this school of philosophy,
this tradition of philosophy,
is going to have to concern itself with in the immediate future?
\end{quotation}

Marcuse [34:59]:

\begin{quotation}
Well I can in this respect only talk of myself and I would say that
far more attention should be paid to the women's liberation movement.
I see in the women's liberation movement today a very strong radical potential.
Now I would have to give a lecture in order to explain and why I do that.
Unfortunately I cannot.

Let me at least try to say it in two sentences.
All domination in recorded history up to [to]day was patriarchal domination.
So if we should indeed live to see, not only equality of the woman before the law,
whatever it is, but the deployment of what is called the specific feminine
qualities throughout the society, for example non-violence, receptivity,
tenderness, this would indeed be, or perhaps could be the beginning of a
qualitatively different society, the very antithesis to male domination with its
violent and brutal character.

No I'm myself perfectly conscious of the fact that these so-called specific
feminine qualities are socially conditioned and \emph{[I say there are people
who would regard it as sexist]} to say all right now I don't care.
There are socially conditioned but to a great extent they are available
they are there so why not use them the way they are regardless of the question
as to their origin.
\end{quotation}

Note that this interview took place in the same year as the Combahee River Collective Statement \cite{combahee-statement}, see section \ref{Combahee}.

\section{Peter Marshall}

\subsection{Demanding the Impossible - A History of Anarchism \cite{marshallHA}}

\section{Charles Murray \cite{murrayc-tbc,murrayc-hd}}

\section{Douglas Murray}

\cite{murrayd-vi,murrayd-sde,murrayd-tmc}

\section{George Orwell}

\subsection{1984}

\cite{orwell-1984}

\subsection{Animal Farm \cite{orwell-af}}

\section{Camille Paglia}

\subsection{Free Women, Free Men: Sex, Gender, Feminism \cite{paglia-fw}}

\section{Steven Pinker}

\cite{pinker-tbs,pinker-angels,pinker-en}

\section{Helen Pluckrose}

\subsection{The Evolution of Postmodern Thought}\label{EPT}

This following is extracted from the talk \cite{pluckrose-evolution}.

A timeline:
\begin{itemize}
\item[5:41] late 1960s, 66-70: beginning of era of post-modernity.
  Loss of confidence in ``modernity'' which stood for scientific and social progress and objective truth.
  Both revolutionary socialism and liberalism were modernist.
  Loss of credibility in Marxism and liberalism.
  The void was filled by ``post-modernism''.
   Jean Baudrillard, Giles Deluge, Felix Guattari.
   Descriptive, despairing, aimless.
 \item[11:20]Post-structuralist and deconstructionist thinkers:

   \begin{itemize}

   \item[11:30] Jean Francois Lyotard: ``The Postmodern Condition'' (1979); scepticism towards meta-narratives (christianity, marxism, science).
     Language of science inseparable from the language of power and government.
     Instead of meta-narratives, we need lots of mini-narratives, moral and factual relativism.

   \item[12:50] Jacques Derrida: sceptical about the possibility of conveying meaning by language, ``words only refer to other words so meaning indefinitely deferred'', but can be used to express comparisons such as ``men superior to women''.  So he advocated inverting these, to expose and challenge them.  Possibly justifying inverted oppression to redress the balance.

   \item[14:12] Michel Foucault: episteme, power-knowledge, discourses, biopower.
     Knowledge as cultural construct.  We decide what is true and what is known through categories and narratives created and enforced culturally (an episteme).
     Those in power set the episteme, this is power-knowledge.
   \end{itemize}
   The imperative then, of postmodern approaches, is to study the discourses of society, to find the Foucian power-knowledge, invert the Derridian binaries and empower the Lyotardian mini-narratives.
   \item[17:57] This yields the following ``plan'':
     \begin{enumerate}[i)]
     \item there is no way of obtaining objective truth, everything is culturally constructed
     \item society is dominated by systems of power and privilege that people just accept as common sense
     \item these vary from culture to culture and subculture to subculture
     \item none of them is right or superior to any other
     \item the categories that we use to understand things, like fact and fiction, reason and emotion, science and art and male and female, are false
     \item they operate in the service of power need to be examined, broken down and complicated
     \item language is immensely powerful and it is used to construct oppressive social realities, therefore it must be regarded with suspicion and scrutinized to find the discourses of power
     \item the intention of the speaker is no more authoritative than the interpretation of the hearer
     \item the idea of the autonomous individual is a myth, the individual is also a construct of culture programmed by his or her place in relation to power
     \item the idea of a universal human nature is also a myth, it is constructed by what
powerful forces deemed to be the right way to be, therefore it is white Western masculine and heterosexual.
     \end{enumerate}
     These are the ideas from post-modernism which have survived and now appear in the social justice movement.
     Thus ends the ``High deconstructive'' phase of post-modernism.
   \item[19:36] Late 1980s: new generation of leftist academics, legal support for various oppressions disappears leaving only attitudes to be addressed, to which post-modernist ideas can be applied.
   \item[21:06] Post colonialism: Offshoot of post-modernism headed by Foucauldian Edward Said who argued ``The west constructed the East as its inferior in order to construct itself in noble terms'' and that previous colonies should now reconstruct the East for themselves.
     Spivak and Bhabha followed but were more Derridian, and hence incomprehensible.
   \item[21:41] 1989: In ``critical legal studies'' and ``critical race theory'' Kimberlé Crenshaw began developing her concept of intersectionality, which she described as contemporary politics linked to postmodern theory.
     She accepted the cultural constructivism of post-modernism in relation to the concepts of race and gender but believed in the objective reality of oppressive cultural constructs around race and gender.She thought liberalism inadequate despite evidence of its success.
     Liberalism was too universal and an intense focus on identity politics was needed.
     Mary Poovey Adopted a similar stance in relation to feminism, reconciling postmodernist deconstruction with he objective reality of the category of women.
     She advocated a ``toolbox'' approach using post-modern techniques when helpful and not otherwise.
     Judith Butler belief in  objective categories was the problem.
     Queer theory purest form of post-modernism currently in existence.
     
     \item[24] It was now objectively true that social reality was culturally constructed by specific systems of power.

\item white privilege - Peggy Macintosh
\item white complicity - Barbara Applebaum
\item white fragility - Robin DiAngelo
\item re-ified postmodernism - The Creed:
  \begin{enumerate}[i)]
\item  racism exists today in both traditional and modern forms
\item  racism is an institutionalized multi-layered multi-level system that distributes unequal power and resources between white people and people of color, as socially identified, and disproportionately benefits White's
\item  all members of society are socialized to participate in the system of racism albeit in various social locations
\item all white people benefit from racism regardless of their intentions
\item no one chose to be socialized into racism so no one is bad, but no one is neutral so not to act against racism is to support racism
\item racism must be continually identified analyzed and challenged no one is ever done
\item the question is not did racism take place but how did racism manifest in that situation
\item the racial status quo is uncomfortable for most White's therefore anything that maintains white comfort is suspect
\item the racially oppressed have a more intimate insight via experiential knowledge into the system of race than their racial oppressors but they're not bad
  \item however white professors will be seen as having more legitimacy thus positionality must be intentionally engaged (means you must always mention your race gender and sexuality and how it impacts on what you're saying)
\item resistance is a predictable reaction to anti-racist education and must be explicitly and strategically addressed
\end{enumerate}

\end{itemize}

\subsection{Cynical Theories}\label{CT}

How Activist Scholarship Made Everything about Race, Gender and Sexuality - and Why this Harms Everybody \cite{pluckrose-cynical}.

An overall structure something like this:

\begin{enumerate}

\item Postmodernism (1966-1989)

   - Foundational postmodern principles held that objective knowledge is impossible, that knowledge is a construct of power, and that society is made up of systems of power and privilege that need to be deconstructed.

\item Applied postmodernism (80s, 90s)

  \begin{itemize}
  \item postcolonial theory
  \item queer theory
  \item critical race theory
  \item intersectional feminism
  \item disability studies
  \item fat studies
  \end{itemize}

\item since 2010, concretized in the combined intersectional Social Justice scholarship and activism
\end{enumerate}


\subsubsection{Postmodernism}
A revolution in knowledge and power.

Two principles of postmodernism:

\begin{enumerate}
\item Knowledge Principle

  Radical scepticism about objective truth and knowledge, commitment to cultural constructivism.
  
\item Political Principle

  Society is formed of systems of power and hierarchies which determine what can be known and how.
  \end{enumerate}

Four Themes:

\begin{enumerate}
\item blurring of boundaries
\item power of language
\item cultural relativism
\item loss of the individual and the universal
\end{enumerate}

These are core elements of postmodern ``Theory'' which have persisted throughout the evolution of postmodernism and its applications from its initial ``hopelessness'' to its recent strident activism.

\subsubsection{Postcolonial Theory}

Aims to deconstruct the West in order to effect \emph{decolonisation}, the systematic undoing of colonialism.
Prominent in this undertaking are the postmodern knowledge principle (rejection of objective truth) and the postmodern political principle (systems of power and priviledge determining what can be known).

\subsubsection{Queer Theory}

\subsubsection{Critical Race Theory}

\subsubsection{Feminism and Gender Studies}

\subsubsection{Disability and Fat Studies}

\subsubsection{Social Justice Scholarship}

\subsubsection{Social Justice in Action}

\subsubsection{An Alternative to the Ideology of Social Justice}

\section{Karl Popper}

\subsection{The Open Society and its Enemies}

\cite{popperOSE1,popperOSE2}

\subsection{The Poverty of Historicism}

\cite{popperPOH}

\section{Roger Scruton}

\cite{scruton85,scruton15}


\section{Marc Sidwell}\label{Sidwell}

\subsubsection{The Long March \cite{sidwell-long}}

This strategy, named after the long march securing the victory of Communist Party of China, was first enunciated by the German student activist Rudi Dutchke, supported by Herbert Marcuse, and was a replacement for the failed revolutionary expectations of Marxism.

The idea is that student activists should join the various key professions, become competent and progress to positions of power so that these institutions can be subverted from within.
Educational institutions were a first priority, since they educated the new blood which would go into the other professions and institutions, and teacher training of particular interest since activist teachers could then begin the work with younger and more impressionable subjects.

This strategy has proved very successful in propagating critical theories through key social institutions in the UK and in other English-speaking developed nations (USA, Canada, Australia) such as education, the media, and the arts, and thence through quangos and into industry via Human Resource departments.
Other key institutions such as the police and the judiciary have also been effectively targetted through their training professionals and through the bodies writing guidance for police officers, judges and court officials.
In the USA the preponderance of strong left wing politics in tertiary education, particularly in social sciences and humanities, is sufficient to silence many of the few conservatives who survive in that context on any matter which is politically sensitive.
The reach of critical theory leaves little out of its scope, now making ground in the supposedly objective and non-partisan STEM curriculum including mathematics.
The United Kingdom follows in the wake, even issues conspicuously local to the USA readily crossing the atlantic.

\begin{itemize}
\item[CHAPTER ONE] Gramsci’s Ghost 1
\item[CHAPTER TWO] Meet the Blob 11
\item[CHAPTER THREE] The ‘Culture Industry’ Industry 21
\item[CHAPTER FOUR] Wolves in Sheep’s Clothing 31
\item[CHAPTER FIVE] Mao, Marx and Marcuse 40
\item[CHAPTER SIX] The Thatcher Revolution 51
\item[CHAPTER SEVEN] From Political Class to Identity Politics 65
\item[CHAPTER EIGHT] Failing Upwards 77
\item[CHAPTER NINE] The Art of Cultural Resistance 89
\item[CHAPTER TEN] Downstream of Politics 99
\end{itemize}

\section{Peter Singer}

Probably doesn't belong here, but he appears because I am thinking of Ethics as authoritarian and sharing with metaphysics an important \emph{a priori} core, thus epistemologially suspect.
This is not an essential element of ethics but one in which utilitarians are suspects because of their attempt to provide a total (or almost total) moral ordering over our choices, and hence to leave little or no discretion to the moral individual.

Anyway, he came up while we were discussing \emph{equality}, leading into \emph{equity} and at first by my viewing a very short video \cite{singer-eqvid}, after which I went through the free sample of his book on

He is also relevant as a scholar of Hegel and Marx, and the interview with Magee on that topic \cite{magee-singer} is instructive, see Section \ref{Sin-Heg}.

\section{Debra Soh}

\subsection{The End of Gender: Debunking the Myths about
Sex and Identity in Our Society \cite{soh-end}}

\section{Thomas Sowell}

\cite{sowell-barbarians}


\section{Miscellany}

\subsection{A compendium of Free Black Thought \cite{free-black-thought}}.

\chapter{Some Issues}

\section{Social Justice and Identity Politics}

These are underpinned by a variety of ``critical theories'' which are in turn underpinned by the ``Critical Theory'' of the Frankfurt School and Postmodern Philosophy.

\subsection{pre-history}

\subsubsection{Hegel and Marx}

{\bf\emph{Notes from Magee and Singer}}.\label{Sin-Heg}

Magee\cite{magee-singer} summary:

\begin{enumerate}
\item understanding reality = understanding a process (of perpertual change)
\item what is changing - ``geist'' (mind/spirit)
\item why changing? - because in a state of alienation
\item what is the process of change? - the dialectical process
\item where is it going? - politically: to organic society; philosophically: to absolute knowledge
\end{enumerate}

Note that (in this sketch) Magee does not mention Hegel's notion of freedom, and does not give particular prominence to the alleged logical necessity in the dialectic. 

Left and Right Hegelians (young and old)

Right (conservative) took Hegel to be describing the Prussian state and therefore that no radical changes were required.

Left (radical) took Hegel to be concerned with overcoming the conflict between reason and desire or between morality and self interest, a very large undertaking, hence requiring revolution (though Hegel did not advocate it).

Marx was a left Hegelian and carried over 1 and 3-5 above, but eliminates ``geist'' as the subject of the process in favour of matter, hence ``dialectical materialism''.

{\bf\emph{Notes from Sabine}}.\label{Sab-Heg}

Sabine \cite{sabine63} seems to me to penetrate deeper in his first explanation of Hegel's logic, along the following lines.

Hegel's logic is intended to overcome the constraints advertised (e.g. by Hume) on analytic (deductive) logic and implicit in the tripartite division between truths of reason, empirical truths and value judgements.
Hegel offers in his dialectic a kind of reason which tells us about how history progresses and gives us a logical justification for matters including morals and religion.

Sabine mentions ``three vaguely similar generalisations'':

\begin{enumerate}
\item universal human progress (inherited from the enlightenment, Turgo, Condorcet)
  
\item logically necessary historical development (of national cultures)
  
\item Darwin's theory of ``organic evolution''
\end{enumerate}

The failure to distinguish these very different notions of progress (of which (2) is Hegel's historicism) caused great confusion.
Sabine considers (3) to be irrelevant to (1) and (2), and considers (1) to be revolutionary in tendency while (2) in Hegel's conception is conservative (but is later transformed by Marx into a revolutionary theory).

The main distinctive feature of (2) is that it is held to be a matter of \emph{logic} rather than of \emph{empirical causation}.
Hegel, and later Marx, regarded (1) (and (3)?) as ``philosophically superficial'' \emph{because} they are empirical theories.
Hegel's purpose was to demonstrate the logically necessary stages by which human reason approximates the absolute.

\subsubsection{Nietzsche}

Its not clear to me that this is important to critical theory, but it is a line of thinking leading to totalitarian politics.

Notes from Magee and Stern \cite{magee-stern}.
The main traditions in Western Civilisation which Nietzsce attacked were:

\begin{enumerate}
  
\item Christian morality

Dismisses christian virtues such as ``turning the other cheek'', compassion, humility.
  
\item Secular morality

Dismisses the generalised moral codes which occurs in secular moral theory.

\item Herd morality

The heroic individual should be a law unto himself.
  
\item Some traditions from ancient Greece (Socrates)
  
\end{enumerate}


\subsection{roots}

The three pillars which support Social Justice ideologies (collectively ``critical social justice'' of which ``critical race theory'' is perhaps the most controversial) are:

\begin{itemize}
\item Critical Theory
\item Postmodern Philosophy
\item The Long March through the Institutions
\end{itemize}

The first two are philosophical, the last is an activist strategy.

\subsubsection{Critical Theory}

``Critical Theory'' as capitalised refers to the theories of the Frankfurt School, the later more activist descendents (e.g. critical race theory) lose the capitals.

The Frankfurt School was established as the Institute for Social Research by Felix Weil in Frankfurt with an endowment from his father Herman Weil.
At first a group of orthodox Marxists, it developed from 1930 under the leadership of Max Horkheimer its own distinctive ``Critical Theory'' of Society \cite{horkheimer-trad,horkheimer-crit}.


Some aspects of Critical Theory:

\begin{itemize}
\item Integration of Theoretical, Practical and Normative

  Critical Theory must explain what is wrong with current social reality, identify the actors to change it, and provide both clear norms for criticism and achievable practical goals for social transformation.
  
\item Emancipation of the individual
  
\item Democratisation of society

  “all conditions of social life that are controllable by human beings depend on real consensus” in a rational society (Horkheimer \cite{horkheimer-crit})

  \item Intolerance
\end{itemize}

\subsubsection{Postmodern Philosophy}

See Pluckrose and Lindsay Sections \ref{EPT} and \ref{CT} 

\subsubsection{The Long March}\label{LongMarch}

See Marc Sidwell, Section \ref{Sidwell}

\subsection{Gender}

\subsection{Race}

\section{Critical Pedagogy}

See: Gottesman section \ref{Gottesman} and Friere section \ref{Friere}, Henry Giroux \ref{Giroux}.


\section{Freedom of Speech}

\section{Due Process}

\section{Conflicts of Interest}

\section{The Nature of Democracy and The Risk of Subversion}

\section{Pushback}

Here some notes about those who are trying to reverse some of the  trends we have been considering.

First a few words about the phrase \emph{anti-democratic}, which concerns us here because that's the purpose of this document, even though we might debate whether all of the things noted here really are anti-democratic, or whether that's their most important disadvantage.

Many, perhaps most, allegations of anti-democratic or authoritarian are made by the political left-of-centre, about the political right-of centre, sometimes associated with an accusation of fascism.

For present purposes I am considering counter-reactions to doctrines which may be considered anti-democratic because they involve deceit, sometimes in the form of denial of scientifically well established facts, and/or systematic attempts to prevent the articulation not just of particular values or preferences, but also of objective factual knowledge on which they are based.

A healthy democracy depends upon openness and honesty, in default of which confidence in the democratic process will wither and alternative means of resolving differences will be incentivised.
For this reason the line of thought coming down into social activism from Hegel and Marx through Critical Theory, Postmodern Philosophy and the various applied critical theories can be considered compromised and my attention is here directed to some of the pushback against these lines of thought.

Notwithstanding their heritage, it is the continued practice of manipulation of language and self-serving unjustifiable alternative epistemologies which I would like push nack against, but the most substantial pushback will inevitably be provoked by the socal consequences rather than the philosophical mis-steps which contributed.

\subsection{Gender Critical Feminism}

I'm not convinced that it is correct to include ``feminism'' in the title here, since the sources which I will mention here are primarily concerned with exposing various facts the denial of which is facilitating what many consider an assault on ``womens rights'', but are also alleged to underpin practices which are damaging to minors, both psychologically and physically.




\phantomsection
\addcontentsline{toc}{section}{Bibliography}
\bibliographystyle{rbjfmu}
\bibliography{rbj}

%\addcontentsline{toc}{section}{Index}\label{index}
%{\twocolumn[]
%{\small\printindex}}

%\vfill

%\tiny{
%Started 2020/01/17


%\href{http://www.rbjones.com/rbjpub/www/papers/p032.pdf}{http://www.rbjones.com/rbjpub/www/papers/p033.pdf}

%}%tiny

\end{document}

% LocalWords:
