% $Id: p034.tex $fi
% bibref{rbjp034} pdfname{p034}
\documentclass[10pt,titlepage]{book}
\usepackage{makeidx}
\usepackage{turnstile,amssymb}
\newcommand{\ignore}[1]{}
\usepackage{graphicx}
\usepackage[unicode]{hyperref}
\pagestyle{plain}
\usepackage[paperwidth=5.25in,paperheight=8in,hmargin={0.75in,0.5in},vmargin={0.5in,0.5in},includehead,includefoot]{geometry}
\hypersetup{pdfauthor={Roger Bishop Jones}}
\hypersetup{pdftitle={Miscellany}}
\hypersetup{colorlinks=true, urlcolor=red, citecolor=blue, filecolor=blue, linkcolor=blue}
%\usepackage{html}
\usepackage{paralist}
\usepackage{relsize}
\usepackage{verbatim}
\usepackage{enumerate}
\usepackage{longtable}
\usepackage{url}
\newcommand{\hreg}[2]{\href{#1}{#2}\footnote{\url{#1}}}
\makeindex

\title{Miscellany}
\author{Roger~Bishop~Jones}
\date{\small 2021/12/05}


\begin{document}
\frontmatter

%\begin{abstract}
% This is a brute consolidation of many abortive fragments (p034-p051) of writing, as a temporary staging post while I decide what to do with it all (if anything).
%\end{abstract}
                               
\begin{titlepage}
\maketitle

%\vfill

%\begin{centering}

%{\footnotesize
%copyright\ Roger~Bishop~Jones;
%}%footnotesize

%\end{centering}

\end{titlepage}

\ \

\ignore{
\begin{centering}
{}
\end{centering}
}%ignore

\setcounter{tocdepth}{2}
{\parskip-0pt\tableofcontents}

%\listoffigures

\mainmatter

\pagebreak

\part{Evolution}

\chapter{Philosophy and Evolution}

The ideas explored here arise primarily from my dismay on discovering (somewhat belatedly) the extent to which ``Critical Theory'' and its explicitly irrational progeny, (gathered together under the rubric ``Critical Social Justice'' or more popularly ``Equity''), have penetrated the social institutions of the most prosperous and humane societies in history.

My reaction has been to try to understand how it could have been possible, and to seek a defence of some aspects of our cultural heritage, for example the notion of objective truth, the fundamental (if still controversial) distinctions between truths of logic, empirical facts and value judgements, and the merits of liberal democracy.
In trying to understand both the roots of rationality and of those social tendencies which fly in its face, the theory of evolution could not be ignored, and as I struggled in that endeavour my own sense of the significance of the theory of evolution to philosophy grew.

On the other hand, studying evolution from a philosophical perspective may also cast light, and it is my hope in this essay to give an account of evolution, insofar as it relates to these questions, which benefits from my own philosophical perspective.
I seek both philosophical and biological enlightenment by considering each through the lense of the other, and I hope in the essay that the understanding which I am approaching might possibly have some interest to others.

This essay is not a culmination.
I have separated out this relationship, between philosophy and evolution, so that I can consider it detached from the controversies which lead me to think it important, and I look forward to discovering what I can make of it thus detached.
It is a progression whose merit lies in its continuation rather than its completion.

\section{Introduction}

The words `philosophy' and `evolution' have been used in more ways than I can ever know.
Though this essay is broadly scoped, it does not aim completely to comprehend that breadth of usage.

My aim in this essay is to defend aspects of that cultural tradition which is first exhibited in the writings of Plato and Aristotle, after the astounding mathematical successes and embarassing philosophical disarray of the pre-socratic `philosophers', all attempting with varied degrees of success to secure knowledge by observation and reason.
The greatest substantive fruit from this tradition did not appear until it spawned the industrial revolution whose economic consequences transformed, and continue to transform, the prosperity and well being of \emph{homo sapiens}.
These are, I suggest, the fruits of \emph{rationality}, and it is under the umbrella of rationality that my defence of this tradition will be conducted.
This involves me in taking a particular stance both on what rationality is, and on what is rational, which will gradually unfold through the essay.

It is something of a surprise to me that I should be engaged on such an enterprise, since for most of my life I have been more inclined to criticise than laud many of the contemporary manifestations of that tradition.
My critical inclinations were nevertheless very much \emph{from within}.
My desire to see the tradition as a whole demolished and discarded is nil.
So this will be a qualified defence.
A defence against wholesale destruction, but not against forceful critique.
Forceful critique is of course part of the tradition, that is, critique from within a tradition of diverse thought and custom of some part of that tradition, not of the whole, which only becomes a single whole in the ideas of those seeking to undermine the whole that they create.

What is the threat against which a defence is warranted?

It is the `Critical Theory' of the Frankfurt school of neo-marxist scholars, the scepticism of Postmodern philosophy and the subversive activism spawned by `Critical Social Justice' pseudo-scholarship.
I name these as inspiring this response, but will give only the barest sketch of the challenge which they present.
A broader defence is intended, one which addresses also problems more fully within the tradition, one intended to have relevance even to those not convinced of the all the threats I perceive, and which may have continuing relevance when those threats are past.

How does evolution figure in this?
There were two initial provocations for me to have thought that an evolutionary perspective might help.

First, we have the explicit critique of rationality which comes with critical social justice scholarship and activism, possibly derived in part from the scepticism of postmodern philosphy.
This critique identifies all aspects of the western tradition as a white supermacist power-play.
It is shocking that any defence should be needed against this position, but I must reluctantly concede that it is.
It is to the origins of language, reason and culture that I turn, and it is to the theory of evolution that I look for enlightenment on those origins.

Alongside my attempt to cast light on how we humans came to be capable of rationality I shall consider the genesis of the ways in which we continue to be capable of casting aside reason.
Contemporary manifestations of Critical Theory may seem the most egregious illustrations of that capability, but cultural evolution furnishes a broad range of examples pertinent to the conundrum of how the tender flower of liberal democracy can be preserved against the totalitarian vandalism of ideological high fashion.

By taking an evolutionary perspective on philosophy with a philosophical view of evolution I invite chicken and egg issues throughout.
As a result the essay will be a smorgasborg of historical narratives, sometimes passing over the same ground more than once with a new or refined perspective.

Before considering evolution at all, I will sketch aspects of the development of the idea of scientific theory over the last couple of millenia, in the light of which the theory of evolution and its development over the last two centuries will be lightly drawn.
Then our idea of evolution is refined by considering the various distinct kinds of evolution which have lead us into our present predicament.

\section{What is Rationality?}

Before looking at the theory of evolution, let me say a few words about the direction from which I will be approaching it.
I have already mentioned the important place of rationality in the tradition which I seek to defend, and I will briefly expand upon that by clarifying how I propose to use that concept and how the essay will revolve around it.

I will take as fundamental the concept of \emph{instrumental} rationality, which is exhibited when actions are taken with some purpose in mind which are likely to realise that purpose.
This gloss is a good starting point, but will not be religiously adhered to.
It may be that no course of action is likely to achieve some desperate purpose which cannot be cast aside, and in that case it is surely rational to adopt the course least likely to fail.
So perhaps I should be demanding as rational, that course most likely to succeed, however unlikely that may be.
Perhaps we should also take into account the state of knowledge or ignorance of the agent involved, for surely he cannot be held irrational for adopting after due diligence, the course which seems most likely, even if in fact it is unlikely to succeed for reasons not readily discovered.

In the further analaysis of the concept of rationality we may distinguish two general kinds of consideration.
Those which bear upon the \emph{meaning} of the concept, and those which are concerned, supposing meaning to have been elucidated, with the question of what then falls under the concept.

Leaving for the moment questions of meaning.
It has been important, in the tradition which I am defending, to establish epistemic standards for scientific scholarship which are more rigorous than may be necessary in everyday life.
That is to say, standards which determine (or influence) when some supposed scientific truth is to be considered established.
To have such standards, we may argue, is instrumentally rational, for the purpose of securing prosperity and well-being.
Adherence to such standards we will call \emph{epistemic} rationality which is therefore an aspect of instrumental rationality.
There is a reasonable presumption here that scientific knowledge will be generally applicable, and that accuracy in this body of knowledge is instrumental even if no particular purposes are under consideration.
I do not seek here to deny the importance of choices about what scientific research `should' be conducted, be that a moral or an instrumental consideration, but rather to suggest that if research is to be undertaken, for whatever purpose (or none), then getting the right answer will be instrumental for that purpose (or for any unantipated applications that come up).

We derive the notion of epistemic rationality from instrumental rationality even though establishing the truth of our hypotheses does not have a specific purpose or end, because we consider true knowledge as a resource generally applicable to whatever purposes we have in mind.
This applies whether or not we can anticipate the applications which it may have, because we know of many cases where knowledge proves to have applications which had not been anticipated.
This does not mean that resources for research should not be allocated taking into account the probable applications, but it does help to justify the funding of fundamental research for which applications have yet to be discovered.

\subsection{Head and Heart}

According to Blaise Pascal ``The heart has its reasons which reason does not know''.
According to popular culture, reason and emotion are opposed, rational and emotional behaviour are distinct.

The view of rationality which I take here is more conciliatory.
I acknowledge that an act precipitately spurred by emotion is unlikely to be instrumentally rational, but emotions are motivating, they influence our purposes and in that way feature in rational behaviour, directing rather than conflicting with the dictates of reason.
Emotions direct our attention to important but unexpected happenings or outcomes, things which demand attention.

As well as influencing our purposes and hence the actions which are intrumental in realising those purposes, emotions may provide some guidance on the feasibility of a course of action.
If our own proposed contribution to effecting some plan seems to us morally repugnant or personally unpleasant, this may tell us something about the likelyhood that the plan will succeed.
It is not likely to be instrumental to ignore emotions in determining the course of action most likely to succed.

The everyday question of whether emotional sensitivity should mitigate or overturn a preference for rational effectiveness was played out on a larger canvass in the cultural transition from the age of enlightenment to the subsequent era of romanticism, according to some perspectives on the complex cultural trends around the turn of the 19$^{th}$ century.
This will feature in the discussion later in this essay of the origins of irrational ideologies which became major social influences in contemporary English speaking societies.
That perspective sees the triumph of rational science celebrated in the enlightenment over religion authority as underpinning a secular authoritarianism which sought to order society following a science and philosophy which told us not only about the physical world, but also about hearts and minds, the workings of society, and ultimately about how we should be governed, without needing to pay too much attention to individual preferences.
The desire to determine what is good by rational deliberation has tempted philosophers throughout the ages, but it has no place in the conception of rationality here.
The rationalist philosophers were not in that sense instrumentally rational, for the attempted to determine by purely rational means matters which are in practice only resolvable, on the one hand, by empirical reasearch, and on another by consulting the heart and soul of all.

\subsection{Rationality and Logic}

Often, rationality is thought of as \emph{being logical}, as associated with some particular way of thinking which excludes the intuitive and emotional.
This is close to the way in which rationalist philosophers are thought of, as holding that true knowledge comes only from reason, and perhaps only that particularly rigorous form of reasoning which we call \emph{deductive}.
But then, the kind of reasoning for which Sherlock Holmes is renown is thought of as deductive.

Deductive reason, as applied particularly but not exclusively in the proof of mathematical theorems, has a solid reputation, which is wholly deserved, as an unimpeachable imprimatur of authenticity.
Its reputation has therefore been envied and usurped over and over, by those who care little for confining it to its proper domain.

Hume had it right, in confining its scope to `relations between ideas', and hence incapable of supporting unaided any truth about the material world we inhabit.
A fortiori, following Hume, deduction alone yields no knowledge of moral truths.
But those who believe that they know virtue, rarely aknowledge the tenuous foundations on which it rests, and if they cannot give it devine authority, the pretence to logical proof may seem essential.
By contrast, Hume's scepticism in this matter was to prove helpful to religious dissidents keen to assert the sufficiency of faith against reason.

If our conception of rationality is \emph{instrumental} then we must acknowledge that the reasoning which has ultimately been instrumental for our health and prosperity lacks anything close to absolute authority, but suffices for the purpose.
The logic of science is to some extent deductive in its application, but much more tenuous in establishing any principles from which empirical conclusions can be derived.
Beyond that ideal \emph{nomological-deductive} model, the great mass of practical scientific knowledge is more informal, and the truths of `evolutionary theory' are difficult to corral under any clarly articulated scientific method.
Nevertheless, it is to some understanding of evolution that I turn for insight into human nature, to understand how we might secure the instrumental advantages of rationality and mitigate the potentially destructive forces of tribal ideologies.

\subsection{The Rationality of Evolution}

The process of evolution often produces results which \emph{seem} to be rational.
A classic example is the intricate structure of the human eye, which appears to have been rationally designed to enable us humans to see, even though it is generally considered that evolution proceeds with no purpose in mind.
It is an interesting feature of work in synthetic biology that having developed the technology to undertake evolution \emph{by design}, it turns out that using a customised evolutionary process is sometimes the best way to get the desired results.
In that case the evolution does have a purpose, conceived by a biologist, and is a rational approach to realising that purpose.

The rationality of evolution is not merely evident in the outcome of particular evolutionary adaptations.
Evolution itself evolves, and we can see that this process itself is rational, insofar as the transformations make evolution itself more effective.
The process of evolution designs processes, it crafts ways of living, and it is often possible to see the role of any particular feature in the evolved behaviour in facilitating the replication of the organism.
Insofar as evolution is effective in advancing the reproductive fitness of the organism it will be because the behaviours it engenders are instrumentally rational for the purpose of replicating the organism.
Thus we can expect that evolved organisms will be designed to \emph{behave} rationally.
As organismd and their nervous systems become more complex, and are exposed to climatic volatility an natural crises, it becomes possible, and is intrumentally rational, for them to have less rigid and more adaptable behaviours.
The evolution of nervous systems which balance established behaviours and dispositions with more or less intelligent adaptability in the face of new environmental challenges is a rational effect of evolution, and it plausible that the adaptability is itself delivers more effective responses than would have resulted from the prior more rigid behaviour patterns.
Here the rationality has stepped up a level.

A further step upwards occurs with the evolution of culture and the subsequent cultural evolution.
At this stage we see the emergence of something quite close to the kind of rationality which we might hope to see in homo sapiens.
Evolution has engineered by rational means, a species which is capable of ratiocination and may think and behave rationally.

\subsection{Rationality, Objectivity and Absoluteness}

Whether some course of action is instrumental or not depends upon the purpose for which it was undertaken, which may not be explict or clear.
Even where there is an explicit and clear purpose in play, it may be that the agent adopting the course of action is in fact covertly motivated by some other purpose.

This is relevant to the establishment of institutions, for one of the cases in which this divergence may take place is where a course of action on the part of an employee of an instution  which would be instrumental in promoting the ends of the institution is contrary to the interests of the individual.
In consequence, an institution may act irrationally even though its employees can be seen to be acting rationally.
It is therefore desirable to ensure that an organisation is so structured that when its members acti in their own interests they are acting in a way likely to contribute in the intended way to the achievement or the organisations purposes.

The potential divergence in consideration of instrumental rationality, according to whose purposes are definitive, may expressed by saying that judgements of instrumental rationality are not absolute, but are relative to purpose.

Epistemic rationality is not quite the same.
In this case purpose does not enter in (or we may say that the purpose is to establish truth).
The question of the objectivity of truth does matter.
If truth is objective, then epistemic rationality may be absolute.
Otherwise, it will be relative to those same considerations which influence truth, relative to which truth is determinate.

I will later argue that definiteness of meaning, or more particularly of truth conditions, is an essential condition of the selective advantages which enables language to evolve, that objectivity of truth follows from it, and hence that epistemic rationality has claim to being an absolute notion.
Notwithstanding that claim, I recognise that meaning will never by \emph{abolutely} precise and unambiguous, and that some important domains of discourse are enormously difficult to articulate precisely and consequently sustain language which may be virtually meaningless.


\section{What is a Scientific Theory?}


I'm hoping to apply the theory of evolution to illuminating the nature and origin of rationality and of the forces which work against rational progress.
To understand the cultural tradition which has most explicitly emerged from the rationality of humans, we will be tracing the development of that culture to the present day, and one aspect of the culture which we will trace is the development of scientific methods.

Before looking at the theory of evolution, some consideration of the `scientific method' exemplified by that theory is appropriate.
If not definitive of epistemic rationality, received scientific method is an important part of how we judge whether scientific theories are rational, and the history of scientific method therefore belongs within a broader account of the cultural evolution of rationality.
This first sketch of elements of that history precedes discussion of evolution, and will be augmented in the light of the evolutionary theory in due course.

I first sketch some thoughts about the development of scientific method, in order to determine what sort of a `theory' the science of evolutionary biology might have.

\subsection{Axiomatic Method}

The first theoretical science was mathematics, which began, as such, in ancient Greece, approximately 600 years before Christ.
In its first 300 years Greek mathematics was stunningly successful.
It established cumulatively a body of proven mathematical knowledge which was collected together in Euclid's Elements \cite{euclidEL1}, using a clearly documented and reliable systematic method, `The Axiomatic Method', based around \emph{deductive} inference.
Euclidean geometry was so solid that once knowledge of it returned to Western Europe after the dark ages, it continued to be taught in schools until well into the 20$^{th}$ Century.

The axiomatic method involved adopting certain principles, notably axioms and definitions, and deriving an extensive body of knowledge from those principles by deductive proof.
There is at this stage no known disussion of what deduction is, the meta-theory of deduction, and more broadly of demonstrative science, begins with Aristotle (who explicitly lays claim to precedence in this).

\subsection{Plato's Two Worlds}

\subsection{Aristotle's Demonstrative Science}

\subsection{Empirical Science}

\subsection{Hume and Positivism}

\subsection{Carnap's Logical Positivism}

In the conception of philosophical analysis put forward by Rudolf Carnap early in the 20$^{th}$ century, the product of such analysis must be those logical propositions which were called `analytic' sometimes glossed as `true in virtue of meaning'.
So strict a conception of philosophy left no room for practical philosophy to reach conclusions which bear on how we conduct our lives, and entailed a clear distinction between philosophical analysis and empirical science.
Carnap was however, by no means an ivory tower rationalist.
He was an empiricist (even a positivist), and his philosophical programme saw analytic philosophy as handmaiden to empirical science, providing the analytic tools necessary for science to be conducted in a logically rigorous manner, just as before him Aristotle had offered his works on logic not as science but as tools for conducting `demonstrative science'.

The uncompromising ideal pursued by these men, separated by two millenia of philosophical and scientific controversy and progress, remains tantalisingly beyond practical reach for most science.

\subsection{Naturalised Philosophy}


\subsection{Evolution as Tautology or Method}

Biology, the discipline most clearly transformed by the theory of evolution, and sociobiology or 
evolutionary psychology, present particular problems for rigorous conceptions of scientific methods, but will be essential to the discussion of how rationality came to be, and why it so often seems to have been abandoned.

The peculiar status of evolution in respect of scientific method may be seen in the two first books by Richard Dawkins.
In ``The Selfish Gene''\cite{dawkinsSG} Dawkins assures us that his central thesis, around which the entire book turns, is a \emph{tautology} (along the lines that a gene will proliferate in the gene pool if the phenotype it codes for is conducive to the replications of the gene, though that claim falls short of being a tautology).
Whether or not Dawkins' claim is correct is not important here, what's important is that Dawkins was comfortable building an empirical text apparently by deduction from a non-empirical foundation.
In his second book, ``The Extended Phenotype''\cite{dawkinsEP} Dawkins makes a different observaton about the first ``if adaptations are to be regarded as `for the good of' something, then that something is the gene'', and observes of the central thesis of the new book ``since it is not a factual position I am advocating, I warn the reader not to expect `evidence' in the normal sense of the word``.


\section{The Theory of Evolution}

\subsection{Darwin's Theory}

This is a very lightweight discussion of Darwin's \emph{theories}, and some of the subsequent work which has added to the \emph{theory} of evolution.
This is to be understood as distinct from the very much greater volume and detail of the study of what actually happened in the evolution of life on earth.

To clarify that distinction just a little, Darwin had a theory, primarily about the evolution of species, and he spent a lifetime gathering data about evolution in order to show as conclusively as possible, the truth of his theory.
The theory itelf has a relevance which transcends the historical (geological) data which supports it.
If true, it tells us not only about how life evolved, but also about how it will evolve in the future.
The facts about what actually happened don't do that, until they are transformed into general theories describing enduring features of evolutionary biology which will continue to be true into the future.

Darwin was not the first to have ideas about the evolution of life, but he was the first to have a theory which has stood the test of time and of extended empirical investigation.
In its most concise form that theory is:
\begin{enumerate}
\item Living organisms on earth fall into \emph{species}, which are groups of similar but not identical organisms.
\item An essential feature of living organisms is that in their normal environment they are capable of reproduction, which is to say, of producing new and similar (but not identical) organisms.
\end{enumerate}

  \subsection{The Modern Syntheses}

  \subsection{The Gene-centric Variation}

  \subsection{Extended Evolutionary Syntheses}

  \subsection{Tinbergen's Four Questions}

  \begin{itemize}
  \item[First:] what is the function of a given trait (if any)? Why does it exist compared to many other traits that could exist?
  \item[Second:] what is the history of the trait as it evolved over multiple generations?
  \item [Third:] what is its physical mechanism? All traits, even behavioral traits, have a physical basis that must be understood in addition to their functions.
    \item [Fourth:] how does the trait develop during the lifetime of the organism?
  \end{itemize}
  \footnote{from Tinbergen \cite{tinbergen-oame} as presented by Sloan Wilson in \cite{wilson-tvl}}
  
  \section{Stages in Evolution}

  \subsection{Pre-Biotic Chemical Evolution}

  \subsection{Prokaryotic Evolution}

  \subsection{Sexual Selection}

  \subsection{Culture}

\chapter{The Evolution of Culture}

\section{The Beginning of Culture}

When culture begins depends upon what we consider culture to be, how we \emph{define} the term.
The broader the definition, the more it encompasses and the sooner we may consider it begun.
In this context culture should be understood as knowledge transmitted from one generation to the next \emph{non-genetically}.
Other candidate criteria include:

\begin{enumerate}
  \item transmission of adaptations
  \item and also shared within a generation, not just passed from parent to child.
  \item communicated by language
\end{enumerate}

\section{Where are We?}

Before considering where evolution will take us in the future, a few words about where we now are may provide some basis for projection into the future.

\subsection{The State of Evolution}

Evolution itself is at a point of inflection.
Biological evolution, the evolution of the human genome and the ecosphere we share, may be about to experience its most profound transformation.

Arguably, the dominant form of human evolution is now by \emph{cultural} selection.
To an increasing extent the evolution of the rest of the earth's ecosphere is influenced, for better or worse, by human culture, the technology we have developed and its intended and accidental effects.

To an unprecedented extent technology and the welfare state have marginalised the significance of genetically determined phenotypic factors which might limit human proliferation.
Though it is at this moment of marginal significance, the scope of cultural selection can now be extended by the techniques of synthetic biology.
These will allow cultural selection to take effect by specific genetic interventions rather than purely by selection of mating partners.
These interventions at present can be undertaken by sequencing of embryos and selection among them on the basis of detailed knowledge of the genomes, and will likely eventually to embrace editing of the genome, or even synthesis of a genome, techniques colloquially said to yield ``designer babies''.

These methods, to the extent to which they may be permitted,  belong to the future.
The novelties which characterise the present state of evolution are ones which arise from cultural evolution, and from the technical advances which have accelerated that process.


\part{Political}

\chapter{A Story of Enlightenment}

Speculating about Enlightennent 2.0 on the basis of ideas about its predecessors.


\section*{Preface}

I have considered for some time the origins of and development of `Enlightenment' values and methods and of the present deluge of their antitheses.
The term `enlightenment' is now geneally associated with a period of history spanning perhaps 50 perhaps 150 years, during which at least some intellectuals sought progress through the application of reason and science, rather than through deference to authority, scripture or divine revelation.

My aim here is to enter into the kind of controversy which was typical of the enlightenment by working towards some 

It may be best to think of the essay as a work of fiction, based on what I believe the history of \emph{homo sapiens}.
The fiction draws out some hypotheses about why things are as they are, how we came to be, leading to some ideas about how our future might go.
The aim is constructive, the ideas
What is intended here is a blueprint of some of the key features of a global society which facilitates the fulfilment%
\footnote{satisfaction or happiness as a result of fully developing one's potential}
of all of our species and its progeny.
Despite that forward looking aspiration, the method is to look at the past for help in understanding the present, and to address primarily how we should be now to secure the future we desire and deserve.

Though I talk about what we `should' do, I do not offer these ideas as moral imperatives.
They are simply my preferences, which I recommend for your consideration.


The aspects of interest here are primarily sociopolitical.


\footnote{There may be ``hyperlinks'' in the PDF version of this document which either link to another point in the document  (if coloured blue) or to an internet resource  (if coloured red) giving direct access to the materials referred to (e.g. a Youtube video) if the document is read using some internet connected device.
Important links also appear explicitly in the bibiography.}

\pagebreak
\section{Introduction}

One of the responses to various strains of radical progressive thought and activism, particularly those most critical of ``Western Civilisation'' has been to reject the objectivity and the evidential basis for the critique, to highlight the evidence for continuing progress in most objective measures of well-being and to reassert the values of the enlightenment and the instrumental attitude to incremental progression which they facilitate.
Stephen Pinker is a prolific advocate for similar ideas \cite{pinker-angels,pinker-en}.

The idea that one might return, lock stock and barrel, to the values of certain intellectuals in the 18th Century is not plausible, and will be even less acceptable to the many progressives and revolutionaries to my left.
Its not one I attribute to Steven Pinker, despite his book ``Enlightenment now''\cite{pinker-en}, which nevertheless provides fodder for my ruminations here.
If there are aspects of enlightenment thought which deserve to be preserved, then they need to be distinguished from those aspects which do not, and perhaps some account of how they could and should be restored attempted.
For the sake of that desire for progress which is held by Pinker to be a part of enlightenment ideals, our conception of ``enlightenment values'' must surely must surely recognise the diversity of thought in that period of history, and the insights which the passage a quarter of a century now makes possible into its wisdom.

My aim in this essay is to strip out or `reconstruct' from enlightenment thought and values a core which might form a politically neutral basis for the future of a diverse democratic society.
I am looking for a core of theory and value which are as close to politically neutral as once can get while trying to establish and maintain a healthy democratic society.

That this cannot be entirely neutral politically is ensured by the existence of ideologies which are essentially (if not explicitly) antithetical to democracy.
Herbert Marcuse held that capitalism could not be reformed from within \cite{marcuse-repressive, marcuse-liberation}, but must be entirely dismantled before a just society could emerge, and advocated intolerance of any contrary opinion.
He was associated (at least as far as endorsement) with the non-violent (if not wholly democratic) strategem, advanced by the student political activist Rudi Dutchke, which was named ``The Long March Through the Institutions'' \cite{sidwell-long}.

The basic complex of ideas and principles thus concocted I refer to in this essay as ``Enlightenment 2.0'' to reflect the desire to carry forward the most crucial aspects of the first enlightenment (Enlightenment 1.0) augmented to cope with the special challenges and opportunities of the present.

Many of the ideas and principles which are fundamental to democracy but which have nevertheless been challenged in 21st and late 20th Century substantially predate the enlightenment.
Some of those are first recorded in classical Greece, which can reasonably be thought of as a major predecessor of the enlighenment (``Enlightenment 0.0''), but others are so fundamental that they date back to the origin of anatomically modern homo sapiens, and are therefore pre-historic antecedents.

Giving credit to those three prominent contributors, the essay falls into four sections:

\begin{itemize}
\item An Overview
\item The Evolution of Homo Sapiens
\item The origins and nature of language and culture
\item A Greek dawn for reflective rationality
\item Some enlightenment philosophy
\item Essential Enlightenment 2.0
\end{itemize}

Interwoven with this progressive enunciation of a positive core of doctrine and value I provide commentary on contrary ideas and tendencies, in which the promiment features include:

\begin{itemize}
\item The eclipse of democracy in Classical Greece and the rise of Christianity and Islam.
\item The romantic sequel to Enlightenment 1.0
\item Hegelian dialectical method
  \item Critical Theory and Postmodern Philosophy
\end{itemize}

In seeking a core for Enlightenment 2.0 which hinges upon arguably necessary conditions for a thriving democratic society my aim is for these to be consistent with as broad a range of political opinion as possible.
It may therefore be instructive to consider where the limits lie, what kinds of doctrine could not be encompassed, and also, how that failure should be expected to work.
This connects with words of Karl Popper on ``The Paradox of Tolerance'' (at first a mere footnote in \cite{popper-ose}) and the writings of Herbert Marcuse who countered with his lengthy essay on ``Repressice Tolerance'' \cite{marcuse-repressive}.
One doctrine which is in this respect beyond the pale is that the society we live in effectively prevents progressive reform from within, and that progress is only possible after its complete demise.
What that means is unclear to me.
In Marx it is the violent overthrow of the \emph{ancien regime} by the proletariat, but later exponents of the same point of view, for example, once again Marcuse (perhap in \cite{marcuse-liberation}) have eschewed that violent prospect in favour of the \emph{Long March} \cite{sidwell-long}.
The strategy and tactics of the long march are also beyond the pale, though one could easily imagine a long march which was not, or indeed, conceive of the Enlightenment 2.0 project (if such it be) which is itself a long march.

%\cite{pluckrose-cynical,lindsay-racemarx,friere-poled,gottesman-criturn}

That I divide the essay into four phases in the history of ``enlightenment'' might suggest that these will be presented sequentially.
I think it would work better to do otherwise, one reason for which is that the significance of the features I would like to highlight of the earlier phases is easier to draw out if at least some of the later developments have been discussed.

\section{An Overview}

\subsection{A Timeline}

``The Enlightenemnt'' is a historical period in which certain ideas and values were promulgated which have been credited for much subsequent progress, but which have remained con


\begin{itemize}
\item Antiquity 800BC-500AD
\item The Middle Ages 500-1500AD
    \begin{itemize}
  \item Early Middle Ages 500-1000
  \item Hight Middle Ages 1000-1300 Scholasticism
  \item Late Middle Ages 1300-1500 Renaissance    
    \end{itemize}
\item Modernity 1500-1960AD
  \begin{itemize}
  \item Renaissance
  \item Reformation
  \item Enlightenment    
    \end{itemize}
\item Post Modernity 1960AD- 
\end{itemize}


\section{Some Scepticism}

The enlightenment is a product of scepticism, but is itself tainted by dogmatism.
In the broader conceptions of ``The Enlightenment'' as a historical period, the ``early enlightenment may by considered to have begin with Descartes, whose work was the most influential of many attemots to secure science by the use of sceptical arguments (drawn from the phyrrhonism of Sextus Empiricus \cite{sextusempiricusOOP}, rediscovered during the renaissance) to secure its independence of ecclesiastical authority while avoiding the potential impact of those arguments on the new science itself.
The progression of scepticism before and during the early enlightenment is documented by Richard Popkin in his ``History of Scepticism'' (from Savarolla to Bayle) \cite{popkin03}.


\section{The Evolution of Homo Sapiens}



\subsection{Central Concerns}

This essay is concerned with the resolution of tensions between two apparent dichotomies which are found in human nature and human customs and institutions.
The first is that betweem violent and consensual mediation of actual or potential conflicts.
The second is the contrast between rational deliberation and `tribal' allegiance.

Our ability to reason, and to apply reason to solving everyday problems has evolved into a distinctive capability of homo sapiens.
Why we have evolved a faculty capable of the heights of creative rationality in accomplishments which could not even have been contemplated during that period of evolution is still, I think, something of a mystery.

At more or less the same time (in geological timescales) the evolution of human culture began.
An essential requirement for the evolution of culture is both cutural diversity and cultural uniformity.
In order for cultural differences to be effective, there must be a reasonable degree of cultural uniformity in some population which is effectively in competition with other populations with distinct cultures.
If we use the term ``tribe'' to denote a culturally uniform population group, then the intellectual diversity to which our rational capability leads may be thought of as taking place within more or less strict bounds determined the dominant, or tribal, cultural constraints.
In matters of domestic or professional life which are neutral relative to this culture, individuals will be free to reason towards the most effective solutions to their problems, which will include the accumulation of objective knowledge which proves useful in that process.





\section{Enlightenment 0.1}

Insofar as we may associate enlightenment with reason and rationality, and even if it is also associated with advances in governance, the enlightenment may be considered to have a precursor in classical Greece.

\section{Enlightenment 1.0}

By this I mean that period in the history of European thought which is normally called ``The Enlightenment'', and which is typically identified in the first instance with the work of the French \emph{Philosophe}s (Voltaire, Montesquieu, D'Alembert, Diderot, Rousseau, ...) in the mid-18th Century.

I wlll be synthesising my own take, not specifically on what were he essential elements of ``The Enlightenment'', but rather, on What I consider the  most important desiderata for ``Elightenments 2.0''.
Here however, I'm aiming to draw out a variety of opinions on what the enlightenment was both from a handful of contemporary philosophers (Hume, Rousseau and Kant) and from some later scholars looking back at it (Berlin, Pinker, William Bristow (SEP)).

\subsection{Kant}

\subsection{Isaiah Berlin}

Isaiah Berlin's take on The Enlightenment comes in two parts.
First three legs upon which the whole Western tradition rested:
\begin{enumerate}
  \item All genuine questions have an answer.

    In principle, by someone.
\item  The answers are knowable.
\item All the answers are compatible.
\end{enumerate}

and then, the extra twist added by the Enlightenment:
\begin{quotation}
That the knowledge is not to be obtained by revelation, tradition, dogma, introspection..., only by the correct use of reason, deductive or inductive as appropriate to the subject matter.

This extends not only to the mathematical and natural sciences, but to all other matters including ethics, aesthetics and politics.
\end{quotation}
and... that virtue is knowledge.


\subsection{Steven Pinker}

\subsection{David Hume}

\subsection{Rousseau}

\section{Enlightenment 2.0}

\ignore{

\appendix

\section{Pinker on Progress}

Is the world getting better or worse? A look at the numbers.

https://youtu.be/yCm9Ng0bbEQ


Many people face the news each morning
with trepidation and dread.
Every day, we read of shootings,
inequality, pollution, dictatorship,
war and the spread of nuclear weapons.
These are some of the reasons
that 2016 was called the "Worst. Year. Ever."

0:35
Until 2017 claimed that record --
and left many people longing for earlier decades,
when the world seemed safer, cleaner and more equal.

0:46
But is this a sensible way to understand the human condition
in the 21st century?
As Franklin Pierce Adams pointed out,
"Nothing is more responsible for the good old days
than a bad memory."

1:01
You can always fool yourself into seeing a decline
if you compare bleeding headlines of the present
with rose-tinted images of the past.
What does the trajectory of the world look like
when we measure well-being over time using a constant yardstick?

\paragraph{}

Let's compare the most recent data on the present
with the same measures 30 years ago.
\begin{itemize}
\item[Last year(2017)], Americans killed each other at a rate of 5.3 per hundred thousand,
had seven percent of their citizens in poverty
and emitted 21 million tons of particulate matter
and four million tons of sulfur dioxide.
\item[30 years ago], the homicide rate was 8.5 per hundred thousand,
poverty rate was 12 percent
and we emitted 35 million tons of particulate matter
and 20 million tons of sulfur dioxide.
\end{itemize}
1:51
What about the world as a whole?
\begin{itemize}
\item[Last year]
  \begin{description}
    \item[] the world had
    \item[12] ongoing wars,
\item[60] autocracies,
\item[10] percent of the world population in extreme poverty
  and more than
\item[10,000] nuclear weapons.
  \end{description}
\item[30 years ago]
  \begin{description}
  \item[]  there were
    \item[23] wars,
\item[85] autocracies,
\item[37] percent of the world population in extreme poverty
  and more than
\item[60,000] nuclear weapons.
   \end{description}
\end{itemize}

True, last year was a terrible year for terrorism in Western Europe,
with 238 deaths,
but 1988 was worse with 440 deaths.

What's going on?
Was 1988 a particularly bad year?
Or are these improvements a sign that the world, for all its struggles,
gets better over time?

2:39
Might we even invoke the admittedly old-fashioned notion of progress?

2:45
To do so is to court a certain amount of derision,
because I have found that intellectuals hate progress.
And intellectuals who call themselves progressive really hate progress.
Now, it's not that they hate the fruits of progress, mind you.
Most academics and pundits
would rather have their surgery with anesthesia than without it.
It's the idea of progress that rankles the chattering class.
If you believe that humans can improve their lot, I have been told,
that means that you have a blind faith
and a quasi-religious belief in the outmoded superstition
and the false promise of the myth of the onward march
of inexorable progress.
You are a cheerleader for vulgar American can-doism,
with the rah-rah spirit of boardroom ideology,
Silicon Valley and the Chamber of Commerce.
You are a practitioner of Whig history,
a naive optimist, a Pollyanna and, of course, a Pangloss,
alluding to the Voltaire character who declared,
"All is for the best in the best of all possible worlds."

3:58
Well, Professor Pangloss, as it happens, was a pessimist.
A true optimist believes there can be much better worlds
than the one we have today.

4:06
But all of this is irrelevant,
because the question of whether progress has taken place
is not a matter of faith
or having an optimistic temperament or seeing the glass as half full.
It's a testable hypothesis.
For all their differences,
people largely agree on what goes into human well-being:
life, health, sustenance, prosperity, peace, freedom, safety, knowledge,
leisure, happiness.
All of these things can be measured.
If they have improved over time, that, I submit, is progress.

Let's go to the data,
beginning with the most precious thing of all, life.

4:43

\begin{itemize}
\item For most of human history, life expectancy at birth was around 30.
Today, worldwide, it is more than 70,
and in the developed parts of the world,
more than 80.
\item 250 years ago, in the richest countries of the world,
a third of the children did not live to see their fifth birthday,
before the risk was brought down a hundredfold.
5:05
Today, that fate befalls less than six percent of children
in the poorest countries of the world.
\item Famine is one of the Four Horsemen of the Apocalypse.
It could bring devastation to any part of the world.
Today, famine has been banished
to the most remote and war-ravaged regions.
200 years ago, 90 percent of the world's population
subsisted in extreme poverty.
Today, fewer than 10 percent of people do.
\end{itemize}

5:32
For most of human history,
the powerful states and empires
were pretty much always at war with each other,
and peace was a mere interlude between wars.
Today, they are never at war with each other.
The last great power war
pitted the United States against China 65 years ago.
More recently, wars of all kinds have become fewer and less deadly.
The annual rate of war has fallen from about 22 per hundred thousand per year
in the early '50s to 1.2 today.
Democracy has suffered obvious setbacks
in Venezuela, in Russia, in Turkey
and is threatened by the rise of authoritarian populism
in Eastern Europe and the United States.
Yet the world has never been more democratic
than it has been in the past decade,
with two-thirds of the world's people living in democracies.

6:24
Homicide rates plunge whenever anarchy and the code of vendetta
are replaced by the rule of law.
It happened when feudal Europe was brought under the control of centralized kingdoms,
so that today a Western European
has 1/35th the chance of being murdered
compared to his medieval ancestors.
It happened again in colonial New England,
in the American Wild West when the sheriffs moved to town,
and in Mexico.
Indeed, we've become safer in just about every way.

6:54
Over the last century, we've become
\begin{itemize}
\item 96 percent less likely
to be killed in a car crash,
\item 88 percent less likely to be mowed down on the sidewalk,
\item 99 percent less likely to die in a plane crash,
\item 95 percent less likely to be killed on the job,
\item 89 percent less likely to be killed by an act of God,
such as a drought, flood, wildfire, storm, volcano,
landslide, earthquake or meteor strike,
presumably not because God has become less angry with us
but because of improvements in the resilience of our infrastructure.
\end{itemize}

7:30
And what about the quintessential act of God,
the projectile hurled by Zeus himself?
Yes, we are 97 percent less likely to be killed by a bolt of lightning.

7:43
Before the 17th century,
no more than 15 percent of Europeans could read or write.
Europe and the United States achieved universal literacy
by the middle of the 20th century,
and the rest of the world is catching up.

7:56
Today, more than 90 percent of the world's population
under the age of 25 can read and write.
In the 19th century, Westerners worked more than 60 hours per week.
Today, they work fewer than 40.

Thanks to the universal penetration of running water and electricity
in the developed world
and the widespread adoption of washing machines, vacuum cleaners,
refrigerators, dishwashers, stoves and microwaves,
the amount of our lives that we forfeit to housework
has fallen from 60 hours a week
to fewer than 15 hours a week.

8:32
Do all of these gains in health, wealth, safety, knowledge and leisure
make us any happier?
The answer is yes.
In 86 percent of the world's countries,
happiness has increased in recent decades.

Well, I hope to have convinced you
that progress is not a matter of faith or optimism,
but is a fact of human history,
indeed the greatest fact in human history.
And how has this fact been covered in the news?

9:03
A tabulation of positive and negative emotion words in news stories
has shown that during the decades in which humanity has gotten healthier,
wealthier, wiser, safer and happier,
the "New York Times" has become increasingly morose
and the world's broadcasts too have gotten steadily glummer.
Why don't people appreciate progress?
Part of the answer comes from our cognitive psychology.
We estimate risk using a mental shortcut called the "availability heuristic."
The easier it is to recall something from memory,
the more probable we judge it to be.
The other part of the answer comes from the nature of journalism,
captured in this satirical headline from "The Onion,"
"CNN Holds Morning Meeting to Decide
What Viewers Should Panic About For Rest of Day."

9:57
News is about stuff that happens, not stuff that doesn't happen.
You never see a journalist who says,
"I'm reporting live from a country that has been at peace for 40 years,"
or a city that has not been attacked by terrorists.

Also, bad things can happen quickly,
but good things aren't built in a day.
The papers could have run the headline,
"137,000 people escaped from extreme poverty yesterday"
every day for the last 25 years.
That's one and a quarter billion people leaving poverty behind,
but you never read about it.

10:31
Also, the news capitalizes on our morbid interest
in what can go wrong,
captured in the programming policy, "If it bleeds, it leads."
Well, if you combine our cognitive biases with the nature of news,
you can see why the world has been coming to an end
for a very long time indeed.

10:50
Let me address some questions about progress
that no doubt have occurred to many of you.
First, isn't it good to be pessimistic
to safeguard against complacency,
to rake the muck, to speak truth to power?

Well, not exactly.
It's good to be accurate.
Of course we should be aware of suffering and danger
wherever they occur,
but we should also be aware of how they can be reduced,
because there are dangers to indiscriminate pessimism.

11:18
One of them is fatalism.
If all our efforts at improving the world
have been in vain,
why throw good money after bad?
The poor will always be with you.
And since the world will end soon --
if climate change doesn't kill us all,
then runaway artificial intelligence will --
a natural response is to enjoy life while we can,
eat, drink and be merry, for tomorrow we die.

11:42
The other danger of thoughtless pessimism is radicalism.
If our institutions are all failing and beyond hope for reform,
a natural response is to seek to smash the machine,
drain the swamp,
burn the empire to the ground,
on the hope that whatever rises out of the ashes
is bound to be better than what we have now.

12:03
Well, if there is such a thing as progress,
what causes it?

12:07
Progress is not some mystical force or dialectic lifting us ever higher.
It's not a mysterious arc of history bending toward justice.
It's the result of human efforts governed by an idea,
an idea that we associate with the 18th century Enlightenment,
namely that if we apply reason and science
that enhance human well-being,
we can gradually succeed.

12:32
Is progress inevitable? Of course not.
Progress does not mean that everything becomes better
for everyone everywhere all the time.
That would be a miracle, and progress is not a miracle
but problem-solving.
Problems are inevitable
and solutions create new problems which have to be solved in their turn.

12:53
The unsolved problems facing the world today are gargantuan,
including the risks of climate change
and nuclear war,
but we must see them as problems to be solved,
not apocalypses in waiting,
and aggressively pursue solutions
like Deep Decarbonization for climate change
and Global Zero for nuclear war.

13:15
Finally, does the Enlightenment go against human nature?
This is an acute question for me,
because I'm a prominent advocate of the existence of human nature,
with all its shortcomings and perversities.

13:28
In my book "The Blank Slate,"
I argued that the human prospect is more tragic than utopian
and that we are not stardust, we are not golden
and there's no way we are getting back to the garden.

13:42
But my worldview has lightened up
in the 15 years since "The Blank Slate" was published.
My acquaintance with the statistics of human progress,
starting with violence
but now encompassing every other aspect of our well-being,
has fortified my belief
that in understanding our tribulations and woes,
human nature is the problem,
but human nature, channeled by Enlightenment norms and institutions,
is also the solution.

14:09
Admittedly, it's not easy to replicate my own data-driven epiphany
with humanity at large.

Some intellectuals have responded
with fury to my book "Enlightenment Now,"
saying first how dare he claim that intellectuals hate progress,
and second, how dare he claim that there has been progress.

14:32
With others, the idea of progress just leaves them cold.
Saving the lives of billions,
eradicating disease, feeding the hungry,
teaching kids to read?

Boring.

14:44
At the same time, the most common response I have received from readers is gratitude,
gratitude for changing their view of the world
from a numb and helpless fatalism
to something more constructive,
even heroic.

14:57
I believe that the ideals of the Enlightenment
can be cast a stirring narrative,
and I hope that people with greater artistic flare
and rhetorical power than I
can tell it better and spread it further.

15:09
It goes something like this.
We are born into a pitiless universe,
facing steep odds against life-enabling order
and in constant jeopardy of falling apart.
We were shaped by a process that is ruthlessly competitive.
We are made from crooked timber,
vulnerable to illusions, self-centeredness
and at times astounding stupidity.
Yet human nature has also been blessed with resources
that open a space for a kind of redemption.
We are endowed with the power to combine ideas recursively,
to have thoughts about our thoughts.
We have an instinct for language,
allowing us to share the fruits of our ingenuity and experience.
We are deepened with the capacity for sympathy,
for pity, imagination, compassion, commiseration.
These endowments have found ways to magnify their own power.
The scope of language has been augmented
by the written, printed and electronic word.
Our circle of sympathy has been expanded
by history, journalism and the narrative arts.
And our puny rational faculties have been multiplied
by the norms and institutions of reason,
intellectual curiosity, open debate,
skepticism of authority and dogma
and the burden of proof to verify ideas
by confronting them against reality.

As the spiral of recursive improvement
gathers momentum,
we eke out victories against the forces that grind us down,
not least the darker parts of our own nature.
We penetrate the mysteries of the cosmos, including life and mind.
We live longer, suffer less, learn more,
get smarter and enjoy more small pleasures
and rich experiences.

16:57
Fewer of us are killed, assaulted, enslaved, exploited
or oppressed by the others.
From a few oases, the territories with peace and prosperity are growing
and could someday encompass the globe.
Much suffering remains
and tremendous peril,
but ideas on how to reduce them have been voiced,
and an infinite number of others are yet to be conceived.
We will never have a perfect world,
and it would be dangerous to seek one.
But there's no limit to the betterments we can attain
if we continue to apply knowledge to enhance human flourishing.

17:34
This heroic story is not just another myth.
Myths are fictions, but this one is true,
true to the best of our knowledge, which is the only truth we can have.
As we learn more,
we can show which parts of the story continue to be true and which ones false,
as any of them might be and any could become.
And this story belongs not to any tribe
but to all of humanity,
to any sentient creature with the power of reason
and the urge to persist in its being,
for it requires only the convictions
that life is better than death,
health is better than sickness,
abundance is better than want,
freedom is better than coercion,
happiness is better than suffering
and knowledge is better than ignorance and superstition.

18:22
Thank you.

}%ignore

\section{A Synthesis}

In this section I present a conception of what enlightenment might be.

I begin with a purpose, a sense of what enlightenment might be expected to realise.

\subsection{Purpose}

Enlightenment, to a first approximation, is independent of purpose, in the same way that instrumental rationality consists in an appropriate approach to whatever objective we puesue, one likely to the best of our knowledge to reaslise that objective.
This is indeed what we find in some of the narratives written during ``The Enlightenemnt'', such as Kant's short essay \cite{kant-en} and Paine on ``The Age of Reason'' \cite{paine-reason}. 

\cite{pinker-angels}

\chapter{On Critical Theory}

An attempt to put together an intelligible presentation of my evolving understanding of the intellectual maelstrom which is Planet Earth as it is transformed by near instant, near universally accessible, global communications.

\section{Introduction}

The conceit (we may call it) of enlightenment thought (as presented by some intellectuals) that truth could be judged by individuals for themselves rather than being dispensed by authority, seems to have been an historical anomaly, and the liberal democracies which followed, embracing both those enlightenment ideas and values and much of the `romanticism' which became more prominent in the nineteenth century, are now at risk of succumbing beneath the power of ideological authoritarianism.


``Critical Theory'' was the term adopted by Horkheimer \cite{horkheimer-trad, horkheimer-crit} for the work of the Frankfurt school after he took over the directorship of the school in 1930.
When capitalised the term usually refers to the work of the Frankfurt school, but the term is also used for a broader range of more recent theories eventually coming together under the umbrella term ``Critical Social Justice''.

The resulting ``theories'' are shaped in part by a thoroughgoing rejection of the idea of rationality as it was understood in the enlightenment, and are therefore, by design, radically incoherent and rationally irredeemable.
The departure from what Horkheim called ``traditional'' theory and the rational standards which it advocated did not take place in an abrupt transition, but in a number of innovations spanning two Centuries, prominent among which are contributions from Kant, Hegel, Marx, the Frankfurt School, Postmodern philosophy and a variety of subsequent critical theorists.

I shall take David Hume as a solid representative of ``traditional theory'' in the Enlightenment ($18^{th}$C), and will consider the subsequent divergence step by step from that traditional perspective.

To begin here, I simply enumerate the steps which I will be considering:

\begin{enumerate}
\item Humean positivism
\item Kant's Critique
\item Hegelian Dialectical Logic
\item Marx's Dialectical Materialism
\item Horkheimer's Critical Theory
  \begin{enumerate}
  \item The conflation of philosophy science and activism
  \item The conflation of disadvantage with enslavement
  \item Totalitarian democracy
  \end{enumerate}
  \item False consciousness
\item Marcuse's Repressive Tolerance
\item Postmodern Selective Radical Scepticism
  \begin{enumerate}
  \item Meaning
  \item Truth
  \item Power
    \end{enumerate}
\item Applied critical theory
\end{enumerate}


\section{Hume}

Hume has been posthumously nominated the first positivist, that tendency in philosophy named by August Comte, which was particularly mentioned by Max Horkheimer when speaking of the ``traditional theory'' against which Critical Theory was to be contrasted \cite{horkheimer-trad}.
Hume became famous for his scepticism, a term provoked by his sharp understanding of the limits of deductive reason and his belief that even empirical facts cannot by themselves warrant moral conclusions.
The two distinctions, between verbal and factual, and between factual and moral, both sometimes referred to as ``Hume's fork'', are the first clear delineation of categories which had been important but elusive throughout the prior history of Western philosophy.
No sooner were they enunciated than they began to be unravelled.

Two points to note here.
First, the making of the distinctions does not entail that academics should confine themselves to one realm, even though there are important differences in methods across the realms which may corral enquiry.
Even if originality may be confined to one realm, there is applicability of some realms to others.
Thus, physicists need mathematics, use mathematics, but need not be innovators in mathematics and may not have the kind of competence required for that.
Engineers need physics (and its mathematics) ...
More on this when we come to Critical Theory.

\section{Rousseau}

Rousseau gets a mention here in relation to two subthemes.
While Hume is perhaps atypical as an enlightenment philosopher because his scepticism led him to doubt that reason could conclusively establish either empirical science or value judgements (or belief in God, hence underpinning counter enlightenment religious beliefs founded in faith rather than reason), Rousseau is atypical in completely different ways, and may be perhaps the enlightenment philosopher most easily seen as preparing the way for post-enlightenment romanticism.\cite{berlinRR}


He may therefore be seen as placed at the head of the division sometimes perceived between \emph{continental philosophy} and the more anglo-saxon analytic tradition.
Rousseau is the philosopher most closely associated with the French revolution, a champion of freedom and democracy, who nevertheless corrupted language with the doctrine that freedom consisted in acting according to `the general will'.
This equivocation was later elaborated for the totalitarian regimes of the $20^{th}$ Century, before being denounced by Orwell \cite{orwell-1984} only to be taken to yet greater extremes in the elaboration of Critical Theory in the $21^{th}$.


\section{Kant}

It is Kant who first raises objections to Hume's simple scheme.
To do this he introduces new terminology, usually translated into English as \emph{analytic} and \emph{synthetic}.
The words were not new, but the Kants usage was.
There was for example, in ancient Greece a distinction between analytic and synthetic proof, the former proceeding by analysis of the proposition to be proven, the latter by synthesis of that proposition from others.
More recently, Leibniz had used the concept of analytic, but his conception of the analytic was impacted by his theology, which attributed infinite analytic power to God and which extended the analytic beyond anything which a human might be able to establish.

After Kant the concept of analytic became a \emph{semantic} notion, but in Kant there is a dual characterisation with one of those seemingly more proof theoretic.
The characterisation which we are inclined to count semantic was that an analytic proposition is one in which the subject \emph{is contained in} the object, taking the containment to be containment of concept rather than of extension.
The `proof theoretic' characterisation was that a proposition is analytic if it is derivable from the law of contradiction.
In either case, synthetic propositions are those true propositions which are not analytic.

Using this mew terminology, Kant was able to differ from Hume by asserting that certain kinds of knowledge consist of synthetic \emph{a priori} propositions.
That is, rephrasing in more Humean language, that the scope of \emph{a priori} knowledge extends beyond Hume's relations between ideas.
This result is obtained in part by taking a more restrictive idea of the scope of `relations between ideas', notably by considering mathematics to be synthetic (thought still \emph{a priori}, and in part by taking a more liberal idea of what is \emph{a priori}, in this case by admitting the categorical imperative.

In this story, which begins by observing how the tight constraints upon what can be known with certainty according to Hume are progressively eroded to make way for the dogmatic totalitarian ideologies of the $20^{th}$ Century and the woke revolution of the $21^{st}$, Kant makes the first 

\section{Hegel}

My knowledge of Hegel is superficial.
I mention him here to point out some salient aspects of his dialectical logic.

\section{Horkheimer}



\chapter{The Tyranny of the Tribe and the Roots of Reason}

An attempt to understand rationality and those forces which undermine it in the form of a historical narrative.

This is a short essay with broad scope, and hence a collage of oversimplifications.
One of those is committed by my title, which suggests a normative contrast as clear as that between good and evil.
For this I credit the political climate as I write, in which the dangers of ``tribalism'' seem to me more proximate than those of rationalism.

In many aspects of the matters here discussed, mutually contradictory tendencies push evolution to congenial and necessary balance, which facilitates a progression, if not always progress.
By tracing these we may come to a better understanding of where we are and how to find our way home.

I have late in life suddenly come to appreciate that the values which I absorbed as a child and a young man are being challenged by alternatives which seem to have no merit, and whose main purpose seems rather to destroy than to improve.
It is startling that this agenda has pressed forward in plain sight, but unseen by most.
In another climate the two alternatives here juxtaposed might be given a more balanced presentation, but now, despite considerable reservations on the matter which will become apparent, I am intent on a defence of reason against ideological assault.

\section{Introduction}

No other species on Earth is capable of rational intelligence in the manner of \emph{homo sapiens}, and none other so well equipped, in apparent abandonment of that gift, to undertake mass self-mutilation while blinded by ideology.

In this essay I sketch the results of my efforts to understand these seemingly contradictory phenomena, my attempt to comprehend certain aspects of human nature and the social realities influenced by them, and to probe the possibilities for progress, because of and despite these characteristics.

To understand \emph{where} we are, we must understand \emph{who} we are, the nature of humanity.
To understand who we are, it helps to understand how we came to be.
To peer into the future, we must understand where we are, the process which created us, and the ways in which that process is itself evolving.
This is my striving to see.

The story which follows is built around evolution, the many kinds of evolution, the ways in which evolution has and will continue to evolve, and the way it has shaped humanity and society.
It is a historical narrative which (somewhat arbitrarily) falls into four main parts:

\begin{enumerate}
\item The evolution of homo sapiens
\item The evolution of culture to `\emph{the Age of Reason?}' 
\item Counter-enlightenments and revolutions
\item Evolution by design -- synthetic biology and artificial intelligence
\end{enumerate}

Through these stages the pace of evolution has accelerated.
The first stage took four billion years, the second two hundred thousand, the third two hundred years.
The fourth stretches into the future.
The acceleration has been enabled by changes in the nature of evolution enabled by the advances previously made.

Only the later part of the first stage consists of that kind of biological evolution identified by Darwin as ``the evolution of species by natural selection''.
Thereafter the most significant change is cultural, outpacing continued genetic evolution.
In the last phase, changes to the biosphere and perhaps even to the human genome will increasingly be {\it by design}.

\section{Themes}

Through this chronological narrative a number of themes and subthemes are threaded.

\ignore{
  The two principle themes are rationality and democracy in the account of which a number of other themes and subthemes are helpful.

\subsection{Rationality}

There is one principal theme, which is the evolution, biological and cultural, of rationality and of those human or social tendencies which seem most seriously to diverge from it (or are perhaps a necessary complement to it).
For the purposes of this essay I take an instrumental notion of rationality to be fundamental, and all others (of which the espistemic is the most important) as derivative.

\subsection{Democracy}

It is common for a species to exhibit distinct behaviours, perhaps even distinct physical structure, according to environmental conditions.
The availability or scarcity of food may trigger such variations, procreation being favoured in conditions of pleanty, and foraging favoured when supplies are short.
In humans the treatment of offspring in their infancy may determine adult character in a profound way, reduced physical contact between parent and child (perhaps caused by more urgent demands on parental time) leading to more aggressive adult character, suitable for seeking (and possibly fighting for) resources.

Different forms of social organisation and behavioural norms may similarly depend upon context.
In harsh times, competition for scarce resource may be physical.
In prosperous conditions it may be possible to resolve differences without violence.
This may be seen as the principle aim of democratic governance, something which seems only to have flourished in modern relative prosperity, and remains a fragile flower.

}%ignore

\subsection{Evolution}

I counterpose two evolved phenomena.
On the one hand our capability for \emph{rationality}, eventually manifesting in the advanced technologies which have made the lives of most people healthy, prosperous and fertile.
On the other the strong social tendencies which may cause rational beings to act \emph{en masse} in apparently irrational ways.
Both of these are enabled by physiological features, coded in our genomes, but are manifest in ways which are shaped by the cultural evolution which those genetically determined traits enabled.


\subsection{The Theory of Evolution}

Here a brief sketch of the development of the theory of evolution.
Later the history of evolution on earth will be considered particularly in relation to my two headline concerns, rationality and its nemesis.

Ideas about evolution date back to antiquity, but the modern scientific account of evolution begins properly with Darwin's theory of the evolution of species by natural selection.
This was developed in the $19^{th}$ Century concurrently with Mendel's work on genetics, which however was not widely disseminated until the $20^{th}$ Century, so Darwin did not have a good understanding of inheritance when he undertook his research and wrote his magnum opus. \cite{darwin-oos}.

Darwinian evolution therefore pre-dated a scientific understanding of inheritance, which latter has continued to advance ever since.
So we have in Darwin a conception of evolution which is broader than most modern ideas of evolution, but which nevertheless fails to encompass all the kinds of evolution which we touch upon here.

Essential to Darwin's conception are that it concerns a biosphere in which:
\begin{itemize}
  \item there are species, each of which is a collection of self-reproducing organisms
  \item organisms reproduce, creating new organisms similar to the original, but subject to some variation
  \item different individuals of the same species will not all be equally successful in reproducing themselves, these differences being attributed to ``natural selection'' (by analogy and contrast with that kind of selection exercised by those who breed domesticated species, selecting as parents individuals exhibiting desired characteristics.)
  \item the differential reproduction results in continuous shifting in the nature of the species which is guided by that natural selection.
\end{itemize}

Though this account of the theory of evolution interprets ``natural selection'' through the reproductive success of individual organisms, Darwin also recognised that altruistic traits emerge which appear to be beneficial to groups even though they do not manifest in reproductive succes for the individual.
He has been interpreted as adhering to a two-level theory in which:

\begin{quote}
`Selfishness beats altruism within groups.
Altruistic groups beat selfish groups.
Everything else is commentary.
\end{quote}
\cite{wilson2007rethinking,wilson2015does}

Though here, ``altruistic group'' means ``a group of altruistic individuals'', but is itself nothing of the sort (in relation to other groups).
The altruism is intragroup not intergroup.

During the first half of the $20^{th}$ Century the further development of evolutionary theory integrated Darwin's theory with an understanding of the genetics coming from the work of Mendel,
The synthesis with Mendelian genetics resulted in `the Modern Synthesis', a term coined by Julian Huxley \cite{huxley-tms}.
There were three further articulations of such a synthesis, Mayr (1959), Stebbins (1966), Dobzhansky (1974), but at the same time as a concensus might thus have seemed possible, divergence was becoming apparent concerning the evolution of altruism and the possibility of group selection.

In that matter the gene-centric presentations by Williams(1966) \cite{williams-ans} popularised by Dawkins \cite{dawkinsSG} were contrasted by work on sociobiology or evolutionary psychology such as E.~O.~Wilson's \emph{Sociobiology: The New Synthesis} \cite{wilson-stns}.
The impetus to take a broader view of evolutionary theory continues with the idea of an \emph{extended Evolutionary Synthesis} which embraces 

The chemistry underlying genetics was clarified by the discovery of the structure of DNA by Crick and Watson in 1953, and the resulting evolutionary theory was popularised by Dawkins accompanied by an interpretation of its consequences for the possibility of genuine altruisn \cite{dawkinsSG} and a strong repudiation of the possibility of selection other than at the level of the gene.

\subsubsection{The Gene-centric Account of Evolution}

Though evolution is varied in character, there is a single abstract model of evolution which is a reasonably good fit with a large part of biological evolution and by contrast with which other kinds of evolution can be described.
This model was given a popular exposition in \emph{The Selfish Gene}\cite{dawkinsSG} by Richard Dawkins.
Its bare essentials are as follows.

Let us call it ``genetic evolution of species by natural selection''.
We take a species to be a population of organisms capable of interbreeding, in some more or less habitable environment.
Each organism has a \emph{genome}, which is a set of genes.
The complete set of genes occurring in the population of the species, together with the number of occurrences of each gene is the \emph{gene pool}.
As time progresses some individuals will die, and they will then be removed from the population and their genes will be deducted from the gene pool.
Others will produce offspring which are added to the population with genomes derived from those of their parents.
This process of reproduction yields individuals whose genomes are primarily drawn from the genes of the parents, but may also include some genes which are not present in the parental genomes.
In sexually reproducing species, the main source of genetic variation in the reproductive process is recombination due to crossover (a process in which a chromosome is obtained by selecting some material from the corresponding male and some from the female parental chromosomes), but some new genes arising from crossovers which occur within a cistron (the sequence coding a single protein), and rarer errors of transcription. 


In this way the gene pool of the species evolves over time.
Both the lifespan and the reproductive success of the individuals of the species are influenced by their genome, and these factors result in differential propagation of genes in the gene pool.
This differential propagation results in progressive changes the gene pool. and hence of the phenotypes of the individuals of the species.

In \emph{The Selfish Gene} Dawkins is particularly concerned to refute the possibility that `true' \emph{altruism} could possibly evolve by such means.
The mechanism alleged to make altruism possible, even in 1967 held to have been rejected by the profession, is \emph{group selection}.
We are here concerned with the evolution and character not only of rationality, but also of social behaviour, and it is to be expected that there may be some tension with Dawkins' uncompromising stance aainst the possibility of altruism.
There remain to this day academics studying evolution who find group selection to be a valuable tool, and take it to be justified by the results it yields, oblivious perhaps to the inadequacy of explanations based on mechanisms which cannot themselves be explained.
I will say here that I accept the model of genetic evolution on which Dawkins (and many others) drew his conclusions, and differ from him, not in positing some other mechanism yeilding selection on the basis of advantage to a social group, but rather in the analysis of what limits flow from it.

\subsubsection{Stages in Evolution}

The evolutionary history we present here falls into five periods all having distinct characteristics from an evolutionary point of view.

\paragraph{pre-biotic}

The standard model is a model of genetic evolution, and the stage in evolution which is termed pre-biotic is that which precedes the existence of life.

\paragraph{prokaryotic}

In the second and third the Modern Synthesis is applicable, with some some supplementary considerations in the third.
The first stage is prebiotic, before the genetic model of inheritance is established.
The fourth stage is dominated by oral and then written culture which fails to comply with the genetic model of inheritence, notwithstanding the tenuous analogy offered by the concept of ``meme''
coined by Dawkins.
If culture took the reins in the fourth stage, partly by guiding the underlying genetic mechanisms no longer arguably directed exclusively by `natural', in the speculative fourth phase the underlying mechanism is derailed by synthetic biology, the ecosphere finally falling within the sphere of intelligent design.

\subsection{Sociality}

There is strength in numbers, one hears, and in consequence life on earth has social groups running through it like lettering in a stick of rock, from beginning to end.
If we admit as ``social'' whatever ways living beings interact in groups with their peers, then we may say that sociality has itself evolved throughout the history of life on earth.
The manner of this social interaction is at first genetically determined, but as life becomes more adaptable through the evolution of increasingly capable central nervous systems, the character of social interaction is progressively transformed, eventually subject to determination largely by culture rather than genes.

The phenomena we seek to understand, contrast and reconcile, are social phenomena a product both of genes and culture, nature and nurture.
Tracing the evolution of social behaviours from their beginnings may help us to understand them better.

\subsection{Communications and Language}

Though it is common among linguists to regard as \emph{language} only those means of communication found in \emph{homo sapiens}, communication \emph{of some kind} is an essential condition of sociality of any kind, counting here among communications even that mere bodily contact with others which may take precedence in some bacteria over life's essentials such as food.
Not only is some kind of communication essential to social life, the kinds of communication available may have a profound effect on social behaviour, and the evolution of information storage and communication technology is coupled with major transformations in social behaviour and all the depends upon it.

An important theme is therefore the development of communication from the most elementary beginnings through to the present day and beyond.
This is particularly important because of its pivotal role in the possibility of oral and then written culture, and because of its fundamental connection with the nature of rationality.

Rather than taking a stance on the question \emph{what is a language?}, I note here a number of kinds of communication which represent an ascending scale of refinement and utility.

\begin{itemize}
\item Observation

  If we consider communication to have taken place when information is transferred from one part to another, then this is possible without any intent to commumnicate on the part of the party from whom information is transferred.
  Thus one organism may observe the behaviour or state of another, and the observation may influence his subsequent behaviour.
  This is an aspect of the most elementary kinds of social behaviour.

  Without attempting an exhaustive classification, the following two obvious cases may be noted.
  First, by observing the behaviour of his peers, an organism may learn where to find resources such as food or water, or how to accomplish some task by a method previously unknown to him.
  Neither of these need involve and intent on the part of the other part to communicate.
  Second, observation may play a role in mimicry, in which one organism observes a certain behaviour, and subsequently mimics that behaviour.
  For example, having seen one organism recoil with fear, an observing organism may subsequently do the same when encountering that same potential threat.

  In the behaviour of flocks or schools each individual must observe the course of those closest to him and adjust his course in order to avoid collisions.
  This is a vital channel of `communication' essential to those collaborative behaviours, but it is not necessary to suppose that there is any intent on those proximate creatures to signal or communicate.

\item Signalling

  Once there is intent to communicate, then an organism may behave in a way which (in part or whole) is intended to communicate to another party, or which we may say, has evolved to fulfill that purpose.
  A suitable criterion here is that what is done is different in some way in order to facilitate the communication than it would otherwise have been.

  Examples of this are the signs of distress which animals may show when some threat is perceived.
  Birds may take to flight on noticing a threat, and this in itself would not constitute a signal, but it is likely also to cry out, which signals to other birds that a threat has been perceived, and has no other apparent role.

  Showing may also fit in this category.
  While being observed exhibiting some skill may not be a deliberate communication, deliberately showing how may involve doing it in a slightly different way to make it clear to the observer exactly what is being done.
  Under this classification this would count as signalling.
  Much behaviour during mating would also count as signalling, typically making conspicuous some attribute which might be persuasive to a potential mate.

  Many linguists would regard these kinds of signalling as not making use of a language, but some philosophers, perhaps Wittgenstein with his ideas about languages as games, might be more accomodating.
  
\item Symbolism and reference

  Many will consider a key feature of language to be symbolism, and the most primitive purpose of symbols to be reference.
  Certainly, it is essential to the notion of symbol that the symbol stands for something else, which is typically known as its referent.
  It may not be unambiguous when this is occurring, could we consider the cry of a distressed bird as symbolising and referring to the distress of the bird?
  I should be inclined myself to look for more deliberate use of symbols.
  Locating the demarcation here will not be important for us.

\item Propositional language

\item Recursive language

\item Written language
  
 
\end{itemize}

\subsection{Epistemology and Rationality}

A huge part of the distinction which concerns us here, between `tribalism' and `rationality' is epistemological, which is to say, that it is concerned with the practice and theory of knowledge.

The evolution of \emph{how we know} as well as the evolution of epistemology (the theory of how we know), is therefore a principle concern and is overlaid upon the consideration of sociality, communications and language.

It is natural perhaps for philosophers to consider as knowledge those things that we can put into or read out of textbooks, but it is also valuable to consider the entire process of evolution of homo sapiens from even before the existence of life through the present and into the future as being an accumulation of knowledge by means which have continually progressed and will continue to evolve.

This is also a perspective from which we can make sense of the notion of rationality as having that same broad scope and associated with the pragmatic nature of evolution in yielding those kinds of organism which are well adapted to self replication in some niche of this moment in the evolution of the biosphere.

Here are some headings under which these matters will be considered.

Knowledge can be considered to be of the following principle kinds:

\begin{itemize}
\item propositional knowledge
  \end{itemize}


\subsection{Subthemes}

Cutting across those themes are some recurring subthemes:
\footnote{These are derived in part from the ideas of Howard Bloom published in \cite{bloom-bs,bloomBRAIN}}

  \begin{itemize}
\item conformity and diversity
\item status and place
\item resource shifters
\item inter group tournaments
  \end{itemize}

  \section{Rationality}
  
The instrumental notion I take as follows:
\begin{quote}
  A course of action is \emph{instrumentally rational} for some purpose if there is good reason to believe that it will realise that purpose.
\end{quote}

It transpires that true knowledge is helpful, but that belief in falsehoods is counterproductive.
So having good ways to distinguish truth from falsity is important, and the use of such criteria yields \emph{epistemic} rationality:

\begin{quote}
Belief in a proposition is \emph{epistemically rational} if there is good reason to believe it true.
\end{quote}

These two definitions don't get us far, there is more to be said.
They simplistically reduce rationality to the having of good reasons for actions or beliefs.

I intend that the concepts should be taken as \emph{normative}, that to say that something is rational is to approve, under the circumstances and in the light of available evidence, of the belief or the course of action.

The standard will depend on context.
In dire circumstance, an action may be considered rational which might otherwise have been thought reckless.
As science and technology progresses it may become possible and rational to exact higher standards of experimental rigour, supported by more advanced and precise equipment and perhaps a fuller understanding of related phenomena.
Thus \emph{rationality} as it is here discussed will be a moveable feast, and much of what is to be said concerns how it has developed in the past and how it might or should progress in the future.

The contrast which concerns us, between belief or behaviour which is rational, and that which may be maligned as ``tribal'', is not a subtle distinction.
In the latter case, there may be a complete disregard of all the kinds of evidence crucial to a rational determination in favour of a rigid or an incoherent idelogically determined prejudice.

The sensitivity of instrumental rationality to \emph{purpose}, may nevertheless render intelligible cases of apparent irrationality.
We may have to acknowledge that the purpose in hand is not what we supposed or hoped it to be, the context creating motives which we had not suspected.
The potential conflict between what might be considered social purpose and the purposes or interest of individuals is germane to our theme, and invites extension of the concept of rationality to social groups, particularly to organisations which have some intended purpose and to the question whether they are organised in the best way to realise that purpose.

\section{The Evolution of Homo Sapiens}

In considering the nature and future of man by investigation of his origins, several kinds of history are relevant.

There is first, the story of how \emph{life on earth} has evolved over the last 4.5 Billion years.
In considering this whole as a process of evolution we are of necessity taking a very broad view of evolution.

Tangible insights from an evolutionary perspective may depend upon recognising that \emph{the process of evolution} has itself evolved over that period, and on seeking an understanding of those stages sufficient to illuminate the aspects of the developmemt of life which are of particular interest.

In telling those two stories it may be helpful to consider the evolution of \emph{our understanding of evolution}, a rather more recent affair which (notwithstanding earlier premonitions) is generally taken to start with Charles Darwin\cite{darwin-oos}.

  \subsubsection{Pre-biotic Evolution}

  A first crucial but mysterious stage of evolution takes place after the formation of the earth but before the appearance of living organisms.
  Our understanding of this is necessarily limited and speculative.
  Evidence of what happened has been wiped clear by the life which subsequently emerged.

  In terms of understanding the nature of evolution and how it is changing at this time, the main problem is that no clear process has so far evolved.
  There is no life, there is no way of coding descriptions of life to hand down through the generations, no genome to be distinguised from its phylotipic expression and subject to variation and selection.

  Nevertheless there is a progression, the manner of which itself evolves over time, and changes to which affect the rate of progression and by favouring certain kinds of progression over others, influence the direction of progress.
  The progression is in the first instance likely to be a gradual change over time in the composition of the watery parts of the planet.
  A drift rather than anything more structured, leading little by little to greater concentrations of slightly more complex molecules.

  Eventually some of the complex molecules formed will be catalysts, providing staging posts for the construction of complex molecules which would otherwise be much likely to form.
  This is one way in which both the direction and pace of evolution may shift.
  Beyond simple catalysis, \emph{auto-catalytic sets} are thought likely to have contributed.
  An autocatalytic set, a group of chemicals which catalyse both the construction of certain molecules from their constituents and also the catalysts themselves, is like a self-reproducing factory.
  It not only builds arbitrarily complex molecules from simpler constituents faster than they might otherwise be created, but is permits continuous exponential growth of the machinery devoted to that purpose.

  Nevertheless, the complexity of the building blocks of life is so great that this kind of free-for-all, abbetted though it might be by catalysts and autocatalytic sets, is barely a beginning the the evolution of the machinery of evolution.
  When we look at the simplest living organisms to have survived to the present day, we find that they share, along with the rest of life on earth, the use of a particular open ended collection of building blocks for life facilitated by \emph{universal} manufacturing facility.
  The building blocks are the proteins, lengthy chains of amino acids.
  The universal machinery, by analogy with a universal turing machine, given the specification of a protean, i.e. an account of what amino acids must be strung together to form the protean, together with an sufficient supply of the amino acids, will follow the recipe carefully stringing together the hundreds of amino acid molecules which will likely be required to make the required molecule of protein.

  \subsubsection {Molecular Machinery}

    In explaining scepticism about evolution, the astronomical improbabilty that we could be here \emph{by accident} is often invoked.
    The theory of evolution has rather more to it, but it does explain why the odds on human beings happening without divine intervention are better than one might imagine.
    Untutored imaginations are poor guides as to what is or is not probable.
    
    What we will see in the following descriptions of how evolution has taken place, is a series of special considerations showing how certain constructions and developments are more probable than we might have expected.

    We begin in the `primeval soup' shortly after the formation of the earth some 4 billion years ago.
    The ocean was then presumed to contain simple molecules, and the stock model of genetic evolution is entirely inapplicable.
    There are no genes, and there are no self-replicating structures, no life.
    
    The very weak sense in which evolution is taking place at this stage is that, little by little, stimulated by energy sources such as lightening and volcanic action, these simple atoms and molecules are combining to form more complex molecules.
    Initially these more complex molecules will be scarce and the ingredients for forming them plentiful.
    The rate of formation stimulated by the available energy sources will exceed the rate at which these more complex molecules break up, and the numbers will rise gradually seeking a point of equilibrium at which losses match formations.
    Thus the consistency of the soup `evolves', equibilibrium proving illusive as the changing mix enables more complex constructions.

    This process is not entirely random, the probability of formation of a compound is not exclusively determined by the availability of components.
    A construction involving three components will be much less likely to happen if all three must be present at once than ifthe construction can take place in stages, if it can be broken down into two constructions each having only two constituents.
    Very complex constructions would be impossible if they could not be conducted stage by stage.

    Some constructions involving more than two constituents can be reduced to a series of simpler (and more probable) constructions with the help of an auxilary molecule which gathers together the constituents one at a time in a stable structure pending the moment at which a complete inventory permits the whole to be assembled and separated from the auxiliary molecule.
    Such molecules are called catalysts.

    A crucial feature of life is self reproduction.
    Reproduction cannot take place in a vacuum, so typically a self-reproducing organism belongs and can survive and reproduce only in the ecological niche in which it evolved and to which it is adapted.
    The first self-replicating systems arising in the primordial soup are most likely autocatalytic sets.
   \emph{ An autocalytic set is a collection of molecular types of which some are inputs freely available in the environment of the autocatalytic set, some are outputs which may be though of either as waste products dispensed into the environment or as complex molecular products which provide ingredients for the contruction of ever more complex molecules, as a result of the autocatalytic process, and a set of catalysts which facilitate the process.}
    Together the inputs and catalysts suffice to produce all the outputs and new copies of the catalysts.
    In the presence of the catalysts the reactions involved can proceed faster than they otherwise would.
    The manufacture of further catalysts permits the scale of the operation to expand.
    In this way autocatalytic sets are like factories for producing complex molecules.

\paragraph{Cell Membranes}
    In the laboratory such reactions would be undertaken in a test tube or flask where a good concentration of the elements of the catalytic set could be ensured.
    In the primordial soup, the products would be likely to disperse, and the realisation or maintenance of sufficient concentration would be difficult.
    This would be much more likely to happen in more confined environments such as small pools.
    For the kinds of complex chemical reactions needed for life to take place, something closer to a test tube is needed, and this is ultimately provided by cell boundaries, the evolution of which remains poorly understood.

\subsubsection {Prokaryotes}

Prokaryotes are single celled organisms lacking a cell nucleus, and are the earliest known living organisms.
Their first appearances in the fossil record have been dated at 3.5B years ago, perhaps 1 Billion years after the formation of the Earth. 


It is interest that they exhibit a variety of social behaviours.

\paragraph{Prokaryotes and their social behaviour}
  \begin{itemize}
  \item  division of labour
  \item making bodily contact with as many other bacteria as possible is more important to an individual [myxobacterium] than sidling up to a food source
  \item much of the genetically determined behaviour is oblivious to the survival of the individual and totally oriented to the survival of the group, particularly that which occurs when times are hard and new resources must be located
  \item form of progeny varies according to conditions, eater/replicators when food is plenteous, explorers when it is scarce
  \item recklessness with life when searching for food, explorers mostly sacrifice themselves and their chances of replication
  \item chemical success/``come hither'' and failure/``avoid'' signals sent out by search parties
    \end{itemize}

\subsubsection{Sexual Reproduction}

\subsubsection{Language}

Homo Sapiens was \emph{defined} by Aristotle as the \emph{rational animal}.
Few are prepared to attribute rationality to any other species.
Along with that rationality, perhaps inextricably intertwined, is the capability for language, providing a vector for evolving culture, and beyond rationality that characteristic we call ``intelligence''.

The most conspicuous and rapidly evolving feature of genus \emph{homo} leading to the emergence of anatomically modern homo sapiens, was the size of brain.
Average brain volume (as measured by cranial capacity) had doubled over the preceding 1.8 Million years.
Language, rationality and intelligence are aspects of information processing which depend upon that increase in brain capacity and seem likely to have co-evolved with it and provided the selective advantage which directed the evolution.

Though it is often said that homo sapiens was the first in these various respects, intelligence, rationality, language, sociality, culture and the complex societies thus enabled, all these had their precursors, some going back to the beginning of life on earth.
To understand our capability for reason and rationality, and to understand why and how these may fail, it may help to consider these precursors, the evolutionary processes leading to homo sapiens, and the cultural evolution which lead down through `the age of reason' to the present.


\chapter{The Paradox of Repressive Tolerance}

\section{Beyond the Paradox}
 
Notes on the "Paradox of Repressive Tolerance".
Discussion of Popper's "paradox of tolerance" and Marcuse's inversion of it as "Repressive Tolerance".

In ``The Open Society and its Enemies'' Karl Popper mentions in a footnote a ``Paradox of Tolerance'' to the effect that a completely tolerant society would be vulnerable to subversion by intolerant ideas, and society should therefore limit its tolerance of ideas to those which are not inconsistent with the continuation of a tolerant society.

That whole work of Popper was devoted to exposing the work of those philosophers whose ideas Popper considred the greatest threats to an ``open society', viz. Plato, Hegel and Marx.

As we have seen, the ideas which Popper implicates are alive and well, having evolved into forms which are much more persuasive and prolific than the originals.
Popper prescribed, however briefly, limits to tolerance, for the purpose of preventing the subversion of liberal democracies by totalitarian regimes, in the context of experience indicating that attempt to implement utopian ideas have resulted in totalitarian dystopias.
Other writers in the mid twentieth century were also inspired by similar motives to expose the workings of such regimes.
Perhaps the best known of these was George Orwell, who approached the expose through fiction in his ``Anmial Farm'' and ``1984'' \cite{orwell-af,orwell-1984,orwell-fd}.
A more scholarly approach was taken by Isaiah Berlin, who might perhaps have been an academic philosopher but for the negative effects of analytic positivism on the standing of political philosophy in mid 20th Century Oxford.
Instead he donned the mantle of historian of ideas and in that way contributed to our understanding.

\section{Introduction}

It is my intention in this short note to discuss a problem which Popper raises in a footnote in Chapter 7 of \emph{The Open Society and its Enemies}\cite{popper-ose}.
In that footnote, Popper does little more than note the problem and assert the need and the right to do something about it.

I propose here to look in a little more detail at the nature of the problem, which might appear in quite diverse ways, and also to consider how liberal democracies might defend against subversion by these means.

I therefore consider two strategies for intolerant subversion exemplified here by the Islamic notion of \emph{Dawa}\cite{ali-dawa,sookhdeo-dawa} and the Marcusian \emph{Repressive Tolerance}\cite{marcuse-repressive}.

It is not crucial to the discussion, or to the purpose of this note, that either of the sketches I give here of these two kinds of ideology is completely accurate, since they stand here as exemplars of certain \emph{kind} of ideology.
The question of interest is how liberal democracy could possibly be protected against idologies of those kinds, which remains of interest even if these kinds had not in fact been instantiated.
To that question I do not offer an answer, though there are some possibilities which are discussed.
My aim is to exhibit the difficulties they pose, and perhaps to add some sense of how great those difficulties are.

\section{Poppers Concern}

``The paradox of tolerance'' is a phrase coined by Popper concerning the defence of a tolerant society from the risk of subversion by an intolerant ideology.

Poppers oft-quoted description of that ``paradox''  is:

\begin{quote}

Less well known is the paradox of tolerance: Unlimited tolerance must lead to the disappearance of tolerance. If we extend unlimited tolerance even to those who are intolerant, if we are not prepared to defend a tolerant society against the onslaught of the intolerant, then the tolerant will be destroyed, and tolerance with them.—In this formulation, I do not imply, for instance, that we should always suppress the utterance of intolerant philosophies; as long as we can counter them by rational argument and keep them in check by public opinion, suppression would certainly be most unwise. But we should claim the right to suppress them if necessary even by force; for it may easily turn out that they are not prepared to meet us on the level of rational argument, but begin by denouncing all argument; they may forbid their followers to listen to rational argument, because it is deceptive, and teach them to answer arguments by the use of their fists or pistols. We should therefore claim, in the name of tolerance, the right not to tolerate the intolerant. We should claim that any movement preaching intolerance places itself outside the law and we should consider incitement to intolerance and persecution as criminal, in the same way as we should consider incitement to murder, or to kidnapping, or to the revival of the slave trade, as criminal.

\end{quote}

\section{Religious Fundamentalism}

The most plausuble instance of religious fundamentalism posing a contemporary threat to liberal democracy is political ISLAM and its clearly documented strategy of DAWA.

Distinctive features of this ``threat'' (if indeed it is) are its \emph{tranparency} and \emph{honesty}.
The desired end-point, a global islamic caliphate under sharia law, is well documented, and the strategy for realising that end is well developed and clearly documented.
To be sure, it is not specifically targeted at the subversion of liberal democracy, which was unknown at its inception, but liberal democracies are nevetheless particularly vulnerable to the same strategy.

The writings which form the Quran fall into two parts, of which the first was written in Mecca and the second in Medina.
It is in the second that Da'wa, as a strategy for the promulgation of ISLAM, was presented, with the ultimate aim of establishing a global Islamic caliphate governed by sharia law.

This exhibits a number of characteristics.

First note the following characteristics of the idiology (in this case a religion) which Dawa is intended to proliferate:

\begin{itemize}
\item It concerns governance of society.
\item It advocates a fixed and unchangeable body of law.
\item It is not democratic.
\item It is underpinned by a rigid authority held to be infallible.
\item It is highly intolerant of challeges to that authority.
\item It advocates lethal violence not only against direct challenges but also against peaceful co-existence.
\end{itemize}

The strategy for proliferation probably is more subtle than I will here present it, but for our discussion it may be considered to have the following characterstics when it is applied to the subversion of a liberal democracy:

\begin{itemize}
\item It seeks to subvert via the ballot box.
\item To secure electoral majorities it adopts the following three methods:
  \begin{itemize}
  \item Immigration
  \item Procreation
  \item Persuasion
\end{itemize}
\item Though Dawa allows for more forceful methods when these are necessary and feasible, up to Jihad, these are not likely to be essential for subversion of a liberal democracy.
\item Once electoral majorities are established, democracy can be dismantled and sharia law established.
\end{itemize}

\section{Neo-Marxism}

Possibly I will here be using the term ``Neo-Marxism'' in a broader way than is usually.
That's not important.

In some ways this subversive strategy is diametrically opposite to the strategy exemplified by Da'wa as described in the previous section.
It is neither clear, transparent, nor honest or coherent.

Rather than progressing some utopian vision of how society should be, it obsesses with a dystopian pseudo-reality and has evolved in the petri-dish of academic freedom within a highly politicised academy to misrepresent how things are and locate all possible grounds for social division to tear down existing social structures and make room for the utopia which they are unable to coherently articulate, and which their own doctrines deny can ever be realised.

A focal point here is the notion of \emph{repressive tolerance} in which Herbert Marcuse adapts Popper's Paradox of Tolerance to exactly that purpose which it was intended to warn us against\cite{marcuse-repressive}.
This is not only a key element in itself, but is a model for systematic inversion.

\subsection{Repressive Tolerance}

\section{Safeguarding Democracy}

\chapter{Philosophical Foundations for Liberal Democracy}

When society proceeds in a largely rational manner, what the average punter understands of the philosophical principles which underpin the social order may not be crucial.
When things go awry, not so good.
It might be a good idea to have a handbook which is as broadly intelligible as possible.
Professional philosophers are rarely good at stripping down the complexity which proves their metal and laying bare key structures.
This is me having a go at that.

\section{Introduction}

\ignore{

``Critical Social Justice''(CSJ) is an ideology which rejects rationality and the epistemological norms associated with ``Western Civilisation''.
These and other supposedly Western norms are rejected by post-colonial theory (an element of CSJ) as part of the systematic oppression by the colonial West of the colonised East.
The doctrines of Western Civilisation (and all other ``meta-narratives'') lack any objective truth and serve only to establish and perpetuate the power of the oppressors and the subjugation of the oppressed.

Those who see difficulties in this point of view, and think perhaps that there may be some redeeming features in the cultural heritage which seems to have abetted rising prosperity around the globe, will find dialogue with the proponents of CSJ unfruitful.
Reasoned dialogue is impossible with those who are unwilling to accept any rational ground rules.

For that reason this essay is not an attempt at dialogue.
It is a laying out, and an advocacy for, those elements of ``Western Culture'' which seem to me most fundamental to the advances which humanity has made over the last three millenia.
The essay addresses those who are not yet under the spell of CSJ ideology, and perhaps some who are already sufficient in doubt about their authodoxy that they will momentarily entertain an alternative viewpoint.
It is especially intended for those whose familiarity with the most fundamental parts of that Western Philosophy rejected by Critical Theory may be limited, who might not have thought these matters important, or who have been educated by schools and universities now intent on looking elsewhere.

My use here of the idea of {\it Western} Philosophy or Culture is perhaps anomalous.
Most of what I speak of is part of our common global heritage, it does not now and never did belong to one hemisphere of our planet.
Nevertheless this term is used, and I don't claim to be giving a broader story.
}%ignore

Where reason prevails liberal democracy may flourish.
Homo sapiens, possibly the only species capable of reason, is, however, also liable to tribal ideology in which belief in absurdities is the litmus test of ideologic purity.

In default of discourse, force may prevail.
Democracy is the choice to resolve differences at the political level without resort to force, and to provide a means (ultimately backed by a state mononopoly over the use of force) for the resolution of differences between individuals and groups within the state, on the basis of a code of law.

Not so long ago, national, let alone global, media were in the hands of tiny elites, and paid little attention to the opinions of the man in the street, which they could hope to shape.
In that context, ignorance of epistemological fundamentals and rational norms on the part of common folk might not have been so significant.
Today, we have reason for greater concern that ``narratives'' may be pressed, public opinion may be swayed, and governments unseated, by activist rhetoric (from anywhere on the political spectrum) in the face of contrary evidence.

In my primary and secondary education, I have no recollection of any attempt to educate me in how to assess the credibility of such claims.
Any ability to do so was acquired by osmosis rather than instruction.

In the contemporary concern over ``fake news'' the suggestion seems to be that one should judge the credibility of the source rather than probe the evidence.
Those sources most likely to be supposed authoritative, mainstream media, have become increasingly partisan, to the point where one may reasonably doubt whether they even recognise a distinction between facts and opinion.

In the new proliferation of sources, it might be helpful to have clear articulation of fundamental principles belonging to epistemology and the philosophy of language, logic and science together with exposure of the most common and egregious ways in which those principles may be flouted.
It is a challenge to see such ideas presented for a broad audience, perhaps together with more detailed and nuanced expansions for those willing to dig a little deeper.

In this essay I'm mainly interested in putting together a story, rather than in curating a presentation which presumes no prior philosophical understanding (though that is where I think I ought to be heading).

It is a discussion of \emph{foundations} and is therefore certain to be ultimately circular, but I hope, nonetheless informative.
This particular circularity involves philosophy of language, of logic, and epistemology, and the concept of \emph{rationality} with which I will begin.

\section{Rationality}

Like most words in the English language which have not been made precise by mathematics or science, the word \emph{rationality}\index{rationality|textbf} has diverse usage which thoroughly obscures the distinction between what is part of the \emph{meaning} of the concept and what is part of our beliefs about what is rational.

For my present purposes I propose to offer a definition of the concept, which is intended to clarify its use in this essay, not to say anything about how it may have previously been used.

My usage is primarily \emph{instrumental} and \emph{normative}.
A course of action may be said to have \emph{instrumental rationality}\index{rationality!instrumental|textbf}  if it is reasonable to suppose that it will realise the purpose for which it was undertaken.
A belief has \emph{epistemic rationality}\index{rationality!epistemic|textbf}, if the belief is held on good grounds, evidence which shows the belief to be most probably true.
We may consider epistemic rationality to be instrumentally rational, insofar as holding true belief will enable the adoption of effective ways of realising our purposes.

\section{Some Kinds of Knowledge}

I will mostly be concerned with aspects of epistemology which are confined to communicable knowledge, but its necessary first to make that distinction.

To that end I suggest that knowledge\index{knowledge} can usefully be considered as falling into three principle types:

\begin{itemize}
\item knowledge by acquaintance (been there, seen that)
\item knowing \emph{how} (done it)
\item knowing \emph{that} (got the tea-shirt?)
\end{itemize}

Knowledge \emph{by acquaintance}\index{knowledge!by acquaintance|textbf} is that familiarity which comes from direct experience, having been there, having seen it, perhaps something you heard, or even knowing how something \emph{feels}.
Maybe some phenomenon you have witnessed, a process you have observed, a ritual or a dance.

Knowing \emph{how}\index{knowing!how|textbf} will usually be a skill acquired by watching and doing, perhaps something which does not really require any \emph{skill} but just a knowledge of what to do acquired by seeing just the once, or trying and discovering.
Sometimes, even in non-human primates, it may be a skill properly mastered only over an extended period of time (years even).

Knowledge by acquaintance and knowing how are both found across a broad swathe of the biosphere, sometimes as innate knowledge buried in genes, sometimes passed from parent to child, or among peers, by example and mimicry, sometimes discovered.

Knowledge \emph{that}\index{knowing!that|textbf} is special and will be the main focus of our attention in this essay.
It appears only when we have descriptive language, in which \emph{propositions}, the content of indicative sentences, can be expressed.

A distinctive feature of this kind of knowledge is that it can be transferred in the absence of the circumstances to which it relates, passed from generation to generation or across large physical distances.
It makes possible an evolving oral culture propagated through story telling and singing.
In a period of unusually volatile climate change, geographical knowledge of where subsistence could be had in different climatic conditions would be valuable, social status would attach to the skills involved and the evolution of the physical and mental infrastructure for linguistic excellence would be accelerated by sexual selection on that basis.
Such conditions did occur in the 600 thousand year period starting just 800 thousand years ago.

During that period typical size of homo sapiens brains grew by about 50\% and then plateaued with the emergence of anatomically modern homo sapiens.
The linguistic abilities marked the beginning of culture and its evolution, an evolution which accelerated throughout the following 200,000 years, in steps which corresponded often to advances in our technology for preserving, replicating and communicating bodies of knowledge.


\section{Language}

\section{Logic}

\section{Metaphysics}

\section{Science}

\section{Skepticisms}

\section{Epistemologies}
\ignore{

\section{Epistemology}

Epistemology is the theory of knowledge.

We can begin with the following classification of kinds of knowledge:

\begin{itemize}
\item knowlege \emph{by acquaintance} 
\item knowing \emph{how}
\item knowing \emph{that}
\end{itemize}

Of which the first two are pre-lingual and are posessed by many species which have no capability for language, but the last is knowing the truth of some proposition, i.e. of that which is expressed by a sentence in a language.

We are concerned here only with \emph{knowing that}, and we find that the different ways of knowing (I shall use the notion of \emph{epistemic status} for this) are related to the meaning of a sentence.

There are important (if controversial) connections between distinctions in epistemology (epistemic status), distinctions in language (meaning or truth conditions), certain metaphysical distinctions (necessity, contingency) and the levels of confidence one can realise (certain knowledge v. speculative opinion).
I will run through these as best I can in short order.

Here is a table:

\begin{table}[h!]
\centering
 
\begin{tabular}{p{1cm} | p{1cm} p{1cm} p{1cm} p{1cm}}
  subject & abstract/math & concrete/science & moral values & personal values \\

  truth conditions & analytic & synthetic & ? & ? \\
  justification & deductive reason &  observation/experiment & god/conscience/reason? & preference/introspection \\
  modal & necessary &  contingent & \\
  confidence & certain & hypothetical &  good & \\
  objectivity &  objective & objective & objective/relative & subjective\\
  \end{tabular}
\end{table}

\section{Language}

The study of language includes two important elements:

\begin{itemize}
\item Pragmatics

  This concerns the ways in which language is used.
  
\item Semantics

  This concerns the \emph{meaning} of indicative sentences.
  
\end{itemize}

An important element of semantics is \emph{truth conditions}, these tell you under what possible circumstances each sentence is true.

}%ignore

\section{CSJ in brief}

From Pluckrose\cite{pluckrose-evolution}.

\begin{enumerate}[i)]
\item  racism exists today in both traditional and modern forms
\item  racism is an institutionalized multi-layered multi-level system that distributes unequal power and resources between white people and people of color, as socially identified, and disproportionately benefits White's
\item  all members of society are socialized to participate in the system of racism albeit in various social locations
\item all white people benefit from racism regardless of their intentions
\item no one chose to be socialized into racism so no one is bad, but no one is neutral so not to act against racism is to support racism
\item racism must be continually identified analyzed and challenged no one is ever done
\item the question is not did racism take place but how did racism manifest in that situation
\item the racial status quo is uncomfortable for most White's therefore anything that maintains white comfort is suspect
\item the racially oppressed have a more intimate insight via experiential knowledge into the system of race than their racial oppressors but they're not bad
  \item however white professors will be seen as having more legitimacy thus positionality must be intentionally engaged (means you must always mention your race gender and sexuality and how it impacts on what you're saying)
\item resistance is a predictable reaction to anti-racist education and must be explicitly and strategically addressed
\end{enumerate}

\section{Postmodern Precursors}

From Pluckrose\cite{pluckrose-evolution}.


The imperative then, of postmodern approaches, is to study the discourses of society, to find the Foucian power-knowledge, invert the Derridian binaries and empower the Lyotardian mini-narratives.

This yields the following ``plan'':
     \begin{enumerate}[i)]
     \item there is no way of obtaining objective truth, everything is culturally constructed
     \item society is dominated by systems of power and privilege that people just accept as common sense
     \item these vary from culture to culture and subculture to subculture
     \item none of them is right or superior to any other
     \item the categories that we use to understand things, like fact and fiction, reason and emotion, science and art and male and female, are false
     \item they operate in the service of power need to be examined, broken down and complicated
     \item language is immensely powerful and it is used to construct oppressive social realities, therefore it must be regarded with suspicion and scrutinized to find the discourses of power
     \item the intention of the speaker is no more authoritative than the interpretation of the hearer
     \item the idea of the autonomous individual is a myth, the individual is also a construct of culture programmed by his or her place in relation to power
     \item the idea of a universal human nature is also a myth, it is constructed by what
powerful forces deemed to be the right way to be, therefore it is white Western masculine and heterosexual.
     \end{enumerate}

\chapter{On Gender}


\section{Introduction}

I have no background which should give anyone reason to pay attention to my views on this topic.
Nevertheless, the controversy around gender is something upon which I feel I need to think through my views carefully.

These notes are my attempt to do that and are not backed up by any serious research.
For that reason I will first of all state the supposed factual basis on which my opinions about what should be done are based, so that readers can judge for themselves whether my opinions are rooted in fact or fiction.

\section{Background ``facts''}

Though not well acquainted with this topic when I first ventured this way, I imagined that I had something to say, and thought to write down the factual basis for the ideas I sought to present.
Not surprisingly I found that difficult, the attempt to explain what I thought I knew quickly lead me into deep water.
The remedy I adopted was to take an existing account of the background and base what I have to say on that foundatoin.
The account I chose was Helen Joyce's book ``Trans''\cite{joyce2021}.

Nevetherless, this section remains, which is now intended to work largely by reference to her that book.

I am primarily here concerned with language, so I will first desxribe my understanding of the development of language concerning sex and gender.

The word sex relates to the phenomenon of sexual reproduction, which in animals results in sexual dimorphism whereby the organisms of a sexually reproducing species produce and rear their offspring.
Individuals in such species fall almost entirely into groups, the females of the species and the males.

Part of the issue at stake here is whether sex is ``binary'' or whether it is a spectrum,
As far as received scientific terminology is concerned, there are just two sexes, male and female, it is not usually difficult to tell whether an individual is male or female, but there are some (a very small number of) cases where this gets more tricky.
This is much like ordinary life has usually been, one can normally easily tell someones sex, but occasionally its not so clear.
In the case of ordinary life, the ability to tell which sex someone is in is substantially aided by the adoption of social stereotypes which serve to make sex evident even if the biological markers of sex are not evident or unambiguous.
Today the number of difficult cases is on the rise, because an increasing number of individuals present as if of the opposite sex.

The existence of sexual stereotypes reflect the enormous importance of sexuality in the evolution of sexually reproducing species.
The choice of a sexual partner with whom to conceive and rear children is of the utmost importance for the continuation of the genetic constitution of the individuals of a species.

Most languages aid and abet the identification of sex by using distinct ways of referring to individuals of different sexes, and the importance of this way of speaking has lead to the linguistic markers (e.g. sex specific pronouns) being used for things which are not biological organisms and have no sex.
Ships, for example.

The word gender came into use (in the 14th Century) as a way of talking about the features of language in question, as classifications of nouns and pronouns based around though possibly extended from the biological sex of the individual referred to.
Likely at that time, the gender was the sex if the referent had a sex, and consequently gender was thought of as a synonym for ``sex'' over that domain (from 15th Century).

It is not untill the 20th Century that gender begins to be used in a way distinct from sex, at first as referring to presentation rather than substance.
After this there seems to be a great deal of fluidity.

[further details]

Notwithstanding the transformation which has occurred in the used of the term gender, the uk government web site dealing with gender recognition certificates asserts that in UK law sex and gender are regarded as synonymous.
This is supported by the fact that once a gender recognition certificate has been geranted, the sex on a birth certificate can be changed.

Nevertheless, when speaking about the equalities act, the UK government speaks of the need to be clear about the distinction between sex and gender, and I belie that the provisions allowing for single sex spaces are specific to sex and allow exclusion of members of the relevant gender who are not of the relevant sex.

It looks as if, at the very least, what the government says about the law is incoherent, and at worst the law really is inconsistent.

This doubtless contributes to the rationale supporting those defenders of single sex spaces who advocate repeal of the gender recognition act.

\section{The ``Cotton Ceiling''}

The term ``Cotton Ceiling'' was introduced, I presume, by someone who regarded it as an injustice to trans identify males that some women are homosexual.

The term homosexual means \emph{a person who is sexually attracted (only) to people of their own sex}.

\section{What to Do}

Women's rights as understood and fought for in the 20th Century (and before) are inconsistent with the conflation of sex with the gender as it has come to be used.
Some people clearly do not want to see the preservation of these rights.
These ideas will not suit them.

A first step to resolution must be to make sure that UK law unambiguously supports the distinction between sex and gender.

Advocates for ``gender self-id'' would like gender be something entirely subjective, such that no evidence of any kind can speak against the conviction of a person about his or her gender.

\section{On the Determination of Sex}

This is a bit of discussion from a non-expert.

In mammals, males and females are distinguished by their roles in procreation.
The child begins as a zygote formed as the union of two gametes, one from a male parent and one from a female parent.
The female gamete is called an ovum or egg which has only half the genetic material normal for members of the species, and becomes fully endowed as a zygote when complemntary genetic material is supplied by union with the male gamete, usually called a sperm.
Though males and females might be distinguished by the type of gamete they are able to produce, infertility is not a bar to the determination of sex.

This can be determined by examination of the chromosomes, which will usually contain chromosomes determining sex.
The genetic material is present in every cell of the body and are not modified by any of the interventions which are customary in ``gender transition''.
The relevant chromosomes are called the X and Y chromosomes, a woman will have two X chromosomes, and a man will normally have an X and a Y chromosome.
In rare cases the genetic material which determines manhood which normally will be in a Y chromosome, may appear in an X chromosome, and in some cases multiple X chromosomes will be present in males.can be determined by sequencing the genome and checking for the presence of a gene called SRY, the presence of which will trigger the development of a male fetus.

Thus one might define a male as an individual whose genome contains the SRY gene.
For that to be definite, one would need an unambiguous definition of the SRY gene.
Genes are coded into DNA as sequences of codons, which represent the sequence of amino acids in a protein.
It is normal for there to be benign variants in the gene pool, which code for proteins which are functionally equivalent to the normal protein.
So to chose one specific sequence of codons or amnino acids in the DNA or in the protein respectively would risk a definition which categorised differently induviduals whose genomes were functionally identical in the relevant areas.

\chapter{Progressive Resistance}

\section{Preface}

Though I have had an account on Twitter for more than a decade, it is only in the last couple of years that is has engaged my attention.
Though social media in general, and twitter in particular, come in for a lot of flack, its clear that they are not all bad.
Through twitter I have become aware of a many things of which I most probably would otherwise have remained blissfully ignorant.

One of those is the migration of US citizens who thought of themselves as ``liberals'' out of the democratic party, sometimes into a central no-mans-land, sometimes as far as the now trump-tainted republicans.
One might suspect that this is symptomatic of that decline in idealism which often accompanies advancing years, but that is certainly not how they perceive it.
The complaint often is, that they have just the same ideals as they had before, but that the democratic party has radicalised and left them adrift.
The doctrines and policies which have been particularly effective in provoking the migration are those dubbed ``identity politics'', the praxis of applied critical Marxism.
They are perceived to have created a new radical strain in the democratic party, the dominance of which was not apparent until after the election of the moderate Joe Biden.
Similar effects in many other English speaking countries.

These ideas and policies leave former progressives fighting a rearguard action, suddenly seeking to conserve valued features of our culture and body politic instead of continuing to press for new policies which address their defects.

That people seeking progress should feel it necessary to devote their energy to preventing what they see as undesirable change is regrettable.
A large proportion of misdirected zeal is progressive in intent.
It would be better if it were possible, to offer an alternative and better way of achieving the advances radicals seek rather than trying to stop them in their tracks.
That would involve getting them to think harder about exactly what progress they are seeking.

This essay is an attempt to articulate how that might be done.

I am not a scholar, not expert in any topic (though better in some than others) and least of all a historian.
Nevertheless, my concern about the direction and impact of modern activists has made me enquire into the history, in the hope that understanding might follow, and my ideas here about how we might make real progress are given context by reference to the history as I understand it.
It will not be difficult for most people to pick holes in my understanding of the history.
The ideas nevertheless are offered for consideration in their own right, and are not original or complex (though the composite may be).
The historical references are primarily to illuminate rather than justify the choices I present.


\section{Introduction}

The ``progressive resistance'' we envisage here is at once resistance to some aspects of the dominant progressive movements of the day not from a conservative spirit but from a perspective which recognises and values the progress that has already been made and seeks to build on it while excising its least desirable tendencies and effects.

If pressed to name one feature of some ``progressive'' actors it is the doctrine, which we associate among others with Herbert Marcuse, that the present system must be completely destroyed as a prelude to, ... to what?
This attitude, is not progressive but revolutionary.
It is the contentment of people with the system which is the principle problem, and which mitigates against any movement for radical change.
This satisfaction is decried as ``false consciousness'', a phrase whose purpose is to legitimise the radical agenda, \emph{divide} and \emph{destroy}.

\begin{itemize}
\item Hegel
\item Marx
\item Horkheimer's Critical Theory
\item Marcuse's Critical Theory
\item Postmodernisms
\item Applied Critical Theories
\item Praxis
\end{itemize}

\section{Hegel}

My knowledge of Hegel is based on slender foundations, which don't include actually reading what he wrote.
I don't include this section in the belief that I know Hegel better than anyone else, but rather by way of explanation of how I have arrived at the ideas which I put forward in this essay.

I did once have Hegel's logic on my bookshelf, but my bullshit detector went apoplectic when I tried to read it, and it went to charity (or probably to pulp) the last time we downsized.
The delusions which I will here recount are therefore inspired by secondary sources, of which the most important are Sabine \cite{sabine63} and Singer interviewed by Magee \cite{magee-singer}.
Long before consulting them I had read Popper's excellent diatribe \cite{popper-ose}, which undoubtedly contributed to my negative opinion of Hegel, but not so far as I am aware to the detail below.

Since the invention of axiomatic geometry, deductive reasoning has had an allure to those who want their opinions to be considered definitive.
Unfortunately, almost no significant doctrines outside of mathematics fall within the scope of deductive logic.
For a good dose of realism about what does and does not fall within that scope, Hume's \emph{Enquiry} \cite{humeECHU} is worth a look, where the relevant claims are said to express ``relations between ideas'' and are clearly distinguished from ``matters of fact'' which are beyond the reach of pure reason.
That dichotomy became known as \emph{hume's fork}, and was supplemented by another Humean dictum, that you cannot derive an \emph{ought} from an \emph{is}\footnote{distinguishing logical and empirical truths from moral imperatives and other value judgements}, also sometimes known as Hume's fork.

Both before and after Hume these distinction have been misunderstood, or simply ignored, by those who want their views to be seen as proven, and for the sake of certainty seek the imprimateur of deductive logical proof.
To be fair, our language is not so precise as to make these distinctions watertight, but they nevertheless have merit in separating claims which it is prudent to separate because the ways in which the claims can be verified differ in fundamental ways.
\emph{Progressive resistance} promotes that kind of precision in language which supports the relevant epistemic differences.
In general, the progressive impulse is better served by noting imperfections, whether in language or in society, and seeking to illiminate or ameliorate them, by contrast with the revolutionary mindset, which seizes and eggagerates inevitable weaknesses to progress the \emph{divide and destroy} path to utopia.

Hegel, having spent most of his life as a historian, was naturally inclined to think history important, and to think that his method of reasoning about its progress unimpeachable.
In record breaking overreach he claimed of his dialectical method that it had the force of logic, that it enabled the prediction of what history would yield, and that not only di the predicted results have the abolute necessity which is commonly associated with deductive reasoning, but that they also constituted a moral imperative.

It is probable that Hegel did not envisage this doctrine fuelling the hubris of revolutionary zealots, but this was not the first or the last time that a philosopher would find his writings headings underpinning developments he would not condone.


\section{Marx}

Philosophically, I gather, Marx whose method is called ``Dialectical Materialism'', is only a small step from Hegel's Dialectic.
Whereas Hegel was an idealist, an a ``right-Hegelian'' and the dialectic he invisaged was of the national culture or spirit evolving dialectically toward some perfect state, Marx was a materialist and the evolution at stake was economic, of the means of production.

This is the most fundamental point on which compromise seems infeasible.
Either you seek to destroy the system or to reform or advance it, there is no between.
in addition to the incompatibility of these two as ends, the means whereby these distinct ends are approached are likely to be highly contrasted, not least because the revolutionary impulse regards incremental progress as counterproductive because of its potential to make the revolution less essential or urgent.
Also, revolution is destructive, incrememental progress is constructive.

Soon, as we transition into Critical Theory, the prospect of violent overthrow of the system loses credibility, and the strategy for revolution becomes cultural rather than military.
The distinction remains, stark, between a strategy predicated on complete destruction of the existing system, and one of progressive evolution, building on strengths and addressing weaknesses.

This is the distinction we sloganise as that between:

\begin{minipage}[t]{0.8\linewidth}

\begin{centering}
\vspace{0.1in}
       {\bf DIVIDE and DESTROY}
       
\vspace{0.1in}
and

\vspace{0.1in}
       {\bf PRESERVE and PROGRESS}
       
\vspace{0.1in}
\        
\end{centering}
\end{minipage}


It is the latter strategy which underpins \emph{Progressive Resistance}, in which we seek to resist destruction of the culture and institutions which have brought us prosperity and humanity so that we can continue the progress which has brought them this far.
That name is chosen to emphasise that the resistence against the revolutionary impulse is not a purely conservative or reactionary force, it can be as solidly rooted in a commitment to progress.
Of course, that is not how a revolutionary would speak of these things, if he were not already convinced that our culture and institutions were fundamentally flawed then he would make that pretence to fuel the insurrection.

From this stage in history I nevertheless take up some elements.
To the extent that some groups of people are unfairly exploited economically, that is something which we would wish to address, and history has surely shown that in economic terms, progress is more likely by incremental than by revoutionary means.
However, 


\section{Critical Theory}

Critical Theory is a derivative of Marxism developed at the Institute for Social Research, initially in Frankfurt.
The innovations in Critical Theory are generally held to respond to the failure of the Marxist prediction of the collapse of Capitalism as a result of a revolution of the proletariat.

In re-thinking Marxism as Critical Theory, primarily at first under the leadership of Horkheimer, we see a mix of ideas which can comfortably fit into a progressive agenda, with some which seem exclusively revolutionary.

The former can be selectively taken into a modern progressive program, the latter do not belong there.

\subsection{Some Philosophy}

First of all let us consider the philosophical and methodological elements.

There is doubtless more from Hegel/Marx here than I am aware of us.
These are the points which have influenced my conception of \emph{progressive resistance}.

The phrase ``Critical Theory'', which seems to be a replacement for ``Marxism'', is contrasted not with Marxism but with ``Classical Theory'' which Horkheimer seems to have a more definite conception of than most philosophers, as perhaps a not quite so extreme version of logical positivism, which was probably just past its peak as Horkheimer took the lead at the Institute of Social Studies.
Nevertheless the content of the proposal has some merit.

Sociology may possibly have been the science least well suited to the model of science offered by physics, which provided an ideal on which philosophies of science are oriented.
Unfortunately, people are unpredictable, in general, but malleable under social influence, both through the imposition of culture in the development of the young, and through the impact of social context on individual behaviour.
During the enlightenment the successes of the hard sciences were to readily supposed replicable in the social sciences, and eventually some alternative conception of the sciences would have to prevail.
One such alternative is the Hegelian dialectic, in which it is historical forces which shape the present and future of society, a model inherited by dialectical materialism.

However, the predictions of Marx had failed to materialise.
A revolution did take place in Russia, but it was not a proletarian revolution.
The anticipated revolution in Germany did not materialise.
Some further development of the ideas was surely necessary (though some Marixsts would disagree, including Herbert Marcuse\footnote{
 In 1977 Marcuse denied that Marxism had been shown to be fallacious, when questioned about this by Brian Magee \cite{magee-marcuse}}, perhaps the most important figure following Horkheimer in Critical Theory).

\subsection{The Aim of Emancipation}

The Marxist goal might be thought of as the emancipation of the proletariat from exploitation by capitalists.

\subsection{Democracy and Liberty}

\chapter{Parents and Education}
Who should know about and/or have influence over, what is taught to children.


\section{Introduction}

The topic is ``\emph{Parents and Education}'', which is expanded in the first instance as:

\begin{enumerate}
\item Should parents
  \begin{itemize}
  \item[(a)] be informed about

    and/or
  \item[(b)] have influence on
    \end{itemize}
what is being taught to their children?

\item Is the answer to (1):
  \begin{itemize}
  \item[(a)] a moral imperative

    and/or
    \item[(b)] a democratic choice?
  \end{itemize}
\item To what extent is UK primary education consistent with the answers,
and if not, what would it take to bring it in line?
\end{enumerate}

These notes will not be tightly focussed on those specific questions.
To understand why this has come to be a matter of controversy some contemporary context is needed.

In brief, a natural response to that question is, ``why?''.
Why would parents feel the need to see in detail what was being taught to their children or to have any unfluence upon it?
I think for most of my life it would not have occurred to me that it should be necessary.
We have professionals who are better qualified to look after these things, public servants who in the end are answerable to democratically elected governments.

Another relevant ``why?'' here is ``why would anyone object to parents knowing what is being taught to their children?'', and in one possible answer to that question we see a reason why parents might seek influence: ``because they are intent on teaching things which parents would find objectionable''.


\section{Controversy on Curriculum and Teaching Methods}

Here are some possible areas of controversy which might lead parents to feel that they need to know or want to influemce what is being taught, or on which we might expect the state to exercise diligence and exercise control.

\begin{itemize}

\item[\bf religion:]

  \begin{itemize}
    
  \item Should fundamentalist religous or radical political views be taught (``as fact'', rather than discussed)?
  \item Should parents be able to influence the content of sex education?
    \item Should religious beliefs which conflict with received science be taught alongside or instead?
  \end{itemize}
  
\item[\bf curriculum balance:]

  STEM v humanities v sports and arts/music

  \item teaching ethos (progressive v. traditional, by discovery v. by instruction)

  \item[\bf political:]
    Radical political groups may seek ideologically based instruction.
    Knowing parents are likely to disapprove, they may conceal curricula detail, hiding it behind inoccuous gloss, and may resist parental or state influence (ignoring parent lobby groups and even the law).

    \item Save Our Schools v. Safe School Alliance
    
\end{itemize}

\section{Three Levels of Engagement}

The three are:

\begin{itemize}
\item Pragmatic/Methodological

  e.g. learn by discovery v. more traditional teaching

\item Progressive/Political

DEI inspired curricula, identity politics
  
\item Revolutionary/Ideological

  advancement of revolutionary agendas by teaching contrary to cultural norms, and radical divisive ideologies intended to create social conflict
\end{itemize}

\section{The Political Progressive Dimension}

This is where those belong who, however radical their views are on how education and society at large should be transformed, are intent on progressive reform rather revolutionary dismantlement.

The naive expectation in the first instance (it is perhaps a testimony to the era in which some of us grew up that such a naive conception might be possible) is that the main issues are not about the purpose of education, but rather about which methods are most effective in realising that purpose and about what kind of content should be prioritised, about the balance across the kinds of things that could be taught.

In that naive state, the idea that the there might be substantive disagreements not about what areas to include in the curriculum, but also about what the truths are in any area.
Beyond the possibility of dispute over the ``facts'', there is the question to what extent education should go beyond facts into values and morals.

In my life, it has been possible to believe that there is a broad consensus about facts, underpinned by various institutions whose business it is to look into the facts, and that we have a shared culture and a large measure of agreement about values (though this retrospective opinion may be unduly influenced by a sense of ongoing deteriorarion).
Educational instutions were primarily charged with teaching facts and skills, while upholding the moral compass of the age, which would primarily be transferred by osmosis and example.

No doubt that's a gross oversimplification or even badly at odds with your perceotions, but perhaps its a useful contrast with the present situation, in which consensus on the facts is disintegrating, trust in any kind of authority is greatly diminished, the educational curriculum has become a target for activists seeking to prevent the transfer of culture from parents to children and teach their own doctrines as not only factual but facts with moral force behind them.

\section{Revolution}

Under this heading are those who regard the existing system as irredemable and in need of complete destruction before anything better can be put in its place.

\cite{pluckrose-cynical,lindsay-racemarx,friere-poled,gottesman-criturn}

\part{Metaphysics}

\chapter{Notes on Orenstein on Quine}

\section{Introduction}

These are notes written in connection with a reading and discussion of Quine based on Orenstein.

\section{Expressing Ontology}

What is an ontology? Why would we want to express one? What is an ontological ``commitment''?
And why does any of this matter, aren't we in angels on pinheads territory here?

Ontology is concerned with existence, when used philosophically it is a branch of metaphysics.

An \emph{ontology} is an account, description or specification of what exists.
It may purport to be an \emph{absolute} account, giving a full inventory of everything that exists, concrete and abstract, or it may be partial, to support articulation of some particular science, or it may be an ontology offered as a pragmatic {\it convention}, applicable to some particular context or language, rather than as absolute truth.

Reasons for articulating an ontology are often scientific, ordinary discourse preferring liberal, flexible and context sensitive attitudes, often speaking {\it as if} a certain kind of entity existed (say, a route, or a detective called Holmes) without intending more than a {\it figure of speech}.
In science, precision of ontology is associated with rigour of reasoning.
Ontological confusion begets paradox and logical incoherence.
This applies particularly to mathematics, the subject matter most conspicuously dominated by long chains of deductive reasoning, whose pre-eminent deductive rigour is underpinned by that ontological precision arising from definition rather than observation.
It is because Quine's philosophy begins at a time when the logical foundations of mathematics have been transformed, while sailing perilously close to incoherence, that ontology looms large.

\subsection{Preliminary Observations}
  
As far as I can see here the only two things which appear here, are original with Quine, and may be significant in the rest of his philosophy are:

\begin{enumerate}
\item The dictum ``to be is to be the value of a variable''.
\item The notion of ``ontological commitment'' 
\end{enumerate}

Context for an understanding of Quine's views on these matters includes:
\begin{enumerate}[(a)]
\item That existence is ``not a predicate'' in modern logic
\item The trajectory of Russell ontological position and its continuation and analysis by Quine. 
  \begin{enumerate}[i]
  \item Russell's Meinongian phase.
  \item Russell's Theory of Descriptions \cite{russell1956,russellOD} and Incomplete Symbols \cite{russell10}
  \item Russell's {\it Theory of Types} \cite{russell1908}
  \item Quine's analysis in his ``Set Theory and its Logic'' \cite{quineSTAIL}
  \end{enumerate}
\end{enumerate}

As to 1), this originates in Quine's analysis of Russell's use of ``incomplete symbols'' in \emph{Principia Mathematica} \cite{russell10} and its application to set theory in Quine's later work.
I really don't think it is of much importance for what follows, but I give a fuller discussion of its origins and significance below.

The notion of ``ontological commitment'' is, as far as I am aware, a neologism of Quine's.
Previously philosophers such as Russell talked of the ``existential import of a proposition'', a topic on which Russell published a paper \cite{russell-existential}.
The main difference, it seems to me, in introducing this terminology, is to suggest that someone using certain forms of language themselves to certain ontological proposition whether they intended to or not.
This may be read as an implicit critique of Carnap's ontological views, which were conventionalistic and anti-metaphysical.
Consequently, I am myself inclined to deprecate the use of this terminology.

The two dictums are connected.
The connection implicates that if you want your variables to range over some kind of thing, then you are committed to the view that these things really do exist, a view which is never held by an anti-metaphysical ontological conventionalist.
The ``anti-metaphysical'' view here is that absolute ontological claims are meaningless, along with the rest of metaphysics.

Historically, questions associated with ontology had a particular imnportance in the context in which Quine was educated, because of the great imnportance of the work in logic which took place around the turn of the 20th Century.

\subsection{The Rigorisation of Analysis and its Logical Foundations}

Well for context let's go back to the year 1900.
This is a new century after a century of major developments in the foundations of mathematics which have latterly focussed on `logical' foundations.

Why did the foundations need fixing?
Well, possibly the most important ever discovery in mathematics was the independent development by Newton and Leibniz of the differential and integral calculus, which paved the way for the quantitative science which underpinned the industrial revolution.
The foundations of mathematics, specifically of the calculus, were attacked at the time by Bishop Berkeley, particularly in relation to the use by Leibniz of ``infinitesimals'' (or fluxions in Newton's version).
However tenuous the idea of infinitesimal might then have been, the calculus worked in practice, and enabled widespread application of Newtonian mechanics to science and engineering.
It was an unstoppable juggernaut sweeping aside intellectual doubts for the duration of the eighteenth century.

Come the nineteenth, even mathematicians began to have doubts.

The mathematician Abel said in 1828:
\begin{quote}
  {\it There are very few theorems in advanced analysis which have been demonstrated in a logically tenable manner. Everywhere one finds this miserable way of concluding from the special to the general, and it is extremely peculiar that such a procedure has lead to so few of the so-called paradoxes.}
\end{quote}

Once mathematicians started to put their house in order a series of developments gradually took the foundations deeper and deeper until the prospect of reducing mathematics to logic (depending on exactly what you mean by that) became real, and it was this that stimulated the developments in logic which completely overtook the logic of Aristotle for the first time in over 2000 years, created the discipline of Mathematical Logic, completely transformed Philosophical Logic and the Philosophy of Logic, and later enabled extensive work in (and applications of) logic and formal semantics in theoretical Computer Science.

The progression took place first by eliminating the need for infinitesimals using limits of sequences of `real' numbers, and then by the arithmetisation of analysis by progressively defining real numbers as sets of rational numbers, rational numbers as equivalence classes of pairs of integers, and integers in terms of the natural numbers (arithmetic). (or something like that, there is more than one way of skinning this cat).
In the latter part of the 19th century attention turned to irrational numbers.
Real numbers were defined by Dedekind as certain sets of rationals.
The theory of rational and natural numbers were then clarified in turn, ultimately reducing all of these systems to set theory and logic.

By these methods the differential and integral calculus, (usually referred to by mathematicians as analysis) had been ``reduced'' to arithmetic, the arithmetisation of analysis was complete.

The mathematical project of improving the rigour of analysis having been successfully undertaken, a philosophical project now takes center stage: logicism the thesis that Mathematics is Logic.
Hume's distinction between ``relations between ideas'' and ``matters of fact'', affirming mathematics and all necessary truth to be in the first category, was disputed by Hume, who introduced the analytic/synthetic terminology and classifying mathematics as synthetic a priori.
Both Frege and Russell disagreed with Kant on the status of mathematics, and both were determined to set the record straight by

All that remained was to show that natural numbers could be given a logical definition and to exhibit a formal system of logic capable of deriving the required properties.

\subsection{Frege's Logicism}

\begin{quote}
{\it  I hope I may claim in the present work to have made it probable that the laws of arithmetic are analytic judgements and consequently \emph{a priori}.
Arithmetic thus becomes simply a development of logic, and every proposition of arithmetic a law of logic, albeit a derivative one.}
\end{quote}
The Foundations of Arithmetic \cite{frege1884}, §87.

To that end he created a new formal logic which he called ``Concept Script'' ({\it Begriffsschrift})\cite{heijenoort67,frege1879}, published in 1879.
This was a ground-breaking work the ``fundamental contributions'' of which are summarised by Van Heijenoort as:

\begin{enumerate}
\item the truth-functional propositional calculus
\item the analysis of proposition into function and argument rather than subject and predicate
\item the theory of quantification
\item a system of logic based exclusive on the syntactic structure of the expressions
  \item a logical definition of the notion of mathematical sequence
  \end{enumerate}


It was the first mathematical work in logic in which formal rules of derivation were codified.
Previous mathematical work relevant in logical matters had been, like Booles work on his eponymous algebra of propositions or truth values, algebraic in character, investigated using the extant and uncodified deductive methods of mathematical proof.

Secondly, it introduced variables and quantification (the universal quantifier which by negation yields the existential), abandoning the Aristotelian propositional forms which had ruled logic for millenia.

\subsubsection{Variables and Quantification}

Aristotle was the first to study and write about logic, and though there had been much further work in the two millenia which followed his innovations, the basic structure which he laid out had remained intact, and was inadequate to give an account of the reasoning undertaken in the queen of the deductive sciences, mathematics.

Aristotle's ``syllogistic'' logic was build around a model of language as ''categorical propositions''.
A categorical proposition is always in ``subject-predicate'' form, i.e. consists of the application of a predicate to some subject.
For example, in the proposition ``All men are mortal'' the predicate `mortal' is asserted of the subject `all men' (using the copula `are').
The subject may be singular or universal, a singular subject being an individual (perhaps `Socrates'), a universal being some category (e.g. `all men').
The predicate may only be applied to a single subject, and therefore some difficulty arises in asserting relations, and the possibility of asserting relations between subjects within the constraints imposed by this conception of categorical proposition was a matter of controversy, asserted for example, by Leibniz and denied by Russell.

There are just four variations on that theme which are admitted in Aristotle's logic as follows.

\begin{table}[h]
\begin{center}
  \caption{forms of categorical proposition}
\begin{tabular}{|l|l|l|}
\hline
form & proposition & paraphrase \\
\hline
Aab &	a belongs to all b & Every b is a \\
Eab &	a belongs to no b & No b is a \\
Iab &	a belongs to some b & Some b is a \\
Oab &	a does not belong to all b & Some b is not a \\
\hline
\end{tabular}
\end{center}
\end{table}

Quantification applied, in effect, to propositional {\it functions} bringing into logic the important mathematical notion of function and the expressive power which it carried.

\subsubsection{The {\it Grundgesetze}}

He went on to apply his {\it Begriffsschrift} (after some further development) to the formal derivation of mathematics in his \emph{Grundgesetze der Arithmetik}\cite{frege1893,frege1903}.
The first volume was published in 1993, but Russell had only become acquainted with Frege's work in 1900.

\subsection{Russell's Paradox}

In the course of completing his manuscript for \emph{The Principles of Mathematics} (sometime between 1900 and 1902) Russell realised that a contradiction could be obtained in Frege's logical system.
The contradiction arose from rules about existence which were too liberal, admitting the existence of what is now sometimes known as {\it the Russell set} and engendering {\it Russell's paradox} (also attributed to Zermelo and Cantor).

This paradox arises from the availability in the system of unrestricted set comprehension, the idea that for any property expressible in the notation there exists a set containing all and only those things which satisfy the property.
This includes the set of sets which do not include themselves, the existence of which proves to be paradoxical, as is seen by asking the question whether this set does or does not contain itself.
An alternative presentation, for those not wholly comfortable with abstract entities is the barber who cuts the hair of everyone in his home town who does not cut his own hair, and cnmsequenmtly both does and does not cut his own hair.

He communicated this discovery to Frege in 1902, just as the second volume of the {\it Grudgesetze} was about to go to press, a devastating blow to his work from which Frege never fully recovered.

Russell remained intent on his own work to the same end, the logicisation of mathematics.

He was henceforth aware of the perilous danger of contradiction from what might seem innocuous ontological principles, and there ensued a period of great intellectual difficulty as he struggled to find a logical system which avoided the paradoxes in a way which was both philosophically rational and technically adequate for the derivation of mathematics.

The paradox having an ontological flavour, the evolution of Russell's views on ontology was important, and feeds into the Quinian doctrines under discussion.
This is a part but by no means the whole of the development of Russell's logical system {\it The Theory of Types} \cite{russell1908}.

But first, by way of showing Russell's dedication to that Fregean enterprise a quotation from the preface to the second edition of {\it The Principles of Mathematics} (1937):

\begin{quote}
  {\it The fundamental thesis of the following pages, that mathematics and logic are identical, is one which I have never since seen any reason to modify.
  }
\end{quote}

\subsection{Descriptions and Other Incomplete Symbols}

At this stage Russell was sympathetic to the lavish ontological conclusions of Meinong \cite{meinong-gegenstandstheorie-a,meinong-gegenstandstheorie-b}, which may be seen in the following quotation from {\it The Principles of Mathematics}.

\begin{quote}
  {\it What does not exist must be something, or it would be meaningless to deny its existence; and hence we need the concept of being, as that which belongs even to the non-existent.}
\end{quote}
(1903) paragraph 427\cite{russell03}

He has thus separated the concepts of {\it existence} and of {\it being}, the latter including many things which do not exist.

His ontology will soon become more spartan, and the notion of {\it being} as distinct from existence will no longer be required.
This is realised through Russell's {\it Theory of Descriptions}, pubished in his paper {\it On Denoting} in 1905 \cite{russellOD}.

Russell's theory of descriptions is a method for expressing in the language of Principia (i.e. Russell's Theory of Types) sentences which assert something of an object identified using a description, where it is possible that nothing satisfies the description.
The method treats the description as having meaning only in context not in isolation, and in a suitable context gives it meaning by translating into an expression in which the description no longer appears.
Because the description does not by itself have any meaning it is called an ``incomplete symbol''.
This trick is used in Principia Mathematica for things other than descriptions, notably for class abstracts, which refer to a class by giving a predicate which characterises (determines) the members of the class.
By this means Russell's Type Theory was considered by Russell a ``no class'' theory, because though Principia uses language which looks like its referring to classs, this is all incomplete symbols which are eliminable by following the prescribed rules for translating these away in the appropriate contexts.

When later Quine is developing ideas for set theoretic (or class theoretic) ways of logically deriving mathematics, he looks closely into how much can be achieved using ``virtual classes'' (i.e. class notations which are incomplete symbols and hence don't really denote classes) and when this technique runs out of steam and you really need classes rather than the pretence.
The answer is, ``when you need to quantify over them'', because if you use virtual classes they don't really exist and so they won't be in the range of the quantifiers, that's really the whole point.

Quine's analysis here is very nice, and may be found in his book ``Set theory and its logic''\cite{quineSTAIL}.
However, its philosophical significance is not in my opinion much to write home about.
Injecting it into philosophical ontological discussion with the dictum ``to be is to be the value of a variable'' is not very impressive, this is a bit ambiguous and is either a truism or falsehood according to how you read it.

It embodies the assumption that quantifiers range over the entirety of some collection of things which in some absolute sense exist.
Some, including Carnap and me, don't believe that there are any ontoogical absolutes, questions about ontology are meaningless except in some context which determines the relevant ontology

\part{Epistemology}

\chapter{An Architecture for Propositional Knowledge}

An exercise in synthetic epistmology, constituting architectural principles for a hypothesized future Galactic cognitive system.


\section{Introduction}

This essay is a thought experiment in synthetic epistemology, the construction of an episteme suitable for an intragalactic cognitive system.
The episteme discussed here is concerned only with propositional knowledge, which may be characterised as knowledge \emph{that} (knowing the truth of propositions) rather than knowing \emph{how} (having some skill, or nohow).

The distinction between knowledge and opinion is addressed obliquely here; logical coherence and deductive soundness are fundamental concerns.
The latter concern is made more critical because I presume that the deductive capability of the system is very high, and that any infelicity in the logical system will quickly lead to false and incoherent conclusions.

I begin by lightly sketching the scenario for which the episteme is intended.
The construction then takes place in stages corresponding to levels in an abstract model of the representation of propositional knowledge by the system.

In explaining and justifying some of the choices made, appeal will sometimes be made to the history of the development of philosophy, mathematics and science.

\section{The Scenario}

I imagine that \emph{homo sapiens} has suceeded in undertaking interstellar travel, and that self-proliferating intelligent systems have spread over a substantial portion of the Milky Way galaxy, perhaps a hundred thousand light years from planet earth at some points.

I imagine that there has been some kind of earthly sponsored space race, and that the bulk of this region of our galaxy is mainly occupied by the kinds of self-proliferating intelligent system which are most effective in proliferating themselves across the galaxy.
This means that they are:
\begin{itemize}
\item not human
\item technologically very advanced
\item dedicated to advancing the knowledge necessary for rapid proliferation across the galaxy
\end{itemize}

The life cycle of these systems runs something like this.
Let us begin the cycle with one of the largest (goegraphical) leaps of which the system is capable, i.e. a basic system has been sent over a large expanse of space into some region devoid of self-proliferating intelligent systems.
There may already have been intelligent probes which have contributed to the decision about where this `seed' should be directed, but on arrival in the inteded area there is some scope for choosing the best place to start development.

The system comes with enough knowledge to make a start, but will need to import knowledge on a large scale from its nearest neighbours in order to develop to the point at which it is capable of sending out a simliar seeds to some yet further destination.
The first task therefore is to begin extracting from its environment the raw products necessary to build out information and communications infrastructue, so that it becomes a functional node in the intragalactic cognitive network.
The information technology being deployed at this stage will be quite unlike any that we find here on earth, since sending or building a \$10B semiconductor plant would not likely be the best way to go.
Much later in the development of the system, that might be feasible, but not yet.

\section{Logical Truth and Truth Conditional Semantics}

The architecture of knowledge which I propose is layered.
The lowest layer is one of logical truths, on which all else is built.

In talking of logical truth we are concerned with \emph{propositional}, or \emph{declaative} language, which may be used to communicate about various subject matters.
The subject matters which may be addressed are called the domain of discourse of the language.
A propositional language will provide expressions which may be used to refer to things in its domain of discourse, and formulae or sentences which express propositions about those things.
A proposition is something which has a truth value depending on exactly how things are in the domain of discourse.
The conditions under which a sentence is true are called its \emph{truth conditions}, which are (perhaps implicitly) known to anyone who understands the language.
A sentence can therefore be used to convey information about the domain of discourse between people who have a common understanding of the truth conditions of the sentences.
The assertion by one party of a sentence communicates to other parties that the domain of discourse satisfies the conditions for truth of the sentence.

We use the term \emph{semantics} to refer to this aspect of the meaning of a language.
For the purposes of logic, it is the truth conditions only which concern us, but these attach only to sentences in the language.
Other expressions in the language which refer, for example to entities in the domain of discourse, have their own characterisation in the semantics which determines their role in setlling the truth of the sentences in which they occur.

Logical truth is a semantic notion, it is a property which attaches to certain sentences in languages with well defined truth conditions.
Truth conditions are a part of the meaning of sentences in a propositional language, sentences which express propositions.

\subsection{defining logical truth}

By \emph{logical}\index{logical} truths I mean those truths which follow deductively from the definitions of the concepts which appear in them.
Reasoning is deductive if the inferences involved are justifiable by reference only to certain aspects of the meaning of the sentences involved in the inferences.

The concept of logical truth and the above characterisation of it are controversial.
Much of the controversy is verbal, i.e. it is about the choice of language and the meaning of the terms.
Most of the technical claims that will be made below are not controversial, though once again the choice of language in which they are expressed may be.

An epicenter of opposition to the position which I here adopt relates to the thesis of \emph{logicism}, a position held by Gottlob Frege, Bertrand Russell and Rudolf Carnap, and arguably attributable retrospectively to David Hume (though not identically expressed by each), the essence of which is that the truths of mathematics are logical truths.
Despite the eminence of all these philosophers, the thesis has been regarded by the majority of philosophers since the middle of the 20th Century as discredited.

The simplest explanation for this is that the concept of ``logic'' became narrower durubg the first half of the 20th Century as the discipline of Mathematical Logic became established.
Early in this period problems arose in the status of abstract ontology, aspects of which at first had appeared uncontroversial and logical in character, but turned out to be perilous, and to some degree arbitrary.
This was first exposed, in the context of the establishment of logicism, by Russell's observation that Frege's formalisation of arithmetic fell foul of what came to be called ``Russell's paradox'' (though it, or very closely related problems,  was already known to Cantor) and could not therefore be relied upon to prove only true statements.

The resolution of Russell's paradox depended upon constraining the principles governing what abstract entities exist, and there seemed to be no wholly convincing argument to determine which restrictions should be adopted.
Seemingly arbitrary choices were needed to establish a coherent ontology.
It was therefore natural to doubt that the ontological principles required for the definition of number and the derivation of the truths of arithmetic could have the character of necessity expected of logical truths.
Against this is may be observed that any language in which arithmetic can be expressed makes use of arbitrarily chosen symbols for the logical and arithmetic operations whic appear in the sentences of arithmetic.
Langauge is conventional, and this is no bar to its ability to express logical truths.

Abstract ontology, by contrast with concrete ontology, corresponds to no observable feature of the world about us.
Some philosophers take the view that abstract entities are mere fictions, and that would provide an adequate basis for the matters discussed here.
However, fictions are usually the kind of thing which one would normally be able to detect empirically, but in fact do not exist, whereas abstract entities are not of that kind, the claim that they exist is not empirically testable.
Questions involving abstract entities only become meaningfull in a context which settles the abstract ontology, amd their resolution depends on reference to that context.

This attitude towards the status abstract entities is similar to that of Rudolf Carnap, who regarded ontological questions as being of two kinds.
The first kind are those which are taken in the course of defining some language, and subject to requirements of consistency are as freely chosen as any other feature od the language (though not without consequences for the utility of the resulting language),
The second kind are those which are posed in the language thus defined, and are to be resolved by reference to the choices made in designing the language, ideally aided by a reasonably complete formal deductive system which supports reasoning within the chosen domains of discourse.

\subsection{basic layer continued}

This most basic layer is also the layer which determines the \emph{abstract semantics} of the vocabulary for this and all other layers in the structure.
Abstract semantics is a technical term related to the intended meaning of the linguistic structures it relates to, and contains sufficient information to determine whether

Before giving more detail about the structure of this first layer, its necessary to explain the concept of logical truth as it is intended in this architecture.
For this purpose a bit of history may be helpful.

\subsection{Some History}

The first known systematic use of deductive reasoning was its use by the philosophers of ancient Greece for the development of mathematics and most successfully in Euclidean Geometry.
At the same time philosophers sought to understand nature and the cosmos using reason and found it to be unreliable and incapable of securing concensus in these matters.
The great philosophical systems of Plato and Aristotle attempted to resolve this descrpancy and promoting and underpinning  the use of reason more broadly than had hitherto succeeded.
In doing so they made the first steps toward identifying the kind of ``logical truth'' whicxh concerns us here.

The first approach to this came in Plato's distinction between his world of abstract forms and that of mere appearances.
The former domain was one which could successfully be comprehended by reason, in whicha domain reason could yield conclusive knowledge.
The world of appearances, of which we learn through our senses, cannot yield certain knowledge, but only mere opinion.
This is a division by subject matter, abstract versus concrete, and by method of enquiry, by reason and by observation, which also distinguishes the degree of trust which can be assigned to the resulting conclusion (certain knowledge versus subjective opinion.
The former domain corresponds to the notion of logical truth as it is intended here, but fails to give a sufficiently precise characterisation of the domain for our purposem.

Aristotle's philosophy sought to resolve some of the difficulties in Plato's ideas, and included six volumes on logic in which an important aim was to establish the notion of \emph{demonstrative science} and the deductive methods appropriate to it.
Many important logical concepts which are closely related to that of logical truth, such as the distinction between essential and accidental predication, that between necessary and contingent propositions, and the notion of demonstrative propositions.
Aristotle's theory of categories provides a way of distinguishing the kind of conceptual inclusions which arise in a taxonomical heirarchy, and hence may be thought essential or to do with meaning, from those between concepts related only by accident.
When we come to Hume, this distinction shifts more decisively away from doubtful metaphysical interpretation to one more easily understood to be concerned with meanings.

Much much later it is the articulation by David Hume of what has been called ``Hume's fork'' that provides the first unproblematic (if still lacking in precision) description of the distinction which is here important.
In a central place in the condensation of his philosophical position as his ``An Enquiry Concerning Human Understanding'' \cite{hume48} Hume speaks of the two kinds of proposition in the following terms:

\begin{quote}
``ALL the objects of human reason or enquiry may naturally be divided
  into two kinds, to wit, Relations of Ideas, and Matters of Fact.'' 
\end{quote}

The first kind, which we call here the truths of logic, he further describes thus:

\begin{quote}
``Of the first kind are the sciences of Geometry, Algebra, and
Arithmetic; and in short, every affirmation which is either
intuitively or demonstratively certain.
That the square of the hypotenuse is equal to the square of the two
sides, is a proposition which expresses a relation between these
figures.
That three times five is equal to the half of thirty, expresses a
relation between these numbers.
Propositions of this kind are discoverable by the mere operation of
thought, without dependence on what is anywhere existent in the
universe.
Though there never were a circle or triangle in nature, the truths
demonstrated by Euclid would for ever retain their certainty and
evidence.''
\end{quote}

I consider the first accurate characterisation of the domain of logical truth, of which are much more detailed technical definition will follow.
You may come to you own opinion once that precise delimitation is in place, though the strength of the correspondence is not essential to the resulting episteme.

Emmanual Kant, who credited Hume for waking him up from a dogmatic slumber disagreed with Hume's characterisation of this split, and specifically disagreed that mathematics belongs in Hume's first classm, introducing a new use for the technical term `analytic'.
The term `analytic' sounds rather like Hume's relation bewteen ideas, but Kant denied that the truths of arithmetic are analyic.
This denial was to prove a spur to the demonstration of the analyticty of arithmetic, the results of which provide the basis for the bottom layer of this architecture.

To give precision to the concept of logical truth (later to be identified with analyticity)  we must look primarily to the mathematicians of the century which followed Hume.

\subsection{Mathematical Logic}

Though mathematics had progressed as a deductive science since the ancient Greeks, the gold standard of rigour as exemplified by Euclidean geometry had not been maintained.
The shortfall had been particularly marked since the invention of the differential and integral calculus by Newton and Leibniz, which had precipitated a huge expansion of mathematics applicable to science and engineering despite a lack of clarity about some of the key concepts involved, notably that of infinitesimal numbers.

In the nineteenth century mathematicians began a process of rigourisation of analysis to put that discipline built on the ideas of Newton and Leibniz on a solid footing.
The first stage in this was to eliminate the use of infinitesimals, using the notion of limit to achieve the same ends.
This required a system of numbers in which limits exist for every convergent series, these were to be called `real' numbers.
Also needed was a domain of mathematical functions which included all the functions likely to arise in this expanding new branch of mathematics.
These issues were resolved by developing \emph{set theory}.
With the aid of set theory the rigourisation of analysis was achieved by arithmetisation, its reduction to the theory of whole numbers.

For the purposes of mathematics that probably would have sufficed, but Gottlob Frege had a bee in his bonnet about Immanual Kant's take on the status of Mathematics, which he claimed did not belong to logic, or in his special terminology, was not \emph{analytic}.
Frege set about demonstrating that ``Arithmetic is Analytic'', one formulation of a thesis which later came to be called ``logicism'', and was often rendered ``Mathematics is (or is reducible to) logic''.
To demonstrate this Frege devised a new formal logical notation which he called \emph{Begriffsschrift}\index{Begriffsschrift} (concept script).

Though intended for the rigorous formal derivation of the truths of arithmetic, Frege did not (as his title suggests) think it confined to that purpose, considering it generally appicable to rigorous formal reasoning.
His description of this notation was published in 1879 \cite{frege1879}, and represented the first major advamce in formal logic since Aristotle formulated his syllogistic.
The limitations of Aristotle's syllogistic were entirely overcome by this new notation, which was as Frege hoped suitable generally for the formalisation of deductive reasoning.

\subsection{Deductive Closure and Consistency}

It is reasonable to expect that the very large scale deductive inference which would be feasible in galactic scale distributed synthetic intelligence will give effective access to the najor part of the deductive closure of the knowledge in its possession.
Consequently it is highly desirable that those principles be logically coherent, otherwise that deductive closure will contain all propositions, true or false.

For this reason the principal method of augmenting the logical layer of our knowledge base will be ny conservative extension, and careful tracke will be kept of non-conservative elements and the conclusions drawn from them.

This remains the case as we ascent to higher layers in the architecture and consider the representation of empirical theories.
It is therefore intended that empirical theories will be couched as defined mathematical theories from which various applicable instances may be deduced.

\subsection{Logical Truth as General Database}

All ways of storing information require some way of referring to that information.
Typically that consists of names or positions, either of which can be interpreted in a logical system constraining heirarchical names.
We do not speak here of how the data is actually stored, just of how it is interprted logically for the purposes of inference.
The computations which one might undertake on these data structures yield logical truths expressing the result of the computation.


\section{Empirical Truth}

The second level of our architecture is concerned with the representation of empirical truth.
For the purpose


\subsection{Factorisation of Semantics}

Sentences expressing empirical propositions have as their subject matter the concrete world rather than abstract entities, though they may nevertheless refer to abstract entities (such as numbers) in the course of expressing an empirical claim.

Physical theories are commonly expressed in mathematical form, and by virtue of their form may be considered to offer an abstract (mathematical) model of a concrete phenomenon.
Reasoning about concrete reality may then be accomplished by logical reasoning about the abstract entities of the mathematical model, considering these entities to have a concrete as well as an abstract interpretation.

If we seek a truth conditional semantics for a language which is intended to speak of the concrete world, it is therefore possible to make use of an abstract model of the concrete phenomemon which would be appropriate for a mathematical model of the astract phenomenon.
The correspondence between the concrete entities concerned and their abstract representatives provides a factorisation of the truth conditional semantics, effectively defining the truth conditions of empirical sentences as those of its abstract representative.

This makes it possible and intelligible to use a purely logical substrate to represent empirical knowledge, provided that it is supplemented by an adequate description of the correspondence between the abstract and concrete entities.
In an axiomatic description of a physical system the physical laws would be presented as axioms.

\ignore{
\begin{quote}
``Matters of fact, which are the second objects of human reason, are not ascertained in the same manner; nor is our evidence of their truth, however great, of a like nature with the foregoing. The contrary of every matter of fact is still possible; because it can never imply a contradiction, and is conceived by the mind with the same facility and distinctness, as if ever so conformable to reality. That the sun will not rise to-morrow is no less intelligible a proposition, and implies no more contradiction than the affirmation, that it will rise. We should in vain, therefore, attempt to demonstrate its falsehood. Were it demonstratively false, it would imply a contradiction, and could never be distinctly conceived by the mind.''
\end{quote}
}%ignore



\chapter{Epistemic Futures}

\section{Introduction}

This is a story about \emph{knowledge}, how it has evolved, how it is, and how it may develop.

In telling that story I connect a broad conception of knowledge and its variety with a number of other related concepts.
The first of these concepts is that of \emph{evolution}, understood as a progressive aggregation of knowledge -  chemical, biological, cultural and technnological.
A second is the notion of \emph{rationality}, closely related both to knowledge and to evolution.
Evolution in all its forms is viewed as aggregating knowledge, and in that way exhibiting both instrumental and epistemic rationality.
As evolution evolves so does the rationality which it exemplifies.

The aggregation of knowledge has predominantly been a social activity, and as such depends upon means of communication, which have themselves evolved continuously.
The evolution of language leads to new forms of knowledge and to oral culture.
The development of written forms better preserves knowledge across generations and spreads it far and wide.
The effectiveness of social groups depends upon an appropriate level of cooperation (while also benefiting from competition) and mechanisms of social cohesion may lead to the apparently irrational behaviours sometimes deprecated as tribal or cultish.

\subsection{The structure}

This work is divided into three parts correponding to historical stages in the evolution of homo-sapiens.
The first considers the period up the apperance of anatomically modern homo sapiens, encompassing a period of pre-biotic chemical evolution, a period of biological evolution effected by undirected variation and `natural' selection, and a period of evolution in which sexual selection became an important factor in determining the direction of evolution.

Anatomically modern homo sapiens appears around 200,000 years ago, and at roughly this time human development due to biological evolution, though continuing, is overtaken by and transformed by the more rapid changes arising from cultural evolution.
This period of cultural development and culturally influenced biological evolution is the subject of the second part, which brings the narrative up to the present day.

The science and technology which eventally flourished as a part of that cultural evolution has now brought us to a point at which we may expect to see the largest ever transformation to the process of evolution, as synthetic biology and informaton technology together permit us to take into our hands the design of future generations of the ecosphere, including the evolution of homo and the engineering of `intelligent' machinery and artificial life forms.
The core imperative of evolution is proliferation, and the third part of this essay concerns what and how we may now begin to proliferate across the cosmos.

The whole is presented as an evolutionary story in which there feature many different kinds of evolution.
As well as distinguishing between the various kinds of evolution involved, I offer an epistemic characterisation of evolution as a whole, in which the progress which we may hope to see from evolution lies in the progressive accumulation of knowledge.

\subsection{Rationality as Focal Point}

I'd like to sketch out here a contemporary issue around which this whole work revolves.
This concerns the contrast between human rationality and its apparent converse exhibited in some radical ideologies.

The seminal notion of rationality here is that which has been called \emph{instrumental} rationality, characterised by the adoption of means to ends which are most likely to realise those ends.
\emph{Epistemic} rationality, by contrast, concerns the relationship between evidence and belief, and is exhibited by those whose beliefs are those consistent with the evidence on which they are based.
Epistemic rationalty may be seen as derivable from ir entailed by instrumental rationality since true belief is generally instrumental and false belief counterproductive.

It is tempting to associate the social activism which has become increasingly forceful in these earty decades of the 21st Century with \emph{irrationality} for a number of reasons.
The first is the explicit rejection of rationality, as a weapon of opression by Western colonialists, supported by the scepticism and relativism which was brought into Critical Theory under the influence of `Postmodern' thinkers.
There are many other aspects of applied critical theory however, which seem to fly in the face of previously accepted rational standards.

Against this we may note the difficulty in making judgements about irrationality where the motives of the agents are not clearly understood, perhaps not even by them.
We must surely ask and hope to understand \emph{why} anyone should reject rationality.
If the motive is understood, the rejection of rationality (possibly only in transition) might then appear rational.

Steven Pinker has argued that arguments against rationality are self-defeating, for the presentation of arguments, he supposes, concedes the relevance of arguments and thus of rationality.
He neglects the rational, indeed deductive, method of \emph{reductio absurdum}, in which one begins a proof by assuming the negation of the proposition to be proven, and also the fallacy of \emph{petitio principii} which is involved in any rational attempt to establish a principle of rationality.

Our context lacks the prerequisites for demonstrative reasoning, we must work with less formality, and less assurance.
A defence of our tradition, even qualified, against fundamental ideological challenges, must not only reinforce our reasons for adopting the methods we use, and our most fundamental system of beliefs, but also must seek to understand and counter-challenge the motives which seeded the conflict and which facilitated its proliferation.

\subsection{An Evolutionary Preview}

Life on earth has evolved.

There have been many changes to the mechanisms involved, but three major transitions which deserve special attention, and one which is now in progress.

Those transitions were:

\begin{description}
\item[life:] The transition from chemistry to biology (4Ba).
\item[sexual:] The transition to sexual reproduction/selection (2Ba).
\item[cultural:] The beginning of culture and its evolution (200ka).
\item[technological:] Will synthetic biology kill natural selection?
\end{description}

In this preliminary discussion I will talk about why these transitions are of importance for the evolutionary process, and why an understanding of the transitions and of the different kinds of evolution they introduce may be worthwhile.

I look for the development of rationality and of those mechanisms of social behaviour which have the power to suspend rationality and secure behaviours which may have no survival or reproductive advantage beyond mere conformance with a social norm which itself has no merit.
These latter I will talk of under the term ``social behaviours'', taking in the first instance a very broad view of social behaviours as pre-cursors to the present day phenomena of interest.

In considering rationality and the relevant social behaviours I will take them as occurring at many levels.

Insofar as the rationality of concern in the first interest is \emph{instrumental}, and taking the purpose at hand to be determined in detail by context but in general as the aim to construct organisms which are successful in self-replication (in a particular ecological and social context), I suggest that evolution itself is \emph{rational}, it realises that purpose.
This is a bit like thinking of evolution as a ``blind watchmaker'', as achieving needed effects which would otherwise require intelligent design.
There are many different kinds of evolution which we will consider, and they do differ in the credibility of such alleged rationality.

Secondly, we may consider that the \emph{results} of the evolutionary process are predominantly rational, i.e. that they are effective in enabling organisms to replicate in some suitable niche.
This attribution is applicable in the first instance and in the most primitive organisms to capabilities and behaviours which are rigidly programmed by the genes of the organism, and thus does not involve anything which we might regard as rational deliberation.

A first step toward such deliberation is the evolution of behaviour which is more flexible, and allows the organism to succeed by adopting varying ways of realising some important end according to circumstances, or which allow the organism to proliferate in a wider range of environments.

\subsubsection{The Genesis of Life}

Direct evidence of life on earth dates back about 3.5 billion years.
Life appeared, it is generally supposed, as a result of a period of ``chemical evolution'' the nature and course of which is not well understood.



Before life evolved evolution was chemical, resulting in the construction of ever more complex molecules and chemical environments gradually more suitable for the support of biological organisms and the evolution of species.

All life on earth shares the characteristic that it consists of organisms which under certain circumstances, in a certain kind of environmental niche, are capable of proliferation, of self-reproduction.
It also shares more specific characteristics which may not be essential to that, such as the use of DNA to mediate in the reproductive process and to practically codify the structures and processes necessary to the life and proliferation of the organism.

The encoding of the structure of the organism in DNA has a profound effect on the process of evolution, and represents the transition of particular interest here, though it may not precisely align with that between inanimate and organic structures, which will depend on a precise definition of `life' which we have not and will not venture here.

It is at this point that evolution can be thought of as evolution of species by `natural selection' as described by Darwin, in which natural conditions select those organisms which survive to reproduce, and thereby gradually evolve the genetic and phenotypical characteristics of population.

\subsubsection{The Merits of Sexual Reproduction}

Important milestones in evolution are often the combined effect of multiple advancements.
Sexual reproduction may be seen in that way.

The earliest known life forms on earth were \emph{prokaryotes}.
Prokaryotes are single cells which do not have a cell nucleus, and reproduce asexually by cell division, a process which usually forns two cells genetically identical to the original.
Without other genetic innovation, the variation on which evolution depends would occur during the copying of the original genome, in default of which the progeny would be identical to the parent.

The evolution of features which required multiple genetic changes would only be possible if all those changes occurred on one line of descent, and they would only become

\subsubsection{Cultural Evolution and Selection}

\subsubsection{Synthetic Evolution}
  
\subsection{Concepts and Vocabulary}

Evolution is the unifying concept under which I discuss the development of certain phenomena of interest over extended periods of time.
Tracing back through the history may help us to understand the phenomena as they appear today and thence anticipate and accomodate their development in the future.

Some of these phenomena, for example language, culture and rationality may be thought exclusively human.
It may nevertheless be helpful to consider from what prior capabilities those human facilities evolved, and how that could have happened.
In doing this, terminology is desirable which reflects the connection with the fully fledged facility as seen in man while maintaining the distinction.

Often qualification will be a good way to do this.
Thus we may speak of the kind of culture whose inception occurs at about the same time as oral language as ``oral culture'' and speak also of the tool making skills passed from one generation of homo erectus to the next as part of a ``pre-lingual culture'', thus facilitating clarity by terminological fiat while side-stepping debates about the precise boundaries of established concepts.

Sometimes the important predecessors are not within reach by that method.
The concept of ``language'' exemplifies the problem.
There is, before any special terminology is attempted, a variation in usage, between professional academics and others and between academics in different disciplines.
Linguists may insist that languages have recursive generative grammers allowing infinitely many sentences of unbounded length and complexity, but others will use the word more liberally.
Nevertheless, when we trace back 

In viewing the whole evolutionary process as a progressive accumulation of knowledge, a great variety of kinds of knowledge are encompassed.
We may begin with the idea that an autocatalytic set embodies knowledge of how certain chemicals can be synthesised, and followed by the encoding in various forms (RNA, DNA) of knowledge about how to build particular proteins and their relevance to the organisms in which the codings are found.

Rationality, thought by Aristotle to be peculiarly human, may similarly be traced back to the origins of life and beyond by analogies based first on the instrumental effectiveness of evolution in realising organisms well adapted to proliferation in their own ecological niche.


In considering these pre-human characteristics it is somtimes natural to extend the scope of the existing concept.
Though some linguists will insist that only humans have language, it is not uncommon to hear the term extended more broadly.
The important distinctions which remain between language in humans and the ``language'' of birds or dolphins, can be preserved and made precise by appropriate qualifications.
If an exclusive feature of human languages is their recursive generative grammars, then perhaps we could call that type of language a recursive language?

Consider the concept of \emph{rationality}.
In its normal use this is thought to be an exclusively human characteristic.

This work flows from a concern about rationality and its contrary, their contributions to our present predicament and prospects.
Intimately bound up in this concern is that for knowledge, the theory of which, epistemology, yields the title of the work.

The method is historical, an exploration of the history leading to the present and the future which may lie beyond.
In looking back for an understanding of the past which will help us understand the present and shape the future, precursors may be instructive.
To understand rationality, it is useful to look back to those related phenomenon which precede rationality.
In distinguishing rationality from its precursors we are limited by the imprecision of our language, in which usage is diverse and boundaries indeterminate.
The approach adopted here is to qualify the concepts, giving us a label for the differning manifestations at each stage in the development.
Which of these stages is to be considered a precursor, and which a fully fledged variant, may then be academic.

Some illustrations may be helpful.
Let us consider language.
Some linguists insist that a bona fide \emph{language} must have a recursive grammar.
A language is a way of transmitting information.
To understand the origin of languages we may consider the function of communication as fundamental to language and consider what means of communication preceded languages.
Another expectation of languages is that they are symbolisms, 

\subsubsection{Knowledge}

\subsubsection{Evolution}

Evolution is a change, not necessarily without interruption.


\chapter{Some Scientific Scepticism}

\section{Preface}

I am not a scientist.
Professionally, I was a software engineer, i.e. I was primarily engaged throughtout my career in one or another aspect of the development of computer sofware.

I have a first class joint honours degree in Mathematics and Philosophy, from an English University not particularly distintinguished in either field.
I earlier spent one year studying mechanical sciences at Churchill College, Cambridge.
I have since then spent many decades learning a lot of other stuff, and continue a life in which I am constantly learning, usually with some purpose in mind.

I am at present trying to write something about rationality and its evolution, and in the course of these attempts am often reminded of some of the points at which I doubt received scientific truths, wondering how to deal with these doubts.
Alonside my persistent scepticisms, there is a continuous process, as I read into the various sources apparently relevant to my enterprise, of assessing the credibility of the various sources.
It makes no difference how well or ill qualified one might be to make such judgements, there is no escaping those decisions about what is worth reading, how seriously to take what it says, and how that can be fitted into a coherent and credible conception of how the world is, has been and will be.

The final provocation to write on this was a foray into the earliest history of the Universe, trying to decide whether in a meaningful way this period might fit into a broadly conceived evolutionary story.
This naturally brought me face-to-face with some of the most tenuous theories of physics and cosmology, and has made me feel the need to come clean about how I think in these areas.

This essay is my attempt to do that.

\section{Some Autobiographical Background}

Philosophy should be concerned, among other things, with the big picture.
Of late I have been trying to understand everything, philosophically, through the lense of evolution.
This essay spans almost the whole of evolution (from a particular vantage point in space time), from the beginnings of life on earth to its proliferation across the galaxy beyond its likely local demise.


This work is not intended to be autobiographical.
It may nevertheless be helpful to the reader to know something about my own person and life while considering whether to read the thoughts herein, if only in saving him from the attempt.
The preface is dedicated to that purpose, the body of the work excluded from it.

I first note my lack of credentials.
In formal education I progressed only so far as a BA in Mathematics and Philosophy, from a University not pre-eminent in either of those subjects.
Professionally I have been engaged primarily in software engineering, with minor excursions toward digital hardware development.

Though later I came to think of philosophy as an unrealisable dystopian vocation, philosophy barely entered into my life until my early twenties, the BA completed only by the age of 28.
Before that, my only recollection of philosophical thinking was of my reflections on sermons endured during my first year at grammar school.
Those reflections made me an atheist at the age of 12, with no further interest in debating the existence of god.

In school my poor memory ensured mediocre performance until I was able to specialise in the sixth form almost exclusively on mathematics and physics.
Accumulating facts I disdained.
Any temptation to attribute poor memory to disinterest or indolence would later be dispelled by acquaintance with effortless recall in others beyond my wildest aspiration.

That shortfall of memory may be the most important of the reasons why I could never have been a scholar.
Others include my antipathy to continuing in the study of an author in whose thought I perceived fatal errors, and the ease and frequency with which I arrive at that diagnosis.
I also read very slowly, and absorb only those parts which can be integrated into my world view (though not necessarily as true, and not necessarily without transformation).
The slowness of my read arises in part from the mental digressions, in default of which I consider a work uninteresting.
That critical dismemberment demands less talent than creative synthesis is a truism amply illustrated by academic philosophy.

\footnote{There may be ``hyperlinks'' in the PDF version of this monograph which either link to another point in the document  (if coloured blue) or to an internet resource  (if coloured red) giving direct access to the materials referred to (e.g. a Youtube video) if the document is read using some internet connected device.
  Important links also appear explicitly in the bibiography.}

\subsection{My Scepticism}

I think myself a sceptic.
Not a radical sceptic holding that no knowledge is possible, but one sceptical about much that is commonly considered authoritative.
Nevertheless, an advocate for rationality, taking scepticism to be an integral part of rationality.
There is no simple story about how my scepticism compares with, say, a typical academic philosopher.
Sure enough I will doubt much that is held by others, but also accept a great deal which is rejected.

I have been shocked by the irrationality of institutions which might have been expected to exemplify rationality, most notably by philosophy in what was for some time called the ``analytic'' tradition.
This has caused me puzzlement.
Certain other philosophical time-lines, notably Marxism reaching ultimately into activist tropes on social media and our streets through the Critical Theory of the Frankfurt school, augmented (or empovershed) by Post-modern scepticism, seem to have abandoned and abjured rationality, and stand testimonial to the power of contrary tendencies.
The apparent attempt to put ideology beyond reach of reasonable discussion is also puzzling (thugh not of course without ample precedent), and concerning.

My response to the puzzlement and concern has been to try and understand, to dig and probe in search of comprehension, in the hope that a contribution to understanding might be possible and helpful.
Why is it that rationality, to which surely is due credit for the mastery of man over his environment and the resulting material prosperity, seems so scarce, and becomes a target for extirpation?

Since my work here depends upon my scepticism, it may be helpful to leverage those doubts in making intelligible my approach and methods, and thence their cautious conclusions.

Though my scepticism has been curated over a lifetime, there have been three particular episodes which have seemed to me of importance and which have influenced the direction of my enquiiry.

These three episodes are, in turn, my deliberations about the incoherence of God, about the Quinean refutation of Logical Positivism and its influence, and my belated acquaintance with the trajectory of Marxist and post-Marxist thought and its 21st Century conscription of the concept of Justice via a complete shredding of any recognisable standard of rationality.

I propose to begin here with a brief account of the principle stages in my disillutionment, and the directions in which they have influenced my enquiries.

The development of my own sceptical thought is more complex and more continuous than the story I will tell here, but there have been four particular episodes which are significant to the development of the main ideas presented in this work, which have given direction to the ``research'' invloved.

The first of these concerns God, whose existence beame problematic for me in my first year at Grammar School.
I went as a boarder, and was obliged to attend a church service every Sunday, at which I therefore was subjected to sermons.
Naturally I tried to understand them, and, not distracted by lesser concenns I tried to understand what God could be.
Evidently not content to be told that he was mysterious and beyond our comprehension.
The difficulty was not so much that of fully comprehending God, as in understanding how anything whatever might have the attributes ascribed to him.
I don;t recall how far I got through the school year before I gave up on this enterprise, but I do remember the end-game.

At the end, the last puzzlement, and perhaps the greatest one, which engaged me before concluding that God did not exist, was that of comprehending how so many important, respected and distinguished men (and women) could believe in the existence of God, if, as I then strongly suspected, no such being existed.
The first memorable stage in the development of my scepticism was the realisation that I could not believe everything I was told even by those who might have been supposed most authoritative.

From that point on, religion, and most especially denate about the existence of God, had no interest for me.
Though I had then no knowledge of logic, my belief then was as close as it could have been to my present view that the concept of God is incoherent, and that no argument for God would be worth considering unless is rested on a definition which fell short of that conception of deity upon which the Christian church is founded.

There followed from this brief episode of spontaneous scepticism little of present interest for many years.
I was an indifferent pupil in most subjects, mostly due dysfunctional memory and minimal effort.
Only toward the end of my fourth year (after a change of teacher) did I rise near the top of the maths class, eventually spending my two years in the sixth form doing double-maths physics to the exclusion of any trace of the humanities.
Before I came to my next sceptical episode, which concerns academic philosophy, I must first have become acquainted with philosophy, of which I was as I left school quite innocent.
My curtailed spell at University undertaking the first year of the Mechanical Sciences tripod, did nothing to improve that situation, unless you count some deliberation on artificial intelligence engendered by my unrestricted access to an IBM 1130 during that year.
The next five years began my career in Software Engineering, primarily involving the implementation of software support for high level programming languages, and it was in my own time that I then broadened my horizons and read a little more widely.

The principle influences on my approach to the Western tradition in philosophy were Bertrand Russell and Logical Positivism, the latter refined later to focus on Rudolf Carnap.
I read Russell's autobiography and his ``History of Western Philosophy'', among the very few books I have ever read more than once.
For a youngster well soaked in mathematics, fascinated by electronic computers and slowly learning about formal logic, the achievement of ``Principia Mathematica'' cast a spell, as it had for many of Russell's most gifted contemporaries and many others since.
Before I got near undergraduate philosophy ``Principia Mathematica to *56'' \cite{russell1970}, the paperback edition of the first parts of the three volume megalith, had secured its place on my shelf, conferring upon me only the barest comprehension of what lay between its covers.
Beyond the formal mathematics, into the philosophy, A.J.Ayer's ``Language Truth and Logic'' \cite{ayer1936} was for me an intelligible account of a palatable attitude to philosophical obscurity which played a large part in shaping the perspective from which I observed my undergraduate instruction in philosophy.

Though I was then barely aware of the source of my own perspective, the fundamental (if disputed) distinctions between truths of logic,  empirical facts and value judgements were, and have remained for me, a bedrock on which all else rests.
Thus was prepared the ground for my scepticism of academic philosophy.
One famous paper was the seed from which my utter disillusion with even analytic philosophy as a rational pursuit ultimately sprang.
That paper was ``Two Dogmas of Empiricism'' by W.V.O. Quine \cite{quine51}, a paper published mid 20th century with the aim and effect of defeating ``Logical Positivism'' as school, and sidelining its most distinguished progenitor Rudolf Carnap.

That I perceived the paper as without merit (for reasons I will pass over as yet) on first acquaintance was not the source of my dismay, for I was not then aware of the impact that the paper had exerted, not only carrying the day against its targets but largely inhibiting further consideration of its central thesis for perhaps half century.
Whether I was correct in my assessment of the paper, my perception of its effects, which supposedly rendered meaningless the most fundamental distinction in the foundations of rational thought.
That distinction, between logical and empirical truth, had evolved through a convoluted 2000 years of Western philosophy and had recently been brought almost to perfection by substantial fundamental advances in the foundations of logic.

In considering a response to this major breach odf rationality among those of whom it might have least been expected (analytic philosophers), it seemed to me clear that a rational response, a logical dissection of the defects of the arguments, could not succeed, and I instead became interested in underpinning the decried distinctions by studying their history over the past 2000 years and perhaps showing how the understanding which could be articulated on the basis of the most modern advances in logic and its philosophy was a worthy culmination of those thousand years of technical and philosophical travail.

That thread of interest persisted for many years without significant issue, and was accompanied by a disdain for contemporary analytic philosophy which was just as complete as my failure to engage with that other kind of philosophy which was once called ``continental philosophy''.
It was only very recently I became aware of the enormous practical effects that some aspects of ``continental philosophy'' were effecting.
The scepticism injected into the Marxist derived ``Critical Theory'' by postmodern philosophy undermined any propspect of rational discourse on matters of great public interest, and seemed to call for a more fundamental defence than could be afforded by a study of the history of philosophy.


\section{Introduction}

Should this essay approach completion then it might address the following areas:

\begin{description}
\item{\bf Entropy}
  The second law of thermodynamics, asserting that in any closed system or the universe as a whole, entropy can never decrease, has spawned a whole variety of alternative conveptions of entropy.
  A prevalent generalisation is that entropy is a measure of disorder, and that the second law entails increasing disorder in the Universe culminating in ``heat death''.
  While aknowledging the value of the concept of entropy in understanding the workings of steam engines, I am pretty sceptical about most else.
\item{\bf Special Relativity}
  The usual explanations of special relativity are incoherent and the evidential basis supporting the theory seems to me unconvincing.
  I can't say I am convinced of the theories falsity, but I'm certainly not convinced of its truth.
\item{\bf General relativity}
  Its a neat trick of Einstein to incorporate gravity into the structure of space-time,  a possibility which arises from the fact that gravity is the only force for which the field determines the acceleration of objects without reference to any of the propoerties such as their mass.
  This may also make the formulation of the theory mathematically more elegant.
  But elegance aside, I have never heard a rationale for not considering the forces of gravity in the same way as all the other forces.
  What this does do is to obscure, at least in lay discourse, the real differences between relativistic and newtonian physics.
  The discussion usually begins with the geometric innovation and never progresses beyond it.
  This is like chosing a new language to talk about some scientific speciality, and never getting beyond talking about the language into discussing the theories expressed in the language.
  Most of the innovations beyond that do not really impact the core elements of Newtoninan theory, viz the laws of motion and of gravity.
  \item{\bf The Expansion of the Universe}
    This theory and the big bang theory which flows from it depends upon the explanation for the plausible correlation between red shift and distance (as due to recession velocity).
    Things get wore when we start talking about the acceleration of the expansion, more so when dark matter is offered as an explanation for the expansion and its acceleration, and reach another plane when we come to the idea that the space which is occupied expands rhather than the universe itself.
  \item{\bf evolution}
    I'm pretty OK with the idea of the evolution of species through natural selection (with some cavils).
    Beyond that thesis the idea that evolution is a ``theory'' of a kind not too dissimilar to Newtons theory of gravitatino, is one which I think desrves and benefits from much sceptical analysis.
    \end{description}

\section{Entropy}

I have not a lot to say here.
I will give only a sketchy account of my negative views on entropy and the second law of thermodynamics, which I don't expect anyone to accept but feel that I nevertheless should mention them.

The shortest statement is as follows:
\begin{enumerate}

\item Though in certain contexts the original definition of entropy is clear and useful (in calculating the maximum work that can be obtained from heat), its subsequent generalisation yields an ill-defined concept.
\item The second law of thermodynamics is therefore meaningless.
  If one were to make this more clearly about heat (rather than about order), then the second law would most likely be false.
  \item The idea that the second law guarantees that the Universe is destined inevtably for some kind of heat death has no empirical basis, and any use of that idea in promulgating other theories (e.g. about evolution or the long term future of humanity) will in my opinion 
\end{enumerate}


\begin{description}
\item[Irreversible Processes]
  \item[]
  \item[]
  \item[]
  \item[]
  \item[]
\end{description}

\section{Thermodynamics}

\subsection{Irreversibility}
\ignore{
  Do the facts about entropy go beyond what is found in Newtonian physics or are they derivable from them?
  Is the Second Law a physical theory or a statistical one?
  How, if ever, does entropy increase?
  Are there absolutely irreversible processes?
}%ignore

    \subsection{}

\section{Special Relativity}

A superficial understanding of the claims of special relativity is easy to achieve.
A few simple formulae tell us about the claims of the theory in terms of strange laws for addition of velocities, time dilation, mass increases and the related conversion between mass and energy.
Making sense of it all, after beginning life with a newtonian understanding of the laws of motion and gravitation is not quite so straightforwaed, and this I have never achieved.

There are many elementary explantions which seek to make special relativity intelligible to the non-specialist, but these have never made any sense to me.
The main stumbling block is with the idea that the speed of light is the same relative to every intertial frame.
This seems to be presented as if it were an empirical observation, but it is in fact a logically incoherent claim in the context of the only concept of space and time which we have prior to special relativity.
It is simply not possible for it to be an experimental observation, though presumably the synthesis of a conception of space in which it is logically coherent could not have happened without experimental observations which could not otherwise be expained.

Certain experimemtal observations are cited to establish the thesis.
These observations are held to refute the two alternatives which are apparent under the prior understanding of space and time.
These are, that light travels in vacuo at a fixed speed relative to some medium (the ether) which may be considered to be at absolute rest, and the second is that it travels with a fixed speed relative to its source.

The evidence against the first alternative is the Michelson-Morely Experiment, which I won't go into because I don't have a problen with it.
The evidence against the second possibility is cited as certain observations of twin stars, and nby contrast with the Michelson-Morely experiment is not explained in elementary accounts.
A few years back I looked at the paper cited, and found its rationale completely unconvincing.

With those two alternatives disposed of we are then presented with Einstein's resolution to the difficulties, which is a major transformation of our conception of space and time.
This is huge, I'm sure that most would agree.
A leap which surely no-one would take without very solid evidence to establish that no more conventional explanation could suffice.
In my particular case, this solution is one which I have never fully understood.

That is my starting point for this exploration, in which I propose to look very closely at Einstein's own painstaking account for the lay person of his theory, with a view to arriving at a good understanding of what is claimed and of the evidence cited for it.

In doing this, I will find it necessary not only to look carefully at the stages in Einstein's explanations, but also to give an account of my prior understanding of space and time, so that the reason's for any difficulties I have with Einstein's alternative may be clearly exposed.

I'm going to review Einstein's account of special relativity and try to amke sense of it, and in the process explain any difficulties I have, as well as those which remain at the end.
The explanation of my own difficulties with this os likely to depend in part on my prior beliefs and understandings which seem to conflict with the theory, and which are not dispelled by Einstein's explanations.

There will be one of three outcomes.
I may still feel that I just don't understand the theory.
I may come to understand the theory, but remain unconvinced of its truth.
I might be convinced of its truth but not of its necessity.
I might be wholly convinced that it is mot only true, but the best account of the truth.

\subsection{Starting Point}

A principal aim of science is to enable prediction.
To achieve some end, it is helpful to construct a plan, the execution of which will lead to the desired outcome.
If you are confident that the execution of you plan is feasible and that it will yield the desired outcome then you can marshall and apply the resources required in the plan.

The ability to undertake this kind of goal oriented behaviour predates by hundreds of thousands of years the invention of science, but the purpose of science is to enable this kind of activity to be undertaken more successfully and on a larger (or smaller) scale.

Science facilitates the process by providing ``laws'' which describe regularities in the behaviour of the world which can be exploited in the construction and execution of such plans.
To articulate laws of nature we need language suitable for precisely describing the state if the universe, or some part of it, at one moment, and thence describe how the state evolves over time thenceforth.

To describe the state of the universe it is necessary to enumerate the things which occur in the system, describe carefully the nature and state of each object, and its place.

\chapter{Chatbots on The Evolution of Epistemology}

\section{by chatGPT}

Epistemology is the branch of philosophy concerned with the nature, origins, and limits of knowledge. It has a long and complex history that spans more than two millennia. Here's an outline of the major figures and ideas in the history of epistemology:

\begin{itemize}
\item[Ancient Greek Philosophy]: The earliest systematic explorations of epistemology can be found in the works of ancient Greek philosophers, such as Plato and Aristotle. Plato's theory of knowledge emphasized the idea that knowledge is innate, while Aristotle's theory of knowledge focused on the role of experience and empirical observation in the acquisition of knowledge.

\item[Medieval Philosophy]: In the Middle Ages, the dominant philosophical tradition was that of Scholasticism. Scholastic philosophers, such as Thomas Aquinas, sought to reconcile the teachings of Aristotle with Christian theology. They developed sophisticated theories of knowledge, including the idea that knowledge could be acquired through both reason and faith.

\item[Modern Philosophy]: The rise of modern science in the 17th century led to a renewed interest in epistemology. René Descartes, for example, sought to establish a foundation for knowledge that was based on reason alone. John Locke, on the other hand, argued that knowledge was derived from sensory experience.

\item[Kantian Epistemology]: Immanuel Kant's Critique of Pure Reason (1781) was a landmark work in the history of epistemology. Kant argued that knowledge is not simply a reflection of reality, but is constructed by the mind using innate structures of understanding. This led to the development of transcendental idealism, which holds that the mind plays an active role in shaping our experience of the world.

\item[Pragmatism]: In the late 19th century, American philosophers such as Charles Peirce and William James developed the philosophical tradition of Pragmatism. Pragmatism emphasizes the practical, experimental approach to knowledge acquisition, and holds that beliefs are true to the extent that they are useful in guiding our actions.

\item[Postmodernism]: In the late 20th century, some philosophers began to question the foundations of traditional epistemology. Postmodernists argued that knowledge is always subjective and context-dependent, and that there is no objective truth that exists independently of human experience.

\end{itemize}

Overall, the history of epistemology reflects an ongoing conversation about the nature and limits of human knowledge, and the ways in which we can acquire it.

\section{by Bard}

Epistemology, the philosophical study of the nature, origin, and limits of human knowledge, has a long and rich history. The ancient Greeks were the first to grapple with the fundamental questions of how we know what we know, and their ideas continue to be debated by philosophers today.

\begin{itemize}

  \item The pre-Socratic philosophers were primarily concerned with the nature of change and the possibility of knowledge. They argued that the world was constantly changing and that knowledge was therefore impossible. Heraclitus, for example, famously said that "everything flows" and that "you cannot step twice into the same river." Parmenides, on the other hand, argued that the world was unchanging and that knowledge was therefore possible. He said that "what exists, exists, and what does not exist, does not exist."

\item Plato was one of the first philosophers to systematically explore the nature of knowledge. He argued that knowledge is not derived from the senses, but rather from a higher realm of Forms. He also argued that there is a difference between knowledge and opinion, and that knowledge is always true.

\item Aristotle was another important figure in the history of epistemology. He argued that knowledge is derived from the senses, but that it is then processed by the mind. He also argued that there are different types of knowledge, such as scientific knowledge and practical knowledge.

\item The medieval philosophers continued to explore the nature of knowledge. Thomas Aquinas, for example, argued that knowledge comes from both the senses and the intellect. He also argued that there are different levels of knowledge, such as sense knowledge and intellectual knowledge.

\item The modern philosophers, such as René Descartes and David Hume, continued to explore the nature of knowledge. Descartes argued that knowledge begins with doubt and that we can only be certain of what we can clearly and distinctly perceive. Hume, on the other hand, argued that all knowledge is based on experience and that we can never be certain of anything beyond what we have experienced.
\end{itemize}

The history of epistemology is a long and complex one. It is a field that has been grappled with by some of the greatest minds in history, and it continues to be debated by philosophers today.

\section{Bing chat}

Epistemology has a long history within Western philosophy, beginning with the ancient Greeks and continuing to the present. It is one of the four main branches of philosophy, along with metaphysics, logic, and ethics1. The term “epistemology” was introduced into English by the Scottish philosopher James Frederick Ferrier (1808–1864)

The history of epistemology begins along with philosophy in Ancient Greece (1200 BC - 146 BC), especially with the works of the Perménides philosophers (5th century BC) and Plato (4th century BC)3. The central focus of ancient Greek philosophy was the problem of motion. Many pre-Socratic philosophers thought that no logically coherent account of motion and change could be given. Plato accepted the Parmenidean constraint that knowledge must be unchanging4. One consequence of that view was that knowledge could be acquired only through reason; hence, Plato held that knowledge is innate or that it can be acquired by pure thought.
Aristotle, in contrast to Plato, believed that knowledge is acquired through experience.

\chapter{Epistemic Architecture}

A discussion of architectural principles for widely distributed collaborative knowledge representation and exploitation systems.


\section{Introduction}

We are at a moment in history when epistemology has become controversial, and is a locus of exceptional risk and opportunity.

The history of `Western' philosophy is replete with opposing tendencies and attempts at synthesis which themselves provoke critique, antithesis and demand further synthesis.
These opposing tendencies often have epistemological cores.

The postmodern philosophy of Michel Foucault denies objective merit to epistemological norms, but the prosperity of human society is substantially attributable to advances in science and technology which the accidents of history predominantly located in certain geographic regions before their utility and power progressed them across the globe.
In terms of their impact on human well being and their contribution to the fulfillment of human aspirations, I cannot myself acquiesce in cultural indifference.

It is moreover a premise of this essay and of the architecture which it sketches, that the most fundamental principles of epistemology are determined not by culture, but by the nature of propositional language and declarative knowledge.
They have come to be understood explicitly only after millenia of industry in appropriate cultural mileu, and are ignored or perverted at cost to humanity,

My interest here is in how to manage knowledge when the following technological developments become significant:

\begin{itemize}
\item The deductive capability of artificial intelligence reaches a level at which it provides effective access to the deductive closure of any body of knowledge which we seek to exploit.
\item That intelligent deductive capability is widely distributed through this and possibly into other galaxies by self replicating intelligent systems.
\end{itemize}

My main interest in this essay is to identify some of the layers in which the various issues of concern from a logical point of view might be structured.
Thus, no attention is given to the physical layers which are concerned, for example, with the technology used to store, retrieve, process and communicate information or knowledge.

Those layers are:

\begin{enumerate}
\item the data layer: a widely distributed heirarchical WORM (write-once-read-many) storage system, associating identifiers with values.
\item the logical name-space: a heirarchical name space in which names denote abstract entities rather than data values, and are characterised by conservative constraints in.
\item abstract syntax
\item core logic
\item abstract ontology
\item logical truth
\item empirical modelling
\item commerce
\item valuations and norms
\end{enumerate}

\section{Data Layer}

In order to give an account of how under this model distinct independent epicenters may combine into a single coherent body of knowledge, we imagine there to be multiple systems distinguished geographically but connected logically, though possibly into more than one logically distinct region.

Each such region is a write-once read-many (WORM) information storage structure with an addressing system broadly similar to our current global internet or to typical digital computer file storage, in which items of information are identified using a hierarchical naming system using finite sequence of names to locate specific items of information.

In this system there will be no fixed top level domain, since connection of logically distinct regions will involve adding higher level domains to one or both of the two name structures.
Addressing will always be relative to a node in the hierarchy, so that addresses within a region are not affected by such a merger.

That this is a write-once structure is necessary to ensure integrity and coherence in the face of distributed asynchronous updates.
Whenever a modification is made at a node it yields in a new node with the same symbolic path but a different sequence number leaving the node with the same symbolic path but distinct sequence number unchanged, and leaving any structure containing or referring to it likewise unchanged.
This will be similar to the effect of an update to a structure performed by a purely functional program, which can only compute a new value without changing the old, and could therefore deliver a history of successive iterations as a list with the laterst version always added at the head of the list.
In that case the references are all memory references rather than symbolic references, and therefore version numbers for named values are not necessary, but in this proposal the relevant references are symbolic, and so a new name is necessary for each new value, and this is obtained by associating a sequence number to the name.
This sequence number will always be larger than the sequence number of any previous version of that named value, and will also be larger than the sequence number of any other value directly or indirectly referred to by the structure.
So not strictly sequence numbers.

\section{Abstract Syntax}

The data layer is a kind of naming system, in which complex names are associated with particular items (possibly large) of stored data.
These names, when fully qualified by sequence numbers, behave like logical constants.
They can be referred to in algorithms to be executed or in the specification of other values, in propositions expressed in suitable propositional languages, and may thence become significant in logical deductions and other means of inference.

In order to be able to talk or write about these values using their names we need languages.
The integrity of the knowledge and the inferences we undertake with it depends on precision in the languages, and the soundness of the deductive systems employed.
In this proposal this may be realised by reduction to a \emph{foundation system}, a logical system with a semantics rich enough for that of most other languages to be expressible in it, and with a deductive system which is sound and strong.

This foundation system need not have a fixed concrete syntax, and many other languages may be understood as special ways of writing down in concrete syntax expressions which belong to the foundational logic.

The abstract syntax which we place at this fundamental place in the architecture is that of a simple polymorphic typed lambda calculus, and the logic itself will be a close derivative of Church's Simple Theory of Types \cite{Church40}.

\part{Logic and Mathematics}

\chapter{Logical Truth and Proof}

An description of how to define logical truth (as analyticity) and how to demonstrate such truths.
This is based on the thesis that set theory is both semantically universal and also is the best way to obtain maximal deductive reach.


\section{Introduction}

In this essay I talk about a particular conception of logical truth, which corresponds closely to the terminology of Rudolf Carnap while he still identified logical truth with analyticity.

I begin with a staightforward definition of analyticity in propositional languages in terms of the `truth conditions' of the languages in which the propositions are expressed, but my main aim is to talk about how the truth conditions can be described.

I then go on to consider how formal renditions of such truth conditions may be obtained, in effect reducing analytic truth across all propositional languages to that in a single closely related family of languages which may be termed a `foundation system'.
The truth conditions thus rendered need (and can) only be \emph{abstract} truth conditions, in which the intended subject matter of a language is rendered in abstract term.
The account therefore depends upon the observation that the property of analyticity will be caputured perfectly through such abstract truth conditions, and will not differ from the results which would have been obtained if concrete truth conditions had been definable in the relevant foundation system.

The most suitable (though not the only adequate) foundation system for this purpose is set theory, and that is therefore the one discussed here.
Thus far we have discussed only the \emph{definition} of logical truth.

The foundational family also provides a basis for the \emph{demonstration} of logical truth.
The definition of abstract truth conditions effectively translates sentences into the language of the foundation system, which are then (if true) candidates for proof in that system, reducing proof in the target language to proof in the foundation system.

\paragraph{quotes and apostrophes}

I use double quotes only when I am reporting verbatim the words written or spoken elsewhere, in which case it would be normal to have an attribution, though sometimes when referring to my own usage.
I use single quotes if I am mentioning rather than using a word, or when I am distancing myself from (despite adopting) a usage which is doubtful or controversial in some way.
There may be other cases, but the main
I use italics for emphasis or when I am introducing some new terminology to highlight the new term.

Apostrophes are something else, in which I hope my usage is not eccentric.

\section{Analyticity}

In the aftermath of the debate on the concept of analyticity which took place between Rudolf Carnap and W.V. Quine \cite{carnap90}, the attempt in this essay to make as solid and precise as possible a conception of analyticity and the means for establishing the analyticity of particular propositions may be thought to be responding to a presumed weakness in the concept of analyticity which renders it rather more in need of attention than other philosophical terminology.

The purpose of this section is to give my reasons for considering that there should never have been credible doubt about the status of the word analyticity, even in the days between the publication of Carnap's `The Logical Syntax of Language' (which was Quine's introduction to Carnap's work) and Quine's supposed demolition of the concept of analyticity in his `Two Dogmas of Empiricism'.
Which is not to say that there was not and is not grounds for further refinement of the concept and elaboration on its ramifications, to which this essay is intended to contribute.

I know of no other philosophical concept which is as precisely definable as the concept of analyticity, matching the level of precision found in the concepts of mathematics.
Any such definition depends for its precision on an exact delimitation of the domain in which it is applicable, for otherwise the terms used in the definition themselves be meaningless.

What is required for the concept of analyticity (as it is used here%
\footnote{The concept has a long history and its meaning has evolved over millenia.}) to be applicable to the sentences of some language, is that the language has definite `truth conditions'\index{truth conditions}.
For a language to have definite truth conditions, it must have a definite delimitation of its scope of applicability, which, if the subject matter of the language is 'the world', would be the range of possible worlds.
If the language is not intended to refer to `the world' then we may in general consider the subject matter of the language to be constituted by a collection of intended interpretations, and I will use that terminology for the general case, including as possible worlds as `interpretations' of a language.

The truth conditions of sentences in a language then consist in:

\begin{enumerate}
\item A set of intended interpretations (the possibiities)
  \item An assignment to each sentence of the language of a subset of the intended interpretations (the cases in which that sentence is true).
\end{enumerate}

The set of interpretations thus assigned to each sentence should be understood as the set of interpretations under which the sentence in question is `true'.
A sentence in such a language is then considered `analytic' if (and only if) the set of sentences in which that sentence is true is the full set of intended interpretations, i.e. if it is \emph{always} true under every possible cirumstance.
\footnote{
Though this concept has been at the centre of controversy in philosophy it is less controversial among mathematical logicians, who are more accustomed to precise definition of the syntax and semantics of their notations and languages.
The property of `completeness' of a deductive system consists in the derivability in that system of every analytic formula, but is generally expressed in other terms.
Some important milestones in the development of modern logic have been proofs of that kind, notably, the completeness of the sentential calculus, proven in the doctoral dissertation of Emil Post \cite{post21}(in which the term `positive' is used for the formulae here called analytic), and the completeness of the first-order predicate calculus, proven in the doctoral dissertation of Kurt G\"{o}del \cite{godel30a} (by which time the term `valid' was usual).
}

Methods for the formal description of the semantics of formal languages were later greatly elaborated by computer scientists and engineers, who also engaged 


The usage of the term `interpretation' here is generic, supports the present narrative and may not exactly correspond to the accepted usage in accounts of particular logical systems.
It encompasses all that must fixed before the truth value of a sentence is determined, which would not normally include values for free variables, since `sentences' may be required to be closed (i.e. to contain no free variables).

In an early treatment of simple formal languages which indicates how comfortable mathematicians are with describing the meanings of their notations well enough for 

\cite{heijenoort67,post21,godel30a,tarski31,tarski56,carnap47}

\section{Natural and Formal Languages}

Deductive reason, which begins in the natural languages which homo sapiens has spoken since the origins of our species, has been shown to be highly reliable and to delineate the particular class of truths here spoken of as ``logical truths''.
Nevertheless, attempts to reason soundly in natural languages have not always been successful.

Both extremes are documented in the works of the philosophers of ancient Greece in the period between 600 and 300 years before Christ.
During that period a body of mathematical theory was established under the heading ``Equlidean geometry'' (and beyond) which survived more than 2000 years before any flaws could be found in it, but when reasoning was applied to understanding the cosmos the conclusions reached were almost immediately disputed, and it was to be readily demonstrated that contradictions and absurdities could ostensibly be demonstrated.

It is reasonable to attribute these disparities to two principle factors, not unrelated.
The first is subject matter.
In mathematics, the subject matter is abstract and has quite simple characteristics which are readily captured precisely by quite simple language.
When we come to talk about the material world, and when we concern ourselves with the nature of that world at very small or very large scale, the language we use is much less clear, and our knowledge of the key principles in those domains is tentative or speculative.

Considering the question of precision of language it may be noted also, that a weakness of clarity translates into a risk of equivocation.
Equivocation is the logical fallacy of using the same term with different meanings in the same argument, which is lethal to sound reasoning and gives ruse to contradictions and the demonstration of falsehoods.

Reliably extending the reach of deductive reason beyond Euclidean geometry was not impossible in mathematics, since in that domain the subject matter not only permitted but made essential notational innovation ensuring that the language of mathematics remained a precise and unambiguous tool for expressing and demonstrating results.
Greek philosopher did seek to capture that same realiability in other domains.
Aristotle, the first philosopher knowm to have studied and written about logic and demonstration, devoted consiferable energy to articulating a conception of demonstrative science to that end, taking important steps toward `formal' treatment of the topic in his syllogistic.
Logic was also a strength of the Stoic tradition in Greek philosophy.

For the study and enunciation of logical principles, further demands are placed upon language, not dissimilar to those familiar in mathematics.
For example, in order to talk about a variety of linguistic structures which have a common logical structure, it is helpful in the study of logic, as it is in mathematics, to make use of `variables', names which have no definite designation but are to be construed as referring to any one of a number of gramatically similar entities.

We will find that for logic to reach its full potential, and for a good account of the conception of logical truth. resort to formal notations, similar in character to the mathematical notations, will be necessary.
In what follows the discussion of the various aspects of logic will at first begin with the way in which these appear in natural languages, but will then progress to speaking of them in formal notations which minimise the great risk in natural languages of reading those concepts in ways not intended, and hence possibly not consistent with the story.

\section{Propositional Logic}

\subsection{Propositions in Natural Languages}

That which is expressed by a sentence in a natural language, which from its form and meaning we would expect to be either true or false, I call a {\it proposition}\index{proposition}.
The word `proposition' is not always used in this way.
Sometimes it is used to refer to sentences themselves, rather than what is expressed by them.
Sometimes `proposition' is a verb.

In speaking of that which is expressed, I speak of the \emph{meaning}, either of a sentence or of some other expression in a language. such as a word or phrase.
What kind of thing a meaning is will remain rather vague, but there are certain aspects of meaning of which I will speak in more precise terms in due course, notably, ``truth conditions'', which tell us under what circumstances a sentence will be true or false.

By `natural language' I mean the kind of language which people normally use when conversing with each other, and which typically has very ancient origins of which we are largely ignorant.
The meanings associated with sentences or expressions in natural languages are to be discovered by studying the way in which those sentences and expressions are used in typical discourse.
Insofar as that usage is imprecise or inconsistent, so will be our understanding of the meaning of the language.
We know the meaning of 'meaning' in a similar way, and this adds confusion to an enquiry into the meaning of natural languages, which is best remedied for our present purposes by adopting and defining technical terms to describe the aspects of language which are of interest.
Because people will learn and make use of languages before they have very broad acquaintance with the usage of any but a few others speakers of the language, the meanings of expressions in these languages is often uncertain, to the extent that not only may each of us not know what they are, but that there may be no definite objective meaning.

This is in contrast with what I will call ``formal language'', which is a language which has been devised and defined for some particular range of usage, often with the desire for a language which is more precise and better defined than natural languages.

\subsection{Sentential Connectives}

The study of propositional logic begins with the connectives in natural languages which become the prototype for operators in a formal language.
For a more extended treatment see \cite{sep-connectives-logic}.

Given one or more sentences we may construct more complex sentences from them using sentential connectives \index{sentemtial connective}.
Our interest here is in those sentential connectives which are \emph{truth functional}\index{truth functional}, which means that the truth of a comppound sentence formed using the connective depends only upon the truth of the sentences from which the compound was formed.

Two simple connectives illustrate this phenomenon.
The connective `and' combine two sentences into a larger compound which is true if and only if both of the two constituent sentences are true, and false otherwise.
Semantically this connective operates upon the propositions expressed by the sentences, and is may therefore also be called a propositional connective or operator, and plays an important role in propositional logic.

We can describe the meaning of truth functional operators using \emph{truth tables}\index{truth table}.
A truth table shows for each combination of truth values for the constituent propositions the truth value of the resulting proposition.

Because the word `and' is not only used in natural English purely as a truth functional connective, we abstract its truth functional role into a formal language where that is its sole purpose, and its semantics is therefore simplified and made more definite.
In that context, instead of `and' it is common to use the symbol '\land'.
The following table gives the truth function which this symbol denotes.

\begin{center}
  \begin{tabular}{c|c|c}
 A & B & A \land{} B\\
 \hline
 F & F & F\\
 F & T & F\\
 T & F & F\\
 T & T & T\\
 \end{tabular}
\end{center}

An even simpler connective, which doesn't really connect because it has onlt one operand, but is nevertheless a truth functional operator of great significance, is expressed in English as `not' and in its use as a truth functional operator inverts the truth value of the sentence it is applied to as is seen in its truth table:


\begin{center}
  \begin{tabular}{c|c}
 A &\lnot{} A\\
 \hline
 F & T\\
 T & F\\
 \end{tabular}
\end{center}


In the language of propositional logic, complex expressions are built from propositional variables, by repeated application of a variety of propositional operators.
The meaning of such expressions is then the truth function which maps the values of the variables which occur in the expression to the resulting value of the expression as a whole.

Just as in the case of single truth funcitonal operators corresponding to sentential connectives, the truth function can be shown using a truth table.
It is then considered to be a logical truth or a tautology, if the resulting value is always true, whatever the truth values of the propositional variab;es which occur in it.

Inspecting the tables above, we can see that neither $A \land B$ nor $\lnot A$ are tautologous, since for some values assigned to A and B they yield the value false.

Taking these two connectives as given, we can define other connectives.
One important connective is `or' rendered symbolically as `\lor{}', which can be defined in terms of \land{} and \lor thus:

\[  A \lor{} B = \lnot{} ((\lnot{} A) \land{} (\lnot{} B)) \]

which says that A or B is true when it is not the case that both A and B are false, with the truth table:

\begin{center}
  \begin{tabular}{c|c|c}
 A & B & A \lor{} B\\
 \hline
 F & F & F\\
 F & T & T\\
 T & F & T\\
 T & T & T\\
 \end{tabular}
\end{center}

We are now in a position to exhibit some elementary logical truths.

\begin{center}
  \begin{tabular}{c|c|c|c|c}
 A & \lnot{} A & A \land{} (\lnot{} A) & A \lor{} (\lnot{} A) & \lnot(A \land{} (\lnot{} A))\\
 \hline
 F & T & F & T & T\\
 T & F & F & T & T\\
 \end{tabular}
\end{center}

In this table the second and third columns both show T on every row, indicating that the formulae at the head of these columns are both logical truths in the propositional calculus, or tautologies.

We do not yet have a `logic', for we lack formal rules which enable us to prove the logical truths which can be expressed in this language.

To introduce the notion of proof we must instroduce one more sentential connective with corresponds to the expession (A therefore B) or (if A then B).
For this we use the technical term `material implication' because of its similarity with the concept of implication, and because there are non truth functional uses of the term `implication' which make the truth functional commective a controversial formalisation of implication.

The symbol we will use here for material implication is `\Rightarrow{}', which can be defined as:


\[ A \Rightarrow{} B = B \lor{} (\lnot{} A)\]

Which allows us to state as a theorem the rule of Modus Ponens, which is important in establishing a formal deductive system for propositional logic.


\[ (A \Rightarrow{} B \land{} A)  \Rightarrow{} B\]

To describe the structure of formal proofs we need some more notation.
The symbol $\vDash{}$ will be used to signify that the following formula of the propositional calculus is a tautology and hence a logical truth.
The symbol  $\vdash{}$ will be used to signify that the following formula is a theorem of propositional logic.

What we seek in an inference system is that it proves only truths:

\begin{center}
  $\vdash{}$A only if $\vDash{}A$
  \end{center}

which is the property of being \emph{sound}\index{sound}.

Ideally it will also be \emph{complete}\index{complete}:

\begin{center}
  if $\vDash{}$A then $\vdash{}$A
\end{center}

The inference system for propositional logic consists of a single rule of inference and a small number of \emph{axioms}\index{asiom}.
It is sound and complete.


\section{The Distribued Information Structure}

The whole could be thought of as simple a natural number, for however much information is stored across the Galaxy, the aggregation of that information will at the end be simply a number.

This is not a particularly helpful way of thinking of it, but it is a starting point.
In order for this to have any value we must be able to refer to particular parts of the information. which would be facilitated by thinking of the number as a sequence of bninary digits.
We may then address particular parts of the information by its location in the binary sequence.
Though that may help, its clearly not much help.
It would be better if we cou;d use some kind of name to refer to particular parts of the structure, and if that location and length of the sections referred to by a name to be flexible, so that changes to one part of the structure which effectively move parts if the structure do not disturb the naming of other structures to which the change would otherwise be irrelevant.

The information structure then becomes a name-space, or a mapping of names to strings of bits (or more elaborate structures represented ultimately by such strings).
Naturally we would want these names to form a structured heirarchy, along the lines of a directory structure, extended as it has been to provide a unique global nameing system as Universal Resource Identifiers, extended yet again to provide a unique naming system for a pan galactic information system.
As information systems expand across the galaxy they will in all likelyhood encounter previously unknown systems and would want to create an integrated distributed system which embraced both systems.
This could be done by adding one of the new systems as a top-level ``domain name'' in the other system, or by creating a new higher level to the heirarchy and allocating distinct names at that level to each of the participants.
This requires a naming scheme which is always relative to a place in the naming structurem. and the number of layers above is not fixed.

For this distributed system to function coherently over the very large distances which would be involved, hundreds of thousands of light years in this galaxy alone, it must be a write-once structure in which any node has only a partial representation of the structure, but it is guaranteed that no two distinct versions of any part of the structure exist.
That means that any changes to the structure create a new version of the structure leaving the older version intact.
This can be thought of as the process of updating a large information structure in a purely functional language, which works from a list of versions of the structure and computes a new list whose tail is the original structure and whose head is an amendment to the head of the old list.
There is nothing to say that an implementation of such a system or of a fragment of the system could not be accomplished by some completely different mechanism, such as a database which keeps a log of transactions (including the original values), or by the methods use in source code control systems to preserve all previous versions of the software.

That this information structure is ``write once'' is essential to the coherence of deductive closure.
If amendments were permitted to data which is referred to in some proposition without that proposition having to be reproven, then contradictions would result and the deductive system would no longer be sound.
So, when the value associated with a name is modified, any theorems which directly or indirectly mention that value would continue to refer to the old value, and only a revised demonstration would be able to refer to the updated value.

\section{Languages}

We have so far only addressed an information store.

Before we are able talk about logical truth and proof we must talk about language.

The conception of logical truth we are addressing is that known as \emph{analyticity}, as that notion was defined by Rudolf Carnap \cite{carnap47,schilpp63} and refuted by Quine \cite{quine51b,quine61a}.
Passing over that refutation, the implication of Carnap's conception of logical truth  is that logical truth is not something which belongs only to sentences in those very special languages which may be called ``logics'', but rather something which may be found in the sentences of any language which is well-defined (in its syntax and in certain details of semantics).

There is no general consensus about what a ``language'' is, between philosophers, logicians, computer scientists, cognitive scientists and linguists.
For some purposes, for example in the logician's notion of ``a first order language'', a language has a fixed vocabulary.
If new terms (known as constants) are introduced to an existing first order language, the result is a new first order language.
On the other hand, when we talk about \emph{programming languages} we speak of fortran, cobol, C, java, python... each as \emph{a} language, of which we expect normal use to consist in elaborating new algorthms, giving them names, and then using them in defining yet more named  algorithms and data structures, until in the fullness of time the functional behaviour required of the program is properly defined.
Introducing new vocabulary is the normal manner of use of these languages, but does not create new languages.

The augmentation of language in the normal course of its use usually goes beyond the mere aggregation of vocabulary, allowing flexibility in the concrete ways in which the use of the vocabulary is presented.
Thus new names for mathematical functions may syntactically appear as prefix, infix, or postfix, appearing before, between or after the values to which the function is to be applied, and may be given priorities which influence the order in which the operations are to be undertaken during the evaluation of expressions.
The need for such flexibility in concrete syntax is felt more strongly in languages which are intended for the formal derivation of logical theories, since their use may aspire to match the flexibility with which mathematical notations have been devised during the historical development of mathematics.
This may be crucial to the adoption of more formal notations by human mathematicians.
In the distant future, when theoretical mathematics is undertaken by synthetic rather than natural intelligence, the historical and continuing need to communicate with humans will make similar demands.
These are not the only issues which complicate the support for languages in our distributed information and knowledge system.

\section{Logical Foundation Systems}

At the birth of modern logic, arguably with the publication by Gottlob Frege of his \emph{Begriffsschrift} \cite{frege1879}, Frege perceived that his new logical notation and its deductive system would suffice for the formal development of arithmentic (and was more generally applicable).

At that point he offered the formula:

\begin{quote}
  mathematics = logic + definitions
\end{quote}

The approach to deductive reason proposed here is rooted in that insight.
The idea was not universally accepted because the conception of ``logic'' which it required was broader than many thought logical necessity should embrace, mainly because it involved the establishment of an abstract ontology in terms of which the mathematical definitions could be understood, which was somewhat arbitrary.

An alternative rendition which proved more acceptable (and has often been offered as a correction to Frege's maxim) was:

\begin{quote}
  mathematics = set-theory + definitions
\end{quote}

In which context the entities of mathematics, such as numbers, were to be construed as sets.

Set theory is not the only basis on which mathematics can be built in a similar manner.

Though there is discretion about abstract ontology, it may still be thought purely a matter of logic, in which there is choice of abstract ontology on which to base mathematics (and all abstract models), but the entire proceeding precede any input from the concrete world.
The philosophical position which I here adopt can reasonably be described as \emph{ontological conventionalism} and its central tenet is that abstract ontology is, like language, a choice which we are free to make, and that logical truths ensue by reasoning in the context of such a choice.
The freedom to make that choice, and the manner of making it, are not entirely without consequence.
The choice of ontology for a foundation system may be made in two different ways which must then be closely connected.
There may be a \emph{semantic} choice in which an informal description of the domain of abstract entities is described in terms not necessarily capable of rendition in an already defined formal notation, and there may be a more formal rendition as axioms in the logical language of the foundation system which admits the formal derivation of truths about the system, but may not be complete in proving all the sentences which are true in the intended and informally described domain.
It is very likely that no single domain is intended, but that several alternative interpretation of the axioms exist and that no single interpretation is given a special status.

A logical founation system is a logic together with a copious abstract ontology in which the main body of mathematics can be derived given suitable definitions of the concepts involved.
We do not expect any such system to be universal, but there are systems which admit of augmentation by the addition of further ontological premises, such that it is plausible that any coherent logical theory could be accomodated gived an appropriate extension, and in which sufficiently strong ontological principles are readilty identified so that practical applications are unlikely to require firther augmentataion of the ontolgical premises (even though some questions about the ontology may never be conclusively settled).

\part{Philosophy of Science}

\chapter{The Mind Body Problem and Consciousness}
A discussion on the nature of consciousness, prefaced by broader considerations about ontology and metaphysics, in which I conclude in favour of the possibility that consciousness is no big deal (though important).

\section{Introduction}

I have some ideas about the nature of consciousness.
I have no qualifications which might persuade anyone that my ideas about consciousness should be taken seriously.
I don't pretend they are `scientific', I am not a scientist.
Nor am I a professional philosopher, though I spend a lot of time thinking about and even attempting to write philosophy.

The origin of these ideas is in my own experience, and to a limited extent in anecdotal evidence from others.

You might think that before writing about consciousness I would pay some attention to what others have said about it.
I confess I have not.
From time to time I make an attempt.
But invariably I have found what people say when they talk about consciousness so unconvincing that I bail out in a really quite ridiculously short space of time.
I just did it a few minutes ago.
I saw on my twitter feed that Lex Fridman had talked to Sam Harris about consciousness, and though before launching into this essay that I would listen to what they had to say.
I gave up less than 4 minutes after Sam started talking.
So you may now be assured that I am thoroughly ignorant of what others have said about consciousness.

\section{Metaphysics and Science}

My discussion of conscousness below is instrumental and may be thought materialistic.
Insofar as ``materialism'' is the denial of any existent other than matter, it is not, even though may mention no other.
Of course, to offer an explanation of consciousness which fails to mention ``mind'' I may bolster the case for materialism, but that is not my intent.
On these matters I am in fact closely aligned with the now thoroughly discredited logical positists, or at least, with Rudolf Carnap the only logical positivist with whose views on these matters I am reasonably aligned.

\section{Lapses of Consciousness}

Here are some observations, partly about my own experience, partly about things which are pretty generally accepted, which I have taken to bear upon the problem.

I once had a fairly serious road traffic accident, in which the main damage was to my head, and resulted in various fractures and concussion.
Of course, I had no memory of the accident.
It occurred round about midnight, and I woke up in the hospital the next day not knowing what had happened, except that I knew I was driving when last conscious.
Though I didn't learn it at the time, sometime later I came to know a medical professional who happened to be around when I was brought in to the hospital, and told me that I had made a great fuss.

Was I conscious when they brought me in, and then forgot about it, or was I just acting out some unconscious mental distress?

\section{Partial Consciousness}

Consciousness is not all or nothing.
When we are not actually unconscious, our consciousness is selective.
We are conscious of only those aspects of our experience to which we are \emph{paying attention}, and arguably not all of that.
We can perhaps be `paying attention' to things which we are not {\it consciously} considering until they actually happen.
A mother may be continually on alert for signs that her child is distressed even when her conscious attention is focussed on some other matter.

I drive, and have a pretty reasonable safety record.
I don't often drive while having a conversation with passengers (or on the phone), and that's something I don't do in a wholly satisfactory way.
In that case, my driving doesn't seem to be any less safe, but my navigation can go completely awry.

More commonly, I do drive and think about other things.
In this case neither the safety nor the navigation seems to be compromised, but my memory of these is compromised.
I have driven on ``autopilot'', and I have little or no memory of the details.
It is tempting to say that it way my unconscious mind which did the driving while my conscious mind was thinking about something else.

There is other anecdotal evidence suggesting that the unconscious mind is no less capable of demanding intellectual feats than a conscious mind.
Famous mathematicians have reported that, having begun in earnest to crack a difficult mathematical problem, the solution appears to them in a dream, or after waking up from a nights sleep.
The anecdotes suggest that once a train of thought has been set in motion by the conscious mind, it can be carried through to conclusion sub-consciously.

The tendency of these anecdotes, is to suggest that the differences between conscious and unconscious thinking are perhaps not very substantial.
But the whole point of this essay is to put forward a suggestion about what the key difference is between these two.
Before doing that I will talk about something which has little to do with consciousness, and then put foward an analogy.

\section{Software Development}

\section{Musical Performance}

\section{Levels of Learning}

\section{Evolution of Memory}

\cite{murray2019evolutionary}


\chapter{Epistemology for AI Safety}

\section{Preface}

When computation was first automated, in many areas the new digital technology was faster, more accurate and more reliable than human computers.
As the complexity of digital hardware and software rose defects became more likely, and even though the hardware remained highly accurate and reliable, defects in the instructions and the data upon which it was operating on made computer systems less reliable.

From the beginning, some aspired to design systems which were not only fast accurate and reliable, but also \emph{intelligent}, and this has proved an illusive goal.
One approach to achieving that goal has been to mimic the way in which human beings achieve their intelligence, to use software which emulates neural nets, and to ``train'' these digital brains rather than programming them.
At first these methods did not achieve impressive results, but as the amount of computational resource which could be devoted to them rose they have had more success, and unexpeted capabilities have emerged in Large Language Models, inspiring a new digital gold rush of companies seeking to continue and to commercially exploit these emerging capabilities.

A hallmark of this new paradigm for advancing machine intelligence is its unreliability.
A large language model performing as chatbot has been trained on a far larger body of information than any human can absorb in his lifetime, and in most areas well-covered by the literature its generative capabilities will reproduce the range of received opinions on the topic in question.
However, these models achieve this effect without any reliable way of telling truth from falsehood, and without a reliable check on coherence.
When asked to speak on esoteric topics sparcely addressed in its training data, an LLM will be creative, resulting in a perhaps quite elaborate and detailed work of fiction.

Along with the enthusiasm for the capabilities of LLMs, in part stimulated by the fear that GAI may be nearer than had been supposed, we are seeing widespread concern about ``Safety'' in AI.
As in the case of fact checking in digital media, we may suspect that often this is more a concern that AI not align with right political views than one about factual accuracy, given that political deviance is often now regarded as dangerous, hateful and akin to physical violence or insurrection.

Our experience with fact checkers who are often appointed more for their political alignment than for any supposed competence in weeding fact from fiction, and are hence often no more reliable than the media they police,  suggests that there are difficult epistemological problems to be addressed for society as a whole, not just for that growing part of human innovation which is undertaken by or with digital intelligence.


This essay is an attempt to analyse these kinds of epistemological problems and to take some initial steps in the synthesis solutions to those problems.
 
\section{The Problem}

Even without the special novelties which are brought into the picture by Artificial INtelligence, the internet has already created a big problem in separating fact from fiction.
This is often spoke of as a fight against ``disinformation'' or ``misinformation'' as if the central problem was not that of deciding what is the truth.
Unfortunately, just as we find so many more voices, and so much more information flooding our senses and intellects, the criteria for judging their authenticity andreliability are disintegrating.
Academia and journalism, which once at least pretended to be objective and unbiased, have become increasingly tranformed into forms of social and political activism, partly under the influence of philosophical tendencies which openly challenge traditional views about language and epistenology and regard the meaning of language and the nature of knowledge as political weapons which have been and should be manipulated in the quest for power.

\section{Philosophical Preliminaries}

The structure of the analysis to follow is informed by a particular perspective on the nature of language and of knowledge.
These are best explained by talking about their origins in the history of philosophy (particularly philosophy of language, epistemology, logic and the philosophy of mathematics and science) and closely related disciplines such as linguistics, mathematical logic and computing science and engineering.


\subsection{}

\subsection{David Hume}

I begin with David Hume for the epistemological distinctions which he drew, first of all between logical and empirical truths (relations between ideas and matter of fact) and between claims about what \emph{is} and what \emph{ought to be} (descriptive and evaluative).

\subsection{Emmanual Kant}



\section{The Structure of Knowledge}

Let us consider the knowledge gathering and exploiting of intelligent life in the universe to be accumulating a single body of shared knowledge, and consider how that knowledge might be organised.

It is safe to assume that such a shared body of knowledge will be managed by digital computers, it already is and they are the most effective way of sharing this kind of commodity.
This in itself does not require intelligent machinery.

Nevertheless, we do want a collaboration in which artificial intelligence can play its part, which we expect to be a growing part, and so as time progresses it is not unreasonable to expect that both the reasearch and development which expands this knowledge base and that which exploits it will be primarily conducted by artificial intelligence.

This is not a proposal about \emph{how to do} AI, but rather one about \emph{what to do} with it.
Even if it were the case that by continuing the present trajectory of training artificial intelligence by exposing it to every possible manifestation of human intelligence which can be digitised we obtain machines which surpass human intelligence in all areas, it would still be necessary to have a shared body of knowledge, for which the weight  in a neural net will not suffice.
It is that shared body of knowledge which is our concern.

In accumulating the body of knowledge which we now share, particularly that knowledge of science, engineering and technology which has been particularly important in securing our growth and prosperity, reason has played an important role.
Digital hardware and, one hopes, machine intelligence greatly improve on the ability to reason soundly, both in establishing and in applying our knowledge.

We would therefore want to see this ongoing body of knowledge organised in a way which is suitable for rigorous automated reasoning.

\part{Definitely Not Philosophy}


\chapter{AI at Cadence by ChatGPT}

\section{Current Research Initiatives at Cadence Involving AI}

Cadence, a leading provider of electronic design automation (EDA) software and hardware, has invested in several research initiatives that involve AI and machine learning technologies. Here's an overview of some of their current projects:

\subsection{AI in Design Automation}

Cadence has been exploring the use of machine learning algorithms to improve the efficiency and accuracy of their design automation tools. This involves developing and training neural networks to recognize patterns in design data and make predictions about design outcomes. Some of the specific areas where Cadence is applying AI include:

\begin{itemize}
\item \textbf{Place-and-Route Optimization:} Cadence is using machine learning algorithms to optimize the placement and routing of electronic components on a chip. This can help reduce design time and improve overall product performance.

\item \textbf{Analog Circuit Design:} Cadence is also applying AI to analog circuit design, where traditional rule-based methods are often inadequate. By training neural networks on large datasets of circuit designs, Cadence aims to develop more accurate and efficient methods for designing analog circuits.
\end{itemize}

\subsection{AI-Powered Verification}

Verification is a critical step in the semiconductor design process, ensuring that the final product meets design specifications and works as intended. Cadence is exploring the use of AI to improve the efficiency and accuracy of this process. Some of the areas where AI is being applied include:

\begin{itemize}
\item \textbf{Coverage-Driven Verification:} Cadence is using machine learning algorithms to optimize coverage-driven verification, where tests are generated to ensure that all parts of the design have been tested. By predicting which parts of the design are most likely to fail, Cadence aims to reduce the number of tests required to achieve full coverage.

\item \textbf{Debugging:} Cadence is also developing AI-powered debugging tools that can automatically identify and diagnose design errors. This involves training neural networks to recognize patterns in design data and identify the root cause of errors more quickly and accurately than traditional manual methods.
\end{itemize}

\subsection{AI for Chip Power Optimization}

Power consumption is a critical consideration in modern chip design, as more and more devices rely on battery power. Cadence is applying AI to power optimization, with the goal of developing more energy-efficient chips. Some of the areas where AI is being applied include:

\begin{itemize}
\item \textbf{Dynamic Power Management:} Cadence is using machine learning algorithms to optimize dynamic power management, where the power consumption of the chip is adjusted dynamically based on the workload. By predicting the workload more accurately, Cadence aims to reduce power consumption while maintaining performance.

\item \textbf{Thermal Optimization:} Cadence is also exploring the use of AI for thermal optimization, where the temperature of the chip is managed to prevent overheating. By predicting the thermal behavior of the chip more accurately, Cadence aims to reduce the risk of chip failure due to overheating.
\end{itemize}

Overall, Cadence is investing in several research initiatives that involve AI and machine learning technologies, with the goal of improving the efficiency, accuracy, and performance of their EDA tools and hardware.

\phantomsection
\addcontentsline{toc}{section}{Bibliography}
\bibliographystyle{rbjfmu}
\bibliography{rbj}

%\addcontentsline{toc}{section}{Index}\label{index}
%{\twocolumn[]
%{\small\printindex}}

%\vfill

%\tiny{
Started 2021/12/05


\href{http://www.rbjones.com/rbjpub/www/papers/p034.pdf}{http://www.rbjones.com/rbjpub/www/papers/p034.pdf}

%}%tiny

\end{document}

% LocalWords:
