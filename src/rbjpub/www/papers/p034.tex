% $Id: p034.tex $fi
% bibref{rbjp034} pdfname{p034}
\documentclass[10pt,titlepage]{book}
\usepackage{makeidx}
\newcommand{\ignore}[1]{}
\usepackage{graphicx}
\usepackage[unicode]{hyperref}
\pagestyle{plain}
\usepackage[paperwidth=5.25in,paperheight=8in,hmargin={0.75in,0.5in},vmargin={0.5in,0.5in},includehead,includefoot]{geometry}
\hypersetup{pdfauthor={Roger Bishop Jones}}
\hypersetup{pdftitle={Some Scientific Scepticism}}
\hypersetup{colorlinks=true, urlcolor=red, citecolor=blue, filecolor=blue, linkcolor=blue}
%\usepackage{html}
\usepackage{paralist}
\usepackage{relsize}
\usepackage{verbatim}
\usepackage{enumerate}
\usepackage{longtable}
\usepackage{url}
\newcommand{\hreg}[2]{\href{#1}{#2}\footnote{\url{#1}}}
\makeindex

\title{\LARGE\bf Some Scientific Scepticism}
\author{Roger~Bishop~Jones}
\date{\small 2021/12/05}


\begin{document}
\frontmatter

%\begin{abstract}
% I often have doubts about supposed scientific truths.
% This is a place where I talk about them.
%\end{abstract}
                               
\begin{titlepage}
\maketitle

%\vfill

%\begin{centering}

%{\footnotesize
%copyright\ Roger~Bishop~Jones;
%}%footnotesize

%\end{centering}

\end{titlepage}

\ \

\ignore{
\begin{centering}
{}
\end{centering}
}%ignore

\setcounter{tocdepth}{2}
{\parskip-0pt\tableofcontents}

%\listoffigures

\mainmatter

\pagebreak

\section*{Preface}

\addcontentsline{toc}{section}{Preface}

\footnote{There are ``hyperlinks'' in the PDF version of this monograph which either link to another point in the document  (if coloured blue) or to an internet resource  (if coloured red) giving direct access to the materials referred to (e.g. a Youtube video) if the document is read using some internet connected device.
Important links also appear explicitly in the bibiography.}

I am not a scientist.
Professionally, I was a software engineer, i.e. I was primarily engaged throughtout my career in one or another aspect of the development of computer sofware.

I have a first class joint honours degree in Mathematics and Philosophy, from an English University not particularly distintinguished in either field.
I earlier spent one year studying mechanical sciences at Churchill College, Cambridge.
I have since then spent many decades learning a lot of other stuff, and continue a life in which I am constantly learning, usually with some purpose in mind.

I am at present trying to write something about rationality and its evolution, and in the course of these attempts am often reminded of some of the points at which I doubt received scientific truths, wondering how to deal with these doubts.
Alonside my persistent scepticisms, there is a continuous process, as I read into the various sources apparently relevant to my enterprise, of assessing the credibility of the various sources.
It makes no difference how well or ill qualified one might be to make such judgements, there is no escaping those decisions about what is worth reading, how seriously to take what it says, and how that can be fitted into a coherent and credible conception of how the world is, has been and will be.

The final provocation to write on this was a foray into the earliest history of the Universe, trying to decide whether in a meaningful way this period might fit into a broadly conceived evolutionary story.
This naturally brought me face-to-face with some of the most tenuous theories of phyusics and cosmology, and has made me feel the need to come clean about how I think in these areas.

This essay is my attempt to do that.

\chapter{Introduction}

Should this essay approach completion then it might address the following areas:

\begin{description}
\item{\bf Entropy}
  The second law of thermodynamics, asserting that in any closed system or the universe as a whole, entropy can never decrease, has spawned a whole variety of alternative conveptions of entropy.
  A prevalent generalisation is that entropy is a measure of disorder, and that the second law entails increasing disorder in the Universe culminating in ``heat death''.
  While aknowledging the value of the concept of entropy in understanding the workings of steam engines, I am pretty sceptical about most else.
\item{\bf Special Relativity}
  The usual explanations of special relativity are incoherent and the evidential basis supporting the theory seems to me unconvincing.
  I can't say I am convinced of the theories falsity, but I'm certainly not convinced of its truth.
\item{\bf General relativity}
  Its a neat trick of Einstein to incorporate gravity into the structure of space-time,  a possibility which arises from the fact that gravity is the only force for which the field determines the acceleration of objects without reference to any of the propoerties such as their mass.
  This may also make the formulation of the theory mathematically more elegant.
  But elegance aside, I have never heard a rationale for not considering the forces of gravity in the same way as all the other forces.
  What this does do is to obscure, at least in lay discourse, the real differences between relativistic and newtonian physics.
  The discussion usually begins with the geometric innovation and never progresses beyond it.
  This is like chosing a new language to talk about some scientific speciality, and never getting beyond talking about the language into discussing the theories expressed in the language.
  Most of the innovations beyond that do not really impact the core elements of Newtoninan theory, viz the laws of motion and of gravity.
  \item{\bf The Expansion of the Universe}
    This theory and the big bang theory which flows from it depends upon the explanation for the plausible correlation between red shift and distance (as due to recession velocity).
    Things get wore when we start talking about the acceleration of the expansion, more so when dark matter is offered as an explanation for the expansion and its acceleration, and reach another plane when we come to the idea that the space which is occupied expands rhather than the universe itself.
  \item{\bf evolution}
    I'm pretty OK with the idea of the evolution of species through natural selection (with some cavils).
    Beyond that thesis the idea that evolution is a ``theory'' of a kind not too dissimilar to Newtons theory of gravitatino, is one which I think desrves and benefits from much sceptical analysis.
    \end{description}

\chapter{Entropy}

I have not a lot to say here.
I will give only a sketchy account of my negative views on entropy and the second law of thermodynamics, which I don't expect anyone to accept but feel that I nevertheless should mention them.

The shortest statement is as follows:
\begin{enumerate}

\item Though in certain contexts the original definition of entropy is clear and useful (in calculating the maximum work that can be obtained from heat), its subsequent generalisation yields an ill-defined concept.
\item The second law of thermodynamics is therefore meaningless.
  If one were to make this more clearly about heat (rather than about order), then the second law would most likely be false.
  \item The idea that the second law guarantees that the Universe is destined inevtably for some kind of heat death has no empirical basis, and any use of that idea in promulgating other theories (e.g. about evolution or the long term future of humanity) will in my opinion 
\end{enumerate}


\begin{description}
\item[Irreversible Processes]
  \item[]
  \item[]
  \item[]
  \item[]
  \item[]
\end{description}

\section{Thermodynamics}

\subsection{Irreversibility}
\ignore{
  Do the facts about entropy go beyond what is found in Newtonian physics or are they derivable from them?
  Is the Second Law a physical theory or a statistical one?
  How, if ever, does entropy increase?
  Are there absolutely irreversible processes?
}%ignore

    \subsection{}

\section{Special Relativity}

A superficial understanding of the claims of special relativity is easy to achieve.
A few simple formulae tell us about the claims of the theory in terms of strange laws for addition of velocities, time dilation, mass increases and the related conversion between mass and energy.
Making sense of it all, after beginning life with a newtonian understanding of the laws of motion and gravitation is not quite so straightforwaed, and this I have never achieved.

There are many elementary explantions which seek to make special relativity intelligible to the non-specialist, but these have never made any sense to me.
The main stumbling block is with the idea that the speed of light is the same relative to every intertial frame.
This seems to be presented as if it were an empirical observation, but it is in fact a logically incoherent claim in the context of the only concept of space and time which we have prior to special relativity.
It is simply not possible for it to be an experimental observation, though presumably the synthesis of a conception of space in which it is logically coherent could not have happened without experimental observations which could not otherwise be expained.

Certain experimemtal observations are cited to establish the thesis.
These observations are held to refute the two alternatives which are apparent under the prior understanding of space and time.
These are, that light travels in vacuo at a fixed speed relative to some medium (the ether) which may be considered to be at absolute rest, and the second is that it travels with a fixed speed relative to its source.

The evidence against the first alternative is the Michelson-Morely Experiment, which I won't go into because I don't have a problen with it.
The evidence against the second possibility is cited as certain observations of twin stars, and nby contrast with the Michelson-Morely experiment is not explained in elementary accounts.
A few years back I looked at the paper cited, and found its rationale completely unconvincing.

With those two alternatives disposed of we are then presented with Einstein's resolution to the difficulties, which is a major transformation of our conception of space and time.
This is huge, I'm sure that most would agree.
A leap which surely no-one would take without very solid evidence to establish that no more conventional explanation could suffice.
In my particular case, this solution is one which I have never fully understood.

That is my starting point for this exploration, in which I propose to look very closely at Einstein's own painstaking account for the lay person of his theory, with a view to arriving at a good understanding of what is claimed and of the evidence cited for it.

In doing this, I will find it necessary not only to look carefully at the stages in Einstein's explanations, but also to give an account of my prior understanding of space and time, so that the reason's for any difficulties I have with Einstein's alternative may be clearly exposed.

I'm going to review Einstein's account of special relativity and try to amke sense of it, and in the process explain any difficulties I have, as well as those which remain at the end.
The explanation of my own difficulties with this os likely to depend in part on my prior beliefs and understandings which seem to conflict with the theory, and which are not dispelled by Einstein's explanations.

There will be one of three outcomes.
I may still feel that I just don't understand the theory.
I may come to understand the theory, but remain unconvinced of its truth.
I might be convinced of its truth but not of its necessity.
I might be wholly convinced that it is mot only true, but the best account of the truth.

\subsection{Starting Point}

A principal aim of science is to enable prediction.
To achieve some end, it is helpful to construct a plan, the execution of which will lead to the desired outcome.
If you are confident that the execution of you plan is feasible and that it will yield the desired outcome then you can marshall and apply the resources required in the plan.

The ability to undertake this kind of goal oriented behaviour predates by hundreds of thousands of years the invention of science, but the purpose of science is to enable this kind of activity to be undertaken more successfully and on a larger (or smaller) scale.

Science facilitates the process by providing ``laws'' which describe regularities in the behaviour of the world which can be exploited in the construction and execution of such plans.
To articulate laws of nature we need language suitable for precisely describing the state if the universe, or some part of it, at one moment, and thence describe how the state evolves over time thenceforth.

To describe the state of the universe it is necessary to enumerate the things which occur in the system, describe carefully the nature and state of each object, and its place.


\phantomsection
\addcontentsline{toc}{section}{Bibliography}
\bibliographystyle{rbjfmu}
\bibliography{rbj}

%\addcontentsline{toc}{section}{Index}\label{index}
%{\twocolumn[]
%{\small\printindex}}

%\vfill

%\tiny{
Started 2021/12/05


\href{http://www.rbjones.com/rbjpub/www/papers/p034.pdf}{http://www.rbjones.com/rbjpub/www/papers/p034.pdf}

%}%tiny

\end{document}

% LocalWords:
