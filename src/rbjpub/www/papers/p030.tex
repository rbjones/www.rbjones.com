% $Id: p030.tex $
% bibref{rbjp030} pdfname{p030}

\documentclass[14pt,titlepage]{extarticle}
\usepackage{fontspec}
\setmainfont{ProofPowerSerif.ttf}[Path=/Users/rbj/.fonts/]
\usepackage{makeidx}
\usepackage{graphicx}
\usepackage[unicode]{hyperref}
\pagestyle{plain}
\usepackage[paperwidth=8.3in,paperheight=11.7in,hmargin={0.3in,0.3in},vmargin={0.5in,0.5in},includehead,includefoot]{geometry}
\hypersetup{pdfauthor={Roger Bishop Jones}}
\hypersetup{pdftitle={Challenges and Opportunities for Practical Philosophy in the 21st Century}}
\hypersetup{colorlinks=true, urlcolor=red, citecolor=blue, filecolor=blue, linkcolor=blue}
%\usepackage{html}
\usepackage{paralist}
\usepackage{relsize}
\usepackage{verbatim}
\usepackage{enumerate}
\makeindex
\newcommand{\ignore}[1]{}

\title{Challenges and Opportunities \\for \\Practical Philosophy \\in \\the 21st Century}
\author{Roger~Bishop~Jones}
\date{\ }


\begin{document}
%\frontmatter

%\begin{abstract}
% A number of factors, conspicuous as we make our way into the 21st century, seem to make a coherent philosophical underpinning for moral, political and economic affairs especially difficult to formulate.
% These notes are intended to facilitate discussion of the challenges and opportunities and a range of possible philosophical responses to them.
%\end{abstract}
                               
\begin{titlepage}
\maketitle

%\vfill

%\begin{centering}

%{\footnotesize
%copyright\ Roger~Bishop~Jones;
%}%footnotesize

%\end{centering}

\end{titlepage}

\ \

\ignore{
\begin{centering}
{\LARGE \bf Challenges \\for \\Practical Philosophy \\in \\the 21st Century\\}
\end{centering}
}%ignore

\setcounter{tocdepth}{1}
{\parskip-0pt\tableofcontents}

%\listoffigures

%\mainmatter

%\pagebreak

\section{Introduction}

We live in times of rapid and accelerating change.
The changes which we now see in progress disrupt established ethical norms and erode the consensus underpinning our policial and economic institutions.

That diversity of custom which lead to relativistic or sceptical philosophical tendencies among the Greek sophists becomes more pressing as technology shrinks the globe.
The diversity which once was seen only when travelling across large distances is now found throughout the planet as mobility and electronic media spread these different customs into environments where they were previously unknown.

The challenges these changes offer to practical philosophy, the underpinnings of our moral political and economic systems, are substantial.

To facilitate discussion I propose to present various aspects of the challenges, some opportunties enabled by the changes, and some ideas for synthesising a way forward.

\section{Challenges}

There are very many aspects of the growing chaos which confronts us and now and threaten to overwhelm our institutions and customs.

A sweeping overall description of the problems might be:

\begin{quote}
Fragmentaton and Polarisation
\end{quote}

Here's an attempt to group them together into general categories.

\

\hfill\begin{minipage}{\dimexpr\textwidth-1cm}
\begin{description}
\item[(α)] the juxtaposition of conflicting cultural and religious diversity
\item[(β)] widening wealth disparities engendered by rapid technological change
\item[(γ)] other widening divides: young/old, progressive/conservative
\end{description}
\end{minipage}

\section{Opportunities}

\hfill\begin{minipage}{\dimexpr\textwidth-1cm}
\begin{description}
\item[(α)] IOT information explosion and transparency
\item[(β)] evolution/erosion/collapse of democracy
\item[(γ)] impact analysis
\item[(δ)] constitutional evolution
\end{description}
\end{minipage}

\section{Syntheses}

\subsection{Priorities}

It important to be clear about priorities, otherwise we may make small advances at the expense of major retrenchments.

Some of the bad things that can happen are liable to mess up everything in very comprehensive ways and so we should think very hard about how to avoid them.
These days many people wll think first in this context of ecological issues like climate change, but it seems to me that there are even worse things than that on the cards, which among their other disadvantages will impair our ability to respond to challenges like global warming.

At the top of my list is war, which is recognised as the principal cause of famine.
Second I would list lawlessness, especially violent crime, or violent conflict within society.

\section{Some Analysis}

%\addcontentsline{toc}{section}{Bibliography}
%\bibliographystyle{alpha}
%\bibliography{rbj2}

%\addcontentsline{toc}{section}{Index}\label{index}
%{\twocolumn[]
%{\small\printindex}}

%\vfill

%\tiny{
%Started 2017-10-09

%Last Change 2017-10-09

%\href{http://www.rbjones.com/rbjpub/www/papers/p028.pdf}{http://www.rbjones.com/rbjpub/www/papers/p028.pdf}

%}%tiny

\end{document}

% LocalWords:
