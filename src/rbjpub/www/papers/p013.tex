% $Id: p013.tex,v 1.1 2008/07/18 14:15:45 rbj Exp $ bibref{rbjp013} pdfname{p013} 
\documentclass[numreferences]{rbjk}
\usepackage{makeidx}
\newcommand{\ignore}[1]{}
\usepackage[unicode,pdftex]{hyperref}
\hypersetup{pdfauthor={Roger Bishop Jones}}
\hypersetup{colorlinks=true, urlcolor=black, citecolor=black, filecolor=black, linkcolor=black}

%\newtheorem{def}{Definition}
%\newtheorem{conj}{Conjecture}

\makeindex
\begin{document}                                                                                   
\begin{article}
\begin{opening}  
\title{Abstract Ontology}
%\runningtitle{Abstract Ontology}
\author{Roger Bishop \surname{Jones}}
\date{$ $\ $ $}
%\runningauthor{Roger Bishop Jones}

\begin{abstract}
A consideration of what abstract entities we should work with.
\end{abstract}

\end{opening}

\vfill

\begin{centering}
\footnotesize{
Created 2008/07/17

Last Change $ $Date: 2008/07/18 14:15:45 $ $

\href{http://www.rbjones.com/rbjpub/www/papers/p013.pdf}{http://www.rbjones.com/rbjpub/www/papers/p013.pdf}

$ $Id: p013.tex,v 1.1 2008/07/18 14:15:45 rbj Exp $ $\\

}%footnotesize
\end{centering}

\newpage
%\def\tableofcontents{{\parskip=0pt\@starttoc{toc}}}
\setcounter{tocdepth}{4}
{\parskip-0pt\tableofcontents}

\section{Introduction}

Abstract entities are fundamental to deductive reasoning, to the foundations of mathematics and hence to large parts of science and engineering.

Our starting point for this discussion is that abstract entities are in many respect like fictions, they are things which we assume to exist for various purposes, rather than things which we objectively discover.
The are not entirely like fictions, for for fictions are usually things which definitely don't exist.
Sherlock Homes is a detective featuring in certain books, and I think it is true to say that there never was such a man, the things we say of him are fictions and they are false.
By contrast, if in set theory we affirm the existence of the empty set, its not so obvious that are asserting a falsehood.

The question whether abstract entities such as the empty set {\it really} exist is {\it metaphysical}.
It is a matter of controversy whether it has a definite meaning, and it is not my present concern.

For present purposes I put it to you that it is convenient in various ways (e.g. for the conduct of mathematics and hence for all those things which mathematics facilitates) to assume the existence of certain abstract objects, and I intend to discuss which such assumptions it is most advantageous to adopt.

\section{The Role of Abstract Objects}

\section{Well-Founded Sets}

\section{Sets}

An ontology of well-founded sets probably does provide the best context to address the most critical foundational issue, that of consistency.

However, in the conduct of mathematics we need in the course of investigating the various kinds of abstract entitu.we need certain other abstract entities to be fulfill auxiliary roles in our investigation.


\subsection{Pragmatic Considerations}

\subsection{The Problem of Consistency}

\subsection{The Construction of Models} 

%{\raggedright
%\bibliographystyle{klunum}
%\bibliography{rbjk}
%} %\raggedright

\twocolumn[\section{Index}\label{Index}]
{\small\printindex}

\end{article}
\end{document}
