% $Id: p013.tex,v 1.3 2010/09/13 11:07:50 rbj Exp $ bibref{rbjp013} pdfname{p013} 

\documentclass[numreferences]{book}
\usepackage{makeidx}
\usepackage{graphicx}
\def\indexname{Index}
\newcommand{\ignore}[1]{}
\pagestyle{headings}
\usepackage[twoside,paperwidth=5.25in,paperheight=8in,hmargin={0.75in,0.5in},vmargin={0.5in,0.5in},includehead,includefoot]{geometry}
\usepackage[unicode,pdftex]{hyperref}
\hypersetup{pdfauthor={Roger Bishop Jones}}
\hypersetup{colorlinks=true, urlcolor=red, citecolor=blue, filecolor=blue, linkcolor=blue}


\makeindex
\newcommand{\indexentry}[2]{\item #1 #2}

\title{Abstract Ontology}
%\runningtitle{Abstract Ontology}
\author{Roger Bishop Jones}
\date{\ }

\begin{document}                                                    
\frontmatter
\begin{titlepage}
\maketitle

%\begin{abstract}
%An informal discussion of abstract ontology.
%\end{abstract}

\vfill

\begin{centering}

\footnotesize{
Created 2008/07/17

Last Change $ $Date: 2010/09/13 11:07:50 $ $

\href{http://www.rbjones.com/rbjpub/www/papers/p013.pdf}{http://www.rbjones.com/rbjpub/www/papers/p013.pdf}

$ $Id: p013.tex,v 1.3 2010/09/13 11:07:50 rbj Exp $ $\\

}%footnotesize
\end{centering}
\end{titlepage}

\newpage
%\def\tableofcontents{{\parskip=0pt\@starttoc{toc}}}
\setcounter{tocdepth}{3}
{\parskip-0pt\tableofcontents}

\mainmatter
\section{Introduction}

Abstract entities are fundamental to deductive reasoning, to the foundations of mathematics and hence to large parts of science and engineering.
More generally even than that (for all but radical scientists) they are fundamental to some conceptions of analytic method of the broadest scope.

Our starting point for this discussion is that abstract entities are in many respect like fictions, they are things which we assume to exist for various purposes, rather than things which we objectively discover.
They are not \emph{entirely} like fictions, for fictions are usually things which definitely don't exist (except possibly \emph{as fictions}).
Sherlock Homes is a detective featured in certain books, and I think it is true to say that there never was such a man, the things we say of him are fictitious and hence false, unless we speak \emph{of the fiction} (as I do here) rather than \emph{of the man}.
By contrast, if in set theory we affirm the existence of the empty set, its not so obvious that we are asserting a falsehood.

The question whether abstract entities such as the empty set {\it really} exist is the kind of question which some philosophers have called  {\it metaphysical}.
It is a matter of controversy whether it has a definite meaning, but it is not my present concern.

For present purposes I put it to you that it is convenient in various ways (e.g. for the conduct of mathematics and hence for all those things which mathematics facilitates) to assume the existence of certain abstract objects, and I intend to discuss which such assumptions it is most advantageous to adopt.
Or at least, I intend to describe to you the various deliberations on this topic in which I have engaged over the past quarter century, and some of the conclusions I have drawn from these deliberations.

\section{The Role of Abstract Objects}

The discussion of abstract ontology will be undertaken in the context of the following ideas about the role of abstract entities.

The broadest conception of that role can be seen in two distinct forms according to whether one takes a universalist or a pluralistic attitude towards language.

\begin{itemize}
\item[1A] For modelling concrete (or other) systems.
\item[1B] For giving to languages an abstract semantics.
\end{itemize}

We think of \emph{1A} as a description of the application of a single general purpose logical system to the great diversity of problems which can be subject to analysis by ``nomologico-deductive'' methods, i.e. by constructing an abstract model of some subject matter (which conforms to certain `laws' observed in the target of the model).

We think of \emph{1B} in the case that problem or domain oriented languages are adopted for modelling, or (in the manner espoused by Rudolf Carnap) when some special language has been devised for the purpose of formalising perhaps an entire scientific discipline.
Deduction in these less universal languages may be underpinned by using abstract objects (possibly in some universalist context) to define aN abstract semantics for the languages and thereby make definite the property of soundness essential to a deductive system for the language.
Giving an abstract semantics to a language which is intended to talk about the concrete world need not, and normally would not be intended merely to provide a technical means for establishing consistency of a deductive system for that language.
It would be normal for the chosen abstract ontology to mirror the relevant structure of the intended concrete subject matter in such a way as to enable a soundness proof relative to the abstract semantics to provide confidence in the soundness of the deductive system relative to the intended concrete subject matter.

The importance of abstract semantic is not confined to its role in supporting the credentials of formal languages with concrete interpretations.
The realm of abstract entities is itself a subject matter in its own right, and may be said to be the subject matter of mathematics as a whole (though not all philosophers of mathematics would concur with this).
Certain fundamental topics in mathematics can only be fully addressed if semantics is taken further in precision than is likely to be necessary for any concrete application.
An important topic of this kind is cardinal arithmetic, in which it is now common to consider meaningless question which are not decided by the first order axioms known as ZFC\index{ZFC}.
However, if it is accepted that certain informal semantic notions are sufficiently definite, in particular the notion of ``all subsets'', then cardinal arithmetic becomes categorical and many important questions (such as the Continuum Hypothesis and the Gerenalised Continuum Hypothesis) are thereby rendered meaningful.
This procedure is an extension of the idea that ``true arithmetic'' (the sentences of the first order theory of arithmetic which are true in relation to the natural numbers) is well defined even though it can have no complete deductive system.

These two perspectives are by no means distinct, the suggested pluralism may be universalistic in relation to semantic foundations.
The two are complementary, and there are pragmatic approaches which fall at various points between the two, among which important examples are methods for supporting pluralism by semantic embedding in universalistic frameworks.

These general perspectives on the role of abstract entities can be supplemented by identifying a number of slightly more specific roles.

They are:

\begin{enumerate}
\item For establishing the consistency of definitions.
\item For comparing structures.
\item For re-use of abstract ideas and the large scale structuring of specifications or theories.
\end{enumerate}

\paragraph{modelling}

Certain important kinds of knowledge can reasonably be presented as consisting in our having abstract models of various aspects of the concrete world.
The value in such abstract models is in permitting us to anticipate the consequences of future actions and with this information to chose those courses of action whose outcomes we prefer.

The advantage of abstract models in this context is in their precision.
This enables us to reason reliably about their characteristics and about the behaviour of any real phenomena of which they provide a good model.

In constructing such a model abstract objects serve as surrogates for the concrete entities in the real world (or of whatever kind of entities we might want to reason about).

\paragraph{consistency}

This kind of modelling is often clearly \emph{mathematical} (sometimes not) and there is a second useful perspective on the role of abstract objects in this context.
In the logical development of mathematics, various mathematical concepts are defined and the consequences of the definitions are then explored by deduction.
It is important that what we say in a definition is \emph{consistent}, for otherwise any conclusion may flow from the definitions.
The usual method used to establish consistency of such definitions is to exhibit something which satisfies the definition, and a substantial abstract ontology is convenient for this purpose.

\paragraph{comparisons}

In order to build and reason about abstract models of the concrete world it is helpful to develop abstract theories of the various kinds of abstract entities which appear in such models.
In the making of such comparisions abstract objects have a special role.
The correspondences between abstract entities which feature in such comparisons are themselves abstract objects, and an insufficient ontology may cause a comparison to fail through no pertinent dissimilarity in the objects compared.

\paragraph{re-use}

The problem of \emph{re-use} arises from constraints which have been imposed in logical system to ensure the consistency of the logic.
Typically these are constraints on abstraction, and hence on the range of abstract objects which can be shown to exist, which are intended to avoid incoherent ontology but which also eliminate some consistent and convenient objects and impose difficulties in abstract mathematica.
Perhaps the best publicised of these is now the difficulties arising in category in relation to categories embracing all algebras with a certain signature, or even the category of categories.


\section{Well-Founded Sets}

\section{Sets}

An ontology of well-founded sets probably does provide the best context to address the most critical foundational issue, that of consistency.

However, in the conduct of mathematics we need in the course of investigating the various kinds of abstract entitu.we need certain other abstract entities to be fulfill auxiliary roles in our investigation.


\subsection{Pragmatic Considerations}

\subsection{The Problem of Consistency}

\subsection{The Construction of Models} 

\backmatter
%{\raggedright
%\bibliographystyle{klunum}
%\bibliography{rbjk}
%} %\raggedright

\addcontentsline{toc}{chapter}{Index}\label{index}
{\twocolumn[]
{\small\printindex}}

\end{document}
