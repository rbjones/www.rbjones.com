% $Id: p013.tex,v 1.2 2010/08/17 12:52:11 rbj Exp $ bibref{rbjp013} pdfname{p013} 
\documentclass[numreferences]{rbjk}
\usepackage{makeidx}
\newcommand{\ignore}[1]{}
\usepackage[unicode,pdftex]{hyperref}
\hypersetup{pdfauthor={Roger Bishop Jones}}
\hypersetup{colorlinks=true, urlcolor=blue, citecolor=blue, filecolor=blue, linkcolor=blue}

%\newtheorem{def}{Definition}
%\newtheorem{conj}{Conjecture}

\makeindex
\begin{document}                                                                                   
\begin{article}
\begin{opening}  
\title{Abstract Ontology}
%\runningtitle{Abstract Ontology}
\author{Roger Bishop \surname{Jones}}
\date{$ $\ $ $}
%\runningauthor{Roger Bishop Jones}

\begin{abstract}
A consideration of what abstract entities we should work with.
\end{abstract}

\end{opening}

\vfill

\begin{centering}
\footnotesize{
Created 2008/07/17

Last Change $ $Date: 2010/08/17 12:52:11 $ $

\href{http://www.rbjones.com/rbjpub/www/papers/p013.pdf}{http://www.rbjones.com/rbjpub/www/papers/p013.pdf}

$ $Id: p013.tex,v 1.2 2010/08/17 12:52:11 rbj Exp $ $\\

}%footnotesize
\end{centering}

\newpage
%\def\tableofcontents{{\parskip=0pt\@starttoc{toc}}}
\setcounter{tocdepth}{4}
{\parskip-0pt\tableofcontents}

\section{Introduction}

Abstract entities are fundamental to deductive reasoning, to the foundations of mathematics and hence to large parts of science and engineering.

Our starting point for this discussion is that abstract entities are in many respect like fictions, they are things which we assume to exist for various purposes, rather than things which we objectively discover.
The are not entirely like fictions, for fictions are usually things which definitely don't exist (except possibly \emph{as a fiction}).
Sherlock Homes is a detective featured in certain books, and I think it is true to say that there never was such a man, the things we say of him are fictions and they are false, unless we speak \emph{of the fiction} (as I do here) rather than \emph{of the man}.
By contrast, if in set theory we affirm the existence of the empty set, its not so obvious that we are asserting a falsehood.

The question whether abstract entities such as the empty set {\it really} exist is {\it metaphysical}.
It is a matter of controversy whether it has a definite meaning, and it is not my present concern.

For present purposes I put it to you that it is convenient in various ways (e.g. for the conduct of mathematics and hence for all those things which mathematics facilitates) to assume the existence of certain abstract objects, and I intend to discuss which such assumptions it is most advantageous to adopt.
Or at least, I intend to describe to you the various deliberations on this topic in which I have engaged over the past quarter century.

\section{The Role of Abstract Objects}

The discussion of abstract ontology will be undertaken in the context of the following conception of the role of abstract entities.

The broadest conception of that role can be seen in two distinct forms according to whether one takes a universalist or a pluralistic attitude towards language.

\begin{itemize}
\item[1A] For modelling concrete (or other) systems.
\item[1B] For giving to languages an abstract semantics.
\end{itemize}

We think of \emph{1A} as a description of the application of a single general purpose logical system to the great diversity of problems which can be subject to analysis by ``nomologico-deductive'' methods, i.e. by constructing an abstract model of some subject matter (which conforms to certain `laws' observed in the target of the model).

We think of \emph{1B} in the case that problem or domain oriented languages are adopted for modelling.
In this case deduction in these less universal languages may be underpinned by using abstract objects to define a semantics for the languages and thereby make definite the property of soundness essential to a deductive system for the language.

These two perspectives are by no means distinct, the suggested pluralism may be universalistic in relation to semantic foundations.
The two are complementary, and there are pragmatic approaches which fall at various points between the two, among which a important examples are methods for supporting pluralism by semantic embedding in universalistic frameworks.

These very general perspectives on the role of abstract entities can be supplemented by identifying a number of slightly more specific roles.

They are:

\begin{enumerate}
\item For establishing the consistency of definitions.
\item For comparing structures.
\item For re-use of abstract ideas and the large scale structuring of specifications or theories.
\end{enumerate}

\paragraph{modelling}

Certain important kinds of knowledge can reasonably be presented as consisting in our having abstract models of various aspects of the concrete world.
The value in such abstract models is in permitting us to anticipate the consequences of future actions and with this information to chose those courses of action whose outcomes we prefer.

The advantage of abstract models in this context is in their precision.
This enables us to reason reliably about their characteristics and about the behaviour of any real phenomena of which they provide a good model.

In constructing such a model abstract objects serve as surrogates for the concrete entities in the real world (or of whatever kind of entities we might want to reason about).

\paragraph{consistency}

This kind of modelling is often clearly \emph{mathematical} (sometimes not) and there is a second useful perspective on the role of abstract objects in this context.
In the logical development of mathematics, various mathematical concepts are defined and the consequences of the definitions are then explored by deduction.
It is important that what we say in a definition is \emph{consistent}, for otherwise any conclusion may flow from the definitions.
The usual method used to establish consistency of such definitions is to exhibit something which satisfies the definition, and a substantial abstract ontology is convenient for this purpose.

\paragraph{comparisons}

In order to build and reason about abstract models of the concrete world it is helpful to develop abstract theories of the various kinds of abstract entities which appear in such models.
In the making of such comparisions abstract objects have a special role.
The correspondences between abstract entities which feature in such comparisons are themselves abstract objects, and an insufficient ontology may cause a comparison to fail through no pertinent dissimilarity in the objects compared.

\paragraph{re-use}




\section{Well-Founded Sets}

\section{Sets}

An ontology of well-founded sets probably does provide the best context to address the most critical foundational issue, that of consistency.

However, in the conduct of mathematics we need in the course of investigating the various kinds of abstract entitu.we need certain other abstract entities to be fulfill auxiliary roles in our investigation.


\subsection{Pragmatic Considerations}

\subsection{The Problem of Consistency}

\subsection{The Construction of Models} 

%{\raggedright
%\bibliographystyle{klunum}
%\bibliography{rbjk}
%} %\raggedright

\twocolumn[\section{Index}\label{Index}]
{\small\printindex}

\end{article}
\end{document}
