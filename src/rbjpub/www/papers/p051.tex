% $Id: p051.tex $fi
% bibref{rbjp051} pdfname{p051}
\documentclass[10pt,titlepage]{article}
\usepackage{makeidx}
\newcommand{\ignore}[1]{}
\usepackage{graphicx}
\usepackage[unicode]{hyperref}
\pagestyle{plain}
\usepackage[paperwidth=5.25in,paperheight=8in,hmargin={0.75in,0.5in},vmargin={0.5in,0.5in},includehead,includefoot]{geometry}
\hypersetup{pdfauthor={Roger Bishop Jones}}
\hypersetup{pdftitle={Epistemic Architecture}}
\hypersetup{colorlinks=true, urlcolor=red, citecolor=blue, filecolor=blue, linkcolor=blue}
%\usepackage{html}
\usepackage{paralist}
\usepackage{relsize}
\usepackage{verbatim}
\usepackage{enumerate}
\usepackage{longtable}
\usepackage{url}
\newcommand{\hreg}[2]{\href{#1}{#2}\footnote{\url{#1}}}
\makeindex

\title{\LARGE\bf Epistemic Architecture}
\author{Roger~Bishop~Jones}
\date{\small 2022/12/09}


\begin{document}

%\begin{abstract}
% A discussion of architectural principles for widely distributed collaborative knowledge representation and exploitation systems.
%\end{abstract}
                               
\begin{titlepage}
\maketitle

%\vfill

%\begin{centering}

%{\footnotesize
%copyright\ Roger~Bishop~Jones;
%}%footnotesize

%\end{centering}

\end{titlepage}

\ \

\ignore{
\begin{centering}
{}
\end{centering}
}%ignore

\setcounter{tocdepth}{2}
{\parskip-0pt\tableofcontents}

%\listoffigures

\pagebreak

\section*{Preface}

\addcontentsline{toc}{section}{Preface}

One of a series looking for ways to do synthetic epistemology as architecture of knowledge.

\footnote{There may be ``hyperlinks'' in the PDF version of this monograph which either link to another point in the document  (if coloured blue) or to an internet resource  (if coloured red) giving direct access to the materials referred to (e.g. a Youtube video) if the document is read using some internet connected device.
Important links also appear explicitly in the bibiography.}

\section{Introduction}

We are at a moment in history when epistemology has become controversial, and is a locus of exceptional risk and opportunity.

The history of `Western' philosophy is replete with opposing tendencies and attempts at synthesis which themselves provoke critique, antithesis and demand further synthesis.
These opposing tendencies typically have epistemological cores, which we may 


The postmodern philosophy of Michel Foucault denies objective merit to epistemological standards, but the prosperity of human society is primarily attributable to advances in science and technology which have appeared only in what is now called Western culture.
The advances to which Western culture 


My interest here is in how to manage knowledge when the following technological developments become significant:

\begin{itemize}
\item The deductive capability of artificial intelligence reaches a level at which it provides effective access to the deductive closure of any body of knowledge which we seek to exploit.
\item That intelligent deductive capability is widely distributed through this and possibly into other galaxies by self replicating intelligent systems.
\end{itemize}



My main interest in this essay is to identify some of the layers in which the various issues of concern from a logical point of view might be structured.
Thus, no attention is given to the physical layers which are concerned, for example, with the technology used to store, retrieve, process and communicate information or knowledge.

Those layers are:

\begin{enumerate}
\item the data layer: a widely distributed heirarchical WORM (write-once-read-many) storage system, associating identifiers with values.
\item the logical name-space: a heirarchical name space in which names denote abstract entities rather than data values, and are characterised by conservative constraints in.

\item derived logical truth
\item empirical modelling
\item commerce
\item valuations and norms
\end{enumerate}

\section{Data Layer}

In order to give an account of how under this model distinct idependent epistemic epicenters may combine into a single coherent body of knowledge, we imagine there to be multiple indendent systems distinguished geographically, possibly by more than one 

\appendix

\phantomsection
\addcontentsline{toc}{section}{Bibliography}
\bibliographystyle{rbjfmu}
\bibliography{rbj}

%\addcontentsline{toc}{section}{Index}\label{index}
%{\twocolumn[]
%{\small\printindex}}

%\vfill

\tiny{
Started 2022/12/09


\href{http://www.rbjones.com/rbjpub/www/papers/p051.pdf}{http://www.rbjones.com/rbjpub/www/papers/p051.pdf}

}%tiny

\end{document}

% LocalWords:
