% $Id: p051.tex $fi
% bibref{rbjp051} pdfname{p051}
\documentclass[10pt,titlepage]{article}
\usepackage{makeidx}
\newcommand{\ignore}[1]{}
\usepackage{graphicx}
\usepackage[unicode]{hyperref}
\pagestyle{plain}
\usepackage[paperwidth=5.25in,paperheight=8in,hmargin={0.75in,0.5in},vmargin={0.5in,0.5in},includehead,includefoot]{geometry}
\hypersetup{pdfauthor={Roger Bishop Jones}}
\hypersetup{pdftitle={Epistemic Architecture}}
\hypersetup{colorlinks=true, urlcolor=red, citecolor=blue, filecolor=blue, linkcolor=blue}
%\usepackage{html}
\usepackage{paralist}
\usepackage{relsize}
\usepackage{verbatim}
\usepackage{enumerate}
\usepackage{longtable}
\usepackage{url}
\newcommand{\hreg}[2]{\href{#1}{#2}\footnote{\url{#1}}}
\makeindex

\title{\LARGE\bf Epistemic Architecture}
\author{Roger~Bishop~Jones}
\date{\small 2022/12/09}


\begin{document}

%\begin{abstract}
% A discussion of architectural principles for widely distributed collaborative knowledge representation and exploitation systems.
%\end{abstract}
                               
\begin{titlepage}
\maketitle

%\vfill

%\begin{centering}

%{\footnotesize
%copyright\ Roger~Bishop~Jones;
%}%footnotesize

%\end{centering}

\end{titlepage}

\ \

\ignore{
\begin{centering}
{}
\end{centering}
}%ignore

\setcounter{tocdepth}{2}
{\parskip-0pt\tableofcontents}

%\listoffigures

\pagebreak

\section*{Preface}

\addcontentsline{toc}{section}{Preface}

One of a series looking for ways to do synthetic epistemology as architecture of knowledge.

\footnote{There may be ``hyperlinks'' in the PDF version of this monograph which either link to another point in the document  (if coloured blue) or to an internet resource  (if coloured red) giving direct access to the materials referred to (e.g. a Youtube video) if the document is read using some internet connected device.
Important links also appear explicitly in the bibliography.}

\section{Introduction}

We are at a moment in history when epistemology has become controversial, and is a locus of exceptional risk and opportunity.

The history of `Western' philosophy is replete with opposing tendencies and attempts at synthesis which themselves provoke critique, antithesis and demand further synthesis.
These opposing tendencies often have epistemological cores.

The postmodern philosophy of Michel Foucault denies objective merit to epistemological norms, but the prosperity of human society is substantially attributable to advances in science and technology which the accidents of history predominantly located in certain geographic regions before their utility and power progressed them across the globe.
In terms of their impact on human well being and their contribution to the fulfillment of human aspirations, I cannot myself acquiesce in cultural indifference.

It is moreover a premise of this essay and of the architecture which it sketches, that the most fundamental principles of epistemology are determined not by culture, but by the nature of propositional language and declarative knowledge.
They have come to be understood explicitly only after millenia of industry in appropriate cultural mileu, and are ignored or perverted at cost to humanity,

My interest here is in how to manage knowledge when the following technological developments become significant:

\begin{itemize}
\item The deductive capability of artificial intelligence reaches a level at which it provides effective access to the deductive closure of any body of knowledge which we seek to exploit.
\item That intelligent deductive capability is widely distributed through this and possibly into other galaxies by self replicating intelligent systems.
\end{itemize}

My main interest in this essay is to identify some of the layers in which the various issues of concern from a logical point of view might be structured.
Thus, no attention is given to the physical layers which are concerned, for example, with the technology used to store, retrieve, process and communicate information or knowledge.

Those layers are:

\begin{enumerate}
\item the data layer: a widely distributed heirarchical WORM (write-once-read-many) storage system, associating identifiers with values.
\item the logical name-space: a heirarchical name space in which names denote abstract entities rather than data values, and are characterised by conservative constraints in.
\item abstract syntax
\item core logic
\item abstract ontology
\item logical truth
\item empirical modelling
\item commerce
\item valuations and norms
\end{enumerate}

\section{Data Layer}

In order to give an account of how under this model distinct independent epicenters may combine into a single coherent body of knowledge, we imagine there to be multiple systems distinguished geographically but connected logically, though possibly into more than one logically distinct region.

Each such region is a write-once read-many (WORM) information storage structure with an addressing system broadly similar to our current global internet or to typical digital computer file storage, in which items of information are identified using a hierarchical naming system using finite sequence of names to locate specific items of information.

In this system there will be no fixed top level domain, since connection of logically distinct regions will involve adding higher level domains to one or both of the two name structures.
Addressing will always be relative to a node in the hierarchy, so that addresses within a region are not affected by such a merger.

That this is a write-once structure is necessary to ensure integrity and coherence in the face of distributed asynchronous updates.
Whenever a modification is made at a node it yields in a new node with the same symbolic path but a different sequence number leaving the node with the same symbolic path but distinct sequence number unchanged, and leaving any structure containing or referring to it likewise unchanged.
This will be similar to the effect of an update to a structure performed by a purely functional program, which can only compute a new value without changing the old, and could therefore deliver a history of successive iterations as a list with the laterst version always added at the head of the list.
In that case the references are all memory references rather than symbolic references, and therefore version numbers for named values are not necessary, but in this proposal the relevant references are symbolic, and so a new name is necessary for each new value, and this is obtained by associating a sequence number to the name.
This sequence number will always be larger than the sequence number of any previous version of that named value, and will also be larger than the sequence number of any other value directly or indirectly referred to by the structure.
So not strictly sequence numbers.

\section{Abstract Syntax}

The data layer is a kind of naming system, in which complex names are associated with particular items (possibly large) of stored data.
These names, when fully qualified by sequence numbers, behave like logical constants.
They can be referred to in algorithms to be executed or in the specification of other values, in propositions expressed in suitable propositional languages, and may thence become significant in logical deductions and other means of inference.

In order to be able to talk or write about these values using their names we need languages.
The integrity of the knowledge and the inferences we undertake with it depends on precision in the languages, and the soundness of the deductive systems employed.
In this proposal this may be realised by reduction to a \emph{foundation system}, a logical system with a semantics rich enough for that of most other languages to be expressible in it, and with a deductive system which is sound and strong.

This foundation system need not have a fixed concrete syntax, and many other languages may be understood as special ways of writing down in concrete syntax expressions which belong to the foundational logic.

The abstract syntax which we place at this fundamental place in the architecture is that of a simple polymorphic typed lambda calculus, and the logic itself will be a close derivative of Church's Simple Theory of Types \cite{Church40}.


\appendix

\phantomsection
\addcontentsline{toc}{section}{Bibliography}
\bibliographystyle{rbjfmu}
\bibliography{rbj}

%\addcontentsline{toc}{section}{Index}\label{index}
%{\twocolumn[]
%{\small\printindex}}

%\vfill

\tiny{
Started 2022/12/09


\href{http://www.rbjones.com/rbjpub/www/papers/p051.pdf}{http://www.rbjones.com/rbjpub/www/papers/p051.pdf}

}%tiny

\end{document}

% LocalWords:
