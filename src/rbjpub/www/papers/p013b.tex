% $Id: p013b.tex,v 1.2 2011/03/11 21:43:51 rbj Exp $

In the first instance this part will consist of no more than a sketch of the modern history as I understand it.

This will cover relevant aspects of the work of:

\begin{itemize}
\item Cantor
\item Frege
\item Russell
\item Hilbert
\item Carnap
\item Schonfinkel
\item Quine
\item Church
\item Curry
\item Woodin
\end{itemize}

The historical sketch will evolve and be filled out as necessary to cover those aspects of the history which are relevant background to my principle narrative on abstract ontology.

For two millennia the mathematics of magnitudes developed with a substantially incomplete understanding of the nature of these magnitudes.
The invention of the infinitesimal calculus inaugurated a new era involving new conceptual difficulties in form of infinitesimal magnitudes alongside a considerable expansion in the development of analysis for application in science.
Despite foundational weakness these developments were highly successful, and it was not until the nineteenth century that mathematicians addressed the foundational issues and put analysis on a sound basis.
This was done first by the elimination of infinitesimals (later shown not to be strictly necessary, but nevertheless a valuable simplification), and then by a precise definition of a number system adequate for the mathematics of magnitude, by construction in stages from the natural numbers.

\section{Woodin}

\subsection{Dehornoy's Survey}

