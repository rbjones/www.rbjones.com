% $Id: p023.tex,v 1.2 2015/04/23 09:58:09 rbj Exp $
% bibref{rbjp023} pdfname{p023}

\documentclass[10pt,titlepage]{article}
\usepackage{makeidx}
\usepackage{graphicx}
\usepackage[unicode,pdftex]{hyperref}
\pagestyle{plain}
\usepackage[paperwidth=5.25in,paperheight=8in,hmargin={0.75in,0.5in},vmargin={0.5in,0.5in},includehead,includefoot]{geometry}
\hypersetup{pdfauthor={Roger Bishop Jones}}
\hypersetup{colorlinks=true, urlcolor=red, citecolor=blue, filecolor=blue, linkcolor=blue}
\usepackage{html}
\usepackage{paralist}
\usepackage{relsize}
\usepackage{verbatim}
\makeindex
\newcommand{\ignore}[1]{}

\title{RBJ Blog drafts}
\author{Roger~Bishop~Jones}
\date{\ }

\begin{document}
%\frontmatter
                               
\begin{titlepage}
\maketitle

\begin{abstract}
This is a place for drafting materials to be posted to my Roger Bishop Jones blog, or elsewhere.
\end{abstract}

%\vfill

%\begin{centering}

%{\footnotesize
%copyright\ Roger~Bishop~Jones;
%}%footnotesize

%\end{centering}

\end{titlepage}

\setcounter{tocdepth}{2}
{\parskip-0pt\tableofcontents}

%\listoffigures

%\mainmatter

\section{Introduction}



\section{Set Theory}

\subsection{}

\section{Thoughts on the History of The Triple-Dichotomy}

\subsection{What Kind of History?}

I am myself more interested in the future than the past, and am in various ways unsuited to historical scholarship.

However, in attempting to articulate a philosophical framework for the future, for the sake of intelligibility I have felt it desirable to relate the intended philosophical position to the history of the subject.
In attempting to do this I find that the superficial nature of my knoweldge of past philosophers is a big problem.

Even if I had a more complete historical knowledge, I don't believe this would fix my problem.
What is happening instead is that my idea of what is needed is gradually being refined, and my conception of what is needed is evolving.

It may be worth referring to the manner in which Kripke addressed a problem raised by Wittgenstein.
He wanted to say something about Wittgenstein's ``private language argument'' without entering into the
question of exactly what Wittgenstein meant.
He considered (I believe) certain things which Wittgenstein \emph{possibly} have meant, and discussed them, without claiming to know whether or not these really were what Wittgenstein had intended.

Kripke described his enterprise thus:

\begin{quote}
``... the present paper should be thought of as expounding neither `Wittgenstein's' argument nor `Kripke's': rather Wittgenstein's argument as it struck Kripke, as it presented a problem for him.''
\end{quote}

In this case we have a historical story for which accuracy is not claimed, or even sought.
The history here is not what actually happened, but a presentation of how the history appears to an author concerned with the triple dictotomy as a cornerstone for a contemporary philosophical system.
An approximation to the history of the topic intended to illuminate the nature and significance of the contemporary ideas by comparison with related ideas which preceded them.

In the exposition of the philosophical position which I have here in mind, one not so far in many respects, from the central tenet's of Rudolf Carnap, it will be helpful to contrast the proposed position with certain alternative views which may be thought of as caricatures of the positions of previous philosophers on the issues at stake.
In this, simple contrasts will serve our purposes best, so that they may be readily grasped by the reader and will then serve to clarify my proposal by contrast.
Philosophers often spend a lifetime developing and articulating their philosophical ideas, and contemporary scholars may devote a large part of a lifetime to the interpretation of an important philosopher.
An accurate account of such a philosopher will not serve our purposes, but using the name of a philosopher as a label for a position closer to his than any other may.

It is the ``triple dichotomy'' which I seek to illuminate in some such manner.
In doing so I propose to begin with a short statment of position, which is to say, a proposal for how the terminology might best be used, and then to further elucidate the proposal and its merits through an account of how it might have evolved, using idealised acccounts of positions not far from those which helped form our present understanding of the issues.

There is to this an informal and a formal side, this being the informal side.
Of the more formal side I suggest that the adoption of the proposed concepts is \emph{without loss of generality}, though I lack anything close to a proof of that claim.
In a sense, it is of the essence that I feel able to make that claim, since an element of the complex of ideas is foundational.
The foundational claim is that in respect of semantics and deductive proof there are systems, or closely related familiies of systems which offer near universal foundations to which alternatives are reducible (somehow).


\subsection{What is it?}

By the ``triple dichotomy'' I mean the following three distinctions:

\begin{itemize}
\item necessary-contingent
\item \emph{a priori}-\emph{a posteriori}
\item analytic-synthetic
\end{itemize}

Of course philosophers may not always have had the same thing in mind when they used these terms,
i.e. they might have had distinct understandings of the \emph{meanings} of the terms, as well as having divergent views on their \emph{extension}s.
Disentangling from what a philosopher has written where he stands on these is very difficult, and I suspect that often philosphers do not have definite views to be discovered.
It is possible to have definite views on what falls under these concepts without having such a definite understanding of what they mean.

It is the concepts as they were understood by Rudolf Carnap which concern us here.
This understanding corresponding quite well the distinction at the centre of David Hume's philosophy sometimes called ``Hume's fork'' (though Carnap's presentation is more refined and precise).
This point exposes one of the difficulties which face comparison between philosophical positions widely separated in time.
There is a world of difference between what Carnap says about the analytic/synthetic distinction and what Hume says about his fork.
There are two ways in which a concept may evolve.
It may evolve into a different concept, or it may evolve into a more precisely drawn exposition of the same concept.

How can it be the same?
I suggest a weak and a strong sense, and assert them both.
Consider the descriptions of the concepts as ``loose''; as not picking out the concept uniquely but as encompassing a collection of concepts which satisfy that same description.
A second ``tighter'' (but still ``loose'') description may be considered a refined description of ``the same'' concept if every concept which satisfies the second description also satisfies the earlier one.
To use the term ``same'' in that sense is controversial, since that kind of ``sameness'' is assymetric.
For a stronger notion of sameness, we might accept that a loose description is intended to refer to a single concept but fails to be sufficiently precise, i.e. is not ``definite'' in the technical sense of being satisfied by just one concept.
Two such descriptions may be said to be of the same concept if we believe that the same single concept is intended by both descriptions.
I doubt that Hume and Carnap had the same concepts in mind in that stronger sense, so I will venture an intermediate notion of sameness, even at risk of introducing ideas which both philosophers might dislike.
I believe that among the concepts which satisfy the two descriptions there is one concept which is much more important, philosophically in some sense more fundamental that the others, and that the two philosophers have that single concept in mind.

Both Hume and Carnap were positivists who rejected certain conceptions of metaphysics.
The rejection is more fully worked out in Carnap, who replaced metaphysics with semantics, making the meaning of concepts a matter of convention embodied in the semantics of languages.
These conventions were to be chosen on pragmatic grounds, but Carnap's idea of the pragmatic encompassed any consideration short of \emph{metaphysics} in that particular and narrow sense in which Carnap uses the term (which allows us free choice of ontology in language design without regarding such a choice as constituting metaphyics).
Despite this rejection of metaphysics, it is clear that both philosophers placed the analytic/synthetic distinction at the center of their philosophy.
There is here a choice which, even if technically considered ``pragmatic'' is in some way fundamental, and which we might reasonably regard as prior to the choice of linguistic conventions; not relative to choice of language, but absolute in character, perhaps even metaphysical.

The roots of this distinction I suggest lie in a particularly fecund abstraction from natural language which provides a basis for the idealised language in which formal science (both \emph{a priori} and \emph{a posteriori}) might be undertaken.


I think of the history of the evolution of these concepts as having its most prominent landmarks in the philosophies of Plato, Aristotle, Leibniz, Hume, Kant, Frege, Wittgenstein and Carnap.
So this history hangs around that list of philosophers.

I have only a superficial knowledge of the history of philosophy.
This is not a contribution to the history of philosophy, but rather, an attempt to elucidate these concepts, which we may regard as mediated by stories about fictional philosophers faintly similar to these well known figures.

\subsection{Entailment and the origins of Deduction}

The notion of analyticity, around which this essay pivots, is closely coupled to two other concepts which may be thought of as its origins.
These in turn are coeval with a particular kind of language or use of language, which we may call ``descriptive''. ``propositional'' or ``truth conditional''.
Wittgenstein's philosophy begin's in his Tractatus with this kind of language, but later repudiates this as a model for language in genera; in favour of a much broader conception of language.
Those broader non-descriptive uses of language remain enormously important and probably preceeded descriptive language, but in certain respects the descriptive use of language is an important advance crucial to the development of mathematics, science and technology.

Descriptive language works in the following way.
Descriptions of the world are communicated by sentences which express propositions which have truth values.
These sentences are like expressions in a programming language which evaluate to a boolean value making use of information about the world as it is.
When one person knows something about the world not known to another, he can communicate this by asserting a proposition which only evaluates to ``true'' if that fact obtains.
If the hearer understands the language he will understand under what conditions a proposition will come out ``true'' and, if he trust the speaker he will correctly infer the fact that the speaker sought to communicate.

For this to work, people have to understand the meaning of words, phrases and sentences in the language.
The meaning of a sentence may be thought of as its truth conditions, the conditions under which it will ``evaluate'' to ``true''.
The meaning of denoting phrases may be thought of as some way of deriving from the state of the world some object referred to by the phrase.
Knowing the meaning of a collective noun is knowing how to tell when something falls under the noun.
Some of these nouns have related meanings.
We know for example, just from our understanding of language, that all mammals are animals, and that cows and sheep are animals.
It is thus an aspect of competence in such a descriptive language that we are able to make certain inferences or deductions, which reflect our knowledge of entailments between the propositions expressed by sentences of the language.

Thus, if we are told that all meats in our local supermarket are reduced to half price, then when we buy pork we will expect to receive that reduction.
We know that the meaning of ``meats'' encompasses ``pork'', so we infer without thought from ``all meats'' to ``all porK''.

I suggest therefore, that competence in a descriptive language brings with it the ability to recognise elementary entailments, and that descriptive language is in that sense co-eval with a kind of deduction.
This does not entail being conscious of such abilities, being able to talk about them, being capable of complex or systematic deductive inference. or being wholly reliable in the inferences we make.

Let me clarify in this context what we mean by ``deductive'', to strengthen the connection with this stage in the development of language.






\subsubsection{Plato}

Long before Plato, Greek mathematicians and philosophers became aware of the power of deductive reasoning, primarily by using it to establish mathematical theorems.
When philosophers tried to reason about the word, things went awry.
Different philosophers arrived at completely opposite conclusions.
A notable example close to Plato's time was the disagreement between Parmenides and Heraclitus, of whom the first held that nothing changes, and the second that nothing remains the same.

Plato tried to square this circle.
He believed in mathematics, in philosophy, and in deductive reasoning as a means to knowledge in both spheres, and he came up with a philosophical system which explained why deduction worked well in mathematics but failed miserably as it had been practiced by philosophers beyond mathematics, and which offered a basis for reliable philosophical reasoning.

Plato had concluded that there were two different worlds which he thought of as the \emph{real} word, a world of ideas or forms accessible to the intellect through reason, and a illusory world of sensory appearances.
The former world was eternal, changeless and accessible to true knowledge, the latter fleeting, in perpetual flux, a matter of opinion rather than true knowledge.

This \emph{metaphysic} explains why philosphers have disagreed, and offers a way in which philosophy can be like mathematics, by undertaking what we might now call \emph{logical analysis}.
In relation to Parmenides and Heralclitus, there is here a kind of resolution to their conflict.
We may think of Parmenides as speaking of the world of forms, eternal and unchangeable, and Heraclitus as confused by the world of appearances, ephemeral and deceptive, forever changing.

More important perhaps is the possibility that philosophers, with the benefit of Plato's wisdom, will confine their thinking to the proper domain of reason, the world of ideal forms, and thereby realise that success so far only found in mathematics, definitive resolutions to philosophical problems.

Plato was of course an enormously influential philosopher, and I wouldn't venture an opinion about the most important part of his philosophical legacy.
But for the purposes of this little narrative on the tripl-dichotomy, what stands out to me is the connection with the views of David Hume.
For David Hume's description of his ``fork'', which we will come to later, is a description which connects with Plato's.
In particular, Plato's characterisation of the difference between reliable knowledge and fleeting opinion is in terms of its subject matter.
Hume does not use the same language to describe these domains, but he too distinguishes in terms of subject matter, and what he calls ``relations between ideas'' is not entirely dissimilar to the study of Plato's ideal forms.
There are big differences, but some similarities of importance.

\subsection{Aristotle}

Here are two aspects of Aristotle's philosophy which seem to me to connect.

The parts of Aristotle's philosophy which seem most relevant are The Organon, a group of six works related to logic, and the Metaphysics.
That part of the Metaphysics which seems most relevant is his Categories, and earlier approach to which is also present in The Organon.

The Organon presents the idea of ``Demonstrative Science'', which we may think of as Aristotle's replacement for Plato's ``real'' world of ideal forms.
It connects with Plato in this area because the two are both concerned with ``reality'', though different ideas on the nature of reality. and because both of these provide a body of truths obtainable by deduction.

Two differences may be noted.
Aristotle provides much greater (unprecedented) detail on the process of deduction.
This appears in Aristotle's ``syllogistic'', the first ever approach to a formal deductive system.
Unfortunately, Aristotle's deductive logic is inadequate for any non-trivlal application, so if we use this to delimit the scope of logical truth then we have a concept far from the notion of logical truth to be found in the triple-dichotomy.
This is a recurring difficulty in the history of the concept.
If the a priori or analyticity are thought of ``proof theoretically'', i.e. as defined by what can be formally derived, then for most of our historical narrative it will be narrowly scoped.

A second area of relevent divergence between Plato and Aristotle is in the nature of the premises from which deductions may proceed.
In Plato, our knowledge of the definitions of the concepts which inhabit the world of forms comes from anamnesis.
Anamnesis refers to the memory possessed by immortal souls from previous incarnations.
Despite this implausible account of our knowledge of definitions, we still have it that our knowledge of the world of ideal forms is analytic, a prioir and necessary.
In Aristotle's demonstrative science, the source of the ``first principles'' is different, but more importantly, the knowledge is of first principles of the various sciences, and it seems doubtful that such first principles can be just definitions of scientific concepts, and we may fear that such principles and the conclusions drawn from them should be considered a posteriori, synthetic and contingent.

On this basis the apparent influence of the idea of demonstrative proof in David Hume's anticipation of the triple dichotomy in his ``fork'' is remarkable


\subsection{Leibniz}



\subsubsection{Conceptual Containment Suffices}

A part of Kant account of the concept of analyticity (which is is credited with having introduced, though I'm not sure he did much more than chose the word to attach to an idea present in Leibniz and Hume) is the assertion that a sentence is analytic if the predicate is contained in the subject.

This seems to entail that all analytic sentences are in Aristotelian subject-predicate form, and this leads many to the belief that later authors, e.g. Frege, had broader concepts of analyticity.

Of course, Kant did deny the analyticity of arithmetic whereas Frege asserted it.
The question is, were they talking at cross purposes (as we might say if their
disagreement was covertly about the meaning of the term ``analytic''), or did they
have a substantive disagreement concerning the extension of the term whilst having
a common understanding of its meaning.

My purpose here is to offer an argument to the effect that ``conceptual containment'' does
provide a reasonable informal account of the notion of analyticity in its broadest sense
(even that of Carnap, which is as broad as it gets), rather than the kind of narrow
conception of analyticity in which analyticity is confined to rather trivial tautologies.

Even though not all sentences are of subject-predicate form (a fact well known to Kant)
they are all logically equivalent to sentences in subject predicate form.


\section{Some History of Necessity}

\section{On Artificial Intelligence}

\subsection{Blending Approaches}

Throughout the history of AI there have been many many different approaches
both to specific problems and to the problem of ``general intelligence''.
An often used categorisation of the approaches distinguishes those which
approach AI by copying or modelling the way the human brain works from
those who approach these problems in ways quite different to the ways in
which human brains work.
The natural labels to give to these are ``neural'' and ``logical''.

Let me point out some principle advantages and disadvantages of these approaches.
If we try to emulate brains and succeed then the kind of intelligence we can expect
is pretty much the same as human intelligence.
Which is illogical and error prone, slow to learn and apt to forget.
If we go the other way, we can keep the accuracy and speed for which digital computers
are know (provided the software is good!), but we might not get the intelligence,
and we might find ourselves unable to program seemingly trivial tasks like
understanding verbal language and recognition of the features in a visual field
(these were not thought to involve intelligence until the AI community found they
were hard to do, after all, even very stupid people do these things!).

Now I'm interesting in getting machines to do as reliably as I know they
can do, some things which can only be done by a minority of people
and are generally and correctly recognised as demanding intellectual challenges.
Among these, proving mathematical theorems, and a variety of technology
design problems.
Useful progress can be made in these areas just by \emph{automation} by
devising and coding up algorithms which do these things better and better.
But still, we find usually find that the best we can do in these areas
still leaves the machines failing to achieve tasks which are simple
for the right kind of person (mathematician, engineer).
When we look into the detail, we often find at its core the need to
find a solution in a space which is very very large.
Brute search is computatinally infeasible, but people can often just
pull the needle out of the haystack.

This is what one hopes neural networks might supply.
Lets suppose that they do.
Do we have to put up with the rest, are can we get the needle finding
goodies and discard the unreliability and error?

Its pretty easy to see ways in which this can be done.
In many problem domains we can use unreliable but possibly inspired
ways of finding solutions, alongside pedestrian but reliable algorithms
which can check where some inspired solution really does solve the
problem.
Sometimes checking the solution is tractably algorithmic but is subject
to ``proof''.
Mathematical proof is the same kind of ``find and check'' process, but
this time we know that the ``check'' part is algorithmic.
To make this work outside of mathematics we use mathematical models.

Here are some capabilities:

\begin{itemize}
\item{Neural Networks}
Learning, associative memory, guessing, picking needles out of haystacks.
\item{RAM. SSD}
Storing stuff. non-associative.
\item{CPU, GPU}
Algorithmic computation, often numeric.
\item{Theorem Proving, GOFAI}
Needs to be integrated with neural nets to narrow the search spaces.
\end{itemize}

Some neural nets advocates (perhaps Jeff Hawkins) might tell us that the entire gamut
of intelligent capabilities can and should be done by neural nets or their superefficient
superabundant electronic counterparts.
But intelligent people don't usually do themselves tasks which are
better done by a computer.
An intelligent neural network can just as well have access to run-of-the-mill
processing power (probably the open-ended resources of the cloud, though possibly
not without budgetary constraint), and will be just as able to see that algorithmic
computation sometimes beats neural nets hands down.
Equally, anyone, man or machine, programming a task which involves intractible
search, will, if it is available, be glad to use an effective neural net whether
soft or hard, to cut his search spaces down.

\section{Metphysical Positivism}

I have for many years used the term ``Metaphysical Positivism'' as a label for the
theoretical side of my positive philosophy.
Calling this ``positivism'' is an aknowledgement of my sense of connection with the
philosophy of Rudolf Carnap (even though the later Carnap prefer to consider himself
an empiricist, perhaps reflecting a slight moderation in maturity).
The qualification ``Metaphysical'' is harder to explain, for it suggests, perhaps misleadingly,
that my attitude towards metaphysics is very different to Carnap's.
The difference between us is almost entirely verbal, so the adjective should be taken as
indicating abstinence from Carnap's rather particular use of the term.

Before making this terminological difference more definite, let me describe concisely the ground
on which I find common cause with Carnap.
I take as the central motivation of Carnap's philosophy the desire to mediate philosophically
in the transformation of the empirical sciences through the adoption of the new logical methods
which Carnap first discovered in the work of Frege.
This was his interpretation of the programme advocated by Russell for ``scientific philosophy'',
that philosophers should realise throught the adoption of formal languages the precision of language
and soundness of reasoning which had so far eluded them, and should facilitate rigour in empirical science by
adapting these methods for those sciences.

\section{Positive Philosophy - 2014-12-19}

This is a really compact survey of \emph{positive philosophy} as I conceive it.

First let me assert the primacy of the practical.
Positive philosophy is intended to help us to decide what to do and how to do it.

However, a major part of positive knowledge is theoretical.
It concerns knowledge, of matters and methods which might possibly prove instrumental for some of us in realising our practical goals.

So, in the presentation of positive philosophy the distinction between theoretical and practical is important.

\section{Positive Philosophy - Chosing a Future}

Philosophy deals with, alongside various minutiae, the greatest problems which face mankind.
It can be approached in a piecemeal fashion, but the greatest works of philosophy are usually in one way or another, \emph{systematic}.

When philosophy is approached systematically (as I hope to do here), it may be possible to encapsulate the philosophical enterprise into a single overarching question.
A popular candidate might be ``what is the meaning of life?''.

For the philosopher adopting this kind of approach, which we might call ``ultimately systematic'', the choice of that initial overarching question, in response to which an entire philosophical system is expected to ensue, will be important.
That is my present concern, and I will approach that question by trial and error.

Let me begin with ``what should we do?''.

The best thing to do depends upon what we hope to achieve, itself influenced by our values and our circumstances, each of which is personal.
Our first candidate is therefore made awkward to answer by its subjective element.
The philosopher, rather than addressing the particular values and circumstances of each individual he considers might prefer to offer some more general wisdom which he supposes will be helpful to individuals in coming to their own determination of how to conduct their lives.

Stepping back from the particular in this way is characteristic of much philosophy, and once the subjective is here aknowledged it may come back again and again to thwart the philosophers search for broad and general truths.
The full aknowledgement of the individual's right to chose, particularly when taken to the extremes which may be found in forms of anarchism, may make it a difficult philosophical problem to adopt any coherent definite position on how society might best be organised.
This is a part of the problem which I hope to grapple here.

\section{Some Philosophy}

I intend here to approach the articulation of a \emph{systematic} philosophy through the analysis of a single question.

It is in the nature of philosophy as I percieve it to have ideals which it can at best imperfectly approach.
That this treatise deliver a clearly articulated systematic whole is one such.
I shall try not to spend too much space apologising for this and similar shortcomings, but there are many points of similar tension which I think worth mentioning.
These tensions might even amount to a subtheme of the text, but it is too early now to elaborate on this point.

The single question I have in mind is ``What should we do?''.

What is special about this question that its analysis might play so crucual role in systematic philosophy?

%\backmatter

%\appendix

%\addcontentsline{toc}{section}{Bibliography}
%\bibliographystyle{alpha}
%\bibliography{rbj}

%\addcontentsline{toc}{section}{Index}\label{index}
%{\twocolumn[]
%{\small\printindex}}

%\vfill

%\tiny{
%Started 2012-10-19

%Last Change $ $Date: 2015/04/23 09:58:09 $ $

%\href{http://www.rbjones.com/rbjpub/www/papers/p019.pdf}{http://www.rbjones.com/rbjpub/www/papers/p019.pdf}

%Draft $ $Id: p023.tex,v 1.2 2015/04/23 09:58:09 rbj Exp $ $
%}%tiny

\end{document}

% LocalWords:
