% $Id: p023.tex$
% bibref{rbjp023} pdfname{p023}

\documentclass[10pt,titlepage]{article}
\usepackage{makeidx}
\usepackage{graphicx}
\usepackage[unicode,pdftex]{hyperref}
\pagestyle{plain}
\usepackage[paperwidth=5.25in,paperheight=8in,hmargin={0.75in,0.5in},vmargin={0.5in,0.5in},includehead,includefoot]{geometry}
\hypersetup{pdfauthor={Roger Bishop Jones}}
\hypersetup{colorlinks=true, urlcolor=red, citecolor=blue, filecolor=blue, linkcolor=blue}
%\usepackage{html}
\usepackage{paralist}
\usepackage{relsize}
\usepackage{verbatim}
\makeindex
\newcommand{\ignore}[1]{}

\title{RBJ Blog drafts}
\author{Roger~Bishop~Jones}
\date{\ }

\begin{document}
%\frontmatter
                               
\begin{titlepage}
\maketitle

%\begin{abstract}
%This is a place for drafting materials to be posted to my Roger Bishop Jones blog, or elsewhere.
%\end{abstract}

%\vfill

%\begin{centering}

%{\footnotesize
%copyright\ Roger~Bishop~Jones;
%}%footnotesize

%\end{centering}

\end{titlepage}

\setcounter{tocdepth}{2}
{\parskip-0pt\tableofcontents}

%\listoffigures

%\mainmatter

\section{Introduction}



\section{Set Theory}

\subsection{}

\section{Thoughts on the History of The Triple-Dichotomy}

\subsection{What Kind of History?}

I am myself more interested in the future than the past, and am in various ways unsuited to historical scholarship.

However, in attempting to articulate a philosophical framework for the future, for the sake of intelligibility I have felt it desirable to relate the intended philosophical position to the history of the subject.
In attempting to do this I find that the superficial nature of my knoweldge of past philosophers is a big problem.

Even if I had a more complete historical knowledge, I don't believe this would fix my problem.
What is happening instead is that my idea of what is needed is gradually being refined, and my conception of what is needed is evolving.

It may be worth referring to the manner in which Kripke addressed a problem raised by Wittgenstein.
He wanted to say something about Wittgenstein's ``private language argument'' without entering into the
question of exactly what Wittgenstein meant.
He considered (I believe) certain things which Wittgenstein \emph{possibly} have meant, and discussed them, without claiming to know whether or not these really were what Wittgenstein had intended.

Kripke described his enterprise thus:

\begin{quote}
``... the present paper should be thought of as expounding neither `Wittgenstein's' argument nor `Kripke's': rather Wittgenstein's argument as it struck Kripke, as it presented a problem for him.''
\end{quote}

In this case we have a historical story for which accuracy is not claimed, or even sought.
The history here is not what actually happened, but a presentation of how the history appears to an author concerned with the triple dictotomy as a cornerstone for a contemporary philosophical system.
An approximation to the history of the topic intended to illuminate the nature and significance of the contemporary ideas by comparison with related ideas which preceded them.

In the exposition of the philosophical position which I have here in mind, one not so far in many respects, from the central tenet's of Rudolf Carnap, it will be helpful to contrast the proposed position with certain alternative views which may be thought of as caricatures of the positions of previous philosophers on the issues at stake.
In this, simple contrasts will serve our purposes best, so that they may be readily grasped by the reader and will then serve to clarify my proposal by contrast.
Philosophers often spend a lifetime developing and articulating their philosophical ideas, and contemporary scholars may devote a large part of a lifetime to the interpretation of an important philosopher.
An accurate account of such a philosopher will not serve our purposes, but using the name of a philosopher as a label for a position closer to his than any other may.

It is the ``triple dichotomy'' which I seek to illuminate in some such manner.
In doing so I propose to begin with a short statment of position, which is to say, a proposal for how the terminology might best be used, and then to further elucidate the proposal and its merits through an account of how it might have evolved, using idealised acccounts of positions not far from those which helped form our present understanding of the issues.

There is to this an informal and a formal side, this being the informal side.
Of the more formal side I suggest that the adoption of the proposed concepts is \emph{without loss of generality}, though I lack anything close to a proof of that claim.
In a sense, it is of the essence that I feel able to make that claim, since an element of the complex of ideas is foundational.
The foundational claim is that in respect of semantics and deductive proof there are systems, or closely related familiies of systems which offer near universal foundations to which alternatives are reducible (somehow).


\subsection{What is it?}

By the ``triple dichotomy'' I mean the following three distinctions:

\begin{itemize}
\item necessary-contingent
\item \emph{a priori}-\emph{a posteriori}
\item analytic-synthetic
\end{itemize}

Of course philosophers may not always have had the same thing in mind when they used these terms,
i.e. they might have had distinct understandings of the \emph{meanings} of the terms, as well as having divergent views on their \emph{extension}s.
Disentangling from what a philosopher has written where he stands on these is very difficult, and I suspect that often philosphers do not have definite views to be discovered.
It is possible to have definite views on what falls under these concepts without having such a definite understanding of what they mean.

It is the concepts as they were understood by Rudolf Carnap which concern us here.
This understanding corresponding quite well the distinction at the centre of David Hume's philosophy sometimes called ``Hume's fork'' (though Carnap's presentation is more refined and precise).
This point exposes one of the difficulties which face comparison between philosophical positions widely separated in time.
There is a world of difference between what Carnap says about the analytic/synthetic distinction and what Hume says about his fork.
There are two ways in which a concept may evolve.
It may evolve into a different concept, or it may evolve into a more precisely drawn exposition of the same concept.

How can it be the same?
I suggest a weak and a strong sense, and assert them both.
Consider the descriptions of the concepts as ``loose''; as not picking out the concept uniquely but as encompassing a collection of concepts which satisfy that same description.
A second ``tighter'' (but still ``loose'') description may be considered a refined description of ``the same'' concept if every concept which satisfies the second description also satisfies the earlier one.
To use the term ``same'' in that sense is controversial, since that kind of ``sameness'' is assymetric.
For a stronger notion of sameness, we might accept that a loose description is intended to refer to a single concept but fails to be sufficiently precise, i.e. is not ``definite'' in the technical sense of being satisfied by just one concept.
Two such descriptions may be said to be of the same concept if we believe that the same single concept is intended by both descriptions.
I doubt that Hume and Carnap had the same concepts in mind in that stronger sense, so I will venture an intermediate notion of sameness, even at risk of introducing ideas which both philosophers might dislike.
I believe that among the concepts which satisfy the two descriptions there is one concept which is much more important, philosophically in some sense more fundamental that the others, and that the two philosophers have that single concept in mind.

Both Hume and Carnap were positivists who rejected certain conceptions of metaphysics.
The rejection is more fully worked out in Carnap, who replaced metaphysics with semantics, making the meaning of concepts a matter of convention embodied in the semantics of languages.
These conventions were to be chosen on pragmatic grounds, but Carnap's idea of the pragmatic encompassed any consideration short of \emph{metaphysics} in that particular and narrow sense in which Carnap uses the term (which allows us free choice of ontology in language design without regarding such a choice as constituting metaphyics).
Despite this rejection of metaphysics, it is clear that both philosophers placed the analytic/synthetic distinction at the center of their philosophy.
There is here a choice which, even if technically considered ``pragmatic'' is in some way fundamental, and which we might reasonably regard as prior to the choice of linguistic conventions; not relative to choice of language, but absolute in character, perhaps even metaphysical.

The roots of this distinction I suggest lie in a particularly fecund abstraction from natural language which provides a basis for the idealised language in which formal science (both \emph{a priori} and \emph{a posteriori}) might be undertaken.


I think of the history of the evolution of these concepts as having its most prominent landmarks in the philosophies of Plato, Aristotle, Leibniz, Hume, Kant, Frege, Wittgenstein and Carnap.
So this history hangs around that list of philosophers.

I have only a superficial knowledge of the history of philosophy.
This is not a contribution to the history of philosophy, but rather, an attempt to elucidate these concepts, which we may regard as mediated by stories about fictional philosophers faintly similar to these well known figures.

\subsection{Entailment and the origins of Deduction}

The notion of analyticity, around which this essay pivots, is closely coupled to two other concepts which may be thought of as its origins.
These in turn are coeval with a particular kind of language or use of language, which we may call ``descriptive''. ``propositional'' or ``truth conditional''.
Wittgenstein's philosophy begin's in his Tractatus with this kind of language, but later repudiates this as a model for language in genera; in favour of a much broader conception of language.
Those broader non-descriptive uses of language remain enormously important and probably preceeded descriptive language, but in certain respects the descriptive use of language is an important advance crucial to the development of mathematics, science and technology.

Descriptive language works in the following way.
Descriptions of the world are communicated by sentences which express propositions which have truth values.
These sentences are like expressions in a programming language which evaluate to a boolean value making use of information about the world as it is.
When one person knows something about the world not known to another, he can communicate this by asserting a proposition which only evaluates to ``true'' if that fact obtains.
If the hearer understands the language he will understand under what conditions a proposition will come out ``true'' and, if he trust the speaker he will correctly infer the fact that the speaker sought to communicate.

For this to work, people have to understand the meaning of words, phrases and sentences in the language.
The meaning of a sentence may be thought of as its truth conditions, the conditions under which it will ``evaluate'' to ``true''.
The meaning of denoting phrases may be thought of as some way of deriving from the state of the world some object referred to by the phrase.
Knowing the meaning of a collective noun is knowing how to tell when something falls under the noun.
Some of these nouns have related meanings.
We know for example, just from our understanding of language, that all mammals are animals, and that cows and sheep are animals.
It is thus an aspect of competence in such a descriptive language that we are able to make certain inferences or deductions, which reflect our knowledge of entailments between the propositions expressed by sentences of the language.

Thus, if we are told that all meats in our local supermarket are reduced to half price, then when we buy pork we will expect to receive that reduction.
We know that the meaning of ``meats'' encompasses ``pork'', so we infer without thought from ``all meats'' to ``all porK''.

I suggest therefore, that competence in a descriptive language brings with it the ability to recognise elementary entailments, and that descriptive language is in that sense co-eval with a kind of deduction.
This does not entail being conscious of such abilities, being able to talk about them, being capable of complex or systematic deductive inference. or being wholly reliable in the inferences we make.

Let me clarify in this context what we mean by ``deductive'', to strengthen the connection with this stage in the development of language.






\subsubsection{Plato}

Long before Plato, Greek mathematicians and philosophers became aware of the power of deductive reasoning, primarily by using it to establish mathematical theorems.
When philosophers tried to reason about the word, things went awry.
Different philosophers arrived at completely opposite conclusions.
A notable example close to Plato's time was the disagreement between Parmenides and Heraclitus, of whom the first held that nothing changes, and the second that nothing remains the same.

Plato tried to square this circle.
He believed in mathematics, in philosophy, and in deductive reasoning as a means to knowledge in both spheres, and he came up with a philosophical system which explained why deduction worked well in mathematics but failed miserably as it had been practiced by philosophers beyond mathematics, and which offered a basis for reliable philosophical reasoning.

Plato had concluded that there were two different worlds which he thought of as the \emph{real} word, a world of ideas or forms accessible to the intellect through reason, and a illusory world of sensory appearances.
The former world was eternal, changeless and accessible to true knowledge, the latter fleeting, in perpetual flux, a matter of opinion rather than true knowledge.

This \emph{metaphysic} explains why philosphers have disagreed, and offers a way in which philosophy can be like mathematics, by undertaking what we might now call \emph{logical analysis}.
In relation to Parmenides and Heralclitus, there is here a kind of resolution to their conflict.
We may think of Parmenides as speaking of the world of forms, eternal and unchangeable, and Heraclitus as confused by the world of appearances, ephemeral and deceptive, forever changing.

More important perhaps is the possibility that philosophers, with the benefit of Plato's wisdom, will confine their thinking to the proper domain of reason, the world of ideal forms, and thereby realise that success so far only found in mathematics, definitive resolutions to philosophical problems.

Plato was of course an enormously influential philosopher, and I wouldn't venture an opinion about the most important part of his philosophical legacy.
But for the purposes of this little narrative on the tripl-dichotomy, what stands out to me is the connection with the views of David Hume.
For David Hume's description of his ``fork'', which we will come to later, is a description which connects with Plato's.
In particular, Plato's characterisation of the difference between reliable knowledge and fleeting opinion is in terms of its subject matter.
Hume does not use the same language to describe these domains, but he too distinguishes in terms of subject matter, and what he calls ``relations between ideas'' is not entirely dissimilar to the study of Plato's ideal forms.
There are big differences, but some similarities of importance.

\subsection{Aristotle}

Here are two aspects of Aristotle's philosophy which seem to me to connect.

The parts of Aristotle's philosophy which seem most relevant are The Organon, a group of six works related to logic, and the Metaphysics.
That part of the Metaphysics which seems most relevant is his Categories, and earlier approach to which is also present in The Organon.

The Organon presents the idea of ``Demonstrative Science'', which we may think of as Aristotle's replacement for Plato's ``real'' world of ideal forms.
It connects with Plato in this area because the two are both concerned with ``reality'', though different ideas on the nature of reality. and because both of these provide a body of truths obtainable by deduction.

Two differences may be noted.
Aristotle provides much greater (unprecedented) detail on the process of deduction.
This appears in Aristotle's ``syllogistic'', the first ever approach to a formal deductive system.
Unfortunately, Aristotle's deductive logic is inadequate for any non-trivlal application, so if we use this to delimit the scope of logical truth then we have a concept far from the notion of logical truth to be found in the triple-dichotomy.
This is a recurring difficulty in the history of the concept.
If the a priori or analyticity are thought of ``proof theoretically'', i.e. as defined by what can be formally derived, then for most of our historical narrative it will be narrowly scoped.

A second area of relevent divergence between Plato and Aristotle is in the nature of the premises from which deductions may proceed.
In Plato, our knowledge of the definitions of the concepts which inhabit the world of forms comes from anamnesis.
Anamnesis refers to the memory possessed by immortal souls from previous incarnations.
Despite this implausible account of our knowledge of definitions, we still have it that our knowledge of the world of ideal forms is analytic, a prioir and necessary.
In Aristotle's demonstrative science, the source of the ``first principles'' is different, but more importantly, the knowledge is of first principles of the various sciences, and it seems doubtful that such first principles can be just definitions of scientific concepts, and we may fear that such principles and the conclusions drawn from them should be considered a posteriori, synthetic and contingent.

On this basis the apparent influence of the idea of demonstrative proof in David Hume's anticipation of the triple dichotomy in his ``fork'' is remarkable


\subsection{Leibniz}



\subsubsection{Conceptual Containment Suffices}

A part of Kant account of the concept of analyticity (which is is credited with having introduced, though I'm not sure he did much more than chose the word to attach to an idea present in Leibniz and Hume) is the assertion that a sentence is analytic if the predicate is contained in the subject.

This seems to entail that all analytic sentences are in Aristotelian subject-predicate form, and this leads many to the belief that later authors, e.g. Frege, had broader concepts of analyticity.

Of course, Kant did deny the analyticity of arithmetic whereas Frege asserted it.
The question is, were they talking at cross purposes (as we might say if their
disagreement was covertly about the meaning of the term ``analytic''), or did they
have a substantive disagreement concerning the extension of the term whilst having
a common understanding of its meaning.

My purpose here is to offer an argument to the effect that ``conceptual containment'' does
provide a reasonable informal account of the notion of analyticity in its broadest sense
(even that of Carnap, which is as broad as it gets), rather than the kind of narrow
conception of analyticity in which analyticity is confined to rather trivial tautologies.

Even though not all sentences are of subject-predicate form (a fact well known to Kant)
they are all logically equivalent to sentences in subject predicate form.


\section{Some History of Necessity}

\section{On Artificial Intelligence}

\subsection{Blending Approaches}

Throughout the history of AI there have been many many different approaches
both to specific problems and to the problem of ``general intelligence''.
An often used categorisation of the approaches distinguishes those which
approach AI by copying or modelling the way the human brain works from
those who approach these problems in ways quite different to the ways in
which human brains work.
The natural labels to give to these are ``neural'' and ``logical''.

Let me point out some principle advantages and disadvantages of these approaches.
If we try to emulate brains and succeed then the kind of intelligence we can expect
is pretty much the same as human intelligence.
Which is illogical and error prone, slow to learn and apt to forget.
If we go the other way, we can keep the accuracy and speed for which digital computers
are know (provided the software is good!), but we might not get the intelligence,
and we might find ourselves unable to program seemingly trivial tasks like
understanding verbal language and recognition of the features in a visual field
(these were not thought to involve intelligence until the AI community found they
were hard to do, after all, even very stupid people do these things!).

Now I'm interesting in getting machines to do as reliably as I know they
can do, some things which can only be done by a minority of people
and are generally and correctly recognised as demanding intellectual challenges.
Among these, proving mathematical theorems, and a variety of technology
design problems.
Useful progress can be made in these areas just by \emph{automation} by
devising and coding up algorithms which do these things better and better.
But still, we find usually find that the best we can do in these areas
still leaves the machines failing to achieve tasks which are simple
for the right kind of person (mathematician, engineer).
When we look into the detail, we often find at its core the need to
find a solution in a space which is very very large.
Brute search is computatinally infeasible, but people can often just
pull the needle out of the haystack.

This is what one hopes neural networks might supply.
Lets suppose that they do.
Do we have to put up with the rest, are can we get the needle finding
goodies and discard the unreliability and error?

Its pretty easy to see ways in which this can be done.
In many problem domains we can use unreliable but possibly inspired
ways of finding solutions, alongside pedestrian but reliable algorithms
which can check where some inspired solution really does solve the
problem.
Sometimes checking the solution is tractably algorithmic but is subject
to ``proof''.
Mathematical proof is the same kind of ``find and check'' process, but
this time we know that the ``check'' part is algorithmic.
To make this work outside of mathematics we use mathematical models.

Here are some capabilities:

\begin{itemize}
\item{Neural Networks}
Learning, associative memory, guessing, picking needles out of haystacks.
\item{RAM. SSD}
Storing stuff. non-associative.
\item{CPU, GPU}
Algorithmic computation, often numeric.
\item{Theorem Proving, GOFAI}
Needs to be integrated with neural nets to narrow the search spaces.
\end{itemize}

Some neural nets advocates (perhaps Jeff Hawkins) might tell us that the entire gamut
of intelligent capabilities can and should be done by neural nets or their superefficient
superabundant electronic counterparts.
But intelligent people don't usually do themselves tasks which are
better done by a computer.
An intelligent neural network can just as well have access to run-of-the-mill
processing power (probably the open-ended resources of the cloud, though possibly
not without budgetary constraint), and will be just as able to see that algorithmic
computation sometimes beats neural nets hands down.
Equally, anyone, man or machine, programming a task which involves intractible
search, will, if it is available, be glad to use an effective neural net whether
soft or hard, to cut his search spaces down.

\section{Metphysical Positivism}

I have for many years used the term ``Metaphysical Positivism'' as a label for the
theoretical side of my positive philosophy.
Calling this ``positivism'' is an aknowledgement of my sense of connection with the
philosophy of Rudolf Carnap (even though the later Carnap prefer to consider himself
an empiricist, perhaps reflecting a slight moderation in maturity).
The qualification ``Metaphysical'' is harder to explain, for it suggests, perhaps misleadingly,
that my attitude towards metaphysics is very different to Carnap's.
The difference between us is almost entirely verbal, so the adjective should be taken as
indicating abstinence from Carnap's rather particular use of the term.

Before making this terminological difference more definite, let me describe concisely the ground
on which I find common cause with Carnap.
I take as the central motivation of Carnap's philosophy the desire to mediate philosophically
in the transformation of the empirical sciences through the adoption of the new logical methods
which Carnap first discovered in the work of Frege.
This was his interpretation of the programme advocated by Russell for ``scientific philosophy'',
that philosophers should realise throught the adoption of formal languages the precision of language
and soundness of reasoning which had so far eluded them, and should facilitate rigour in empirical science by
adapting these methods for those sciences.

\section{Positive Philosophy - 2014-12-19}

This is a really compact survey of \emph{positive philosophy} as I conceive it.

First let me assert the primacy of the practical.
Positive philosophy is intended to help us to decide what to do and how to do it.

However, a major part of positive knowledge is theoretical.
It concerns knowledge, of matters and methods which might possibly prove instrumental for some of us in realising our practical goals.

So, in the presentation of positive philosophy the distinction between theoretical and practical is important.

\section{Positive Philosophy - Chosing a Future}

Philosophy deals with, alongside various minutiae, the greatest problems which face mankind.
It can be approached in a piecemeal fashion, but the greatest works of philosophy are usually in one way or another, \emph{systematic}.

When philosophy is approached systematically (as I hope to do here), it may be possible to encapsulate the philosophical enterprise into a single overarching question.
A popular candidate might be ``what is the meaning of life?''.

For the philosopher adopting this kind of approach, which we might call ``ultimately systematic'', the choice of that initial overarching question, in response to which an entire philosophical system is expected to ensue, will be important.
That is my present concern, and I will approach that question by trial and error.

Let me begin with ``what should we do?''.

The best thing to do depends upon what we hope to achieve, itself influenced by our values and our circumstances, each of which is personal.
Our first candidate is therefore made awkward to answer by its subjective element.
The philosopher, rather than addressing the particular values and circumstances of each individual he considers might prefer to offer some more general wisdom which he supposes will be helpful to individuals in coming to their own determination of how to conduct their lives.

Stepping back from the particular in this way is characteristic of much philosophy, and once the subjective is here aknowledged it may come back again and again to thwart the philosophers search for broad and general truths.
The full aknowledgement of the individual's right to chose, particularly when taken to the extremes which may be found in forms of anarchism, may make it a difficult philosophical problem to adopt any coherent definite position on how society might best be organised.
This is a part of the problem which I hope to grapple here.

\section{Some Philosophy}

I intend here to approach the articulation of a \emph{systematic} philosophy through the analysis of a single question.

It is in the nature of philosophy as I percieve it to have ideals which it can at best imperfectly approach.
That this treatise deliver a clearly articulated systematic whole is one such.
I shall try not to spend too much space apologising for this and similar shortcomings, but there are many points of similar tension which I think worth mentioning.
These tensions might even amount to a subtheme of the text, but it is too early now to elaborate on this point.

The single question I have in mind is ``What should we do?''.

What is special about this question that its analysis might play so crucual role in systematic philosophy?

\section{Speranza on Gricean Corpuscularism}

Really this is about Russell's logical atomism.

This is a place to draft comments on a posting by Speranza to ``The Grice Club''.


\subsection{Atom versus Corpuscle}

Here are some thoughts about Russell's Logical Atomism.

First a sketch.

I consider The following sources which may be regarded as influencing or contributing to the development of Russell's atomism:

\begin{itemize}
\item Bradly
\item Leibniz
\item G.E.Moore
\item Russell's Theory of Types
\item Wittgenstein
\end{itemize}

There are two principal aspects of the theory to be considered here:

\begin{itemize}
\item logical atomism as analytic method
\item logical atomism as metaphysics
\end{itemize}

We might then consider some philosophers influenced by it:

\begin{itemize}
\item Wittgenstein
\item Carnap
\item Grice
\end{itemize}
 
\subsection{Bradly, Leibniz, Moore}

Russell's most prominent attribution when explaining the origin of his logical atomism is to Bradley
(or to Hegel whose philosophy is similar in the relevant respects).
It is Bradley's monistic metaphysic and logic which Russell repudiates in {\it Logical Atomism}, giving
credit to discussions with Moore, and dating his divergence from Bradley in this respect to 1898.

Russell notes a connection between logic and metaphysics, that Aristotle's subject predicate conception of
categorical propositions and syllogistic logic leads naturally to a monistic logic and the doctrine of internal relations.

Though Russell credits discussions with Moore, he says little about the content of those discussions in his accounts of logical atomism.
He is more informative about the necessities arising from his desire to put mathematics on a more rigorous formal footing, and the inadequacy of subject/predicate logic and the doctrine of internal relations for that purpose.
It is because Russell finds relations essential in the development of mathematics that he must progress beyond the subject/predicate logic and its associated monistic metaphysic, to embrace a logic of relations and a pluralistic metaphysic.

This might suggest that the defining characteristic of ``logical atomism'' that its ontology is pluralistic rather than monistic.
However, the term ``logical atomism'' does not seem to have appeared until 1911

\paragraph{origins and influences}

First some observations about its origins.
When Russell speaks of this he is referring to (or perhaps silently implicating) Bradley and his monistic metaphysic.
The essential thesis of Logical Atomism thus presented is simply that the world consists of many independent individuals.
This is not, in itself an {\it atomistic} thesis, for which it would have to he held that the individuals were indivisible.
Russell explicitly denies that his atomism is physical, and this is of course, why he calls it {\it logical} atomism.

There are two other contributory influences which I think worth mentioning here.
Of these the first is emphasised by Russell, who believed that there was an intimate connection between {\it logic} and
 metaphysics, as a result of which the subject-predicate logic of Aristotle leads its supplicants into a monistic metaphysic, since in that logic it seems relations between individuals cannot be expressed.
(the necessity here can now be questioned, since logical systems such as Church's Simple Theory of types have no primitive relations but are able to define relations using higher order functions, but that is a later story).

We therefore find that Russell's atomic facts and propositions are relational, as he considered essential in reasoning about a pluralistic ontology.
One earlier philosopher handicapped by Aristotelian logic, to whom Russell was more sympathetic, was Gottfried Leibniz, whose metaphysic, concisely rendered in the {\it monadology} may have provoked or confirmed Russell's thesis of the connection between logic and metapysics.
I mention Leibniz here, however, because he might also be thought to have contributed to the atomistic aspects of Russell's method and metaphysic.
Leibniz mistakenly supposing Aristotelian logic sufficient for reasoning about all scientific knowledge, concluded that all knowledge could be represented in a manner which admitted an effective decision procedure.
This idea flowed from the discovery by Leibniz of a method for calculating the truth value of subject predicate assertions when the concepts related were given a systematic arithmetic presentation.




This is presented by Russell as being his method for Scientific Philosophy,
which is the first formulation of philosophy as logical analysis.
This conception of philosophy is I would suggest the principal locus oftension between Russell and Grice, though Grice's positive view of Russell
might suggest that he did not see Russell as in fundamental disagreement with him.

By contrast Grice was clearly antipathetic to Carnap, and Russell also rejected logical positivism, even though the conception of scientific philosophy and its characterisation as purely logical is shared between Russell and Carnap, and excludes Grice's ordinary language philosophising since the study of natural languages is an activity of empirical rather than purely logical analysis.

Though Russell enunciated in Logical Atomism this austere conception of scientific philosophy, his own writings were much more broad ranging.
We might call the whole corpus philosophical, but Russell could not at least have thought it {\it scientific} philosophy in the sense which he enunciated.
e

\subsection{Russell and Moore}

We are considering Logical Atomism, and its connections, or friction with the philosophy of Grice.

Sometimes what Russell says about the principal features of his philosophical positions and their origins is hard to believe.
This makes it difficult to judge the degree of fit with the subsequent philosophy of Grice.

I thought I would mention here some principal sources of my own ``puzzlement'' in these matters.

First let me mention my own naive conception of the relationship.
I have hitherto regarded Russell and Moore, the two principals in the revolt against idealism which
took place in Cambridge around the turn of the twentieth century, as setting off with two quite different
ideas of how philosophy might be analytic which were influential and antagonistic through the first half
of the twentieth century.

Russell's conception of philosophical analysis is as a tool based on the new formal mathematical logic which made possible a rigorous scientific philosophy whose method was logic.
Russell is advocating philosophy as an {\it a priori} science like mathematics but differing in subject matter (though possibly both so completely general as to lack a subject matter).

Nothing could be further from this (within the narrow confines of philosophical analysis) than the direction of Moore's philosophy, in which a common sense realism was prominent, and the analysis in question we might consider more {\it linguistic} than purely {\it logical}.

In the sequel we see Wittgenstein beginning with Russell and, in his early philosophy, influencing Russell's conception of Logical Atomism, and presenting in his Tractatus a philosophical perspective influenced by the new mathematical logic as presented in the work of Frege, Russell and Whitehead.
It was in the later stages of this early philosophy that Wittegenstein engaged with the Schlick's circle in Vienna, exerting some influence on the development of logical positivism.
Wittgenstein, quickly moved from this early concern with formalism to a later philosophy more closely aligned with the informal analysis of ordinary language more characteristic of Moore than Russell.

In the first half of the twentieth century there was variety in the approaches to philosophical analysis, but this variety can still be thought of as falling into these two main baskets, the analysis of ordinary language, or of philosophical problems through that of ordinary language, on the one hand, and the avocacy of philosophy as logic and the conception of philosophy as handmaiden to science facilitating clarity and deductive rigour by the use of formal logical notations, and of philosophy as an {\it priori} deductive science.

In the first basket we have Wittgenstein's later philosophy progressed by him and his associates at Cambridge, and in Oxford the rise of ``ordinary language philosophy'' after some nods in the direction of Russell and his footsoldier Rudolf Carnap, for example Ryle's ``systematically misleading expressions'' and the works of the ``infant terrible''\footnote{(A.J.Ayer)} he despatched to find out what was happening in Vienna.


I spoke not merely of divergence but of antagonismm, and we have clear evidence of this.

\subsection{Ontological Observations, concerning Russell and Grice}

I have reflected for while, and researched a little, into the philosophy of Grice and into Russell's logical atomism, provoked in this by Speranza's recent contrasting of Russellian atomism with Gricean corpuscularity.

I found it difficult to come up with anything nice revolving around that particular terminology (atoms v. corpuscles), its not so easy to see what the key elements of Russell's atomism are or whether Grice was a corpuscularian in any way which directly conflicted with that atomism.

However, I did find a little enlightenment by thinking on the ontological side of things, of the relationship between Russell and Grice.

First a rough sketch of the development of Russell's ontology.

\paragraph{Russell's Ontological Progression}

The stages I consider here are:

\begin{enumerate}
\item Idealistic monism
\item Meinongian pluralism
\item Logical Atomism
\end{enumerate}

\paragraph{Monism}

Russell begins in sympathy with Bradley's monism, the thesis that reality is singular, that there is really just one thing.

\paragraph{Lavish Pluralism}

Russell then describes himself as following Moore in forsaking that monistic view in favour of a pluralism.
He talks of it as the beginning of Logical Atomism, but his ontology is at first very different from the ontology more usually associated with logical atomism, and with the ontological aspects of Russell's descriptions of logical atomism.
It is of interest in connection with the later direction of Grice's thinking, for Grice may be thought of as backpedalling a little from the ontologu of logical atomism in the direction of Russell's earlier and more opulent ontology.

The revolt from idealism in which Russell tells us he followed Moore, and which marks the beginnings (in many accounts) of modern analytic philosophy, consists primarily, according to Russell, in the rejection of monism in favour of pluralism.
The monism here is the belief that there is really only one thing, and pluralism is the rejection of that doctrine (nothing like Carnap's pluralism).
These alternatives were closely connected in Russell's thought with the doctrine of internal relations and with the narrow expressive capabilities of Aristotelian term logic.
This kind of monism (not a precursor for Russell's own ``neutral monism'') was problematic for Russell because he sought to make mathematical proofs more rigorous and saw no way in which the relations which were essential to mathematics (Russell cites asymmetric transitive relations) could be construed as .

Central therefore to Russell's revolt was the recognition that there are many things and that these many things are sometimes related by external relations.

At these earliest stages in Russell's post-idealist philosophy his ontology was not merely pluralistic in admitting more than one thing to exist, but lavish also in the kinds of existent it recognised, and has been characterised as ``Meinongian''.
We may think of this as exhibiting too naive a connection between semantics and metaphysics, insofar as Russell is accepting that every conceivable term denotes some existent entity.

In Russell's accounts of Logical Atomism this ontological excess has been severely moderated.
It's significance here is that when Grice comes to consider vacuous names, reacting against Strawson's critique of Russell's theory of descriptions (and othaer alleged dissimilarities between formal logic and natural language) he appears to re-instate some similar principle, that descriptions always denote something.

Russell does not retain this lavish ontology, which would not have provided a good basis for the formalisation of mathematics in Principia Mathematica.
Instead he resorts to Occam's razor and cuts back the ontology to the bone.
This is what we find in his ``Theory of Logical Types'' (1908), used in Principia Mathematica (1910-13) and then providing the conception of logic underpinning his Logical Atomism.

In this more parsimonious ontology we have a collection of ``individuals'', the lowest order in the theory of types, and various ``logical fictions''.
It is convenient to talk as if these logical fictions exist, but they do not in fact constitute real existents beyond the individuals.
From an everyday point of view, we may think of large material structures as being logical fictions constructed from individual material atoms.
In the theory of types, they are propositional functions of various orders, sets, and relations in extension with which the abstract structures of mathematics are defined.

The logical method which Russell advocates as a part of his Logical Atomism is ``the method of logical construction'', exemplified in Principia Mathematica's formalisation of mathematics, but applicable also to the empirical sciences.
The idea is that instead of postulating entities as required, the required ontology is obtained by construction from those already in place.
This kind of construction yields entities over which Russell's theory of types quantifies.
All but the individuals are regarded by Russell as fictions, but by Quine's later ``criterion of ontological commitment'' they are, according to Quine, entities to which Russell is committed.
(Quine's criteria are equally ill fitting with Carnap's philosophy) 

The other important kind of logical fictions are those which Russell call's ``incomplete symbols''.
These are important for our story because they include descriptions, and sets and relations in extension.
The distinguishing feature of an incomplete symbol is that it does not by itself denote anything at all, not even a logical construction.
In certain contexts it plays a role in a convenient means of expression, whose content is established by a translation of the whole in which the ``incomplete symbol'' occurs.
In this translation the incomplete symbol is eliminated in favour of previously defined if perhaps less perspicuous terminology.

Russell's theory of descriptions (as incomplete symbols) is what Strawson later criticised, offering instead an account of the meaning of descriptions which depended upon a ``truth value gap'', sentences which lack a truth value.
Grice later undertook a program of reconciliation between ordinary and formal languages, suggesting that the differences between the two were less substantial than many had supposed and offering new explanations of apparent divergence.
Some of these involved Gricean notions such as conversational implicature (vide Speranza's recent post on the genealogy o  disjunction), but in the case of the theory of descriptions Grice offered a formal account in his System Q (named for Quine).

In his system Q, Grice does not defend Russell's theory of descriptions against Strawson's attack, but responds to Strawson with an account closer perhaps to the logical system which Russell might have come up with if he had never applied Occam's razor to his Meinongian ontology.
In this system, descriptions are no longer incomplete symbols, and denote whether or not their description is satisfied.

At the time of his logical atomism Russell was indecisive about the non-fictional aspects of his ontology.
When he does talk of this, it is frequently in terms of sense impressions, and a phenomenological ontology has also been attributed to Wittgenstein's Tractatus (though not repudiated by Wittgenstein I think).
Russell does come to a more definite ontological position in 1921 as a ``neutral monist'', without thereby reversing his rejection of monism in favour pf pluralism.
Neutral monism does not assert that there is only one thing, but rather, just one kind of things, supplanting Cartesian dualism with an ontology of entities which are neither material nor mental, but from which both matter and mind can be obtained by logical construction.

%\backmatter

%\appendix

%\addcontentsline{toc}{section}{Bibliography}
%\bibliographystyle{alpha}
%\bibliography{rbj}

%\addcontentsline{toc}{section}{Index}\label{index}
%{\twocolumn[]
%{\small\printindex}}

%\vfill

%\tiny{
%Started 2012-10-19

%Last Change $ $Date: 2015/04/23 09:58:09 $ $

%\href{http://www.rbjones.com/rbjpub/www/papers/p019.pdf}{http://www.rbjones.com/rbjpub/www/papers/p019.pdf}

%Draft $ $Id: p023.tex,v 1.2 2015/04/23 09:58:09 rbj Exp $ $
%}%tiny

\end{document}

% LocalWords:
