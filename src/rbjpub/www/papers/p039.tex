% $Id: p039.tex $
% bibref{rbjp039} pdfname{p039}
\documentclass[10pt,titlepage]{article}
\usepackage{makeidx}
\newcommand{\ignore}[1]{}
\usepackage{graphicx}
\usepackage[unicode]{hyperref}
\pagestyle{plain}
\usepackage[paperwidth=5.25in,paperheight=8in,hmargin={0.75in,0.5in},vmargin={0.5in,0.5in},includehead,includefoot]{geometry}
\hypersetup{pdfauthor={Roger Bishop Jones}}
\hypersetup{pdftitle={Oracular AI v2}}
\hypersetup{colorlinks=true, urlcolor=red, citecolor=blue, filecolor=blue, linkcolor=blue}
%\usepackage{html}
\usepackage{paralist}
\usepackage{relsize}
\usepackage{verbatim}
\usepackage{enumerate}
\usepackage{longtable}
\usepackage{url}
\newcommand{\hreg}[2]{\href{#1}{#2}\footnote{\url{#1}}}
\makeindex

\title{\bf\LARGE Oracular AI v3}
\author{Roger~Bishop~Jones}
\date{\small 20212-08-19}


\begin{document}
%\frontmatter

%\begin{abstract}
% A third start on this topic.
%
%\end{abstract}
                               
\begin{titlepage}
\maketitle

%\vfill

%\begin{centering}

%{\footnotesize
%copyright\ Roger~Bishop~Jones;
%}%footnotesize

%\end{centering}

\end{titlepage}

\ \

\ignore{
\begin{centering}
{}
\end{centering}
}%ignore

\setcounter{tocdepth}{2}
{\parskip-0pt\tableofcontents}

%\listoffigures

%\mainmatter

\section*{Preface}

\section{Introduction}

When stored program digital computers were first invented their applications primarily concerned doing large amounts of information processing or computation with almost perfect reliability and at superhuman speeds.
They were accurate and reliable.

As their computational power grew their applications were extended progressively, and this sometimes involved attempts to achieve ends which were much less clearly defined, involving more complex instructions which could less certainly be relied upon to achieve the intended purpose.

The kinds of brute computational power exhibited by these early computers might at first have been thought signs of intelligence, since human skill in computation had certainly been presumed a sign of intelligence.
But brute computational power soon came to be distinguished from intelligence.

As I write, generative AI and Large Language Models have momentarily set a new standard for the unreliability of Artificial Intelligence.
Not designed or trained to be reliable repositories if knowledge, or to be capable of any but the most eleentary reasoning, their exposure to vast quantities of human knowledge enables them to perform in many main-line subject matters in an apparently authoritative way, while morphing in more esoteric areas into fantasy, and failing under the most modest interrogation to demonstrate any but the most shallow comprehension in subjects whose generally accepted facts they can replay.

It may not be so hard to improve on this.
LLMs have proven capable of using tools effectively, and tools such as more reliable ways of saving accurately and reliably knowledge acquired, or reasoning about that knowledge may not be difficult to supply.
The discussion in this essay may be thought of as concerning the use by such an AI with a tool which is capable both of storing knowledge and of deductive reasoning in the context of that knowledge.
The effect alleged would be to enable Artificial Intelligence which is \emph{oracular} in relation to logical truth.

Oracles may be thought of as having great wisdom, possibly derived from divine connection, but here I use the term more narrowly.
For the purposes of this essay the term ``oracle'' is used for something which is always truthful in answering questions, but doesn't always answer.

The oracle of interest here can be asked whether a sentence in a formal language is a \emph{logical truth} a concept which I will try to characterise, but which ultimately cannot be made absolutely definite.

The term ``Logical Truth'' is philosophically controversial.
In my usage of that term I stand on a limb, for my use is very similar to that of Rudolf Carnap,and is synonymous with the term \emph{analytic}, a concept central to Quine's repudiation of the philosophy of Carnap in the mid 20th Century.

Its not my purpose here to argue about the terminology.
Some might insist that my conception of logical truth should more properly be spoken of as set theoretic truth, and I do not intend to argue against that opinion.
Carnap, who until 1952 used the terms ``logical truth'' and ``analytic truth'' synonymously %
\footnote{as is explicit in section 2 of ``Meaning and Necessity'' \cite{carnap56}}%
, eventually accepted defeat and began to use the term ``logical truth'' for a narrower concept%
\footnote{W.V. Quine's noted a defect in Carnap's definition of analyticity in \cite{carnap56}, which followed closely a defect first seen in Wittgenstein's ``Tractatus Logico-philosophicus''\cite{Wittgenstein1921}.
Carnap's response appeared in the paper ``Meaning Postulates''\cite{carnap52} in which for the first time he separates the concept of logical truth from that of analyticity.}.

The term ``Oracular AI'' as used here, refers to what AI might in principle be able to achieve if furnished with an oracle for logical truth.

One of the purposes of this essay is to discuss how thus notion of logical truth can be made precise, to consider the difficulties in implementing such a decision procedure and to talk about the value of approximations which fall short of logical omniscience.

\section{Some Historical Background}

\appendix

\section{Terminological Notes}

\subsection{On the notion of Logical Truth}

\phantomsection
\addcontentsline{toc}{section}{Bibliography}
\bibliographystyle{rbjfmu}
\bibliography{rbj2}

\addcontentsline{toc}{section}{Index}\label{index}
{\twocolumn[]
{\small\printindex}}

%\vfill

%\tiny{
%Started 2023-08-19


%\href{http://www.rbjones.com/rbjpub/www/papers/p039.pdf}{http://www.rbjones.com/rbjpub/www/papers/p039.pdf}

%}%tiny

\end{document}

% LocalWords:
