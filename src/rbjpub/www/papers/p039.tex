% $Id: p039.tex $
% bibref{rbjp039} pdfname{p039}
\documentclass[10pt,titlepage]{article}
\usepackage{makeidx}
\newcommand{\ignore}[1]{}
\usepackage{graphicx}
\usepackage[unicode]{hyperref}
\pagestyle{plain}
\usepackage[paperwidth=5.25in,paperheight=8in,hmargin={0.75in,0.5in},vmargin={0.5in,0.5in},includehead,includefoot]{geometry}
\hypersetup{pdfauthor={Roger Bishop Jones}}
\hypersetup{pdftitle={The Tyrrany of the Tribe and the Roots of Reason}}
\hypersetup{colorlinks=true, urlcolor=red, citecolor=blue, filecolor=blue, linkcolor=blue}
%\usepackage{html}
\usepackage{paralist}
\usepackage{relsize}
\usepackage{verbatim}
\usepackage{enumerate}
\usepackage{longtable}
\usepackage{url}
\newcommand{\hreg}[2]{\href{#1}{#2}\footnote{\url{#1}}}
\makeindex

\title{\bf\LARGE The Tyrrany of the Tribe\\and\\the Roots of Reason}
\author{Roger~Bishop~Jones}
\date{\small 2021-02-02}


\begin{document}

%\begin{abstract}
%
% An attempt to understand rationality and those forces which undermine it in the form of a historical narrative.
% 
%\end{abstract}
                               
\begin{titlepage}
\maketitle

%\vfill

%\begin{centering}

%{\footnotesize
%copyright\ Roger~Bishop~Jones;
%}%footnotesize

%\end{centering}

\end{titlepage}

\ \

\ignore{
\begin{centering}
{}
\end{centering}
}%ignore

\setcounter{tocdepth}{4}
{\parskip-0pt\tableofcontents}

%\listoffigures

\pagebreak

\section*{Preface}
%\phantomsection

\addcontentsline{toc}{section}{Preface}

This is a short essay with broad scope, and hence a collage of oversimplifications.
One of those is committed by my title, which suggests a normative contrast as clear as that between good and evil.
For this I credit the political climate as I write, in which the dangers of ``tribalism'' seem to me more proximate than those of rationalism.

In many aspects of the matters here discussed, mutually contradictory tendencies push evolution to congenial and necessary balance, which facilitates a progression, if not always progress.
By tracing these we may come to a better understanding of where we are and how to find our way home.

I have late in life suddenly come to appreciate that the values which I absorbed as a child and a young man are being challenged by alternatives which seem to have no merit, and whose main purpose seems rather to destroy than to improve.
It is startling that this agenda has pressed forward in plain sight, but unseen by most.
In another climate the two alternatives here juxtaposed might be given a more balanced presentation, but now, despite considerable reservations on the matter which will become apparent, I am intent on a defence of reason against ideological assault.

\section{Introduction}

No other species on Earth is capable of rational intelligence in the manner of \emph{homo sapiens}, and none other so well equipped, in apparent abandonment of that gift, to undertake mass self-mutilation while blinded by ideology.

In this essay I sketch the results of my efforts to understand these seemingly contradictory phenomena, my attempt to comprehend certain aspects of human nature and the social realities influenced by them, and to probe the possibilities for progress, because of and despite these characteristics.

To understand \emph{where} we are, we must understand \emph{who} we are, the nature of humanity.
To understand who we are, it helps to understand how we came to be.
To peer into the future, we must understand where we are, the process which created us, and the ways in which that process is itself evolving.
This is my striving to see.

The story which follows is built around evolution, the many kinds of evolution, the ways in which evolution has and will continue to evolve, and the way it has shaped humanity and society.
It is a historical narrative which (somewhat arbitrarily) falls into four main parts:

\begin{enumerate}
\item The evolution of homo sapiens
\item The evolution of culture to `\emph{the Age of Reason?}' 
\item Counter-enlightenments and revolutions
\item Evolution by design -- synthetic biology and artificial intelligence
\end{enumerate}

Through these stages the pace of evolution has accelerated.
The first stage took four billion years, the second two hundred thousand, the third two hundred years.
The fourth stretches into the future.
The acceleration has been enabled by changes in the nature of evolution enabled by the advances previously made.

Only the later part of the first stage consists of that kind of biological evolution identified by Darwin as ``the evolution of species by natural selection''.
Thereafter the most significant change is cultural, outpacing continued genetic evolution.
In the last phase, changes to the biosphere and perhaps even to the human genome will increasingly be {\it by design}.

\section{Themes}

Through this chronological narrative a number of themes and subthemes are threaded.

\ignore{
  The two principle themes are rationality and democracy in the account of which a number of other themes and subthemes are helpful.

\subsection{Rationality}

There is one principal theme, which is the evolution, biological and cultural, of rationality and of those human or social tendencies which seem most seriously to diverge from it (or are perhaps a necessary complement to it).
For the purposes of this essay I take an instrumental notion of rationality to be fundamental, and all others (of which the espistemic is the most important) as derivative.

\subsection{Democracy}

It is common for a species to exhibit distinct behaviours, perhaps even distinct physical structure, according to environmental conditions.
The availability or scarcity of food may trigger such variations, procreation being favoured in conditions of pleanty, and foraging favoured when supplies are short.
In humans the treatment of offspring in their infancy may determine adult character in a profound way, reduced physical contact between parent and child (perhaps caused by more urgent demands on parental time) leading to more aggressive adult character, suitable for seeking (and possibly fighting for) resources.

Different forms of social organisation and behavioural norms may similarly depend upon context.
In harsh times, competition for scarce resource may be physical.
In prosperous conditions it may be possible to resolve differences without violence.
This may be seen as the principle aim of democratic governance, something which seems only to have flourished in modern relative prosperity, and remains a fragile flower.

}%ignore

\subsection{Evolution}

I counterpose two evolved phenomena.
On the one hand our capability for \emph{rationality}, eventually manifesting in the advanced technologies which have made the lives of most people healthy, prosperous and fertile.
On the other the strong social tendencies which may cause rational beings to act \emph{en masse} in apparently irrational ways.
Both of these are enabled by physiological features, coded in our genomes, but are manifest in ways which are shaped by the cultural evolution which those genetically determined traits enabled.


\subsection{The Theory of Evolution}

Here a brief sketch of the development of the theory of evolution.
Later the history of evolution on earth will be considered particularly in relation to my two headline concerns, rationality and its nemesis.

Ideas about evolution date back to antiquity, but the modern scientific account of evolution begins properly with Darwin's theory of the evolution of species by natural selection.
This was developed in the $19^{th}$ Century concurrently with Mendel's work on genetics, which however was not widely disseminated until the $20^{th}$ Century, so Darwin did not have a good understanding of inheritance when he undertook his research and wrote his magnum opus. \cite{darwin-oos}.

Darwinian evolution therefore pre-dated a scientific understanding of inheritance, which latter has continued to advance ever since.
So we have in Darwin a conception of evolution which is broader than most modern ideas of evolution, but which nevertheless fails to encompass all the kinds of evolution which we touch upon here.

Essential to Darwin's conception are that it concerns a biosphere in which:
\begin{itemize}
  \item there are species, each of which is a collection of self-reproducing organisms
  \item organisms reproduce, creating new organisms similar to the original, but subject to some variation
  \item different individuals of the same species will not all be equally successful in reproducing themselves, these differences being attributed to ``natural selection'' (by analogy and contrast with that kind of selection exercised by those who breed domesticated species, selecting as parents individuals exhibiting desired characteristics.)
  \item the differential reproduction results in continuous shifting in the nature of the species which is guided by that natural selection.
\end{itemize}

Though this account of the theory of evolution interprets ``natural selection'' through the reproductive success of individual organisms, Darwin also recognised that altruistic traits emerge which appear to be beneficial to groups even though they do not manifest in reproductive succes for the individual.
He has been interpreted as adhering to a two-level theory in which:

\begin{quote}
`Selfishness beats altruism within groups.
Altruistic groups beat selfish groups.
Everything else is commentary.
\end{quote}
\cite{wilson2007rethinking,wilson2015does}

Though here, ``altruistic group'' means ``a group of altruistic individuals'', but is itself nothing of the sort (in relation to other groups).
The altruism is intragroup not intergroup.

During the first half of the $20^{th}$ Century the further development of evolutionary theory integrated Darwin's theory with an understanding of the genetics coming from the work of Mendel,
The synthesis with Mendelian genetics resulted in `the Modern Synthesis', a term coined by Julian Huxley \cite{huxley-tms}.
There were three further articulations of such a synthesis, Mayr (1959), Stebbins (1966), Dobzhansky (1974), but at the same time as a concensus might thus have seemed possible, divergence was becoming apparent concerning the evolution of altruism and the possibility of group selection.

In that matter the gene-centric presentations by Williams(1966) \cite{williams-ans} popularised by Dawkins \cite{dawkinsSG} were contrasted by work on sociobiology or evolutionary psychology such as E.~O.~Wilson's \emph{Sociobiology: The New Synthesis} \cite{wilson-stns}.
The impetus to take a broader view of evolutionary theory continues with the idea of an \emph{extended Evolutionary Synthesis} which embraces 

The chemistry underlying genetics was clarified by the discovery of the structure of DNA by Crick and Watson in 1953, and the resulting evolutionary theory was popularised by Dawkins accompanied by an interpretation of its consequences for the possibility of genuine altruisn \cite{dawkinsSG} and a strong repudiation of the possibility of selection other than at the level of the gene.

\subsubsection{The Gene-centric Account of Evolution}

Though evolution is varied in character, there is a single abstract model of evolution which is a reasonably good fit with a large part of biological evolution and by contrast with which other kinds of evolution can be described.
This model was given a popular exposition in \emph{The Selfish Gene}\cite{dawkinsSG} by Richard Dawkins.
Its bare essentials are as follows.

Let us call it ``genetic evolution of species by natural selection''.
We take a species to be a population of organisms capable of interbreeding, in some more or less habitable environment.
Each organism has a \emph{genome}, which is a set of genes.
The complete set of genes occurring in the population of the species, together with the number of occurrences of each gene is the \emph{gene pool}.
As time progresses some individuals will die, and they will then be removed from the population and their genes will be deducted from the gene pool.
Others will produce offspring which are added to the population with genomes derived from those of their parents.
This process of reproduction yields individuals whose genomes are primarily drawn from the genes of the parents, but may also include some genes which are not present in the parental genomes.
In sexually reproducing species, the main source of genetic variation in the reproductive process is recombination due to crossover (a process in which a chromosome is obtained by selecting some material from the corresponding male and some from the female parental chromosomes), but some new genes arising from crossovers which occur within a cistron (the sequence coding a single protein), and rarer errors of transcription. 


In this way the gene pool of the species evolves over time.
Both the lifespan and the reproductive success of the individuals of the species are influenced by their genome, and these factors result in differential propagation of genes in the gene pool.
This differential propagation results in progressive changes the gene pool. and hence of the phenotypes of the individuals of the species.

In \emph{The Selfish Gene} Dawkins is particularly concerned to refute the possibility that `true' \emph{altruism} could possibly evolve by such means.
The mechanism alleged to make altruism possible, even in 1967 held to have been rejected by the profession, is \emph{group selection}.
We are here concerned with the evolution and character not only of rationality, but also of social behaviour, and it is to be expected that there may be some tension with Dawkins' uncompromising stance aainst the possibility of altruism.
There remain to this day academics studying evolution who find group selection to be a valuable tool, and take it to be justified by the results it yields, oblivious perhaps to the inadequacy of explanations based on mechanisms which cannot themselves be explained.
I will say here that I accept the model of genetic evolution on which Dawkins (and many others) drew his conclusions, and differ from him, not in positing some other mechanism yeilding selection on the basis of advantage to a social group, but rather in the analysis of what limits flow from it.

\subsubsection{Stages in Evolution}

The evolutionary history we present here falls into five periods all having distinct characteristics from an evolutionary point of view.

\paragraph{pre-biotic}

The standard model is a model of genetic evolution, and the stage in evolution which is termed pre-biotic is that which precedes the existence of life.

\paragraph{prokaryotic}

In the second and third the Modern Synthesis is applicable, with some some supplementary considerations in the third.
The first stage is prebiotic, before the genetic model of inheritance is established.
The fourth stage is dominated by oral and then written culture which fails to comply with the genetic model of inheritence, notwithstanding the tenuous analogy offered by the concept of ``meme''
coined by Dawkins.
If culture took the reins in the fourth stage, partly by guiding the underlying genetic mechanisms no longer arguably directed exclusively by `natural', in the speculative fourth phase the underlying mechanism is derailed by synthetic biology, the ecosphere finally falling within the sphere of intelligent design.

\subsection{Sociality}

There is strength in numbers, one hears, and in consequence life on earth has social groups running through it like lettering in a stick of rock, from beginning to end.
If we admit as ``social'' whatever ways living beings interact in groups with their peers, then we may say that sociality has itself evolved throughout the history of life on earth.
The manner of this social interaction is at first genetically determined, but as life becomes more adaptable through the evolution of increasingly capable central nervous systems, the character of social interaction is progressively transformed, eventually subject to determination largely by culture rather than genes.

The phenomena we seek to understand, contrast and reconcile, are social phenomena a product both of genes and culture, nature and nurture.
Tracing the evolution of social behaviours from their beginnings may help us to understand them better.

\subsection{Communications and Language}

Though it is common among linguists to regard as \emph{language} only those means of communication found in \emph{homo sapiens}, communication \emph{of some kind} is an essential condition of sociality of any kind, counting here among communications even that mere bodily contact with others which may take precedence in some bacteria over life's essentials such as food.
Not only is some kind of communication essential to social life, the kinds of communication available may have a profound effect on social behaviour, and the evolution of information storage and communication technology is coupled with major transformations in social behaviour and all the depends upon it.

An important theme is therefore the development of communication from the most elementary beginnings through to the present day and beyond.
This is particularly important because of its pivotal role in the possibility of oral and then written culture, and because of its fundamental connection with the nature of rationality.

Rather than taking a stance on the question \emph{what is a language?}, I note here a number of kinds of communication which represent an ascending scale of refinement and utility.

\begin{itemize}
\item Observation

  If we consider communication to have taken place when information is transferred from one part to another, then this is possible without any intent to commumnicate on the part of the party from whom information is transferred.
  Thus one organism may observe the behaviour or state of another, and the observation may influence his subsequent behaviour.
  This is an aspect of the most elementary kinds of social behaviour.

  Without attempting an exhaustive classification, the following two obvious cases may be noted.
  First, by observing the behaviour of his peers, an organism may learn where to find resources such as food or water, or how to accomplish some task by a method previously unknown to him.
  Neither of these need involve and intent on the part of the other part to communicate.
  Second, observation may play a role in mimicry, in which one organism observes a certain behaviour, and subsequently mimics that behaviour.
  For example, having seen one organism recoil with fear, an observing organism may subsequently do the same when encountering that same potential threat.

  In the behaviour of flocks or schools each individual must observe the course of those closest to him and adjust his course in order to avoid collisions.
  This is a vital channel of `communication' essential to those collaborative behaviours, but it is not necessary to suppose that there is any intent on those proximate creatures to signal or communicate.

\item Signalling

  Once there is intent to communicate, then an organism may behave in a way which (in part or whole) is intended to communicate to another party, or which we may say, has evolved to fulfill that purpose.
  A suitable criterion here is that what is done is different in some way in order to facilitate the communication than it would otherwise have been.

  Examples of this are the signs of distress which animals may show when some threat is perceived.
  Birds may take to flight on noticing a threat, and this in itself would not constitute a signal, but it is likely also to cry out, which signals to other birds that a threat has been perceived, and has no other apparent role.

  Showing may also fit in this category.
  While being observed exhibiting some skill may not be a deliberate communication, deliberately showing how may involve doing it in a slightly different way to make it clear to the observer exactly what is being done.
  Under this classification this would count as signalling.
  Much behaviour during mating would also count as signalling, typically making conspicuous some attribute which might be persuasive to a potential mate.

  Many linguists would regard these kinds of signalling as not making use of a language, but some philosophers, perhaps Wittgenstein with his ideas about languages as games, might be more accomodating.
  
\item Symbolism and reference

  Many will consider a key feature of language to be symbolism, and the most primitive purpose of symbols to be reference.
  Certainly, it is essential to the notion of symbol that the symbol stands for something else, which is typically known as its referent.
  It may not be unambiguous when this is occurring, could we consider the cry of a distressed bird as symbolising and referring to the distress of the bird?
  I should be inclined myself to look for more deliberate use of symbols.
  Locating the demarcation here will not be important for us.

\item Propositional language

\item Recursive language

\item Written language
  
 
\end{itemize}

\subsection{Epistemology and Rationality}

A huge part of the distinction which concerns us here, between `tribalism' and `rationality' is epistemological, which is to say, that it is concerned with the practice and theory of knowledge.

The evolution of \emph{how we know} as well as the evolution of epistemology (the theory of how we know), is therefore a principle concern and is overlaid upon the consideration of sociality, communications and language.

It is natural perhaps for philosophers to consider as knowledge those things that we can put into or read out of textbooks, but it is also valuable to consider the entire process of evolution of homo sapiens from even before the existence of life through the present and into the future as being an accumulation of knowledge by means which have continually progressed and will continue to evolve.

This is also a perspective from which we can make sense of the notion of rationality as having that same broad scope and associated with the pragmatic nature of evolution in yielding those kinds of organism which are well adapted to self replication in some niche of this moment in the evolution of the biosphere.

Here are some headings under which these matters will be considered.

Knowledge can be considered to be of the following principle kinds:

\begin{itemize}
\item propositional knowledge
  \end{itemize}


\subsection{Subthemes}

Cutting across those themes are some recurring subthemes:
\footnote{These are derived in part from the ideas of Howard Bloom published in \cite{bloom-bs,bloomBRAIN}}

  \begin{itemize}
\item conformity and diversity
\item status and place
\item resource shifters
\item inter group tournaments
  \end{itemize}

  \section{Rationality}
  
The instrumental notion I take as follows:
\begin{quote}
  A course of action is \emph{instrumentally rational} for some purpose if there is good reason to believe that it will realise that purpose.
\end{quote}

It transpires that true knowledge is helpful, but that belief in falsehoods is counterproductive.
So having good ways to distinguish truth from falsity is important, and the use of such criteria yields \emph{epistemic} rationality:

\begin{quote}
Belief in a proposition is \emph{epistemically rational} if there is good reason to believe it true.
\end{quote}

These two definitions don't get us far, there is more to be said.
They simplistically reduce rationality to the having of good reasons for actions or beliefs.

I intend that the concepts should be taken as \emph{normative}, that to say that something is rational is to approve, under the circumstances and in the light of available evidence, of the belief or the course of action.

The standard will depend on context.
In dire circumstance, an action may be considered rational which might otherwise have been thought reckless.
As science and technology progresses it may become possible and rational to exact higher standards of experimental rigour, supported by more advanced and precise equipment and perhaps a fuller understanding of related phenomena.
Thus \emph{rationality} as it is here discussed will be a moveable feast, and much of what is to be said concerns how it has developed in the past and how it might or should progress in the future.

The contrast which concerns us, between belief or behaviour which is rational, and that which may be maligned as ``tribal'', is not a subtle distinction.
In the latter case, there may be a complete disregard of all the kinds of evidence crucial to a rational determination in favour of a rigid or an incoherent idelogically determined prejudice.

The sensitivity of instrumental rationality to \emph{purpose}, may nevertheless render intelligible cases of apparent irrationality.
We may have to acknowledge that the purpose in hand is not what we supposed or hoped it to be, the context creating motives which we had not suspected.
The potential conflict between what might be considered social purpose and the purposes or interest of individuals is germane to our theme, and invites extension of the concept of rationality to social groups, particularly to organisations which have some intended purpose and to the question whether they are organised in the best way to realise that purpose.

\section{The Evolution of Homo Sapiens}

In considering the nature and future of man by investigation of his origins, several kinds of history are relevant.

There is first, the story of how \emph{life on earth} has evolved over the last 4.5 Billion years.
In considering this whole as a process of evolution we are of necessity taking a very broad view of evolution.

Tangible insights from an evolutionary perspective may depend upon recognising that \emph{the process of evolution} has itself evolved over that period, and on seeking an understanding of those stages sufficient to illuminate the aspects of the developmemt of life which are of particular interest.

In telling those two stories it may be helpful to consider the evolution of \emph{our understanding of evolution}, a rather more recent affair which (notwithstanding earlier premonitions) is generally taken to start with Charles Darwin\cite{darwin-oos}.



  \subsubsection{Pre-biotic Evolution}

  A first crucial but mysterious stage of evolution takes place after the formation of the earth but before the appearance of living organisms.
  Our understanding of this is necessarily limited and speculative.
  Evidence of what happened has been wiped clear by the life which subsequently emerged.

  In terms of understanding the nature of evolution and how it is changing at this time, the main problem is that no clear process has so far evolved.
  There is no life, there is no way of coding descriptions of life to hand down through the generations, no genome to be distinguised from its phylotipic expression and subject to variation and selection.

  Nevertheless there is a progression, the manner of which itself evolves over time, and changes to which affect the rate of progression and by favouring certain kinds of progression over others, influence the direction of progress.
  The progression is in the first instance likely to be a gradual change over time in the composition of the watery parts of the planet.
  A drift rather than anything more structured, leading little by little to greater concentrations of slightly more complex molecules.

  Eventually some of the complex molecules formed will be catalysts, providing staging posts for the construction of complex molecules which would otherwise be much likely to form.
  This is one way in which both the direction and pace of evolution may shift.
  Beyond simple catalysis, \emph{auto-catalytic sets} are thought likely to have contributed.
  An autocatalytic set, a group of chemicals which catalyse both the construction of certain molecules from their constituents and also the catalysts themselves, is like a self-reproducing factory.
  It not only builds arbitrarily complex molecules from simpler constituents faster than they might otherwise be created, but is permits continuous exponential growth of the machinery devoted to that purpose.

  Nevertheless, the complexity of the building blocks of life is so great that this kind of free-for-all, abbetted though it might be by catalysts and autocatalytic sets, is barely a beginning the the evolution of the machinery of evolution.
  When we look at the simplest living organisms to have survived to the present day, we find that they share, along with the rest of life on earth, the use of a particular open ended collection of building blocks for life facilitated by \emph{universal} manufacturing facility.
  The building blocks are the proteins, lengthy chains of amino acids.
  The universal machinery, by analogy with a universal turing machine, given the specification of a protean, i.e. an account of what amino acids must be strung together to form the protean, together with an sufficient supply of the amino acids, will follow the recipe carefully stringing together the hundreds of amino acid molecules which will likely be required to make the required molecule of protein.

  \subsubsection {Molecular Machinery}

    In explaining scepticism about evolution, the astronomical improbabilty that we could be here \emph{by accident} is often invoked.
    The theory of evolution has rather more to it, but it does explain why the odds on human beings happening without divine intervention are better than one might imagine.
    Untutored imaginations are poor guides as to what is or is not probable.
    
    What we will see in the following descriptions of how evolution has taken place, is a series of special considerations showing how certain constructions and developments are more probable than we might have expected.

    We begin in the `primeval soup' shortly after the formation of the earth some 4 billion years ago.
    The ocean was then presumed to contain simple molecules, and the stock model of genetic evolution is entirely inapplicable.
    There are no genes, and there are no self-replicating structures, no life.
    
    The very weak sense in which evolution is taking place at this stage is that, little by little, stimulated by energy sources such as lightening and volcanic action, these simple atoms and molecules are combining to form more complex molecules.
    Initially these more complex molecules will be scarce and the ingredients for forming them plentiful.
    The rate of formation stimulated by the available energy sources will exceed the rate at which these more complex molecules break up, and the numbers will rise gradually seeking a point of equilibrium at which losses match formations.
    Thus the consistency of the soup `evolves', equibilibrium proving illusive as the changing mix enables more complex constructions.

    This process is not entirely random, the probability of formation of a compound is not exclusively determined by the availability of components.
    A construction involving three components will be much less likely to happen if all three must be present at once than ifthe construction can take place in stages, if it can be broken down into two constructions each having only two constituents.
    Very complex constructions would be impossible if they could not be conducted stage by stage.

    Some constructions involving more than two constituents can be reduced to a series of simpler (and more probable) constructions with the help of an auxilary molecule which gathers together the constituents one at a time in a stable structure pending the moment at which a complete inventory permits the whole to be assembled and separated from the auxiliary molecule.
    Such molecules are called catalysts.

    A crucial feature of life is self reproduction.
    Reproduction cannot take place in a vacuum, so typically a self-reproducing organism belongs and can survive and reproduce only in the ecological niche in which it evolved and to which it is adapted.
    The first self-replicating systems arising in the primordial soup are most likely autocatalytic sets.
   \emph{ An autocalytic set is a collection of molecular types of which some are inputs freely available in the environment of the autocatalytic set, some are outputs which may be though of either as waste products dispensed into the environment or as complex molecular products which provide ingredients for the contruction of ever more complex molecules, as a result of the autocatalytic process, and a set of catalysts which facilitate the process.}
    Together the inputs and catalysts suffice to produce all the outputs and new copies of the catalysts.
    In the presence of the catalysts the reactions involved can proceed faster than they otherwise would.
    The manufacture of further catalysts permits the scale of the operation to expand.
    In this way autocatalytic sets are like factories for producing complex molecules.

\paragraph{Cell Membranes}
    In the laboratory such reactions would be undertaken in a test tube or flask where a good concentration of the elements of the catalytic set could be ensured.
    In the primordial soup, the products would be likely to disperse, and the realisation or maintenance of sufficient concentration would be difficult.
    This would be much more likely to happen in more confined environments such as small pools.
    For the kinds of complex chemical reactions needed for life to take place, something closer to a test tube is needed, and this is ultimately provided by cell boundaries, the evolution of which remains poorly understood.

\subsubsection {Prokaryotes}

Prokaryotes are single celled organisms lacking a cell nucleus, and are the earliest known living organisms.
Their first appearances in the fossil record have been dated at 3.5B years ago, perhaps 1 Billion years after the formation of the Earth. 


It is interest that they exhibit a variety of social behaviours.

\paragraph{Prokaryotes and their social behaviour}
  \begin{itemize}
  \item  division of labour
  \item making bodily contact with as many other bacteria as possible is more important to an individual [myxobacterium] than sidling up to a food source
  \item much of the genetically determined behaviour is oblivious to the survival of the individual and totally oriented to the survival of the group, particularly that which occurs when times are hard and new resources must be located
  \item form of progeny varies according to conditions, eater/replicators when food is plenteous, explorers when it is scarce
  \item recklessness with life when searching for food, explorers mostly sacrifice themselves and their chances of replication
  \item chemical success/``come hither'' and failure/``avoid'' signals sent out by search parties
    \end{itemize}

\subsubsection{Sexual Reproduction}

\subsubsection{Language}

Homo Sapiens was \emph{defined} by Aristotle as the \emph{rational animal}.
Few are prepared to attribute rationality to any other species.
Along with that rationality, perhaps inextricably intertwined, is the capability for language, providing a vector for evolving culture, and beyond rationality that characteristic we call ``intelligence''.

The most conspicuous and rapidly evolving feature of genus \emph{homo} leading to the emergence of anatomically modern homo sapiens, was the size of brain.
Average brain volume (as measured by cranial capacity) had doubled over the preceding 1.8 Million years.
Language, rationality and intelligence are aspects of information processing which depend upon that increase in brain capacity and seem likely to have co-evolved with it and provided the selective advantage which directed the evolution.

Though it is often said that homo sapiens was the first in these various respects, intelligence, rationality, language, sociality, culture and the complex societies thus enabled, all these had their precursors, some going back to the beginning of life on earth.
To understand our capability for reason and rationality, and to understand why and how these may fail, it may help to consider these precursors, the evolutionary processes leading to homo sapiens, and the cultural evolution which lead down through `the age of reason' to the present.

\subsection{Sociality}

\subsection{Communication and Language}



\section{}

\cite{shapiro-tabp}

\cite{wilson-tvf}
\cite{berlinRR}
\cite{orwell-fd}


\phantomsection
\addcontentsline{toc}{section}{Bibliography}
\bibliographystyle{rbjfmu}
\bibliography{rbj}

\addcontentsline{toc}{section}{Index}\label{index}
{\twocolumn[]
{\small\printindex}}

%\vfill

%\tiny{
%Started 2021-02-02


%\href{http://www.rbjones.com/rbjpub/www/papers/p039.pdf}{http://www.rbjones.com/rbjpub/www/papers/p039.pdf}

%}%tiny

\end{document}

% LocalWords:
