% $Id: p035.tex $
% bibref{rbjp035} pdfname{p035}
\documentclass[10pt,titlepage]{article}
\usepackage{makeidx}
\newcommand{\ignore}[1]{}
\usepackage{graphicx}
\usepackage[unicode]{hyperref}
\pagestyle{plain}
\usepackage[paperwidth=5.25in,paperheight=8in,hmargin={0.75in,0.5in},vmargin={0.5in,0.5in},includehead,includefoot]{geometry}
\hypersetup{pdfauthor={Roger Bishop Jones}}
\hypersetup{pdftitle={Notes on Orenstein on Quine}}
\hypersetup{colorlinks=true, urlcolor=red, citecolor=blue, filecolor=blue, linkcolor=blue}
%\usepackage{html}
\usepackage{paralist}
\usepackage{relsize}
\usepackage{verbatim}
\usepackage{enumerate}
\usepackage{longtable}
\usepackage{url}
\newcommand{\hreg}[2]{\href{#1}{#2}\footnote{\url{#1}}}
\makeindex

\title{\LARGE\bf Notes on Orenstein on Quine}
\author{Roger~Bishop~Jones}
\date{\small 2020/04/20}


\begin{document}

%\begin{abstract}
% These are notes written in connection with a reading and discussion of Quine based on Orenstein.
% 
%\end{abstract}
                               
\begin{titlepage}
\maketitle

%\vfill

%\begin{centering}

%{\footnotesize
%copyright\ Roger~Bishop~Jones;
%}%footnotesize

%\end{centering}

\end{titlepage}

\ \

\ignore{
\begin{centering}
{}
\end{centering}
}%ignore

\setcounter{tocdepth}{2}
{\parskip-0pt\tableofcontents}

%\listoffigures

\pagebreak

\section*{Preface}

\addcontentsline{toc}{section}{Preface}

There are ``hyperlinks'' in the PDF version of this monograph which either link to another point in the document  (if coloured blue) or to an internet resource  (if coloured red) giving direct access to the materials referred to (e.g. a Youtube video) if the document is read using some internet connected device.
Important links also appear explicitly in the bibiography.

\section{Introduction}

\section{Expressing Ontology}

What is an ontology? Why would we want to express one? What is an ontological ``commitment''?
And why does any of this matter, aren't we in angels on pinheads territory here?

Ontology is concerned with existence, when used philosophically it is a branch of metaphysics.

An \emph{ontology} is an account, description or specification of what exists.
It may purport to be an \emph{absolute} account, giving a full inventory of everything that exists, concrete and abstract, or it may be partial, to support articulation of some particular science, or it may be an ontology offered as a pragmatic {\it convention}, applicable to some particular context or language, rather than as absolute truth.

Reasons for articulating an ontology are often scientific, ordinary discourse preferring liberal, flexible and context sensitive attitudes, often speaking {\it as if} a certain kind of entity existed (say, a route, or a detective called Holmes) without intending more than a {\it figure of speech}.
In science, precision of ontology is associated with rigour of reasoning.
Ontological confusion begets paradox and logical incoherence.
This applies particularly to mathematics, the subject matter most conspicuously dominated by long chains of deductive reasoning, whose pre-eminent deductive rigour is underpinned by that ontological precision arising from definition rather than observation.
It is because Quine's philosophy begins at a time when the logical foundations of mathematics have been transformed, while sailing perilously close to incoherence, that ontology looms large.

\subsection{Preliminary Observations}
  
As far as I can see here the only two things which appear here, are original with Quine, and may be significant in the rest of his philosophy are:

\begin{enumerate}
\item The dictum ``to be is to be the value of a variable''.
\item The notion of ``ontological commitment'' 
\end{enumerate}

Context for an understanding of Quine's views on these matters includes:
\begin{enumerate}[(a)]
\item That existence is ``not a predicate'' in modern logic
\item The trajectory of Russell ontological position and its continuation and analysis by Quine. 
  \begin{enumerate}[i]
  \item Russell's Meinongian phase.
  \item Russell's Theory of Descriptions \cite{russell1956,russellOD} and Incomplete Symbols \cite{russell10}
  \item Russell's {\it Theory of Types} \cite{russell1908}
  \item Quine's analysis in his ``Set Theory and its Logic'' \cite{quineSTAIL}
  \end{enumerate}
\end{enumerate}

As to 1), this originates in Quine's analysis of Russell's use of ``incomplete symbols'' in \emph{Principia Mathematica} \cite{russell10} and its application to set theory in Quine's later work.
I really don't think it is of much importance for what follows, but I give a fuller discussion of its origins and significance below.

The notion of ``ontological commitment'' is, as far as I am aware, a neologism of Quine's.
Previously philosophers such as Russell talked of the ``existential import of a proposition'', a topic on which Russell published a paper \cite{russell-existential}.
The main difference, it seems to me, in introducing this terminology, is to suggest that someone using certain forms of language themselves to certain ontological proposition whether they intended to or not.
This may be read as an implicit critique of Carnap's ontological views, which were conventionalistic and anti-metaphysical.
Consequently, I am myself inclined to deprecate the use of this terminology.

The two dictums are connected.
The connection implicates that if you want your variables to range over some kind of thing, then you are committed to the view that these things really do exist, a view which is never held by an anti-metaphysical ontological conventionalist.
The ``anti-metaphysical'' view here is that absolute ontological claims are meaningless, along with the rest of metaphysics.

Historically, questions associated with ontology had a particular imnportance in the context in which Quine was educated, because of the great imnportance of the work in logic which took place around the turn of the 20th Century.

\subsection{The Rigorisation of Analysis and its Logical Foundations}

Well for context let's go back to the year 1900.
This is a new century after a century of major developments in the foundations of mathematics which have latterly focussed on `logical' foundations.

Why did the foundations need fixing?
Well, possibly the most important ever discovery in mathematics was the independent development by Newton and Leibniz of the differential and integral calculus, which paved the way for the quantitative science which underpinned the industrial revolution.
The foundations of mathematics, specifically of the calculus, were attacked at the time by Bishop Berkeley, particularly in relation to the use by Leibniz of ``infinitesimals'' (or fluxions in Newton's version).
However tenuous the idea of infinitesimal might then have been, the calculus worked in practice, and enabled widespread application of Newtonian mechanics to science and engineering.
It was an unstoppable juggernaut sweeping aside intellectual doubts for the duration of the eighteenth century.

Come the nineteenth, even mathematicians began to have doubts.

The mathematician Abel said in 1828:
\begin{quote}
  {\it There are very few theorems in advanced analysis which have been demonstrated in a logically tenable manner. Everywhere one finds this miserable way of concluding from the special to the general, and it is extremely peculiar that such a procedure has lead to so few of the so-called paradoxes.}
\end{quote}

Once mathematicians started to put their house in order a series of developments gradually took the foundations deeper and deeper until the prospect of reducing mathematics to logic (depending on exactly what you mean by that) became real, and it was this that stimulated the developments in logic which completely overtook the logic of Aristotle for the first time in over 2000 years, created the discipline of Mathematical Logic, completely transformed Philosophical Logic and the Philosophy of Logic, and later enabled extensive work in (and applications of) logic and formal semantics in theoretical Computer Science.

The progression took place first by eliminating the need for infinitesimals using limits of sequences of `real' numbers, and then by the arithmetisation of analysis by progressively defining real numbers as sets of rational numbers, rational numbers as equivalence classes of pairs of integers, and integers in terms of the natural numbers (arithmetic). (or something like that, there is more than one way of skinning this cat).
In the latter part of the 19th century attention turned to irrational numbers.
Real numbers were defined by Dedekind as certain sets of rationals.
The theory of rational and natural numbers were then clarified in turn, ultimately reducing all of these systems to set theory and logic.

By these methods the differential and integral calculus, (usually referred to by mathematicians as analysis) had been ``reduced'' to arithmetic, the arithmetisation of analysis was complete.

The mathematical project of improving the rigour of analysis having been successfully undertaken, a philosophical project now takes center stage: logicism the thesis that Mathematics is Logic.
Hume's distinction between ``relations between ideas'' and ``matters of fact'', affirming mathematics and all necessary truth to be in the first category, was disputed by Hume, who introduced the analytic/synthetic terminology and classifying mathematics as synthetic a priori.
Both Frege and Russell disagreed with Kant on the status of mathematics, and both were determined to set the record straight by

All that remained was to show that natural numbers could be given a logical definition and to exhibit a formal system of logic capable of deriving the required properties.

\subsection{Frege's Logicism}

\begin{quote}
{\it  I hope I may claim in the present work to have made it probable that the laws of arithmetic are analytic judgements and consequently \emph{a priori}.
Arithmetic thus becomes simply a development of logic, and every proposition of arithmetic a law of logic, albeit a derivative one.}
\end{quote}
The Foundations of Arithmetic \cite{frege1884}, §87.

To that end he created a new formal logic which he called ``Concept Script'' ({\it Begriffsschrift})\cite{heijenoort67,frege1879}, published in 1879.
This was a ground-breaking work the ``fundamental contributions'' of which are summarised by Van Heijenoort as:

\begin{enumerate}
\item the truth-functional propositional calculus
\item the analysis of proposition into function and argument rather than subject and predicate
\item the theory of quantification
\item a system of logic based exclusive on the syntactic structure of the expressions
  \item a logical definition of the notion of mathematical sequence
  \end{enumerate}


It was the first mathematical work in logic in which formal rules of derivation were codified.
Previous mathematical work relevant in logical matters had been, like Booles work on his eponymous algebra of propositions or truth values, algebraic in character, investigated using the extant and uncodified deductive methods of mathematical proof.

Secondly, it introduced variables and quantification (the universal quantifier which by negation yields the existential), abandoning the Aristotelian propositional forms which had ruled logic for millenia.

\subsubsection{Variables and Quantification}

Aristotle was the first to study and write about logic, and though there had been much further work in the two millenia which followed his innovations, the basic structure which he laid out had remained intact, and was inadequate to give an account of the reasoning undertaken in the queen of the deductive sciences, mathematics.

Aristotle's ``syllogistic'' logic was build around a model of language as ''categorical propositions''.
A categorical proposition is always in ``subject-predicate'' form, i.e. consists of the application of a predicate to some subject.
For example, in the proposition ``All men are mortal'' the predicate `mortal' is asserted of the subject `all men' (using the copula `are').
The subject may be singular or universal, a singular subject being an individual (perhaps `Socrates'), a universal being some category (e.g. `all men').
The predicate may only be applied to a single subject, and therefore some difficulty arises in asserting relations, and the possibility of asserting relations between subjects within the constraints imposed by this conception of categorical proposition was a matter of controversy, asserted for example, by Leibniz and denied by Russell.

There are just four variations on that theme which are admitted in Aristotle's logic as follows.

\begin{table}[h]
\begin{center}
  \caption{forms of categorical proposition}
\begin{tabular}{|l|l|l|}
\hline
form & proposition & paraphrase \\
\hline
Aab &	a belongs to all b & Every b is a \\
Eab &	a belongs to no b & No b is a \\
Iab &	a belongs to some b & Some b is a \\
Oab &	a does not belong to all b & Some b is not a \\
\hline
\end{tabular}
\end{center}
\end{table}

Quantification applied, in effect, to propositional {\it functions} bringing into logic the important mathematical notion of function and the expressive power which it carried.

\subsubsection{The {\it Grundgesetze}}

He went on to apply his {\it Begriffsschrift} (after some further development) to the formal derivation of mathematics in his \emph{Grundgesetze der Arithmetik}\cite{frege1893,frege1903}.
The first volume was published in 1993, but Russell had only become acquainted with Frege's work in 1900.

\subsection{Russell's Paradox}

In the course of completing his manuscript for \emph{The Principles of Mathematics} (sometime between 1900 and 1902) Russell realised that a contradiction could be obtained in Frege's logical system.
The contradiction arose from rules about existence which were too liberal, admitting the existence of what is now sometimes known as {\it the Russell set} and engendering {\it Russell's paradox} (also attributed to Zermelo and Cantor).

This paradox arises from the availability in the system of unrestricted set comprehension, the idea that for any property expressible in the notation there exists a set containing all and only those things which satisfy the property.
This includes the set of sets which do not include themselves, the existence of which proves to be paradoxical, as is seen by asking the question whether this set does or does not contain itself.
An alternative presentation, for those not wholly comfortable with abstract entities is the barber who cuts the hair of everyone in his home town who does not cut his own hair, and cnmsequenmtly both does and does not cut his own hair.

He communicated this discovery to Frege in 1902, just as the second volume of the {\it Grudgesetze} was about to go to press, a devastating blow to his work from which Frege never fully recovered.

Russell remained intent on his own work to the same end, the logicisation of mathematics.

He was henceforth aware of the perilous danger of contradiction from what might seem innocuous ontological principles, and there ensued a period of great intellectual difficulty as he struggled to find a logical system which avoided the paradoxes in a way which was both philosophically rational and technically adequate for the derivation of mathematics.

The paradox having an ontological flavour, the evolution of Russell's views on ontology was important, and feeds into the Quinian doctrines under discussion.
This is a part but by no means the whole of the development of Russell's logical system {\it The Theory of Types} \cite{russell1908}.

But first, by way of showing Russell's dedication to that Fregean enterprise a quotation from the preface to the second edition of {\it The Principles of Mathematics} (1937):

\begin{quote}
  {\it The fundamental thesis of the following pages, that mathematics and logic are identical, is one which I have never since seen any reason to modify.
  }
\end{quote}

\subsection{Descriptions and Other Incomplete Symbols}

At this stage Russell was sympathetic to the lavish ontological conclusions of Meinong \cite{meinong-gegenstandstheorie-a,meinong-gegenstandstheorie-b}, which may be seen in the following quotation from {\it The Principles of Mathematics}.

\begin{quote}
  {\it What does not exist must be something, or it would be meaningless to deny its existence; and hence we need the concept of being, as that which belongs even to the non-existent.}
\end{quote}
(1903) paragraph 427\cite{russell03}

He has thus separated the concepts of {\it existence} and of {\it being}, the latter including many things which do not exist.

His ontology will soon become more spartan, and the notion of {\it being} as distinct from existence will no longer be required.
This is realised through Russell's {\it Theory of Descriptions}, pubished in his paper {\it On Denoting} in 1905 \cite{russellOD}.

Russell's theory of descriptions is a method for expressing in the language of Principia (i.e. Russell's Theory of Types) sentences which assert something of an object identified using a description, where it is possible that nothing satisfies the description.
The method treats the description as having meaning only in context not in isolation, and in a suitable context gives it meaning by translating into an expression in which the description no longer appears.
Because the description does not by itself have any meaning it is called an ``incomplete symbol''.
This trick is used in Principia Mathematica for things other than descriptions, notably for class abstracts, which refer to a class by giving a predicate which characterises (determines) the members of the class.
By this means Russell's Type Theory was considered by Russell a ``no class'' theory, because though Principia uses language which looks like its referring to classs, this is all incomplete symbols which are eliminable by following the prescribed rules for translating these away in the appropriate contexts.

When later Quine is developing ideas for set theoretic (or class theoretic) ways of logically deriving mathematics, he looks closely into how much can be achieved using ``virtual classes'' (i.e. class notations which are incomplete symbols and hence don't really denote classes) and when this technique runs out of steam and you really need classes rather than the pretence.
The answer is, ``when you need to quantify over them'', because if you use virtual classes they don't really exist and so they won't be in the range of the quantifiers, that's really the whole point.

Quine's analysis here is very nice, and may be found in his book ``Set theory and its logic''\cite{quineSTAIL}.
However, its philosophical significance is not in my opinion much to write home about.
Injecting it into philosophical ontological discussion with the dictum ``to be is to be the value of a variable'' is not very impressive, this is a bit ambiguous and is either a truism or falsehood according to how you read it.

It embodies the assumption that quantifiers range over the entirety of some collection of things which in some absolute sense exist.
Some, including Carnap and me, don't believe that there are any ontoogical absolutes, questions about ontology are never meaningless except in some context which determines the relevant ontology


\phantomsection
\addcontentsline{toc}{section}{Bibliography}
\bibliographystyle{rbjfmu}
\bibliography{rbj}

%\addcontentsline{toc}{section}{Index}\label{index}
%{\twocolumn[]
%{\small\printindex}}

%\vfill

%\tiny{
%Started 2020/07/06


%\href{http://www.rbjones.com/rbjpub/www/papers/p032.pdf}{http://www.rbjones.com/rbjpub/www/papers/p035.pdf}

%}%tiny

\end{document}

% LocalWords:
