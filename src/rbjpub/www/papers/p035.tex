% $Id: p035.tex $fi
% bibref{rbjp035} pdfname{p035}
\documentclass[10pt,titlepage]{book}
\usepackage{makeidx}
\usepackage{turnstile,amssymb}
\newcommand{\ignore}[1]{}
\usepackage{graphicx}
\usepackage[unicode]{hyperref}
\pagestyle{plain}
\usepackage[paperwidth=5.25in,paperheight=8in,hmargin={0.75in,0.5in},vmargin={0.5in,0.5in},includehead,includefoot]{geometry}
\hypersetup{pdfauthor={Roger Bishop Jones}}
\hypersetup{pdftitle={Carnap's Programme and AI Safety}}
\hypersetup{colorlinks=true, urlcolor=red, citecolor=blue, filecolor=blue, linkcolor=blue}
%\usepackage{html}
\usepackage{paralist}
\usepackage{relsize}
\usepackage{verbatim}
\usepackage{enumerate}
\usepackage{longtable}
\usepackage{url}
\newcommand{\hreg}[2]{\href{#1}{#2}\footnote{\url{#1}}}
\makeindex

\title{Miscellany}
\author{Roger~Bishop~Jones}
\date{\small 2023/07/01}


\begin{document}
\frontmatter

%\begin{abstract}
% A sympathetic account of Carnap's philosophical programme, and a reconstruction providing one approach to AI safety (and productivity).
%
%\end{abstract}
                               
\begin{titlepage}
\maketitle

%\vfill

%\begin{centering}

%{\footnotesize
%copyright\ Roger~Bishop~Jones;
%}%footnotesize

%\end{centering}

\end{titlepage}

\ \

\ignore{
\begin{centering}
{}
\end{centering}
}%ignore

\setcounter{tocdepth}{2}
{\parskip-0pt\tableofcontents}

%\listoffigures

\mainmatter

\section*{Preface}

I first came across the work of Rudolf Carnap as an undergraduate in about 1973, but I knew little of the Carnap's ambition, or even of his emnbrace of formal logic, for another 20 years, when I sidled back in the direction of philosophy after a decade of engagement in the development and application  of methods and tools for formal specification and reasoning directed toward secure computing.

It was only then that I began to learn of his programme, and of some of the road blocks which had been placed in its way by prominent philosophers such as Quine and Kripke, and which still seem to obstruct the way.
I have struggled, on and off. for the decades since then, to find a way to articulate my own similar conviction in the important of formal methods for the future of science, engineering and reason in general and its relevance to making machines not only creative and intelligent, but sound (in the logical sense) and safe.

This is my latest approach to that endeavour, in which, for the first time, I put telling Carnap's story front and center while hoping to close with a successor that he might have been happy to support.

\section{Introduction}

It is important in trying to understand any thinker to comprehend what they were trying to do and the historical and biographical context in which those ambitions were conceived, before looking at the details of how the aims were progressed and evaluating their success and considering whether they can and should be furthered.

I have never yet seen a presentation of Carnap's ideas by anyone who seemed to be enthusiastic for the aims of his programme.
Justice can best be done to the programme by someone who shares or sympathises withits purpose, and that is my principle qualification in undertaking the task.

I am no scholar, I shall never be expert in all the details of Carnap's work, so I have relied heavily on the volume of ``The Library of Living Philosophers'' which was published toward the end of his philosophical career and on Carnap's own ``intellectual autobiography'' in that Volume, as well as his description of his positions on the various aspects of his work at that time \cite{carnap63a,carnap63}.


\section{Background}

The spine of this essay is a story about modern efforts to progress the idea that deductive reason can  through the adoption of formal notations and the codification of both the semantics of and inference in such notations can advance the precision and rigour with which all enterprises involving deduction (including but not limited to philosophy, mathematics, science and engineering) are conducted.

Elementary deduction, I suggest, is coeval with human language, for many relatively simple concepts are so transparently compounded from simpler concepts that one cannot be said to understand the concepts with being aware of a relation of entailment between them and the consequent licence to deductively infer one from the other, provide a simple example, though perhaps not one relevant to the camp fire of early homo sapiens.
To a very gross approximation let us say that elementary deduction dates back some two hundred thousand years.

The systematic use of deduction for complex reasoning seems very much more recent, and may coincide with the beginning of Mathematics as a theoretical discipline in ancient Greece from about 600 BCE.
Aristotle (384-322 BE) seems to have been the first to write about logic in a systematic way.
The volumes later gathered together as ``The Organon'' provide, \emph{inter alia} his presentation and analysis of the \emph{syllogism} and a story about how logic contributes to science in his conception of \emph{demonstrative science}, a forerunner of the more modern concept of the hypothetico-deductive science.

Aristotle's syllogistic was the first approach to a formal logical system, but fell well short of encompassing all those kinds of deductive reasoning which were used by his contemporaries in their mathematics.
It was nevertheless was to dominate subsequent teaching in logic for the next two and a half millenia, during which the valuable further research undertaken did not extend the reach of the syllogistic.

The next major advance in this story comes from Leibniz.
It is firmly rooted in Aristotelian Logic, which Leibniz believed provided a way in which science could be mechanised.




\section{The Programme}




\pagebreak
\phantomsection
\addcontentsline{toc}{section}{Bibliography}
\bibliographystyle{rbjfmu}
\bibliography{rbj2}

%\addcontentsline{toc}{section}{Index}\label{index}
%{\twocolumn[]
%{\small\printindex}}

%\vfill

%\tiny{
Started 2023/07/01


\href{http://www.rbjones.com/rbjpub/www/papers/p034.pdf}{http://www.rbjones.com/rbjpub/www/papers/p035.pdf}

%}%tiny

\end{document}

% LocalWords:
