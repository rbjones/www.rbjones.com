% $Id: p019.tex,v 1.1 2012/11/01 15:59:56 rbj Exp $
% bibref{rbjp019} pdfname{p019}

\documentclass[10pt,titlepage]{article}
\usepackage{makeidx}
\usepackage{graphicx}
\usepackage[unicode,pdftex]{hyperref}
\pagestyle{plain}
\usepackage[paperwidth=5.25in,paperheight=8in,hmargin={0.3in,0.3in},vmargin={0.3in,0.3in},includehead,includefoot]{geometry}
\hypersetup{pdfauthor={Roger Bishop Jones}}
\hypersetup{colorlinks=true, urlcolor=red, citecolor=blue, filecolor=blue, linkcolor=blue}
\usepackage{html}
\usepackage{paralist}
\usepackage{relsize}
\usepackage{verbatim}
\bodytext{BGCOLOR="#eeeeff"}
\makeindex
\newcommand{\indexentry}[2]{\item #1 #2}
\newcommand{\glossentry}[2]{\item #1 {\index #1 #2}}
\newcommand{\ignore}[1]{}
\def\Product{ProofPower}
\def\ouml{\"o}
\def\auml{\"a}

\title{What is Philosophy?}
\author{Roger~Bishop~Jones}
\date{\ }

%\begin{abstract}
% A lightweight look at various kinds of philosophy.
%\end{abstract}

\begin{document}
%\frontmatter
                               
\begin{titlepage}
\maketitle

\ignore{
\vfill

\begin{abstract}

\end{abstract}

\begin{centering}

{\footnotesize

\copyright\ Roger~Bishop~Jones;
}%footnotesize

\end{centering}
}%ignore

\end{titlepage}

\setcounter{tocdepth}{2}
{\parskip-0pt\tableofcontents}

%\listoffigures

%\mainmatter

\section{Introduction}

A first answer to the question ``what is philosophy?'' might be ``all things to all men''.

There are many different kinds of philosophy and no single answer will address them all, so we are immediately drawn into the question ``what kinds of philosophy are there?''.

This question too has many different answers, so I will sketch seversl.

\section{Ways of Cutting The Cake}

\subsection{A first cut}

My first inclination was to regard philosophy as of three kinds:

\begin{enumerate}
\item Commonsense philosophy
\item Theological philosophy
\item ``Western'' philosophy
\end{enumerate}

Commonsense philosophy is what ordinary people might be said to be engaged in when they grapple, without the benefit of any kind of special philosophical training, in a relatively abstract way, with certain of the problems which concern them.
Many of the discussions of our group naturally fall into this category, such as ``Love''.

The other two general classifications relate to kinds of philosophy which have been to some degree professionalised.
In libraries and bookshops ``Philosophy'' is commonly juxgtaposed with ``Religion'', and for the larger part of the history of mankind philosophical problems have been discussed primarily in the context of religious belief systems, in which case it may be more a case of receiving instruction from figures of authority than engaging in discussion or debate.

``Western Philosophy'' is a generic term for the philosophising which has taken place in the context of an intellectual tradition which begins in ancient Greece about 2600 years ago.
This tradition is distinctive because of its departure from the handing down of religious or philosophical dogma by figures of authority, in favour of a search for understanding through reason in which individuals make their own judgements rather than being expected to believe what they are told. 

A major figure in the beginnings of Western Philosophy is Aristotle, and it is useful to consider some of his ideas on what philosophy is, in particular in relation to a special kind of philosophy which he called ``first philosophy''.

Not far removed from this analysis of the kinds of philosophy is one given much later by the French philosopher Auguste Comte.

\subsection{Comte's Three States as Kinds of Philosophy}

Comte identified three ``states'' which are three stages in the development of the human mind and correspond to different kinds of philosophy.

They are:

\begin{center}
\begin{description}
\item[theological]

    In which the hidden nature of things is sought by constructing divine beings in man's own image.

\item[metaphysical]

    In which the hidden nature of things is sought without resort to supernatural beings.

\item[positive]

    In which questions about the hidden nature of things are rejected as pathological in some way (e.g. meaningless, or verbal).
\end{description}
\end{center}

    In the positive state science seeks to discover universal laws governing phenomena by observation and experiment. The scientist does not seek beyond these laws governing phenomena for deeper explanations, especially not explanations which involve entities which are not directly observable.

The first two stages correspond to the division between religious philosophy and ``western philosophy'' as initiated in ancient Greece.
The third state comes later, and may be associate with the transition from the metaphysics and science of ancient Greece to modern science and positivistic philosophy.
Prominent figures in this latter transition were Roger Bacon, Gallileo, and David Hume.

In neither of these transitions does the later ``stage'' replace the earlier.
Religious belief systems coexist with scientific, sometimes in harmony sometimes in conflict.
In philosophy there is a long running conflict between metaphysical philosophers and positivistic philosophers, which is related to that between rationalists and empiricists.

\subsection{Aristotelian Divisions}

Its worth bearing in mind in this section and the next, that in Aristotle the divisions between art, science and philosophy are not as sharply drawn as the have become for us.
Aristotle was very interested in defining and classifying, and his terms and classifications were use in education, science and philosophy for thousands of years and are embedded in our language.

One important division within western philosophy is that between theoretical and practical philosophy, and this has its roots in Aristotle.

For Aristotle ``science'' was divided into three parts:

\begin{enumerate}
\item productive sciences
\item practical sciences
\item theoretical sciences
\end{enumerate}

Productive science was concerned with making things, and so corresponds to what we would call engineering rather than science, but was rather broader.

The division between practical sciences and theoretical sciences correponds to a division still relevant to contemporary philosophy.
The practical sciences are those involving values and conduct, theoretical sciences to knowledge.

Besides this three-way division of science, there were certain topics which Aristotle did not regard as sciences in their own right, but which provided a basis on which science could be conducted.
These were Metaphysics and Logic.
They were considered pre-requisites for science rather than science themselves, and the six volumes which Aristotle wrote on Logic were later called ``The Organon'' to reflect this idea that the material were to be used in the conduct of science rather than constituting science.
``Organon'' is the Greek word for ``tool''.
In a modern conception of the division of philosophy metaphysics and logic are considered theoretical rather than practical philosophy.

Thus practical philosophy includes the following subdisciplines of philosophy:

\begin{itemize}
\item Aesthetics
\item Ethics
\item Political Philosophy
\end{itemize}

and theoretical philosophy consists of:

\begin{itemize}
\item Metaphysics
\item Logic
\item Epistemology
\item Philosophy of Language
\item Philosophy of Science
\end{itemize}

The distinction being drawn primarily on the basis of whether the enterprise is purely descriptive or involves values.

\section{Aristotle on First Philosophy}

\subsection{True Wisdom}

The following items exemplify an apparently linear ordering on kinds of knowledge which is used by Aristotle to identify ``Wisdom''.

\begin{itemize}
\item sensation
\item memory
\item experience
\item skilled art
\item usefully creative art
\item purely creative art
\end{itemize}

The references to ``art'' here are not to be taken as contrasting with ``science'', for Aristotle takes ``the mathematical sciences'' to be a prime example of what I have named ``purely creative art'' (for the distinction between science and art he refers us to his Ethics).

Connected with this ordering there are various other ideas.
As we progress along this scale greater intelligence is involved, and greater good.
The knowledge becomes more ``theoretical'', the posessor more suited for teaching rather than learning from others, and for giving rather than receiving instructions or ordered, for being in authority.
An aspect of the characterisation of the scale is the transition from mere acquaintance with a phenomenon, understanding the general rules according to which it operates, and understanding the reasons or causes of that manner of operation.
It is this knowledge of reasons or causes which becomes the dominant characterisation of true wisdom, which is distinguished from lesser forms of theoretical knowledge by the kinds of causes it deals with, the \emph{first} causes.

Aristotle takes ``wisdom'', the subject matter of first philosophy, to be the highest point on this scale, and he takes it to be concerned with certain principles and causes.

At this earliest point we can distinguish between Aristotle's characterisation of first philosophy and his conclusions about its subject matter.
In the latter part of the twentieth century there was among some philosophers a rejection of the idea of ``first philosophy'', and among others a resurgence of interest in metaphysics.
Possibly these were just different groups of philosophers who disagreed, but quite possibly those who rejected ``first philosophy'' were not rejecting metaphysics in general, and those who persued metaphysics did not conceive of themselves as practicing ``first philosophy''.

\subsection{Metaphysics}

%\backmatter

%\appendix

%\addcontentsline{toc}{section}{Bibliography}
%\bibliographystyle{alpha}
%\bibliography{rbj}

%\addcontentsline{toc}{section}{Index}\label{index}
%{\twocolumn[]
%{\small\printindex}}

%\vfill

%\tiny{
%Started 2012-10-19

%Last Change $ $Date: 2012/11/01 15:59:56 $ $

%\href{http://www.rbjones.com/rbjpub/www/papers/p019.pdf}{http://www.rbjones.com/rbjpub/www/papers/p019.pdf}

%Draft $ $Id: p019.tex,v 1.1 2012/11/01 15:59:56 rbj Exp $ $
%}%tiny

\end{document}

% LocalWords:
