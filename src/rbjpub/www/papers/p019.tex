% $Id: p019.tex,v 1.5 2014/11/08 19:43:29 rbj Exp $
% bibref{rbjp019} pdfname{p019}

\documentclass[10pt,titlepage]{article}
\usepackage{makeidx}
\usepackage{graphicx}
\usepackage[unicode]{hyperref}
\pagestyle{plain}
\usepackage[paperwidth=5.5in,paperheight=8in,hmargin={0.3in,0.3in},vmargin={0.3in,0.3in},includehead,includefoot]{geometry}
\hypersetup{pdftitle={What is Philosophy?}}
\hypersetup{pdfauthor={Roger Bishop Jones}}
\hypersetup{colorlinks=true, urlcolor=red, citecolor=blue, filecolor=blue, linkcolor=blue}
%\usepackage{html}
\usepackage{paralist}
\usepackage{relsize}
\usepackage{verbatim}
\makeindex
\newcommand{\ignore}[1]{}

\title{What is Philosophy?}
\author{Roger~Bishop~Jones}
\date{\ }

\begin{document}
%\frontmatter
                               
\begin{titlepage}
\maketitle

%\begin{abstract}
%Some thoughts about the nature of philosophy and the kinds of philosophy.
%\end{abstract}

%\vfill

%\begin{centering}

%{\footnotesize
%copyright\ Roger~Bishop~Jones;
%}%footnotesize

%\end{centering}

\end{titlepage}

\setcounter{tocdepth}{2}
{\parskip-0pt\tableofcontents}

%\listoffigures

%\mainmatter

\section{Preface}

This document contains several approaches to the question ``What is philosophy'', which I have tried to make diverse, and have not attempted to integrate, creating a new section every time I return to the topic.
The introduction is intended to provide a brief survey.

\section{Introduction}

There are many different kinds of philosophy and no single answer will address them all, so we are immediately drawn into the question ``what kinds of philosophy are there?'', and section \ref{CuttingCake} approaches the problem from this point of view, considering some of the ways in which the different kinds of philosophy might be characterised.
Among others, we consider here Aristotle's classification of the sciences.
The division between science and philosophy as we know it today is not present in Aristotle, so this classification of {\it science} has more bearing on philosophy than than might otherwise be expected.

In section \ref{AristotleMetaphysics} we dig deeper into Aristotle's wisdom, looking specifically at what he says about ``First Philosophy''.

In section \ref{Scruton} we review the ways in which Roger Scruton tries to answer the question ``What is Philosophy?'' in his book ``Modern Philosophy''\cite{scruton97}.

In section \ref{Answers} the characterisation of philosophy is approached through the question ``Does Philosophey provide answers?'', or should we expect that {\it philosophical} questions are ultimately unanswerable.


\section{Ways of Cutting The Cake}\label{CuttingCake}

\subsection{A first cut}

My first inclination was to regard philosophy as of three kinds:

\begin{enumerate}
\item Commonsense philosophy
\item Theological philosophy
\item ``Western'' philosophy
\end{enumerate}

Commonsense philosophy is what ordinary people might be said to be engaged in when they grapple, without the benefit of any kind of special philosophical training, in a relatively abstract way, with certain of the problems which concern them.
Many of the discussions of our group naturally fall into this category, such as ``Love''.

The other two general classifications relate to kinds of philosophy which have been to some degree professionalised.
In libraries and bookshops ``Philosophy'' is commonly juxtaposed with ``Religion'', and for the larger part of the history of mankind philosophical problems have been discussed primarily in the context of religious belief systems, in which case it may be more a case of receiving instruction from figures of authority than engaging in discussion or debate.

``Western Philosophy'' is a generic term for the philosophising which has taken place in the context of an intellectual tradition which begins in ancient Greece about 2600 years ago.
This tradition is distinctive because of its departure from the handing down of religious or philosophical dogma by figures of authority, in favour of a search for understanding through reason in which individuals make their own judgements rather than being expected to believe what they are told. 

A major figure in the beginnings of Western Philosophy is Aristotle, and it is useful to consider some of his ideas on what philosophy is, in particular in relation to a special kind of philosophy which he called ``first philosophy''.

Not far removed from this analysis of the kinds of philosophy is one given much later by the French philosopher Auguste Comte.

\subsection{Comte's Three States as Kinds of Philosophy}

Comte identified three ``states'' which are three stages in the development of the human mind and correspond to different kinds of philosophy.

They are:

\begin{center}
\begin{description}
\item[theological] \ 

    In which the hidden nature of things is sought by constructing divine beings in man's own image.

\item[metaphysical] \ 

    In which the hidden nature of things is sought without resort to supernatural beings.

\item[positive] \ 

    In which questions about the hidden nature of things are rejected as pathological in some way (e.g. meaningless, or verbal).
\end{description}
\end{center}

    In the positive state science seeks to discover universal laws governing phenomena by observation and experiment. The scientist does not seek beyond these laws governing phenomena for deeper explanations, especially not explanations which involve entities which are not directly observable.

The first two of Comte's stages correspond to the division between religious philosophy and ``western philosophy'' as initiated in ancient Greece.
The third state comes later, and may be associate with the transition from the metaphysics and science of ancient Greece to modern science and positivistic philosophy.
Prominent figures in this latter transition were Roger Bacon, Gallileo, and David Hume.

In neither of these transitions does the later ``stage'' replace the earlier.
Religious belief systems coexist with scientific, sometimes in harmony sometimes in conflict.
In philosophy there is a long running conflict between metaphysical philosophers and positivistic philosophers, which is related to that between rationalists and empiricists.

\subsection{Aristotelian Divisions}

Its worth bearing in mind in this section and the next, that in Aristotle the divisions between art, science and philosophy are not as sharply drawn as they have become for us.
Aristotle was very interested in defining and classifying.
His terms and classifications were used in education, science and philosophy for thousands of years and are embedded in our language.

One important division within western philosophy is that between theoretical and practical philosophy, and this has its roots in Aristotle.

For Aristotle ``science'' was divided into three parts:

\begin{itemize}

\item theoretical sciences - concerning knowledge

\item practical sciences - concerning conduct

\item productive sciences - concerning the making of useful or beautiful objects

\end{itemize}

The division between practical sciences and theoretical sciences correponds to a division still relevant to contemporary philosophy.
The practical sciences are those involving values and conduct, theoretical sciences to knowledge.

Thus the theoretical sciences were mathematics, physics and theology or metaphysics.

Logic does not fall under any of these categories for Aristotle, being concerned with methods common to and prerequisite for all these kinds of science.
The six volumes which Aristotle wrote on Logic were later called ``The Organon'' (the Greek word for ``tool'') to reflect this idea that the ideas and techniques were to be used in the conduct of science rather than themselves constituting science.

This distinction between theoretical and practical is now used specifically for philosophy, and philosophical logic is considered a part of theoretical philosophy (mathematical logic is a part of mathematics).

Thus practical philosophy, concerned with values and conduct, includes the following subdisciplines of philosophy:

\begin{itemize}
\item Aesthetics
\item Ethics
\item Political Philosophy
\item ...
\end{itemize}

but also includes the doctrines of those schools of thought whose primary concerns were with the conduct of life, such as scepticism, stoicism, epicureanism (in ancient Greece) and more modern schools such as existentialism.

Theoretical philosophy includes:

\begin{itemize}
\item Metaphysics
\item Logic
\item Epistemology
\item Philosophy of Language
\item Philosophy of Science
\item Philosophy of Mind
\end{itemize}

and many other specialities.

The distinction being drawn primarily on the basis of whether the enterprise is purely descriptive or involves values.

\ignore{
\subsection{Rationalism and Empiricism}

In the early modern period of philosophy certain philosophers have been 

\subsection{Hume's Forks}

There is 
}%ignore

\subsection{A 20th Century Divide - Analytic v. Continental}

The twentieth century was called ``the age of analysis'', the idea that philosophy should consist of logical or linguistic analysis having been introduced by Russell and Moore around the turn of the twentieth century.
This proved highly influential, and this conception of philosophy (which is entirely theoretical) dominated academic philosophy for most of the twentieth century.
However, in continental Europe, particularly in France, a different approach to philosophy remained important, and was called by Anglo-Saxon's ``continental philosophy''.
The best known school of continental philosophy is Existentialism, and its most famous 20th century exponent Jean Paul Sarte a philosopher/novelist.
Existentialism is substantially practical rather than theoretical.

Possibly this divide may be seen as following from an early divide between a predominantly continental ``rationalist'' tendency in which metaphysics plays a prominent role, and a British empiricist tradition in which knowledge is seen as firmly rooted in empirical evidence and metaphysics is repudiated.
However, there are many counterexamples to these generalisations, for example, positivism, a variety of empiricism is as much (if not more) a continental phenomenon as British.

\subsection{Hume's Forks}

David Hume noticed two cleavages in knowledge.
One between descriptive and evaluative claims, which he separated with the dictum that one cannot derive an ``ought'' from an ``is''.
The other a fundamental divide within descriptive knowledge which he described as concerning either ``relations between ideas'' or ``matters of fact''.
These correspond to logical or analytic truths on the one hand, and synthetic, empirical, contingent truths on the other.

In treating both of these kinds of knowledge as important Hume provides a kind of synthesis between the rationalism and empiricism, but a synthesis which leaves no place for metaphysics, and so remains within the empiricist tradition.
This is regarded by many as the beginning of positivism (though before that term is coined) and this distinction plays a key role in the subsequent history of positivism.

Though Hume banished metaphysics as this is often understood, in his own language he uses the term metaphysics more as a synonym for philosophy and considers himself to be offering a new kind of metaphysics.
The place for this in Hume's classification is in the ``matters of fact'', since he advocates that philosophy should be conducted following newtonian scientific method as an empirical enquiry into human understanding.
Later in the twentieth century Rudolf Carnap gives this same classification of knowledge a fundamental place in his own positivistic philosophy (\emph{Logical} positivism), but now within the new conception of philosophy as analysis considers philosophy to belong to the ``relations between ideas'' category, the analytic propositions (though substantially concerned with methods rather than doctrine).

\section{Aristotle on First Philosophy}\label{AristotleMetaphysics}

The topic ``Metaphysics'' was first used after Aristotle to name a book he wrote on ``first philosophy''.
In examining Aristotle's conception of ``first philosophy'' and the modern conception of ``metaphysics'' it is useful to consider four stages, to consider each of these stages on their own merits and to consider the relationship between the stages.

The stages are:
\begin{enumerate}
\item Aristotle's examination of which kinds of knowledge should be considered ``true wisdom''.
\item The conception of ``first philosophy'' which flows from it.
\item Aristotle's taking ``substance'' as the subject matter of ``first philosophy''.
\item Modern conceptions of ``metaphysics''.
\end{enumerate}

\subsection{True Wisdom}

The following items exemplify an apparently linear ordering on kinds of knowledge which is used by Aristotle to identify ``Wisdom''.

\begin{itemize}
\item sensation

\item memory

Animals with memory are ``more intelligent and apt at learning'' than those without.
Animals with hearing can also be taught.

\item experience

The several memories of the same thing produce finally the capacity for a single experience.
Science and art come to men through experience.

\item skilled art

``Art arises when from many notions gained by experience one universal judgement about a class of objects is produced.''

``To have a judgement that when Callias was ill of this
disease this did him good, and similarly in the case of Socrates and in many
individual cases, is a matter of experience; but to judge that it has done good
to all persons of a certain constitution, marked off in one class, when they
were ill of this disease, e.g. to phlegmatic or bilious people when burning
with fevers - this is a matter of art.''

``experience is knowledge of individuals, art of universals''

``knowledge and understanding belong to art rather than to experience''

``masterworkers in each craft ... are wiser than the manual workers, because they know the causes''

``artists can teach, and men of mere experience cannot''

\item usefully creative art

``At first he who invented any art whatever that went beyond the common
perceptions of man was naturally admired ... because he was thought wise
and superior to the rest.''

\item purely creative art

``as more arts were invented, and some were directed to the necessities of life, others to recreation, the inventors of the
latter were naturally always regarded as wiser than the inventors of the former''

\end{itemize}

``the point of our present discussion is
this, that all men suppose what is calledWisdom to deal with the first causes
and the principles of things; so that, as has been said before, the man of experience
is thought to be wiser than the possessors of any sense-perception
whatever, the artist wiser than the men of experience, the masterworker than
the mechanic, and the theoretical kinds of knowledge to be more of the nature
of Wisdom than the productive. Clearly then Wisdom is knowledge
about certain principles and causes.''

The references to ``art'' here are not to be taken as contrasting with ``science'', for Aristotle takes ``the mathematical sciences'' to be a prime example of what I have named ``purely creative art'' (for the distinction between science and art he refers us to his Ethics).

\subsection{First Philosophy}

Connected with this ordering (of kinds of knowledge) there are various other ideas, it is highly value laden and bound up with considerations of social status.

As we progress along the scale greater intelligence is involved, and greater good.
The knowledge becomes more ``theoretical'', the posessor more suited for teaching rather than learning from others, and for giving rather than receiving instructions or orders, for being in authority.

An aspect of the characterisation of the scale is the transition from mere acquaintance with a phenomenon, understanding the general rules according to which it operates, and understanding the reasons or causes of that manner of operation.
It is this knowledge of reasons or causes which becomes the dominant characterisation of true wisdom, which is distinguished from lesser forms of theoretical knowledge by the kinds of causes it deals with, the \emph{first} causes.

Aristotle takes ``wisdom'', the subject matter of first philosophy, to be the highest point on this scale, and he takes it to be concerned with certain principles and causes.

Supposing we accept Aristotle's ordering of the kinds of knowledge, and we go with him in taking one end of this spectrum as being the topic of our interest, should we accept his conclusion that this is ``first philosophy'' the study of first causes?
(We should bear in mind that Aristotle's notion of cause is much broader than modern usage of the term.)

\subsection{Being Qua Being and Substance}

Before getting down to serious work in Aristotle further refines his characterisation of ``first philosophy'' by closer examination of the idea of first cause.
By eliminating all that is specific to the sciences associated with particular kinds of object Aristotle arrives at the idea that the study of these first causes is the study of ``being qua being''.
This is by contrast with, say biology as the study of being \emph{as an animal}, physics the study of material beings, and so on.
When all such particular considerations are set aside we are left with the most general science concerned with being in itself, being \emph{qua being}.

This brings us to the study of substance, which is that which underlies the various particular attributes of things.

\subsection{Metaphysics}

Since Aristotle the term ``metaphysics'' has come to be used for that subject matter which he called ``first philosophy''.
It is no longer specifically associated with the notion of ``being qua being'' and ``substance'', but more generally with ontology (what exists), and with all that can be known \emph{a priori} about the nature of the world.
It has been, however, a matter of controversy, whether anything about the world can be known \emph{a priori} (i.e. prior to or without reference to sensory experience).
This has been contested by empiricists and positivist, and forms a principle battleground between empiricists and rationalists and their contemporary progeny.

\pagebreak
\section{Roger Scruton}\label{Scruton}

This section contains notes on Roger Scruton's chapter ``The Nature of Philosophy'' in his {\it Modern Philosophy}\cite{scruton97}.

Note that Scruton is speaking only of philosophy ``as this subject is taught in English-speaking universities''!
Thus only {\it academic} and {\it English}.
Note, that this does not mean English-speaking nations, Universities all over the world teach philosophy in English.
It is nevertheless curious that this should warrant mention, and it is because much of this kind of philosophy is deeply rooted in language.
However, the classic texts are in many languages -- Greek, Latin, and German being very important as well as English.

This is the structure of Scruton's chapter:

\begin{enumerate}
\item What is Philosophy?
\item What is the subject matter of philosophy?
\item Does Philosophy have a distinctive method?
\item The {\it a priori} and the Empirical
\item Branches of Philosophy
\item History of Philosophy and the History of Ideas
\end{enumerate}

This refines a core dichotomy between a discipline defined by its subject matter (as most are, but which is awkward for philosophy because it claims relevance to all problem domains in some way), or by its methods (which is also controversial in philosophy, insofar as no single method commands universal assent, and methods are generally not often clearly articulated).
Mention of the {\it a priori} has a bearing on both these possible methods of characterisation, since many distinguish philosophy from empirical science by its {\it a priori} character.
Talk of the branches of philosophy is a more detailed and piecemeal attempt at definition by subject matter.
Mention of history is important because of the special place which history of philosophy has within philosophy, and because of the difficulty or impossibility of studying the history of philosophy without actually doing philosophy.

In his preliminary remarks Scruton offers an account of the meaning of ``modern'' as intended in the phrase ``modern philosophy'', distinguishing it from certain other uses of the adjective in relation to philosophy.

\begin{quote}
  ``English philosophy is modern in the true sense of the word -- the sense in which science, mathematics and the common law are modern.
  It attempts to build on past results, and, where they are inadequate supersede them.
  Hence English philosophy pays scrupulous attention to arguments, the validity of which it is constantly assessing; it is, like science a collective endeavour, recognising the contribution of many workers in the field; its problems and solutions too are collective, emerging often `by an invisible hand' from the process of debate and scholarship.''
\end{quote}

Contrast that with the following quote from Betrand Russell:

\begin{quote}
  ``The following lectures are an attempt to show, by means of examples, the nature, capacity, and limitations of the logical-analytic method in philosophy. This method, of which the first complete example is to be found in the writings of Frege, has gradually, in the course of actual research, increasingly forced itself upon me as something perfectly definite, capable of embodiment in maxims, and adequate, in all branches of philosophy, to yield whatever objective scientific knowledge it is possible to obtain. Most of the methods hitherto practised have professed to lead to more ambitious results than any that logical analysis can claim to reach, but unfortunately these results have always been such as many competent philosophers considered inadmissible. Regarded merely as hypotheses and as aids to imagination, the great systems of the past serve a very useful purpose, and are abundantly worthy of study. But something different is required if philosophy is to become a science, and to aim at results independent of the tastes and temperament of the philosopher who advocates them. In what follows, I have endeavoured to show, however imperfectly, the way by which I believe that this desideratum is to be found.
''
\end{quote}

and:

\begin{quote}
  ``Philosophy, from the earliest times, has made greater claims, and achieved fewer results, than any other branch of learning. Ever since Thales said that all is water, philosophers have been ready with glib assertions about the sum-total of things; and equally glib denials have come from other philosophers ever since Thales was contradicted by Anaximander. I believe that the time has now arrived when this unsatisfactory state of things can be brought to an end. In the following course of lectures I shall try, chiefly by taking certain special problems as examples, to indicate wherein the claims of philosophers have been excessive, and why their achievements have not been greater. The problems and the method of philosophy have, I believe, been misconceived by all schools, many of its traditional problems being insoluble with our means of knowledge, while other more neglected but not less important problems can, by a more patient and more adequate method, be solved with all the precision and certainty to which the most advanced sciences have attained.''
\end{quote}

Both from {\it Our Knowledge of the External World as a Field for Scientific Method in Philosophy} \cite{russell21}.

\subsection{What is Philosophy?}

Scruton makes the observation that the history of philosophy is a prolonged search for its own definition.

In this section he delimits philosophy by the {\it character} of the problems it studies.
The characteristics he identiifies are:

\begin{itemize}

\item[(a)] abstraction
  
\item[(b)] ultimacy
  
\item[(c)] the interest in truth
  
\end{itemize}

\subsection{What is the Subject-Matter of Philosophy?}

\begin{itemize}

\item[(a)] another realm of being
  
\item[(b)] {\it anything}
  
\item[(c)] {\it everything} (a theory of {\it the whole} of things)
  
\end{itemize}

\subsection{Does Philosophy Have a Distinctive Method?}

\begin{itemize}

\item[(a)] Thomism
  
\item[(b)] {\it Linguistic or `conceptual' analysis}
  
\item[(c)] {\it Critical philosophy}
  
\item[(d)] {\it phenomenology}

\end{itemize}

No mention here of {\it logic}, {\it logical construction} or {\it logical analysis}, which are the kinds of method which Russell or Carnap would stipulate.

\subsection{The {\it a priori} and the {\it Empirical}}

This is a kind of characterisation {\it by method}, though the method is not very definite.
It is {\it by contrast} with the methods of ``science'', by which he means of course {\it empirical science}, and he seems to acquesce in the idea that philosophy is therefore an {\it a priori} discipline without quite wanting to go so far as calling it an {\it a priori science} (which is what Russell and Carnap held).
This I think is because he does not want to endorse anything more than that philosophy is non-empirical, Scruton does not want to stipulate any more definite method (there is no concensus among philosophers on any such method).

However, this does not bear close scrutony.
Its certainly true that philosophers do seem to think of themselves as engaged in an {\it a priori} pursuit, but their results are not confined to those which plausibly can be established {\it a priori} and they do make use of much empirical knowledge, without actually conducting the kind of carefully documented observations or experiments which are considered essential in empirical science.
The empirical knowledge used by philosophers in their arguments is likely to be common sense knowledge, or the kind of knowledge of our language which might be assumed familiar us all (if perhaps better understood by philsophers.

\subsection{Branches of Philosophy}

Classication by enumeration of subject matters?

\begin{itemize}
\item[(a)] Pure philosophy
\begin{itemize}

\item[(i)] Logic

\item[(ii)] Epistemology
  
\item[(iii)] Metaphysics
  
\item[(iv)] Ethics and Aesthetics

\end{itemize}

\item[(b)] Applied philosophy
\begin{itemize}

\item[(i)] of Religion
  
\item[(ii)] of Science
  
\item[(iii)] of Ethics
  
\item[(iv)] of Politics
  
\item[(v)] Applied Ethics

\end{itemize}

\end{itemize}

\section{Can Philosophy Provide Answers?}\label{Answers}

Is philosophy just:
\begin{itemize}
\item a matter of opinion or culturally relative, or
\item do some or all philosophical questions have definite answers which we can hope to discover and establish?
\end{itemize}

Connected with that dichotomy are the following two caricatures of what a philosopher might be:
\begin{itemize}
\item a sage, who we may consult and whose word we should accept because he is venerable and wise, perhaps even oracular.
\item someone who has mastered scientifically rigorous methods for reasoning soundly about any subject matter and who uses these methods to extend our knowledge into new problem domains.
\end{itemize}

Of course, scepticisn, not just in relation to philosophy but also more broadly based, has a long history and will outlive us all.
It is, I suggest, rational to doubt that {\it absolute} certainty in any matter is attainable.

The word `Magi' was originally applied exclusively to members of a priestly caste of the Medes and Persian [sic] who had esoteric skills in interpreting dreams. However, the use of the word broadened to embrace various categories of persons who were marked out by their superior knowledge and ability, including astrologers, soothsayers, and even oriental sages.

Before the ancient Greeks the closest we had to philosophers were religious leaders and these magi or sages.
The ``Weatern'' philosophical tradition begins with the ancient Greek ``philosophers'', lovers of knowledge whose authority was secular and rested on ``reason'' (though it would be a while before that idea was clarified and articulated, in Aristotle).
Of course the distinction between science and philosophy did not exist yet.

Quine, in his {\it Homage to Carnap}:
\begin{quote}
  ``We beamed with partisan pride when he countered a diatribe of Arthur Lovejoy's in his characteristically reasonable way, explaining that if Lovejoy means A then {\it p}, and if he means B then {\it q}.
  I had let to learn how unsatisfying this way of Carnap's could sometimes be.''
\end{quote}



%\backmatter

\appendix

\addcontentsline{toc}{section}{Bibliography}
\bibliographystyle{alpha}
\bibliography{rbj2}

%\addcontentsline{toc}{section}{Index}\label{index}
%{\twocolumn[]
%{\small\printindex}}

%\vfill

%\tiny{
%Started 2012-10-19

%Last Change $ $Date: 2014/11/08 19:43:29 $ $

%\href{http://www.rbjones.com/rbjpub/www/papers/p019.pdf}{http://www.rbjones.com/rbjpub/www/papers/p019.pdf}

%Draft $ $Id: p019.tex,v 1.5 2014/11/08 19:43:29 rbj Exp $ $
%}%tiny

\end{document}

% LocalWords:
b
