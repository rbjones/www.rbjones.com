% $Id: p019.tex,v 1.2 2012/11/24 20:22:24 rbj Exp $
% bibref{rbjp019} pdfname{p019}

\documentclass[10pt,titlepage]{article}
\usepackage{makeidx}
\usepackage{graphicx}
\usepackage[unicode,pdftex]{hyperref}
\pagestyle{plain}
\usepackage[paperwidth=5.5in,paperheight=8in,hmargin={0in,0in},vmargin={0in,0in},includehead,includefoot]{geometry}
\hypersetup{pdfauthor={Roger Bishop Jones}}
\hypersetup{colorlinks=true, urlcolor=red, citecolor=blue, filecolor=blue, linkcolor=blue}
\usepackage{html}
\usepackage{paralist}
\usepackage{relsize}
\usepackage{verbatim}
\makeindex
\newcommand{\ignore}[1]{}

\title{What is Philosophy?}
\author{Roger~Bishop~Jones}
\date{\ }

\begin{document}
%\frontmatter
                               
\begin{titlepage}
\maketitle

%\begin{abstract}
%Some thoughts about the nature of philosophy and the kinds of philosophy.
%\end{abstract}

%\vfill

%\begin{centering}

%{\footnotesize
%copyright\ Roger~Bishop~Jones;
%}%footnotesize

%\end{centering}

\end{titlepage}

\setcounter{tocdepth}{2}
{\parskip-0pt\tableofcontents}

%\listoffigures

%\mainmatter

\section{Introduction}

A first answer to the question ``what is philosophy?'' might be ``all things to all men''.

There are many different kinds of philosophy and no single answer will address them all, so we are immediately drawn into the question ``what kinds of philosophy are there?''.

This question too has many different answers, so I will sketch seversl.
A related question which may also help to position philosophy is ``what kinds of knowledge are there?''.
Some of the answers here provide a classification both of knowledge in general and of kinds of philosophy.

\section{Ways of Cutting The Cake}

\subsection{A first cut}

My first inclination was to regard philosophy as of three kinds:

\begin{enumerate}
\item Commonsense philosophy
\item Theological philosophy
\item ``Western'' philosophy
\end{enumerate}

Commonsense philosophy is what ordinary people might be said to be engaged in when they grapple, without the benefit of any kind of special philosophical training, in a relatively abstract way, with certain of the problems which concern them.
Many of the discussions of our group naturally fall into this category, such as ``Love''.

The other two general classifications relate to kinds of philosophy which have been to some degree professionalised.
In libraries and bookshops ``Philosophy'' is commonly juxtaposed with ``Religion'', and for the larger part of the history of mankind philosophical problems have been discussed primarily in the context of religious belief systems, in which case it may be more a case of receiving instruction from figures of authority than engaging in discussion or debate.

``Western Philosophy'' is a generic term for the philosophising which has taken place in the context of an intellectual tradition which begins in ancient Greece about 2600 years ago.
This tradition is distinctive because of its departure from the handing down of religious or philosophical dogma by figures of authority, in favour of a search for understanding through reason in which individuals make their own judgements rather than being expected to believe what they are told. 

A major figure in the beginnings of Western Philosophy is Aristotle, and it is useful to consider some of his ideas on what philosophy is, in particular in relation to a special kind of philosophy which he called ``first philosophy''.

Not far removed from this analysis of the kinds of philosophy is one given much later by the French philosopher Auguste Comte.

\subsection{Comte's Three States as Kinds of Philosophy}

Comte identified three ``states'' which are three stages in the development of the human mind and correspond to different kinds of philosophy.

They are:

\begin{center}
\begin{description}
\item[theological] \ 

    In which the hidden nature of things is sought by constructing divine beings in man's own image.

\item[metaphysical] \ 

    In which the hidden nature of things is sought without resort to supernatural beings.

\item[positive] \ 

    In which questions about the hidden nature of things are rejected as pathological in some way (e.g. meaningless, or verbal).
\end{description}
\end{center}

    In the positive state science seeks to discover universal laws governing phenomena by observation and experiment. The scientist does not seek beyond these laws governing phenomena for deeper explanations, especially not explanations which involve entities which are not directly observable.

The first two pf Comte's stages correspond to the division between religious philosophy and ``western philosophy'' as initiated in ancient Greece.
The third state comes later, and may be associate with the transition from the metaphysics and science of ancient Greece to modern science and positivistic philosophy.
Prominent figures in this latter transition were Roger Bacon, Gallileo, and David Hume.

In neither of these transitions does the later ``stage'' replace the earlier.
Religious belief systems coexist with scientific, sometimes in harmony sometimes in conflict.
In philosophy there is a long running conflict between metaphysical philosophers and positivistic philosophers, which is related to that between rationalists and empiricists.

\subsection{Aristotelian Divisions}

Its worth bearing in mind in this section and the next, that in Aristotle the divisions between art, science and philosophy are not as sharply drawn as they have become for us.
Aristotle was very interested in defining and classifying.
His terms and classifications were use in education, science and philosophy for thousands of years and are embedded in our language.

One important division within western philosophy is that between theoretical and practical philosophy, and this has its roots in Aristotle.

For Aristotle ``science'' was divided into three parts:

\begin{itemize}

\item theoretical sciences - concerning knowledge

\item practical sciences - concerning conduct

\item productive sciences - concerning the making of useful or beautiful objects

\end{itemize}

The division between practical sciences and theoretical sciences correponds to a division still relevant to contemporary philosophy.
The practical sciences are those involving values and conduct, theoretical sciences to knowledge.

Thus the theoretical sciences were mathematics, physics and theology or metaphysics.

Logic does not fall under any of these categories for Aristotle, being concerned with methods common to and prerequisite for all these kinds of science.
The six volumes which Aristotle wrote on Logic were later called ``The Organon'' (the Greek word for ``tool'') to reflect this idea that the ideas and techniques were to be used in the conduct of science rather than themselves constituting science.

This distinction between theoretical and practical is now used specifically for philosophy, and philosophical logic is considered a part of theoretical philosophy (mathematical logic is a part of mathematics).

Thus practical philosophy, concerned with values and conduct, includes the following subdisciplines of philosophy:

\begin{itemize}
\item Aesthetics
\item Ethics
\item Political Philosophy
\item ...
\end{itemize}

but also includes the doctrines of those schools of thought whose primary concerns were with the conduct of life, such as scepticism, stoicism, epicureanism (in ancient Greece) and more modern schools such as existentialism.

Theoretical philosophy includes of:

\begin{itemize}
\item Metaphysics
\item Logic
\item Epistemology
\item Philosophy of Language
\item Philosophy of Science
\item Philosophy of Mind
\end{itemize}

and many other specialities.

The distinction being drawn primarily on the basis of whether the enterprise is purely descriptive or involves values.

\ignore{
\subsection{Rationalism and Empiricism}

In the early modern period of philosophy certain philosophers have been 

\subsection{Hume's Forks}

There is 
}%ignore

\subsection{A 20th Century Divide - Analytic v. Continental}

The twentieth century was called ``the age of analysis'', the idea that philosophy should consist of logical or linguistic analysis having been introduced by Russell and Moore around the turn of the twentieth century.
This proved highly influential, and this conception of philosophy (which is entirely theoretical) dominated academic philosophy for most of the twentieth century.
However, in continental Europe, particularly in France, a different approach to philosophy remained important, and was called by Anglo-Saxon's ``continental philosophy''.
The best known school of continental philosophy is Existentialism, and its most famous 20th century exponent Jean Paul Sarte a philosopher/novelist.
Existentialism is substantially practical rather than theoretical.

Possibly this divide may be seen as following from an early divide between a predominantly continental ``rationalist'' tendency in which metaphysics plays a prominent role, and a British empiricist tradition in which knowledge is seen as firmly rooted in empirical evidence and metaphysics is repudiated.
However, there are many counterexamples to these generalisations, for example, positivism, a variety of empiricism is as much (if not more) a continental phenomenon as British.

\subsection{Hume's Forks}

David Hume noticed two cleavages in knowledge.
One between descriptive and evaluative claims, which he separated with the dictum that one cannot derive an ``ought'' from an ``is''.
The other a fundamental divide within descriptive knowledge which he described as concerning either ``relations between ideas'' or ``matters of fact''.
These correspond to logical or analytic truths on the one hand, and synthetic, empirical, contingent truths on the other.

In treating both of these kinds of knowledge as important Hume provides a kind of synthesis between the rationalism and empiricism, but a synthesis which leaves no place for metaphysics, and so remains within the empiricist tradition.
This is regarded by many as the beginning of positivism (though before that term is coined) and this distinction plays a key role in the subsequent history of positivism.

Though Hume banished metaphysics as this is often understood, in his own language he uses the term metaphysics more as a synonym for philosophy and considers himself to be offering a new kind of metaphysics.
The place for this in Hume's classification is in the ``matters of fact'', since he advocates that philosophy should be conducted following newtonian scientific method as an empirical enquiry into human understanding.
Later in the twentieth century Rudolf Carnap gives this same classification of knowledge a fundamental place in his own positivistic philosophy (\emph{Logical} positivism), but now within the new conception of philosophy as analysis considers philosophy to belong to the ``relations between ideas'' category, the analytic propositions (though substantially concerned with methods rather than doctrine).

\section{Aristotle on First Philosophy}

The topic ``Metaphysics'' was first used after Aristotle to name a book he wrote on ``first philosophy''.
In examining Aristotle's conception of ``first philosophy'' and the modern conception of ``metaphysics'' it is useful to consider four stages, to consider each of these stages on their own merits and to consider the relationship between the stages.

The stages are:
\begin{enumerate}
\item Aristotle's examination of which kinds of knowledge should be considered ``true wisdom''.
\item The conception of ``first philosophy'' which flows from it.
\item Aristotle's taking ``substance'' as the subject matter of ``first philosophy''.
\item Modern conceptions of ``metaphysics''.
\end{enumerate}

\subsection{True Wisdom}

The following items exemplify an apparently linear ordering on kinds of knowledge which is used by Aristotle to identify ``Wisdom''.

\begin{itemize}
\item sensation

\item memory

Animals with memory are ``more intelligent and apt at learning'' than those without.
Animals with hearing can also be taught.

\item experience

The several memories of the same thing produce finally the capacity for a single experience.
Science and art come to men through experience.

\item skilled art

``Art arises when from many notions gained by experience one universal judgement about a class of objects is produced.''

``To have a judgement that when Callias was ill of this
disease this did him good, and similarly in the case of Socrates and in many
individual cases, is a matter of experience; but to judge that it has done good
to all persons of a certain constitution, marked off in one class, when they
were ill of this disease, e.g. to phlegmatic or bilious people when burning
with fevers - this is a matter of art.''

``experience is knowledge of individuals, art of universals''

``knowledge and understanding belong to art rather than to experience''

``masterworkers in each craft ... are wiser than the manual workers, because they know the causes''

``artists can teach, and men of mere experience cannot''

\item usefully creative art

``At first he who invented any art whatever that went beyond the common
perceptions of man was naturally admired ... because he was thought wise
and superior to the rest.''

\item purely creative art

``as more arts were invented, and some were directed to the necessities of life, others to recreation, the inventors of the
latter were naturally always regarded as wiser than the inventors of the former''

\end{itemize}

``the point of our present discussion is
this, that all men suppose what is calledWisdom to deal with the first causes
and the principles of things; so that, as has been said before, the man of experience
is thought to be wiser than the possessors of any sense-perception
whatever, the artist wiser than the men of experience, the masterworker than
the mechanic, and the theoretical kinds of knowledge to be more of the nature
of Wisdom than the productive. Clearly then Wisdom is knowledge
about certain principles and causes.''

The references to ``art'' here are not to be taken as contrasting with ``science'', for Aristotle takes ``the mathematical sciences'' to be a prime example of what I have named ``purely creative art'' (for the distinction between science and art he refers us to his Ethics).

\subsection{First Philosophy}

Connected with this ordering (of kinds of knowledge) there are various other ideas, it is highly value laden and bound up with considerations of social status.

As we progress along the scale greater intelligence is involved, and greater good.
The knowledge becomes more ``theoretical'', the posessor more suited for teaching rather than learning from others, and for giving rather than receiving instructions or orders, for being in authority.

An aspect of the characterisation of the scale is the transition from mere acquaintance with a phenomenon, understanding the general rules according to which it operates, and understanding the reasons or causes of that manner of operation.
It is this knowledge of reasons or causes which becomes the dominant characterisation of true wisdom, which is distinguished from lesser forms of theoretical knowledge by the kinds of causes it deals with, the \emph{first} causes.

Aristotle takes ``wisdom'', the subject matter of first philosophy, to be the highest point on this scale, and he takes it to be concerned with certain principles and causes.

Supposing we accept Aristotle's ordering of the kinds of knowledge, and we go with him in taking one end of this spectrum as being the topic of our interest, should we accept his conclusion that this is ``first philosophy'' the study of first causes?
(We should bear in mind that Aristotle's notion of cause is much broader than modern usage of the term.)

\subsection{Being Qua Being and Substance}

Before getting down to serious work in Aristotle further refines his characterisation of ``first philosophy'' by closer examination of the idea of first cause.
By eliminating all that is specific to the sciences associated with particular kinds of object Aristotle arrives at the idea that the study of these first causes is the study of ``being qua being''.
This is by contrast with, say biology as the study of being \emph{as an animal}, physics the study of material beings, and so on.
When all such particular considerations are set aside we are left with the most general science concerned with being in itself, being \emph{qua being}.

This brings us to the study of substance, which is that which underlies the various particular attributes of things.

\subsection{Metaphysics}

Since Aristotle the term ``metaphysics'' has come to be used for that subject matter which he called ``first philosophy''.
It is no longer specifically associated with the notion of ``being qua being'' and ``substance'', but more generally with ontology (what exists), and with all that can be known \emph{a priori} about the nature of the world.
It has been, however, a matter of controversy, whether anything about the world can be known \emph{a priori} (i.e. prior to or without reference to sensory experience).
This has been contested by empiricists and positivist, and forms a principle battleground between empiricists and rationalists and their contemporary progeny.

%\backmatter

%\appendix

%\addcontentsline{toc}{section}{Bibliography}
%\bibliographystyle{alpha}
%\bibliography{rbj}

%\addcontentsline{toc}{section}{Index}\label{index}
%{\twocolumn[]
%{\small\printindex}}

%\vfill

%\tiny{
%Started 2012-10-19

%Last Change $ $Date: 2012/11/24 20:22:24 $ $

%\href{http://www.rbjones.com/rbjpub/www/papers/p019.pdf}{http://www.rbjones.com/rbjpub/www/papers/p019.pdf}

%Draft $ $Id: p019.tex,v 1.2 2012/11/24 20:22:24 rbj Exp $ $
%}%tiny

\end{document}

% LocalWords:
