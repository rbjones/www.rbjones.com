% $Id: p027.tex $
% bibref{rbjp027} pdfname{p027}

\documentclass[14pt,titlepage]{extarticle}
\usepackage{makeidx}
\usepackage{graphicx}
\usepackage[unicode,pdftex]{hyperref}
\pagestyle{plain}
\usepackage[paperwidth=8.3in,paperheight=11.7in,hmargin={0.3in,0.3in},vmargin={0.5in,0.5in},includehead,includefoot]{geometry}
\hypersetup{pdfauthor={Roger Bishop Jones}}
\hypersetup{pdftitle={Scepticism and Positivism}}
\hypersetup{colorlinks=true, urlcolor=red, citecolor=blue, filecolor=blue, linkcolor=blue}
%\usepackage{html}
\usepackage{paralist}
\usepackage{relsize}
\usepackage{verbatim}
\usepackage{enumerate}
\makeindex
\newcommand{\ignore}[1]{}

\title{Scepticism and Positivism}
\author{Roger~Bishop~Jones}
\date{\ }


\begin{document}
%\frontmatter

%\begin{abstract}
%Notes for a philosophical discussion on scepticism and positivism
%\end{abstract}
                               
\begin{titlepage}
\maketitle

%\vfill

%\begin{centering}

%{\footnotesize
%copyright\ Roger~Bishop~Jones;
%}%footnotesize

%\end{centering}

\end{titlepage}

\begin{centering}
{\LARGE \bf Scepticism and Positivism}
\end{centering}

\setcounter{tocdepth}{1}
{\parskip-0pt\tableofcontents}

%\listoffigures

%\mainmatter

%\pagebreak

\section{Introduction}

Scruton talks about scepticism in chapter two of his book ``Modern Philosophy - an introduction and survey'' \cite{scruton97}, and follows in the next chapter with material about a number of other ``-isms'' many of which are connected with scepticism (concessions or responses to it), thus giving an idea of the influence of sceptical arguments beyond the ranks of radical sceptics.

\section{Levels of Scepticism}

There are many kinds and degrees of scepticism.
For today's purposes the following classification may be helpful:

\begin{itemize}
\item Common-sense Scepticism

  This is what we use when we ignore a transparent scam in our mailbox, or when we doubt the word of a politician.

\item Hard-core Scepticism

  This is the scepticism of Plato's Academy (after Plato) and of the followers of Pyrrho of Ellis.
  It says that NO knowledge is possible (Academic Scepticism) or that even to hold an opinion counts as dogmatism and that we should strive to suspend judgement on all matters.
  This is the kind of thing that present day academics usually mean when they talk of ``scepticism'' (or ``skepticism'').

\item Selective Scepticisms

  Sometimes philosophers are sceptical about a whole class of propositions, but not about all knowledge or belief.
  Often we have special names for these kinds of scepticism, e.g. Nomimalism (scepticism about universals or abstract entities), Idealism (scepticism about anything but ideas)

\item Graduated Scepticisms

  Hard-core and selective scepticisms are rather black and white, large classifications of claims are either admotted of unequivocally rejected (as possible knowledge or reasonable beliefs).
  Graduated scepticisms are more subtle, they may allow an intermediate status for some claim, or more subtle criteria separating acceptable from unacceptable claims in some particular area.
  Positivism may be though of in this way.
  Broad swathes of philosophy are condemned by positivism but not all of it, science is the model of propositions which are OK, but only sciencen done proper, which is called {\it positive} science.
\end{itemize}

In addition to these levels of scepticism, it may be noted that much academic interest in scepticism might best be described as {\it analytic} scepticism. which on its face is not sceptical at all, but is just the study of scepticism and the analysis of the strength of the various sceptical stances and the grounds for adopting or rejecting them.
However, in a way, this is in itself a form of scepticism, leaning towards pyrrhonism, since this kind of study of scepticism may be entirely non-commital, the academics involved may in the end neither endorse nor reject any particular kind of scepticism, they just {\it analyse} but suspend judgement.

\ignore{
I am proposing that we should discuss scepticism and also touch upon positivism (a moderated or constructive scepticism which Scruton does not talk about but which connects with several of the -isms he does mention).

I propose that we start out discussing the kinds of scepticism appropriate in everyday life, and the disadvantages and dangers of being gullible or credulous, and then move on to the character and significance of scepticism through the history of philosophy, leading into the connection with positivism.
}%ignore

\section{The Discussion}

I propose that we break the discussion into two parts:
\begin{enumerate}
\item Common-sense scepticism
\item Other scepticisms
\end{enumerate}

Not much introduction is needed for the first part, just a few questions to consider.
I propose to introduce the second part with a very brief history of scepticism and positivism.

\section{Common-sense Scepticism}

In living our lives its not a good idea to believe everything we are told.
Often information come from sources which have an interest in our believing some proposition whether or not it is actually true, because of some likely influence that belief may have on our actions, for example the purchase of some product, or the way we vote in an election.

Apart from this kind of common-sense scepticism, it may be noted that an element of scepticism is built into our legal system, which recognises that prosecuting authorities should not be trusted but should rather be required to persuade a jury of the truth of their accusations, ``beyond reasonable doubt''.

In our lifetimes there has been considerable change in the attitude of ordinary people to what they are told by figures of ``authority'' who are purported to be reliable sources of information or advice.

\pagebreak

Here are some questions to consider:

\begin{itemize}

\item Would you rather be considered ``sceptical'' or ``gullible''?

\item What are the advantages and disadvantages of being sceptical (on the one hand) or gullible (on the other)?

\item Do you believe everything you are told, or are you sometimes sceptical about what people say?

\item How do you decide what to believe, what to disbelieve, or when to suspend judgement?

\item Do you believe claims coming from:


\begin{itemize}
\item Teachers/academics
\item     Doctors
\item     Police/courts
\item     Politicians
\item     Scientists
\item     Pressure groups
\item     Consumer organisations (e.g. which)
\item     Books
\item     Papers
\item     The Web
\item     Social Media
\item     Philosophers
\item     Commercial advertisments
\item     Cold callers
\end{itemize}

add your own...

\item Where the answer is ``sometimes'' how do you tell what to believe?

\item Has your trust in these ``authorities'' grown or declined over the years?
\end{itemize}

\pagebreak

\section{Some Notes on the History of Scepticism}

\nocite{sextusempiricus33}
\nocite{popkin03}

Stages:

\begin{enumerate}
\item Precursors
  \begin{itemize}
  \item Pre-socratics
    \begin{itemize}
    \item The contrast between the success of reason in mathematics and in ``philosophy'':
      \begin{itemize}
      \item mathematics reliable, enduring, cumulative
      \item philosophy ephemeral, dubitable, contradictory
      \end{itemize}
    \item Illustrated by the seminal contradiction between:
      \begin{itemize}
      \item Heraclitus (c 535-475 bc) - everything is in constant flux
      \item Parmenides (c 500-450 bc) - nothing changes
      \end{itemize}
    \end{itemize}

  \item Socrates and the Sophists, oppose moderately sceptical relativism by conceptual analysis
  \item Plato and Aristotle - the great classical system builders
    \begin{itemize}
    \item Plato reconciles Heraclitus and Parmenides using his two worlds:
      \begin{itemize}
      \item the perfect and eternal world of ideals, of which we have reliable and durable knowledge through reason
      \item the elusive and shifting world of appearances of which we form opinions through our senses but of which we can have no true knowledge
      \end{itemize}
    \item Aristotle begins the study of logic and puts forward his conception of ``demonstrative science''
    \end{itemize}
  \end{itemize}
\item Academic and Pyrrhonean Scepticism (hard core scepticism)

  After Plato and Aristotle Greek philosophy becomes predominantly ``practical'' rather than theoretical, i.e. concerned primarily with ethics, politics, how to live a life.
  Strangely enough this includes the two major schools of radical scepticism which span a period of 300+ years between 300 bc and 200 ad.
  These two are the Academic Sceptics (in Plato's academy), and the Phyrrhoneans, originating with Phyrro of Ellis.
  The ``practical'' motivation in Phyrrhonean philosophy is the search for tranquility which is held to result from the suspension of judgement, in turn a consequences of arguments which establish ``equipollence'' of the two alternatives for each proposition that it is true or false.
  
\item Later Influence
  \begin{itemize}
  \item Saint Augustine (354-430)

    Argued against academic scepticism
  \item Savonarola (1452-1498), Luther

    The work of Sextus Empiricus was ``rediscovered'' in the $15^{th}$ century and the problem of justifying a criterion for true and certain religious knowledge featured in Savonarola's dispute with the authority of the pop in 1497/8 (culminating in his execution) and in Luther's break with Rome.

    Savonarola's crusade involved abolishing philosophy through skepticism so that people could turn to revelation instead. 
    
  \item Moderated and constructive scepticisms
  \item Rationalist and empiricist philosophy
    
    Descartes, Spinoza and Leibniz.

    Locke, Berkeley and Hume.
  \end{itemize}
\item Positivism \cite{kolakowski66}
  \begin{itemize}
  \item Hume
  \item Carnap and Popper
  \item The rout of positivism
  \end{itemize}
\end{enumerate}

\section{Further Reading}

There are several online articles on various aspects of scepticism at the Stanford Encyclopaedia of Philosophy.

%\subsection{Applications}

%\paragraph{}

%\subsection{Theory}



%\backmatter

%\appendix

\addcontentsline{toc}{section}{Bibliography}
\bibliographystyle{alpha}
\bibliography{rbj2}

%\addcontentsline{toc}{section}{Index}\label{index}
%{\twocolumn[]
%{\small\printindex}}

%\vfill

%\tiny{
%Started 2012-10-19

%Last Change $ $Date: 2014/11/08 19:43:30 $ $

%\href{http://www.rbjones.com/rbjpub/www/papers/p019.pdf}{http://www.rbjones.com/rbjpub/www/papers/p019.pdf}

%Draft $ $Id: p022.tex,v 1.1 2014/11/08 19:43:30 rbj Exp $ $
%}%tiny

\end{document}

% LocalWords:
