% $Id: p027.tex $
% bibref{rbjp027} pdfname{p027}

\documentclass[14pt,titlepage]{extarticle}
\usepackage{makeidx}
\usepackage{graphicx}
\usepackage[unicode,pdftex]{hyperref}
\pagestyle{plain}
\usepackage[paperwidth=8.3in,paperheight=11.7in,hmargin={0.3in,0.3in},vmargin={0.5in,0.5in},includehead,includefoot]{geometry}
\hypersetup{pdfauthor={Roger Bishop Jones}}
\hypersetup{pdftitle={Rationality}}
\hypersetup{colorlinks=true, urlcolor=red, citecolor=blue, filecolor=blue, linkcolor=blue}
%\usepackage{html}
\usepackage{paralist}
\usepackage{relsize}
\usepackage{verbatim}
\usepackage{enumerate}
\makeindex
\newcommand{\ignore}[1]{}

\title{Scepticism and Positivism}
\author{Roger~Bishop~Jones}
\date{\ }


\begin{document}
%\frontmatter

%\begin{abstract}
%Notes for a philosophical discussion on scepticism and positivism
%\end{abstract}
                               
\begin{titlepage}
\maketitle

%\vfill

%\begin{centering}

%{\footnotesize
%copyright\ Roger~Bishop~Jones;
%}%footnotesize

%\end{centering}

\end{titlepage}

\begin{centering}
{\LARGE \bf Scepticism and Positivism}
\end{centering}

\setcounter{tocdepth}{1}
{\parskip-0pt\tableofcontents}

%\listoffigures

%\mainmatter

%\pagebreak

\section{Introduction}

Scruton talks about scepticism in chapter two of his book ``Modern Philosophy'' \cite{scruton}, and follows in the next chapter with material about a number of other ``-isms'' many of which are connected with scepticism (concessions or responses to it).

I am proposing that we should discuss scepticism and also touch upon positivism (a constructive scepticism which Scruton does not talk about but which connects with several of the isms he does mention).

Though Scruton's lecture is worth a read, I feel that it doesn't provide a good basis for the kind of discussion I would like us to have, so I will be leading us in a slightly different direction, which I think will help to make the topic more accessible.

I propose that we begin by discussing the notion of scepticism as it is used outside of academic philosophy, and how it bears upon our daily lives (which I think it does).


\pagebreak

\section{Outline Plan}

\begin{enumerate}
\item Preliminaries
\begin{enumerate}
\item Suggested discussion plan.
\item How are the words {\it rational} and {\it irrational} generally used (if at all)?
\item The distinction between {\it unreasonable} and {\it irrational} in ordinary language.
\item The wide variety of special usages in different academic disciplines, e.g. economics, sociology, law.
\item In relation to its use in philosophy:
\begin{enumerate}
\item The relationship between {\it rationality} and {\it reason}.
\item The difference between {\it reason} and {\it rhetoric}.
\item {\it Theoretical} and {\it Practical} Rationality (definitions?).
\end{enumerate}
\end{enumerate}
\item Some reasons why the concept is important:
\begin{enumerate}
\item Reason as an alternative to violence (connection with fundamentalist terrorism)
\item The role of reason in the genesis and supposed character of Western Philosophy
\item Reason as the foundation for science, technology and of our material prosperity.
\end{enumerate}
\item Some applications for consideration:
\begin{enumerate}
\item Refugees, Asylum, Terrorism
\item NICE, health care ethics and politics
\item EU membership ``debate'' and politics generally
\item the rationality of Philosophy (and academia in general) - ``academic freedom''.
\end{enumerate}
\item Some general (theoretical?) Questions:
\begin{enumerate}
\item rationality and morality (is rationality good?)
\item rationality and emotion (enlightenment and romanticism)
\end{enumerate}
\end{enumerate}

\pagebreak

\section{Some Notes}

\subsection{Preliminaries}

\begin{itemize}
\item[(b)] Not used a lot?
The words ``logical'', ``illogical'' has similar meaning, perhaps used more.
Irrational is a strongly pejorative term.
Irrational beliefs those which fly in the face of (are contradicated by) evidence known to the believer.
Irrational actions are those which the agent knows to be incapable of securing his purposes. 

\item[(c)] Unreasonable means something different, connotes an unwillingness to compromise.
Unreasonable behaviour is more common than irrational behaviour.

\item[(e)]

\begin{itemize}

\item [(i)]
In philosophy the relationship between {\it rationality} and {\it reason} is typically much closer,
so that rational beliefs and actions are those which conform to reason.

\item [(ii)]


Rhetoric seeks to persuade by means of appeals to emotions, prejudices or other non-rational means.
Reason, seeks objective conclusions which are based on evidence and are not influenced by other matters.

\item [(iii)]

Thus philosophy may be concerned (since Aristotle) with theoretical and practical science, and
we see in Kant ideas about the scope and limits of ``pure reason'' (``Critique of Pure Reason'')
considered under the headings of ``Theoretical Rationality'' and``Practical Rationality''.

Theoretical rationality concerns our beliefs, practical rationality concerns actions.

Our beliefs are rational if they are consistent (logically?) with the evidence upon which they are
based.
Our actions are rational if, in the context of our beliefs and values, they can reasomably be expected 
to contribute the the realisation of the ends.
To consider the rationaity of actions, it helps to understand what they are intended to achieve.
The ends themselves are not in themselves the subject of the judgement, rather whether the
means can reasonably be expected to realise those ends.

\end{itemize}

\end{itemize}

\subsection{Motivators}

Why is {\it rationality} an interesting and important concept to discuss philosophically?

\begin{itemize}

\item[(a)]


Reason provides a civilised alternative
to physical violence in resolving conflicts of interest between individuals and groups.
When such reasonable methods of conflict resolution are systemised into large scale
social institutions, they facilitate a society which is fair and just in which individuals
can flourish.
The antithesis of reason, the resort to violence, is a prominent feature of terrorist
organisations intent on imposing extreme fundamentalist religious ideologies.
It seems important to keep a firm grip on these values in responding to the threats
we face, and in doing so, we should not be blind to the imperfections of our own
society.

In these times, when our fundamental beliefs are under attack, it is important to
be clear about what they are, and to ensure that they are not obscured or buried
by multi-culturalism.
Freedom of thought and speech are among the most important of these fundamental
values.

\item[(b)]

The origins of the ``Western'' tradition in philosophical thinking are usually traced to
the great flourishing of creative thought and culture located around Greece from about
600 BC.
A hallmark of this thinking is its relative freedom from contraint by authority, or established
ideology (and the relatively large class of citizens free to spend their time on such things).

These early philosophers sought truth by observing and reasoning about the world around them.
They were also influenced by the effectiveness of deductive methods in establishing mathematics
as a theoretical discipline, and sought to apply these methods more widely.

The distinctive feature of the resulting philosophical tradition which had dominated philosophy
in the Western Hemisphere (and now globally), it that it aims to be {\it rational}.

\item[(c)]

Arguably, it is our ability to reason about our environment which is the principle cause
of success as a species, insofar as science and technology result from observing closely
and reasoning about the world around us.

\end{itemize}



%\subsection{Applications}

%\paragraph{}

%\subsection{Theory}



%\backmatter

%\appendix

%\addcontentsline{toc}{section}{Bibliography}
%\bibliographystyle{alpha}
%\bibliography{rbj}

%\addcontentsline{toc}{section}{Index}\label{index}
%{\twocolumn[]
%{\small\printindex}}

%\vfill

%\tiny{
%Started 2012-10-19

%Last Change $ $Date: 2014/11/08 19:43:30 $ $

%\href{http://www.rbjones.com/rbjpub/www/papers/p019.pdf}{http://www.rbjones.com/rbjpub/www/papers/p019.pdf}

%Draft $ $Id: p022.tex,v 1.1 2014/11/08 19:43:30 rbj Exp $ $
%}%tiny

\end{document}

% LocalWords:
