% $Id: p031.tex $
% bibref{rbjp031} pdfname{p031}

\documentclass[14pt,titlepage]{extarticle}
\usepackage{makeidx}
\usepackage{graphicx}
\usepackage[unicode]{hyperref}
\pagestyle{plain}
\usepackage[paperwidth=8.3in,paperheight=11.7in,hmargin={0.3in,0.3in},vmargin={0.5in,0.5in},includehead,includefoot]{geometry}
\hypersetup{pdfauthor={Roger Bishop Jones}}
\hypersetup{pdftitle={Should we have a Written Constitution - and what might it say?}}
\hypersetup{colorlinks=true, urlcolor=red, citecolor=blue, filecolor=blue, linkcolor=blue}
%\usepackage{html}
\usepackage{paralist}
\usepackage{relsize}
\usepackage{verbatim}
\usepackage{enumerate}
\usepackage{longtable}
\makeindex
\newcommand{\ignore}[1]{}
\newcommand{\hreg}[2]{\href{#1}{#2}%\footnote{\tt$#1$}
}

\title{\LARGE \bf Should we have a \\Written Constitution \\- \\and what might it say?}
\author{Roger~Bishop~Jones}
\date{\ }


\begin{document}
%\frontmatter

%\begin{abstract}
% The pace of constitutional change has accelerated over recent decades,
% and the departure of the UK from the European Union not only in itself
% requires major constitutional changes, but also exposes weaknesses in
% the constitution which may provoke further change.
% Because we do not have a written constitution which has a special status
% relative to other legislation, controversial changes to the constitution
% may be undertaken on the basis of a simple parliamentary majority with little
% or no public consultation.
% This risks changes which are partisan rather than consensual, and that respond
% to intellectual or ideological trends alien to our historical traditions including
% our democracy and the free speech on which it depends.
% This document provides notes for a discussion of these matters.
%\end{abstract}
                               
\begin{titlepage}
\maketitle

%\vfill

%\begin{centering}

%{\footnotesize
%copyright\ Roger~Bishop~Jones;
%}%footnotesize

%\end{centering}

\end{titlepage}

\ \

\ignore{
\begin{centering}
{\LARGE \bf Should we have a \\Written Constitution?\\ - \\and what might it say?\\}
\end{centering}
}%ignore

\setcounter{tocdepth}{2}
{\parskip-0pt\tableofcontents}

%\listoffigures

%\mainmatter

%\pagebreak

\section*{Preface}

\addcontentsline{toc}{section}{Preface}

The pace of constitutional change has accelerated over recent decades,
and the departure of the UK from the European Union not only in itself
requires major constitutional changes, but also exposes weaknesses in
our constitution which may provoke further change.
 
Because we do not have a written constitution which has a special status
relative to other legislation, controversial changes to the constitution
may be undertaken on the basis of a simple parliamentary majority with little
or no public consultation.
This risks changes which are partisan rather than consensual, and that respond
to intellectual or ideological trends alien to our historical traditions such as
our democracy and the free speech on which it depends.
 
This document provides notes for a discussion of these matters.

There are ``hyperlinks'' in this PDF document which either link to another point in the document  (if coloured blue) or to an internet resource  (if coloured red) giving direct access to the materials referred to (e.g. a Youtube video) if the document is read using some internet connected device.

\section{Introduction}

Every now and then over the course of my life I have read a book which transformed the way I thought about the world.
I sometimes thought of these as my ``book of the decade''.

In the course of preparing for this discussion over the last couple of months, I have experienced just such a feeling, but not this time as a result of reading a book, rather of viewing a rich seam of (often controversial) thought on YouTube videos.
I have provided in appendix \ref{heroines} links to some of these thought provoking videos.
None of them is by a professional philosopher.

At times I have felt that I was dragged away from the ostensible topic of my researches, by very interesting and stimulating but peripheral matters, but I have come to feel that this all bears upon constitutions and constitutional change, and deserves to be touched upon in that context.

I propose therefore, that our discussion be undertaken in two parts, first dealing with the kinds of constitutional issue which have arisen over recent decades (which I label derisively 'parochial', but nevertheless consider of some importance), and whether a written constitution might have improved the outcomes, the second addressing some potentially much greater and actually more speculative and controversial threats to our way of life, asking whether a written consitution might provide a degree of protection.

But first a few points of clarification.

\paragraph{Written v Protected}

It is generally said that the UK does not have a written conmstitution.
Nevertheless, a discussion of our constitution exposes a plethora of written materials which substantially determine that constitution.
Apendix \ref{ConstitutionalLaw} sketches some of this material and Appendix \ref{ActList} lists many more.
As well as Constitutional Law there are documented conventions such as the parliamentary practice extensively documented in Erskin May, and there is a body of legal precedent, important both for the interpretation of statutory law and for the development of common law.

This body of written documentation for our constitution is quite different to the kind of written constitution commonly posessed by sovereign states.
For this reason some people prefer to speak of the latter as ``codifed''.

Here are some principal differences as I see it (though many may differ, and ultimately it is up to those who draft the constitution):

\begin{itemize}
\item A written constitution is more comprehensive, though less detailed than constitutional law.
\item The constitution is concerned with fundamental principles, and may therefore defer details of implementation to statutes (albeit constitutional law).
\item A written constitution, partly because comprehensive, will make matters clear that may not be covered clearly by ad hoc constitutional legislation (e.g. the scope of ``royal prerogative'').
\item Crucially, a written constitution may enjoy greater protection from change than statutory law.
  e.g. it may require a larger than 50\% majority in the legislature, and/or other means of estblishing near universal assent (in USA 75\% of states must assent).
  Some kinds of change might possibly require assent of the people through a referendum, also with greater than 50\% majority.
  
\item A written constitution is likely to have greater authority than statutory law, constraining what can be legislated by statute.
  It may provide a basis on which acts of parliament can be declared `unconstitutional' (and hence invalid) by the courts.
  Consequently, those matters regarded as most fundamental to our way of life might be given protection from political whim by being incorporated in the constitution.
  Freedom of opinion and expression, the right to trial by jury, and the principle of innocence until proven guilty are examples, all of which have been eroded or threatened in recent decades. 
\end{itemize}

\section{Part 1 - Parochial Considerations}

\subsection{Constitutional Changes and Controversies}

A substantial number of constitutional amendments have been undertaken over recent decades on the basis of simple majorities in parliament or in referenda.

Many of these concern major changes such as devolution of government, independence referenda, entering the European Community and Union, ceding further powers to the Union through additional treaties, and exiting the European Union.
It is a frequent complaint that the public were mislead about these changes.

Here are some of the consitutional changes which have taken place in recent decades, arguably without sufficiently broad explicit support:

\begin{itemize}
\item actual devolution and independence referenda
\item joining the common market and entering to a series of subsequent treaties extending its powers
  \item legislation seriously impeding our right to discuss matters relating to protected minorities (where claims of offensiveness may be made).
  \end{itemize}

Further to that some points at which government and `the people' have been wrong footed because of the lack of clarity inherent in a non-codified constitution.

\begin{itemize}
\item The intervention of the courts in relation to the invocation of article 50.
\item The intervention of the courts on the exercise of the royal prerogative to prorogue parliament.
\end{itemize}

In both these matters the government took a course which it believed to be within its powers, and the courts ruled that not to be the case.
Furthermore, because of the lack of clarity in the constitution, many lay observers looked upon this as overreach by the supreme court.

A further important consequence it seems to me is the ready accusations by all parties that some outcome is ``un-democratic'', as if any perceived majority cannot be refused its demands.
It is our consitution which should clearly define the democratic process, and the term ``un-democratic'' would then be completely superceded in significamce by the more precise term ``un-consitutional'', the constitution defining clearly what democracy is in the United Kingdom.

\subsection{How a Written Constitution Might Help}

\begin{itemize}
\item the government is less likely to misconstrue its constitutional powers, (e.g. the scope of Royal Prerogative), and hence less likely to be frustrated by the courts.
\item the people are less likley to feel that intervention by the courts is anti-democratic if the basis for that intervention in the constitution is explicit and clear.
\item There will be fewer occasions on which the general public make reference to the woolly critique ``un-democratic'' if the nature of our democracy is properly codified in a written constitution.
\item essential aspects of our way of life can be given protection from partisan whim enacted by simple parliamentary majority if incorporated into a written constitution which can only be amended by a parliamentary supermajority (say 66\%) or by other more elaborate safeguards (e.g. referenda).
  Thinking here of freedom of belief and expression, equality before the law, innocence until proven utility, fundamental human rights...
\item Children would be taught about the constitution, would understand better how it is supposed to work and how important it is to our way of life.
\item A clear basis for limits to the tolerance extended through multi-cultural toleration would be in place, to ensure that immigrants could be expected to understand that their culture can only be replicated in the United Kingdom to the extent that it is consistent with our constitution.
\end{itemize}


\section{Part 2 - The Enemy Within}

\subsection{On Tolerance}

Karp Popper on `The Paradox of Intolerance':

\begin{quote}
Less well known is the paradox of tolerance: Unlimited tolerance must lead to the disappearance of tolerance. If we extend unlimited tolerance even to those who are intolerant, if we are not prepared to defend a tolerant society against the onslaught of the intolerant, then the tolerant will be destroyed, and tolerance with them. — In this formulation, I do not imply, for instance, that we should always suppress the utterance of intolerant philosophies; as long as we can counter them by rational argument and keep them in check by public opinion, suppression would certainly be unwise. But we should claim the right to suppress them if necessary even by force; for it may easily turn out that they are not prepared to meet us on the level of rational argument, but begin by denouncing all argument; they may forbid their followers to listen to rational argument, because it is deceptive, and teach them to answer arguments by the use of their fists or pistols. We should therefore claim, in the name of tolerance, the right not to tolerate the intolerant.
\end{quote}

Multi-culturalism seems to recognise no bounds to the tolerance which it extends to intolerant religious minorities (which might in due course grow in to majorities).
Astonishingly, in the face of death penalties imposed by fatwa on the author even of fictional works the parliament has responded with legislation which makes it an offence to `offend'.
Equivocation on practices such as female genital mutilation, forced marriage and `honour killings', is a betrayal of the rights and protections which should be conferred on the genuinely oppressed in our society in favour of those who reject the most fundamental principles of our own culture and heritage.

A written constitution could and should provide some protection against this kind of regression.

\subsection{Is `Moderate Islam' a Trojan Horse?}

See Ayan Hirsi Ali on  \href{https://www.youtube.com/watch?v=Bx2hEc7Dlcg}{on the West, Dawa, and Islam} and other videos.

\subsection{The New Face of Marxism - Identity Politics}

Identity politics as characterised by Douglas Murray in his book ``The Madness of Crowds'', his videos such as \href{https://www.youtube.com/watch?v=m2zZMg7SNWA}{Political Correctness and Vacuous Wokeness} and as excoriated by a growing constellation of American conservatives and apostate soft lefties, has come to us, we hear, from Marxism via postmodernism and the Frankfurt school of ``critical theory''.

This new left wing ideological orthodoxy corrupts western society from within without help even from the moderate face of alien religious fanaticisms.
Not so long ago I reflected on how long past 1984 we have come without falling into the dystopia envisaged by George Orwell, but now we see how a word out of place on social media destroys lives.
It is not only necessary to think pure thoughts, but to be seen to have understood and fluently mastered the newspeak and doublethink of identity politics, intersectionality and grievance studies.

\paragraph{Some of the more extreme idiocies}

\begin{itemize}
\item any disavantage is understood as the result of oppression, the opressors being those who do not share the disadvantage (and are thus `privileged').
\item equality of opportunity does not suffice, equality of outcome is required, thus whenever any minority is under-represented in an occupation, that must be because of prejudice on the part of the institution, and steps must be taken to correct this.
  This despite scientific research showing that the greater the equality of opportunity, the greater the diversity (inequality) of outcome.
  To get equal representation of women in engineering, or of men in the caring professions (where the ratos are 40:1 in Sweden) would require severe sexual discrimination (which would prpbably not suffice) and would result in many people in occupations to which they are neither suited nor inclined.
\item Some of those seeking to improve the conviction rate for sexual offenses (particularly rape) advocate that the presumption of innocence for defendents be dropped.
\item It is an offense in Canada to refuse to refer to someone by their chosen pronoun.
  There are now about 50 different pronouns for `trans' persons.
\item to cause offense to a disadvantaged person, primarily by use of proscribed language, even if unintentional, is interpreted as an act of violence, to which a violent response is justified.
  ``no platforming'' is an aspect of this attitude.
\item Germain Greer is not a feminist, Candace Owens (a black conservative) is a white supremacist, Peter Thiel (a Californian venture capitalist who came out for Trump) may sleep with men, but in no way is he gay!
  \item The Lindsay Shepherd affair.  See appendix \ref{heroines}.
  \end{itemize}

Some of this just deserves to be laughed at, but some threatens important parts of our culture which should be vigorously defended, just as we fought for freedom and democracy in the second world war.

A written constitution might be the first line of defence.

\section{Postscript}

\subsection{Unstoppable Juggernauts: Global China and Big Tech}

Its quite a novel experience for me to get seriously concerned about domestic political trends.
Until a few months ago I would have identified the economic impact of the growing pace and reach of technological change as by far the greatest, potientially worrying, dynamic.
After that the growing significance on the world stage of China, and the whole panoply of authoritarian regimes.

Now, it is easier for me to see China as a bulwark against religion, a nation reacting vigorously to ideologies which it (possibly correctly) perceives as threatening.

\subsection{Back to the Constitution}

Two closing questions:

\begin{enumerate}
\item Are these perceived threats real or is this just paranoia?

  I don't know.
  On both sides of this debate there may be paranoia, exaggeration and misrepresentation.
  But already it seems to me that we have conceded too much ground to ideologies which are in fundamental contradiction with our values and way of life.
  Some responses to the mere possibility of radical subversion are safe enough that we should undertake them whether or not we believe that the threat is as great as it is painted.
  The codification of our constitution in a way which protects us from frivolous corruption is one such moderate response.

\item Do I really think that a written constitution can protect us?

  No.
  Even a constitution which admits no change whatsoever could not do that, if a converse ideology secured overwhelming support.
  It would simply be discarded.
  But a well written constition may be a bulwark against frivolous degradation, it might hold the fort for a while to give sanity a chance to prevail, and it would prevent constitutional change from slipping in under the radar without the silent majority even noticing until too late.
\end{enumerate}

\subsection{Revisiting `The Challenge'}

Going back to my previous topic, {\it the challenges to practical philosophy in the 21st Century}, the question of whether a written constitution can help to protect us from the regressive left and the medina islamists is a small part of that.

In response to the `regressive left', some otherwise left leaning people turn to conservatism, and it is conservatism which is served by constitutional protections.
But we face too many problems, both in the present and the future, not to want to see progress, and the challenge to practical philosophy I see as coming up with a genuinely progressive social and political philosophy which is divorced from the pathologies we see today on the left and provides room for the future which technological advancement will bring.

\pagebreak
{\bf \Large Appendices}
\appendix

\addcontentsline{toc}{section}{Appendices}

\section{Some Links and References}

\begin{itemize}
\item \href{https://www.putneydebates2019.co.uk/New-Putney-Debates}{New Putney Debates 2017 and 2018}
\item \href{https://www.putneydebates2019.co.uk/}{New Putney Debates 2019}
\item \href{https://www.parliament.uk/business/committees/committees-a-z/commons-select/political-and-constitutional-reform-committee/inquiries/parliament-2010/mapping-the-path-to-codifying---or-not-codifying---the-uks-constitution/report-a-new-magna-carta/}{Three blueprints of what form a codified constitution for the UK could take}
  \item \href{https://erskinemay.parliament.uk/}{Erskin May online}
  \item \href{https://en.wikipedia.org/wiki/Erskine_May:_Parliamentary_Practice}{Erskin May: Parliamentary practice - at wikipedia}
\end{itemize}

\section{Constitutional Law}\label{ConstitutionalLaw}

Although the UK has an unwritten constitution, many important elements of it are found in statutes that have been enacted by Parliament. The following are of most importance to the constitution and civil liberties:

\begin{itemize}
  
\item Magna Carta 1215. This embodies the principle that government must be conducted according to the law and with the consent of the governed.

\item Bill of Rights 1689. This imposed limitations on the powers of the monarch and provided that Parliament should meet on a regular basis.

\item Act of Settlement 1701. This prohibited Catholics from succeeding to the throne and gave precedence to male heirs. It also established the constitutional independence of the judiciary.

\item Acts of Union 1706-07. These united England and Scotland under a single Parliament of Great Britain (that is, the Westminster Parliament).

\item Parliament Acts 1911 and 1949. These ensured that the will of the elected House of Commons would prevail over that of the unelected House of Lords by enabling legislation to be enacted without the consent of the House of Lords.

\item European Communities Act 1972. This incorporated EU law and EU legal systems into domestic law (see Question 11).

\item Police and Criminal Evidence Act 1984. This provides the police with wide powers of arrest, search and detention as well as accompanying safeguards to ensure that the police do not abuse such powers.

\item Public Order Act 1986. This allows limitations to be placed on the rights of citizens to hold meetings and demonstrations in public places.

\item Human Rights Act 1998. This incorporates the European Convention on Human Rights into domestic law and allows citizens to raised alleged breaches of their human rights before the domestic courts.

\item Acts of devolution (for example, Scotland Act 1998). These created a devolved system of government in parts of the UK, establishing a Scottish Parliament and assemblies in Wales and Northern Ireland.

\item Constitutional Reform Act 2005. This reformed the office of Lord Chancellor by transferring his powers as head of the judiciary to the Lord Chief Justice. It also created the Supreme Court and a new Judicial Appointments Committee.

\item Case law
The common law is an important source of key legal principles, particularly in relation to the preservation of the rights of the individual against the state and the rule of law.
\end{itemize}

\section{Treaties and Acts of Parliament}\label{ActList}

{\small
\begin{longtable}{l l}
Treaty of Union & 1706\\
Acts of Union & 1707\\
Succession to the Crown Act &  1707\\
Septennial Act & 1716\\
Wales and Berwick Act & 1746\\
Constitution of Ireland (1782) & 1782\\
Acts of Union 1800 & 1800\\
HC (Disqualifications) Act &  1801\\
Reform Act &  1832\\
Scottish Reform Act  & 1832\\
Irish Reform Act  & 1832\\
Judicial Committee Act  & 1833\\
Judicial Committee Act  & 1843\\
Judicial Committee Act  & 1844\\
Colonial Laws Validity Act & 1865\\
British North America Act  & 1867\\
Representation of the People Act &  1867\\
Reform Act (Scotland)  & 1868\\
Reform Act (Ireland)  & 1868\\
Irish Church Act & 1869\\
Royal Titles Act  & 1876\\
Appellate Jurisdiction Act & 1876\\
Reform Act  & 1884\\
Interpretation Act  & 1889\\
Cth of Australia Constitution Act & 1900\\
Parliament Act & 1911\\
Aliens Restriction Act & 1914\\
Status of Aliens Act  & 1914\\
Government of Ireland Act  & 1914\\
Welsh Church Act & 1914\\
Royal Proclamation of  & 1917\\
Representation of the People Act &  1918\\
Church of England Assembly (Powers) Act & 1919\\
Government of Ireland Act & 1920\\
Anglo-Irish Treaty & 1921\\
Church of Scotland Act  & 1921\\
Irish Free State (Agreement) Act & 1922\\
Irish Free State Constitution Act & 1922\\
Ireland (Confirmation of Agreement) Act &  1925\\
Balfour Declaration of &  1926\\
Royal and Parliamentary Titles Act & 1927\\
Representation of the People Act &  1928\\
Eire (Confirmation of Agreement) Act &  1929\\
Statute of Westminster & 1931\\
HM Declaration of Abdication Act &  1936\\
Regency Act &  1937\\
Regency Act  & 1943\\
Indian Independence Act & 1947\\
Burma Independence Act & 1947\\
British Nationality Act &  1948\\
Representation of the People Act &  1948\\
Ireland Act  & 1949\\
Statute of the Council of Europe & 1949\\
Parliament Act  & 1949\\
Regency Act  & 1953\\
Royal Titles Act  & 1953\\
European Convention on Human Rights & 1953\\
Interpretation Act (NI) & 1954\\
HC Disqualification Act  & 1957\\
Life Peerages Act & 1958\\
Commonwealth Immigrants Act  & 1962\\
Peerage Act & 1963\\
West Indies Act &  1967\\
Commonwealth Immigrants Act  & 1968\\
Immigration Act & 1971\\
EC Treaty of Accession & 1972\\
NI (Temporary Provisions) Act & 1972\\
European Communities Act & 1972\\
Local Government Act & 1972\\
The UK joins the European Communities & 1973\\
Local Government (Scotland) Act & 1973\\
NI border poll & 1973\\
NI Constitution Act & 1973\\
House of Commons Disqualification Act & 1975\\
Referendum Act & 1975\\
EC membership referendum & 1975\\
Interpretation Act & 1978\\
Scotland Act  & 1978\\
Wales Act  & 1978\\
Scottish devolution referendum & 1979\\
Welsh devolution referendum & 1979\\
British Nationality Act  & 1981\\
Representation of the People Act  & 1983\\
Representation of the People Act  & 1985\\
Maastricht Treaty & 1993\\
Local Government (Wales) Act & 1994\\
Local Government etc. (Scotland) Act & 1994\\
Referendums (Scotland \& Wales) Act & 1997\\
Scottish devolution referendum & 1997\\
Welsh devolution referendum & 1997\\
Good Friday Agreement & 1998\\
Northern Ireland Act & 1998\\
Government of Wales Act & 1998\\
Human Rights Act & 1998\\
Scotland Act & 1998\\
House of Lords Act & 1999\\
Representation of the People Act  & 2000\\
Political Parties, Elections and Referendums Act & 2000\\
Constitutional Reform Act & 2005\\
Government of Wales Act  & 2006\\
Northern Ireland Act  & 2009\\
Lisbon Treaty & 2009\\
Parl Voting System and Constituencies Act & 2011\\
Welsh devolution referendum & 2011\\
2011 United Kingdom Alternative Vote referendum & 2011\\
European Union Act &  2011\\
Fixed-term Parliaments Act & 2011\\
Scotland Act &  2012\\
Succession to the Crown Act &  2013\\
Scottish independence referendum & 2014\\
House of Lords Reform Act & 2014\\
Wales Act &  2014\\
HL (Expulsion and Suspension) Act & 2015\\
European Union Referendum Act & 2015\\
EU membership referendum & 2016\\
Scotland Act &  2016\\
Wales Act &  2017\\
EU (Notification of Withdrawal) Act & 2017\\
Invocation of Article 50 & 2017\\
European Union (Withdrawal) Act & 2018\\
EU Withdrawal Act &  2019\\
EU Withdrawal Act (No. 2) & 2019\\
Early Parliamentary General Election Act & 2019\\
The UK leaves the European Union & 2020\\
\end{longtable}
}%small

\section{Un-Woke Hero(ine)s}\label{heroines}

\begin{tabular}{l l}
Christopher Hitchins & \href{https://www.youtube.com/watch?v=4Z2uzEM0ugY}{Free Speech (and ISLAM)}\\
Douglas Murray & \href{https://www.youtube.com/watch?v=m2zZMg7SNWA}{Political Correctness and Vacuous Wokeness}\\
Dave Rubin & \href{https://www.youtube.com/watch?v=Tq86Beh3T70}{The Left is No Longer Liberal}\\
Jordan Peterson & \href{https://www.youtube.com/watch?v=_iudkPi4_sY}{interview | SVT/TV 2/Skavlan}\\
Sam Harris & \href{https://www.youtube.com/watch?v=Nb-o6NZiWrw&t=2001s}{Ben Shapiro Clashes With Sam Harris On Religion And Morality}\\
Stephen Fry & \href{https://www.youtube.com/watch?v=eJQHakkViPo}{On Political Correctness and Clear Thinking}\\
  Ayan Hirsi Ali & \href{https://www.youtube.com/watch?v=Bx2hEc7Dlcg}{on the West, Dawa, and Islam}\\
  Ayan Hirsi Ali and Maajid Nawaz & \href{https://www.youtube.com/watch?v=SV_GMeZ_XmA}{at JW3 cultural centre in London}\\
Yasmine Mohammed & \href{https://www.youtube.com/watch?v=_PXfMY6YqBY}{The Reality of Islam In The West}\\
Katie Hopkins & Rude! - \href{https://www.youtube.com/watch?v=vTo0crpK8Zs}{at the Cambridge Union}\\
Posie Parker & \href{https://www.youtube.com/watch?v=Pdpc2r4cBxQ}{Trans Women Aren't Women}\\
Candace Owens & \href{https://www.youtube.com/watch?v=MFQsjgXyJjk}{Confronting Violence - Congressional testimony}\\

\end{tabular}

\paragraph{The Lindsay Shepherd Affair}

\begin{itemize}
\item \href{https://www.youtube.com/watch?v=kasiov0ytEc&t=2625s}{Genders, Rights and Freedom of Speech}
\item \href{https://www.youtube.com/watch?v=PkNv4LFpGf4}{The Lindsay Shepherd Affair: Update}
\item \href{https://www.youtube.com/watch?v=YWVmDSMl30s}{Deconstruction: The Lindsay Shepherd Affair}
\end{itemize}

%\section{Some Analysis}

%\addcontentsline{toc}{section}{Bibliography}
%\bibliographystyle{alpha}
%\bibliography{rbj2}

%\addcontentsline{toc}{section}{Index}\label{index}
%{\twocolumn[]
%{\small\printindex}}

%\vfill

%\tiny{
%Started 2017-10-09

%Last Change 2017-10-09

%\href{http://www.rbjones.com/rbjpub/www/papers/p028.pdf}{http://www.rbjones.com/rbjpub/www/papers/p028.pdf}

%}%tiny

\end{document}

% LocalWords:
