% $Id: p031.tex $
% bibref{rbjp031} pdfname{p031}

\documentclass[14pt,titlepage]{extarticle}
\usepackage{makeidx}
\usepackage{graphicx}
\usepackage[unicode]{hyperref}
\pagestyle{plain}
\usepackage[paperwidth=8.3in,paperheight=11.7in,hmargin={0.3in,0.3in},vmargin={0.5in,0.5in},includehead,includefoot]{geometry}
\hypersetup{pdfauthor={Roger Bishop Jones}}
\hypersetup{pdftitle={Should we have a Written Constitution - and what might it say?}}
\hypersetup{colorlinks=true, urlcolor=red, citecolor=blue, filecolor=blue, linkcolor=blue}
%\usepackage{html}
\usepackage{paralist}
\usepackage{relsize}
\usepackage{verbatim}
\usepackage{enumerate}
\usepackage{longtable}
\usepackage{url}
\renewcommand{\refname}{Links and References}
\newcommand{\hreg}[2]{\href{#1}{#2}\footnote{\url{#1}}}
\newcommand{\hreh}[3]{\href{#1}{#2}\cite{#3}}
\makeindex
\newcommand{\ignore}[1]{}

\title{\LARGE \bf Should we have a \\Written Constitution \\- \\and what might it say?}
\author{Roger~Bishop~Jones}
\date{\small 2020/02/09 21:01}


\begin{document}
%\frontmatter

%\begin{abstract}
% The pace of constitutional change has accelerated over recent decades,
% and the departure of the UK from the European Union not only in itself
% requires major constitutional changes, but also exposes weaknesses in
% the constitution which may provoke further change.
% Because we do not have a written constitution which has a special status
% relative to other legislation, controversial changes to the constitution
% may be undertaken on the basis of a simple parliamentary majority with little
% or no public consultation.
% This risks changes which are partisan rather than consensual, and that respond
% to intellectual or ideological trends alien to our historical traditions including
% our democracy and the free speech on which it depends.
% This document provides notes for a discussion of these matters.
%\end{abstract}
                               
\begin{titlepage}
  \maketitle
  
%\vfill
 

%\begin{centering}

%{\footnotesize
%copyright\ Roger~Bishop~Jones;
%}%footnotesize

%\end{centering}
 

\end{titlepage}

\ \

\ignore{
\begin{centering}
{\LARGE \bf Should we have a \\Written Constitution?\\ - \\and what might it say?\\}
\end{centering}
}%ignore

\setcounter{tocdepth}{2}
{\parskip-0pt\tableofcontents}

%\listoffigures

%\mainmatter

%\pagebreak

\section*{Preface}

\addcontentsline{toc}{section}{Preface}

The pace of constitutional change has accelerated over recent decades,
and the departure of the UK from the European Union not only in itself
requires major constitutional changes, but also exposes weaknesses in
our constitution which may provoke further change.
 
Because we do not have a written constitution which has a special status
relative to other legislation, controversial changes to the constitution
may be undertaken on the basis of a simple parliamentary majority with little
or no public consultation.
This risks changes which are partisan rather than consensual, and that respond
to intellectual or ideological trends alien to our historical traditions such as
our democracy and the free speech on which it depends.
 
This document provides notes for discussion of these matters.

There are ``hyperlinks'' in this PDF document which either link to another point in the document  (if coloured blue) or to an internet resource  (if coloured red) giving direct access to the materials referred to (e.g. a Youtube video) if the document is read using some internet connected device.
Citations in the form of numerals in square brackets refer to entries in the list at the end of the document (and are also hyperlinks)

External links are at risk of being broken by changes to or disappearance of the target website, in which case the relevant resource might possibly be found by a search on the title.

\section{Introduction}

Every now and then over the course of my life I have read a book which transformed the way I thought about the world.
I sometimes thought of these as my ``book of the decade''.

In the course of preparing for this discussion over the last few months, I have experienced just such a feeling, but not this time as a result of reading a book, rather of viewing a rich seam of (often controversial) thought on YouTube videos.
I have provided in appendix \ref{heroines} links to some of these thought provoking videos.
None of them is by a professional philosopher.

At times I have felt that I was dragged away from the ostensible topic of my researches, by very interesting and stimulating but peripheral matters, but I have come to feel that this all bears upon constitutions and constitutional change, and deserves to be touched upon in that context.

My first proposal was that our discussion be undertaken in two parts, first (section \ref{part1}) dealing with the kinds of constitutional issue which have arisen over recent decades (which I label derisively 'parochial', but nevertheless consider of some importance), and whether a written constitution might have improved the outcomes, the second (section \ref{part2}) addressing some potentially much greater and actually more speculative and controversial threats to our way of life, asking whether a written constitution might provide a degree of protection.
We more or less followed that prescription in the January meeting, the topic creating lively discussion, and it was decided that the same topic should be continued in February.

I am now proposing a different approach for the February meeting.
Again I suggest taking the discussion in two parts, but the parts I suggest this time are different:
\begin{itemize}
\item
    First I suggest we reconsider the pros and cons of written constitutions, starting from a very condensed account of the 21 pros and 21 cons enumerated by the Kings College report \cite{rbjw003}.
\item
  Second I recommend discussing the preamble to the constitution, based on the winning entry to the competition held after the Kings College report 2010-11 \cite{rbjw011}.
  The preamble is our most compact account of the purpose and content of the constitution.

\end{itemize}

I have added a two new section (sections \ref{part3} \ref{part4}) covering these suggestions.


But first a few points of clarification.

\paragraph{Written v Protected}

It is generally said that the UK does not have a written constitution.
Nevertheless, a discussion of our constitution exposes a plethora of written materials which substantially determine that constitution.
Appendix \ref{ConstitutionalLaw} sketches some of this material and Appendix \ref{ActList} lists many more.
As well as Constitutional Law there are documented conventions such as the parliamentary practice extensively documented in Erskine May, and there is a body of legal precedent, important both for the interpretation of statutory law and for the development of common law.

This body of written documentation for our constitution is quite different to the kind of written constitution commonly possessed by sovereign states.
For this reason some people prefer to speak of the latter as ``codified''.

Here are some principal differences as I see it (though many may differ, and ultimately it is up to those who draft the constitution):

\begin{itemize}
\item A written constitution is more comprehensive, though less detailed than constitutional law.
\item The constitution is concerned with fundamental principles, and may therefore defer details of implementation to statutes (albeit constitutional law).
\item A written constitution, partly because comprehensive, will make matters clear that may not be covered clearly by ad hoc constitutional legislation (e.g. the scope of ``royal prerogative'').
\item Crucially, a written constitution may enjoy greater protection from change than statutory law.
  e.g. it may require a larger than 50\% majority in the legislature, and/or other means of establishing near universal assent (in USA 75\% of states must assent).
  Some kinds of change might possibly require assent of the people through a referendum, also with greater than 50\% majority.
  
\item A written constitution is likely to have greater authority than statutory law, constraining what can be legislated by statute.
  It may provide a basis on which acts of parliament can be declared `unconstitutional' (and hence invalid) by the courts.
  Consequently, those matters regarded as most fundamental to our way of life might be given protection from political whim by being incorporated in the constitution.
  Freedom of opinion and expression, the right to trial by jury, and the principle of innocence until proven guilty are examples, all of which have been eroded or threatened in recent decades. 
\end{itemize}

\subsection{Some Prior Discussion}

There has been much prior discussion of our constitution in general and of the possibility of a written constitution in particular and there is a great deal of relevant material available on line to help constitutional deliberations.

A distinguished but rare predecessor to contemporary constitutional debates was a series of debates held in St Mary’s Church, Putney in 1647 between the two English Civil Wars.
After the UK referendum on leaving the EU further constitutional debates were held in the same Church, first in 2017 \cite{rbjw001}, and again in 2018 and 2019 \cite{rbjw002}.

Prior to that the House of Commons agreed in June 2010 that a select committee should be established, for the duration of the 2010 Parliament, to consider political and constitutional reform.
This committee, the Political and Constitutional Reform Committee, commissioned work on the matter from Kings College London.
The resulting report \cite{rbjw003}, included elaboration of the cases for and against codification of the constitution, and three blueprints \cite{rbjw004} for alternative ways of codifying the constitution.

After the publication of that report an open public competition was held to chose the best Preamble - or introductory statement - for a modern written constitution.
The winning entry from the public was by Richard Elliot \cite{rbjw011}.

\section{Part 1 - Parochial Considerations}\label{part1}

\subsection{Constitutional Changes and Controversies}

A substantial number of constitutional amendments have been undertaken over recent decades on the basis of simple majorities in parliament or in referenda.

Many of these concern major changes such as devolution of government, independence referenda, entering the European Community and Union, ceding further powers to the Union through additional treaties, and exiting the European Union.
It is a frequent complaint that the public were mislead about these changes.

Here are some of the constitutional changes which have taken place in recent decades, arguably without sufficiently broad explicit support:

\begin{itemize}
\item actual devolution and independence referenda
\item joining the common market and entering to a series of subsequent treaties extending its powers
  \item legislation seriously impeding our right to discuss matters relating to protected minorities (where claims of offensiveness may be made).
  \end{itemize}

Further to that some points at which government and `the people' have been wrong footed because of the lack of clarity inherent in a non-codified constitution.

\begin{itemize}
\item The intervention of the courts in relation to the invocation of article 50.
\item The intervention of the courts on the exercise of the royal prerogative to prorogue parliament.
\end{itemize}

In both these matters the government took a course which it believed to be within its powers, and the courts ruled that not to be the case.
Furthermore, because of the lack of clarity in the constitution, many lay observers looked upon this as overreach by the supreme court.

A further important consequence it seems to me is the ready accusations by all parties that some outcome is ``un-democratic'', as if any perceived majority cannot be refused its demands.
It is our constitution which should clearly define the democratic process, and the term ``un-democratic'' would then be completely superceded in significance by the more precise term ``un-constitutional'', the constitution defining clearly what democracy is in the United Kingdom.

\subsection{How a Written Constitution Might Help}

\begin{itemize}
\item The government is less likely to misconstrue its constitutional powers, (e.g. the scope of Royal Prerogative), and hence less likely to be frustrated by the courts.
\item The people are less likely to feel that intervention by the courts is anti-democratic if the basis for that intervention in the constitution is explicit and clear.
\item There will be fewer occasions on which the general public make reference to the woolly critique ``un-democratic'' if the nature of our democracy is properly codified in a written constitution.
\item Essential aspects of our way of life can be given protection from partisan whim enacted by simple parliamentary majority if incorporated into a written constitution which can only be amended by a parliamentary supermajority (say 66\%) or by other more elaborate safeguards (e.g. referenda).
  Thinking here of freedom of belief and expression, equality before the law, innocence until proven utility, fundamental human rights...
\item Children would be taught about the constitution, would understand better how it is supposed to work and how important it is to our way of life.
\item A clear basis for limits to the tolerance extended through multi-cultural toleration would be in place, to ensure that immigrants could be expected to understand that their culture can only be replicated in the United Kingdom to the extent that it is consistent with our constitution (and law).
\end{itemize}

\section{Part 2 - The Enemy Within}\label{part2}

\subsection{On Tolerance}

Karp Popper on `The Paradox of Intolerance':

\begin{quote}
Less well known is the paradox of tolerance: Unlimited tolerance must lead to the disappearance of tolerance. If we extend unlimited tolerance even to those who are intolerant, if we are not prepared to defend a tolerant society against the onslaught of the intolerant, then the tolerant will be destroyed, and tolerance with them. — In this formulation, I do not imply, for instance, that we should always suppress the utterance of intolerant philosophies; as long as we can counter them by rational argument and keep them in check by public opinion, suppression would certainly be unwise. But we should claim the right to suppress them if necessary even by force; for it may easily turn out that they are not prepared to meet us on the level of rational argument, but begin by denouncing all argument; they may forbid their followers to listen to rational argument, because it is deceptive, and teach them to answer arguments by the use of their fists or pistols. We should therefore claim, in the name of tolerance, the right not to tolerate the intolerant.
\end{quote}

Multiculturalism seems to recognise no bounds to the tolerance which it extends to intolerant religious minorities (which might in due course grow in to majorities).
Astonishingly, in the face of death penalties imposed by fatwa on the author even of fictional works the parliament has responded with legislation which makes it an offence to `offend'.
Equivocation on practices such as female genital mutilation, forced marriage and `honour killings', is a betrayal of the rights and protections which should be conferred on the genuinely oppressed in our society in favour of those who reject the most fundamental principles of our own culture and heritage.

A written constitution could and should provide some protection against this kind of regression.

\subsection{Is `Moderate Islam' a Trojan Horse?}

See Ayan Hirsi Ali on  \hreg{https://www.youtube.com/watch?v=Bx2hEc7Dlcg}{on the West, Dawa, and Islam} and other videos.

\subsection{The New Face of Marxism - Identity Politics}

Identity politics as characterised by Douglas Murray in his book ``The Madness of Crowds'', his videos such as \hreg{https://www.youtube.com/watch?v=m2zZMg7SNWA}{Political Correctness and Vacuous Wokeness} and as excoriated by a growing constellation of American conservatives and apostate soft lefties, has come to us, we hear, from Marxism via postmodernism and the Frankfurt school of ``critical theory''.

This new left wing ideological orthodoxy corrupts western society from within without help even from the moderate face of alien religious fanaticisms.
Not so long ago I reflected on how long past 1984 we have come without falling into the dystopia envisaged by George Orwell, but now we see how a word out of place on social media destroys lives.
It is not only necessary to think pure thoughts, but to be seen to have understood and fluently mastered the newspeak and doublethink of identity politics, intersectionality and grievance studies.

\paragraph{Some of the more extreme idiocies}

\begin{itemize}
\item Any disadvantage is understood as the result of oppression, the oppressors being those who do not share the disadvantage (and are thus `privileged').
\item Equality of opportunity does not suffice, equality of outcome is required, thus whenever any minority is under-represented in an occupation, that must be because of prejudice on the part of the institution, and steps must be taken to correct this.
  This despite scientific research showing that the greater the equality of opportunity, the greater the diversity (inequality) of outcome.
  To get equal representation of women in engineering, or of men in the caring professions (where the ratios are 40:1 in Sweden) would require severe sexual discrimination (which would probably not suffice) and would result in many people in occupations to which they are neither suited nor inclined.
\item Some of those seeking to improve the conviction rate for sexual offences (particularly rape) advocate that the presumption of innocence for defendants be dropped.
\item It is an offence in Canada to refuse to refer to someone by their chosen pronoun.
  There are now about 50 different pronouns for `trans' persons.
\item To cause offence to a disadvantaged person, primarily by use of proscribed language, even if unintentional, is interpreted as an act of violence, to which a violent response is justified.
  ``no platforming'' is an aspect of this attitude.
\item Germain Greer is not a feminist, Candace Owens (a black conservative) is a white supremacist, Peter Thiel (a Californian venture capitalist who came out for Trump) may sleep with men, but in no way is he gay!
  \item The Lindsay Shepherd affair.  See appendix \ref{heroines}.
  \end{itemize}

Some of this just deserves to be laughed at, but some threatens important parts of our culture which should be vigorously defended, just as we fought for freedom and democracy in the second world war.

A written constitution might be the first line of defence.

\section{Part 3 - Pros and Cons of a Written Constitution}\label{part3}

The following analysis of the pros and cons is derived from the arguments presented in the Kings College report \cite{rbjw003}.
They form Part 1 of the Appendix to the report, beginning at page 19.

In undertaking the analysis I first prepared a draconian precis of the `arguments', for and against, presented in the report.
These are shown below in Appendices \ref{For} and \ref{Against}.
The numbering in the appendices follows that in the original report, so for a fuller understanding of any particular item refer to the original.
In the following analysis the numeric references are to the items in these lists of pros and cons.

Many of these give reasons for or against by citing features or consequences of a written constitution which are not exclusive to a written constitution (perhaps they would accrue from any written account, whether in a constitution or in constitutional legislation, or even in a descriptive rather than prescriptive account of the existing constitution).
In considering each case we may ask whether tha claim is in fact correct (would the alleged consequence actually follow) and we have to consider whether the consequence is good or bad.
These speak to the validity of the claim.
Having thus weeded out the wheat from the chaff the question arises whether the positives outweigh the negatives.

I therefore attempt an analysis of that kind.

Preliminary to that some terminological clarification.

For a conclusive case, we must rest most heavily upon considerations which are exclusive to a written constitution, rather than on benefits which might accrue from lesser measures,
To make that distinction we need to be clear about what, for the purposes of this discussion, a written constition is.
I suggest the following working definition.

A ``written constitution'':
\begin{itemize}
\item[c1)] is a single written document
\item[c2)] has an initial overview which would be intelligible to a comfortable majority of adults and is suitable for teaching children about our constitution
\item[c3)] has legal force and has priority over all other UK law
  \item[c4)] is protected from frivolous and partisan change by requiring stronger evidence of general support for any proposed change than is required for normal legislation.
\end{itemize}

So that we can consider whether benefits or disadvantages might accrue from something short of a written constitution I suggest here a brief classification of what those lesser measures might be.

\begin{itemize}
\item[w1)] written (prescriptive) constitutional legislation
\item[w2)] written descriptions of various constitutional conventions
\item[w3)] a written overview giving a comprehensive guide to constitutional legislation and other documentation
\item[w4)] a written description of the constitution at a level and in terms suitable for the general public and for teaching in schools
\item[w5)] a comprehensive programme of constitutional legislation to bring all constitutional matters into written form and to provide a generally intelligible statement of the main principles underlying the constitution
\end{itemize}

\subsection{Arguments For}

These fall under the following main categories:


\paragraph{clarity [2,3,4,13,15,17]}  

  accessibility, intelligibility, teachability, disambiguation/clarity

\ 
  
  I think most people would think that these are desirable and could be realised by a written constitution, but might also be realised by some other kind of constitutional documentation.

\paragraph{constitutional change}

strengthen popular sovereignty [8], protection for local government [10], clarification of devolution [11], safeguards to ensure good governance [12], separation of powers [16], constitutional reform [19], systematisation [18]

\ 

These might not all be as uncontroversial as the previous set, and whether a written constitution would include them is moot.
Some of these probably are benefits which would only accrue by inclusion in a written constitution, for example safeguards for good governance and separation of powers (which are typically key elements of constitutions).

\paragraph{benefits of process}
popular participation and consent [9], popular support [14], consensus [20], symbolism [21].

\

The establishment of a written constitution depends on the first three, and if they are also seen as resulting from one, so much the better.
However, including them in a case for a written sounds circular, you need other grounds to argue for support.

\paragraph{special status}

special status for constitutional fundamentals [6], protection from constitutional change [7]

In my opinion these are the two most important substantive features which are desirable ends realisable only by a written constitution.

\subsection{Arguments Against}

\paragraph{not needed [1,2,8,9,11,15,16,20]}

\

\

That we might manage without is not a strong argument against, particularly if it can be argued that though not essential a written constitution would be an improvement.

\paragraph{disadvantages of process}

time consuming[17], divisive[19], no consensus[21]

\

\

Proponents would accept the first and hope that the time consuming process would gradually become less divisive and lead to consensus.
Ultimately, if no sufficient consensus is reached then the constitution will never be accepted, so the question becomes whether the odds are good enough to justify making the effort.

\paragraph{special status}

not flexible [3,10]

\ 

\

There is no doubt that this is one of the primary supposed advantages, so it's important that only those things which we don't want to change often or much should go into the constitution.

\paragraph{meaningless[12], alien[13], disruptive[18]}

\

\

Clearly those in favour would like to avoid these possible features!

\paragraph{transfers power to courts [4,5,6]}

\

\ 

My guess is that clarification of the powers of the various parties would make it less likely that resort to the courts would be necessary, but the constraints on legislation imposed by a constitution would introduce the possibility that acts of parliament could be challenged in the courts.
There is room for debate about the extent to which this is transfer of substantive power to the courts, they are arguably even when nullifying legislation, adjudicating conflicts between the actions of the legislature and the power of the people as encoded in the constitution.


\section{Part 4 - Constitutional Preambles}\label{part4}

In 2010 a ``Modern Written Constitution for UK competition'' was held in which members of the public were asked to submit a preamble for a modern written constitution.

I paraphrase here for discussion the main aims of the constitution stated in the winning preamble by Richard Elliot.

\begin{itemize}
\item Every citizen an equal partner in government.
\item Equality under the law.
\item Equality of opportunity for all citizens.
\item To eradicate poverty and want throughout the nation.
\item To protect and cultivate community identities within the countries of the union.
\item To preserve our environment.
\item To safeguard freedom of thought, conscience, assembly and peaceable dissent.
\item To protect these fundamental rights.
\end{itemize}

\section{Postscript}

\subsection{Unstoppable Juggernauts: Global China and Big Tech}

Its quite a novel experience for me to get seriously concerned about domestic political trends.
Until a few months ago I would have identified the economic impact of the growing pace and reach of technological change as by far the greatest, potentially worrying, dynamic.
After that the growing significance on the world stage of China, and the whole panoply of authoritarian regimes.

Now, it is easier for me to see China as a bulwark against religion, a nation reacting vigorously to ideologies which it (possibly correctly) perceives as threatening.

\subsection{Back to the Constitution}

Two closing questions:

\begin{enumerate}
\item Are these perceived threats real or is this just paranoia?

  I don't know.
  On both sides of this debate there may be paranoia, exaggeration and misrepresentation.
  But already it seems to me that we have conceded too much ground to ideologies which are in fundamental contradiction with our values and way of life.
  Some responses to the mere possibility of radical subversion are safe enough that we should undertake them whether or not we believe that the threat is as great as it is painted.
  The codification of our constitution in a way which protects it from frivolous corruption is one such moderate response.

\item Do I really think that a written constitution can protect us?

  No.
  Even a constitution which admits no change whatsoever could not do that, if a converse ideology secured overwhelming support.
  It would simply be discarded.
  But a well written constitution may be a bulwark against frivolous degradation, it might hold the fort for a while to give sanity a chance to prevail, and it would prevent constitutional change from slipping in under the radar without the silent majority even noticing until too late.
\end{enumerate}

\subsection{Revisiting `The Challenge'}

Going back to my previous topic, {\it the challenges to practical philosophy in the 21st Century}, the question of whether a written constitution can help to protect us from the regressive left and the medina islamists is a small part of that.

In response to the `regressive left', some otherwise left leaning people turn to conservatism, and it is conservatism which is served by constitutional protections.
But we face too many problems, both in the present and the future, not to want to see progress, and the challenge to practical philosophy I see as coming up with a genuinely progressive social and political philosophy which is divorced from the pathologies we see today on the left and provides room for the future which technological advancement will bring.

This is something I would like to take further, and I am finding the problem of formulating a constitution to be a good way to approach a forward looking political and social philosophy.
The constraint imposed by a search for a constitution is an antidote to the tribalism characteristic of our increasingly polarised political ``dialogue'' (which is more an exchange of name-calling and reciprocal misrepresentation).
A viable constitution can only be the result of an arduous search for common ground, putting aside for a moment the hopefully less fundamental issues which have driven us apart.

To that end I am now beginning work on philosophical project to come up with a philosophical analysis of the issues arising in such an undertaking, which I will call ``Prolegomena to a Progressive Democratic Constitution for the United Kingdom'' \cite{rbjp032}, and which I hope will provide fodder for some future philosophical discussions.


\pagebreak
{\bf \Large Appendices}
\appendix

\addcontentsline{toc}{section}{Appendices}


\section{Constitutional Law}\label{ConstitutionalLaw}

Although the UK has an unwritten constitution, many important elements of it are found in statutes that have been enacted by Parliament. The following are of most importance to the constitution and civil liberties:

\begin{itemize}
  
\item Magna Carta 1215. This embodies the principle that government must be conducted according to the law and with the consent of the governed.

\item Bill of Rights 1689. This imposed limitations on the powers of the monarch and provided that Parliament should meet on a regular basis.

\item Act of Settlement 1701. This prohibited Catholics from succeeding to the throne and gave precedence to male heirs. It also established the constitutional independence of the judiciary.

\item Acts of Union 1706-07. These united England and Scotland under a single Parliament of Great Britain (that is, the Westminster Parliament).

\item Acts of Union 1800. These united the Kingdom of Great Britain and the Kingdom of Ireland.

  \item  Anglo-Irish Treaty of December 1921, which established the Irish Free State as a dominion of the British Empire. 

\item Parliament Acts 1911 and 1949. These ensured that the will of the elected House of Commons would prevail over that of the unelected House of Lords by enabling legislation to be enacted without the consent of the House of Lords.

\item European Communities Act 1972. This incorporated EU law and EU legal systems into domestic law (see Question 11).

\item Police and Criminal Evidence Act 1984. This provides the police with wide powers of arrest, search and detention as well as accompanying safeguards to ensure that the police do not abuse such powers.

\item Public Order Act 1986. This allows limitations to be placed on the rights of citizens to hold meetings and demonstrations in public places.

\item Human Rights Act 1998. This incorporates the European Convention on Human Rights into domestic law and allows citizens to raised alleged breaches of their human rights before the domestic courts.

\item Acts of devolution (for example, Scotland Act 1998). These created a devolved system of government in parts of the UK, establishing a Scottish Parliament and assemblies in Wales and Northern Ireland.

\item Constitutional Reform Act 2005. This reformed the office of Lord Chancellor by transferring his powers as head of the judiciary to the Lord Chief Justice. It also created the Supreme Court and a new Judicial Appointments Committee.

\item Case law
The common law is an important source of key legal principles, particularly in relation to the preservation of the rights of the individual against the state and the rule of law.
\end{itemize}

\section{Other Constitutional Documentation}

The workings of government are also documented in various government publications which include:

\begin{itemize}
\item The Cabinet Manual \cite{rbjw005}
  
  - the main laws, rules and conventions affecting the conduct and operation of government.

\item The Ministerial Code \cite{rbjw006}

  - the standards of conduct expected of ministers and how they discharge their duties.
  
\item The Civil Service Code \cite{rbjw007}

  - the statutory basis for the management of the Civil Service as set out in Part 1 of the Constitutional Reform and Governance Act 2010.

\item The Osmotherly Rules \cite{rbjw008}
  
  - guidance for officials from departments and agencies on giving evidence to Parliamentary Select Committees
\item Erskine May \cite{rbjw009,rbjw010}

  - the most authoritative and influential work on parliamentary procedure and constitutional conventions affecting Parliament
\end{itemize}

\pagebreak
\section{Treaties and Acts of Parliament}\label{ActList}

{\small
\begin{longtable}{l l}
Treaty of Union & 1706\\
Acts of Union & 1707\\
Succession to the Crown Act &  1707\\
Septennial Act & 1716\\
Wales and Berwick Act & 1746\\
Constitution of Ireland (1782) & 1782\\
Acts of Union 1800 & 1800\\
HC (Disqualifications) Act &  1801\\
Reform Act &  1832\\
Scottish Reform Act  & 1832\\
Irish Reform Act  & 1832\\
Judicial Committee Act  & 1833\\
Judicial Committee Act  & 1843\\
Judicial Committee Act  & 1844\\
Colonial Laws Validity Act & 1865\\
British North America Act  & 1867\\
Representation of the People Act &  1867\\
Reform Act (Scotland)  & 1868\\
Reform Act (Ireland)  & 1868\\
Irish Church Act & 1869\\
Royal Titles Act  & 1876\\
Appellate Jurisdiction Act & 1876\\
Reform Act  & 1884\\
Interpretation Act  & 1889\\
Cth of Australia Constitution Act & 1900\\
Parliament Act & 1911\\
Aliens Restriction Act & 1914\\
Status of Aliens Act  & 1914\\
Government of Ireland Act  & 1914\\
Welsh Church Act & 1914\\
Royal Proclamation of  & 1917\\
Representation of the People Act &  1918\\
Church of England Assembly (Powers) Act & 1919\\
Government of Ireland Act & 1920\\
Anglo-Irish Treaty & 1921\\
Church of Scotland Act  & 1921\\
Irish Free State (Agreement) Act & 1922\\
Irish Free State Constitution Act & 1922\\
Ireland (Confirmation of Agreement) Act &  1925\\
Balfour Declaration of &  1926\\
Royal and Parliamentary Titles Act & 1927\\
Representation of the People Act &  1928\\
Eire (Confirmation of Agreement) Act &  1929\\
Statute of Westminster & 1931\\
HM Declaration of Abdication Act &  1936\\
Regency Act &  1937\\
Regency Act  & 1943\\
Indian Independence Act & 1947\\
Burma Independence Act & 1947\\
British Nationality Act &  1948\\
Representation of the People Act &  1948\\
Ireland Act  & 1949\\
Statute of the Council of Europe & 1949\\
Parliament Act  & 1949\\
Regency Act  & 1953\\
Royal Titles Act  & 1953\\
European Convention on Human Rights & 1953\\
Interpretation Act (NI) & 1954\\
HC Disqualification Act  & 1957\\
Life Peerages Act & 1958\\
Commonwealth Immigrants Act  & 1962\\
Peerage Act & 1963\\
West Indies Act &  1967\\
Commonwealth Immigrants Act  & 1968\\
Immigration Act & 1971\\
EC Treaty of Accession & 1972\\
NI (Temporary Provisions) Act & 1972\\
European Communities Act & 1972\\
Local Government Act & 1972\\
The UK joins the European Communities & 1973\\
Local Government (Scotland) Act & 1973\\
NI border poll & 1973\\
NI Constitution Act & 1973\\
House of Commons Disqualification Act & 1975\\
Referendum Act & 1975\\
EC membership referendum & 1975\\
Interpretation Act & 1978\\
Scotland Act  & 1978\\
Wales Act  & 1978\\
Scottish devolution referendum & 1979\\
Welsh devolution referendum & 1979\\
British Nationality Act  & 1981\\
Representation of the People Act  & 1983\\
Representation of the People Act  & 1985\\
Maastricht Treaty & 1993\\
Local Government (Wales) Act & 1994\\
Local Government etc. (Scotland) Act & 1994\\
Referendums (Scotland \& Wales) Act & 1997\\
Scottish devolution referendum & 1997\\
Welsh devolution referendum & 1997\\
Good Friday Agreement & 1998\\
Northern Ireland Act & 1998\\
Government of Wales Act & 1998\\
Human Rights Act & 1998\\
Scotland Act & 1998\\
House of Lords Act & 1999\\
Representation of the People Act  & 2000\\
Political Parties, Elections and Referendums Act & 2000\\
Constitutional Reform Act & 2005\\
Government of Wales Act  & 2006\\
Northern Ireland Act  & 2009\\
Lisbon Treaty & 2009\\
Constitutional Reform and Governance Act & 2010\\
Parl Voting System and Constituencies Act & 2011\\
Welsh devolution referendum & 2011\\
2011 United Kingdom Alternative Vote referendum & 2011\\
European Union Act &  2011\\
Fixed-term Parliaments Act & 2011\\
Scotland Act &  2012\\
Succession to the Crown Act &  2013\\
Scottish independence referendum & 2014\\
House of Lords Reform Act & 2014\\
Wales Act &  2014\\
HL (Expulsion and Suspension) Act & 2015\\
European Union Referendum Act & 2015\\
EU membership referendum & 2016\\
Scotland Act &  2016\\
Wales Act &  2017\\
EU (Notification of Withdrawal) Act & 2017\\
Invocation of Article 50 & 2017\\
European Union (Withdrawal) Act & 2018\\
EU Withdrawal Act &  2019\\
EU Withdrawal Act (No. 2) & 2019\\
Early Parliamentary General Election Act & 2019\\
The UK leaves the European Union & 2020\\
\end{longtable}
}%small

\pagebreak
\section{Arguments For a Written Constitution}\label{For}

\begin{itemize}
  
\item[1]
  We don't have one. We should!

\item[2]
  Many basic rules of our constitution are unwritten understandings or traditions and are inaccessible or unintelligible to ordinary people.

\item[3]
  The constitution should be accessible and intelligible to all, not just the politicians who
are running the country.

\item[4]
  Constitutional history is no longer taught in any depth in schools today, making it important that the constitution is written down in a single document.

\item[5]
  Some constitutional rules remain unclear or ambiguous, for example, whether parliamentary approval is necessary before the government enters into armed conflict abroad.
On such matters, there should be clarity which a codified written constitution would provide.

\item[6]
  The rules about our core institutions of government, especially the
executive (ministers and civil servants), the legislature (the two Houses of
Parliament) and judiciary, are of a fundamental and different character to other
kinds of law. They should be clearly distinguished from ordinary law and codified
in a special document, becoming the United Kingdom's 'written' constitution.

\item[7]
  Codification would protect the constitution from partisan tampering, setting down a minimum set of procedures that govern major constitutional changes.

\item[8]
  A written constitution might express the sovereignty of the people
and circumscribe the powers and duties of members of Parliament in both
Houses.

\item[9]
  The process of establishing a codified constitution would give the people a
role for the first time in determining the central principles of the constitution of
the UK on which they have never been fully consulted.

\item[10]
  A written constitution could protect local government from the centralising interference of westminster.

\item[11]
  A written constitution would clarify the principles for national and regional governance.

\item[12]
  To help buttress public confidence in
the political system, a clear structure of controls and safeguards needs to be
codified into a written constitution that ensures their integrity and standards.

\item[13]
  Any
Historic institutions and ceremonies of past centuries that remain valuable for
today, including the monarchy, can codified into a written constitution
but with clarity over their modern roles, duties and functions.

\item[14]
 Opinion polls show clear popular support for a written constitution. 

\item[15]
 A written constitution would have grear symbolic and educational importance.

\item[16]
 The present constitution has become one of 'elective dictatorship', lacking separation in powers.

\item[17]
  A written constitution could codify the prerogative powers and replace grey areas of constitutional conduct by clear legal provisions.

\item[18]
 Piecemeal codification (constitutional legislation) needs to be joined up and completed in one comprehensive and coherent document forming a written constitution.

\item[19]
  Codification would provide an opportunity for various much needed constitutional reforms.


\item[20]
 It would be easier than you might think, there is broad consensus on many fundamental principles.

\item[21]
  A written constitution would be a confident expression of the United
Kingdom's national identity.

\end{itemize}

\section{Arguments Against a Written Constitution}\label{Against}

\begin{itemize}
  
\item[1]
 `if it ain't broke, don't fix it'
  
\item[2]
 None of the circumstances which typically call for the codification of a constitution exist in the United Kingdom today.
  
\item[3]
 A written constitution is more difficult to change.
  
\item[4]
 A written constitution transfers power from parliament to the courts.
  
\item[5]
 A written constitution would increase politically motivated litigation in the
 courts.
 
\item[6]
 A written constitution would politicise the judiciary, requiring the courts
 to rule on questions of a political nature.
  
\item[7]
 A written constitution would detract from the Crown serving as the United
Kingdom's symbol of state.
  
\item[8]
 Ad hoc constitutional legislation is preferable to a comprehensive scheme of
codification in a written constitution, mainly because more flexible.
  
\item[9]
 If advocates of a written constitution seek to argue that stronger judicial
controls over government are needed, these are already extensive and can be further extended without codification of the constitution.
  
\item[10]
 A written constitution would curb the ability of elected representatives and
their officials to act quickly and flexibly to meet citizens' needs.
  
\item[11]
 If advocates of a written constitution maintain that there are insufficient
institutional checks and balances on the actions, decisions and policies of the
executive (an `elective dictatorship'), this is a simplistic analysis and the reality is
very different.
  
\item[12]
 If advocates of a written constitution propose to codify the status quo, this
will need to include some elementary guiding principles of the existing constitution.
To include such declarations would be meaningless.
  
\item[13]
 A written constitution would most likely enshrine the doctrine of
separation of powers, which in its strict form is alien to the essence of the UK
constitution.
  
\item[14]
 If a watered-down form of written constitution were to be proposed,
codifying the status quo and making it either non-legal in effect or not enforceable
by the judiciary, this would be a futile exercise.
  
\item[15]
  Those who support a written constitution because they regard it as a device
  for implementing a wide-ranging or `joined-up' package of reforms are misguided,
  because each particular reform is better undertaken on its own merits.
  
\item[16]
 There is in reality no popular demand for a written constitution.
  
\item[17]
 The preparation of a written constitution would involve a huge and
disproportionate amount of time and effort and there are more important things to be doing.

\item[18]
  If enacted, a written constitution containing any substantive reforms would cause disruption to the normal workings of government.
  
\item[19]
 To initiate the process towards a written constitution would be to invite
division and controversy at a time when the government should be seeking
national unity.
  
\item[20]
  Written constitutions are only adopted in extreme circumstances.
  Our present circumstances do not justify such extreme measures.
  
\item[21]
  The political parties would never agree on a constitution..

\end{itemize}

\pagebreak
\section{Un-Woke Hero(ine)s}\label{heroines}

\begin{longtable}{l l}
Christopher Hitchins & \hreg{https://www.youtube.com/watch?v=4Z2uzEM0ugY}{Free Speech (and ISLAM)}\\
Douglas Murray & \hreg{https://www.youtube.com/watch?v=m2zZMg7SNWA}{Political Correctness and Vacuous Wokeness}\\
Dave Rubin & \hreg{https://www.youtube.com/watch?v=Tq86Beh3T70}{The Left is No Longer Liberal}\\
Jordan Peterson & \hreg{https://www.youtube.com/watch?v=_iudkPi4_sY}{interview | SVT/TV 2/Skavlan}\\
%Sam Harris & \hreg{https://www.youtube.com/watch?v=Nb-o6NZiWrw&t=2001s}{\small Ben Shapiro Clashes With Sam Harris On Religion And Morality}\\
Stephen Fry & \hreg{https://www.youtube.com/watch?v=eJQHakkViPo}{On Political Correctness and Clear Thinking}\\
  Ayan Hirsi Ali & \hreg{https://www.youtube.com/watch?v=Bx2hEc7Dlcg}{on the West, Dawa, and Islam}\\
  Ayan Hirsi Ali and Maajid Nawaz & \hreg{https://www.youtube.com/watch?v=SV_GMeZ_XmA}{at JW3 cultural centre in London}\\
Yasmine Mohammed & \hreg{https://www.youtube.com/watch?v=_PXfMY6YqBY}{The Reality of Islam In The West}\\
Katie Hopkins & Rude! - \hreg{https://www.youtube.com/watch?v=vTo0crpK8Zs}{at the Cambridge Union}\\
Posie Parker & \hreg{https://www.youtube.com/watch?v=Pdpc2r4cBxQ}{Trans Women Aren't Women}\\
Candace Owens & \hreg{https://www.youtube.com/watch?v=MFQsjgXyJjk}{Confronting Violence - Congressional testimony}\\
\end{longtable}

\paragraph{The Lindsay Shepherd Affair}

\begin{itemize}
\item \hreg{https://www.youtube.com/watch?v=kasiov0ytEc&t=2625s}{Genders, Rights and Freedom of Speech}
\item \hreg{https://www.youtube.com/watch?v=PkNv4LFpGf4}{The Lindsay Shepherd Affair: Update}
\item \hreg{https://www.youtube.com/watch?v=YWVmDSMl30s}{Deconstruction: The Lindsay Shepherd Affair}
\end{itemize}



\pagebreak

\phantomsection
\addcontentsline{toc}{section}{Links and References}\label{LinksAndReferences}
\bibliographystyle{rbjfmu}
\bibliography{rbj}

%\addcontentsline{toc}{section}{Index}\label{index}
%{\twocolumn[]
%{\small\printindex}}

%\vfill

%\tiny{
%Started 2017-10-09

%Last Change 2017-10-09

%\hreg{http://www.rbjones.com/rbjpub/www/papers/p028.pdf}{http://www.rbjones.com/rbjpub/www/papers/p028.pdf}

%}%tiny

\end{document}

% LocalWords:
