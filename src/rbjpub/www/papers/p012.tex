% $Id: p012.tex,v 1.3 2010/12/16 20:10:15 rbj Exp $ bibref{rbjp012} pdfname{p012} 
\documentclass[numreferences]{rbjk}
\usepackage{makeidx}
\newcommand{\ignore}[1]{}
\usepackage[unicode,pdftex]{hyperref}
\hypersetup{pdfauthor={Roger Bishop Jones}}
\hypersetup{colorlinks=true, urlcolor=black, citecolor=black, filecolor=black, linkcolor=black}

%\newtheorem{def}{Definition}
%\newtheorem{conj}{Conjecture}

\makeindex
\begin{document}                                                                                   
\begin{article}
\begin{opening}  
\title{Language Planning and Design Automation}
\author{Roger Bishop \surname{Jones}}
\date{$ $\ $ $}
\runningauthor{Roger Bishop Jones}

\begin{abstract}
A development of aspects of the philosophies of Leibniz and Carnap in the direction of design automation.
\end{abstract}

\end{opening}

\vfill

\begin{centering}
\footnotesize{
Created 2006/12/23

Last Change $ $Date: 2010/12/16 20:10:15 $ $

\href{http://www.rbjones.com/rbjpub/www/papers/p012.pdf}{http://www.rbjones.com/rbjpub/www/papers/p012.pdf}

$ $Id: p012.tex,v 1.3 2010/12/16 20:10:15 rbj Exp $ $\\

}%footnotesize
\end{centering}

\newpage
%\def\tableofcontents{{\parskip=0pt\@starttoc{toc}}}
\setcounter{tocdepth}{4}
{\parskip-0pt\tableofcontents}

\section{Introduction}

Language planning is one of the headings under which Rudolf Carnap described his work in the ``intellectual biography''\cite{carnap63} which he wrote for the volume of ``The Library of Living Philosophers'' devoted to him\cite{carnap63a}.

In this section Carnap speaks of the two kinds of Universal Language which Leibniz had pursued, and the problem arising from his own linguistic pluralism (expressed in his ``principle of tolerance'').
Most of that material is devoted to universal natural languages, with little said about the problem more central to his own philosophical preoccupations of planning the diversity of formal languags necessary for the formalisation of science and for the conduct of philosophy along scientific lines.

Many other aspects of Carnap's work can be seen as relevant to this strategic problem of ``language planning'', such as his work on logical syntax and the later work on semantics, but on the general problem there does not seem to be a substantial body of work.

This document is not about Carnap or about Leibniz, it is about those aspects of my own thinking which might be seen as falling under a ``language planning'' header.
A substantial part of my own philosophical thinking and of my more detailed technical are intended to progress ideas similar to those of Leibniz and Carnap.
In brief this is the enterprise of formalisation and automation science, and such other aspects of our knowledge for which that may be beneficial, notably design automation.

My own interest dates several hundred years after Leibniz and half a century later than Carnap, and my own thinking therefore benefits from some aquaintaince with a more fully developed field of mathematical logic and a lifetime's exposure to the digital computer some of that in the application of formal techniques to the development of digital hardware and software.
It is natural that my own perspective will be very different to that of Leibniz and Carnap, and there is therefore some point in sketching out the significance of `language planning' in my own systematic thinking.

\section{A Most General Sketch}

Both Leibniz and Carnap were interested in two most general conceptions of what Carnap thought of as `language planning'.

The first of these was the problem of devising a universal `natural' language.
In this context `natural' should be understood by contrast with `formal' rather than by contrast with `artificial' or `constructed'.
This is the problem of devising a language for use by people in writing and in speech.
The desire was for a language which was universal in two respects.
In the first respect the language would be universal in being understood by all, in the second in being suitable for all purposes and all subject matters.

The second problem of interest to both Leibniz and Carnap was the construction of formal languages in which knowledge could be expressed in such a way as to admit formal reasoning to derive logical consequences of the knowledge.
In this aspect of `language planning' Leibniz was a universalist, desiring a single formal calculus to encompass all human knowledge, whereas Carnap, though initially following Russell as a universalist, became a pluralist very early in his career, and did not so far as I am aware ever return to a universalist perspective.
\footnote{There is some point in going into this in greater detail, as to why Carnap abandoned his early universalist position and whether we should regard his reasons as compelling.}

Another point of difference between the two which may be worth mentioning at this point is the apparent purposes for formalisation.
A conspicuous difference is that Leibniz saw formalisation as a route to automation of reasoning, and expected automation of reasoning to yied a decision procedure for problems in relation to the body of formalised scientific knowledge (not as a way of determining the correctness of science).
Carnap, on the other hand,  does not seem to have shown much interest in this possible application of formal languages.
His interest was more in the use of formality to achieve the highest standards of rigour in science and scientific philosophy, in this and the next level of detail in what he conceived of as the contribution of philosophy Carnap's position is, as one might expect, positivistic.
Leibniz's program was more rationalistic, in his conception the role of formality was not so much in the establishment of scientific fact, but in its application.

My own position, in similar brevity, is in part a hybrid of the two.
I am not myself interested in contributing to the advancement of universal natural languages, but this does not quite mean that the problem of `language planning' is for me exclusively concerned with formal languages.
My own interest extending beyond the formal is in understanding the limits of formality, understanding what kinds of endeavour could be formalised, understanding when this is likely to be beneficial, understanding how we do, can or should deal with those kinds of knowledge or entrprise which are not suitable, how we might increase the scope of applicability of formal methods, how problems and projects might be migrated as they progress from the informal to the formal.

However, my primary interest here is properly within the domain of the formal, rather than in these matters concerning its relation to the informal.
In this I embrace the primary motivations both of Leibniz and of Carnap.
On the one hand in the prospect that formalisation will facilitate automation of reason, and the automated solution of problems on the basis of a formalisation of scientific and technological knowledge.
On the other hand in the potential of formalisation for increasing rigour both in science and in certain kinds of philosophy.
The enterprise as a whole is not one which can be expected to be realised by purely academic efforts.
There is potential for substantial economic benefit, at the cost of industrial scale development, and it is my inclination to invest my own rather small and relatively abstract contribution in ways which seem most relevant to delivering applications which will act as economic drivers.




%{\raggedright
%\bibliographystyle{klunum}
%\bibliography{rbjk}
%} %\raggedright

\twocolumn[\section{Index}\label{Index}]
{\small\printindex}

\end{article}
\end{document}
