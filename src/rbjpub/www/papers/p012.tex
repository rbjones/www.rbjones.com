% $Id: p012.tex,v 1.5 2011/02/12 09:14:19 rbj Exp $ bibref{rbjp012} pdfname{p012} 
\documentclass[numreferences]{rbjk}
\usepackage{makeidx}
\newcommand{\ignore}[1]{}
\usepackage[unicode,pdftex]{hyperref}
\usepackage[twoside,paperwidth=5.25in,paperheight=8in,hmargin={0.75in,0.5in},vmargin={0.5in,0.5in},includehead,includefoot]{geometry}
\hypersetup{pdfauthor={Roger Bishop Jones}}
\hypersetup{colorlinks=true, urlcolor=black, citecolor=black, filecolor=black, linkcolor=black}

%\newtheorem{def}{Definition}
%\newtheorem{conj}{Conjecture}

\makeindex
\begin{document}                                                                                   
\begin{article}
\begin{opening}  
\title{Language Planning and Design Automation}
\author{Roger Bishop \surname{Jones}}
\date{$ $\ $ $}
\runningauthor{Roger Bishop Jones}

\begin{abstract}
A development of aspects of the philosophies of Leibniz and Carnap in the direction of design automation.
\end{abstract}

\end{opening}

\vfill

\begin{centering}
\footnotesize{
Created 2006/12/23

Last Change $ $Date: 2011/02/12 09:14:19 $ $

\href{http://www.rbjones.com/rbjpub/www/papers/p012.pdf}{http://www.rbjones.com/rbjpub/www/papers/p012.pdf}

$ $Id: p012.tex,v 1.5 2011/02/12 09:14:19 rbj Exp $ $\\

}%footnotesize
\end{centering}

\newpage
%\def\tableofcontents{{\parskip=0pt\@starttoc{toc}}}
\setcounter{tocdepth}{4}
{\parskip-0pt\tableofcontents}

\section{Introduction}

Language planning is one of the headings under which Rudolf Carnap described his work in the ``intellectual biography''\cite{carnap63} which he wrote for the volume of ``The Library of Living Philosophers'' devoted to him\cite{carnap63a}.

In this section Carnap speaks of the two kinds of Universal Language which Leibniz had pursued, and the problem arising from his own linguistic pluralism (expressed in his ``principle of tolerance'').
Most of that material is devoted to universal natural languages, with little said about the problem more central to his own philosophical preoccupations of planning the diversity of formal languages necessary for the formalisation of science and for the conduct of philosophy along scientific lines.

Many other aspects of Carnap's work can be seen as relevant to this strategic problem of ``language planning'', such as his work on logical syntax and the later work on semantics, but on the general problem there does not seem to be a substantial body of work.

This document is not about Carnap or about Leibniz, it is about those aspects of my own thinking which might be seen as falling under a ``language planning'' header.
A substantial part of my own philosophical thinking and of my more detailed technical are intended to progress ideas similar to those of Leibniz and Carnap.
In brief this is the enterprise of formalisation and automation science, and such other aspects of our knowledge for which that may be beneficial, notably design automation.

My own interest dates several hundred years after Leibniz and half a century later than Carnap, and my own thinking therefore benefits from some acquaintance with a more fully developed field of mathematical logic and a lifetime's exposure to the digital computer some of that in the application of formal techniques to the development of digital hardware and software.
It is natural that my own perspective will be very different to that of Leibniz and Carnap, and there is therefore some point in sketching out the significance of `language planning' in my own systematic thinking.

\section{A Most General Sketch}

Both Leibniz and Carnap were interested in two most general conceptions of what Carnap thought of as `language planning'.

The first of these was the problem of devising a universal `natural' language.
In this context `natural' should be understood by contrast with `formal' rather than by contrast with `artificial' or `constructed'.
This is the problem of devising a language for use by people in writing and in speech.
The desire was for a language which was universal in two respects.
In the first respect the language would be universal in being understood by all, in the second in being suitable for all purposes and all subject matters.

The second problem of interest to both Leibniz and Carnap was the construction of formal languages in which knowledge could be expressed in such a way as to admit formal reasoning to derive logical consequences of the knowledge.
In this aspect of `language planning' Leibniz was a universalist, desiring a single formal calculus to encompass all human knowledge, whereas Carnap, though initially following Russell as a universalist, became a pluralist very early in his career, and did not so far as I am aware ever return to a universalist perspective.
\footnote{There is some point in going into this in greater detail, as to why Carnap abandoned his early universalist position and whether we should regard his reasons as compelling.}

Another point of difference between the two which may be worth mentioning at this point is the apparent purposes for formalisation.
A conspicuous difference is that Leibniz saw formalisation as a route to automation of reasoning, and expected automation of reasoning to yield a decision procedure for problems in relation to the body of formalised scientific knowledge (not as a way of determining the correctness of science).
Carnap, on the other hand,  does not seem to have shown much interest in this possible application of formal languages.
His interest was more in the use of formality to achieve the highest standards of rigour in science and scientific philosophy, in this and the next level of detail in what he conceived of as the contribution of philosophy Carnap's position is, as one might expect, positivistic.
Leibniz's program was more rationalistic, in his conception the role of formality was not so much in the establishment of scientific fact, but in its application.

My own position, in similar brevity, is in part a hybrid of the two.
I am not myself interested in contributing to the advancement of universal natural languages, but this does not quite mean that the problem of `language planning' is for me exclusively concerned with formal languages.
My own interest extending beyond the formal is in understanding the limits of formality, understanding what kinds of endeavour could be formalised, understanding when this is likely to be beneficial, understanding how we do, can or should deal with those kinds of knowledge or enterprise which are not suitable, how we might increase the scope of applicability of formal methods, how problems and projects might be migrated as they progress from the informal to the formal.

However, my primary interest here is properly within the domain of the formal, rather than in these matters concerning its relation to the informal.
In this I embrace the primary motivations both of Leibniz and of Carnap.
On the one hand in the prospect that formalisation will facilitate automation of reason, and the automated solution of problems on the basis of a formalisation of scientific and technological knowledge.
On the other hand in the potential of formalisation for increasing rigour both in science and in certain kinds of philosophy.
The enterprise as a whole is not one which can be expected to be realised by purely academic efforts.
There is potential for substantial economic benefit, at the cost of industrial scale development, and it is my inclination to invest my own rather small and relatively abstract contribution in ways which seem most relevant to delivering applications which will act as economic drivers.

In the following sections I will attempt:

\begin{enumerate}
\item a preliminary sketch, to provide a structure for consideration and presentation of the more detailed material.
\item following that structure some ideas on desiderata
\item some ideas on how the desiderata might be realised
\item a reconsideration of the nature of `language planning' and of how much of the material presented belongs properly belongs to that activity
\end{enumerate}

\section{A Preliminary Sketch}

It is my aim in this section to break down the problem of language planning into a hierarchy of sub-problems intended both to serve as structure for the remainder of this essay and as a classification for the kinds of related work which fall within the scope of language planning but do not constitute language planning.

A first coarse classification lies along an axis (not necessarily linear) at the extremes of which are the Yin/Yang of Dao and the Boolean dichotomies of modern formal logic.
Here are the areas which fall under this classification:

\begin{itemize}
\item[Yin/Yang] This is a domain in which the kinds of concepts involved do not seem to obey the law of non-contradiction, in which one may find a unity of opposites.
\item[Informal] This is the domain in which knowledge comes as prose in natural languages (or may not be written at all).
It is important to understand whether and how there can be knowledge which resists formalisation.
\item[Mathematical] This is a domain in which numerical or mathematical ideas are applicable, but are not applied formally.
Importantly the main thrust of design by simulation belongs here, and is an important preoccupation.
\item[Formal] This is the domain in which formal models are applicable (and most of my language planning ideas belong here).
The extent to which the formal can and should subsume the less formal design by simulation is of considerable interest.
\end{itemize}

At its most woolly language planning as I conceive it is concerned with the clarification or improvement of this classification, and with the migration of knowledge between these categories.

Rather than presuming, as positivists sometimes have, that no true knowledge is beyond the remit of formal languages, I am interested in understanding what kinds of knowledge are impossible or difficult to render formally, in how we may distinguish between what can be migrated across these categories and what in its essence might have to stay where it is.
This does make it sound like there is a presumption that migration in the direction of formality is desirable, which I don't hold.

Those four groupings can first be informally explicated as follows.

\paragraph{Yin/Yang}
It is perhaps natural to think of the oriental notion of Yin/Yang (which I believe originates in the ancient Chinese philosophy known as Dao) as the advocacy of balence, rather like a generalisation of the contemporary idea of a balenced diet, in all aspects of life.
One might think of this as a denial of false dichotomies, the idea that some choices should not be made but fudged, that one should not chose one thing or its opposite, but compromise with a blend of the two.
Taking one step further from such a denial of dichotomy, which is to deny that one must chose one or the other, one may in some important cases deny the apparent opposition, affirming their unity.

An example can be drawn from another concept of Dao, \emph{Wu-Wei} sometimes paraphrased `actionless action'.
Apart from the apparent contradiction in the phrase `actionless action' Wu-Wei may be thought of as unifying the opposites of spontaneity and self-discipline.
In relation to this I think of the art of a concert pianist such as Lang-Lang whose performances realise a spontaneous expressiveness which is possible only because of a technical facility attained through many years of dedication to mastery of his discipline.

The point here is, that it seems possible that some of the most profound insights into the human condition lead us into language which seems to fly in the face of the principle of contradiction.

\paragraph{Informal}

Empiricist philosophers such as Locke have emphasised that the mind works by association of ideas.
The idea of the mind as an associative engine predates the very modern notion of deductive reason (even supposing that to be coeval with descriptive language), and runs beyond the origin of our species back through evolution to the some of the earliest forms of life.
It is arguably the case that of the knowledge which enables us to function the great majority is in the form of experience acquired by an associative engine, and there is reason to doubt that such knowledge is an appropriate basis for deductive reasoning, or can beneficially be formalised.

Descriptive language, when made sufficiently precise, provides a basis for deductive reasoning, if knowledge of the world howsoever acquired can be rendered in descriptive language it becomes a candidate for migration either into (informal) mathematical models or into formal models.
In such cases it is the operation of the mind as an associative engine which results first in some kind of intuitive understanding of the phenomena and then in the formulation of scientific laws or models of the phenomena which in turn may be formalised.

\paragraph{Mathematical}

This classification concerns pre-formal mathematical modelling.





This `prelimary sketch' will be filled out in parallel with the subsequent sections.

\section{Desiderata}

The perhaps surprisingly broad top level classification in our sketch is intended to facilitate a discussion of what knowledge or possible knowledge there might be of importance which falls outside of the domain of the formalisable.
It may be that there is little to be said here about how one can approach those kinds of knowledge which could not be formalised, but I am unable to follow Leibniz and Carnap in the supposition that all knowledge, or even all scientific knowledge, must be formalisable, and it is therefore desirable not just to have some idea of how informal knowledge might be migrated into formality, but also to have some idea of what kinds of knowledge or domains of knowledge fall outside its scope.

The first desiderata it therefore some understanding of these different kinds of knowledge, either a refinement of the given classification or its replacement by something more appropriate.
In this we seek not merely a static characterisation, but an appreciation of the extent of misplacement and scope for migration.
In practice the contemporary domain of the formalised is minute, but those of us sympathetic to the ambitions of Leibniz and Carnap will hope that benefits may acrue from considerable expansion.

In talking of misplacement we are at first in need of criteria, and criteria may be principled or pragmatic.
The principles may tell us what kinds of knowledge may be dealt with in what ways, but grounds for migration, and grounds for investing in the means of migration, depend on prospective benefits of not entirely theoretical nature.

Leibniz had perhaps a greater sense of these pragmatic benefits, or at least was more substantially motivated by them, and in the 21st century, with information technology to hand, the prospect that formalisation will extend the scope and improve the quality of automation of economically significant intellectual industry.
From this point of view, we may look for an account at all levels which addresses the potential for and difficulty in realising economic benefits, and this is what I seek here.

A middle ground provides a third element to be sought, which corresponds better with the motivations of Russell and Carnap.
This middle ground between the purely theoretical question of what may kinds of knowledge may be dealt with, and the brutally pragmatic question of what economic benefits may accrue, is the matter of rigour, the prospect of eliminating error, at least in deduction.
This has been of particular interest in philosophy, if only to a minority of philosophers, for the sake of eliminating that most stark contrast between the growth and solidity of mathematical knowledge and the less durable products of deductive reason in philosophy.

Thus a central purpose, as I see it, of `language planning', is the question at this level of classification of kinds of language and their comparative merits.
In this my inclination is to work with classifications in which there is no dross.
The classification above is not intended to include categories exhibited for the sake of avoidance, I believe that each category has its particular strengths and weaknesses.
This is not to say that I endorse this as a status quo, clearly it is a central purpose of this work to facilitate migration toward formal treatment and greatly to expand what can be and what in fact is done by such means.

\subsection{In Relation to Formal Languages}

Let us now consider matter specific to the exploitation of formal systems.

The following matters deserve consideration:

\begin{itemize}
\item Universalism and Pluralism
\item 

\end{itemize}

\section{Ideas on Realising the Desiderata}


%{\raggedright
%\bibliographystyle{klunum}
%\bibliography{rbjk}
%} %\raggedright

\twocolumn[\section{Index}\label{Index}]
{\small\printindex}

\end{article}
\end{document}
