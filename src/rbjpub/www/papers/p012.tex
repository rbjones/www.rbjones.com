% $Id: p012.tex,v 1.1 2007/01/02 20:41:09 rbj01 Exp $ bibref{rbjp012} pdfname{p012} 
\documentclass[numreferences]{rbjk}
\usepackage{makeidx}
\newcommand{\ignore}[1]{}
\usepackage[unicode,pdftex]{hyperref}
\hypersetup{pdfauthor={Roger Bishop Jones}}
\hypersetup{colorlinks=true, urlcolor=black, citecolor=black, filecolor=black, linkcolor=black}

%\newtheorem{def}{Definition}
%\newtheorem{conj}{Conjecture}

\makeindex
\begin{document}                                                                                   
\begin{article}
\begin{opening}  
\title{X-Logic - realising the dream of Leibniz}
\runningtitle{X-Logic}
\author{Roger Bishop \surname{Jones}}
\date{$ $\ $ $}
\runningauthor{Roger Bishop Jones}

\begin{abstract}
A project aiming to contribute to the realisation of Leibniz's lingua characteristica calculus ratiocinator, making use of XML technologies where appropriate.
\end{abstract}

\end{opening}

\vfill

\begin{centering}
\footnotesize{
Created 2006/12/23

Last Change $ $Date: 2007/01/02 20:41:09 $ $

\href{http://www.rbjones.com/rbjpub/www/papers/p012.pdf}{http://www.rbjones.com/rbjpub/www/papers/p012.pdf}

$ $Id: p012.tex,v 1.1 2007/01/02 20:41:09 rbj01 Exp $ $\\

}%footnotesize
\end{centering}

\newpage
%\def\tableofcontents{{\parskip=0pt\@starttoc{toc}}}
\setcounter{tocdepth}{4}
{\parskip-0pt\tableofcontents}

\section{Introduction}

Leibniz conceived of the idea of:

\begin{itemize}
\item a lingua characteristica

which would be a universal language in which all problems could be expressed with sufficient precision that there might be a ...

\item a calculus ratiocinator

consisting of a method for calculating the answer to any question posed in the lingua characteristica

\end{itemize}

Leibniz was himself a philosopher, a mathematician, a logician, and an engineer (who designed mechanical calculators).
There are probably many more people today with similar interests than there ever have been, and some of these are today progressing objectives related to those ideas of Leibniz.

I belong to this loose community and this document sketches my own preferred approach to the realisation of Leibniz's dream.

For those readers inclined to think it naive today to imagine that these ideas could possibly be realised, (or even that it is now well established that they are not) I offer the reassurance that there will be some moderation of the ideas.

\section{Moderations}

Since the time of Leibniz many advances have been made in our knowledge of matters central to his enterprise, so that we can now see that Leibniz could not have hoped to come close to realising his objectives, and can see that we are now much better placed.

The most important of these advances are:
\begin{itemize}
\item the emergence of mathematical logic
\item the invention of the digital electronic computer
\end{itemize}

Mathematical logic emerged during the 19th Century, becoming an important new branch of mathematics which during the 20th century completely transformed and radically extended our knowledge of formal logic.

Mathematical logic anticipated in its theoretical advances the universal digital computer.
Universal digital computers, though conceived and designed earlier by Charles Babbage, depended upon the electronics for effective realisatoni.

Both these advances were essential to the realisation of anything close to Leibniz's project, so with hindsight we can see that the task he had set himself was to far ahead of the then state of the art in both logic and computation to be achievable in his lifetime.

Today these limitations are gone, and a great deal of energy has been applied to projects similar to the calculus ratiocinator, most notably in the related fields of ``artificial intelligence'' (the engineering side of cognitive science) and ``automation of reason''.
Success on the scale envisaged by Leibniz has been elusive.


%{\raggedright
%\bibliographystyle{klunum}
%\bibliography{rbjk}
%} %\raggedright

\twocolumn[\section{Index}\label{Index}]
{\small\printindex}

\end{article}
\end{document}
