% $Id: p011.tex,v 1.5 2012/03/23 15:41:40 rbj Exp $
% bibref{rbjp011} pdfname{p011} 
\documentclass{article}
\usepackage{makeidx}
\newcommand{\ignore}[1]{}
\usepackage[unicode]{hyperref}
\hypersetup{pdftitle={Non Well-Founded Foundational Ontologies}}
\hypersetup{pdfauthor={Roger Bishop Jones}}
\hypersetup{colorlinks=true, urlcolor=black, citecolor=black, filecolor=black, linkcolor=black}

%\newtheorem{def}{Definition}
%\newtheorem{conj}{Conjecture}

\makeindex
\newcommand{\indexentry}[2]{\item #1 #2}
\newcommand{\indexname}[0]{Index}

\begin{document}                                                                                   
\begin{article}
\begin{opening}  
\title{PolySets}
\subtitle{Non Well-Founded Foundational Ontologies}
\author{Roger Bishop \surname{Jones}}
\date{$ $Date: 2012/03/23 15:41:40 $ $}
\runningauthor{Roger Bishop Jones}

\begin{abstract}
An informal sketch of the development of my ideas about non well-founded ontologies for the foundations of mathematics.
\end{abstract}
\end{opening}

%\def\tableofcontents{{\parskip=0pt\@starttoc{toc}}}
\setcounter{tocdepth}{3}
{\parskip-0pt\tableofcontents}

\section{Introduction}

I have been interested in non-well founded foundation systems since approximately 1985 and have an intermittently progressing line of research in this topic, which is rather poorly documented.
This document began as an informal description of one of the systems I have looked at but which is now of purely historical interest.
It is now being expanded into a more broadly scoped account of the ideas.
It is my hope that the story will make it easier for interested parties (should there be any) to understand the approach which I am now progressing.

The material is presented as occurring in three phases.

The first spanned the period from 1985 through to about 2007.
There is no uniform tendency in this period but there are a small number of ideas which have subsequently proved important and so these are worth mentioning in their original context.


\section{Early Ideas}

The relevant history begins with my deciding in 1985 that a non-well founded system was desirable as the basis for a ``knowledge management'' system, and coming to a preliminary conclusion that an illative combinatory logic would be suitable, without then understanding the limitation of the known kinds of illative combinatory logic (of which the most important is that they are weak, and there is no obvious way to arbitrarily strengthen them. as there is for well-founded set theories through large cardinal axioms).
For this reason I found myself slipping from the attempt to \emph{chose} a foundation system into the desire to \emph{devise} one.


\subsection{First Thoughts about Combinatory Logic}

When I first decided that non well-founded foundations were a good idea I was thinking about ``knowledge management'' and the of the knowledge base as a pair the first element being a complex proposition in some formal language and the second a function which applies to the whole to perform both an inference on, and/or conservative extension to, the propositional element and some kind of transformation on the functional component.
Because of this \emph{reflexive} character I supposed that the logic involved would need to accomodate functions which are self applicable and hence do not have well-founded graphs.


\subsection{Creative Foundations}

\subsection{Well-founded Ontologies}



\subsection{Stealing Strength}

\section{Non-well Founded Set Theories}


\subsection{Polysets}

For the beginnings of a formal treatment of this topic see \cite{rbjt020}.

\subsubsection{Motivation}

The polysets are a conception of set which includes the pure well-founded collections (as many of them as you like) and also a similar number of non-well-founded collections.

The non-well-founded sets are designed to encompass two kinds of entity.

Firstly there are some functions similar to the well typed terms of the pure lambda calculus.
This is not of course the same as those of the simply typed lambda calculus, for these have well-founded graphs and are already available to us.
The special feature of the well-typed functions of the type-free calculus, is that the same function has many different types.
In the type free calculus we have an identity function which can be applied to anything,
In the typed calculus we have many identity functions, each defined only over some part of the universe.

Secondly there are the kinds of non-well-founded entities which would abound in category theory if only they existed.
Any kind of algebraic structure, like the groups, gives rise to a category which, because the objects in it are too diverse, is not well-founded.
There does not exist a category formed from all the groups, because there are groups over every set, and the category would therefore be too close for comfort to a universal set.
This is the kind of foundational problem which Saunders Mac Lane described at the beginning of his book ``Category Theory for the Working Mathematician''.
When Saunders Mac Lane came later to make definite foundational proposals they were completely irrelevant to this kind of problem, and so from the category theoretic point of view this foundational proposal is one which is intended to address Mac Lanes early foundational misgivings rather than those later ones (perhaps!) which led to his set theoretic ideas.

\subsubsection{Intuition}

The intuition about these kinds of structure which informs the notion of a \emph{polyset} comes from just a little knowledge of how functions are implemented in some polymorphically typed functional programming languages.
The situation is similar to that of type assignment in the pure type-free lambda calculus.
The type discipline is diagnostically and in other ways valuable, but a polymorphic function is just a single function, not a family of functions.
When we consider how it is possible for the same polymorphic function in these languages to apply to objects of different types, there is a simple answer.
This can be illustrated by the case of the polymorphic \emph{length} function over lists.
The reason why a single function can compute the length of a list without knowing the type of the values in the list, is that the values in the list are irrelevant to the length of the list and hence to the evaluation of the function.
The function needs to know how lists are structured.
It does not need to know anything about the structure of the members of the list.

Such functions examine only the superficial structure of their arguments, and by the time the computation reaches values whose type is not known (except as a type variable) they have already extracted all the information they need and probe no deeper.

In the case of computable functions, the superficial structure of arguments which will be used in a computation are finite.
In the case of mathematical functions more generally, this need not be the case, but the idea that a polymorphically-typable function depends only upon (some generous conception of) the superficial structure of its arguments provides us with an extension of the notion of a function which can be made to yield a model for a non-well-founded set theory in which such functions exist.

The more extended notion of superficiality is as follows.
A set is superficial in this way if every element in it is in it because it conforms to a ``pro-forma'', such that every other set conforming to that pro-forma is also a member.
A pro-forma is a set with some ``free variables'' in it, and a set conforms to that pro forma if it is an instance of it obtained by uniformly replacing the free variables by sets.
The most simple examples are firstly the universe V, which is obtained with a proforma consisting of the ordinal zero, and the identity function, for which the proforma would be the ordered pair of zero with itself.
These have extremely superficial structure, but structure may be arbitrarily deep and still be superficial in the required sense.
The distinction between superficial and non-superficial is very loosely analogous to that between set and class.
We may talk of a collection being a class if it is ``too large'' to be a set, even though there is no limit to how large a set might be.
A class is perhaps too large to have a size.
A better analogy is perhaps with the distinction between well-founded and non-well-founded.
Superficial structure is always well-founded, there is a bound on the length of the descending paths through the superficial structure.

\subsubsection{Construction}
To make this idea more precise I proceed as follows.

I construct representatives of the polysets in a well-founded set theory, then define a new membership relation over these representatives reflecting the intended extension of the polysets, finally this relation is lifted to operate over sets of the original representatives (unit sets in the first instance) and extensionalise the relationship by taking the smallest equvalence relation over the representation relative to which the lifted membership relation is extensional.

\paragraph{Version 1}

The representatives of the polysets are defined by transfinite recursion, relative to some standard well-founded set theory.
The definition is done in three stages, first sero, then the Von Neumann ordinals and then the rest.

\begin{quote}\label{def:polyset rep}
{\it
The polyset rep of the empty set is the empty set.
}
\end{quote}
\index{polyset rep}

\begin{quote}
\emph{
The polyset rep of a non-zero ordinal is the ordered pair with the empty set on the left and and the set of polyset reps of its members on the right.
}
\end{quote}

The polyset reps of the ordinals are called the polyset ordinal reps.

A polyset membership relationship is defined over the polyset ordinal reps as follows:

\begin{quote}
\emph{
The polyset ordinal rep is a polyset member of some other polyset ordinal rep if it is a member of the set on the right of the ordered pair.
}
\end{quote}

The polyset reps are then defined:

\begin{quote}
\emph{
A polyset rep is an ordered pair of which the first element is a polyset ordinal rep and the second a set of polyset reps.
}
\end{quote}

Using the terms ``lhs'' and ``rhs'' respectively for the left and right hand members of an ordered pair, the membership relationship is then defined over the polyset reps as follows:

\begin{quotation}
{\it
A polyset rep A is a member of a polyset rep B if there exists:
\begin{enumerate}
\item a function \textsf{subs} defined over the (poly-)members of the polyset ordinal rep \textsf{lhs B} with values in the polyset reps
\item a member \textsf{m} of \textsf{rhs B}
\end{enumerate}
such that A is obtained by instantiating \textsf{m} regarded as a proforma with the values for free variables determined by \textsf{subs} (the occurrences in \textsf{m} of ordinals in the domain of \textsf{subs} are considered to be free variables).
}
\end{quotation}

If the ordinal \textsf{lhs B} is zero, then this indicates that the proformas in \textsf{rhs B} have no free variables.
Only the empty substitution may be applied, and \textsf{A} must then be a member of \textsf{rhs B} to meet the polyset membership requirement.

The details of how to instantiate a pro-forma are of course crucial here and this is defined by transfinite recursion as follows.

\begin{quote}
{\it
The instance of a polyset rep \textsf{A} resulting from a substitution \textsf{subs} is:
\begin{itemize}

\item if \textsf{A} is a polyset ordinal rep which is in \textsf{dom(subs)} then the value of the instance is the value of \textsf{subs} at \textsf{A}.

\item if \textsf{A} is a polyset ordinal rep which is not in \textsf{dom(subs)} and \textsf{B} is the a polyset ordinal rep such that \textsf{dom(Subs) + B = A} then the value of the instance is \textsf{B}.

\item if \textsf{A} is not a polyset ordinal rep, but is the ordered pair \textsf{(C, D)} where \textsf{C} is a polyset ordinal rep and \textsf{D} is a set of polyset reps, then the instance is the ordered pair \textsf{(C,E)} where E is the set of polyset reps obtained by applying substition \textsf{subs} to each of the members of \textsf{D}.
\end{itemize}.
}
\end{quote}

\paragraph{Version 2}

I define an interpretation of the first order language of set theory by defining within a well-founded (extensional) membership structure \textsf{(WF,R)} a new structure \textsf{(PS,R')} where \textsf{PS} is a subset of \textsf{WF}, and \textsf{R'} is a non-well-founded (and non-extensional) relation over \textsf{PS}. From this an extensional structure is obtained by taking a quotient.

\textsf{PS}, the set of polysets, is defined in two stages.
It will contain an isomorphic image of the original membership structure the members of which are called the ``hereditarily low'' polysets.
It is convenient to define the injection from \textsf{WF} to these polysets first,

The image of the empty set is the empty set.
The image of every other well-founded set is an ordered pair with the empty set on the left and the set of images of the elements of the set on the right.
The restriction of \textsf{R'} to the hereditarily low polysets is that induced by this injection so that they form an isomorphic copy of \textsf{(WF,R)} in \textsf{(PS,R')}.
The restriction of \textsf{(PS,R')} to these sets is well-founded and extensional, and is preserved unchanged when quotients are taken to recover extensionality across the whole of \textsf{R'}.

The \emph{polyset ordinals} are the polysets which are in the image of the Von-Neumann ordinals under this injection.

The polysets are then the well-founded sets which are hereditarily either the empty set or an ordered pair with a polyset ordinal on the left and a set of polysets on the right.

\textsf{R'} is then defined:

\textsf{R' x y} iff y is the ordered pair \textsf{(n,s)} and:

\begin{enumerate}
\item n is the empty set (the zero polyset ordinal) and \textsf{R x s} 

or
\item n is not the empty set and there exists an assignment of polysets to the ordinals below n and a member of s such that when s (taken as a pro-forma over n-1 variables) is instantiated using the assignment, the result is x 
\end{enumerate}

When a poly set is ``taken as a proforma over n-1 variables'' then occurrences in it of the first n-1 ordinals are taken as free variables.
When an instantiation takes place according to some assignment, occurrences of the free variables are replaced by their assigned values, and the number of free variables is deducted from the value of any other occurrences of ordinals (i.e. of ordinals not less than n).

Lets try that again!

Instantiation of a polyset according to a variable assignment is defined as follows.

The instance of a polyset \emph{s} according to an assignment \emph{a} is:
\begin{itemize}
\item if \emph{s} is a polyset ordinal less than the number of variables in the assignment then the value assigned to that variable (\emph{s-o}) is the result of the instantiation
\item if \emph{s} is a polyset ordinal greater than the number of variables then its value is decreased by the number of variables.
\item if \emph{s} is not a polyset ordinal but is $(n,t)$ then its value after instantiation is $(n,t')$ where $t'$ is the set of instances of members of $t$ according to assignment $a$.
\end{itemize}

We only ever instantiate a PolySet with an assignment whose domain is the left hand element of the polyset,

\paragraph{Version 3}

$(x,y)$ is the ordered pair of $x$ and $y$.

Let $WF$ be $V(\alpha)$ for some reasonably large ordinal $\alpha$ (say, a Mahlo cardinal).

An injection from $WF$ into the PolySets is defined thus:

\begin{displaymath}
ps(\{\}) = \{\}
\end{displaymath}
for non-empty \emph{x}:
\begin{displaymath}
ps(x) = (\{\}, \{ ps(y)\ |\ y \in x\})
\end{displaymath}

Let $PsOn$ be the image under $ps$ of the ordinals in $WF$:

\begin{displaymath}
PsOn = ps `` Ord
\end{displaymath}

The $PolySets$ are then:

\begin{displaymath}
PolySets = \{(o,s)\ |\ o \in PsOn \land s \in WF \land s \subseteq PolySets\}
\end{displaymath}

Membership over the $PolySets$ is defined using a substitution operation.
A $PolySet$ assignment to a PsOn $Po$ is a function with domain $Po$ (a PolySet of $PolySet$ ordinals) with $PolySet$ values.
An instance of a PolySet $(Po, Pss)$ may be obtained by instantiating the PolySet according to an assignment of PolySets to $Po$.

Instantiation is defined as follows, in which `$+$' is $ordinal$ addition.

\begin{enumerate}
\item
If $ps \in domain(va)$ then:
\begin{displaymath}
 inst va ps = va (ps)
\end{displaymath}
\item
If $ps \in PsOn\ \verb!\!\ domain(va)$
\begin{displaymath}
 inst\ va\ ps\ =\ the\ ps'\ s.t.\ dom(va)+ps'=ps
\end{displaymath}
\item
\begin{displaymath}
inst\ va\ (o,pss)\ =\ (o,\ \{inst\ va\ psse\ |\ psse\ \in\ pss\})
\end{displaymath} 
\end{enumerate}

Membership over PolySets is then defined:
\begin{displaymath}
ps\ \not\in\ \{\}
\end{displaymath}
\begin{displaymath}
ps\ \in\ \{o,pss\}\\
\Leftrightarrow\\
\exists va,\ ps2.\ domain\ va\ =\ o\ \wedge\ ps\ =\ inst\ va\ ps2
\end{displaymath}

\subsubsection{Extensionality}

We finally lift the membership operation over equivalence classes of polyset reps, taking the smallest equivalence relation such that the membership relation induced on the equivalence classes is extensional.
The induced relation is: [A] is a member of [B] if there exist C and D such that $C \backsimeq A$ and $D \backsimeq B$ and C is a (polyset) member of D.

\subsubsection{Properties}

In the following discussion of the properties of this structure I use the term {\it mono} or {\it low} for a poly set with a representative whose lhs is the empty set, poly for other polyset reps, and WF or H$_{low}$ for the hereditarily mono polyset reps.

It is not my intention that the polysets be the subject of a first order theory, in the manner of ZFC or NF.
It is intended that they form a stage in a series of constructions which ends in some kind of type theory.

The key distinguishing feature is polyset abstraction, which would I think be difficult to formalise conveniently in a first order language, but can be formalised in a higher order logic.
In either case its probably not an easy axiom to use, so the aim ultimately would be to have a type system designed so that well-typeable functional abstractions in that type theory always yield polysets.

The discussion falls into four parts.
\begin{enumerate}
\item well-founded and other mono polysets.
\item poly polysets for ML-like polymorphic functions
\item poly polysets for locales and abstract algebra
\item comparison with other non-well-founded set theories
\end{enumerate}

\subsubsection{Mono PolySets}

The well-founded PolySets will I hope be isomorphic to the original well-founded structure.

I am inclined to work with systems with `galaxies' (aka universes) which are closed under power set union and replacement, and these should allow traditional mathematical domains to be treated in the usual way.
If there are galaxies then the poly PolySets will also be members of galaxies though the closure properties relate only to the mono constituents of the galaxies.

\subsubsection{ML-like Polymorphism}

The main motivation and intuition behind this construction comes from ML.
The idea is to produce an ontology which has all the classical well-founded sets together with the polymorphic functions definable in a language like ML in such a way that one can instantiate local definitions of polymorphic functions (or in which instantiation of polymorphic functions in not required).

I am still some way off having any arguments to support this intuition, but my present view is that the interpretation will achieve this objective.

\subsubsection{Locales and Abstract Algebra}

I had also hoped that the interpretation would be good for the kind of structuring of formal theories which is desirable for doing and applying abstract algebra, or more generally for good structuring and reusability in large scale formal mathematics.

I think the onology is unlikely to satisfy this requirement.

The desire here is to be able to give, not just local definitions of polymorphic functions, but also local specifications supplying a context in which the theory is to be interpreted,
This might include for example stating or importing the concept of a group and expecting to be able to make multiple applications of the concept in the body of the specification.

It now seems unlikely to me that abstract algebraic structures will be PolySets.
It is possible that further filling out of the ontology, maybe with Church-Oswald constructions, might realise this kind of ambition, but I don't have a good intuition about what kind of filling out would to the trick.

It is worth asking, given that the ontology seems too skinny, whether this ontology is encompassed by any other set theory, NF or `positive' set theory.

The conjectures are based on the hunch that the final stage in the process described above, in which the relation is extensionalised, really doesn't do much else, and that the closure properties of the non-extensional version are retained.

The term ``set'' refers to something in the domain of the original well-founded membership structure, the term ``polyset'' is used for the sets in the constructed non-well-founded membership structure.
The term ``polyset rep'' refers to members of the domain of the non-extensional non-well-founded membership structure over which the ``polyset'' are a quotient. 

The terminology of ``low'' and ``high'' (poly-)sets associated with Church-Oswald constructions (as in \cite{forster92,forster2005}) is used, and (the closest analogue of) Forster's definition of low is adopted.
A polyset is low if it is the empty set or the lhs of (any one of) its ordered pair representatives is the empty set.
Otherwise it is high.

\subsubsection{Higher Order Axioms}

It is natural to expect that this set theory can be axiomatised with three axioms as follows.
There are two primitive constants, a binary membership relation and the predicate ``Low'' (whose negation is ``High'').

\begin{enumerate}
\item Full extensionality.

\item Every set is a member of a Galaxy.

A galaxy is a set which is closed under:

\begin{itemize}
\item [full low power set]\ 

All the subsets of a low set are low sets and are collected in a low set.

\item [low sumset]\ 

All low sets have sumsets, low if all the members of the set are low, otherwise high.

\item [low replacement]\ 

The image of a low set under a functional relation is a low set.

\end{itemize}

\item polyset abstraction.

Any set together with an ordinal determines a high set whose members are those sets which can be obtained by instantiating a member of the first set by a family of sets indexed by the ordinal.

\end{enumerate}

\subsubsection{Other Properties}

\begin{itemize}

\item[No gratuitous failures of $\in$ foundation]\ 

This heading comes from Forster's paper \cite{forster2006} and I use it here because we seem to have extreme versions of some of the characteristics he is considering there.

All high polysets have the same size and are larger than any low polyset.
The only self-membered set is V.
All $\in$ loops involve at least one high polyset.
$\in$ restricted to sets smaller than V (i.e. to low sets) is well-founded.

\item[Properties of CO constructions]\ 

The polysets are not obtained using a Church-Oswald construction (see: \cite{forster2005}) though perhaps they could have been.
They seem to have several of the properties of CO constructions proven by Forster in \cite{forster2005}.
Many of these are obvious consequences of properties already stated,

\begin{itemize}

\item[Low] Forsters definition of a ``low'' set is one for which k=0.
In our case the ordered pairs are the other way round, but with that adjustment the definition is good.
The low sets are the special case of PolySets in which there are no pattern variables.

\item[H$_{low}$]\ 

I suspect that H$_{low}$ (as defined by Forster\cite{forster2005}) is (here) the set of hereditarily low sets.

\item[Low Comprehension]\ 

There will be something analogous to low comprehension.
Any set of polysets is a low polyset.
High polysets are all classes of polysets.
(interpret ``set'' and ``class'' here as if in a set theoretic metalanguage)

\item[12.] The set of low polysets is not a polyset (not even a high one).

\item[13.] An image of a low polyset is low, subsets of low polysets are low.

\item[14.] A low polyset has a low power set.

\item[15.] Low polysets have sumsets. Low polysets of low polysets have low sumsets.

\item[30.] H$_{low}$ is isomorphic to the original universe assuming that the original was V($\alpha$) for some $\alpha$, probably much weaker assumptions would suffice.

\item[32.] a canonical injection from the original universe is readily definable, but not in quite the same way as for a Church-Oswald construction.

\item[34.] The range of the canonical injection from the well-founded sets into the (hereditarily low) polysets is not a polyset (or a set).

\item[37.] 

\end{itemize}
\end{itemize}

\subsection{A General Method}

There is a general method for the construction of models of set theories which can be extracted from the construction of the polysets and which can be seen to be related to but more general than the method called by Forster the ``Church-Oswald'' method.
It was not conceived of as an elaboration of the Church-Oswald construction, and it has a different character.
My own awareness of the connection arose subsequent to the Polyset construction when I was trying to relate that idea to previous methods for constructing models, by reference to \cite{forster92}.

The general method which is exemplified by the polyset construction is that of devising an infinitary notation for sets using something like an abstract syntax coded in set theory and giving a semantics to that notation using an inductive definition of the membership relationship o over that syntax.
This does not require that membership be well-founded, only that the definition of the membership relationship be well-founded.
It is not required that the notation for sets be \emph{canonical} in the sense that each set is represented by only one syntactic expression, the intention is that the elements of the domain will be equivalence classes of representatives.
So there is an obligation to prove that the obvious way of obtaining the required equivalence relation gives a non-trivial result.
This is to be expected because the heirarchy of representatives of pure well=founded sets will not collapse since no two distinct such sets will ever appear to be equivalent.

\subsection{Infinitary Comprehension}

I was not very far into the Polysets before it became clear that there would not be enough of them to provide a good foundation for mathematics (though there are inaccessibly many, so this is about the qualities rather than the quantities).
However, when I thought about how to improve the closure properties, it seemed probably that merely adding complements would effectively ytield full comprehension, and hence would not deliver a consistent set theory.

I thus faced the difficulty of deciding on an extension to the polysets which would be stronger but still coherent.
Potentially this would involve several attempts, in each case involving defining a new syntax of representatives, defining membership over that syntax and then proving that the collapse to give extensionality yields a satisfactory result.

The method does not make a significant contribution to the resolution of the problem, the crux of which is to identify which sets exist in a proposed model.
I don't think at this stage I perceived how little the method achieved, but it occurred to me that parts of this process involved in checking out a model could be done once-and-forall.
If we work with the idea that sets are the extensions of concepts, and hence that every set corresponds to some formula, having as members just those sets which satisfy the formula, then we can devise a single notation which will embrace all the possibilities, and we could formulate the semantics relative to some given domain (which would be a subset of the representatives).
To ensure that all the pure well-founded sets are represented (up to some chosen large cardinal) an infinitary notation is used.
After several attempts at this I arrived at \cite{rbjt026}, in which first base in establishing the properties of the semantics fails, for reasons not yet understood.
This is an attempted proof of the conjecture that any total least fixed point of the semantic functor (over some subclass of the infinitary formulae) yields an extensional interpretation of the first order language of set theory.

I still do not know whether this conjecture is true.
When I later returned to the problem I decided that two difficulties in obtaining non well-founded set theories by this method would disappear if an analogous method were applied directly to an illative lambda calculus instead of attempting to obtain a set theory first.
Those two problems relate to junk and extensionality.
The ontology of a set theory must be well-behaved in ways which are not necessary in a lambda calculus or a combinatory logic.
A set must have a definite membership and must be extensional.
To achieve this it is necessary to select a suitable subset of the representatives over which domain the semantic functor has a total fixed point which is extensional.
If we work with combinators we do not need to insist on analogous conditions, the need to identify a suitable subset of the domain is obviated, we can just take them all.

\section{Illative Lambda Calculi}

Though I am looking for a lambda calculus, the underlying ontology might nevertheless just be combinators.

{\raggedright
\bibliographystyle{rbjfmu}
\bibliography{rbj}
} %\raggedright

\twocolumn[\section{Index}\label{Index}]
{\small\printindex}

\end{article}
\end{document}
