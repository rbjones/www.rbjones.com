% $Id: p011.tex,v 1.1 2006/11/09 12:05:24 rbj01 Exp $
% bibref{rbjp011} pdfname{p011} 
\documentclass[numreferences]{rbjk}
\usepackage{makeidx}
\newcommand{\ignore}[1]{}
\usepackage[unicode,pdftex]{hyperref}
\hypersetup{pdfauthor={Roger Bishop Jones}}
\hypersetup{colorlinks=true, urlcolor=black, citecolor=black, filecolor=black, linkcolor=black}

%\newtheorem{def}{Definition}
%\newtheorem{conj}{Conjecture}

\makeindex
\begin{document}                                                                                   
\begin{article}
\begin{opening}  
\title{Poly-Sets}
\runningtitle{Poly-Sets}
\author{Roger Bishop \surname{Jones}}
\date{$ $Date: 2006/11/09 12:05:24 $ $}
\runningauthor{Roger Bishop Jones}

\begin{abstract}
A concise presentation of the motivation for, intuition behind, construction of, and conjectured properties of a model for a set theory with a universal set.
\end{abstract}
\end{opening}

%\def\tableofcontents{{\parskip=0pt\@starttoc{toc}}}
\setcounter{tocdepth}{4}
{\parskip-0pt\tableofcontents}

\section{Introduction}

For a formal treatment of this topic see \cite{rbjt020}.

\subsection{Motivation}

The poly-sets are a conception of set which includes the pure well-founded collections (as many of them as you like) and also a similar number of non-well-founded collections.

The non-well-founded sets are designed to encompass two kinds of entity.

Firstly there are some functions similar to the well typed terms of the pure lambda calculus.
This is not of course the same as those of the simply typed lambda calculus, for these have well-founded graphs and are already available to us.
The special feature of the well-typed functions of the type-free calculus, is that the same function has many different types.
In the type free calculus we have an identity function which can be applied to anything,
In the typed calculus we have many identity functions, each defined only over some part of the universe.

Secondly there are the kinds of non-well-founded entities which would abound in category theory if only they existed.
Any kind of algebraic structure, like the groups, gives rise to a category which, because the objects in it are too diverse, is not well-founded.
There does not exist a category formed from all the groups, because there are groups over every set, and the category would therefore be too close for comfort to a universal set.
This is the kind of foundational problem which Saunders Mac Lane described at the beginning of his book ``Category Theory for the Working Mathematician''.
When Saunders Mac Lane came later to make definite foundational proposals they were completely irrelevant to this kind of problem, and so from the category theoretic point of view this foundational proposal is one which is intended to address Mac Lanes early foundational misgivings rather than those later ones (perhaps!) which led to his set theoretic ideas.

\subsection{Intuition}

The intuition about these kinds of structure which informs the notion of a \emph{poly-set} comes from just a little knowledge of how functions are implemented in some polymorphically typed functional programming languages.
The situation is similar to that of type assignment in the pure type-free lambda calculus.
The type discipline is diagnostically and in other ways valuable, but a polymorphic function is just a single function, not a family of functions.
When we consider how it is possible for the same polymorphic function in these languages to apply to objects of different types, there is a simple answer.
This can be illustrated by the case of the polymorphic \emph{length} function over lists.
The reason why a single function can compute the length of a list without knowing the type of the values in the list, is that the values in the list are irrelevant to the length of the list and hence to the evaluation of the function.
The function needs to know how lists are structured.
It does not need to know anything about the structure of the members of the list.

Such functions examine only the superficial structure of their arguments, and by the time the computation reaches values whose type is not known (except as a type variable) they have already extracted all the information they need and probe no deeper.

In the case of computable functions, the superficial structure of arguments which will be used in a computation are finite.
In the case of mathematical functions more generally, this need not be the case, but the idea that a polymorphically-typable function depends only upon (some generous conception of) the superficial structure of its arguments provides us with an extension of the notion of a function which can be made to yield a model for a non-well-founded set theory in which such functions exist.

The more extended notion of superficiality is as follows.
A set is superficial in this way if every element in it is in it because it conforms to a ``pro-forma'', such that every other set conforming to that pro-forma is also a member.
A pro-forma is a set with some ``free variables'' in it, and a set conforms to that pro forma if it is an instance of it obtained by uniformly replacing the free variables by sets.
The most simple examples are firstly the universe V, which is obtained with a proforma consisting of the ordinal zero, and the identity function, for which the proforma would be the ordered pair of zero with itself.
These have extremely superficial structure, but structure may be arbitrarily deep and still be superficial in the required sense.
The distinction between superficial and non-superficial is very loosely analogous to that between set and class.
We may talk of a collection being a class if it is ``too large'' to be a set, even though there is no limit to how large a set might be.
A class is perhaps too large to have a size.
A better analogy is perhaps with the distinction between well-founded and non-well-founded.
Superficial structure is always well-founded, there is a bound on the length of the descending paths through the superficial structure.

\section{Construction}
To make this idea more precise I proceed as follows.

I construct representatives of the poly-sets in a well-founded set theory, then define a new membership relation over these representatives reflecting the intended extension of the poly-sets, finally this relation is lifted to operate over sets of the original representatives (unit sets in the first instance) and extensionalise the relationship by taking the smallest equvalence relation over the representation relative to which the lifted membership relation is extensional.

\subsection{Version 1}

The representatives of the poly-sets are defined by transfinite recursion, relative to some standard well-founded set theory.
The definition is done in three stages, first sero, then the Von Neumann ordinals and then the rest.

\begin{quote}\label{def:poly-set rep}
{\it
The poly-set rep of the empty set is the empty set.
}
\end{quote}
\index{poly-set rep}

\begin{quote}
\emph{
The poly-set rep of a non-zero ordinal is the ordered pair with the empty set on the left and and the set of poly-set reps of its members on the right.
}
\end{quote}

The poly-set reps of the ordinals are called the poly-set ordinal reps.

A poly-set membership relationship is defined over the poly-set ordinal reps as follows:

\begin{quote}
\emph{
The poly-set ordinal rep is a poly-set member of some other poly-set ordinal rep if it is a member of the set on the right of the ordered pair.
}
\end{quote}

The poly-set reps are then defined:

\begin{quote}
{\it
A poly-set rep is an ordered pair of which the left hand is a poly-set ordinal rep and the right hand is a set of poly-set reps.
}
\end{quote}

Using the terms ``lhs'' and ``rhs'' respectively for the left and right hand members of an ordered pair, the membership relationship is then defined over the poly-set reps as follows:

\begin{quotation}
{\it
A poly-set rep A is a member of a poly-set rep B if there exists:
\begin{enumerate}
\item a function \textsf{subs} defined over the (poly-)members of the poly-set ordinal rep \textsf{lhs B} with values in the poly-set reps
\item a member \textsf{m} of \textsf{rhs B}
\end{enumerate}
such that A is obtained by instantiating \textsf{m} regarded as a proforma with the values for free variables determined by \textsf{subs} (the occurrences in \textsf{m} of ordinals in the domain of \textsf{subs} are considered to be free variables).
}
\end{quotation}

If the ordinal \textsf{lhs B} is zero, then this indicates that the proformas in \textsf{rhs B} have no free variables.
Only the empty substitution may be applied, and \textsf{A} must then be a member of \textsf{rhs B} to meet the poly-set membership requirement.

The details of how to instantiate a pro-forma are of course crucial here and this is defined more carefully, by transfinite recursion as follows.

\begin{quote}
{\it
The instance of a poly-set rep \textsf{A} resulting from a substitution \textsf{subs} is:
\begin{itemize}

\item if \textsf{A} is a poly-set ordinal rep which is in \textsf{dom(subs)} then the value of the instance is the value of \textsf{subs} at \textsf{A}.

\item if \textsf{A} is a poly-set ordinal rep which is not in \textsf{dom(subs)} and \textsf{B} is the a poly-set ordinal rep such that \textsf{dom(Subs) + B = A} then the value of the instance is \textsf{B}.

\item if \textsf{A} is not a poly-set ordinal rep, but is the ordered pair \textsf{(C, D)} where \textsf{C} is a poly-set ordinal rep and \textsf{D} is a set of poly-set reps, then the instance is the ordered pair \textsf{(C,E)} where E is the set of poly-set reps obtained by applying substition \textsf{subs} to each of the members of \textsf{D}.
\end{itemize}.
}
\end{quote}

\subsection{Version 2}

Version 1 seemed a bit tortuous, so I thought I'd try a more concise statement.

I define a model for ``poly-set'' theory by defining within a well-founded (extensional) membership structure \textsf{(WF,R)} a new structure \textsf{(PS,R')} where \textsf{PS} is a subset of \textsf{WF}, and \textsf{R'} is a non-well-founded (and non-extensional) relation over \textsf{PS}. From this an extensional structure is obtained by taking a quotient.

\textsf{PS}, the set of poly-sets, is defined in two stages.
It will contain an isomorphic image of the original membership structure the members of which are called the ``hereditarily low'' poly-sets.
It is convenient to define the injection from \textsf{WF} to these poly-sets first,

The image of the empty set is the empty set.
The image of every other well-founded set is an ordered pair with the empty set on the left and the set of images of the elements of the set on the right.
The restriction of \textsf{R'} to the hereditarily low poly-sets is that induced by this injection so that they form an isomorphic copy of \textsf{(WF,R)} in \textsf{(PS,R')}.
The restriction of \textsf{(PS,R')} to these sets is well-founded and extensional, and is preserved unchanged when quotients are taken to recover extensionality across the whole of \textsf{R'}.

The \emph{poly-set ordinals} are the poly-sets which are in the image of the Von-Neumann ordinals under this injection.

The poly-sets are then the well-founded sets which are hereditarily either the empty set or an ordered pair with a poly-set ordinal on the left and a set of poly-sets on the right.

\textsf{R'} is then defined:

\textsf{R' x y} iff y is the ordered pair \textsf{(n,s)} and:

\begin{enumerate}
\item n is the empty set (the zero poly-set ordinal) and \textsf{R x s} 

or
\item n is not the empty set and there exist an assignment of poly-sets to the ordinals below n and a member of s such that when s (taken as a pro-forma over n-1 variables) is instantiated using the assignment, the result is x 
\end{enumerate}

When a poly set is ``taken as a proforma over n-1 variables'' then occurrences in it of the first n-1 ordinals are taken as free variables.
When an instantiation takes place according to some assignment, occurrences of the free variables are replaced by their assigned values, and the number of free variables is deducted from the value of any other occurrences of ordinals (i.e. of ordinals not less than n).

Lets try that again!

Instantiation of a poly-set according to a variable assignment is defined as follows.

The instance of a poly-set \emph{s} according to an assignment \emph{a} with an offset (count of bound variables) \emph{o} is:

The instance of a poly-set \emph{s} according to an assignment \emph{a} is:
\begin{itemize}
\item if \emph{s} is a poly-set ordinal less than the number of variables in the assignment then the value assigned to that variable (\emph{s-o}) is the result of the instantiation
\item if \emph{s} is a poly-set ordinal greater than the number of variables then its value is decreased by the number of variables.
\item if \emph{s} is not a poly-set ordinal but is $(n,t)$ then its value after instantiation is $(n,t')$ where $t'$ is the set of instances of members of $t$ according to assignment $a$.
\end{itemize}

We only ever instantiate a polyset with an assignment whose domain is the left hand element of the poly-set,

\subsection{Extensionality}

We next lift the membership operation to operate over equivalence classes of poly-set reps, and take the smallest equivalence relation such the membership relation induced on the equivalence sets is extensional.
The induced relation is: [A] is a member of [B] if there exist C and D such that $C \backsimeq A$ and $D \backsimeq B$ and C is a member of D.

In the following discussion of the properties of this structure I use the term {\it mon} for a poly set with a representative whose lhs is the empty set, poly for other poly-set reps, and WF for the hereditarily mono poly-set reps.

The characteristics of the resulting membership structure depend upon those of the original.
Certain minimal conditions are necessary for the construction to yield an extensional membership relation, and these give some further properties of that relation.
Thereafter the properties of the resulting membership relation depend on the properties of the original, in particularly 

\begin{enumerate}
\item The restriction to the hereditarily mono poly-sets is isomorphic to the initial well-founded set theory.
\item 
\end{enumerate}

\section{Properties}

It is not my intention that the poly-sets be the subject of a first order theory, in the manner of ZFC or NF.
It is intended that they form a stage in a series of constructions which ends in some kind of type theory.

The key distinguishing feature is poly-set abstraction, which would I think be difficult (though not impossible) to formalise in a first order language, but can be formalised in a higher order logic.
In either case its probably not an easy axiom to use, so the aim ultimately would be to have a type system designed so that well-typeable functional abstractions in that type theory always yield poly-sets.

The conjectures are based on the hunch that the final stage in the process described above, in which the relation is extensionalised, really doesn't do much else, and that the closure properties of the non-extensional version are retained.
I also assume that the original well-founded relation on which the construction is based is standard (full power sets) and tall enough that every set is a member of a set closed under union, power set and (higher order) replacement.

The term ``set'' refers to something in the domain of the original well-founded membership structure, the term ``poly-set'' is used for the sets in the constructed non-well-founded membership structure.
The term ``poly-set rep'' refers to members of the domain of the non-extensional non-well-founded membership structure over which the ``poly-set'' are a quotient. 

The terminology of ``low'' and ``high'' (poly-)sets associated with Church-Oswald constructions (as in \cite{forster92,forster2005}) is used, and (the closest analogue of) Forster's definition of low is adopted.
A poly-set is low if it is the empty set or the lhs of (any one of) its ordered pair representatives is the empty set.
Otherwise it is high.

\subsection{Higher Order Axioms}

It is natural to expect that this set theory can be axiomatised with three axioms as follows.
There are two primitive constants, a binary membership relation and the predicate ``Low'' (whose negation is ``High'').

\begin{enumerate}
\item Full extensionality.

\item Every set is a member of a Galaxy.

A galaxy is a set which is closed under:

\begin{itemize}
\item [full low power set]\ 

All the subsets of a low set are low sets and are collected in a low set.

\item [low sumset]\ 

All low sets have sumsets, low if all the members of the set are low, otherwise high.

\item [low replacement]\ 

The image of a low set under a functional relation is a low set.

\end{itemize}

\item Poly-set abstraction.

Any set together with an ordinal determine a high set whose members are those sets which can be obtained by instantiating a member of the first set by a family of sets indexed by the ordinal.

\end{enumerate}

\subsection{Other Properties}

\begin{itemize}

\item[No gratuitous failures of $\in$ foundation]\ 

This heading comes from Forster's paper \cite{forster2006} and I use it here because we seem to have extreme versions of some of the characteristics he is considering there.

All high poly-sets have the same size and are larger than any low poly-set.
The only self-membered set is V.
All $\in$ loops involve at least one high poly-set.
$\in$ restricted to sets smaller than V (i.e. to low sets) is well-founded.

\item[Properties of CO constructions]\ 

The poly-sets are not obtained using a Church-Oswald construction (see: \cite{forster2005}) though perhaps they could have been.
They seem to have several of the properties of CO constructions proven by Forster in \cite{forster2005}.
Many of these are obvious consequences of properties already stated,

\begin{itemize}

\item[H$_{low}$]\ 

I suspect that H$_{low}$ (as defined by Forster\cite{forster2005}) is (here) the set of hereditarily low sets.

\item[Low Comprehension]\ 

There will be something analogous to low comprehension.
Any set of poly-sets is a low poly-set.
High poly-sets are all classes of poly-sets.

\item[12.] The set of low poly-sets is not a poly-set.

\item[13.] An image of a low poly-set is low, subsets of low poly-sets are low.

\item[14.] A low poly-set has a low power set.

\item[15.] Low poly-sets have sumsets. Low poly-sets of low poly-sets have low sumsets.

\end{itemize}
\end{itemize}

{\raggedright
\bibliographystyle{klunum}
\bibliography{rbjk}
} %\raggedright

\twocolumn[\section{Index}\label{Index}]
{\small\printindex}

\end{article}
\end{document}
