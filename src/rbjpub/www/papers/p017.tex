% $Id: p017.tex,v 1.3 2012/11/01 15:59:56 rbj Exp $
% bibref{rbjp017} pdfname{p017}

\documentclass[10pt,titlepage]{article}
\usepackage{makeidx}
\usepackage{graphicx}
\usepackage[unicode,pdftex]{hyperref}
\pagestyle{headings}
\usepackage[twoside,paperwidth=5.25in,paperheight=8in,hmargin={0.75in,0.5in},vmargin={0.5in,0.5in},includehead,includefoot]{geometry}
\hypersetup{pdfauthor={Roger Bishop Jones}}
\hypersetup{colorlinks=true, urlcolor=red, citecolor=blue, filecolor=blue, linkcolor=blue}
%\usepackage{html}
\usepackage{paralist}
\usepackage{relsize}
\usepackage{verbatim}
%\bodytext{BGCOLOR="#eeeeff"}
\makeindex
\newcommand{\indexentry}[2]{\item #1 #2}
\newcommand{\glossentry}[2]{\item #1 {\index #1 #2}}
\newcommand{\ignore}[1]{}
\def\Product{ProofPower}
\def\ouml{\"o}
\def\auml{\"a}

\title{Set Theory: Forcing and Semantics}
\author{Roger~Bishop~Jones}
\date{\ }

\begin{document}
%\frontmatter
                               
\begin{titlepage}
\maketitle

%\begin{abstract}
% Issues properly belonging to the semantics of set theory are now being addressed using results about forcing.
% This essay considers the relevance of forcing to semantics.
%\end{abstract}

\ignore{
\vfill

\begin{abstract}

\end{abstract}

\begin{centering}

{\footnotesize

\copyright\ Roger~Bishop~Jones;
}%footnotesize

\end{centering}
}%ignore

\end{titlepage}

\setcounter{tocdepth}{2}
{\parskip-0pt\tableofcontents}

%\listoffigures

%\mainmatter

\addcontentsline{toc}{section}{Preface}

\section{Introduction}

This essay responds to some of the recent publications in the philosophy of mathematics which either are or might be connected with \emph{the semantics} of set theory.
It is concerned particularly with the relevance of \emph{forcing} to questions in the semantics of set theory or to the question of what axioms should be used for set theory.
I am concerned more specifically to two major recent developments in which forcing plays a central role, the work of Woodin and Koellner on the resolution of CH, and the work of Hamkins in connection with the multiverse theory, the modal logic of forcing, and set theoretic geology.

These lines of research are highly technical.
It is not intended to engage with or contribute to the technical developments, but rather to consider the grounds for believing that the technical results are relevant to the issues.
Thus, in the case of the resolution of CH, various technical results are held to provide evidence for or against CH, and we are concerned to understand why this is thought to be the case.
The rational for multiverse theory is based upon considerations arising from generic extensibility, and the ancillary investigations into the modal logic and set theoretic geology investigate set theory by examining the relationship between models and their forcing extensions.
Both the arguments for the multiverse theory and the reasons for giving such a prominent role to forcing in the study of that multiverse are to be considered.

The analysis is undertaken from a particular philosophical standpoint which is similar in character to the philosophy of Rudolf Carnap.
Peter Koellner has criticised Carnap's pluralism and his own approach to the resolution of CH (and the more general problem of discovering or justifying new axioms for set theory) is presented by him in contrast to Carnap's pluralism.
The question of pluralism and the related question of whether there are absolutely undecidable problems are intimately bound up with his work on CH.

My own understanding of Carnap's philosophy, and more specifically of his pluralism, differs from that of Koellner, is not open to the same objections, and furnishes a standpoint from which a fruitful analysis and critique of the work of Koellner and Hamkins can be approached.
I therefore begin with a presention of key elements of that standpoint, contrasting it with the Koellner's interpretation of Carnap.

\section{Semantic Logicism}

In this section I describe a \emph{semantic} standpoint from which matters pertaining to the foundations of mathematics may be addressed.

This is described by comparison with a selection of alternative positions.

These descriptions are not offered as historically accurate descriptions of the beliefs of philosophers, though various philosophers are mentioned as exemplars (or even authors) of the features discussed.

The closest predecessor to the position I describe is found in the philosophy of Rudolf Carnap.
Like Carnap's philosophy of mathematics, it may be thought of a synthesis of logicism and formalism, in which the logicism is semantic rather than metaphysical.

To give a fuller description the following varieties of logicism and formalism are considered.

\subsection{formalism}

The notion of formalism I consider here is one which I associate with Hilbert, but which is distinct from the programme normally associated with Hilbert whose purpose was to estabish the consistency of the foundations of mathematics.

It is more closely related to the doctrine, against which Frege argued, to the effect that mathematics is a purely formal discipline in which symbols are manipulated without assigning to them any meaning or significance.
The

In the first instance I distinguish \emph{radical}, \emph{narrow} and \emph{broad} formalism. 

\subsection{Some Notes on Carnap}

Carnap attended lecture courses given by Frege as an undergraduate.
He was impressed by the precision of the methods and their contrast with the standard of arguments in philosophy.

\section{Forcing}

The plan at present for this section is:

\begin{itemize}
\item Three observations:

\begin{enumerate}
\item There are no generic extensions of V. 
\item $(ZFC \vdash (M \models P) \Rightarrow$\\
  $\exists B, G (G$ is a generic ultrafilter in B over M$))))$\\
  $\Rightarrow (ZFC \vdash \lnot P)$
\item A fallacious proof method.
\end{enumerate}

\item Their application in Jech.
\item Other interpretations of Jech.
\item Possible causes.
\item Applications elsewhere.
\end{itemize}

\subsection{Preliminaries}

In the following we are concerned with what is provable in the first order theory ZFC, primarily with results about ZFC and its models.
Though ZFC is a pure theory of sets, it is customary to simulate talk about classes using class abstracts as what Russell would have called incomplete symbols or Quine, virtual classes. 
Quantification of classes is not possible, but theorem schemata can be proven.

If $M$ is a transitive model (set or class), $B$ is an $M$ complete boolean algebra and $G$ is a generic ultrafilter over $B$, then the generic extension $M[G]$ is the transitive collapse of the quotient of the boolean valued model $M^B$ by the ultrafilter $G$ (the collapse of $M^B/G$).
 
Bell and Jech both prove that under these conditions:

%\begin{equation}
%G \not\in M \label{1}
%\end{equation}
\begin{equation}
G \in M[G] \label{2}
\end{equation}
\begin{equation}
M \subset M[G] \label{3}
\end{equation}

The notion of a generic extension of a transitive model (possibly a class) $M$ is defined in Jech 14.30 and 14.32 and Bell p97.

\subsection{A Basic Fact}

Here is a first brute fact:

There are no generic extensions of V.

In this the notion of generic extension is as defined in Jech 14.30 and 14.32 and Bell p97.
The notation for this in both cases is $M[G]$ where $M$ is a transitive set or class and $G$ is an $M$ generic ultrafilter over some boolean algebra complete in $M$.
 
V is the class abstract $\{x | T\}$.

\begin{equation}
ZFC \vdash \forall x (x \not\in V)
\end{equation}
\begin{equation}
ZFC \vdash V \not\subset M
\end{equation}

Jech \cite{jech2002} (4.22) and Bell \cite{bell2005} (14.32) both prove:

If M[G]


%backmatter

\appendix

\subsection{Cohen on Avoiding SM}

In his book on the continuum hypothesis\cite{cohenSTCH} Paul Cohen proceeds from the assumption that there exists a standard model (SM) of ZFC.
In Chapter IV \S 11 he then explains how this assumption can be dispensed with ``\emph{if one does not care about the construction of actual models}''.



\addcontentsline{toc}{section}{Bibliography}
\bibliographystyle{alpha}
\bibliography{rbj}

\addcontentsline{toc}{section}{Index}\label{index}
{\twocolumn[]
{\small\printindex}}

\vfill

\tiny{
Started 2011-09-01

Last Change $ $Date: 2012/11/01 15:59:56 $ $

\href{http://www.rbjones.com/rbjpub/www/papers/p017.pdf}{http://www.rbjones.com/rbjpub/www/papers/p017.pdf}

Draft $ $Id: p017.tex,v 1.3 2012/11/01 15:59:56 rbj Exp $ $
}%tiny

\end{document}

% LocalWords:
