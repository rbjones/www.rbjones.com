% $Id: p016.tex,v 1.1 2011/05/16 21:40:55 rbj Exp $
% bibref{rbjp016} pdfname{p016} 

\documentclass[10pt,titlepage]{book}
\usepackage{makeidx}
\usepackage{graphicx}
\usepackage[unicode,pdftex]{hyperref}
\pagestyle{headings}
\usepackage[twoside,paperwidth=5.25in,paperheight=8in,hmargin={0.75in,0.5in},vmargin={0.5in,0.5in},includehead,includefoot]{geometry}
\hypersetup{pdfauthor={Roger Bishop Jones}}
\hypersetup{colorlinks=true, urlcolor=red, citecolor=blue, filecolor=blue, linkcolor=blue}
%\usepackage{html}
\usepackage{paralist}
\usepackage{relsize}
\usepackage{verbatim}
%\bodytext{BGCOLOR="#eeeeff"}
\makeindex
\newcommand{\indexentry}[2]{\item #1 #2}
\newcommand{\glossentry}[2]{\item #1 {\index #1 #2}}
\newcommand{\ignore}[1]{}
\def\Product{ProofPower}
\def\ouml{\"o}
\def\auml{\"a}

\title{A Foundational Programme}
\author{Roger~Bishop~Jones}
\date{\ }

\begin{document}
\frontmatter
                               
\begin{titlepage}
\maketitle

\ignore{
\vfill

%\begin{abstract}
% A description of my efforts in the foundations of mathematics over the past 25+ years.
%\end{abstract}

\begin{centering}

{\footnotesize

\copyright\ Roger~Bishop~Jones;
}%footnotesize

\end{centering}
}%ignore

\end{titlepage}

\setcounter{tocdepth}{2}
{\parskip-0pt\tableofcontents}

%\listoffigures

\mainmatter

\addcontentsline{toc}{section}{Preface}

\section*{Preface}

Early versions of this document are primarily intended to give a set of references or links to the fragmentary documentation of my foundational thinking.
Later perhaps a fuller exposition of where I think I am heading.

\chapter{Introduction}

\section{First Ideas}

My interest in the foundations of mathematics dates back to at least 1971, my first acquaintance with the topic probably was Russell's \emph{Introduction to Mathematical Philosophy}\cite{russell1919}, unless one counts what Ayer had to say on the topic in \emph{Language Truth and Logic}\cite{ayer1936}.
From these two sources I became acquainted with two ideas which have remained important to me since.

First is the logicist thesis that \emph{mathematics is logic}.
Though Russell's exposition on this is much more detailed, Ayer's presentation of it as the observation that mathematics is \emph{analytic} and that its status as logic (in what must now be called `the broad sense') is independent of whatever arrangements we may chose for demonstrating mathematical propositions, has been the sense in which I have remained a logicist (without a hint of perturbation or doubt) ever since.

Second is the idea of a \emph{formal logical foundation system for mathematics}.
This idea is found in Frege and in Russell.
It is most clearly articulated by Frege's dictum that:

\begin{centering}
\begin{quote}
Mathematics = Logic + Definitions
\end{quote}
\end{centering}

In which the `logic' is what I have called above a \emph{logical foundation system}.
It is unimportant (for me) whether we call this central core `logic' or `set theory' or something else altogether.
The important thing 

\subsection{Combinatory Logic}

\emph{Persistent Applicative Heaps and Knowledge Bases}\cite{jones85}

\emph{Logical foundations and formal verification}\cite{jones86a}

\emph{Creative foundations and for program verification}\cite{jones86b}

\emph{Logical Necessity and the Foundations of Mathematics}\cite{jones87}

\section{Alternate Ontologies}

\subsection{Well-Founded}

\subsection{Non-Well-Founded}

\subsection{Polysets}

\emph{The Theory of PolySets and its Applications}\cite{jones06a}

\emph{PolySets: foundational ontologies for formal mathematics (presentation notes)}\cite{rbjn003}\cite{rbjo003}

\emph{PolySet Theory}\cite{rbj020}

\subsection{Infinitary Comprehension}

\emph{Set Theory as Consistent Infinitary Comprehension}\cite{rbjt021}

\emph{Infinitarily Definable Non-Well-Founded Sets}\cite{rbjt024}

\emph{Infinitarily Definable Sets}\cite{rbjt026}

\emph{Infinitary First Order Set Theory}\cite{rbjt027}

\subsection{Infinitary Combinators and Finitary Illative Lambda-Calculus}

\emph{An Illative Lambda-Calculus}\cite{rbjt041}

\section{Foundational Architecture}


\backmatter

%\chapter*{Glossary}\label{glossary}
%\addcontentsline{toc}{chapter}{Glossary}
%
%\begin{description}
%\item[]
%\end{description}

\addcontentsline{toc}{chapter}{Bibliography}
\bibliographystyle{alpha}
\bibliography{rbj}

\addcontentsline{toc}{chapter}{Index}\label{index}
{\twocolumn[]
{\small\printindex}}

\vfill

\tiny{
Started 2011-04-01

Last Change $ $Date: 2011/05/16 21:40:55 $ $

\href{http://www.rbjones.com/rbjpub/www/papers/p016.pdf}{http://www.rbjones.com/rbjpub/www/papers/p016.pdf}

Draft $ $Id: p016.tex,v 1.1 2011/05/16 21:40:55 rbj Exp $ $
}%tiny

\end{document}

% LocalWords:  Arist
