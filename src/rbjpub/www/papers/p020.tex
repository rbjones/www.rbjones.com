% $Id: p020.tex,v 1.1 2014/11/08 19:43:29 rbj Exp $
% bibref{rbjp020} pdfname{p020}

\documentclass[10pt,titlepage]{article}
\usepackage{makeidx}
\usepackage{graphicx}
\usepackage[unicode,pdftex]{hyperref}
\pagestyle{plain}
\usepackage[paperwidth=5.5in,paperheight=8in,hmargin={0in,0in},vmargin={0in,0in},includehead,includefoot]{geometry}
\hypersetup{pdfauthor={Roger Bishop Jones}}
\hypersetup{colorlinks=true, urlcolor=red, citecolor=blue, filecolor=blue, linkcolor=blue}
\usepackage{html}
\usepackage{paralist}
\usepackage{relsize}
\usepackage{verbatim}
\makeindex
\newcommand{\ignore}[1]{}

\title{Tao and Anarchy}
\author{Roger~Bishop~Jones}
\date{\ }

\begin{document}
%\frontmatter
                               
\begin{titlepage}
\maketitle

%\begin{abstract}
%The ancient Chinese philosophy of Dao provides ideas which may be seen as fundamental to contemporary ideas in ethics, politics and economics and therefore which together provide a basis for new approaches to practical philosophy.
%\end{abstract}

%\vfill

%\begin{centering}

%{\footnotesize
%copyright\ Roger~Bishop~Jones;
%}%footnotesize

%\end{centering}

\end{titlepage}

\setcounter{tocdepth}{2}
{\parskip-0pt\tableofcontents}

%\listoffigures

%\mainmatter

\section{Introduction}

My aim is to connect some of the ideas in Dao philosophy with contemporary trends,
suggesting that Dao may have some relevance to the way the world is moving.

The pace of change of contemporary society continues to accelerate, and these changes are ushered in increasingly by individuals and groups who are organised in less clearly structured ways, eschewing autocratic hierarchies in favour of an-archic networks.
This is the ``information age'' in which an increasing proportion of people no longer perform to order, but instead use their intelligence to perceive a need and use their creativity, ingenuity and technical skill to respond to (or exploit) that need.
These innovations may be destructive of less creative job opportunities, as the labour content in manufacturing is eroded by increasingly sophisticated automation, replacing that labour content by the higher level skills of engineers who develop new technologies, products and automate their manufacture .

Changes in how people thrive economically create a demand for new economic and political systems.
The distinction between socialistic or other command economies and capitalistic free-market economies is itself an instance of the distinction betwen autocracy and anarchy, and western observers presume that transition to free market economies creates a demand for less autocratic (more democratic) political institutions. 

Entrepreneurs in California and elsewhere, living in a fast moving world where businesses valued in tens of billions of dollars can be created in but a few years, easily grow impatient with political processes which effect change at glacial pace, and with the constraints imposed on innovation by the dead hand of government.

How in this context are we to understand what the future holds?
How can we judge to what changes we should consent or contribute?

Suprisingly, perhaps, there are timeless sources of wisdom which may have important contributions to make.
The ancient chinese philosophy of Dao is one which seems to me apt, many aspects of the developments in progress seem connected with elements of this system of ideas, and these ideas may help us find our way.

\section{Central Principals of Dao}

The following five ideas are sometimes listed:

\begin{description}
\item[The Dao]
The universe as a harmonised whole, `the way' it proceeds, the interconnectedness of all life.
\item [De]
Virtue, inner strength, as alignment with the Tao, acting in harmony with the whole.
\item [Yin Yang]
The underlying unity of apparent opposites.
\item [Wu Wei]
Spontaneous, natural and effortless action, in harmony with the Dao.
\item [The Sage]
The truly wise person whose actions are in complete harmony with his environment, 
\end{description}

\subsection{Ineffability}

The Dao cannot be captured in words.
This is sometimes said ``The Dao is nameless'' and ``that which can be named is not the Dao'', where I guess here ``named'' should be read as ``defined'' or ``described''.
So all that is written here or elsewhere about ``Dao'' is bound to be not quite right, or not at all right.

\subsection{Dao, De and The Sage}

The word Dao means ``way''.
In confucianism this is the path which should be taken by the individual, an ethical code of conduct.

In Daoism the word Dao is used for the path taken by the whole rather than the individual, the ethical code of conduct is replaced by the idea of ``Wu Wei'', effortless spontaneity, in harmony with the path of the whole.  

``De'' is the Daoist conception of virtue and strength in the individual, and consists in or derives from harmony with the Dao, in going with the flow of the Dao.
The sage is a supreme example of such virtue, one in complete harmony with the Dao.

The Daoist conception of Dao contrasts with the Confucian.

The doctrines at this point appear to be quietistic, one accepts things as they are, rather than putting up a fight.
However...

\subsection{Wu Wei, Yin Yang}

The ineffability of Dao becomes more conspicuous when we consider the notions of Yin and Yang.
There seem to be multiple conceptions of Yin Yang, both in Dao and elsewhere in oriental culture.
So Yin and Yang may be thought of metaphysically as kinds of universally pervasive substance which must be kept in balance.
Alternatively the notion of Yin Yang may serve as an injunction to balance and harmony in all aspects of life, rather than to particular substances.

A third notion of Yin Yang seems of greater importance in Dao, the unity of opposites.
The idea here is that many important concepts or ideas, when fully carried through, appear to involve their opposites so intimately that we may describe the two (the idea and its contrary) as two sides of the same coin, so that the opposition is only apparent, and one cannot fully realise the one without the other.

This idea may also be taken as moderating the sense that Dao is quietistic.
The quiet harmony which is advocated need not be an abnegation of energy and industry, instead it might simply be an inner sense of harmony underpinning an outer vigorous and robust enterprise.

The notion of Wu Wei seems contradictory in itself, sometimes translated ``actionless action'', perhaps more readily intelligible as acting spotaineously rather than deliberately.
Talk is also of living ``in the moment'' rather than planning for the future or reflecting on the past.
This seems at first glance an impractical recipe for a harmonious life.
A house cannot be built without planning, and life without shelter may be uncomfortable and full of strife.
However, if we look at this idea through the lense of Yin Yang, understanding Wu Wei as uniform with its opposite, then we may find a deeper sense which is of some value.

\subsection{Socio-Political Aspects}

I have written as if Dao was concerned with the whole and the individual and the relationship between them.
However, the ideas apply to social groups, to the relations betweem these groups or between groups and individuals.
On of the earliest writings on Dao, Lao Tzu's ``Tao Te Ching'' is written as advice to a ruler.

\section{Political Anarchism}

``political'' anarchism is the application of anarchist principals to politics.
It comes in many flavours, but for present purposes these can be represented as two general kinds, social anarchism, and anarcho-capitalism.

\subsection{Social Anarchism}

Social anarchism, is a radically libertarian socialism.
These ideologies regard the state as using force to maintain an unjust social order in which the many are deprived of the fruits of their labour by the few weathy capitalists.
They advocate abolition of the state and the organisation of society through voluntary associations.


\subsection{Anarcho-Capitalism}


\section{Tech-Utopia}


\section{Practical Philosophy and Dao}


%\backmatter

%\appendix

%\addcontentsline{toc}{section}{Bibliography}
%\bibliographystyle{alpha}
%\bibliography{rbj}

%\addcontentsline{toc}{section}{Index}\label{index}
%{\twocolumn[]
%{\small\printindex}}

%\vfill

%\tiny{
%Started 2012-10-19

%Last Change $ $Date: 2014/11/08 19:43:29 $ $

%\href{http://www.rbjones.com/rbjpub/www/papers/p019.pdf}{http://www.rbjones.com/rbjpub/www/papers/p019.pdf}

%Draft $ $Id: p020.tex,v 1.1 2014/11/08 19:43:29 rbj Exp $ $
%}%tiny

\end{document}

% LocalWords:
