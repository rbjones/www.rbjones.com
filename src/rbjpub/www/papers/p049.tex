% $Id: p049.tex $
% bibref{rbjp049} pdfname{p049}
\documentclass[10pt,titlepage]{article}
\usepackage{makeidx}
\newcommand{\ignore}[1]{}
\usepackage{graphicx}
\usepackage[unicode]{hyperref}
\pagestyle{plain}
\usepackage[paperwidth=5.25in,paperheight=8in,hmargin={0.75in,0.5in},vmargin={0.5in,0.5in},includehead,includefoot]{geometry}
\hypersetup{pdfauthor={Roger Bishop Jones}}
\hypersetup{pdftitle={Synthetic Epistemology}}
\hypersetup{colorlinks=true, urlcolor=red, citecolor=blue, filecolor=blue, linkcolor=blue}
%\usepackage{html}
\usepackage{paralist}
\usepackage{relsize}
\usepackage{verbatim}
\usepackage{enumerate}
\usepackage{longtable}
\usepackage{url}
\newcommand{\hreg}[2]{\href{#1}{#2}\footnote{\url{#1}}}
\makeindex

\title{\LARGE\bf Synthetic Epistemology}
\author{Roger~Bishop~Jones}
\date{\small 2022/03/13}

\begin{document}

%\begin{abstract}
% The outline of a possible book attempting to construct an epistemology for the future, in the light of the evolution of epistemology to the present day and the technological advances in progress which impact on how our knowledge of the world is expanded.
%\end{abstract}
                               
\begin{titlepage}
\maketitle

%\vfill

%\begin{centering}

%{\footnotesize
%copyright\ Roger~Bishop~Jones;
%}%footnotesize

%\end{centering}

\end{titlepage}

\ \

\ignore{
\begin{centering}
{}
\end{centering}
}%ignore

\setcounter{tocdepth}{2}
{\parskip-0pt\tableofcontents}

%\listoffigures

\pagebreak

\section*{Preface}
\addcontentsline{toc}{section}{Preface}

This is intended to be a sketch of a larger, book sized, project and a contender for the introduction to the book, if ultimately this kind of introduction seems a good idea.
Otherwise, its still a stage in working out the ideas to be addressed in the larger format.

\footnote{There may be ``hyperlinks'' in the PDF version of this document which either link to another point in the document  (if coloured blue) or to an internet resource  (if coloured red) giving direct access to the materials referred to (e.g. a Youtube video) if the document is read using some internet connected device.
Important links also appear explicitly in the bibiography.}

\section{Introduction}

Epistemology is the theory of knowledge, a branch of philosophy.
The postmodern philosopher Michel Foucault introduced the term ``episteme'' which I will use here to refer to those aspects of a culture or subculture which determine or influence what us accepted as knowledge.
The term ``synthetic epistemology'' I have coined to talk about the philosophical activity of systhesising epistemes, or reasoning about such syntheses.

The discussion is divided into three parts, following a classification derived from the philosophy of David Hume substantially as as refined in the works of Rudolf Carnap.
Hume classification came as two distinctions both of which have sometimes been called ``Hume's fork''.
The first is the distinction between, as Hume put it, relations between ideas, and matters of fact.
In the more recent terminology this is the distinction between logical and empirical propositions.
The second is that suggested by Hume's dictum, ``you cannot derive and ought from an is'', which presumes a clear distinction moral judgements and factual or logical claims.
This latter distinction can be extended to include with moral judgements other (though not all) evaluative claims.

These distinctions have a long history, perhaps not of precise articulation, but of faltering approaches, gradually homing in on the fundamental distinctions behind those tenuos descriptions, some of which I will sketch below.

It is not my intention here to discover truths of epistemology.
It is for us to chose how to discover, evaluate, organise and apply knowledge, and though common standards are highly desirable, there is always room for debate about what they should be.
This is written as a contribution to the continous evolution of epistemological standards.

In that description of my enterprise you may immediately sense my divergemce from at least two other attitudes to epistemology.
The first is any kind of epistemological absolutism.
It is up to us where we place the boundary between cpmjecture and fact, or indeed, whether we have such a dichotomy rather than a more complex classification.
\footnote{Just as hot and cold were displaced for many purposes by quantitative measures, some philosophers have sought quantitative evaluation of the evidential support for a scientific hypothesis.}
The second is the the radical relativism which we might see in Michel Foucault, and weaponisation of epistemology through the standpoint epistemology which was enabled by it.
Knowledge is important, it enables those who possess it to flourish, and facilitates their fulfillment.
Epistemes facilitate the acquisition of knowledge and guard against the degradation of our understanding of the world.
They are not all equally effective, they evolve with us, and this enterprise is part of the evolution of that evolutionary process.

\section{Matters of Logic}

The distinction between logical and empirical truth is fundamental to the epistemic developments discussed in this work, and it will be helpful in clarifying that distinction to talk about its history.

This account of the history of the development of logic is focussed on giving makingintelligible a particular view of the nature of logical truth and the purposes for which and ways in which a clear understanding of this conception of logical truth can be instrumental.
The notion of logical truth has been controversial, as indeed most philosophical concepts can be from time to time, particularly because of the centrality of that notion in the doctrines of Rudolf Carnap and the logical positivists, and (I would say) the importance to American philosophers of finding their own way and asserting their strengthg in mid 20th Century philosophical politics.
There is no intention here to revive the debate about what is ``logical truth'', my use of that term, even though it is very close to the preferred usage if Rudolf Carnao, is not a claim about the meaning of the term, it is a choice of terminology which is convenient for describing the \emph{episteme} to the construction of which this work is devolted.

The distinction between logical and empirical truth is rooted in the nature of language, and some practical application of logical truth is implicit in the skills which constitute linguistic competence.
The part of language which we can call `propositional' and consists in the use of indicative sentences depends for its use upon an understanding of the meaning of the conceots which can be asserted of the world or its constituents.
One aspect of that knowledge is the understanding of various conceptual inclusions.
A rich source of these is taxonomic heirarchies, and the knowledge of the conceptual inclusion in such heirarchies yields logical truths of which mere comprehension of tha language assures us, without need to inspect the world for corroboration.

For signs of systematic use of logical inference we have to wait quite a long time, from the birth of language, maybe coeval with homo sapiens some two or three hundred thousand years ago to the advent of mathematics as a theoretical discipline in ancient Greece.
Prior mathematics, Bablylonian arithmetic and Egyptian geometry, consisted in techniques for achieving certain practical ends, which Greek antiquity would have called logicstics rather than mathematics.
From about 600 BC the development of mathematics as a theoretical discipline began, seeking theoretical principles established not by successful application, but by deductive proof.

The deductive method was applied throughout the development of Greek mathematics, but was best exhibited in the development of Euclidean geometry, which showed it to be a highly reliable method for establishing geometric truths.
These early Greek philosophers also sought to understand the world and the cosmos through reason, and in these broader realms were rather less successful.
Theories advanced in these matters were not supported by deductive proof, were frequently contradicted, and eluded concensus.

In this way, the axiomatic method deployed for geometry and later systematised by Aristotle and Euclid came to be envied by all who wanted their opinions to be taken as gospel, and would often be misapplied or its merits fraudulently claimed for less reliable methods.
In clarifying the nature and boundaries of \emph{logical truth} the scope of deductive methods can be made clear.
This makes a proper evaluation of other kinds of claims to knowledge more readily achievable.

After the period of pre-Socratic philosophy, in which the disparate achievements of reason were laid bare, resolution of the disparities was sought in the two great classical philosophical systems, those of Plato and Aristotle.
It is here that much of the terminology connected with the distinction between logical and empirical truth discussed, as well as the ideas behind later terminological developments.

Plato was influenced by the philosophers Heraclitus\index{Heraclitus} and Parmenides\index{Parmenides}.
Heraclitus held that everything was in constant flux, Parmenides that change was impossible.
Parmenides arrived at his position by reason, Heraclitus was acquainted with the chaos of the world through his senses.
Plato's resolved these apparently contradictory philosophies by identifying two distinct domains, a realm of ideal forms, immutable and accessible to reason, of which certain knowledge could therefore be had, and a realm of appearances, fleeting and ephemeral, brought to us by our senses, of which we could have only opinion, not knowledge.
The realm of Platonic ``forms'' or universals, and one imperfectly revealed by our senses.
A classical predecessor to what might now be rendered as a realm of abstract objects, and one of material entities.

David Hume offered an intermediate position between this classical philosophies of Plato and Aristotle and more contemporary philosophy of Rudolf Carnap.

Hume makes the distinction thus:

\begin{quote}
``ALL the objects of human reason or enquiry may naturally be divided
  into two kinds, to wit, Relations of Ideas, and Matters of Fact.'' 
\end{quote}

Which we may characterise as sharing with Plato a classification by subject matter with multiple dimensions of significance.

He gives the following description of ``Relations of ideas'':

\begin{quote}
``Of the first kind are the sciences of Geometry, Algebra, and
Arithmetic; and in short, every affirmation which is either
intuitively or demonstratively certain.
That the square of the hypotenuse is equal to the square of the two
sides, is a proposition which expresses a relation between these
figures.
That three times five is equal to the half of thirty, expresses a
relation between these numbers.
Propositions of this kind are discoverable by the mere operation of
thought, without dependence on what is anywhere existent in the
universe.
Though there never were a circle or triangle in nature, the truths
demonstrated by Euclid would for ever retain their certainty and
evidence.''
\end{quote}

I'll go through this in some detail.
This connects with Plato in that the primary distinction is about subject matter, and that the distinction is connected with certain other distinctions.
The other distinctions found in Plato are combined into one as the distinction that 
we obtain knowledge about the forms by reason, but only have opinion about the world of appearances derived from our senses.
This distinguishes two ways of coming to kowledge ot opinion, and two correponding results of such enquiries.
What we have here might be called a ``triple dicholtomy'' in which three different characterisations are held to yield closely related distinctions, and we will see that progress in clarifying the notion of logical truth has advanced through the refinement of similar dichotomies.

The things being classified are usually sentences, judgements or propositions.
The distinctions made of of the following kinds:
\begin{itemize}

\item[Semantic] Concerned with the meaning of a sentence, or its subject matter, what the sentence is about or what it says about its subject.
  
\item[Epistemological] Concerned with how knowledge or opinion about the truth of the sentence or proposition is to be obtained, and/or justified.

\item[Confidence] Whether one can have certainty on the matter or not.

\item[Modal Status] Whether the sentence or proposition is necessary or contingent.
  
\end{itemize}


\subsection{Temporary Sketch}

The stages:

\begin{itemize}
\item the origin of language
\item the beginnings of deductive mathematics
\item the Platonic dichotomies
\item Aristotelian logic and metaphysics
  \begin{itemize}
\item  The Logic of Syllogism
\item  Category Theory and the Necessary/Contingent dichotomy
\item  Demonstrative Science
  \end{itemize}
\item Leibniz and the mechanisation of logic
  The mechanisability of Logic
\item The Humean dichotomies
\item Kant's retreat
  \begin{itemize}
\item   The concept of analyticity
\item   The synthetic a priori
\item   \end{itemize}
\item Bolzano
  \begin{itemize}
\item   The break with Aristotle
\item   Logic as Theory of Science
\item   Proposition as meaning of a sentence
  \end{itemize}

\end{itemize}

\section{Conjectures about the Future}

The epistemic synthesis approached here is based in large part on speculations about the very long term future of intelligent systems.

The earth is now about 4 billion years old, and we project the future of humanity 4 billion years into the future.
The influence of humanity into the future and across the universe might be broadly classified in terms of distance (in some unspecifiable metric) from the epicenter.

I present this here as a map of the future structure of humanity and its progeny and products.
This map is topologically similar to an onion, at the centre of which is our solar system

The innermost part is void, there being little prospect of intelligence, natural or synthetic remaining.
It is reasonable to expect that this will include the whole of the solar system.
At the nearest feasible distance from the solar system, which is likely to be a nearby stat system with a younger star.



\section{Matters of Fact}


\section{Values and Morals}


\phantomsection
\addcontentsline{toc}{section}{Bibliography}
\bibliographystyle{rbjfmu}
\bibliography{rbj2}

%\addcontentsline{toc}{section}{Index}\label{index}
%{\twocolumn[]
%{\small\printindex}}

%\vfill

\tiny{
Started 2022/10/09

\href{http://www.rbjones.com/rbjpub/www/papers/p049.pdf}{http://www.rbjones.com/rbjpub/www/papers/p049.pdf}

}%tiny

\end{document}

% LocalWords:
