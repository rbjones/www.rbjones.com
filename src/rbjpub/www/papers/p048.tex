% $Id: p048.tex $
% bibref{rbjp046} pdfname{p048}
\documentclass[10pt,titlepage]{article}
\usepackage{makeidx}
\newcommand{\ignore}[1]{}
\usepackage{graphicx}
\usepackage[unicode]{hyperref}
\pagestyle{plain}
\usepackage[paperwidth=5.25in,paperheight=8in,hmargin={0.75in,0.5in},vmargin={0.5in,0.5in},includehead,includefoot]{geometry}
\hypersetup{pdfauthor={Roger Bishop Jones}}
\hypersetup{pdftitle={A History of Enlightenment}}
\hypersetup{colorlinks=true, urlcolor=red, citecolor=blue, filecolor=blue, linkcolor=blue}
%\usepackage{html}
\usepackage{paralist}
\usepackage{relsize}
\usepackage{verbatim}
\usepackage{enumerate}
\usepackage{longtable}
\usepackage{url}
\newcommand{\hreg}[2]{\href{#1}{#2}\footnote{\url{#1}}}
\makeindex

\title{\LARGE\bf A History of Enlightenment}
\author{Roger~Bishop~Jones}
\date{\small 2022/03/13}

\begin{document}

%\begin{abstract}
%Speculating about Enlightennent 2.0 on the basis of ideas about its predecessors.
%\end{abstract}
                               
\begin{titlepage}
\maketitle

%\vfill

%\begin{centering}

%{\footnotesize
%copyright\ Roger~Bishop~Jones;
%}%footnotesize

%\end{centering}

\end{titlepage}

\ \

\ignore{
\begin{centering}
{}
\end{centering}
}%ignore

\setcounter{tocdepth}{2}
{\parskip-0pt\tableofcontents}

%\listoffigures

\pagebreak

\section*{Preface}

I have considered for some time the origins of Enlightenment values and methods and of the present deluge of their antithesis, flowing, at least in part, from the $20^{th}$ Century evolution of Marxist philosophy.
This is an attempt at writing something about a way forward.

It may be best to think of the essay as a work of fiction, inspired by what I know of the history of \emph{homo sapiens}.
The fiction draws out some hypotheses about why things are as they are, how we came to be, in a way which plausibly supports some ideas about how our future might go.
In relation to the future, it leans on the thesis that the best way to predict the future is to create it.
What is intended here is a blueprint of some of the key features of a global society which facilitates the fulfilment%
\footnote{satisfaction or happiness as a result of fully developing one's potential}
of all of our species and its progeny.

The aspects of interest here are primarily sociopolitical.

\addcontentsline{toc}{section}{Preface}

\footnote{There may be ``hyperlinks'' in the PDF version of this document which either link to another point in the document  (if coloured blue) or to an internet resource  (if coloured red) giving direct access to the materials referred to (e.g. a Youtube video) if the document is read using some internet connected device.
Important links also appear explicitly in the bibiography.}

\section{Introduction}

One of the responses to various strains of radical progressive thought and activism, particularly those most critical of ``Western Civilisation'' has been to reject the objectivity and the evidential basis for the critique, to highlight the evidence for continuing progress in most objective measures of well-being and to reassert the values of the enlightenment and the instrumental attitude to incremental progression which they facilitate.
Stephen Pinker is a prolific advocate for similar ideas \cite{pinker-angels,pinker-en}.

The idea that one might return, lock stock and barrel, to the values of certain intellectuals in the 18th Century is not attractive to me, and will be even less acceptable to the many progressives and revolutionaries to my left.
If there are aspects of enlightenment thought which deserve to be preserved, then they need to be distinguished from those aspects which do not, and perhaps some account of how they could and should be restored attempted.
But for the sake of that desire for progress which is held by Pinker to be a part of enlightenment ideals, the whole conception must surely be uprated to take account of what we have learned, and the evolution of our values over the last two hundred years.

My aim in this essay is to strip out from enlightenment thought and values a core which could form a politically neutral basis for the future of a diverse democratic society.
Specifically I am looking for a core of theory and value which are as close to politically neutral as once can get while trying to establish and maintain a healthy democratic society.

That this cannot be entirely neutral politically is ensured by the existence of ideologies which are essentially antithetical to democracy.
More difficult are ideas like those of Herbert Marcuse, who held that capitalism could not be reformed from within \cite{marcuse-repressive, marcuse-liberation}, but must be entirely dismantled before a just society could emerge, and advocated intolerance of any contrary opinion, but who was associated with the non-violent (if not wholly democratic) strategem which was named ``The Long March Through the Institutions'' \cite{sidwell-long}.

The basic complex of ideas and principles thus concocted I refer to in this essay as ``Enlightenment 2.0'' to reflect the desire to carry forward the most crucial aspects of the first enlightenment (Enlightenment 1.0) augmented to cope with the special challenges and opportunities of the present.

Many of the ideas and principles which are fundamental to democracy but which have nevertheless been challenged in 21st and late 20th Century substantially predate the enlightenment.
Some of those are first recorded in classical Greece, which can reasonably be thought of as a major predecessor of the enlighenment (``Enlightenment 0.0''), but others are so fundamental that they date back to the origin of anatomically modern homo sapiens, and are therefore pre-historic antecedents.

Giving credit to those three prominent contributors, the essay falls into four sections:

\begin{itemize}
\item The origins and nature of language
\item A Greek dawn for reflective rationality
\item Some enlightenment philosophy
\item Essential Enlightenment 2.0
\end{itemize}

Interwoven with this progressive enunciation of a positive core of doctrine and value I provide commentary on contrary ideas and tendencies, in which the promiment features include:

\begin{itemize}
\item The eclipse of democracy in Classical Greece and the rise of Christianity and Islam.
\item The romantic sequel to Enlightenment 1.0
\item Hegelian dialectical method
  \item Critical Theory and Postmodern Philosophy
\end{itemize}

In seeking a core for Enlightenment 2.0 which hinges upon arguably necessary conditions for a thriving democratic society my aim is for these to be consistent with as broad a range of political opinion as possible.
It may therefore be instructive to consider where the limits lie, what kinds of doctrine could not be encompassed, and also, how that failure should be expected to work.
This connects with words of Karl Popper on ``The Paradox of Tolerance'' (at first a mere footnote in \cite{popper-ose}) and the writings of Herbert Marcuse who countered with his lengthy essay on ``Repressice Tolerance'' \cite{marcuse-repressive}.
One doctrine which is in this respect beyond the pale is that the society we live in effectively prevents progressive reform from within, and that progress is only possible after its complete demise.
What that means is unclear to me.
In Marx it is the violent overthrow of the \emph{ancien regime} by the proletariat, but later exponents of the same point of view, for example, once again Marcuse (perhap in \cite{marcuse-liberation}) have eschewed that violent prospect in favour of the \emph{Long March} \cite{sidwell-long}.
The strategy and tactics of the long march are also beyond the pale, though one could easily imagine a long march which was not, or indeed, conceive of the Enlightenment 2.0 project (if such it be) which is itself a long march.

%\cite{pluckrose-cynical,lindsay-racemarx,friere-poled,gottesman-criturn}

That I divide the essay into four phases in the history of ``enlightenment'' might suggest that these will be presented sequentially.
I think it would work better to do otherwise, one reason for which is that the significance of the features I would like to highlight of the earlier phases is easier to draw out if at least some of the later developments have been discussed.


\section{The Evolution of Homo Sapiens}

This essay is concerned with the resolution of tensions between two apparent dichotomies which are found in human nature and human customs and institutions.
The first is that betweem violent and consensual mediation of actual or potential conflicts.
The second is the contrast between rational deliberation and `tribal' allegiance.

\section{Enlightenment 1.0}

By this I mean that period in the history of European thought which is normally called ``The Enlightenment'', and which is typically identified in the first instance with the work of the French \emph{Philosophe}s (Voltaire, Montesquieu, D'Alembert, Diderot, Rousseau, ...) in the mid-18th Century.

I wlll be synthesising my own take, not specifically on what were he essential elements of ``The Enlightenment'', but rather, on What I consider the  most important desiderata for ``Elightenments 2.0''.
Here however, I'm aiming to draw out a variety of opinions on what the enlightenment was both from a handful of contemporary philosophers (Hume, Rousseau and Kant) and from some later scholars looking back at it (Berlin, Pinker, William Bristow (SEP)).

\subsection{Kant}

\subsection{Isaiah Berlin}

Isaiah Berlin's take on The Enlightenment comes in two parts.
First three legs upon which the whole Western tradition rested:
\begin{enumerate}
  \item All genuine questions have an answer.

    In principle, by someone.
\item  The answers are knowable.
\item All the answers are compatible.
\end{enumerate}

and then, the extra twist added by the Enlightenment:
\begin{quotation}
That the knowledge is not to be obtained by revelation, tradition, dogma, introspection..., only by the correct use of reason, deductive or inductive as appropriate to the subject matter.

This extends not only to the mathematical and natural sciences, but to all other matters including ethics, aesthetics and politics.
\end{quotation}
and... that virtue is knowledge.


\subsection{Steven Pinker}

\subsection{David Hume}

\subsection{Rousseau}

\section{Enlightenment 2.0}

\appendix

%\section{Critical Pedagogy}


\cite{friere-poled,gottesman-criturn}


\phantomsection
\addcontentsline{toc}{section}{Bibliography}
\bibliographystyle{rbjfmu}
\bibliography{rbj}

%\addcontentsline{toc}{section}{Index}\label{index}
%{\twocolumn[]
%{\small\printindex}}

%\vfill

\tiny{
Started 2022/03/13


\href{http://www.rbjones.com/rbjpub/www/papers/p048.pdf}{http://www.rbjones.com/rbjpub/www/papers/p048.pdf}

}%tiny

\end{document}

% LocalWords:
