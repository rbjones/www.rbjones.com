% $Id: p048.tex $
% bibref{rbjp046} pdfname{p048}
\documentclass[10pt,titlepage]{article}
\usepackage{makeidx}
\newcommand{\ignore}[1]{}
\usepackage{graphicx}
\usepackage[unicode]{hyperref}
\pagestyle{plain}
\usepackage[paperwidth=5.25in,paperheight=8in,hmargin={0.75in,0.5in},vmargin={0.5in,0.5in},includehead,includefoot]{geometry}
\hypersetup{pdfauthor={Roger Bishop Jones}}
\hypersetup{pdftitle={A Story of Enlightenment}}
\hypersetup{colorlinks=true, urlcolor=red, citecolor=blue, filecolor=blue, linkcolor=blue}
%\usepackage{html}
\usepackage{paralist}
\usepackage{relsize}
\usepackage{verbatim}
\usepackage{enumerate}
\usepackage{longtable}
\usepackage{url}
\newcommand{\hreg}[2]{\href{#1}{#2}\footnote{\url{#1}}}
\makeindex

\title{\LARGE\bf A Story of Enlightenment}
\author{Roger~Bishop~Jones}
\date{\small 2022/03/13}

\begin{document}

%\begin{abstract}
%Speculating about Enlightennent 2.0 on the basis of ideas about its predecessors.
%\end{abstract}
                               
\begin{titlepage}
\maketitle

%\vfill

%\begin{centering}

%{\footnotesize
%copyright\ Roger~Bishop~Jones;
%}%footnotesize

%\end{centering}

\end{titlepage}

\ \

\ignore{
\begin{centering}
{}
\end{centering}
}%ignore

\setcounter{tocdepth}{2}
{\parskip-0pt\tableofcontents}

%\listoffigures

\pagebreak

\section*{Preface}

I have considered for some time the origins of `Enlightenment' values and methods and of the present deluge of their antitheses, flowing, at least in part, from the $20^{th}$ Century evolution of Marxist philosophy.
This is my assay at writing something about a way forward.

It may be best to think of the essay as a work of fiction, inspired by what I know of the history of \emph{homo sapiens}.
The fiction draws out some hypotheses about why things are as they are, how we came to be, in a way which plausibly supports some ideas about how our future might go.
In relation to the future, it leans on the thesis that the best way to predict the future is to create it, and offers a proposal for some aspects of the future that I would like us to build.
What is intended here is a blueprint of some of the key features of a global society which facilitates the fulfilment%
\footnote{satisfaction or happiness as a result of fully developing one's potential}
of all of our species and its progeny.
Despite that forward looking aspiration, the method is to look at the past for help in understanding the present, and to address primarily how we should be now to secure the future we desire and deserve.

Though I talk about what we `should' do, I do not offer these ideas as moral imperatives.
They are simply my preferences, which I recommend for your consideration.


The aspects of interest here are primarily sociopolitical.

\addcontentsline{toc}{section}{Preface}

\footnote{There may be ``hyperlinks'' in the PDF version of this document which either link to another point in the document  (if coloured blue) or to an internet resource  (if coloured red) giving direct access to the materials referred to (e.g. a Youtube video) if the document is read using some internet connected device.
Important links also appear explicitly in the bibiography.}

\section{Introduction}

One of the responses to various strains of radical progressive thought and activism, particularly those most critical of ``Western Civilisation'' has been to reject the objectivity and the evidential basis for the critique, to highlight the evidence for continuing progress in most objective measures of well-being and to reassert the values of the enlightenment and the instrumental attitude to incremental progression which they facilitate.
Stephen Pinker is a prolific advocate for similar ideas \cite{pinker-angels,pinker-en}.

The idea that one might return, lock stock and barrel, to the values of certain intellectuals in the 18th Century is not one I subscribe to, and will be even less acceptable to the many progressives and revolutionaries to my left.
If there are aspects of enlightenment thought which deserve to be preserved, then they need to be distinguished from those aspects which do not, and perhaps some account of how they could and should be restored attempted.
For the sake of that desire for progress which is held by Pinker to be a part of enlightenment ideals, the whole conception must surely be uprated to take account of what we have learned, and of the evolution of our values over the last two hundred years.

My aim in this essay is to strip out or `reconstruct' from enlightenment thought and values a core which might form a politically neutral basis for the future of a diverse democratic society.
I am looking for a core of theory and value which are as close to politically neutral as once can get while trying to establish and maintain a healthy democratic society.

That this cannot be entirely neutral politically is ensured by the existence of ideologies which are essentially (if not explicitly) antithetical to democracy.
Herbert Marcuse held that capitalism could not be reformed from within \cite{marcuse-repressive, marcuse-liberation}, but must be entirely dismantled before a just society could emerge, and advocated intolerance of any contrary opinion.
He was associated (at least as far as endorsement) with the non-violent (if not wholly democratic) strategem, advanced by the student political activist Rudi Dutchke, which was named ``The Long March Through the Institutions'' \cite{sidwell-long}.

The basic complex of ideas and principles thus concocted I refer to in this essay as ``Enlightenment 2.0'' to reflect the desire to carry forward the most crucial aspects of the first enlightenment (Enlightenment 1.0) augmented to cope with the special challenges and opportunities of the present.

Many of the ideas and principles which are fundamental to democracy but which have nevertheless been challenged in 21st and late 20th Century substantially predate the enlightenment.
Some of those are first recorded in classical Greece, which can reasonably be thought of as a major predecessor of the enlighenment (``Enlightenment 0.0''), but others are so fundamental that they date back to the origin of anatomically modern homo sapiens, and are therefore pre-historic antecedents.

Giving credit to those three prominent contributors, the essay falls into four sections:

\begin{itemize}
\item An Overview
\item The origins and nature of language
\item A Greek dawn for reflective rationality
\item Some enlightenment philosophy
\item Essential Enlightenment 2.0
\end{itemize}

Interwoven with this progressive enunciation of a positive core of doctrine and value I provide commentary on contrary ideas and tendencies, in which the promiment features include:

\begin{itemize}
\item The eclipse of democracy in Classical Greece and the rise of Christianity and Islam.
\item The romantic sequel to Enlightenment 1.0
\item Hegelian dialectical method
  \item Critical Theory and Postmodern Philosophy
\end{itemize}

In seeking a core for Enlightenment 2.0 which hinges upon arguably necessary conditions for a thriving democratic society my aim is for these to be consistent with as broad a range of political opinion as possible.
It may therefore be instructive to consider where the limits lie, what kinds of doctrine could not be encompassed, and also, how that failure should be expected to work.
This connects with words of Karl Popper on ``The Paradox of Tolerance'' (at first a mere footnote in \cite{popper-ose}) and the writings of Herbert Marcuse who countered with his lengthy essay on ``Repressice Tolerance'' \cite{marcuse-repressive}.
One doctrine which is in this respect beyond the pale is that the society we live in effectively prevents progressive reform from within, and that progress is only possible after its complete demise.
What that means is unclear to me.
In Marx it is the violent overthrow of the \emph{ancien regime} by the proletariat, but later exponents of the same point of view, for example, once again Marcuse (perhap in \cite{marcuse-liberation}) have eschewed that violent prospect in favour of the \emph{Long March} \cite{sidwell-long}.
The strategy and tactics of the long march are also beyond the pale, though one could easily imagine a long march which was not, or indeed, conceive of the Enlightenment 2.0 project (if such it be) which is itself a long march.

%\cite{pluckrose-cynical,lindsay-racemarx,friere-poled,gottesman-criturn}

That I divide the essay into four phases in the history of ``enlightenment'' might suggest that these will be presented sequentially.
I think it would work better to do otherwise, one reason for which is that the significance of the features I would like to highlight of the earlier phases is easier to draw out if at least some of the later developments have been discussed.

\section{A Timeline}

``The Enlightenemnt'' is a historical period in which certain ideas and values were promulgated which have been credited for much subsequent progress, but which have remained con


\begin{itemize}
\item Antiquity 800BC-500AD
\item The Middle Ages 500-1500AD
    \begin{itemize}
  \item Early Middle Ages 500-1000
  \item Hight Middle Ages 1000-1300 Scholasticism
  \item Late Middle Ages 1300-1500 Renaissance    
    \end{itemize}
\item Modernity 1500-1960AD
  \begin{itemize}
  \item Renaissance
  \item Reformation
  \item Enlightenment    
    \end{itemize}
\item Post Modernity 1960AD- 
\end{itemize}

\section{The Evolution of Homo Sapiens}

This essay is concerned with the resolution of tensions between two apparent dichotomies which are found in human nature and human customs and institutions.
The first is that betweem violent and consensual mediation of actual or potential conflicts.
The second is the contrast between rational deliberation and `tribal' allegiance.

\section{Enlightenment 1.0}

By this I mean that period in the history of European thought which is normally called ``The Enlightenment'', and which is typically identified in the first instance with the work of the French \emph{Philosophe}s (Voltaire, Montesquieu, D'Alembert, Diderot, Rousseau, ...) in the mid-18th Century.

I wlll be synthesising my own take, not specifically on what were he essential elements of ``The Enlightenment'', but rather, on What I consider the  most important desiderata for ``Elightenments 2.0''.
Here however, I'm aiming to draw out a variety of opinions on what the enlightenment was both from a handful of contemporary philosophers (Hume, Rousseau and Kant) and from some later scholars looking back at it (Berlin, Pinker, William Bristow (SEP)).

\subsection{Kant}

\subsection{Isaiah Berlin}

Isaiah Berlin's take on The Enlightenment comes in two parts.
First three legs upon which the whole Western tradition rested:
\begin{enumerate}
  \item All genuine questions have an answer.

    In principle, by someone.
\item  The answers are knowable.
\item All the answers are compatible.
\end{enumerate}

and then, the extra twist added by the Enlightenment:
\begin{quotation}
That the knowledge is not to be obtained by revelation, tradition, dogma, introspection..., only by the correct use of reason, deductive or inductive as appropriate to the subject matter.

This extends not only to the mathematical and natural sciences, but to all other matters including ethics, aesthetics and politics.
\end{quotation}
and... that virtue is knowledge.


\subsection{Steven Pinker}

\subsection{David Hume}

\subsection{Rousseau}

\section{Enlightenment 2.0}

\appendix

\section{Pinker on Progress}

Is the world getting better or worse? A look at the numbers.

https://youtu.be/yCm9Ng0bbEQ


Many people face the news each morning
with trepidation and dread.
Every day, we read of shootings,
inequality, pollution, dictatorship,
war and the spread of nuclear weapons.
These are some of the reasons
that 2016 was called the "Worst. Year. Ever."

0:35
Until 2017 claimed that record --
and left many people longing for earlier decades,
when the world seemed safer, cleaner and more equal.

0:46
But is this a sensible way to understand the human condition
in the 21st century?
As Franklin Pierce Adams pointed out,
"Nothing is more responsible for the good old days
than a bad memory."

1:01
You can always fool yourself into seeing a decline
if you compare bleeding headlines of the present
with rose-tinted images of the past.
What does the trajectory of the world look like
when we measure well-being over time using a constant yardstick?

Let's compare the most recent data on the present
with the same measures 30 years ago.
\begin{itemize}
\item[Last year(2017)], Americans killed each other at a rate of 5.3 per hundred thousand,
had seven percent of their citizens in poverty
and emitted 21 million tons of particulate matter
and four million tons of sulfur dioxide.
\item[30 years ago], the homicide rate was 8.5 per hundred thousand,
poverty rate was 12 percent
and we emitted 35 million tons of particulate matter
and 20 million tons of sulfur dioxide.
\end{itemize}
1:51
What about the world as a whole?
\begin{itemize}
\item[Last year]
  \begin{description}
    \item[] the world had
    \item[12] ongoing wars,
\item[60] autocracies,
\item[10] percent of the world population in extreme poverty
  and more than
\item[10,000] nuclear weapons.
  \end{description}
\item[30 years ago]
  \begin{description}
  \item[]  there were
    \item[23] wars,
\item[85] autocracies,
\item[37] percent of the world population in extreme poverty
  and more than
\item[60,000] nuclear weapons.
   \end{description}
\end{itemize}

True, last year was a terrible year for terrorism in Western Europe,
with 238 deaths,
but 1988 was worse with 440 deaths.

What's going on?
Was 1988 a particularly bad year?
Or are these improvements a sign that the world, for all its struggles,
gets better over time?

2:39
Might we even invoke the admittedly old-fashioned notion of progress?

2:45
To do so is to court a certain amount of derision,
because I have found that intellectuals hate progress.
And intellectuals who call themselves progressive really hate progress.
Now, it's not that they hate the fruits of progress, mind you.
Most academics and pundits
would rather have their surgery with anesthesia than without it.
It's the idea of progress that rankles the chattering class.
If you believe that humans can improve their lot, I have been told,
that means that you have a blind faith
and a quasi-religious belief in the outmoded superstition
and the false promise of the myth of the onward march
of inexorable progress.
You are a cheerleader for vulgar American can-doism,
with the rah-rah spirit of boardroom ideology,
Silicon Valley and the Chamber of Commerce.
You are a practitioner of Whig history,
a naive optimist, a Pollyanna and, of course, a Pangloss,
alluding to the Voltaire character who declared,
"All is for the best in the best of all possible worlds."

3:58
Well, Professor Pangloss, as it happens, was a pessimist.
A true optimist believes there can be much better worlds
than the one we have today.

4:06
But all of this is irrelevant,
because the question of whether progress has taken place
is not a matter of faith
or having an optimistic temperament or seeing the glass as half full.
It's a testable hypothesis.
For all their differences,
people largely agree on what goes into human well-being:
life, health, sustenance, prosperity, peace, freedom, safety, knowledge,
leisure, happiness.
All of these things can be measured.
If they have improved over time, that, I submit, is progress.

Let's go to the data,
beginning with the most precious thing of all, life.

4:43

\begin{itemize}
\item For most of human history, life expectancy at birth was around 30.
Today, worldwide, it is more than 70,
and in the developed parts of the world,
more than 80.
\item 250 years ago, in the richest countries of the world,
a third of the children did not live to see their fifth birthday,
before the risk was brought down a hundredfold.
5:05
Today, that fate befalls less than six percent of children
in the poorest countries of the world.
\item Famine is one of the Four Horsemen of the Apocalypse.
It could bring devastation to any part of the world.
Today, famine has been banished
to the most remote and war-ravaged regions.
200 years ago, 90 percent of the world's population
subsisted in extreme poverty.
Today, fewer than 10 percent of people do.
\end{itemize}

5:32
For most of human history,
the powerful states and empires
were pretty much always at war with each other,
and peace was a mere interlude between wars.
Today, they are never at war with each other.
The last great power war
pitted the United States against China 65 years ago.
More recently, wars of all kinds have become fewer and less deadly.
The annual rate of war has fallen from about 22 per hundred thousand per year
in the early '50s to 1.2 today.
Democracy has suffered obvious setbacks
in Venezuela, in Russia, in Turkey
and is threatened by the rise of authoritarian populism
in Eastern Europe and the United States.
Yet the world has never been more democratic
than it has been in the past decade,
with two-thirds of the world's people living in democracies.

6:24
Homicide rates plunge whenever anarchy and the code of vendetta
are replaced by the rule of law.
It happened when feudal Europe was brought under the control of centralized kingdoms,
so that today a Western European
has 1/35th the chance of being murdered
compared to his medieval ancestors.
It happened again in colonial New England,
in the American Wild West when the sheriffs moved to town,
and in Mexico.
Indeed, we've become safer in just about every way.

6:54
Over the last century, we've become
\begin{itemize}
\item 96 percent less likely
to be killed in a car crash,
\item 88 percent less likely to be mowed down on the sidewalk,
\item 99 percent less likely to die in a plane crash,
\item 95 percent less likely to be killed on the job,
\item 89 percent less likely to be killed by an act of God,
such as a drought, flood, wildfire, storm, volcano,
landslide, earthquake or meteor strike,
presumably not because God has become less angry with us
but because of improvements in the resilience of our infrastructure.
\end{itemize}

7:30
And what about the quintessential act of God,
the projectile hurled by Zeus himself?
Yes, we are 97 percent less likely to be killed by a bolt of lightning.

7:43
Before the 17th century,
no more than 15 percent of Europeans could read or write.
Europe and the United States achieved universal literacy
by the middle of the 20th century,
and the rest of the world is catching up.

7:56
Today, more than 90 percent of the world's population
under the age of 25 can read and write.
In the 19th century, Westerners worked more than 60 hours per week.
Today, they work fewer than 40.

Thanks to the universal penetration of running water and electricity
in the developed world
and the widespread adoption of washing machines, vacuum cleaners,
refrigerators, dishwashers, stoves and microwaves,
the amount of our lives that we forfeit to housework
has fallen from 60 hours a week
to fewer than 15 hours a week.

8:32
Do all of these gains in health, wealth, safety, knowledge and leisure
make us any happier?
The answer is yes.
In 86 percent of the world's countries,
happiness has increased in recent decades.

Well, I hope to have convinced you
that progress is not a matter of faith or optimism,
but is a fact of human history,
indeed the greatest fact in human history.
And how has this fact been covered in the news?

9:03
A tabulation of positive and negative emotion words in news stories
has shown that during the decades in which humanity has gotten healthier,
wealthier, wiser, safer and happier,
the "New York Times" has become increasingly morose
and the world's broadcasts too have gotten steadily glummer.
Why don't people appreciate progress?
Part of the answer comes from our cognitive psychology.
We estimate risk using a mental shortcut called the "availability heuristic."
The easier it is to recall something from memory,
the more probable we judge it to be.
The other part of the answer comes from the nature of journalism,
captured in this satirical headline from "The Onion,"
"CNN Holds Morning Meeting to Decide
What Viewers Should Panic About For Rest of Day."

9:57
News is about stuff that happens, not stuff that doesn't happen.
You never see a journalist who says,
"I'm reporting live from a country that has been at peace for 40 years,"
or a city that has not been attacked by terrorists.

Also, bad things can happen quickly,
but good things aren't built in a day.
The papers could have run the headline,
"137,000 people escaped from extreme poverty yesterday"
every day for the last 25 years.
That's one and a quarter billion people leaving poverty behind,
but you never read about it.

10:31
Also, the news capitalizes on our morbid interest
in what can go wrong,
captured in the programming policy, "If it bleeds, it leads."
Well, if you combine our cognitive biases with the nature of news,
you can see why the world has been coming to an end
for a very long time indeed.

10:50
Let me address some questions about progress
that no doubt have occurred to many of you.
First, isn't it good to be pessimistic
to safeguard against complacency,
to rake the muck, to speak truth to power?

Well, not exactly.
It's good to be accurate.
Of course we should be aware of suffering and danger
wherever they occur,
but we should also be aware of how they can be reduced,
because there are dangers to indiscriminate pessimism.

11:18
One of them is fatalism.
If all our efforts at improving the world
have been in vain,
why throw good money after bad?
The poor will always be with you.
And since the world will end soon --
if climate change doesn't kill us all,
then runaway artificial intelligence will --
a natural response is to enjoy life while we can,
eat, drink and be merry, for tomorrow we die.

11:42
The other danger of thoughtless pessimism is radicalism.
If our institutions are all failing and beyond hope for reform,
a natural response is to seek to smash the machine,
drain the swamp,
burn the empire to the ground,
on the hope that whatever rises out of the ashes
is bound to be better than what we have now.

12:03
Well, if there is such a thing as progress,
what causes it?

12:07
Progress is not some mystical force or dialectic lifting us ever higher.
It's not a mysterious arc of history bending toward justice.
It's the result of human efforts governed by an idea,
an idea that we associate with the 18th century Enlightenment,
namely that if we apply reason and science
that enhance human well-being,
we can gradually succeed.

12:32
Is progress inevitable? Of course not.
Progress does not mean that everything becomes better
for everyone everywhere all the time.
That would be a miracle, and progress is not a miracle
but problem-solving.
Problems are inevitable
and solutions create new problems which have to be solved in their turn.

12:53
The unsolved problems facing the world today are gargantuan,
including the risks of climate change
and nuclear war,
but we must see them as problems to be solved,
not apocalypses in waiting,
and aggressively pursue solutions
like Deep Decarbonization for climate change
and Global Zero for nuclear war.

13:15
Finally, does the Enlightenment go against human nature?
This is an acute question for me,
because I'm a prominent advocate of the existence of human nature,
with all its shortcomings and perversities.

13:28
In my book "The Blank Slate,"
I argued that the human prospect is more tragic than utopian
and that we are not stardust, we are not golden
and there's no way we are getting back to the garden.

13:42
But my worldview has lightened up
in the 15 years since "The Blank Slate" was published.
My acquaintance with the statistics of human progress,
starting with violence
but now encompassing every other aspect of our well-being,
has fortified my belief
that in understanding our tribulations and woes,
human nature is the problem,
but human nature, channeled by Enlightenment norms and institutions,
is also the solution.

14:09
Admittedly, it's not easy to replicate my own data-driven epiphany
with humanity at large.

Some intellectuals have responded
with fury to my book "Enlightenment Now,"
saying first how dare he claim that intellectuals hate progress,
and second, how dare he claim that there has been progress.

14:32
With others, the idea of progress just leaves them cold.
Saving the lives of billions,
eradicating disease, feeding the hungry,
teaching kids to read?

Boring.

14:44
At the same time, the most common response I have received from readers is gratitude,
gratitude for changing their view of the world
from a numb and helpless fatalism
to something more constructive,
even heroic.

14:57
I believe that the ideals of the Enlightenment
can be cast a stirring narrative,
and I hope that people with greater artistic flare
and rhetorical power than I
can tell it better and spread it further.

15:09
It goes something like this.
We are born into a pitiless universe,
facing steep odds against life-enabling order
and in constant jeopardy of falling apart.
We were shaped by a process that is ruthlessly competitive.
We are made from crooked timber,
vulnerable to illusions, self-centeredness
and at times astounding stupidity.
Yet human nature has also been blessed with resources
that open a space for a kind of redemption.
We are endowed with the power to combine ideas recursively,
to have thoughts about our thoughts.
We have an instinct for language,
allowing us to share the fruits of our ingenuity and experience.
We are deepened with the capacity for sympathy,
for pity, imagination, compassion, commiseration.
These endowments have found ways to magnify their own power.
The scope of language has been augmented
by the written, printed and electronic word.
Our circle of sympathy has been expanded
by history, journalism and the narrative arts.
And our puny rational faculties have been multiplied
by the norms and institutions of reason,
intellectual curiosity, open debate,
skepticism of authority and dogma
and the burden of proof to verify ideas
by confronting them against reality.

As the spiral of recursive improvement
gathers momentum,
we eke out victories against the forces that grind us down,
not least the darker parts of our own nature.
We penetrate the mysteries of the cosmos, including life and mind.
We live longer, suffer less, learn more,
get smarter and enjoy more small pleasures
and rich experiences.

16:57
Fewer of us are killed, assaulted, enslaved, exploited
or oppressed by the others.
From a few oases, the territories with peace and prosperity are growing
and could someday encompass the globe.
Much suffering remains
and tremendous peril,
but ideas on how to reduce them have been voiced,
and an infinite number of others are yet to be conceived.
We will never have a perfect world,
and it would be dangerous to seek one.
But there's no limit to the betterments we can attain
if we continue to apply knowledge to enhance human flourishing.

17:34
This heroic story is not just another myth.
Myths are fictions, but this one is true,
true to the best of our knowledge, which is the only truth we can have.
As we learn more,
we can show which parts of the story continue to be true and which ones false,
as any of them might be and any could become.
And this story belongs not to any tribe
but to all of humanity,
to any sentient creature with the power of reason
and the urge to persist in its being,
for it requires only the convictions
that life is better than death,
health is better than sickness,
abundance is better than want,
freedom is better than coercion,
happiness is better than suffering
and knowledge is better than ignorance and superstition.

18:22
Thank you.

\cite{pinker-angels}
%\cite{friere-poled,gottesman-criturn}


\phantomsection
\addcontentsline{toc}{section}{Bibliography}
\bibliographystyle{rbjfmu}
\bibliography{rbj}

%\addcontentsline{toc}{section}{Index}\label{index}
%{\twocolumn[]
%{\small\printindex}}

%\vfill

\tiny{
Started 2022/03/13


\href{http://www.rbjones.com/rbjpub/www/papers/p048.pdf}{http://www.rbjones.com/rbjpub/www/papers/p048.pdf}

}%tiny

\end{document}

% LocalWords:
