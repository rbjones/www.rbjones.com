% $Id: p032.tex $
% bibref{rbjp032} pdfname{p032}
\documentclass[10pt,titlepage]{book}
\usepackage{makeidx}
\newcommand{\ignore}[1]{}
\usepackage{graphicx}
\usepackage[unicode]{hyperref}
\pagestyle{plain}
\usepackage[paperwidth=5.25in,paperheight=8in,hmargin={0.75in,0.5in},vmargin={0.5in,0.5in},includehead,includefoot]{geometry}
\hypersetup{pdfauthor={Roger Bishop Jones}}
\hypersetup{pdftitle={Prolegomena to A Progressive Democratic Constitution for the UK}}
\hypersetup{colorlinks=true, urlcolor=red, citecolor=blue, filecolor=blue, linkcolor=blue}
%\usepackage{html}
\usepackage{paralist}
\usepackage{relsize}
\usepackage{verbatim}
\usepackage{enumerate}
\usepackage{longtable}
\usepackage{url}
\newcommand{\hreg}[2]{\href{#1}{#2}\footnote{\url{#1}}}
\makeindex

\title{\LARGE\bf Prolegomena to \\A \\Progressive Democratic Constitution \\for the UK}
\author{Roger~Bishop~Jones}
\date{\small 2020/01/17}


\begin{document}
\frontmatter

%\begin{abstract}
% This is an assay at the formulation of that part of a progressive liberal constitution for
% the United Kingdom which provides an account of the fundamental philosophical principles
% which underly the constitution.
%\end{abstract}
                               
\begin{titlepage}
\maketitle

%\vfill

%\begin{centering}

%{\footnotesize
%copyright\ Roger~Bishop~Jones;
%}%footnotesize

%\end{centering}

\end{titlepage}

\ \

\ignore{
\begin{centering}
{}
\end{centering}
}%ignore

\setcounter{tocdepth}{2}
{\parskip-0pt\tableofcontents}

%\listoffigures

\mainmatter

\pagebreak

\section*{Preface}

\addcontentsline{toc}{section}{Preface}

There are ``hyperlinks'' in this PDF document which either link to another point in the document  (if coloured blue) or to an internet resource  (if coloured red) giving direct access to the materials referred to (e.g. a Youtube video) if the document is read using some internet connected device.

\hreg{http://rbjones.com/with_underbar}{test link}

\chapter{Introduction}

This document is conceived of as a part of a rational approach to the formulation and
adoption of a constitution for the government of the United Kingdom.

Let us suppose that such a constitution would begin with a statement of the fundamental principles which informed its composition and which it was intended to enshrine, promote and protect.
It would then go into some detail about how the government should be organised to embody those principles.
It would not be the purpose of the constitution to undertake analysis or to examine alternatives, such matters having been addressed prior to formulation of the constitution, which would present the chosen solution.
This prolegomena is offered as my own take on the kind of prior analysis of fundamentals which is desirable, and as such is offered as perinent philosophical analysis.

For this reason I begin with a chapter on \emph{Analytic Method} which is the most technical part of my essay, and provides a compact account of some relevant history of philosophical analysis as a prelude to a more accessible account of the kind of analysis which I propose to undertake, which is intended to be intelligible even if the preceeding historical context is skimmed or skipped.

\chapter{Analytic Method}

The term analysis is used for a broad range of methods; even qualified as \emph{philosophical} analysis it has no definite meaning out of context.
The kind of analysis here intended harks back to the kind of formal logical analysis
advocated in the 20th century by Rudolf Carnap, rooted in the technical advances in
formal logic in the second half of the nineteenth century and the philosophical attitude to
the opportunities thus created in whicb Gottlob Frege played an important role.
These strictly formal methods are most readilly applicable in the \emph{a priori} sciences,
notably in mathematics, though even in that domain the level of rigour required made the
application of these methods very arduous, as illustrated by Russell’s monumental Principia
Mathematica. The difficulty in achieving that standard of rigour progressively increases as
we move from a prioir to empirical science, and fromn ‘hard’ sciences such as physics into
the social sciences, so that for polical science or political philosophy we may consider strict
fornmality beyond practicable and settle for seeking rigour through some kind of informal
simulacrum of key features which we may associate with formal analysis.

\section{Background}



\section{The Method}

\chapter{Context}

The context is unusual in the following ways:
•
•
•

\chapter{The Purposes of a Constitution}


%\addcontentsline{toc}{section}{Bibliography}
%\bibliographystyle{alpha}
%\bibliography{rbj2}

%\addcontentsline{toc}{section}{Index}\label{index}
%{\twocolumn[]
%{\small\printindex}}

%\vfill

%\tiny{
%Started 2020/01/17


%\href{http://www.rbjones.com/rbjpub/www/papers/p032.pdf}{http://www.rbjones.com/rbjpub/www/papers/p032.pdf}

%}%tiny

\end{document}

% LocalWords:
