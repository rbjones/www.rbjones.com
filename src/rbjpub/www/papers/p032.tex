% $Id: p032.tex $fi
% bibref{rbjp032} pdfname{p032}
\documentclass[10pt,titlepage]{book}
\usepackage{makeidx}
\newcommand{\ignore}[1]{}
\usepackage{graphicx}
\usepackage[unicode]{hyperref}
\pagestyle{plain}
\usepackage[paperwidth=5.25in,paperheight=8in,hmargin={0.75in,0.5in},vmargin={0.5in,0.5in},includehead,includefoot]{geometry}
\hypersetup{pdfauthor={Roger Bishop Jones}}
\hypersetup{pdftitle={Prolegomenon to A Progressive Democratic Constitution for the Unitewd Kingdom}}
\hypersetup{colorlinks=true, urlcolor=red, citecolor=blue, filecolor=blue, linkcolor=blue}
%\usepackage{html}
\usepackage{paralist}
\usepackage{relsize}
\usepackage{verbatim}
\usepackage{enumerate}
\usepackage{longtable}
\usepackage{url}
\newcommand{\hreg}[2]{\href{#1}{#2}\footnote{\url{#1}}}
\makeindex

\title{\LARGE\bf Prolegomenon \\to a \\Progressive \\Democratic Constitution \\for the \\United Kingdom}
\author{Roger~Bishop~Jones}
\date{\small 2020/02/12}


\begin{document}
\frontmatter

%\begin{abstract}
% This is an assay at the formulation of that part of a progressive liberal constitution for
% the United Kingdom which provides an account of the fundamental philosophical principles
% which underly the constitution.
%\end{abstract}
                               
\begin{titlepage}
\maketitle

%\vfill

%\begin{centering}

%{\footnotesize
%copyright\ Roger~Bishop~Jones;
%}%footnotesize

%\end{centering}

\end{titlepage}

\ \

\ignore{
\begin{centering}
{}
\end{centering}
}%ignore

\setcounter{tocdepth}{2}
{\parskip-0pt\tableofcontents}

%\listoffigures

\mainmatter

\pagebreak

\section*{Preface}

\addcontentsline{toc}{section}{Preface}

There are ``hyperlinks'' in the PDF version of this monograph which either link to another point in the document  (if coloured blue) or to an internet resource  (if coloured red) giving direct access to the materials referred to (e.g. a Youtube video) if the document is read using some internet connected device.
Important links also appear explicitly in the bibiography.

\chapter{Introduction}

This document is conceived of as a part of an analytic approach to the formulation and
adoption of a constitution for the government of the United Kingdom.

Let us suppose that such a constitution would begin with a statement of the fundamental principles which informed its composition, which principles it was intended to enshrine, promote and protect.
It would then go into some detail about how the government should be organised to embody those principles, and to enshrine their most essential features.

It would not be the purpose of the constitution to examine alternatives or to undertake analysis, such matters having been addressed prior to formulation of the constitution, which would present the chosen solution.
This prolegomenon is offered as my own take on the kind of prior analysis of fundamentals which is desirable, and as such is offered as pertinent philosophical analysis.

This material is intended to speak to the kind of person who might be involved in the drafting of a constitution; perhaps some parts suitable for a wider audience, among those who ultimately would be asked to consent to the constitution.
Failing that level of intelligibility, it might nevertheless contribute to the kind of discussion among the more philosophically inclined which could indirectly find its way into the ``public debate''.

The monograph is structured as follows:

\paragraph{Chapter \ref{Method} - rational discourse and analytic method}
I begin with a chapter on \emph{analytic method}, which is the most technical part of the essay and provides a compact account of some relevant history of philosophical analysis as a prelude to a more accessible account of the kind of analysis which I propose to undertake.
The concluding account is intended to be intelligible even if the preceeding historical context is skimmed or skipped.
The discussion here serves two distinct but similar purposes:
\begin{itemize}
\item
  It is intended to clarify the method adopted in this monograph ({\it philosophical analysis}), and also...
\item
  ...to say something about the kind of {\it rational discourse} which one might hope to find in the operation of the democracy which the proposed constitution would describe.
\end{itemize}
...but this is not some {\it idée fixe} of origin prior to the endeavour.
The accounts of {\it philosophical analysis} and of {\it rational discourse} are intended to be continuously adapted and refined as the work progresses.
The enterprise is therefore, in some measure, meta-theoretic and methodological.

\paragraph{Chapter \ref{Context} - context}

We will see in the methodological discussion that much depends on what we bring into the discussion as factual beliefs or moral convictions, and that a logical analysis or a persuasive rational case will be based on a clear and explicit understanding of what these (possibly contentious) premises are.

Scope is also significant.
It is clear {\it ab initio}, that only democratic constitutions are at stake here; if you are interested in other forms of government then this may not be a broad enough analysis, or may be completely {\it off piste}, for you.
Consideration of the purposes we might have in mind for the the system of government and the role of the constitution in securing those purposes is included in consudering the scope of the analysis.

My choice of scope and the premises which I accept without justification are important ways in which values are imported into a discussion which might otherwise (in principle) have been of a purely logical character, and it is my hope that these are laid bare here before kernel of the analysis is broached.

\chapter{Rational Discourse and Analytic Method}\label{Method}

The term analysis is used for a broad range of methods; even qualified as \emph{philosophical} analysis it has no definite meaning out of context.
The kind of analysis here intended harks back to the kind of formal logical analysis
advocated in the 20th century by Rudolf Carnap, rooted in the technical advances in
formal logic in the second half of the nineteenth century and the philosophical attitude to
the opportunities thus created in whicb Gottlob Frege played an important role.
These strictly formal methods are most readilly applicable in the \emph{a priori} sciences,
notably in mathematics, though even in that domain the level of rigour required made the
application of the methods very arduous, as illustrated by Russell \& Whiteheads’s monumental Principia Mathematica\cite{russell10}.
The difficulty in achieving that standard of rigour progressively increases as we move from a priori to empirical science, and fromn ‘hard’ sciences such as physics into
the social sciences, so that for polical science or political philosophy we may consider strict
fornmality beyond practicable and settle for seeking rigour through some kind of informal
simulacrum of key features which we may associate with formal analysis.

\section{Background}

\nocite{heijenoort67,sabine63,berlin-liberty}

We may speculate that reason predates language, since its most elementary manifestations are evident in primates other than homo sapiens.

Whenever we observe some regularity in nature, our knowledge of that regularity provides a premise from which we may be said to reason when we adapt our behaviour to anticipate future exemplifications of that regularity to our advantage.
When our experience in finding a particular kind of fruit in a certain species of tree leads us to the opinion that in the right season all or most such trees bear that fruit, we have formed a pre-scientific hypothesis.
In science this step is sometimes called inductive reasoning.
When we apply this knowledge to indulge a fondness of that fruit by seeking out such a tree, we have reasoned from that general hypothesis together with knowledge of certain particulars (such as the location of suitable trees) to a specific expectation of where the fruit may be found.

Mathematics enables more elaborate kinds of practical reasoning.
For example, the use of geometry in ancient Egypt for the purpose of fairly redistributing agricultural land along the Nile as the river shifts its course.

\paragraph{Euclid\cite{euclidEL1}}

\paragraph{Aristotle\cite{aristotleL325,aristotleL391,aristotleL400}}

\paragraph{Plato\cite{plato-republic}}

\paragraph{Leibniz\cite{leibniz-pw,leibnizNCT}}

\paragraph{Newton\cite{newton-pm}}

\paragraph{Locke\cite{locke-stg,locke-t,locke-echu}}

\paragraph{Hune\cite{humeTHN,humeECHU}}

\paragraph{The Enlightenment}

\paragraph{James Mill and Classical Liberalism}

\paragraph{John Stuart Mill and Social Liberalism}

\paragraph{Kant\cite{kant-cpr}}

\paragraph{Frege\cite{fregeTPWF,frege1879,frege1884,frege1893,frege1903}}

\paragraph{Russell}

\paragraph{Carnap}

\paragraph{Quine}

\section{The Method}

By describing my intended method as {\it Philosophical analysis} I intend the following.

  Firstly it may be useful to distinguish the terms {\it politics}, {\it political theory} and {\it political philosophy} as they are intended here.

  \begin{description}
  \item{\bf politics} is a decision making process concerning the governance of a country or of other groups or organisations.
  \item{\bf political theory} is the scientific study of political institutions, it should be objective and empirical.
  \item{\bf political philosophy} is the further analysis of political institutions typicallly building on political and ethical theory.
    It sometimes becomes ideological advocacy concerning the justification of existing institution s, tbe promotion of alternatives, or the evoluton of the {\it status quo}.
  \end{description}

  It is intended that the discussion be {\it rational} and contribute to the establishment of a written constitution by reasoned discussion among the interested parties (primarily the citizens of the United Kingdom).
  However, rational does not mean purely logical or deductive, for we recognise Hume's admonitions about the severely restricted scope of logical necessity.
  It is acknowledged that such logic as may be mustered proceeds from premises of whose certainty we cannot be absolutely assured.
  For this reason, the premises upon which we build are to be made explicit and clear.

  Furthermore, we aim to clearly distinguish premises into (alleged) {\it statements of fact} (or hypotheses), and {\it statements of preference}.

  The explicit expisure of all premises helps render analytic what might otherwise be more polemical, but we do not pretend that the analysis presented is neutral or impartial.
  The values we hold are exposed by the choices we make in which hypotheses to work from and which alternaive characteristics of constitutions we countenance.

  There are some further aspects of the proposed method which are worth mentioning here.

  Political philosophy is often concerned with the justification of some particular systen of governance, and often the justification will rest upon some supposed moral characteristics.
  A good example of this is the philosophical underpinning of the classical liberal view of Jeremy Bentham, whose system of jurisprudent was intended as a systematic progran of legislative reform guided by a utilitarian ethical theory.

  The analysis here is intended to contribute to a process of formulating a written consituttion for the United Kingdom, and it is intended to facilitate the formation of a sufficiently broad consensus that a national referendum would result in a substantial majority in favour of adopting the constitution.
  In this conntext, the justification of the constitution and of the powers it confers is that the consittution has been chosen by the people as evidenced by their overwhelming support in a such a referendum.
  Because we consider only {\it democratic} constitutions, this {it justification by choice} is promulgated tbrough the entire process of governamnce through the democratic choices of the people in the resulting democracy.

  It is to be expected that questions of morals and of justification may feature in the voting choices made by the citizens of this democracy, but questions of justification and morality do not play an important role in the analysis which follows.

\chapter{Context}\label{Context}

\cite{berlin91}

In reaction to the excesses of social rights activists and the growing polatisation in the present moment, some contemporary pundits have invoked and seek to revive and protect the insights both invoking the ideology of some earlier time whose principle merits are being undermined and need to be protected, are

\section{The Place of National Constitutions}


\chapter{}

\phantomsection
\addcontentsline{toc}{section}{Bibliography}
\bibliographystyle{rbjfmu}
\bibliography{rbj}

%\addcontentsline{toc}{section}{Index}\label{index}
%{\twocolumn[]
%{\small\printindex}}

%\vfill

%\tiny{
%Started 2020/01/17


%\href{http://www.rbjones.com/rbjpub/www/papers/p032.pdf}{http://www.rbjones.com/rbjpub/www/papers/p032.pdf}

%}%tiny

\end{document}

% LocalWords:
