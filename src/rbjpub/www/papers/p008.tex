% $Id: p008.tex,v 1.7 2009/06/02 19:13:56 rbj Exp $
% bibref{rbjp008} pdfname{p008} 
\documentclass{rbjk}
\usepackage{makeidx}
\newcommand{\ignore}[1]{}
\usepackage[unicode,pdftex]{hyperref}
\hypersetup{pdfauthor={Roger Bishop Jones}}
\hypersetup{colorlinks=true, urlcolor=blue, citecolor=blue, filecolor=black, linkcolor=blue}

\begin{document}                                                                                   
\begin{article}
\begin{opening}  
\title{Metaphysical Positivism}
\author{Roger Bishop \surname{Jones}}
\runningauthor{Roger Bishop Jones}

\begin{abstract}
Fragmentary presentation of aspects of the philosophy of metaphysical positivism.
\end{abstract}
\end{opening}

%\def\tableofcontents{{\parskip=0pt\@starttoc{toc}}}
\setcounter{tocdepth}{4}
{\parskip-0pt\tableofcontents}

\section{INTRODUCTION}

This document began (or was last seen) under the title of ``constructive scepticism'' but had not progressed very far.
Since I am now trying to write about metaphysical positivism, which is perhaps just another name, or perhaps a name which encompasses the scepticism and more, I have decided to change the title and do some work in here.
 
What little there was in its predecessor remains at present as Section %\ref{SCEPTICISM}.

What I am now working on is a little discussion of metaphysical positivism as a mediating position which might possibly be rendered intelligible both to Carnap and to Grice, see Section %\ref{CARNAPANDGRICE}.

\section{SCEPTICISM}\label{SCEPTICISM}

Sceptical ideas appear very early in the history of ``Western'' philosophy, which is generally considered to have begun in Greece around 600 years before Christ.
It is the distinctive feature of this philosophical tradition that it is, or purports to be, rational.
Mathematics is also held by mathematicians to have begun at roughly the same time, and is distinguished from (mere) arithmetic (and more generally the Greek ``logistic'') by being a theoretical rather than a practical discipline.
The litmus test for the practice of this kind of mathematics is the concepts ``theorem'' and ``proof'' which turn mathematics into a deductive science.

The startling feature of deductive mathematics (as later to be codified in ``the axiomatic method'') is its very high levels of reliability and stability.
To a high degree mathematical proofs inspire confidence, and results once established by uncontroversial proofs are rarely later overturned.
Philosophers have frequently envied mathematics this definitive rigour, and many have sought to philosophise in like manner, with little success.

The failure of philosophy to emulate the rigour of mathematics may be analysed into two main elements.
The first is that philosophers sometimes seek knowledge about the physical world, which they base to a greater or lesser extent upon observation, but the process of observation is unreliable, and the results even if true, provide an insufficient basis for deductive inference to the kind of general laws which were of interest to early philosophers.
The second is that deductive inference, while highly productive and reliable in certain domains (notably mathematics) proves less satisfactory in other domains (and perhaps least satisfactory of all in philosophy).

These two areas of weakness may be associated with the earliest known sceptical thinking, in the writings of Heraclitus and Zeno.
Heraclitus was sceptical about the senses and build his philosophy on this scepticism.

Pyrrhonean scepticism is supposed to arise from the failure of an earnest and ongoing search for truth.
However, the literature is unconvincing, appearing to represent an exclusive interest in the refutation of dogma, with little evidence of a genuine attempt to seek truth.

In general, philosophical movements which are claimed to be sceptical have similar characteristics.
When elements of scepticism are combined with significant amounts of positive theory the result is rarely called scepticism, except perhaps by its opponents.
Descartes' system in which systematic and radical scepticism lays the ground for a dogmatic metaphysical system is called, simply, Cartesianism.
Hume's radical scepticism is called ``empiricism'' or ``positivism''.

There were before Hume in the seventeenth century, philosophers such as Mersenne and Gassendi, who combined a respect for sceptical arguments with a positive attitude toward science, and sought to reconcile these in a mitigated or constructive scepticism.
These may be thought of a precursors of many subsequent philosophies in which the same ideals are sought but are described not as varieties of scepticism but rather as kinds of empiricism, positivism or pragmatism.

My enterprise here is broadly similar in character.
My primary motivation is speculation about our futures which encompasses consideration of how various kinds of knowledge can best be sought and applied.
In this it is recognised that ill-founded dogma has a disutility offsetting the benefits of applicable knowledge, and that a balanced and open-minded attitude may be of greatest value.

My concern in this essay is only with scepticism and how it can be encompassed within a positive weltanschauung.

\subsection{Varieties of Scepticism}

\section{CARNAP AND GRICE}
%\label{CARNAPANDGRICE}

This began as \href{http://rbjones.com/pipermail/hist-analytic_rbjones.com/2009q2/000267.html}{a threat on my part} to J.L. Speranza and subscribers to the \href{http://www.hist-analytic.org/}{hist-analytic mailing list} to prepare a piece in which I attempt to explain to Carnap why metaphysics might not always be very bad and how the study of Aristotle's metaphysics might have some point even for a positivist.

It is well understood that if you invite Speranza to a party, he will bring Grice along, so of course Grice is soon in the conversation and we are \href{http://rbjones.com/pipermail/hist-analytic_rbjones.com/2009q2/000268.html}{learning more from Speranza} of Grice's apparent antagonism towards positivism.
This seemed to me to run counter to \href{http://rbjones.com/pipermail/hist-analytic_rbjones.com/2009q2/000254.html}{my own earlier} touching upon the similarity in their pragmatic attitudes towards ontological questions, and though I defer to Speranza on Gricean exegesis, I want to explore points at which things don't seem to square up.

In thinking about metaphysical positivism I had in mind, Carnap of course to whose philosophy it comes closest, and Kripke, with whom it concurs only in admitting metaphysics.
Though there is I hope a serious embracing of some king of metaphysics, it is not of Kripke's kind.
I thought about the source, and the opposition, Grice has not featured.
However, since Grice has insinuated himself into my thoughts, it seems an opportunity to introduce a symmetry.
On the one hand I present metaphysical positivism as reconciling Carnap's metaphysics with a positivistically tenable notion of metaphysics, on the other, I wonder whether the metaphysically relaxed Grice might be persuaded that in metaphysical positivism there is a kind of positivism which he might at least converse if not concur. 
Here are some thoughts along these lines.

I'm going to say a few things about Carnap and Grice first, before getting into metaphysical positivism. 
But before either, some b\^{e}tes noires.

\subsection{B\^{e}tes Noires}

Speranza talks about Grice's b\^{e}tes noires, which we will come to later.
It will help to explain what I am trying to do if I mention one or two of mine.

Gladiatorial philosophy is one of them.
I'll contrast this with {\it conversational} philosophy.
The contrast is most pertinent to philosophical dialogue, in the flesh, on a mailing list, but applies also to papers and books.

Here are some features:

\begin{description}

\item[gladiatorial] \ 

In gladiatorial philosophy, one engages with one or more opponents and aims to prevail.
An outright defeat conceded by the opponent is best, but a kind of point victory is better than nothing.
It is not the purpose of the exercise to understand your opponent or to learn anything from him (you already know better than he does).
It is not necessary to criticise the position he actually holds, this would often involve considerable effort in first coming to an understanding of that position.
It is necessary only to refute what he says, understood in whatever way is most convenient for a convincing dismissal.

\item[conversational] \ 

Conversation philosophy is a game played by two or more philosophers having a common interest in some subject matter who believe that through dialogue they can come to understand that subject better than they otherwise might.
The idea is that seeing this problem through from other persons point of view will enrich ones own understanding, even if that point of view is not necessarily ``better'' than your own.
To obtain this benefit it may be necessary to invest considerable effort in coming to an understanding of the others position, and in conveying to him an understanding of your own.
Not all of it, the bits which bear upon the issues.

Unless all partners are committed to this process it will not work.
\end{description}

It is my perception, that in philosophical discussion on the internet, it is very difficult to achieve what I have called a conversational dialogue.
It is the norm rather than the exception that people from the first assume that they understand what you are saying, and one may then find many occasions to complain that an adversary criticises a position that one has never held.

This is however, not a problem exclusive to the internet fringe.
It is pervasive in the writings of professional philosophers that a philosopher will criticise an opponent by attacking a position which he has never held.
If the philosopher is at the van of a new philosophical trend this can be done with impunity, readers sympathetic to the new philosophical trend will not trouble to check the accuracy of the picture presented of some obsolescent point of view.

Famous works of this kind include, Austin's ``Sense and Sensibilia'', Quine's ``Two Dogmas'' and Kripke's ``Naming and Necessity''.

Sometimes a failure to converse is not due to bad faith.
Two philosophers earnestly wishing to understand each other may nevertheless fail for reasons which are mysterious and insurmountable, and we shall touch upon some of these.

The point of this little diatribe is to explain that my aim in discussing Carnap and Grice here is to explore whether they might have had a conversation, and whether some of the ideas of metaphysical positivism might help in providing a basis for that dialogue.
To say some things on either side which might help to make conversation possible.
Locating the points at which incomprehension, may possibly facilitate their isolation, and permit conversation elsewhere.

\subsection{Carnap Towards  Metaphysics}

I think I have a reasonable (if not scholarly) grasp of Carnap's attitude towards metaphysics.

I shall sketch it here in three parts, a technical, an intuitive, and a radical rejection.

\begin{itemize}
\item[techical]
The technical rejection is the least important, though it often takes the headlines.
Metaphysics has sometimes be characterised as necessary knowledge of synthetic truths.
The technical rejection arises from adopting a conceptual framework in which there can be no such thing (on pain of contradiction).
This Carnap does through an understanding of the concepts of necessity and analyticity in which the only difference between them is that the former applies to propositions and the latter to those things which express propositions (of which propositions are the meanings).

The effect if this is to define out of existence a certain kind of metaphysics, viz. metaphysical propositions which are held by some philosopher to be necessary but synthetic.
The propositions do not go away, they are likely then to be regarded as analytic (at least, if the claim to necessity is sustainable).
In Carnap's terminology, the internal question becomes less controversial, but the issue is exported to the ``external question''.


\item[intuitive]
Carnap had I believe a quite genuine incomprehension of certain kinds of philosophical question, which he was therefore inclined to consider meaningless, and thus either to proscribe as metaphysical.or to regard as matters of pragmatics.

\item[radical]


\end{itemize}








Next I'd like to address the differing perceptions of Seperanza and myself in relation to Grice's attitude towards such as Carnap.

\subsection{Grice Towards Positivism}

Grice thought of himself as a kind of ``ordinary language philospher'' or at the least, a philosopher who saw some point and philosophical value in studying ``ordinary language'' as it is.
His predecessor Austin severely criticised aspects of positivism in his sense and sensibility, and appeared an implacable opponent.

Grice's position was more moderate, several parts of his work made him appear to me less antagonistic.



%{\raggedright
%\bibliographystyle{klunamed}
%\bibliography{rbj,fmu}
%} %\raggedright

\end{article}
\end{document}
