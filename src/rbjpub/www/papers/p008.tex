% $Id: p008.tex,v 1.5 2007/09/28 15:24:53 rbj01 Exp $
% bibref{rbjp008} pdfname{p008} 
\documentclass{rbjk}
\usepackage{makeidx}
\newcommand{\ignore}[1]{}
\usepackage[unicode,pdftex]{hyperref}
\hypersetup{pdfauthor={Roger Bishop Jones}}
\hypersetup{colorlinks=true, urlcolor=black, citecolor=black, filecolor=black, linkcolor=black}

\begin{document}                                                                                   
\begin{article}
\begin{opening}  
\title{Philosophia Universalis}
\author{Roger Bishop \surname{Jones}}
\runningauthor{Roger Bishop Jones}

\begin{abstract}
A philosophical tract.
\end{abstract}
\end{opening}

%\def\tableofcontents{{\parskip=0pt\@starttoc{toc}}}
\setcounter{tocdepth}{4}
{\parskip-0pt\tableofcontents}

\section{Introduction}

I aim in this essay to describe a philosophical weltanschauung.
This is not an undifferentiated body of demonstrated, or even supposed, truths.

Some elements of the viewpoint I consider more important than others.
Questions of relative importance or of primacy or subsidiarity are an integral part of the viewpoint.
Practical questions have primacy, theoretical questions are subsidiary, their {\it importance} arising from their contribution to the furtherance of practical objectives. 

Of course I am here speaking of practical philosophy, which I take to include ethics, politics and economics, but also and more importantly questions of a more personal nature.
How to conduct ones life.
Within this practical realm there are relationships too.
We may consider, for example, that the purpose of ethics, politics and economics is to further the interests of individuals, and that principles in the former domains should be reducible in some way to principles in the latter.
Nothing quite so straightforward is envisaged here,

In addressing questions in practical philosophy, a difficulty arises from the intrusion of matters of a subjective or discretionary nature.
Our conclusions about the nature of the ideal society and the institutions which may realise such a society, depend inevitably upon our own values,
My own values are not universal, others have quite different values.
I do not desire to impose my values on others, but nor could I willingly tolerate some of the more extreme violations which others might countenance.

Social institutions provide ways of resolving such value conflicts.

\section{Evolution}

The theory of evolution is an old theory which has itself evolved quite rapidly in recent times.
During this period it has influenced the views of philosophers and others on socio-political matters, and has often been used in the justification of political, economic or ethical doctrines.

My own limited understanding of evolution and its impact upon human nature and society is an important element in my weltanschauung, and however imperfect, must feature in the rationale my {\it philosophia universalis}.

Here the ideas about evolution which are significant for my philosophy are presented.

\subsection{Scope}

For present purposes the scope of the theory of evolution is much as it was conceived by Darwin.
It is a theory about the origin of species by natural selection.
My interest is in what this tells us about people and society.
Evolution cast in a more general way, as a theory about the universe as a whole, has not as yet made any contribution to my understanding.

\subsection{Some Principles}.

To the idea of evolution by natural selection we must add first genetics, a mechanism which enables the reproduction of very complex organisms.
The discovery of the structure of DNA and its role as repository of genetic information advanced our understanding of the mechanics of reproduction and evolution.






%{\raggedright
%\bibliographystyle{klunamed}
%\bibliography{rbj,fmu}
%} %\raggedright

\end{article}
\end{document}
