% $Id: p008.tex,v 1.11 2010/02/10 11:17:12 rbj Exp $
% bibref{rbjp008} pdfname{p008} 

\documentclass[10pt,titlepage]{book}
\usepackage{makeidx}
\usepackage[unicode,pdftex]{hyperref}
\pagestyle{headings}
\usepackage[twoside,paperwidth=5.25in,paperheight=8in,hmargin={0.75in,0.5in},vmargin={0.5in,0.5in},includehead,includefoot]{geometry}
\hypersetup{pdfauthor={Roger Bishop Jones}}
\hypersetup{colorlinks=true, urlcolor=black, citecolor=black, filecolor=black, linkcolor=black}
\usepackage{html}
\usepackage{paralist}
\usepackage{relsize}
\bodytext{BGCOLOR="#eeeeff"}
\makeindex
\newcommand{\indexentry}[2]{\item #1 #2}
\newcommand{\glossentry}[2]{\item #1 {\index #1 #2}}
\newcommand{\ignore}[1]{}
\def\Product{ProofPower}
\def\ouml{\"o}

\title{A Conversation Between Carnap and Grice\\ {\small as it might have been in the 21st Century}}
\author{Roger~Bishop~Jones and J.L.~Speranza
}
\date{\ }

\begin{document}
\frontmatter
                               
\begin{titlepage}
\maketitle

\vfill

%\begin{abstract}
%A speculation about what the fundamental differences between the philosophies of Rudolf Carnap and Paul Grice might have been had they survived into the twenty first century.
%\end{abstract}

\vfill

\begin{centering}

\vfill

\footnotesize{
Started 2010-01

Last Change $ $Date: 2010/02/10 11:17:12 $ $

\href{http://www.rbjones.com/rbjpub/www/papers/p008.pdf}{http://www.rbjones.com/rbjpub/www/papers/p008.pdf}

Draft $ $Id: p008.tex,v 1.11 2010/02/10 11:17:12 rbj Exp $ $

\copyright\ Roger~Bishop~Jones and J.L.~Speranza;

}%footnotesize
\end{centering}

\end{titlepage}

\setcounter{tocdepth}{3}
{\parskip-0pt\tableofcontents}

\mainmatter

\chapter{Introduction}

This is a speculative discussion of the relationship between the philosophies of Carnap and Grice.
Structure a bit fluid at present.

\section{Carnap and Grice}\label{CARNAPANDGRICE}

This began as \href{http://rbjones.com/pipermail/hist-analytic_rbjones.com/2009q2/000267.html}{a threat on my part} to J.L. Speranza and subscribers to the \href{http://www.hist-analytic.org/}{hist-analytic mailing list} to prepare a piece in which I attempt to explain to Carnap why metaphysics might not always be very bad and how the study of Aristotle's metaphysics might have some point even for a positivist.

It is well understood that if you invite Speranza to a party, he will bring Grice along, so of course Grice is soon in the conversation and we are \href{http://rbjones.com/pipermail/hist-analytic_rbjones.com/2009q2/000268.html}{learning more from Speranza} of Grice's apparent antagonism towards positivism.
This seemed to me to run counter to \href{http://rbjones.com/pipermail/hist-analytic_rbjones.com/2009q2/000254.html}{my own earlier} touching upon the similarity in their pragmatic attitudes towards ontological questions, and though I defer to Speranza on Gricean exegesis, I want to explore points at which things don't seem to square up.

In thinking about {\it metaphysical} positivism I had in mind, Carnap of course to whose philosophy it comes closest, and Kripke, with whom it concurs only in admitting metaphysics.
Though there is I hope a serious embracing of some kind of metaphysics, it is not of Kripke's kind.
In my thought about the source, and the opposition, Grice has not featured.

However, since Grice has insinuated himself into my thoughts, it seems an opportunity to introduce a symmetry.
On the one hand I present metaphysical positivism as a reconciling of Carnap's positivism with a positivistically tenable notion of metaphysics, on the other, I wonder whether the metaphysically relaxed Grice might be persuaded that in metaphysical positivism there is a kind of positivism which he might at least converse if not concur. 
Here are some thoughts along these lines.

{[\it There is a slight difficulty here now since, I am beginning to feel that ``metaphysical positivism'' is dead!
I will need to change this story a bit, though I don't think it will greatly affect the attempted reconciliation of Carnap and Grice.]}

I'm going to say a few things about Carnap and Grice first, before getting into metaphysical positivism. 
But before either, some b\^{e}tes noires.

\subsection{B\^{e}tes Noires}

Speranza talks about Grice's b\^{e}tes noires, which we will come to later.
It will help to explain what I am trying to do if I mention one or two of mine.

Gladiatorial philosophy is one of them.
I'll contrast this with {\it conversational} philosophy.
The contrast is most pertinent to philosophical dialogue, in the flesh, on a mailing list, but applies also to papers and books.

Here are some features:

\begin{description}

\item[gladiatorial] \ 

In gladiatorial philosophy, one engages with one or more opponents and aims to prevail.
An outright defeat conceded by the opponent is best, but a kind of point victory is better than nothing.
It is not the purpose of the exercise to understand your opponent or to learn anything from him (you already know better than he does).
It is not necessary to criticise the position he actually holds, this would often involve considerable effort in first coming to an understanding of that position.
It is necessary only to refute what he says, understood in whatever way is most convenient for a convincing dismissal.

\item[conversational] \ 

Conversation philosophy is a game played by two or more philosophers having a common interest in some subject matter who believe that through dialogue they can come to understand that subject better than they otherwise might.
The idea is that seeing this problem from another persons point of view will enrich ones own understanding, even if that point of view is not necessarily ``better'' than your own.
To obtain this benefit it may be necessary to invest considerable effort in coming to an understanding of the others position, and in conveying to him an understanding of your own.
Not all of it, the bits which bear upon the issues.

Unless all partners are committed to this process it will not work.
\end{description}

It is my perception, that in philosophical discussion on the Internet, it is very difficult to achieve what I have called a conversational dialogue.
It is the norm rather than the exception that people from the first assume that they understand what you are saying, and one may then find many occasions to complain that an adversary criticises a position that one has never held.

This is however, not a problem exclusive to the Internet fringe.
It is pervasive in the writings of professional philosophers that a philosopher will criticise an opponent by attacking a position which he has never held.
If the philosopher is at the van of a new philosophical trend this can be done with impunity, readers sympathetic to the new philosophical trend will not trouble to check the accuracy of the picture presented of some obsolescent point of view.

Famous works of this kind include, Austin's ``Sense and Sensibilia'', Quine's ``Two Dogmas'' and Kripke's ``Naming and Necessity''.

Sometimes a failure to converse is not due to bad faith.
Two philosophers earnestly wishing to understand each other may nevertheless fail for reasons which are mysterious and insurmountable, and we shall touch upon some of these.

The point of this little diatribe is to explain that my aim in discussing Carnap and Grice here is to explore whether they might have had a conversation, and whether some of the ideas of metaphysical positivism might help in providing a basis for that dialogue.
To say some things on either side which might help to make conversation possible.
Locating the points at which incomprehension, may possibly facilitate their isolation, and permit conversation elsewhere.

\chapter{Grice}

\section{Grice the Metaphysician}

A question here is whether Grice's metaphysics was of a kind which Carnap would have been inclined to reject.

\section{Grice's Demons and Perilous Places}

\begin{quote}
"As I thread my way unsteadily along the   tortuous mountain 
path which is supposed to lead, in the long  distance,  
to the City of Eternal Truth, I find myself beset by a  
multitude of demons and perilous places, bearing names like  
Extensionalism, Nominalism, Positivism, Naturalism, Mechanism,  
Phenomenalism, Reductionism, Physicalism, Materialism,  
Empiricism, Scepticism, and Functionalism. 

...

After a more tolerant (permissive)  middle age, I have come to  
entertain strong opposition to *all of them*, 
perhaps partly as a result of the strong connection between a  number 
of them and the philosophical technologies which used to appeal  
to me a good deal more than they do  now"
\end{quote}
\footnote{"The Life and Opinions of Paul Grice", by Paul Grice).}

\subsection{The Beasts after Speranza}\label{GriceGripes}

This is the list of Grice's B\^etes Noires as posted to hist-analytic on 7/2/2010, together with Speranza's commentary (and to be followed mine).

\paragraph{empiricism}
Nothing wrong with it. And it is the perfect pronoun for a bete 
 noire, because ISMUS was neuter in Latin, unless it was masculine. Locke 
was  one, Grice was one, Mill was one. Grice PLAYED with being a rationalist 
alla  Kant, just to be irreverent.
I rather am scared by RATIONALISM -- but 
don't  spread the word!
 
\paragraph{extensionalism}
Well. He does say that the way he quantifies into (WoW:5)  
is enough to give an extensionalist the trembles. But the fact that he was 
so  self-conscious about logical form (e.g. his "Vacuous Names") and the fact 
that  he never used triangles and squares to symbolise serious modalities 
like poss.  and nec. makes you wonder.
 
\paragraph{functionalism}
Ned Block, the big one, lists Grice's Method in  
philosophical psychology as the most functionalist a philosopher can BE. I think  Grice 
is thinking of identity-thesis alla Smart that he need not go into. He was  
a multiple realisability functionalist of properties, not states. Etc. 
Schiffer  has tried to elucidate this in pre-apostatic writings.
 
\paragraph{materialism}
What's the mind? Never matter, or vice versa. This must have  
to do with Grice's ontological marxism: if they work, they exist. By 'they' 
he  means things like 'mental predicates'. But I don't think he was into res 
 cogitans itself. So if he wasn't a materialist he wasn't a DUALIST. And 
DUALISM  does scare me. Also ANIMISM.
 
\paragraph{mechanism}
This is the idea in "Method" that there's a mechanist  
explanation that leaves you cold when you want to say that you scratch your head  
because it itches. But the TOE is trying to reconcile these aspects. It may 
also  have to do with computer modelling: heuristic, abduction, etc. are 
difficult to  model mechanistically, but not impossible.
 
\paragraph{naturalism}
He does say that mean-N is the basis for mean-NN, so I think,  
or am pretty sure he means here a scheme that leaves VALUE out of the 
picture.  Especially concerned with the non-naturalistic basis of reason or 
rationality:  if rationality is a faculty OVER our pre-rational, natural, 
dispositions, it  cannot be "natural" herself. Etc.
 
\paragraph{nominalism}
This must be a joke unless he is thinking of those ridiculous  
theories by Scheffler. Type/token Grice always used. He uses x to symbolise  
token, X to symbolise type. He may be objecting to an extensional treatment 
of  'classes'. Etc. He may be thinking of higher-order predicate-calculus 
where we  can substantivise over properties, etc. alla Strawson, Subject and 
predicate in  logic and grammar.
 
\paragraph{phenomenalism}
This is the early early Grice and we know Carnap rejected  
this too. The opposite, Physicalism, actually scares me much more. I do love  
phenomenalism, even if inappropriate, as a good way of understanding the  
paintings of Picasso. He must be having in mind solipsism as a consequence of 
 phenomenalism, and the paradoxes of Berkeley brought to reality by Dr. 
Johnson  when kicking a stone.

\paragraph{positivism}
I should leave to Jones to expand on this. The antonym,  
negativism, is much more of a scarer. I think he must be meaning what he  
elsewhere calls, disrespectfully, the 'rednecks of Vienna' -- as if the sun  there 
were so strong! (I love Vienna).
 
\paragraph{physicalism}
Well, if this is not the antonym of phenomenalism, he must be  
meaning something alla Smart, identity thesis. Neutralism, Monism, I'm 
surprised  don't challenge him. The opposite, Spritualism, is more of a scarer, 
too.
 
\paragraph{reductionism}
We see his problem with reductive AND reductionist analysis.  
So here it's eliminationism he objects. And he does it because, once a  
linguistic botaniser, allways (sic) a linguistic botaniser. What's the good of  
having learned English if Stich and Churchland and the rest of them are 
going to  tell you that, roughly, is all false (cf. Jones on Formal versus 
Natural  Languages, though).
 
\paragraph{scepticism}
This is loose Grice. He thinks Gettier etc are too rigid. We  
know more than we care to admit. A schoolboy knows that the battle of 
Trafalgar  was in 1811, etc. So no need to be Phyrronian. I see Jones's pdf. has a 
section  on my favourite philosopher of Antiquity: Phyrro, and so I'm ready 
to  distinguish between good and bad sceptics. They were all good, honest 
people in  fact. I think it's the French philosophers, Voltaire, etc. who gave 
scepticism a  bad name.

\subsection{Positivism}

Next I'd like to address the differing perceptions of Speranza and myself in relation to Grice's attitude towards such as Carnap.

Grice thought of himself as a kind of ``ordinary language philosopher'' or at the least, a philosopher who saw some point and philosophical value in studying ``ordinary language'' as it is.
His predecessor Austin severely criticised aspects of positivism in his ``Sense and Sensibility'', and appeared an implacable opponent.

Grice's position was more moderate, several parts of his work made him appear to me less antagonistic.

\subsection{Metaphysics}

Clearly Grice was willing to induldge in Metaphysics.

Did he, would he, understand Carnap's reservations about metaphysics?
Was his metaphysics of the kind which Carnap would deprecate, or of the kind that Carnap would not call metaphysics?

\subsection{Reductionism}

We want to know whether the very specific kind of reductionism which Grice's rejected would embrace any or all of Carnap's reductionist tendencies.

Now, according to Speranxa, the specific notion of reductionism which Grice rejected was that of a conception of reductive analysis of semantics in which the reduction takes place to some single kind of entity.
This looks like it is intended to include dogmatic phenomenalism as a doctrine about semantics, which, apart from the dogmatic bit is what Carnap was doing in the Aufbau.



\chapter{Carnap}

\section{Carnap Towards Metaphysics}

I think I have a reasonable (if not scholarly) grasp of Carnap's attitude towards metaphysics.

I shall sketch it here in three parts, a technical, an intuitive, and a radical rejection.

\subsection{technical}
The technical rejection is the least important, though it often takes the headlines.
Metaphysics has sometimes been characterised as necessary knowledge of synthetic truths.
The technical rejection arises from adopting a conceptual framework in which there can be no such thing (on pain of contradiction).
This Carnap does through an understanding of the concepts of necessity and analyticity in which the only difference between them is that the former applies to propositions and the latter to those things which express propositions (of which propositions are the meanings).

The effect if this is to define out of existence a certain kind of metaphysics, viz. metaphysical propositions which are held by some philosopher to be necessary but synthetic.
The propositions do not go away, they are likely then to be regarded as analytic (at least, if the claim to necessity is sustainable).
In Carnap's terminology, the internal question becomes less controversial, but the issue is exported to the ``external question''.

\subsection{intuitive}
Carnap had, I believe, a quite genuine incomprehension of certain kinds of philosophical question, which he was therefore inclined to consider meaningless, and thus either to proscribe as metaphysical or to regard as matters of pragmatics.

\subsection{radical}
There are some radical skeptical rejections of ``metaphysics'' to be found in some positivist thinkers.
These are connected with the idea that science should not go beyond the evidence, but should confine itself to description of the observables.

How this kind of dictum is to be understood depends on how you construe observation.
In its most radical forms the observational data will be sense data, and the inference to the existence of the external world is a bit of ``metaphysics''.
Phenomenalism is, if taken in this way, a very radical rejection of an extremely broad conception of metaphysics.

Those who associate Carnap with the Aufbau and are not aware of or consider unimportant any of Carnap's later views may therefore consider Carnap to represent this kind of radical anti-metaphysics.
However, by Carnap's own account he was not such a simplistic phenomenalist even in the days of the Aufbau.

Later, his principle of tolerance and his interest not only in phenomenalistic but also in physicalistic (materialist) and theoretical languages provides further evidence against the view that Carnap was dogmatic and radical in his opposition to metaphysics.

Carnap's liberal attitude to ontology is best seen in ``Empiricism, Semantics and Ontology'', but this can be read in two ways.
Carnap is liberal about language and the ontology it presupposes.
But this is a pragmatic stance.
He aknowledges that it may be useful to adopt an ontology of abstract or theoretical objects, but he does not admit that they ``really exist''.
He doesn't even understand the question.

The middle ground which Carnap adopts here between affirming and denying what he calls the ``external questions'' is for some still an anti-metaphysical stance, tantamount to denial.

There are too levels at which ones credentials as a metphysician may be judged.
At first level, a metaphysician is someone who admits an extravagent ontology, and the anti-metaphysicians are ontological nominalists (this is a key thread in positivism).
But at the next level, which is the one you have to think of to understand Carnap, the question is not ``what exists'' but ``what ontological questions have objective (rather than conventional or pragmatic) answers''.
At this level the arch metaphysician will perhap say ``all'', but Carnap says ``none'' and so is from this perspective a radical critic of metaphysics.

\section{On Grice's Gripes}

Here are short reactions to the short accounts in Section \ref{GriceGripes}.
Longer discussions in many cases will be necessary.

First however a general reservation.

Though Carnap seems to have become progressively more liberal as he grew older, Speranza's Grice quote is painting Grice as one who turned round in middle age to become less tolerant.
That is consistent with the tone of his {\it b\^etes noires}, which looks superficially like an extended exercise in what I would call ``negative dogmatism''.
If this were as it appears then the prospects of a rapprochement between Grice and Carnap in the afterlife would take a heavy knock.
My aim must be therefore to test this interpretation, and to enquire whether Grice had or might have been nudged into, a more temperate view on these matters. 

It is not impossible that Grice is descending into a kind of Wittgensteinian denial that philosophers should have theories at all, this being a natural expectation from a proscription of ``-isms'', since that's what you get when you give a name to a theory.
Slightly less disasterously, the underlying idea might conceivably be that it is just when you think a theory important enough to give a name to it that your theory is transformed into a beast.
Is it an imprecation against taking things too far?

We have this back to front, you should read Section \ref{GriceGripes} first.

\paragraph{empiricism}

Speranza seems to be telling us that Grice wasn't really against empiricism, is that right?
 
\paragraph{extensionalism}

I think I need to have a more precise idea of what he might have been referring to here, because in my world ``extensional'', and even more its converse ``intensional'' are used for two many different things.

However, on the face of it he does mean more or less what Carnap is talking about in his ``Method of Extension and Intensions'' and if that is the case then we at least have a difference that would have to be seriously discussed before it would go away.

For my part, as our present $21^{st}$ Century proto-Carnap, I am an advocate of abstract semantics and their formulation in an extensional set theory, and I like to think that for such purposes (i.e. for an account of the semantics of arbitrary languages with sufficient detail to establish or refute the soundness of their deductive systems) set theory suffices.
I don't think Carnap thought about semantics in such a purely set theoretic way, but I believe he did think an extensional metalanguage would suffice, and I think I could persuade him that set theory is as good as any.

The place where Quine would have worried would have been where we quantify into a modal context, but Kripke and others showed that the semantics of this could be dealt with extensionally.

\paragraph{functionalism}

Speranza has this as a being specific to ``mind-brain identity'' discussions.
In which case its another thread of anti-reductionism.

I don't know whether Carnap said anything specific about this.
My guess is that he would be pragmatic on both sides, certainly allowing functionalist models.
Also allowing theories which are no so ``functionalist'' if they actually deliver the goods.

\paragraph{materialism}

What's the mind? Never matter, or vice versa. This must have  
to do with Grice's ontological marxism: if they work, they exist. By 'they' 
he  means things like 'mental predicates'. But I don't think he was into res 
 cogitans itself. So if he wasn't a materialist he wasn't a DUALIST. And 
DUALISM  does scare me. Also ANIMISM.

It sounds like Grice's antagonism here is one which might fit with Carnap.
I think they both accept that concrete ontology could be materialistic, but neither feels obliged to stick with that.
They are both ontologically liberal.
 
\paragraph{mechanism}

This is another aspect of reductionism.
What is the bottom line here?
Is the complaint about mechanistic reductions which fail to cover all there is to cover, or is it that in some domains, e.g. mental or moral, such a reduction is impossible in principle and should therefore be ruled out.

Though Carnap might here be attempting things which Grice deprecated, it is only if Grice too the dogmatic ``can't be done'' position in some domain that one might have a difficult conflict.
 
\paragraph{naturalism}

\ignore{
He does say that mean-N is the basis for mean-NN, so I think,  
or am pretty sure he means here a scheme that leaves VALUE out of the 
picture.  Especially concerned with the non-naturalistic basis of reason or 
rationality:  if rationality is a faculty OVER our pre-rational, natural, 
dispositions, it  cannot be "natural" herself. Etc.
} 

As far as I am aware Carnap did not do any serious philosophical thinking about morality.
What he shared with most logical positivist was the view that moral claims lack empirical content.
The logical positivist stance on this is rather poorly worked out, to talk of moral claims as expressions of emotion is not very satisfactoryl, but I don't sense any dogmatic stance here, if Carnap had been pressed into serious work in this area I expect he would have come up with a more plausible story.

If there were to be a sticking point between them, I suspect it might concern the objectivity of moral claims.

So I have two questions here from Grice.
The first is for clarification of the difficulty which he has with naturalism, if it were only in ethics then we would want that under a different heading.
Then as far as ethics is concerned I need to know something about Grice's position to begin to consider whether it would be a problem. 

\paragraph{nominalism}

\ignore{
This must be a joke unless he is thinking of those ridiculous  
theories by Scheffler. Type/token Grice always used. He uses x to symbolise  
token, X to symbolise type. He may be objecting to an extensional treatment 
of  'classes'. Etc. He may be thinking of higher-order predicate-calculus 
where we  can substantivise over properties, etc. alla Strawson, Subject and 
predicate in  logic and grammar.
} 

\paragraph{phenomenalism}

\ignore{
This is the early early Grice and we know Carnap rejected  
this too. The opposite, Physicalism, actually scares me much more. I do love  
phenomenalism, even if inappropriate, as a good way of understanding the  
paintings of Picasso. He must be having in mind solipsism as a consequence of 
 phenomenalism, and the paradoxes of Berkeley brought to reality by Dr. 
Johnson  when kicking a stone.
}

\paragraph{positivism}

\ignore{
I should leave to Jones to expand on this. The antonym,  
negativism, is much more of a scarer. I think he must be meaning what he  
elsewhere calls, disrespectfully, the 'rednecks of Vienna' -- as if the sun  there 
were so strong! (I love Vienna).
}

The redneck thing sounds more like a clash of cultures or even a class thing than an ideological difference.

The later Carnap's positiivism is so attenuated that it is said he prefered to call himself a logical empiricist (isn't that what people call Quine?).
Many of these -isms relate to aspects of positivism, and so it may be worth looking here just for any aspects of Carnap's positivism which are not elsewhere covered.

One key element is the place of the analytic/synthetic distinction, in which we have Grice coming out in support.
 
\paragraph{physicalism}

\ignore{
Well, if this is not the antonym of phenomenalism, he must be  
meaning something alla Smart, identity thesis. Neutralism, Monism, I'm 
surprised  don't challenge him. The opposite, Spritualism, is more of a scarer, 
too.
}

Another reductionism.
 
\paragraph{reductionism}

\ignore{
We see his problem with reductive AND reductionist analysis.  
So here it's eliminationism he objects. And he does it because, once a  
linguistic botaniser, allways (sic) a linguistic botaniser. What's the good of  
having learned English if Stich and Churchland and the rest of them are 
going to  tell you that, roughly, is all false (cf. Jones on Formal versus 
Natural  Languages, though).
}

Well I don't think Carnap is a reductionist, but we need a clearer understanding of the indictment.

\paragraph{scepticism}

\ignore{
This is loose Grice. He thinks Gettier etc are too rigid. We  
know more than we care to admit. A schoolboy knows that the battle of 
Trafalgar  was in 1811, etc. So no need to be Phyrronian. I see Jones's pdf. has a 
section  on my favourite philosopher of Antiquity: Phyrro, and so I'm ready 
to  distinguish between good and bad sceptics. They were all good, honest 
people in  fact. I think it's the French philosophers, Voltaire, etc. who gave 
scepticism a  bad name.
}

I tend to regard positivism as a mitigated or moderated scepticism, and it certainly is in my case.
In my case not really Phyrrhonean, more Carneadean.
The Pyrrhoneans were hard line negative dogmatists (in my book), and aspects of that negative dogmatism re-appear in the extreme interpretations of positivism (dogmatic phenomenalism particularly).
Carnap is not that kind of sceptic.
But he probably is my kind of sceptic.
Confirmation theory is confirmation.

So here we need to ask subtle questions about Grice's anti-sceptical stance to know what he would have against our kind of scepticism.
I don't think this is a debate about the meaning of the word ``know'', if it were, Carnap would give it to him under the principle of tolerance.


\ignore{
\subection{Reductionism}

One of Grice's B\^etes Noires is reductionism, but his gripe is about a particular kind of reductive analysis, not necessarily everything which might be called reductionism.
}

\backmatter

%\chapter*{Glossary}\label{glossary}
%\addcontentsline{toc}{chapter}{Glossary}
%
%\begin{description}
%\item[]
%\end{description}

\addcontentsline{toc}{chapter}{Bibliography}
\bibliographystyle{alpha}
\bibliography{rbj}

\addcontentsline{toc}{chapter}{Index}\label{index}
\twocolumn[]
{\small\printindex}

\end{document}
