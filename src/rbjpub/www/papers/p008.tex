% $Id: p008.tex,v 1.2 2006/10/21 17:18:21 rbj01 Exp $
% bibref{rbjp008} pdfname{p008} 
\documentclass{rbjk}

\begin{document}                                                                                   
\begin{article}
\begin{opening}  
\title{Metaphysical Positivism}
\runningtitle{Metaphysical Positivism}
\author{Roger Bishop \surname{Jones}}
\runningauthor{Roger Bishop Jones}

\begin{abstract}
An outline of a skeptical philosophy.
\end{abstract}
\end{opening}

%\def\tableofcontents{{\parskip=0pt\@starttoc{toc}}}
\setcounter{tocdepth}{4}
{\parskip-0pt\tableofcontents}

\section{Introduction}

{\it Metaphysical Positivism} is the name I have chosen for what would be a philosophical system, if only I could get it together.
I will talk of it, for convenience, within limits, as if it were a reasonably complete and coherent system.
It is a kind of radical scepticism in what I understand to be the original sense of that term.

A skeptic is one who seeks true knowledge, but fails to find it, and comes to doubt what others accept or claim as knowledge.
A dogmatist is one who believes or professes to have true knowledge.
The word in its ordinary use suggests that beliefs are held to be knowledge in spite of grounds for doubt.
A skeptic may regard all belief as dogmatic, because he comes to believe that reason for doubt can always be found.

{\it mataphysical positivism} is a radical but constructive skepticism.
An explanation of what this means is not straightforward, and constitutes the central meta-philosophical problem involved in the articulation of metaphysical positivism.
In fact, metaphysical positivism is almost entirely meta-philosophical, it is engaged in perpetuity in the search for knowledge, at each stage in its evolution the constructive element of the philosophy provides new ways of approaching the search for knowledge, and new aproaches to living without it.

\section{Radical Skepticism}

I'm using the work ``skeptic'' in what I understand to have been its original Greek sense.
A skeptic is one who seeks knowledge.
It came to be used by those whose search for {\it true} knowledge proved fruitless, and who doubted the knowledge professed by others.

When I first began to think of myself as a positivist, I supposed positivism to fall short of the most extreme skepticism (and so far as I can see most previous positivists have not thought of themeselves as radical sleptcs).
Recently however I have come to believe that positivism could possibly be, and perhaps should be, the most radical kind of skepticism.

To explain why this might be its worth considering some of the pitfalls which skeptics may encounter.
Its natural to think of a radical skeptic as one who denies all knowledge, but of course, if skeptic claims to know that neither he nor anyone else has true knowledge, then he has fallen into inconsistency.
This supposed extreme form of skepticism is in some respects dogmatic.

Some of the Greek skeptics may have made this error, others did not.
A skeptic who takes care not to fall into this trap, may find more subtle difficulties.







%{\raggedright
%\bibliographystyle{klunamed}
%\bibliography{rbj,fmu}
%} %\raggedright

\end{article}
\end{document}
