% $Id: p008.tex,v 1.14 2010/02/21 12:06:47 rbj Exp $
% bibref{rbjp008} pdfname{p008} 

\documentclass[10pt,titlepage]{book}
\usepackage{makeidx}
\usepackage[unicode,pdftex]{hyperref}
\pagestyle{headings}
\usepackage[twoside,paperwidth=5.25in,paperheight=8in,hmargin={0.75in,0.5in},vmargin={0.5in,0.5in},includehead,includefoot]{geometry}
\hypersetup{pdfauthor={Roger Bishop Jones}}
\hypersetup{colorlinks=true, urlcolor=blue, citecolor=blue, filecolor=blue, linkcolor=blue}
\usepackage{html}
\usepackage{paralist}
\usepackage{relsize}
\usepackage{verbatim}
\bodytext{BGCOLOR="#eeeeff"}
\makeindex
\newcommand{\indexentry}[2]{\item #1 #2}
\newcommand{\glossentry}[2]{\item #1 {\index #1 #2}}
\newcommand{\ignore}[1]{}
\def\Product{ProofPower}
\def\ouml{\"o}

\title{A Conversation\\{\small between}\\Carnap and Grice\\{\small as it might have been}}
\author{Roger~Bishop~Jones\\\small{and}\\J.L.~Speranza}
\date{\ }

\begin{document}
\frontmatter
                               
\begin{titlepage}
\maketitle

\vfill

%\begin{abstract}
%A speculation about what the fundamental differences between the philosophies of Rudolf Carnap and Paul Grice might have been had they survived into the twenty first century.
%\end{abstract}

\begin{centering}

{\footnotesize
\begin{verbatim}

Library of Congress Cataloging-in-Publication Data

Jones, Roger Bishop
Speranza, J. L.
    A Conversation between Grice and Carnap /
          Roger Bishop Jones and J.L. Speranza
     p. cm.
ISBN 1-40......
  1. Carnap, Rudolf.
  2. Grice, H. P. (H. Paul)

\end{verbatim}

\copyright\ Roger~Bishop~Jones and J.L.~Speranza;
}%footnotesize

\end{centering}
\end{titlepage}

\setcounter{tocdepth}{3}
{\parskip-0pt\tableofcontents}

\mainmatter

Bibliography test:
\cite{grice41}
\cite{grice86}
\cite{grice87}
\cite{grice88}
\cite{grice88a}
\cite{grice88b}
\cite{grice88c}
\cite{grice89}
\cite{grice91}
\cite{grice01}
\cite{grice57}
\cite{pears57}
\cite{speranza89}
\cite{speranza91a}
\cite{speranza91b}
\cite{speranza95}

\addcontentsline{toc}{section}{Preface}
\section*{Preface}


\ignore{
This preface is for things relevant to discussion and development of this document but not intended to be in the final work.
It is written by RBJ in the first person.
}

\ignore{
This began as a threat%
\footnote{\href{http://rbjones.com/pipermail/hist-analytic_rbjones.com/2009q2/000267.html}{http://rbjones.com/pipermail/hist-analytic_rbjones.com/2009q2/000267.html}}%
 by Roger Jones to J.L.~Speranza and other subscribers to the hist-analytic mailing list %
\footnote{\href{http://www.hist-analytic.org/}{See: www.hist-analytic.org}}%
 to prepare a piece in which he would attempt to explain to Carnap why metaphysics might not always be very bad, and how the study of Aristotle's metaphysics might have some point even for a positivist.

It is well understood that if you invite Speranza to a party, he will bring Grice along, so of course Grice is soon in the conversation, and we are learning more from Speranza %
\footnote{See: \href{http://rbjones.com/pipermail/hist-analytic_rbjones.com/2009q2/000268.html}{http://rbjones.com/pipermail/hist-analytic_rbjones.com/2009q2/000268.html}}%
 of Grice's apparent antagonism towards positivism.
This seemed to Jones to run counter to his earlier %
\footnote{See: \href{http://rbjones.com/pipermail/hist-analytic_rbjones.com/2009q2/000254.html}{http://rbjones.com/pipermail/hist-analytic_rbjones.com/2009q2/000254.html}}%
 touching upon the similarity in their pragmatic attitudes towards ontological questions, and though deferring to Speranza on Gricean exegesis, he want to explore points at which things don't seem to square up.

In thinking about {\it metaphysical} positivism Jones had in mind, Carnap of course to whose philosophy it comes closest, and Kripke, with whom it concurs only in admitting metaphysics.
Though there is I hope a serious embracing of some kind of metaphysics, it is not of Kripke's kind.
In his thinking about the source (of his positivism), and the opposition, Grice had not featured.

However, since Grice has insinuated himself into my thoughts, it seems an opportunity to introduce a symmetry.
On the one hand I present metaphysical positivism as a reconciling of Carnap's positivism with a positivistically tenable notion of metaphysics, on the other, I wonder whether the metaphysically relaxed Grice might be persuaded that in metaphysical positivism there is a kind of positivism which he might at least converse if not concur. 
Here are some thoughts along these lines.

The last wrinkle in this prelude is that, having implicated Speranza fairly thoroughly, I proposed to take him on board as a co-conspirator.
Speranza having accepted, we are now two; he appears on the front page and I must stop writing in the first person.

Blue ink is for hyperlinks in the PDF version, all will be black when printed.
}

\subsection{B\^{e}tes Noires}

We talk later about Grice's {\it b\^{e}tes noires}.
It will be helpful if we introduce some of our own.

Gladiatorial philosophy is one of them.
I'll contrast this with {\it conversational} philosophy.
The contrast is most pertinent to philosophical dialogue, in the flesh, on a mailing list, but applies also to papers and books.

Here are some features:

\begin{description}

\item[gladiatorial] \ 

In gladiatorial philosophy, one engages with one or more opponents and aims to prevail.
An outright defeat conceded by the opponent is best, but a kind of point victory is better than nothing.
It is not the purpose of the exercise to understand your opponent or to learn anything from him (you already know better than he does).
It is not necessary to criticise the position he actually holds, this would often involve considerable effort in first coming to an understanding of that position.
It is necessary only to refute what he says, understood in whatever way is most convenient for a convincing dismissal.

\item[conversational] \ 

Conversation philosophy is a game played by two or more philosophers having a common interest in some subject matter who believe that through dialogue they can come to understand that subject better than they otherwise might.
The idea is that seeing this problem from another persons point of view will enrich ones own understanding, even if that point of view is not necessarily ``better'' than your own.
To obtain this benefit it may be necessary to invest considerable effort in coming to an understanding of the others position, and in conveying to him an understanding of your own.
Not all of it, the bits which bear upon the issues.

Unless all partners are committed to this process it will not work.
\end{description}

It is my perception, that in philosophical discussion on the Internet, it is very difficult to achieve what I have called a conversational dialogue.
It is the norm rather than the exception that people from the first assume that they understand what you are saying, and one may then find many occasions to complain that an adversary criticises a position that one has never held.

This is however, not a problem exclusive to the Internet fringe.
It is pervasive in the writings of professional philosophers that a philosopher will criticise an opponent by attacking a position which he has never held.
If the philosopher is at the van of a new philosophical trend this can be done with impunity, readers sympathetic to the new philosophical trend will not trouble to check the accuracy of the picture presented of some obsolescent point of view.

\ignore{
Famous works of this kind include, Austin's ``Sense and Sensibilia'', Quine's ``Two Dogmas'' and Kripke's ``Naming and Necessity''.
}

Sometimes a failure to converse is not due to bad faith.
Two philosophers earnestly wishing to understand each other may nevertheless fail for reasons which are mysterious and may possibly be insurmountable, and we shall touch upon some of these.

The point of this little diatribe is to explain that one of our aims in discussing Carnap and Grice here is to explore whether they might have had a conversation, and whether some of the ideas of metaphysical positivism might help in providing a basis for that dialogue.
To say some things on either side which might help to make conversation possible.
Locating the points at which incomprehension, may possibly facilitate their isolation, and permit conversation elsewhere.

\chapter{Introduction}

This is a speculative discussion of the relationship between the philosophies of Carnap and Grice.

We will begin with some background on Carnap and Grice before trying to locate the areas in which they seem to be at odds.

\ignore{
However, it may be worth noting here that there is in twentieth century analytic philosophy, up to about 1970, two very broad streams into which it may be possible to slot most philosophers.
The difference between the two streams is similar to the difference between the early and the later philosophies of Wittgenstein.
Enthusiasts for Wittgenstein sometimes take him to be the fountainhead of both streams.
Less controversially, Russell may be thought of as the source of the formalistic or logicist stream of analytic philosophy, and G.E.Moore of the other stream, characterised by a greater interest in ordinary language than formal discourse.

The heat of the controversy between these streams has made conflict, rather than conversation, the order of the day.

In this scheme Carnap belongs to the formal, and Grice to the ordinary language stream.

Does this, by itself, sound the death knoll for any prospect of conversation let alone reconciliation?
}


\chapter{Carnap}

Carnap wrote an ``Intellectual Autobiography''\cite{carnap63} for the volume on his philosophy\cite{carnap63a} in the ``Library of Living Philosophers'' series, which is our primary source.
Our aim is to present sufficient background on Carnap's philosophy to make intelligible the discussion which follows of the principal differences between his ideas and those of Grice.

To this end the most important general features are outlined, and are supplemented by greater detail only in areas of possible or actual disagreement with Grice.

Carnap was a systematic philosopher with a strong sense of direction and purpose in his work which was stable throughout his career.
In pursuit of his long term goals he was eager to learn from his mistakes or from the criticisms of others, and this resulted in quite substantial changes in the strategies he adopted to secure his ends.

It is therefore useful to understand first these central themes and purposes in Carnap's philosophy, the most fundamental of which trace right back to his student days.

\section{autobiographical}

\subsection{Student Years}

Carnap was born in 1891 in Germany, and studied from 1910 to 1914 at the Universities of Jena and Freiberg/i.B.
At first his principal subjects were Philosophy and Mathematics, later Physics and Philosophy.
In philosophy he was interested in the theory of knowledge and the philosophy of science, but lost interest in philosophical teaching on logic once he had discovered the logic of Frege.

Carnap had the benefit of learning logic from Frege, in whose second course on the {\it Begriffsschrift} Carnap shared Frege's attentions with just one other student.
He learned mostly from books and conversations, rather than lectures.
The ``most fruitful inspiration'' he obtained from lectures came from Frege on symbolic logic and the foundations of mathematics.

As a student Carnap turned away from religion, finding it incompatible with the theory of evolution and determinism in physics, and took an interest in the freethinker movement in Germany and was sympathetic to ``their insistence that the scientific method was the only way of obtaining well-founded systematically coherent knowledge and with their humanistic aim of improving the life of mankind by rational means''.

Before completing his examinations Carnap's studies were interrupted by the Great War.
On his return from the war he completed his examinations and began to think about a doctoral dissertation.

We have already seen in Carnap's thought many important and enduring elements of his later mature philosophy. 

\begin{itemize}
\item[a]An interest in philosophy and science.
\item[b]A preference for physics among the sciences, because of the greater clarity and precision of its concepts.
\item[c]An interest in Frege's new mathematical logic, but not in established philosophical logic.
\item[d]An abandonment of religious belief.
\end{itemize}

\section{Chronology of Ideas}

\subsection{Student Years}

\paragraph{}


\chapter{Grice}

If Carnapians are ever grateful that Carnap was able to deliver his charming "Intellectual Autobiography", Griceans cherish Grice's  "Prejudices and predilections, which become the life and opinions of Paul Grice"\cite{grice88c} in \cite{grice88a}.
The handwritten version is in the Grice Collection, Bancroft, but most of it is repr. as \cite{grice88b} (as section II of "Reply to Richards" -- where he drops the "prejudices and predilections").
(We find that Grice is at his wittiest best in unpublication).

\section{Biographical Notes (RBJ)}

Based on \cite{grice88c}.

Grice first approached philosophy with a \emph{temperament} of irreverent, conservative, dissenting, rationalism.

Grice's initial rationalism was developed under the guidance of W.F.R.~Hardie consisted in the belief that philosophical questions are to be settled by reason, i.e., by \emph{argument}.
This he found not to be simply a matter of seeing logical connections.

His impression of Ayer's ``Language Truth and Logic'' \cite{ayer36} was of `crudities and dogmas', but the stir created by that work was interrupted by the war, and after the war things were different.

After the war an important part of Grice's work was in collaboration with his former student Peter Strawson, and the principal results of this collaboration were ``In Defence of a Dogma'' \cite{grice56} and unfinished material on predication and Aristotelian categories.

This collaboration with Strawson was very intense, and was later followed by collaborations (of varying intensity) with Austin (on Aristotle), Warnock (perception), Pears and Thomson (philosophy of action), Staal (philosophical-linguistic questions), Myro (metaphysics) and Baker (ethics).

Another important feature of Oxford philosophy during this period was Austin's ``play group'' (so called by Grice), which was regarded by many as the hot-bed of the ``Oxford School'' of ordinary language philosophy.
Grice emphasises the diversity of Oxford philosophy at that time, there being multiple similar groups (influenced by Ryle or by Wittgenstein for example) and considerable diversity within these groups.  

\section{Central Themes}

We draw here on Grice's \cite{grice} in the ``Retrospective Epilogue'' of which he lists some of the main strands in his work.

\begin{enumerate}
\item Philosophy of Perception
\begin{itemize}
\item The causal analysis of perception.
\item The experential quality of different senses.
\item The analysis of statements describing objects of perception.
\end{itemize}

\item Defence of the Analytic/Synthetic distinction.

\item Defence of the rights of the ordinary man and common sense vis-a-vis the professional philosopher.

\item Meaning:
\begin{itemize}
\item ``necessary to distinguish between a notion of meaning which is relativised to users of words or expressions and one that is not''
\item un-relativised meaning must be understood in terms of the relativised meaning
\end{itemize}

\item Further meaning distinctions:
\begin{itemize}
\item conventional and non-conventional meaning
\item what is asserted and what is implicated
\end{itemize}

\item Parallels between language and other rational activities.

\item That phrases like ``the King of France'' should be considered as genuinely rather than ostensibly referential.

\item That formal logic can be amended to meet the above requirement for phrases to be genuinely referential.
\end{enumerate}

\chapter{Points of Contention}

\section{Metaphysics}

The single best known feature of positivist philosophies is its repudiation of metaphysics.
Since Grice defended and engaged in metaphysics, one might suspect a fundamental difficulty between him and Carnap in this.

Carnap did concur with the positivist tradition in some such repudiation, but this was in the context of very specific and narrow conception of metaphysics, by which the repudiation is severely mitigated.

It is in relation to ontology that one can most readily see that Carnap and Grice are closer on metaphysics than might have been expected, for both these philosophers espoused pragmatic attitudes in matters of ontology.
Grice's pragmatic ontological tolerance seems is incompatible with a belief in the objective truth of ontological claims, and in default of such belief Grice in this matter can probably not be considered as engaging in metaphysics as this is to be understood in Carnap's proscription.

That still leaves scope for dischord on other matters metaphysical, but even before a detailed examination we can outline other ways in which Grice might have undertaken metaphysics without falling out with Carnap.
Two further very substantial categories of metaphysics fall outside of Carnap's conception, viz. exegetical and descriptive metaphysics.
In both of these find an activity of analysis applied to some empirical subject matter.
In the first case the historical subject matter is the philosophy of some previous metaphysician, in Grice's case Plato, Aristotle and Kant come to mind.
In the second case the subject matter is ordinary language.
Both these kinds of metaphysical investigation may be considered as kinds of applied nomologico-deductive analysis.
These involve the construction of some kind of model of the metaphysics (as proposed by some historical figure, or as presupposed in normal usage of everyday language).
The models and their analysis fall properly within the scope of analytic philosophy as conceived by Carnap.
Only the question of the fidelity of the model falls outside.
The situation is therefore analogous to the relationship which Carnap envisaged between philosophy and science, which itself is not dissimilar to that between theoretial and experimental science, the former being deductive and the latter empirical science.

What, allows such metaphysicians to escape from Carnap's obloquay is their abstention from asserting supposedly objective metaphysical theses about reality.
Provided that they merely {\it analyse} the metaphysic discoverable, in some historical figure or in some aspect of ordinary usage, then they risk at worst slipping into science rather than what Carnap considers as metaphysics.

From this sketch we proceed first to spell out more carefully the narrow conception of metaphysics which exited Carnap's disdain, and then to examine in slightly greater detail whether Gricean metaphysics crosses the line.

\subsection{Carnap Towards Metaphysics}

A sketch in three parts, a technical, an intuitive, and a radical rejection.

\subsubsection{technical}
The technical rejection is the least important, though it often takes the headlines.
Metaphysics has sometimes been characterised as necessary knowledge of synthetic truths.
The technical rejection arises from adopting a conceptual framework in which there can be no such thing (on pain of contradiction).
This Carnap does through an understanding of the concepts of necessity and analyticity in which the only difference between them is that the former applies to propositions and the latter to those things which express propositions (of which propositions are the meanings).

The effect if this is to define out of existence a certain kind of metaphysics, viz. metaphysical propositions which are held by some philosopher to be necessary but synthetic.
The propositions do not go away, they are likely then to be regarded as analytic (at least, if the claim to necessity is sustainable).
In Carnap's terminology, the internal question becomes less controversial, but the issue is exported to the ``external question''.

\subsubsection{intuitive}
Carnap had, I believe, a quite genuine incomprehension of certain kinds of philosophical question, which he was therefore inclined to consider meaningless, and thus either to proscribe as metaphysical or to regard as matters of pragmatics.

\subsubsection{radical}
There are some radical sceptical rejections of ``metaphysics'' to be found in some positivist thinkers.
These are connected with the idea that science should not go beyond the evidence, but should confine itself to description of the observables.

How this kind of dictum is to be understood depends on how you construe observation.
In its most radical forms the observational data will be sense data, and the inference to the existence of the external world is a bit of ``metaphysics''.
Phenomenalism is, if taken in this way, a very radical rejection of an extremely broad conception of metaphysics.

Those who associate Carnap with the Aufbau and are not aware of or consider unimportant any of Carnap's later views may therefore consider Carnap to represent this kind of radical anti-metaphysics.
However, by Carnap's own account he was not such a simplistic phenomenalist even in the days of the Aufbau.

Later, his principle of tolerance and his interest not only in phenomenalistic but also in physicalistic (materialist) and theoretical languages provides further evidence against the view that Carnap was dogmatic and radical in his opposition to metaphysics.

Carnap's liberal attitude to ontology is best seen in ``Empiricism, Semantics and Ontology''\cite{carnap50,carnap47}, but this can be read in two ways.
Carnap is liberal about language and the ontology it presupposes.
But this is a pragmatic stance.
He acknowledges that it may be useful to adopt an ontology of abstract or theoretical objects, but he does not admit that they ``really exist''.
He doesn't even understand the question.

The middle ground which Carnap adopts here between affirming and denying what he calls the ``external questions'' is for some still an anti-metaphysical stance, tantamount to denial.

There are too levels at which ones credentials as a metaphysician may be judged.
At first level, a metaphysician is someone who admits an extravagant ontology, and the anti-metaphysicians are ontological nominalists (this is a key thread in positivism).
But at the next level, which is the one you have to think of to understand Carnap, the question is not ``what exists'' but ``what ontological questions have objective (rather than conventional or pragmatic) answers''.
At this level the arch metaphysician will perhaps say ``all'', but Carnap says ``none'' and so is from this perspective a radical critic of metaphysics.

\subsection{Grice the Metaphysician}

Clearly Grice was willing to indulge in Metaphysics.

Did he, would he, understand Carnap's reservations about metaphysics?
Was his metaphysics of the kind which Carnap would deprecate, or of the kind that Carnap would not call metaphysics?


\section{Grice's Demons and Perilous Places}

\begin{quote}
"As I thread my way unsteadily along the   tortuous mountain 
path which is supposed to lead, in the long  distance,  
to the City of Eternal Truth, I find myself beset by a  
multitude of demons and perilous places, bearing names like  
Extensionalism, Nominalism, Positivism, Naturalism, Mechanism,  
Phenomenalism, Reductionism, Physicalism, Materialism,  
Empiricism, Scepticism, and Functionalism. 

...

After a more tolerant (permissive)  middle age, I have come to  
entertain strong opposition to *all of them*, 
perhaps partly as a result of the strong connection between a  number 
of them and the philosophical technologies which used to appeal  
to me a good deal more than they do  now"
\end{quote}
\footnote{"The Life and Opinions of Paul Grice", by Paul Grice).}

\subsection{General Discussion}

We will take Grice's list seriously for it contains a number of -isms of which Carnap might be accused.
But this declaration by itself is not a good basis for the investigation, for we have so little information about exactly what Grice is abhorring or why.
Furthermore, he is talking about ``names like'', so this is not to be taken as a definitive list of deprecated doctrines.

We find (elsewhere, more detail to be included) that these are all regarded as minimalisms, and it seems possible that Grice's principal objection is to these as dogmatic minimalisms (this needs checking out).
He also talks specifically about reductionism, and several of the beasts are kinds of reductionism.

It is moot whether going through them one by one is the best way to deal with them, since this risks repetition and may fail to bring out common features.

First however a general reservation.

Though Carnap seems to have become progressively more liberal as he grew older, Speranza's Grice quote is painting Grice as one who turned round in middle age to become less tolerant.
That is consistent with the tone of his {\it b\^etes noires}, which looks superficially like an extended exercise in what I would call ``negative dogmatism''.
If this were as it appears then the prospects of a rapprochement between Grice and Carnap in the afterlife would take a heavy knock.
My aim must be therefore to test this interpretation, and to enquire whether Grice had or might have been nudged into, a more temperate view on these matters. 

It is not impossible that Grice is descending into a kind of Wittgensteinian denial that philosophers should have theories at all, this being a natural expectation from a proscription of ``-isms'', since that's what you get when you give a name to a theory.
Slightly less disastrously, the underlying idea might conceivably be that it is just when you think a theory important enough to give a name to it that your theory is transformed into a beast.
Is it an imprecation against taking things too far?

[This now looks less likely.  Whether or not it need be discussed is moot.]

\subsection{Definitions of the -isms}

This is a set run off by JLS for consideration.

\paragraph{scepticism} For any claim C, there is an anti-claim C' which is equally  
grounded on the available evidence. Therefore rendering the uttering of  claim 
C an otiose thing to do. Varieties to consider here: Phyrronism:  which 
yield to silence. Carneadism: which introduce probability as a licensing  
operator. So that 'Probably, C' becomes consistent with "Probably C'". 
 
\paragraph{naturalism} For any claim C in any realm of philosophical discourse, it is  
possible and indeed recommendable if not mandatory that the observational 
terms  mentioned be of the realm of a (favoured) notion of nature. To explain 
nature  naturally is natural enough. Naturalism becomes interesting when it 
aims at  explaining, for example, psychology (behaviourism) or, better, 
morality.  Flourishing-ethics or virtue ethics (teleological) as naturalist. 
Moore's  critique of the 'naturalist' fallacy. The twin monster, 
non-naturalism not  faring any better.
 
\paragraph{nominalism} A claim C is nominalist if it excludes talk of 'classes', or  
allows them only extensionally defined.
As an answer to the alleged problem of 'universalia', contrasts with Realism.
As a rejection of abstract entities it is Quine's claim to infame.
 
\paragraph{functionalism} Any state of a functional unity (or organism) is a  
functional one if it correlates some input (usually in terms of perceptual  sensory 
data) with an output (in terms of observable behaviour). Functionalism  as a 
way to explain 'mentalistic' talk which is no behaviourism and no dualism.  
Represented by Lewis and the early Grice of "Method in philosophical  
psychology". 
 
\paragraph{empiricism} Any claim C is to be derived from sense-datum reports. The  
backing of the claim is seen as a matter of 'de iure' concerns, not the 
'facto'.  It is not claimed that all our claims in fact derive from the senses 
('nihil est  in intellectu quod non fuerit prius in sensu") but that they 
should do be.  Locke-Hume-Berkeley as the empiricist triad. Mill as a later date 
one. (cfr.  Grice to the Mill).
 
\paragraph{reductionism} A statement is reductionist if it provides necessary and  
sufficient conditions of the alleged analysans in terms of an analysandum which 
 does not make reference to the realm in which the analysans feels best at 
home.  One reduces psychological talk to physiological talk, for example. 
One reduces  logic to algebra, or algebra to logic. We disallow mere 
'reductive' analysis,  which are the bread-and-butter of the philosopher, and aim at 
Eliminationism,  for we find that it's only by eliminating the original 
concept that progress is  made in philosophy.
 
\paragraph{mechanism} A statement of teleology is properly reduced to a mechanist  
claim C iff C contains only physiological physicalist terms to it. A mechanist  
claim, unlike a teleological one which it reduces, is, more importantly,  
verifiable and should be at least verified, or alternatively falsified. 
George's  doctrines in psychology and Mace's are what both Carnap and Grice may 
be having  in mind here. Cf. the demon of Emergentism: the idea that biology 
involves  non-mechanist claims as a matter of necessity.
 
\paragraph{materialism} A claim C is materialist if it disallows terms which are  
non-observational. Even in the functionalist scheme, the input and the output  
should be understood materialistically in this sense. As a metaphysical or  
physical claim it ventures to stick to a proper account of what there is in  
these terms. It's the old Cartesian project of the res extensa (vs. res  
cogitans) made feasible.
 
\paragraph{extensionalism} A claim is extensionalist if it disallows intensional  
isomorphism of the type that Carnap and the latter Grice find appealing (Grice's 
 only detailed attack in "Reply to Richards" is against extensionalism).
 
\paragraph{phenomenalism} A claim is phenomenalist if rendered in phenomenal terms,  
where this is properly or strictly understood as observer's predicates  
(perspectivism). The phsyicist's theory-laden observational claims fall under  
this category, too. A consequence of empiricism.
 
\paragraph{physicalism} A claim is physicalist if 'dressed' in third-person objective  
predicates from the language of physics, or at least in terms of the 
protocol  statements that the phsyicist formulates in the lab. It becomes 
interesting when  meant to reduce psychological statements in such terms. Watson's 
behaviourism as  a sort of physicalism.
 
\paragraph{positivism} A claim is positivist if it aims at the unified science. The  
varieties of positivism center around logical positivism -- there is a 
hierarchy  of our cathedral of learning, and the positivist spirit is respected 
when this  hiearchy is respected and followed. The anti-positivist credo being 
either a  non-reductionist primitivism, where each allegedly scientific 
claim is made not  to be reduced to the hierarchically lower theory that 
sustains it.
 
\subsection{Some Preliminary Observations on the -ism}

There follows an interleaving, pro-tem, until a better structure is found.
Under each paragraph we have first Speranza's comments then mine.


\paragraph{empiricism}\ 

JLS

Nothing wrong with it. And it is the perfect pronoun for a bete 
 noire, because ISMUS was neuter in Latin, unless it was masculine. Locke 
was  one, Grice was one, Mill was one. Grice PLAYED with being a rationalist 
alla  Kant, just to be irreverent.
I rather am scared by RATIONALISM -- but 
don't  spread the word!
 
RBJ

Speranza seems to be telling us that Grice wasn't really against empiricism, is that right?
If so, why is it on his list?
 
\paragraph{extensionalism}\ 

JLS

Well. He does say that the way he quantifies into (WoW:5)  
is enough to give an extensionalist the trembles. But the fact that he was 
so  self-conscious about logical form (e.g. his "Vacuous Names") and the fact 
that  he never used triangles and squares to symbolise serious modalities 
like poss.  and nec. makes you wonder.

RBJ

I think I need to have a more precise idea of what he might have been referring to here, because in my world ``extensional'', and even more its converse ``intensional'' are used for two many different things.

However, on the face of it he does mean more or less what Carnap is talking about in his ``Method of Extension and Intensions'' and if that is the case then we at least have a difference that would have to be seriously discussed before it would go away.

For my part, as our present $21^{st}$ Century proto-Carnap, I am an advocate of abstract semantics and their formulation in an extensional set theory, and I like to think that for such purposes (i.e. for an account of the semantics of arbitrary languages with sufficient detail to establish or refute the soundness of their deductive systems) set theory suffices.
I don't think Carnap thought about semantics in such a purely set theoretic way, but I believe he did think an extensional metalanguage would suffice, and I think I could persuade him that set theory is as good as any.

The place where Quine would have worried would have been where we quantify into a modal context, but Kripke and others showed that the semantics of this could be dealt with extensionally.

\paragraph{functionalism}\ 

JLS

Ned Block, the big one, lists Grice's Method in  
philosophical psychology as the most functionalist a philosopher can BE.
I think  Grice is thinking of identity-thesis \'a la Smart that he need not go into.
He was a multiple realisability functionalist of properties, not states. Etc. 
Schiffer has tried to elucidate this in pre-apostatic writings.

RBJ

Speranza has this as a being specific to ``mind-brain identity'' discussions.
In which case its another thread of anti-reductionism.

I don't know whether Carnap said anything specific about this.
My guess is that he would be pragmatic on both sides, certainly allowing functionalist models.
Also allowing theories which are no so ``functionalist'' if they actually deliver the goods.

\paragraph{materialism}\ 

JLS

What's the mind? Never matter, or vice versa. This must have  
to do with Grice's ontological Marxism: if they work, they exist. By 'they' 
he  means things like 'mental predicates'. But I don't think he was into res 
 cogitans itself. So if he wasn't a materialist he wasn't a DUALIST. And 
DUALISM  does scare me. Also ANIMISM.

RBJ

It sounds like Grice's antagonism here is one which might fit with Carnap.
I think they both accept that concrete ontology could be materialistic, but neither feels obliged to stick with that.
They are both ontologically liberal.
 
\paragraph{mechanism}\ 

JLS

This is the idea in "Method" that there's a mechanist  
explanation that leaves you cold when you want to say that you scratch your head  
because it itches. But the TOE is trying to reconcile these aspects. It may 
also  have to do with computer modelling: heuristic, abduction, etc. are 
difficult to  model mechanistically, but not impossible.
 
RBJ

This is another aspect of reductionism.
What is the bottom line here?
Is the complaint about mechanistic reductions which fail to cover all there is to cover, or is it that in some domains, e.g. mental or moral, such a reduction is impossible in principle and should therefore be ruled out.

Though Carnap might here be attempting things which Grice deprecated, it is only if Grice too the dogmatic ``can't be done'' position in some domain that one might have a difficult conflict.
 
\paragraph{naturalism}\ 

JLS

He does say that mean-N is the basis for mean-NN, so I think,  
or am pretty sure he means here a scheme that leaves VALUE out of the 
picture.  Especially concerned with the non-naturalistic basis of reason or 
rationality:  if rationality is a faculty OVER our pre-rational, natural, 
dispositions, it  cannot be "natural" herself. Etc. 

RBJ

As far as I am aware Carnap did not do any serious philosophical thinking about morality.
What he shared with most logical positivist was the view that moral claims lack empirical content.
The logical positivist stance on this is rather poorly worked out, to talk of moral claims as expressions of emotion is not very satisfactorily, but I don't sense any dogmatic stance here, if Carnap had been pressed into serious work in this area I expect he would have come up with a more plausible story.

If there were to be a sticking point between them, I suspect it might concern the objectivity of moral claims.

So I have two questions here from Grice.
The first is for clarification of the difficulty which he has with naturalism, if it were only in ethics then we would want that under a different heading.
Then as far as ethics is concerned I need to know something about Grice's position to begin to consider whether it would be a problem. 

\paragraph{nominalism}\ 

JLS

This must be a joke unless he is thinking of those ridiculous  
theories by Scheffler. Type/token Grice always used. He uses x to symbolise  
token, X to symbolise type. He may be objecting to an extensional treatment 
of  'classes'. Etc. He may be thinking of higher-order predicate-calculus 
where we  can substantivise over properties, etc. \'a la Strawson, Subject and 
predicate in  logic and grammar.

RBJ

I don't see that Carnap can be considered a nominalist, even though he engaged with Quine and Goodman in some of their excursions into nominalistic mathematics.
In this Carnap was a fellow traveller, given immunity by his principle of tolerance.
The dominant characteristic of his ontology is its pragmatism and flexibility.

\paragraph{Phenomenalism}\ 

JLS

This is the early early Grice and we know Carnap rejected  
this too. The opposite, Physicalism, actually scares me much more. I do love  
Phenomenalism, even if inappropriate, as a good way of understanding the  
paintings of Picasso. He must be having in mind solipsism as a consequence of 
 Phenomenalism, and the paradoxes of Berkeley brought to reality by Dr. 
Johnson  when kicking a stone.

RBJ

As in all these metaphysical matters, Carnap eviscerates them as metaphysics.
He accepts or rejects these ontologies or reductive schemes on a pragmatic basis, never engaging with the question which is objectively true.
These are his ``external questions'' to which he needs no answer.

\paragraph{positivism}\ 

JLS

I should leave to Jones to expand on this. The antonym,  
negativism, is much more of a scarer. I think he must be meaning what he  
elsewhere calls, disrespectfully, the 'rednecks of Vienna' -- as if the sun  there 
were so strong! (I love Vienna).

RBJ

The redneck thing sounds more like a clash of cultures or even a class thing than an philosophical difference.

The later Carnap's positivism is so attenuated that it is said he preferred to call himself a logical empiricist (isn't that what people call Quine?).
Many of these -isms relate to aspects of positivism, and so it may be worth looking here just for any aspects of Carnap's positivism which are not elsewhere covered.

One key element is the place of the analytic/synthetic distinction, on which we have Grice coming out in support.
 
\paragraph{Physicalism}\ 

JLS

Well, if this is not the antonym of Phenomenalism, he must be  
meaning something \'a la Smart, identity thesis. Neutralism, Monism, I'm 
surprised  don't challenge him. The opposite, Spiritualism, is more of a scarer, 
too.

RBJ

Another reductionism.
Whether a conflict arises here depends on whether Carnap's objection is moderate or extreme.
Does he merely reject a doctrinaire and dogmatic metaphysical physicalism.
Or does he also reject a pragmatic, opportunist, non-committal manner of speaking physicalism?

\paragraph{reductionism}\ 

JLS

We see his problem with reductive AND reductionist analysis.  
So here it's eliminationism he objects. And he does it because, once a  
linguistic botaniser, allways (sic) a linguistic botaniser. What's the good of  
having learned English if Stich and Churchland and the rest of them are 
going to  tell you that, roughly, is all false (cf. Jones on Formal versus 
Natural  Languages, though).

RBJ

Well I don't think Carnap is a reductionist, but we need a clearer understanding of the indictment.

\paragraph{scepticism}\ 

JLS

This is loose Grice. He thinks Gettier etc are too rigid. We  
know more than we care to admit. A schoolboy knows that the battle of 
Trafalgar  was in 1811, etc. So no need to be Pyrrhonean. I see Jones's pdf. has a 
section  on my favourite philosopher of Antiquity: Pyrrho, and so I'm ready 
to  distinguish between good and bad sceptics. They were all good, honest 
people in  fact. I think it's the French philosophers, Voltaire, etc. who gave 
scepticism a  bad name.

RBJ

I tend to regard positivism as a mitigated or moderated scepticism, and it certainly is in my case.
In my case not really Pyrrhonean, more Carneadean.
The Pyrrhoneans were hard line negative dogmatists (in my book), and aspects of that negative dogmatism re-appear in the extreme interpretations of positivism (dogmatic Phenomenalism particularly).
Carnap is not that kind of sceptic.
But he probably is my kind of sceptic.
Confirmation theory is confirmation.

So here we need to ask subtle questions about Grice's anti-sceptical stance to know what he would have against our kind of scepticism.
I don't think this is a debate about the meaning of the word ``know'', if it were, Carnap would give it to him under the principle of tolerance.

\ignore{
\subsection{Positivism}

Next I'd like to address the differing perceptions of Speranza and myself in relation to Grice's attitude towards such as Carnap.

Grice thought of himself as a kind of ``ordinary language philosopher'' or at the least, a philosopher who saw some point and philosophical value in studying ``ordinary language'' as it is.
His predecessor Austin severely criticised aspects of positivism in his ``Sense and Sensibility'', and appeared an implacable opponent.

Grice's position was more moderate, several parts of his work made him appear to me less antagonistic.

\subsection{Reductionism}\index{reductionism}

We want to know whether the very specific kind of reductionism which Grice's rejected would embrace any or all of Carnap's reductionist tendencies.

Now, according to Speranza, the specific notion of reductionism which Grice rejected was that of reductive analysis of semantics in which the reduction takes place to some single kind of entity.
This looks like it is intended to include dogmatic Phenomenalism as a doctrine about semantics, which, apart from the dogmatic bit is what Carnap was doing in the Aufbau.
}


\backmatter

%\chapter*{Glossary}\label{glossary}
%\addcontentsline{toc}{chapter}{Glossary}
%
%\begin{description}
%\item[]
%\end{description}

\addcontentsline{toc}{chapter}{Bibliography}
\bibliographystyle{alpha}
\bibliography{rbj}

\addcontentsline{toc}{chapter}{Index}\label{index}
{\twocolumn[]
{\small\printindex}}

\vfill

\tiny{
Started 2010-01

Last Change $ $Date: 2010/02/21 12:06:47 $ $

\href{http://www.rbjones.com/rbjpub/www/papers/p008.pdf}{http://www.rbjones.com/rbjpub/www/papers/p008.pdf}

Draft $ $Id: p008.tex,v 1.14 2010/02/21 12:06:47 rbj Exp $ $
}%tiny

\end{document}
