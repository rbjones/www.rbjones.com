% $Id: p008.tex,v 1.3 2007/02/20 22:14:05 rbj01 Exp $
% bibref{rbjp008} pdfname{p008} 
\documentclass{rbjk}

\begin{document}                                                                                   
\begin{article}
\begin{opening}  
\title{A Utopian Philosophical System}
%\runningtitle{Metaphysical Positivism}
\author{Roger Bishop \surname{Jones}}
\runningauthor{Roger Bishop Jones}

\begin{abstract}
An outline of a philosophical system in three layers.
The most fundamental layer is a systematic sceptical philosophy.
The central layer is a positive philosophy covering a broad range of philosophical topics.
The upper layer consists of a utopian method and a utopian conception built on the underlying positive philosophy.
\end{abstract}
\end{opening}

%\def\tableofcontents{{\parskip=0pt\@starttoc{toc}}}
\setcounter{tocdepth}{4}
{\parskip-0pt\tableofcontents}

\section{Introduction}

This document is a linearised account of the philosophical system partially presented in convoluted hypermedia at the RBJones.com website.

The philosophy is presented in three layers.
The most fundamental layer has its roots in pyrrhonean scepticism, but is termed {\it naive} for reasons which will be explained shortly.

The next layer is the closest we have to systematic philosophy in the analytic tradition, but seeks also to encompass features more closely associated with romantic philosophy.

The uppermost layer is utopian,

\section{Naive Philosophy}

\subsection{Naivety}

\subsection{Skepticisms}

I'm using the work ``skeptic'' in what I understand to have been its original Greek sense.
A skeptic is one who seeks knowledge.
It came to be used by those whose search for {\it true} knowledge proved fruitless, and who doubted the knowledge professed by others.

When I first began to think of myself as a positivist, I supposed positivism to fall short of the most extreme skepticism (and so far as I can see most previous positivists have not thought of themeselves as radical sleptcs).
Recently however I have come to believe that positivism could possibly be, and perhaps should be, the most radical kind of skepticism.

To explain why this might be its worth considering some of the pitfalls which skeptics may encounter.
Its natural to think of a radical skeptic as one who denies all knowledge, but of course, if skeptic claims to know that neither he nor anyone else has true knowledge, then he has fallen into inconsistency.
This supposed extreme form of skepticism is in some respects dogmatic.

Some of the Greek skeptics may have made this error, others did not.
A skeptic who takes care not to fall into this trap, may find more subtle difficulties.

\section{Positive Philosophy}

The middle layer contains philosophy addressing a broad range of topics and falls into two principle parts, metaphysical and existential positivism.

\subsection{Metaphysical Positivism}

{\it Metaphysical Positivism} is the name I have chosen for what would be a philosophical system, if only I could get it together.
I will talk of it, for convenience, within limits, as if it were a reasonably complete and coherent system.
It is a kind of radical scepticism in what I understand to be the original sense of that term.

A skeptic is one who seeks true knowledge, but fails to find it, and comes to doubt what others accept or claim as knowledge.
A dogmatist is one who believes or professes to have true knowledge.
The word in its ordinary use suggests that beliefs are held to be knowledge in spite of grounds for doubt.
A skeptic may regard all belief as dogmatic, because he comes to believe that reason for doubt can always be found.

{\it mataphysical positivism} is a radical but constructive skepticism.
An explanation of what this means is not straightforward, and constitutes the central meta-philosophical problem involved in the articulation of metaphysical positivism.
In fact, metaphysical positivism is almost entirely meta-philosophical, it is engaged in perpetuity in the search for knowledge, at each stage in its evolution the constructive element of the philosophy provides new ways of approaching the search for knowledge, and new aproaches to living without it.

\section{Utopian Philosophy}

The uppermost layer comes in two parts.
The first part is methodological, considering methodological questions relevant to utopianism.
The second part is utopian, and is conducted along the lines suggested by the methodological material

\subsection{Utopian Factasy}

\subsection{The Factasian Utopia}



%{\raggedright
%\bibliographystyle{klunamed}
%\bibliography{rbj,fmu}
%} %\raggedright

\end{article}
\end{document}
