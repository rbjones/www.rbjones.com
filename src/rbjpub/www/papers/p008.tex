% $Id: p008.tex,v 1.6 2008/09/09 20:21:11 rbj Exp $
% bibref{rbjp008} pdfname{p008} 
\documentclass{rbjk}
\usepackage{makeidx}
\newcommand{\ignore}[1]{}
\usepackage[unicode,pdftex]{hyperref}
\hypersetup{pdfauthor={Roger Bishop Jones}}
\hypersetup{colorlinks=true, urlcolor=black, citecolor=black, filecolor=black, linkcolor=black}

\begin{document}                                                                                   
\begin{article}
\begin{opening}  
\title{Constructive Scepticism}
\author{Roger Bishop \surname{Jones}}
\runningauthor{Roger Bishop Jones}

\begin{abstract}
A sceptical position.
\end{abstract}
\end{opening}

%\def\tableofcontents{{\parskip=0pt\@starttoc{toc}}}
\setcounter{tocdepth}{4}
{\parskip-0pt\tableofcontents}

\section{Introduction}

Sceptical ideas appear very early in the history of ``Western'' philosophy, which is generally considered to have begun in Greece around 600 years before Christ.
It is the distinctive feature of this philosophical tradition that it is, or purports to be, rational.
Mathematics is also held by mathematicians to have begun at roughly the same time, and is distinguished from (mere) arithmetic (and more generally the Greek ``logistic'') by being a theoretical rather than a practical discipline.
The litmus test for the practice of this kind of mathematics is the concepts ``theorem'' and ``proof'' which turn mathematics into a deductive science.

The startling feature of deductive mathematics (as later to be codified in ``the axiomatic method'') is its very high levels of reliability and stability.
To a high degree mathematical proofs inspire confidence, and results once established by uncontroversial proofs are rarely later overturned.
Philosophers have frequently envied mathematics this definitive rigour, and many have sought to philosophise in like manner, with little success.

The failure of philosophy to emulate the rigour of mathematics may be analysed into two main elements.
The first is that philosophers sometimes seek knowledge about the physical world, which they base to a greate or lesser extent upon observation, but the process of observation is unreliable, and the results even if true, provide an insufficient basis for deductive inference to the kind of general laws which were of interest to early philosophers.
The second is that deductive inference, while highly productive and reliable in certain domains (notably mathematics) proves less satisfactory in other domains (and perhaps least satisfactory of all in philosophy).

These two areas of weakness may be associated with the earliest known sceptical thinking, in the writings of Heraclitus and Zeno.
Heraclitus was sceptical about the senses and build his philosophy on this scepticism.



Pyrrhonian scepticism is supposed to arise from the failure of an earnest and ongoing search for truth.
However, the literature is unconvincing, appearing to represent an exclusive interest in the refutation of dogma, with little evidence of a genuine attempt to seek truth.

In general, philosophical movements which are claimed to be sceptical have similar characteristics.
When elements of scepticism are combined with significant amounts of positive theory the result is rarely called scepticism, except perhaps by its opponents.
Descartes' system in which systematic and radical scepticism lays the ground for a dogmatic metaphysical system is called, simply, Cartesianism.
Hume's radical scepticism is called ``empiricism'' or ``positivism''.

There were before Hume in the seventeenth century, philosophers such as Mersenne and Gassendi, who combined a respect for sceptical arguments with a positive attitude toward science, and sought to reconcile these in a mitigated or constructive scepticism.
These may be thought of a precursors of many subsequent philosophies in which the same ideals are sought but are described not as varieties of scepticism but rather as kinds of empiricism, positivism or pragmatism.

My enterprise here is broadly similar in character.
My primary motivation is speculation about our futures which encompasses consideration of how various kinds of knowledge can best be sought and applied.
In this it is recognised that ill-founded dogma has a disutility offsetting the benefits of applicable knowledge, and that a balanced and open-minded attitude may be of greatest value.

My concern in this essay is only with scepticism and how it can be encompassed within a positive weltanschauung.

\section{Varieties of Scepticism}

 

%{\raggedright
%\bibliographystyle{klunamed}
%\bibliography{rbj,fmu}
%} %\raggedright

\end{article}
\end{document}
