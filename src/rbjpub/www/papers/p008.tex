% $Id: p008.tex,v 1.8 2010/02/06 16:19:48 rbj Exp $
% bibref{rbjp008} pdfname{p008} 
\documentclass{article}
\usepackage{makeidx}
\newcommand{\ignore}[1]{}
\usepackage[unicode,pdftex]{hyperref}
\hypersetup{pdfauthor={Roger Bishop Jones}}
\hypersetup{colorlinks=true, urlcolor=blue, citecolor=blue, filecolor=black, linkcolor=blue}

\title{A Conversation Between Grice and Carnap}
\author{Roger Bishop Jones}
\date{\ }

\begin{document}
                               
\begin{titlepage}
\maketitle
\begin{abstract}
A speculation about what the fundamental differences between the philosophies of Rudolf Carnap and Paul Grice might have been had they survived into the twenty first century.
\end{abstract}

\vfill

\begin{centering}

\vfill

\footnotesize{
Started 

Last Change $ $Date: 2010/02/06 16:19:48 $ $

\href{http://www.rbjones.com/rbjpub/www/papers/p008.pdf}{http://www.rbjones.com/rbjpub/www/papers/p008.pdf}

Draft $ $Id: p008.tex,v 1.8 2010/02/06 16:19:48 rbj Exp $ $

\copyright\ Roger Bishop Jones;

}%footnotesize
\end{centering}

\end{titlepage}

\newpage
%\def\tableofcontents{{\parskip=0pt\@starttoc{toc}}}
\setcounter{tocdepth}{4}
{\parskip-0pt\tableofcontents}

\section{Introduction}

This is a speculative discussion of the relationship between the philosophies of Carnap and Grice.
Structure a bit fluid at present.

\section{Carnap and Grice}\label{CARNAPANDGRICE}

This began as \href{http://rbjones.com/pipermail/hist-analytic_rbjones.com/2009q2/000267.html}{a threat on my part} to J.L. Speranza and subscribers to the \href{http://www.hist-analytic.org/}{hist-analytic mailing list} to prepare a piece in which I attempt to explain to Carnap why metaphysics might not always be very bad and how the study of Aristotle's metaphysics might have some point even for a positivist.

It is well understood that if you invite Speranza to a party, he will bring Grice along, so of course Grice is soon in the conversation and we are \href{http://rbjones.com/pipermail/hist-analytic_rbjones.com/2009q2/000268.html}{learning more from Speranza} of Grice's apparent antagonism towards positivism.
This seemed to me to run counter to \href{http://rbjones.com/pipermail/hist-analytic_rbjones.com/2009q2/000254.html}{my own earlier} touching upon the similarity in their pragmatic attitudes towards ontological questions, and though I defer to Speranza on Gricean exegesis, I want to explore points at which things don't seem to square up.

In thinking about metaphysical positivism I had in mind, Carnap of course to whose philosophy it comes closest, and Kripke, with whom it concurs only in admitting metaphysics.
Though there is I hope a serious embracing of some kind of metaphysics, it is not of Kripke's kind.
I thought about the source, and the opposition, Grice has not featured.
However, since Grice has insinuated himself into my thoughts, it seems an opportunity to introduce a symmetry.
On the one hand I present metaphysical positivism as reconciling Carnap's metaphysics with a positivistically tenable notion of metaphysics, on the other, I wonder whether the metaphysically relaxed Grice might be persuaded that in metaphysical positivism there is a kind of positivism which he might at least converse if not concur. 
Here are some thoughts along these lines.

I'm going to say a few things about Carnap and Grice first, before getting into metaphysical positivism. 
But before either, some b\^{e}tes noires.

\subsection{B\^{e}tes Noires}

Speranza talks about Grice's b\^{e}tes noires, which we will come to later.
It will help to explain what I am trying to do if I mention one or two of mine.

Gladiatorial philosophy is one of them.
I'll contrast this with {\it conversational} philosophy.
The contrast is most pertinent to philosophical dialogue, in the flesh, on a mailing list, but applies also to papers and books.

Here are some features:

\begin{description}

\item[gladiatorial] \ 

In gladiatorial philosophy, one engages with one or more opponents and aims to prevail.
An outright defeat conceded by the opponent is best, but a kind of point victory is better than nothing.
It is not the purpose of the exercise to understand your opponent or to learn anything from him (you already know better than he does).
It is not necessary to criticise the position he actually holds, this would often involve considerable effort in first coming to an understanding of that position.
It is necessary only to refute what he says, understood in whatever way is most convenient for a convincing dismissal.

\item[conversational] \ 

Conversation philosophy is a game played by two or more philosophers having a common interest in some subject matter who believe that through dialogue they can come to understand that subject better than they otherwise might.
The idea is that seeing this problem from another persons point of view will enrich ones own understanding, even if that point of view is not necessarily ``better'' than your own.
To obtain this benefit it may be necessary to invest considerable effort in coming to an understanding of the others position, and in conveying to him an understanding of your own.
Not all of it, the bits which bear upon the issues.

Unless all partners are committed to this process it will not work.
\end{description}

It is my perception, that in philosophical discussion on the Internet, it is very difficult to achieve what I have called a conversational dialogue.
It is the norm rather than the exception that people from the first assume that they understand what you are saying, and one may then find many occasions to complain that an adversary criticises a position that one has never held.

This is however, not a problem exclusive to the Internet fringe.
It is pervasive in the writings of professional philosophers that a philosopher will criticise an opponent by attacking a position which he has never held.
If the philosopher is at the van of a new philosophical trend this can be done with impunity, readers sympathetic to the new philosophical trend will not trouble to check the accuracy of the picture presented of some obsolescent point of view.

Famous works of this kind include, Austin's ``Sense and Sensibilia'', Quine's ``Two Dogmas'' and Kripke's ``Naming and Necessity''.

Sometimes a failure to converse is not due to bad faith.
Two philosophers earnestly wishing to understand each other may nevertheless fail for reasons which are mysterious and insurmountable, and we shall touch upon some of these.

The point of this little diatribe is to explain that my aim in discussing Carnap and Grice here is to explore whether they might have had a conversation, and whether some of the ideas of metaphysical positivism might help in providing a basis for that dialogue.
To say some things on either side which might help to make conversation possible.
Locating the points at which incomprehension, may possibly facilitate their isolation, and permit conversation elsewhere.

\section{Carnap}

\subsection{Carnap Towards Metaphysics}

I think I have a reasonable (if not scholarly) grasp of Carnap's attitude towards metaphysics.

I shall sketch it here in three parts, a technical, an intuitive, and a radical rejection.

\subsubsection{technical}
The technical rejection is the least important, though it often takes the headlines.
Metaphysics has sometimes been characterised as necessary knowledge of synthetic truths.
The technical rejection arises from adopting a conceptual framework in which there can be no such thing (on pain of contradiction).
This Carnap does through an understanding of the concepts of necessity and analyticity in which the only difference between them is that the former applies to propositions and the latter to those things which express propositions (of which propositions are the meanings).

The effect if this is to define out of existence a certain kind of metaphysics, viz. metaphysical propositions which are held by some philosopher to be necessary but synthetic.
The propositions do not go away, they are likely then to be regarded as analytic (at least, if the claim to necessity is sustainable).
In Carnap's terminology, the internal question becomes less controversial, but the issue is exported to the ``external question''.

\subsubsection{intuitive}
Carnap had, I believe, a quite genuine incomprehension of certain kinds of philosophical question, which he was therefore inclined to consider meaningless, and thus either to proscribe as metaphysical or to regard as matters of pragmatics.

\subsubsection{radical}
There are some radical skeptical rejections of ``metaphysics'' to be found in some positivist thinkers.
These are connected with the idea that science should not go beyond the evidence, but should confine itself to description of the observables.

How this kind of dictum is to be understood depends on how you construe observation.
In its most radical forms the observational data will be sense data, and the inference to the existence of the external world is a bit of ``metaphysics''.
Phenomenalism is, if taken in this way, a very radical rejection of an extremely broad conception of metaphysics.

Those who associate Carnap with the Aufbau and are not aware of or consider unimportant any of Carnap's later views may therefore consider Carnap to represent this kind of radical anti-metaphysics.
However, by Carnap's own account he was not such a simplistic phenomenalist even in the days of the Aufbau.

Later, his principle of tolerance and his interest not only in phenomenalistic but also in physicalistic (materialist) and theoretical languages provides further evidence against the view that Carnap was dogmatic and radical in his opposition to metaphysics.

Carnap's liberal attitude to ontology is best seen in ``Empiricism, Semantics and Ontology'', but this can be read in two ways.
Carnap is liberal about language and the ontology it presupposes.
But this is a pragmatic stance.
He aknowledges that it may be useful to adopt an ontology of abstract or theoretical objects, but he does not admit that they ``really exist''.
He doesn't even understand the question.

The middle ground which Carnap adopts here between affirming and denying what he calls the ``external questions'' is for some still an anti-metaphysical stance, tantamount to denial.

There are too levels at which ones credentials as a metphysician may be judged.
At first level, a metaphysician is someone who admits an extravagent ontology, and the anti-metaphysicians are ontological nominalists (this is a key thread in positivism).
But at the next level, which is the one you have to think of to understand Carnap, the question is not ``what exists'' but ``what ontological questions have objective (rather than conventional or pragmatic) answers''.
At this level the arch metaphysician will perhap say ``all'', but Carnap says ``none'' and so is from this perspective a radical critic of metaphysics.

\subsection{Reductionism}

One of Grice's B\^etes Noires is reductionism, but his gripe is about a particular kind of reductive analysis, not necessarily everything which might be called reductionism.

So here I consider what kind of a reduc

\section{Grice}

\subsection{Positivism}

Next I'd like to address the differing perceptions of Speranza and myself in relation to Grice's attitude towards such as Carnap.

Grice thought of himself as a kind of ``ordinary language philosopher'' or at the least, a philosopher who saw some point and philosophical value in studying ``ordinary language'' as it is.
His predecessor Austin severely criticised aspects of positivism in his ``Sense and Sensibility'', and appeared an implacable opponent.

Grice's position was more moderate, several parts of his work made him appear to me less antagonistic.

\subsection{Metaphysics}

Clearly Grice was willing to induldge in Metaphysics.

Did he, would he, understand Carnap's reservations about metaphysics?
Was his metaphysics of the kind which Carnap would deprecate, or of the kind that Carnap would not call metaphysics?

\subsection{Reductionism}

We want to know whether the very specific kind of reductionism which Grice's rejected would embrace any or all of Carnap's reductionist tendencies.

Now, according to Speranxa, the specific notion of reductionism which Grice rejected was that of a conception of reductive analysis of semantics in which the reduction takes place to some single kind of entity.
This looks like it is intended to include dogmatic phenomenalism as a doctrine about semantics, which, apart from the dogmatic bit is what Carnap was doing in the Aufbau.



{\raggedright
\bibliographystyle{alpha}
\bibliography{rbj,fmu}
} %\raggedright

\end{document}
