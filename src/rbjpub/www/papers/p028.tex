% $Id: p028.tex $
% bibref{rbjp028} pdfname{p028}

\documentclass[14pt,titlepage]{extarticle}
\usepackage{makeidx}
\usepackage{graphicx}
\usepackage[unicode]{hyperref}
\pagestyle{plain}
\usepackage[paperwidth=8.3in,paperheight=11.7in,hmargin={0.3in,0.3in},vmargin={0.5in,0.5in},includehead,includefoot]{geometry}
\hypersetup{pdfauthor={Roger Bishop Jones}}
\hypersetup{pdftitle={Open Society - some notes}}
\hypersetup{colorlinks=true, urlcolor=red, citecolor=blue, filecolor=blue, linkcolor=blue}
%\usepackage{html}
\usepackage{paralist}
\usepackage{relsize}
\usepackage{verbatim}
\usepackage{enumerate}
\makeindex
\newcommand{\ignore}[1]{}

\title{Open Society - some notes}
\author{Roger~Bishop~Jones}
\date{\ }


\begin{document}
%\frontmatter

%\begin{abstract}
%Notes for a philosophical discussion on The Open Society
%\end{abstract}
                               
\begin{titlepage}
\maketitle

%\vfill

%\begin{centering}

%{\footnotesize
%copyright\ Roger~Bishop~Jones;
%}%footnotesize

%\end{centering}

\end{titlepage}

\ \

\pagebreak

\begin{centering}
{\LARGE \bf Open Society - some notes}
\end{centering}

\setcounter{tocdepth}{1}
{\parskip-0pt\tableofcontents}

%\listoffigures

%\mainmatter

%\pagebreak

\section{Introduction}

Three stages:

\begin{enumerate}
\item Some history and background
\item What does ``Open Society'' mean or emtail
\item Which freedoms are for sale?  Do economics trump all?
\end{enumerate}

\section{History and Other Background}

Term ``Open Society'' coined by Bergson, then used by Popper (in a rather different way) in his ``THe Open Society and its Enemies'', an anti-totalitarian tract inspired by the examples of Nazi Germany and Stalinist Russia, and later institionalised by Popper's student Soros, financier and philosopher, in the Open Society Foundations.

\subsection{Open Society}

The following quoted from:

Encyclopedia of Science, Technology, and Ethics \\
COPYRIGHT 2005 Thomson Gale \\

\begin{quote}
The term open has a special salience in such phrases as ``open markets,'' ``open records,'' ``open government,'' and ``open-ended'' discussion or project. In such contexts it denotes both freedom and transparency, two fundamental values of a democratic society. Indeed, the term open society has itself become almost synomous with democracy, and is sometimes used to name the ideal of both the scientific and the non-scientific social orders.

Although Henri Bergson (1859–1941) first employed the term open society in The Two Sources of Morality and Religion (1935) and Eric Voegelin (1901–1985) made Bergson's interpretation a key concept in his philosophy of history, it was The Open Society and Its Enemies (1945) by Karl R. Popper (1902–1994) that gave the phrase wide currency. The concept of the open society has since sparked numerous scholarly debates as well as practical applications. Although based on core values such as equality in social relations, freedom of inquiry and speech, and transparancy in decision making and knowledge production, the precise meaning of an open society has never been settled. Furthermore, globalization and the increasing threat of terrorism are reshaping conventional understandings of closed and open societies.
\end{quote}

\subsection{Popper}

Accordng to Wikipedia:

\begin{quote}
Sir Karl Raimund Popper CH FBA FRS[7] (28 July 1902 – 17 September 1994) was an Austrian and British philosopher and professor.
He is generally regarded as one of the 20th century's greatest philosophers of science.

...

In political discourse, he is known for his vigorous defence of liberal democracy and the principles of social criticism that he came to believe made a flourishing open society possible.
His political philosophy embraces ideas from all major democratic political ideologies and attempts to reconcile them: socialism/social democracy, libertarianism/classical liberalism and conservatism.
\end{quote}

Popper enunciated ``the paradox of tolerance'':

\begin{quote}
The paradox of tolerance, first described by Karl Popper in 1945, is a decision theory paradox.
The paradox states that if a society is tolerant without limit, their ability to be tolerant will eventually be seized or destroyed by the intolerant.
Popper came to the seemingly paradoxical conclusion that in order to maintain a tolerant society, the society must be intolerant of intolerance.
\end{quote}

\subsection{Soros}

Soros was born in Budapest.
He survived Nazi Germany-occupied Hungary and emigrated to England in 1947, studied philosophy under Popper at the London School of Economics, but then went on to be a hedge fund manager famous for having been instrumental in the speculations which caused stirling to drop out of the EMU.

He created the Open Society Foundations, and by 2017 had given away some \$12 Billion before very recently transferring a further \$19B of his personal wealth to the Open Society Foundations.

\section{What does ``Open Society'' Mean and Entail}

Central is freedom of thought and expression, and (possibly in a qualified way) of actiom.

This freedom is crucially to cover governance.
Giving the individual these freedoms may be thought best realised through democracy.

In economic matters freedom of choice and action may be thought to demand a free-market, capitalist, economy.

When the Soviet Union collapsed, it was assumed that the connections between these freedoms and prosperity was the cause of the collapse.
A totalitarian system and collectivist economy must fail economically, the freedoms were thought economically essential.

Key associations:

\begin{itemize}
\item freedom of information, thought and speech
\item democratic institutions
\item free markets
\item human rights
\end{itemize}

The fall of the Berlin Wall was taken to be symbolic of a connection between freedom and wealth.
The thesis (held by many) at the time was that totalitarian socialist systems inevitably proved economically ineffective and would eventually collapse for that reason.
Another supposition which was then prevalent (and persist today) was that socialism was inevitably totalitarian;  only oppression could prevent independant (rather than communal) economic enterprise and the free markets and capitalistic trappings that result from it.

At this stage the ``Open Society'' seemed to be advancing, and this impression was at first enhanced by the phenomenon known as the arab spring, which added the further connection that freedom of information (particularly of the kind provided by free access to the internet) would inevitably erode acceptance of a closed society and lead via popular unrest to open and democratic instutions.

The arab spring however, turned sour, led to the disintegration of Libya and interminable civil war in Syria.
In states which peacefully transitioned to apparently more democratic institutions, such as Egypt, conservative majorities overwhelmed the progressive elements in the polls and installed fundamentalist regimes intent on suppressing the newly democratic institutions.

At this stage we begin to see the retreat of the Open Society ideal, which is now compounded by the retreat first of Russia and later some of its erstwhile eastern european satellites into pseudo democracies.

\section{Freedom's for Sale}

The ideal of the Open Society is now truly on its back foot, and one of the perceived spanners in the spokes is the perceived willingness of ordinary people to trade freedoms for economic prosperity.

Among the illustrations of this are:

\begin{itemize}
\item China

  China disproves the supposition that autocratic regimes cannot deliver economic prosperity.
  For a while it was thoufgt by some that improved standards of living and education might result in greater disatisfaction among the Chineese people and force a transition to more democratic institions, but this does not appear to be the case.
  It looks like most Chinese are pleased enough to be better off, and don't care to rock the boat.
  They may be perceived as ``selling'' the freedoms (even though they never had them).
  
\item Singapore

  The city state of Singapore is built on a similar model, and is even more successful than China, having an extremely high \emph{per capita} income realised by a non-democratic authoritarian regime which has invested heavily in creating a high-tech high trade economy.
  
\item Anti-Brexit arguments.

  It is rare to hear an argument against Brexit other than the opinion that we will be economically worse off for it, and usually this is presented as if it were in itself sufficient, presuming that no other consideration might justify accepting reduced wealth (even though many Brexit supporters remain steadfast and careless of possible impact on wealth).
  
\end{itemize}

%\backmatter

%\appendix

%\addcontentsline{toc}{section}{Bibliography}
%\bibliographystyle{alpha}
%\bibliography{rbj2}

%\addcontentsline{toc}{section}{Index}\label{index}
%{\twocolumn[]
%{\small\printindex}}

%\vfill

%\tiny{
%Started 2017-10-09

%Last Change 2017-10-09

%\href{http://www.rbjones.com/rbjpub/www/papers/p028.pdf}{http://www.rbjones.com/rbjpub/www/papers/p028.pdf}

%}%tiny

\end{document}

% LocalWords:
