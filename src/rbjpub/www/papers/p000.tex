% $Id: p000.tex,v 1.8 2007/01/02 20:41:09 rbj01 Exp $
% bibref{rbjp000} pdfname{p000} 
\documentclass[numreferences]{rbjk}
\usepackage{makeidx}
\newcommand{\ignore}[1]{}
\usepackage[unicode,pdftex]{hyperref}
\hypersetup{pdfauthor={Roger Bishop Jones}}
\hypersetup{colorlinks=true, urlcolor=black, citecolor=black, filecolor=black, linkcolor=black}
\makeindex

\begin{document}                                                                                   
\begin{article}
\begin{opening}  
\title{Things by Roger Jones}
\subtitle{mostly philosophical}
\runningtitle{The Philosophy of Roger Bishop Jones}
\author{Roger Bishop \surname{Jones}}
\date{\ }

\begin{abstract}
An introduction to and an overview of various work in progress.
\end{abstract}
\end{opening}

\vfill

\begin{centering}
{\footnotesize
Created 2005/01/26

Last Changed $ $Date: 2007/01/02 20:41:09 $ $

\href{http://www.rbjones.com/rbjpub/www/papers/p000.pdf}
{http://www.rbjones.com/rbjpub/www/papers/p000.pdf}

$ $Id: p000.tex,v 1.8 2007/01/02 20:41:09 rbj01 Exp $ $

}%footnotesize
\end{centering}

\newpage

%\def\tableofcontents{{\parskip=0pt\@starttoc{toc}}}
\setcounter{tocdepth}{4}
{\parskip=0pt\tableofcontents}

\section{Introduction}

I am now in writing in XML converted to HTML for web pages and in a
series of documents written in LaTeX for delivery as PDF.

The PDF documents are in two series.

Of these, one series is devoted to progressing philosophical problems
by formal methods, the other by informal means.

This document introduces directly the informal essays, with some
reference to the related content of the formal documents.

The more formal documents, whose overview and introduction is \cite{rbjt000}, makes use not only of {\LaTeX} but also of {\it ProofPower}, a system for preparing and processing documents containing formal specifications and proofs.

\section{The Themes of The Work}

\subsection{The Foundations of Abstract Semantics}

\subsection{Metaphysical Positivism}

Knowledge is classified methodologically, by how it may be discovered or how it should be established, the latter generally being the more important characteristic.
It is an expectation here that the account will not be anthropological, since we must admit the possibility that something other than a human being might acquire knowledge.

The classification therefore impinges upon the design of artefacts which might assume the relevant roles (of discovering, establishing or applying knowledge).

The first crude distinction is then made along the traditional if controversial analytic/synthetic/evaluative lines.

\subsection{X-Logic}

It is primarily under the {\it X-Logic} heading that I propose to engage in what I am now calling ``constructive philosophy'', though possibly this may go beyond what even I am prepared to offer as philosophy.
Constructive philosophy is a method which belongs to metaphysical philosophy without being the only philosophical method which that philosophical system embraces.

\section{Abstracts}

\include{p000abs}

%\begin{thebibliography}{}
%\bibitem[\protect\citeauthoryear{MacKensie}{2001}]{MacKensie}
Donald MacKensie: 2001,
\newblock {\it Mechanising Proof - Computing, Risk and Trust},
\newblock {The MIT Press}.

\bibitem[\protect\citeauthoryear{Russell}{1908}]{Russell08}
Bertrand Russell: 1908,
\newblock {Mathematical logic as based on the theory of types},
\newblock {in van Heijenoort (ed.) {\it From Frege to Godel - A source book in mathematical logic 1879-1931}, Harvard University Press, Cambridge Massachusetts, 1967, pp. 150-182}.

\bibitem[\protect\citeauthoryear{Tarski}{1908}]{Tarski36}
Alfred Tarski: 1936,
\newblock {On the concept of logical consequence},
\newblock {in John Corocoran (ed.) {\it Logic, Semantics, Meta-mathematics}, Oxford University Press, 1956, pp. 409-420}.
%\end{thebibliography}

{\raggedright
\bibliographystyle{klunum}
\bibliography{rbjk}
} %\raggedright

\twocolumn[\section{INDEX}\label{INDEX}]
{\small\printindex}

\end{article}
\end{document}




