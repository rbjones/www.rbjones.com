% $Id: p000.tex,v 1.4 2006/09/11 10:11:46 rbj01 Exp $
\documentclass[numreferences]{rbjk}
\usepackage{hyperref}
\newcommand{\ignore}[1]{}

\begin{document}                                                                                   
\begin{article}
\begin{opening}  
\title{Things by Roger Jones}
\subtitle{mostly philosophical}
\runningtitle{the philosophy of Roger Jones}
\author{Roger Bishop \surname{Jones}}
\runningauthor{Roger Jones}
%\runningtitle{}

\begin{abstract}
An introduction to and an overview of various work in progress.
\end{abstract}
\end{opening}

\section{Introduction}

I am now in working with two distinct series of documents.
The series are distinguished not so much by the content of the documents as by the technology required to prepare and process the documents, though the technological demands are related to the content.

This introduction is the first of a series of documents prepared using the {\LaTeX} document preparation system in a format not dissimilar to that required for journals such as {\it Synthese}.

Another series of documents, whose overview and introduction is \cite{rbjt000}, makes use not only of {\LaTeX} but also of {\it ProofPower}, a system for preparing and processing documents containing formal specifications and proofs.

Since there is a separate introduction \cite{rbjt000} to the documents making use of {\it ProofPower}, such detail as we have here primarily concerns the informal series.

At times these papers have been conceived of as work towards a book.
This is not my present conception, which is more focussed on developing the web site.
For the time being I am working on similar material both in hypermedia and more linear prose and this is where the prose versions go. 

\section{The Themes of The Work}

The work is intended to be at once philosophy, and an architectural design for an intelligent system.

The central theme of the philosophy is knowledge.
From a philosophical standpoint the design of an intelligent artefact is a philosophical thought experiment.From an engineering standpoint the philosophy is a theoretical underpinning which determines the character of the proposed system.

It is intended that the philosophical and the engineering viewpoints are reasonably well balanced in the book, but if either should be dominant then it will be the philosophical.
As an architectural design the book will be somewhat nebulous, as a philosophical system it will be, I hope, more definite than might be expected.

I am expecting a problem integrating the two perspectives.

The obvious approach is to organise the philosophy and the engineering around a classification of kinds of knowledge.
I think this will be part of the story, but its not a rich enough organisational principle to suffice.

\subsection{Metaphysical Positivism}

Knowledge is classified methodologically, by how it may be discovered or how it should be established, the latter generally being the more important characteristic.
It is an expectation here that the account will not be anthropological, since we must admit the possibility that something other than a human being might acquire knowledge.

The classification therefore impinges upon the design of artefacts which might assume the relevant roles (of discovering, establishing or applying knowledge).

The first crude distinction is then made along the traditional if controversial analytic/synthetic/evaluative lines.

\subsection{X-Logic}

It is primarily under the {\it X-Logic} heading that I propose to engage in what I am now calling ``constructive philosophy'', though possibly this may go beyond what even I am prepared to offer as philosophy.
Constructive philosophy is a method which belongs to metaphysical philosophy without being the only philosophical method which that philosophical system embraces.

\section{The Papers}

At this stage the idea that this series of papers contributes to my book project is scarcely recognisable.

\begin{enumerate}
\item Semantic Foundations for Deductive Methods \cite{rbjp001}
\item Analyticity and Abstraction \cite{rbjp002}
\item On How Many Things There Might Be \cite{rbjp003}
\item ProofPower \cite{rbjp004}
\item Notes on the Philosophy of Leibniz \cite{rbjp005}
\item Philosophical Themes \cite{rbjp006}
\item An Assurance Calculus for X-Logic \cite{rbjp007}
\item Metaphysical Positivism \cite{rbjp008}
\item Metaphysical Problems \cite{rbjp009}
\end{enumerate}


%\begin{thebibliography}{}
%\bibitem[\protect\citeauthoryear{MacKensie}{2001}]{MacKensie}
Donald MacKensie: 2001,
\newblock {\it Mechanising Proof - Computing, Risk and Trust},
\newblock {The MIT Press}.

\bibitem[\protect\citeauthoryear{Russell}{1908}]{Russell08}
Bertrand Russell: 1908,
\newblock {Mathematical logic as based on the theory of types},
\newblock {in van Heijenoort (ed.) {\it From Frege to Godel - A source book in mathematical logic 1879-1931}, Harvard University Press, Cambridge Massachusetts, 1967, pp. 150-182}.

\bibitem[\protect\citeauthoryear{Tarski}{1908}]{Tarski36}
Alfred Tarski: 1936,
\newblock {On the concept of logical consequence},
\newblock {in John Corocoran (ed.) {\it Logic, Semantics, Meta-mathematics}, Oxford University Press, 1956, pp. 409-420}.
%\end{thebibliography}

{\raggedright
\bibliographystyle{klunum}
\bibliography{rbjk}
} %\raggedright

\end{article}
\end{document}




