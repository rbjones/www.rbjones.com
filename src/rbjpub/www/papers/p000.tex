% $Id: p000.tex,v 1.1 2005/01/26 20:53:38 rbj Exp $
\documentclass[numreferences]{rbjk}

\newdisplay{guess}{Conjecture}
\newdisplay{prop}{Proposition}
\newcommand{\ignore}[1]{}

\begin{document}                                                                                   
\begin{article}
\begin{opening}  
\title{Things by Roger Jones}
\subtitle{mostly philosophical}
\runningtitle{the philosophy of Roger Jones}
\author{Roger Bishop \surname{Jones}}
\runningauthor{Roger Jones}
%\runningtitle{}

\begin{abstract}
An introduction to and an overview of various work in progress.
\end{abstract}
\end{opening}

\section{Introduction}

I am now in working with two distinct series of documents.
The series are distinguished not so much by the content of the documents as by the technology required to prepare and process the documents, though the technological demands are related to the content.

This introduction is the first of a series of documents prepared using the {\LaTeX} document preparation system in a format not dissimilar to that required for journals such as {\it Synthese}.

Another series of documents, whose overview and introduction is \cite{rbjt000}, makes use not only of {\LaTeX} but also of {\it ProofPower}, a system for preparing and processing documents containing formal specifications and proofs.

Since there is a separate introduction \cite{rbjt000} to the documents making use of {\it ProofPower}, such detail as we have here primarily concerns the informal series.
However, the enterprise is a common one, and the more general discussions here relate to both series.
The relationship between the two series is expected to be along the following lines.
The real meat, whenever possible, involves formalization and will therefore appear in the formal series.
In the informal series there will be a mixture of gentler or more expansive treatments of topics also addressed in the formal series and material which is beyond the scope of what can plausbily be given a formal treatment.
In particular, metaphilosophical discussion is more likely to be found in the found in the informal than the formal series.

\section{Semantic Positivism}

What {\it kind} of philosophy am I attempting?

In short, {\it semantic positivism}, a conception of philosophy which is nowhere articulated, but, very briefly here.

\section{The Papers}

\begin{enumerate}
\item Semantic Foundations for Deductive Methods
\item Analyticity and Abstraction
\item On How Many Things There Might Be
\item Z in HOL
\item Metaphysical Positivism
\item Logico-Philosophical Themes
\end{enumerate}


%\begin{thebibliography}{}
%\bibitem[\protect\citeauthoryear{MacKensie}{2001}]{MacKensie}
Donald MacKensie: 2001,
\newblock {\it Mechanising Proof - Computing, Risk and Trust},
\newblock {The MIT Press}.

\bibitem[\protect\citeauthoryear{Russell}{1908}]{Russell08}
Bertrand Russell: 1908,
\newblock {Mathematical logic as based on the theory of types},
\newblock {in van Heijenoort (ed.) {\it From Frege to Godel - A source book in mathematical logic 1879-1931}, Harvard University Press, Cambridge Massachusetts, 1967, pp. 150-182}.

\bibitem[\protect\citeauthoryear{Tarski}{1908}]{Tarski36}
Alfred Tarski: 1936,
\newblock {On the concept of logical consequence},
\newblock {in John Corocoran (ed.) {\it Logic, Semantics, Meta-mathematics}, Oxford University Press, 1956, pp. 409-420}.
%\end{thebibliography}

{\raggedright
\bibliographystyle{klunum}
\bibliography{rbj,fmu}
} %\raggedright

\end{article}
\end{document}




