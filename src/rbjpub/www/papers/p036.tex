% $Id: p036.tex $
% bibref{rbjp036} pdfname{p036}
\documentclass[10pt,titlepage]{article}
\usepackage{makeidx}
\newcommand{\ignore}[1]{}
\usepackage{graphicx}
\usepackage[unicode]{hyperref}
\pagestyle{plain}
\usepackage[paperwidth=5.25in,paperheight=8in,hmargin={0.75in,0.5in},vmargin={0.5in,0.5in},includehead,includefoot]{geometry}
\hypersetup{pdfauthor={Roger Bishop Jones}}
\hypersetup{pdftitle={Equality or Equity - where's the justice? and how would we know?}}
\hypersetup{colorlinks=true, urlcolor=red, citecolor=blue, filecolor=blue, linkcolor=blue}
%\usepackage{html}
\usepackage{paralist}
\usepackage{relsize}
\usepackage{verbatim}
\usepackage{enumerate}
\usepackage{longtable}
\usepackage{url}
\newcommand{\hreg}[2]{\href{#1}{#2}\footnote{\url{#1}}}
\makeindex

\title{\bf{\LARGE Equality or Equity\\-\\where's the justice?}\small\\and how would we know?}
\author{Roger~Bishop~Jones}
\date{\small 2020/12/13}


\begin{document}

%\begin{abstract}
% A critique of and defence against Critical Social Justice starting with the distinction between equality and equity.
% 
%\end{abstract}
                               
\begin{titlepage}
\maketitle

%\vfill

%\begin{centering}

%{\footnotesize
%copyright\ Roger~Bishop~Jones;
%}%footnotesize

%\end{centering}

\end{titlepage}

\ \

\ignore{
\begin{centering}
{}
\end{centering}
}%ignore

\setcounter{tocdepth}{2}
{\parskip-0pt\tableofcontents}

%\listoffigures

\pagebreak

\section*{Preface}
\phantomsection

\addcontentsline{toc}{section}{Preface}

There are ``hyperlinks'' in the PDF version of this monograph which either link to another point in the document  (if coloured blue) or to an internet resource  (if coloured red) giving direct access to the materials referred to (e.g. a Youtube video) if the document is read using some internet connected device.
Important links also appear explicitly in the bibiography.

\section{Introduction}

Classic and social liberalism of the 19th and 20th centuries regarded certain kinds of equality as important.
Most fundamentally perhaps, the democratic principle of ``one person one vote'' gave an equal voice to each citizen in electing a government.
Equally important was the rule of law and the principle of equal treatment under the law.
In practice, these are not without qualification.

The idea of equality of {\it opportunity}, though it may a play role in contrasting liberal values with later less liberal trends, is, without further elaboration, close to meaningless.
Nevertheless, that contrast, between equality of opportunity and the notion of ``equity'' as it emerges from Critical Social Justice Scholarship into social activism, is where I begin in this essay.

Despite equal rights in these and progressively many other matters, classical and social liberalism did not seek to make their citizens equal in all respects.
Liberalism has traditionally been concerned more tenaciously with freedom of the individual.
Freedom not only from imposition by his fellow citizens, but, especially, from oppression by government.
Of these freedoms, perhaps the most fundamental is the right of a man to better his lot through his own endeavours and the freedom to give his children the best possible start in life by similar means.

Socialism, Marxism, and a broad variety of academic theorising fanning out from the Frankfurt School's ``Critical Theory'', supercharged by Postmodern Philosophy and culminating in ``Critical Social Justice'', have placed a much greater premium on equality at the expense of individual freedoms, and (among many other conceptual twists) have adopted the concept of ``equity'' as a cornerstone of their demands.

On its face, in this context, the demand for equity is a call for equality of \emph{outcome}.
Equality, not in any particular kind of outcome, nor for individuals in general so much as for ``identity groups''.

\phantomsection
\addcontentsline{toc}{section}{Bibliography}
\bibliographystyle{rbjfmu}
\bibliography{rbj}

%\addcontentsline{toc}{section}{Index}\label{index}
%{\twocolumn[]
%{\small\printindex}}

%\vfill

%\tiny{
%Started 2020/07/06


%\href{http://www.rbjones.com/rbjpub/www/papers/p032.pdf}{http://www.rbjones.com/rbjpub/www/papers/p036.pdf}

%}%tiny

\end{document}

% LocalWords:
