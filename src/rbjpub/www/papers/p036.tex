% $Id: p036.tex $
% bibref{rbjp036} pdfname{p036}
\documentclass[10pt,titlepage]{article}
\usepackage{makeidx}
\newcommand{\ignore}[1]{}
\usepackage{graphicx}
\usepackage[unicode]{hyperref}
\pagestyle{plain}
\usepackage[paperwidth=5.25in,paperheight=8in,hmargin={0.75in,0.5in},vmargin={0.5in,0.5in},includehead,includefoot]{geometry}
\hypersetup{pdfauthor={Roger Bishop Jones}}
\hypersetup{pdftitle={Philosophy and Evolution}}
\hypersetup{colorlinks=true, urlcolor=red, citecolor=blue, filecolor=blue, linkcolor=blue}
%\usepackage{html}
\usepackage{paralist}
\usepackage{relsize}
\usepackage{verbatim}
\usepackage{enumerate}
\usepackage{longtable}
\usepackage{url}
\newcommand{\hreg}[2]{\href{#1}{#2}\footnote{\url{#1}}}
\makeindex

\title{\bf{\LARGE Philosophy\\and\\Evolution}}
\author{Roger~Bishop~Jones}
\date{\small 2021/04/16}


\begin{document}

%\begin{abstract}
% A philosophical perspective on evolution, and an evolutionary approach to philosophy.
% 
%\end{abstract}
                               
\begin{titlepage}
\maketitle

%\vfill

%\begin{centering}

%{\footnotesize
%copyright\ Roger~Bishop~Jones;
%}%footnotesize

%\end{centering}

\end{titlepage}

\ \

\ignore{
\begin{centering}
{}
\end{centering}
}%ignore

\setcounter{tocdepth}{2}
{\parskip-0pt\tableofcontents}

%\listoffigures

\pagebreak

\section*{Preface}
\phantomsection

\addcontentsline{toc}{section}{Preface}


\footnote{There are ``hyperlinks'' in the PDF version of this essay which either link to another point in the document  (if coloured blue) or to an internet resource  (if coloured red) giving direct access to the materials referred to (e.g. a Youtube video) if the document is read using some internet connected device.
Important links also appear explicitly in the bibiography.}

The ideas explored here arise primarily from my dismay on discovering (somewhat belatedly) the extent to which ``Critical Theory'' and its explicitly irrational progeny, (gathered together under the rubric ``Critical Social Justice'' or more popularly ``Equity''), have penetrated the social institutions of the most prosperous and humane societies in history.

My reaction has been to try to understand how it could have been possible, and to seek a defence of some aspects of our cultural heritage, for example the notion of objective truth, the fundamental (if still controversial) distinctions between truths of logic, empirical facts and value judgements, and the merits of liberal democracy.
In trying to understand both the roots of rationality and of those social tendencies which fly in its face, the theory of evolution could not be ignored, and as I struggled in that endeavour my own sense of the significance of the theory of evolution to philosophy grew.

On the other hand, studying evolution from a philosophical perspective may also cast light, and it is my hope in this essay to give an account of evolution, insofar as it relates to these questions, which benefits from my own philosophical perspective.
I seek both philosophical and biological enlightenment by considering each through the lense of the other, and I hope in the essay that the understanding which I am approaching might possibly have some interest to others.

This essay is not a culmination.
I have separated out this relationship, between philosophy and evolution, so that I can consider it detached from the controversies which lead me to think it important, and I look forward to discovering what I can make of it thus detached.
It is a progression whose merit lies in its continuation rather than its completion.

\section{Introduction}

The words `philosophy' and `evolution' have been used in more ways than I can ever know.
Though this essay is broadly scoped, it does not aim completely to comprehend that breadth of usage.

My aim in this essay is to defend aspects of that cultural tradition which is first exhibited in the writings of Plato and Aristotle, after the astounding mathematical successes and embarassing philosophical disarray of the pre-socratic `philosophers', all attempting with varied degrees of success to secure knowledge by observation and reason.
The greatest substantive fruit from this tradition did not appear until it spawned the industrial revolution whose economic consequences transformed, and continue to transform, the prosperity and well being of \emph{homo sapiens}.
These are, I suggest, the fruits of \emph{rationality}, and it is under the umbrella of rationality that my defence of this tradition will be conducted.
This involves me in taking a particular stance both on what rationality is, and on what is rational, which will gradually unfold through the essay.

It is something of a surprise to me that I should be engaged on such an enterprise, since for most of my life I have been more inclined to criticise than laud many of the contemporary manifestations of that tradition.
My critical inclinations were nevertheless very much \emph{from within}.
My desire to see the tradition as a whole demolished and discarded is nil.
So this will be a qualified defence.
A defence against wholesale destruction, but not against forceful critique.
Forceful critique is of course part of the tradition, that is, critique from within a tradition of diverse thought and custom of some part of that tradition, not of the whole, which only becomes a single whole in the ideas of those seeking to undermine the whole that they create.

What is the threat against which a defence is warranted?

It is the `Critical Theory' of the Frankfurt school of neo-marxist scholars, the scepticism of Postmodern philosophy and the subversive activism spawned by `Critical Social Justice' pseudo-scholarship.
I name these as inspiring this response, but will give only the barest sketch of the challenge which they present.
A broader defence is intended, one which addresses also problems more fully within the tradition, one intended to have relevance even to those not convinced of the all the threats I perceive, and which may have continuing relevance when those threats are past.

How does evolution figure in this?
There were two initial provocations for me to have thought that an evolutionary perspective might help.

First, we have the explicit critique of rationality which comes with critical social justice scholarship and activism, possibly derived in part from the scepticism of postmodern philosphy.
This critique identifies all aspects of the western tradition as a white supermacist power-play.
It is shocking that any defence should be needed against this position, but I must reluctantly concede that it is.
It is to the origins of language, reason and culture that I turn, and it is to the theory of evolution that I look for enlightenment on those origins.

Alongside my attempt to cast light on how we humans came to be capable of rationality I shall consider the genesis of the ways in which we continue to be capable of casting aside reason.
Contemporary manifestations of Critical Theory may seem the most egregious illustrations of that capability, but cultural evolution furnishes a broad range of examples pertinent to the conundrum of how the tender flower of liberal democracy can be preserved against the totalitarian vandalism of ideological high fashion.

By taking an evolutionary perspective on philosophy with a philosophical view of evolution I invite chicken and egg issues throughout.
As a result the essay will be a smorgasborg of historical narratives, sometimes passing over the same ground more than once with a new or refined perspective.

Before considering evolution at all, I will sketch aspects of the development of the idea of scientific theory over the last couple of millenia, in the light of which the theory of evolution and its development over the last two centuries will be lightly drawn.
Then our idea of evolution is refined by considering the various distinct kinds of evolution which have lead us into our present predicament.

\section{What is Rationality?}

Before looking at the theory of evolution, let me say a few words about the direction from which I will be approaching it.
I have already mentioned the important place of rationality in the tradition which I seek to defend, and I will briefly expand upon that by clarifying how I propose to use that concept and how the essay will revolve around it.

I will take as fundamental the concept of \emph{instrumental} rationality, which is exhibited when actions are taken with some purpose in mind which are likely to realise that purpose.
This gloss is a good starting point, but will not be religiously adhered to.
It may be that no course of action is likely to achieve some desperate purpose which cannot be cast aside, and in that case it is surely rational to adopt the course least likely to fail.
So perhaps I should be demanding as rational, that course most likely to succeed, however unlikely that may be.
Perhaps we should also take into account the state of knowledge or ignorance of the agent involved, for surely he cannot be held irrational for adopting after due diligence, the course which seems most likely, even if in fact it is unlikely to succeed for reasons not readily discovered.

In the further analaysis of the concept of rationality we may distinguish two general kinds of consideration.
Those which bear upon the \emph{meaning} of the concept, and those which are concerned, supposing meaning to have been elucidated, with the question of what then falls under the concept.

Leaving for the moment questions of meaning.
It has been important, in the tradition which I am defending, to establish epistemic standards for scientific scholarship which are more rigorous than may be necessary in everyday life.
That is to say, standards which determine (or influence) when some supposed scientific truth is to be considered established.
To have such standards, we may argue, is instrumentally rational, for the purpose of securing prosperity and well-being.
Adherence to such standards we will call \emph{epistemic} rationality which is therefore an aspect of instrumental rationality.
There is a reasonable presumption here that scientific knowledge will be generally applicable, and that accuracy in this body of knowledge is instrumental even if no particular purposes are under consideration.
I do not seek here to deny the importance of choices about what scientific research `should' be conducted, be that a moral or an instrumental consideration, but rather to suggest that if research is to be undertaken, for whatever purpose (or none), then getting the right answer will be instrumental for that purpose (or for any unantipated applications that come up).

We derive the notion of epistemic rationality from instrumental rationality even though establishing the truth of our hypotheses does not have a specific purpose or end, because we consider true knowledge as a resource generally applicable to whatever purposes we have in mind.
This applies whether or not we can anticipate the applications which it may have, because we know of many cases where knowledge proves to have applications which had not been anticipated.
This does not mean that resources for research should not be allocated taking into account the probable applications, but it does help to justify the funding of fundamental research for which applications have yet to be discovered.

\subsection{Head and Heart}

According to Blaise Pascal ``The heart has its reasons which reason does not know''.
According to popular culture, reason and emotion are opposed, rational and emotional behaviour are distinct.

The view of rationality which I take here is more conciliatory.
I acknowledge that an act precipitately spurred by emotion is unlikely to be instrumentally rational, but emotions are motivating, they influence our purposes and in that way feature in rational behaviour, directing rather than conflicting with the dictates of reason.
Emotions direct our attention to important but unexpected happenings or outcomes, things which demand attention.

As well as influencing our purposes and hence the actions which are intrumental in realising those purposes, emotions may provide some guidance on the feasibility of a course of action.
If our own proposed contribution to effecting some plan seems to us morally repugnant or personally unpleasant, this may tell us something about the likelyhood that the plan will succeed.
It is not likely to be instrumental to ignore emotions in determining the course of action most likely to succed.

The everyday question of whether emotional sensitivity should mitigate or overturn a preference for rational effectiveness was played out on a larger canvass in the cultural transition from the age of enlightenment to the subsequent era of romanticism, according to some perspectives on the complex cultural trends around the turn of the 19$^{th}$ century.
This will feature in the discussion later in this essay of the origins of irrational ideologies which became major social influences in contemporary English speaking societies.
That perspective sees the triumph of rational science celebrated in the enlightenment over religion authority as underpinning a secular authoritarianism which sought to order society following a science and philosophy which told us not only about the physical world, but also about hearts and minds, the workings of society, and ultimately about how we should be governed, without needing to pay too much attention to individual preferences.
The desire to determine what is good by rational deliberation has tempted philosophers throughout the ages, but it has no place in the conception of rationality here.
The rationalist philosophers were not in that sense instrumentally rational, for the attempted to determine by purely rational means matters which are in practice only resolvable, on the one hand, by empirical reasearch, and on another by consulting the heart and soul of all.

\subsection{Rationality and Logic}

Often, rationality is thought of as \emph{being logical}, as associated with some particular way of thinking which excludes the intuitive and emotional.
This is close to the way in which rationalist philosophers are thought of, as holding that true knowledge comes only from reason, and perhaps only that particularly rigorous form of reasoning which we call \emph{deductive}.
But then, the kind of reasoning for which Sherlock Holmes is renown is thought of as deductive.

Deductive reason, as applied particularly but not exclusively in the proof of mathematical theorems, has a solid reputation, which is wholly deserved, as an unimpeachable imprimatur of authenticity.
Its reputation has therefore been envied and usurped over and over, by those who care little for confining it to its proper domain.

Hume had it right, in confining its scope to `relations between ideas', and hence incapable of supporting unaided any truth about the material world we inhabit.
A fortiori, following Hume, deduction alone yields no knowledge of moral truths.
But those who believe that they know virtue, rarely aknowledge the tenuous foundations on which it rests, and if they cannot give it devine authority, the pretence to logical proof may seem essential.
By contrast, Hume's scepticism in this matter was to prove helpful to religious dissidents keen to assert the sufficiency of faith against reason.

If our conception of rationality is \emph{instrumental} then we must acknowledge that the reasoning which has ultimately been instrumental for our health and prosperity lacks anything close to absolute authority, but suffices for the purpose.
The logic of science is to some extent deductive in its application, but much more tenuous in establishing any principles from which empirical conclusions can be derived.
Beyond that ideal \emph{nomological-deductive} model, the great mass of practical scientific knowledge is more informal, and the truths of `evolutionary theory' are difficult to corral under any clarly articulated scientific method.
Nevertheless, it is to some understanding of evolution that I turn for insight into human nature, to understand how we might secure the instrumental advantages of rationality and mitigate the potentially destructive forces of tribal ideologies.

\subsection{The Rationality of Evolution}

The process of evolution often produces results which \emph{seem} to be rational.
A classic example is the intricate structure of the human eye, which appears to have been rationally designed to enable us humans to see, even though it is generally considered that evolution proceeds with no purpose in mind.
It is an interesting feature of work in synthetic biology that having developed the technology to undertake evolution \emph{by design}, it turns out that using a customised evolutionary process is sometimes the best way to get the desired results.
In that case the evolution does have a purpose, conceived by a biologist, and is a rational approach to realising that purpose.

The rationality of evolution is not merely evident in the outcome of particular evolutionary adaptations.
Evolution itself evolves, and we can see that this process itself is rational, insofar as the transformations make evolution itself more effective.
The process of evolution designs processes, it crafts ways of living, and it is often possible to see the role of any particular feature in the evolved behaviour in facilitating the replication of the organism.
Insofar as evolution is effective in advancing the reproductive fitness of the organism it will be because the behaviours it engenders are instrumentally rational for the purpose of replicating the organism.
Thus we can expect that evolved organisms will be designed to \emph{behave} rationally.
As organismd and their nervous systems become more complex, and are exposed to climatic volatility an natural crises, it becomes possible, and is intrumentally rational, for them to have less rigid and more adaptable behaviours.
The evolution of nervous systems which balance established behaviours and dispositions with more or less intelligent adaptability in the face of new environmental challenges is a rational effect of evolution, and it plausible that the adaptability is itself delivers more effective responses than would have resulted from the prior more rigid behaviour patterns.
Here the rationality has stepped up a level.

A further step upwards occurs with the evolution of culture and the subsequent cultural evolution.
At this stage we see the emergence of something quite close to the kind of rationality which we might hope to see in homo sapiens.
Evolution has engineered by rational means, a species which is capable of ratiocination and may think and behave rationally.

\subsection{Rationality, Objectivity and Absoluteness}

Whether some course of action is instrumental or not depends upon the purpose for which it was undertaken, which may not be explict or clear.
Even where there is an explicit and clear purpose in play, it may be that the agent adopting the course of action is in fact covertly motivated by some other purpose.

This is relevant to the establishment of institutions, for one of the cases in which this divergence may take place is where a course of action on the part of an employee of an instution  which would be instrumental in promoting the ends of the institution is contrary to the interests of the individual.
In consequence, an institution may act irrationally even though its employees can be seen to be acting rationally.
It is therefore desirable to ensure that an organisation is so structured that when its members acti in their own interests they are acting in a way likely to contribute in the intended way to the achievement or the organisations purposes.

The potential divergence in consideration of instrumental rationality, according to whose purposes are definitive, may expressed by saying that judgements of instrumental rationality are not absolute, but are relative to purpose.

Epistemic rationality is not quite the same.
In this case purpose does not enter in (or we may say that the purpose is to establish truth).
The question of the objectivity of truth does matter.
If truth is objective, then epistemic rationality may be absolute.
Otherwise, it will be relative to those same considerations which influence truth, relative to which truth is determinate.

I will later argue that definiteness of meaning, or more particularly of truth conditions, is an essential condition of the selective advantages which enables language to evolve, that objectivity of truth follows from it, and hence that epistemic rationality has claim to being an absolute notion.
Notwithstanding that claim, I recognise that meaning will never by \emph{abolutely} precise and unambiguous, and that some important domains of discourse are enormously difficult to articulate precisely and consequently sustain language which may be virtually meaningless.


\section{What is a Scientific Theory?}


I'm hoping to apply the theory of evolution to illuminating the nature and origin of rationality and of the forces which work against rational progress.
To understand the cultural tradition which has most explicitly emerged from the rationality of humans, we will be tracing the development of that culture to the present day, and one aspect of the culture which we will trace is the development of scientific methods.

Before looking at the theory of evolution, some consideration of the `scientific method' exemplified by that theory is appropriate.
If not definitive of epistemic rationality, received scientific method is an important part of how we judge whether scientific theories are rational, and the history of scientific method therefore belongs within a broader account of the cultural evolution of rationality.
This first sketch of elements of that history precedes discussion of evolution, and will be augmented in the light of the evolutionary theory in due course.

I first sketch some thoughts about the development of scientific method, in order to determine what sort of a `theory' the science of evolutionary biology might have.

\subsection{Axiomatic Method}

The first theoretical science was mathematics, which began, as such, in ancient Greece, approximately 600 years before Christ.
In its first 300 years Greek mathematics was stunningly successful.
It established cumulatively a body of proven mathematical knowledge which was collected together in Euclid's Elements \cite{euclidEL1}, using a clearly documented and reliable systematic method, `The Axiomatic Method', based around \emph{deductive} inference.
Euclidean geometry was so solid that once knowledge of it returned to Western Europe after the dark ages, it continued to be taught in schools until well into the 20$^{th}$ Century.

The axiomatic method involved adopting certain principles, notably axioms and definitions, and deriving an extensive body of knowledge from those principles by deductive proof.
There is at this stage no known disussion of what deduction is, the meta-theory of deduction, and more broadly of demonstrative science, begins with Aristotle (who explicitly lays claim to precedence in this).

\subsection{Plato's Two Worlds}

\subsection{Aristotle's Demonstrative Science}

\subsection{Empirical Science}

\subsection{Hume and Positivism}

\subsection{Carnap's Logical Positivism}

In the conception of philosophical analysis put forward by Rudolf Carnap early in the 20$^{th}$ century, the product of such analysis must be those logical propositions which were called `analytic' sometimes glossed as `true in virtue of meaning'.
So strict a conception of philosophy left no room for practical philosophy to reach conclusions which bear on how we conduct our lives, and entailed a clear distinction between philosophical analysis and empirical science.
Carnap was however, by no means an ivory tower rationalist.
He was an empiricist (even a positivist), and his philosophical programme saw analytic philosophy as handmaiden to empirical science, providing the analytic tools necessary for science to be conducted in a logically rigorous manner, just as before him Aristotle had offered his works on logic not as science but as tools for conducting `demonstrative science'.

The uncompromising ideal pursued by these men, separated by two millenia of philosophical and scientific controversy and progress, remains tantalisingly beyond practical reach for most science.

\subsection{Naturalised Philosophy}


\subsection{Evolution as Tautology or Method}

Biology, the discipline most clearly transformed by the theory of evolution, and sociobiology or 
evolutionary psychology, present particular problems for rigorous conceptions of scientific methods, but will be essential to the discussion of how rationality came to be, and why it so often seems to have been abandoned.

The peculiar status of evolution in respect of scientific method may be seen in the two first books by Richard Dawkins.
In ``The Selfish Gene''\cite{dawkinsSG} Dawkins assures us that his central thesis, around which the entire book turns, is a \emph{tautology} (along the lines that a gene will proliferate in the gene pool if the phenotype it codes for is conducive to the replications of the gene, though that claim falls short of being a tautology).
Whether or not Dawkins' claim is correct is not important here, what's important is that Dawkins was comfortable building an empirical text apparently by deduction from a non-empirical foundation.
In his second book, ``The Extended Phenotype''\cite{dawkinsEP} Dawkins makes a different observaton about the first ``if adaptations are to be regarded as `for the good of' something, then that something is the gene'', and observes of the central thesis of the new book ``since it is not a factual position I am advocating, I warn the reader not to expect `evidence' in the normal sense of the word``.


\section{The Theory of Evolution}

\subsection{Darwin's Theory}

This is a very lightweight discussion of Darwin's \emph{theories}, and some of the subsequent work which has added to the \emph{theory} of evolution.
This is to be understood as distinct from the very much greater volume and detail of the study of what actually happened in the evolution of life on earth.

To clarify that distinction just a little, Darwin had a theory, primarily about the evolution of species, and he spent a lifetime gathering data about evolution in order to show as conclusively as possible, the truth of his theory.
The theory itelf has a relevance which transcends the historical (geological) data which supports it.
If true, it tells us not only about how life evolved, but also about how it will evolve in the future.
The facts about what actually happened don't do that, until they are transformed into general theories describing enduring features of evolutionary biology which will continue to be true into the future.

Darwin was not the first to have ideas about the evolution of life, but he was the first to have a theory which has stood the test of time and of extended empirical investigation.
In its most concise form that theory is:
\begin{enumerate}
\item Living organisms on earth fall into \emph{species}, which are groups of similar but not identical organisms.
\item An essential feature of living organisms is that in their normal environment they are capable of reproduction, which is to say, of producing new and similar (but not identical) organisms.
\end{enumerate}

  \subsection{The Modern Syntheses}

  \subsection{The Gene-centric Variation}

  \subsection{Extended Evolutionary Syntheses}

  \subsection{Tinbergen's Four Questions}

  \begin{itemize}
  \item[First:] what is the function of a given trait (if any)? Why does it exist compared to many other traits that could exist?
  \item[Second:] what is the history of the trait as it evolved over multiple generations?
  \item [Third:] what is its physical mechanism? All traits, even behavioral traits, have a physical basis that must be understood in addition to their functions.
    \item [Fourth:] how does the trait develop during the lifetime of the organism?
  \end{itemize}
  \footnote{from Tinbergen \cite{tinbergen-oame} as presented by Sloan Wilson in \cite{wilson-tvl}}
  
  \section{Stages in Evolution}

  \subsection{Pre-Biotic Chemical Evolution}

  \subsection{Prokaryotic Evolution}

  \subsection{Sexual Selection}

  \subsection{Culture}



\phantomsection
\addcontentsline{toc}{section}{Bibliography}
\bibliographystyle{rbjfmu}
\bibliography{rbj}

%\addcontentsline{toc}{section}{Index}\label{index}
%{\twocolumn[]
%{\small\printindex}}

%\vfill

%\tiny{
%Started 2020/07/06


%\href{http://www.rbjones.com/rbjpub/www/papers/p032.pdf}{http://www.rbjones.com/rbjpub/www/papers/p036.pdf}

%}%tiny

\end{document}

% LocalWords:
