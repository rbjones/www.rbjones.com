% $Id: p036.tex $
% bibref{rbjp036} pdfname{p036}
\documentclass[10pt,titlepage]{book}
\usepackage{makeidx}
\newcommand{\ignore}[1]{}
\usepackage{graphicx}
\usepackage[unicode]{hyperref}
\pagestyle{plain}
\usepackage[paperwidth=5.25in,paperheight=8in,hmargin={0.75in,0.5in},vmargin={0.5in,0.5in},includehead,includefoot]{geometry}
\hypersetup{pdfauthor={Roger Bishop Jones}}
\hypersetup{pdftitle={An Epistemological Synthesis}}
\hypersetup{colorlinks=true, urlcolor=red, citecolor=blue, filecolor=blue, linkcolor=blue}
%\usepackage{html}
\usepackage{paralist}
\usepackage{relsize}
\usepackage{verbatim}
\usepackage{enumerate}
\usepackage{longtable}
\usepackage{url}
\newcommand{\hreg}[2]{\href{#1}{#2}\footnote{\url{#1}}}
\makeindex

\title{\bf An Epistemological Synthesis}
\author{Roger~Bishop~Jones}
\date{\small 2024/05/15}


\begin{document}

%\begin{abstract}
% An epistemology constructed for the advancement of science and technology in the age of interstellar hybrid (human, synthetic, post human) intelligence.
% 
%\end{abstract}

                               
\begin{titlepage}
\maketitle

%\vfill

%\begin{centering}

%{\footnotesize
%copyright\ Roger~Bishop~Jones;
%}%footnotesize

%\end{centering}

\end{titlepage}

\ \

\ignore{
\begin{centering}
{}
\end{centering}
}%ignore

\setcounter{tocdepth}{2}
{\parskip-0pt\tableofcontents}

%\listoffigures

\hfill
\ 
\pagebreak

%\addcontentsline{toc}{section}{Index}\label{index}
%{\twocolumn[]
%{\small\printindex}}

%\vfill

%\tiny{
%Started 2023/07/21


%\href{http://www.rbjones.com/rbjpub/www/papers/p032.pdf}{http://www.rbjones.com/rbjpub/www/papers/p036.pdf}

%}%tiny

\section*{About}

\section*{Prelude}

\chapter{Introduction}

Epistemology is that part of western philosophy which is concerned with the theory of knowledge.
Its roots lie among the pre-socratic philosophers of ancient Greece at a time when the term \emph{philosophia} meant \emph{love of knowledge} which is distinguished by the endeavours of philosophers to understand the world through reason rather than accepting traditions attributing the important events in life in terms of divine interventions.

For most of the history of epistemology the focus has been on the propositional or declarative knowledge which can be expressed in language and the ways in which human beings acquire and apply such knowledge.


The epistemological synthesis here presented is best understood from an evolutionary perspective, in which that particularly human kind of knowledge which comes only with declarative language is seen alongside its many predecessors and as leading beyond what we now have.

-----------

The epistemological synthesis presented in this monograph is built around the idea of a paradigm shift towards a deductive paradigm for knowledge representation, management and application.

In this introduction I attempt a concise introduction to the most important elements of the proposed synthesis, to enable to reader to make an early assessment of its credibility.
Inevitably, intelligibility will be the first casualty, particularly for those whose philosophical or technical background is less closely related to the key ideas involved.

The central thesis is that human progress in modern times, at least in terms of material prosperity, has been enabled by the advancement of science and technology along lines which since the first attempts of Aristotle to articulate scientific method, have depended vitally on the application of scientific laws by deduction from natural laws.

Despite the efforts of Aristotle and later philosophers and mathematicians, extensive use of deductive reasoning proved successful only where the domain of discourse is severely constrained, usually to ideal domains of mathematics as described by small systems of axioms.
Where the subject matters were less clearly determined, supposedly deductive reasoning would lead different philosophers to distinct and contradictory conclusions.

Though the ability to make simple deductive inferences is inherent in an understanding of declarative language, and likely dates back some quarter of a million years, the conditions necessary for the kind of elaborate reasoning found in mathematics were not understood until the 20th Century.




\appendix

\section{references}

\footnote{Some references for future use:
\cite{arthan1991formal}
\cite{beeson2012foundations}
\cite{centrone2019reflections}
\cite{dzamonja2019}
\cite{gettier1963justified}
\cite{jones1992a,jones1992b}
\cite{kline1990mathematical1}
\cite{kline1990mathematical2}
\cite{kline1990mathematical3}
\cite{kumar2016self}
\cite{kuhn2000structure}
\cite{kuhn2012structure}
\cite{oliveira2006unifying}
\cite{shapiro1991foundations}
\cite{shapiroHPML}
\cite{tarski31}
\cite{tarski56}
}

\phantomsection
\addcontentsline{toc}{section}{Bibliography}
\bibliographystyle{rbjfmu}
\bibliography{rbj2}

\end{document}

% LocalWords:
