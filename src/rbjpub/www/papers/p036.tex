% $Id: p036.tex $
% bibref{rbjp036} pdfname{p036}
\documentclass[10pt,titlepage]{book}
\usepackage{makeidx}
\newcommand{\ignore}[1]{}
\usepackage{graphicx}
\usepackage[unicode]{hyperref}
\pagestyle{plain}
\usepackage[paperwidth=5.25in,paperheight=8in,hmargin={0.75in,0.5in},vmargin={0.5in,0.5in},includehead,includefoot]{geometry}
\hypersetup{pdfauthor={Roger Bishop Jones}}
\hypersetup{pdftitle={An Epistemological Synthesis}}
\hypersetup{colorlinks=true, urlcolor=red, citecolor=blue, filecolor=blue, linkcolor=blue}
%\usepackage{html}
\usepackage{paralist}
\usepackage{relsize}
\usepackage{verbatim}
\usepackage{enumerate}
\usepackage{longtable}
\usepackage{url}
\newcommand{\hreg}[2]{\href{#1}{#2}\footnote{\url{#1}}}
\makeindex

\title{\bf An Epistemological Synthesis}
\author{Roger~Bishop~Jones}
\date{\small 2024/05/27}


\begin{document}

%\begin{abstract}
% An epistemology constructed for the advancement of science and technology in the age of interstellar hybrid (human, synthetic, post human) intelligence.
% 
%\end{abstract}

                               
\begin{titlepage}
\maketitle

%\vfill

%\begin{centering}

%{\footnotesize
%copyright\ Roger~Bishop~Jones;
%}%footnotesize

%\end{centering}

\end{titlepage}

\ \

\ignore{
\begin{centering}
{}
\end{centering}
}%ignore

\setcounter{tocdepth}{2}
{\parskip-0pt\tableofcontents}

%\listoffigures

\hfill
\ 
\pagebreak

%\addcontentsline{toc}{section}{Index}\label{index}
%{\twocolumn[]
%{\small\printindex}}

%\vfill

%\tiny{
%Started 2023/07/21


%\href{http://www.rbjones.com/rbjpub/www/papers/p032.pdf}{http://www.rbjones.com/rbjpub/www/papers/p036.pdf}

%}%tiny

\section*{About}

\section*{Autobiographical Prelude}

It may be helpful to readers of this monograph to know a little about the author, his background and his motivations in attempting the intended `epistemological synthesis'.
This is not so much to bare my scan credentials as in the hope of making my project more intelligible by exposing certain of its roots.

As a child I had the good fortune to get through the 11+ exam which was used in those days to separate out those more academically minded pupils who might eventually number among the small proportion who went on to University, rather than some more vocationally oriented tertiary education (or non at all).

At Ermysted's Grammar school I was mediocre in all but mathematics, the only subject separately streamed, and perhaps French, at which I came bottom of the class despite enthusiasm which could not compensate for a memory which was clearly not designed for retaining discrete and disjointed facts (such as vocabulary)

For the first four years at Grammar school I progressed unremarkably in the middle of three streams, except for mathematics, in which I appeared near the lower end of the top stream.
Towards the end of the fourth year a new mathematics teacher catalysed a sudden improvement in my mathematics, bringing me up towards the top of the stream.

It was clear that my bent was scientific rather than literary\footnote{and, in retrospect neuro-atypical, especially in relation to attention and memory}, and among the limit subject choices for A-levels I eventually plumped for double mathematics and physics.
All other combinations included chemistry, which I avoided because it needed a better memory than I had been gifted.

When it came to applying for a University place, I was already well saturated with mathematics and preferred to go for something more applied.
There was only one University at that time in the UK which did a first degree in Computer Science, which was Manchester, and that was my preference, but I wanted also to put Cambridge in my list, alongside other Engineering disciplines.


\chapter{Introduction}




\chapter{Introduction 24:05:27}

The \emph{epistemological synthesis} which I present here is intended for this world in which the shared knowledge of humanity is held by and accessed through information technology.
It is synthesised in the belief that such a shared repository should consist (as it primarily does) as declarative knowledge in well-defined languages structured into logically coherent complexes at various levels.


It is already the case that the shared knowledge of humanity is now to be found in the memory systems of electronic computers, widely accessible through global networks using electronic devices which are increasingly accessible across both geographic and economic demographics.
But much of this knowledge is stored as \emph{data} without any well-defined or explicit semantics, with no systematic presentation of the semantics of the different notations or data structures in which the knowledge is encoded.

In the present context, as artificial intelligence in the form of `Large Language Models', pervades ever more deeply into the way we use information systems and access the knowledge of which they are the custodians, the AI community driving this transformation seems content to regard the parameters of a neural net as the prime repository of knowledge.



\chapter{Introduction 2024:05:26}

Epistemology is that part of western philosophy which is concerned with the theory of knowledge.
Its roots lie among the pre-socratic philosophers of ancient Greece at a time when the term \emph{philosophia} meant \emph{love of knowledge} which is distinguished by the endeavours of philosophers to understand the world through reason rather than accepting traditions attributing the important events in life in terms of divine interventions.

For most of the history of epistemology the focus has been on the propositional or declarative knowledge which can be expressed in language and the ways in which human beings acquire and apply such knowledge.


The epistemological synthesis here presented is best understood from an evolutionary perspective, in which that particularly human kind of knowledge which comes only with declarative language is seen alongside its many predecessors and as leading beyond what we now have.

-----------

The epistemological synthesis presented in this monograph is built around the idea of a paradigm shift towards a deductive paradigm for knowledge representation, management and application.

In this introduction I attempt a concise introduction to the most important elements of the proposed synthesis, to enable to reader to make an early assessment of its credibility.
Inevitably, intelligibility will be the first casualty, particularly for those whose philosophical or technical background is less closely related to the key ideas involved.

The central thesis is that human progress in modern times, at least in terms of material prosperity, has been enabled by the advancement of science and technology along lines which since the first attempts of Aristotle to articulate scientific method, have depended vitally on the application of scientific laws by deduction from natural laws.

Despite the efforts of Aristotle and later philosophers and mathematicians, extensive use of deductive reasoning proved successful only where the domain of discourse is severely constrained, usually to ideal domains of mathematics as described by small systems of axioms.
Where the subject matters were less clearly determined, supposedly deductive reasoning would lead different philosophers to distinct and contradictory conclusions.

Though the ability to make simple deductive inferences is inherent in an understanding of declarative language, and likely dates back some quarter of a million years, the conditions necessary for the kind of elaborate reasoning found in mathematics were not understood until the 20th Century.




\appendix

\section{references}

\footnote{Some references for future use:
\cite{arthan1991formal}
\cite{beeson2012foundations}
\cite{centrone2019reflections}
\cite{dzamonja2019}
\cite{gettier1963justified}
\cite{jones1992a,jones1992b}
\cite{kline1990mathematical1}
\cite{kline1990mathematical2}
\cite{kline1990mathematical3}
\cite{kumar2016self}
\cite{kuhn2000structure}
\cite{kuhn2012structure}
\cite{oliveira2006unifying}
\cite{shapiro1991foundations}
\cite{shapiroHPML}
\cite{tarski31}
\cite{tarski56}
}

\phantomsection
\addcontentsline{toc}{section}{Bibliography}
\bibliographystyle{rbjfmu}
\bibliography{rbj2}

\end{document}

% LocalWords:
