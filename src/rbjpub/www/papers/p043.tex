% $Id: p043.tex $fi
% bibref{rbjp043} pdfname{p043}
\documentclass[10pt,titlepage]{article}
\usepackage{makeidx}
\newcommand{\ignore}[1]{}
\usepackage{graphicx}
\usepackage[unicode]{hyperref}
\pagestyle{plain}
\usepackage[paperwidth=5.25in,paperheight=8in,hmargin={0.75in,0.5in},vmargin={0.5in,0.5in},includehead,includefoot]{geometry}
\hypersetup{pdfauthor={Roger Bishop Jones}}
\hypersetup{pdftitle={A Position on Gender}}
\hypersetup{colorlinks=true, urlcolor=red, citecolor=blue, filecolor=blue, linkcolor=blue}
%\usepackage{html}
\usepackage{paralist}
\usepackage{relsize}
\usepackage{verbatim}
\usepackage{enumerate}
\usepackage{longtable}
\usepackage{url}
\newcommand{\hreg}[2]{\href{#1}{#2}\footnote{\url{#1}}}
\makeindex

\title{\LARGE\bf A Position On Gender}
\author{Roger~Bishop~Jones}
\date{\small 2021/12/08}


\begin{document}

%\begin{abstract}
% An attempt to put together a coherent position on the shifting language around sex and gender.
%\end{abstract}
                               
\begin{titlepage}
\maketitle

%\vfill

%\begin{centering}

%{\footnotesize
%copyright\ Roger~Bishop~Jones;
%}%footnotesize

%\end{centering}

\end{titlepage}

\ \

\ignore{
\begin{centering}
{}
\end{centering}
}%ignore

\setcounter{tocdepth}{2}
{\parskip-0pt\tableofcontents}

%\listoffigures

\pagebreak

\section*{Preface}

\addcontentsline{toc}{section}{Preface}

\footnote{There may be ``hyperlinks'' in the PDF version of this monograph which either link to another point in the document  (if coloured blue) or to an internet resource  (if coloured red) giving direct access to the materials referred to (e.g. a Youtube video) if the document is read using some internet connected device.
Important links also appear explicitly in the bibiography.}

\section{Introduction}

I have no background which should give anyone reason to pay attention to my views on this topic.
Nevertheless, the controversy around gender is something upon which I feel I need to think through my views carefully.

These notes are my attempt to do that and are not backed up by any serious research.
For that reason I will first of all state the supposed factual basis on which my opinions about what should be done are based, so that readers can judge for themselves whether my opinions are rooted in fact or fiction.

\section{Background ``facts''}

Though not well acquainted with this topic when I first ventured this way, I imagined that I had something to say, and thought to write down the factual basis for the ideas I sought to present.
Not surprisingly I found that difficult, the attempt to explain what I thought I knew quickly lead me into deep water.
The remedy I adopted was to take an existing account of the background and base what I have to say on that foundatoin.
The account I chose was Helen Joyce's book ``Trans''\cite{joyce2021}.

Nevetherless, this section remains, which is now intended to work largely by reference to her that book.

I am primarily here concerned with language, so I will first desxribe my understanding of the development of language concerning sex and gender.

The word sex relates to the phenomenon of sexual reproduction, which in animals results in sexual dimorphism whereby the organisms of a sexually reproducing species produce and rear their offspring.
Individuals in such species fall almost entirely into groups, the females of the species and the males.

Part of the issue at stake here is whether sex is ``binary'' or whether it is a spectrum,
As far as received scientific terminology is concerned, there are just two sexes, male and female, it is not usually difficult to tell whether an individual is male or female, but there are some (a very small number of) cases where this gets more tricky.
This is much like ordinary life has usually been, one can normally easily tell someones sex, but occasionally its not so clear.
In the case of ordinary life, the ability to tell which sex someone is in is substantially aided by the adoption of social stereotypes which serve to make sex evident even if the biological markers of sex are not evident or unambiguous.
Today the number of difficult cases is on the rise, because an increasing number of individuals present as if of the opposite sex.

The existence of sexual stereotypes reflect the enormous importance of sexuality in the evolution of sexually reproducing species.
The choice of a sexual partner with whom to conceive and rear children is of the utmost importance for the continuation of the genetic constitution of the individuals of a species.

Most languages aid and abet the identification of sex by using distinct ways of referring to individuals of different sexes, and the importance of this way of speaking has lead to the linguistic markers (e.g. sex specific pronouns) being used for things which are not biological organisms and have no sex.
Ships, for example.

The word gender came into use (in the 14th Century) as a way of talking about the features of language in question, as classifications of nouns and pronouns based around though possibly extended from the biological sex of the individual referred to.
Likely at that time, the gender was the sex if the referent had a sex, and consequently gender was thought of as a synonym for ``sex'' over that domain (from 15th Century).

It is not untill the 20th Century that gender begins to be used in a way distinct from sex, at first as referring to presentation rather than substance.
After this there seems to be a great deal of fluidity.

[further details]

Notwithstanding the transformation which has occurred in the used of the term gender, the uk government web site dealing with gender recognition certificates asserts that in UK law sex and gender are regarded as synonymous.
This is supported by the fact that once a gender recognition certificate has been geranted, the sex on a birth certificate can be changed.

Nevertheless, when speaking about the equalities act, the UK government speaks of the need to be clear about the distinction between sex and gender, and I belie that the provisions allowing for single sex spaces are specific to sex and allow exclusion of members of the relevant gender who are not of the relevant sex.

It looks as if, at the very least, what the government says about the law is incoherent, and at worst the law really is inconsistent.

This doubtless contributes to the rationale supporting those defenders of single sex spaces who advocate repeal of the gender recognition act. 

\section{What to Do}

Women's rights as understood and fought for in the 20th Century (and before) are inconsistent with the conflation of sex with the gender as it has come to be used.
Some people clearly do not want to see the preservation of these rights.
These ideas will not suit them.

A first step to resolution must be to make sure that UK law unambiguously supports the distinction between sex and gender.

Advocates for ``gender self-id'' would like gender be something entirely subjective, such that no evidence of any kind can speak against the conviction of a person about his or her gender.


\appendix
\section{On the Determination of Sex}

This is a bit of discussion from a non-expert.

In mammals, males and females are distinguished by their roles in procreation.
The child begins as a zygote formed as the union of two gametes, one from a male parent and one from a female parent.
The female gamete is called an ovum or egg which has only half the genetic material normal for members of the species, and becomes fully endowed as a zygote when complemntary genetic material is supplied by union with the male gamete, usually called a sperm.
Though males and females might be distinguished by the type of gamete they are able to produce, infertility is not a bar to the determination of sex.

This can be determined by examination of the chromosomes, which will usually contain chromosomes determining sex.
The genetic material is present in every cell of the body and are not modified by any of the interventions which are customary in ``gender transition''.
The relevant chromosomes are called the X and Y chromosomes, a woman will have two X chromosomes, and a man will normally have an X and a Y chromosome.
In rare cases the genetic material which determines manhood which normally will be in a Y chromosome, may appear in an X chromosome, and in some cases multiple X chromosomes will be present in males.can be determined by sequencing the genome and checking for the presence of a gene called SRY, the presence of which will trigger the development of a male fetus.

Thus one might define a male as an individual whose genome contains the SRY gene.
For that to be definite, one would need an unambiguous definition of the SRY gene.
Genes are coded into DNA as sequences of codons, which represent the sequence of amino acids in a protein.
It is normal for there to be benign variants in the gene pool, which code for proteins which are functionally equivalent to the normal protein.
So to chose one specific sequence of codons or amnino acids in the DNA or in the protein respectively would risk a definition which categorised differently induviduals whose genomes were functionally identical in the relevant areas.




\phantomsection
\addcontentsline{toc}{section}{Bibliography}
\bibliographystyle{rbjfmu}
\bibliography{rbj}

%\addcontentsline{toc}{section}{Index}\label{index}
%{\twocolumn[]
%{\small\printindex}}

%\vfill

\tiny{
Started 2021/12/08


\href{http://www.rbjones.com/rbjpub/www/papers/p043.pdf}{http://www.rbjones.com/rbjpub/www/papers/p043.pdf}

}%tiny

\end{document}

% LocalWords:
