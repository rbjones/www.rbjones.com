% $Id: p033.tex $fi
% bibref{rbjp032} pdfname{p033}
\documentclass[10pt,titlepage]{book}
\usepackage{makeidx}
\newcommand{\ignore}[1]{}
\usepackage{graphicx}
\usepackage[unicode]{hyperref}
\pagestyle{plain}
\usepackage[paperwidth=5.25in,paperheight=8in,hmargin={0.75in,0.5in},vmargin={0.5in,0.5in},includehead,includefoot]{geometry}
\hypersetup{pdfauthor={Roger Bishop Jones}}
\hypersetup{pdftitle={Failing Democracies and How to Fix Them}}
\hypersetup{colorlinks=true, urlcolor=red, citecolor=blue, filecolor=blue, linkcolor=blue}
%\usepackage{html}
\usepackage{paralist}
\usepackage{relsize}
\usepackage{verbatim}
\usepackage{enumerate}
\usepackage{longtable}
\usepackage{url}
\newcommand{\hreg}[2]{\href{#1}{#2}\footnote{\url{#1}}}
\makeindex

\title{\LARGE\bf Failing Democracies \\and \\How to Fix Them}
\author{Roger~Bishop~Jones}
\date{\small 2020/04/03}


\begin{document}
\frontmatter

%\begin{abstract}
% We seem at this moment in history to be unusually well endowed with tales of how our democracies are failing.
% Against these there is "push back".
% To push back you need to spot what is going on, to describe it clearly, and to understand and articulate why it is pathological.
% There are many recent examples of this, some of which are mentioned in this essay.
%
% Thinking philosophically about these phenomena may provide more compelling support for core values which are threatened against
% the broadest range of subversive strategies and tactics.
% Here we take democracy as the fundamental value and seek to analyse a broad range of contemporary erosions and consider what kinds of defence might be mounted against them, though the principle defence the light of scrutiny, on the thesis that once we see clearly the threats and the values which they threated, their strength will be undermined.
%\end{abstract}
                               
\begin{titlepage}
\maketitle

%\vfill

%\begin{centering}

%{\footnotesize
%copyright\ Roger~Bishop~Jones;
%}%footnotesize

%\end{centering}

\end{titlepage}

\ \

\ignore{
\begin{centering}
{}
\end{centering}
}%ignore

\setcounter{tocdepth}{2}
{\parskip-0pt\tableofcontents}

%\listoffigures

\mainmatter

\pagebreak

\section*{Preface}

\addcontentsline{toc}{section}{Preface}

There are ``hyperlinks'' in the PDF version of this monograph which either link to another point in the document  (if coloured blue) or to an internet resource  (if coloured red) giving direct access to the materials referred to (e.g. a Youtube video) if the document is read using some internet connected device.
Important links also appear explicitly in the bibiography.

\chapter{Introduction}


I don't really have much idea how to do this, but I think quite a lot of digging is in order, and I will start by making notes here on what others have uncovered.

\chapter{Some People and their Books}

\section{Michelle Alexander}

\cite{alexander-tnjc}

\section{Ta-Nehisi Coates}

\cite{coatestnh-bwm,coatestnh-wweyip}

\section{Heather Mac Donald}

\cite{macdonald-bbi, macdonald-woc, macdonald-tdd}

\section{Charles Murray}

\cite{murrayc-tbc,murrayc-hd}

\section{Douglas Murray}

\cite{murrayd-vi,murrayd-sde,murrayd-tmc}

\section{Steven Pinker}

\cite{pinker-tbs,pinker-en}

\chapter{Some Issues}

\section{Social Justice and Identity Politics}

\section{Other Activist Foci}

\section{Freedom of Speech}

\section{Due Process}

\section{Conflicts of Interest}

\section{The Nature of Democracy and The Risk of Subversion}

\phantomsection
\addcontentsline{toc}{section}{Bibliography}
\bibliographystyle{rbjfmu}
\bibliography{rbj}

%\addcontentsline{toc}{section}{Index}\label{index}
%{\twocolumn[]
%{\small\printindex}}

%\vfill

%\tiny{
%Started 2020/01/17


%\href{http://www.rbjones.com/rbjpub/www/papers/p032.pdf}{http://www.rbjones.com/rbjpub/www/papers/p032.pdf}

%}%tiny

\end{document}

% LocalWords:
