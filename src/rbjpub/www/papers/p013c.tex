
\section{Hallett on Cantor}

This section concerns issues arising for me from Hallett's book \emph{Cantorian Set Theory and Limitation of Size}\cite{hallettCSTALS}.

\paragraph{Why I am Looking at This Book}

There is a very great difference in perspective between myself of Hallett on set theory, which it might be illuminating to explore.
To draw this out I speak here of what I am hoping to find in his book, and how this differs from what he tried to deliver.

My dominant interest is in understanding abstract ontology, because of its foundational significance in the application of abstract modelling in nomologico-deductive methods generally.

\paragraph{Drawing Out the Differences}

I'm going to scrutinise some minutiae from very early in the book

\subsection{Preface}

Hallett begins with the question whether set theory is simple or not.
It is his point that it is not so simple.

I think he makes it seem less simple than it is.
In doing this he is at one with most philosophers or philosophically minded logicians and set theorists.

I think it is important to discriminate between various different aspects of set theory, some of which are very difficult and complex, and others of which really are very simple.

In examining these distinctions I begin by saying that the notion of set which I intend to discuss here is the notion of a pure well-founded set.
In this it is useful to distinguish two questions:
\begin{enumerate}
\item What is a set?
\item What sets are there?
\end{enumerate}

The first question does have a simple answer.
A set is something whose essence consists in the having of members, and in the members it has.
This is captured by two features its first order formalisations.
The first is that membership is a binary relation between sets, so that of any two sets it is either true or it is false that the first is a member of the second.
The second is the axiom of extensionality, which tells us that no two distinct sets have exactly the same members.
From this we discover that there is in the essence of a set nothing beyond its members, no supplementary objects or information, no particulars about how the members are arranged in the set of which they are members. 

Sets, by themselves are not very useful.
It is only when we have quite a lot of them that we find sets to be useful.
A ``set theory'' is a theory in which the objects in the domain of discourse are sets.

\section{Dehornoy on Woodin on CH}

The following are my notes on reading the Dehornoy's paper \href{http://www.math.unicaen.fr/~dehornoy/Surveys/DgtUS.pdf}{Recent Progress on the Continuum Hypothesis (after Woodin)}.
The points I raise may not of course be relevant to Woodin, they points may only connect with Dehornoy's exposition.

\subsection{Conjecture 1}

First note the way in which Dehornoy introduces conjecture 1:

\begin{quote}
``For the first time, there appear a global explanation for the hierarchy of large cardinals, and, chiefly, a
realistic perspective to decide the Continuum Hypothesis namely in the negative.''
\end{quote}

and this is the conjecture:

\begin{quote}
``Every set theory that is compatible with the existence of large cardinals
and makes the properties of sets with hereditary cardinality at most $\aleph_1$ invariant under forcing
implies that the Continuum Hypothesis be false.''
\end{quote}

I don't actually understand this conjecture.
Perhaps by the end of the paper I will, but here are some indications of what is unclear to me about it at this stage.

\subsubsection{The existence of large cardinals}

Does ``compatible with the existence of large cardinals'' in relation to a set theory mean that the theory has models in which large cardinals exist (for every possible cardinality?), or does it mean that it does not contradict any large cardinal axiom?

If the first, I will assume that the intention is that for every large cardinal there exist a model in which that large cardinal appears.
There remains a question about what it means to say that a certain large cardinal exists in some interpretation of set theory.

It is probably reasonable to assume that the notion of cardinality involved here is internal, i.e. cardinality in an interpretation of set theory is determined by reference to the available bijections in that interpretation rather than in some other (e.g. a standard interpretation, to give an absolute notion of cardinality).

At the end of this section Dehornoy explains the idea of axioms, A, being compatible with the existence of large cardinals as ``in the sense that no large cardinal axiom contradicts A''.
This leaves open the question, ``what is a large cardinal axiom''?
This point arises in connection with Dehornoy's view that it is reasonable to consider large cardinal axioms true, or at least to doubt any axiom which is contradicted by a large cardinal axiom.

For my part, I consider the explication of the notion of large cardinal axiom crucial to the case for considering them to be justified by the iterative conception of set.
It is essential that a large cardinal axiom be in some sense ``pure'', in doing nothing but placing a lower bound on the size of the largest cardinal.
An obvious way to get an ``impure'' large cardinal axiom would be to conjoin a large cardinal with CH or its negation.
One needs in some way to be reassured of any putative large cardinal principle that it does not covertly impose other conditions, before accepting its truth in some initial segment of the cumulative hierarchy.
Furthermore, to prove results in which the phrase ``all large cardinal axioms'' appear, it would seem to be essential to settle the meaning of that phrase.

\subsubsection{properties invariant under forcing}


\subsubsection{Implies that the Continuum Hypothesis be false}

Presumably this means ``includes the denial of the continuum hypothesis'' (in the theory).

\subsubsection{Further remarks}

While talking about the realistic perspective, Dehornoy nevertheless conceives of his problem as finding appropriate axiomatisations of set theory, rather than simply establishing the truth of CH (though that may be the proximate aim of the axiomatisation).
Feodora, oddly, he talks about axioms possibly ``completing'' ZFC, which of course we know to be impossible, at least in the usual sense, or even in the limited sense of making ZFC arithmetically complete.
At this point I can only suppose, not that he is unaware of this point but rather that he is using the term to mean some further extension of ZFC which might be considered \emph{sufficiently} complete for some purposes.

\section{Arithmetic, Incompleteness and Forcing}

Here Dehornoy begins ``let V be the collection of all sets''!
Of course, there is no such collection.

In a footnote he qualifies this as just meaning the ``pure'' sets.
He doesn't add the constraint ``well founded'', but he gives a definition which does seem to entail well-foundedness.
However, there can be no such collection, for that collection would be a set according to the definition, and would be well-founded, and hence could not be the collection of all pure well-founded sets.

Later, even more oddly after this definition, we find talk about generic extensions of V.

Dehornoy now talks about the incompleteness of ZFC and makes a distinction between the kinds of incompleteness which correspond to and are demonstrated by G{\ouml}del's first incompleteness result, and other ``higher level'' undecided statements such as the continuum hypothesis.
The distinction is found in forcing, and in the explanation we begin with the definition of ``invariant under forcing''.

\begin{quote}
Let H be a definable set.
We say that the properties of the structure (H, $\in$) are invariant under forcing if, for every sentence $\psi$, every model M, and every generic extension M[G] of M, the sentence ``(H, $\in$) satisfies $\psi$'' is satisfied in M if and only if it is satisfied in M[G].
\end{quote}

A footnote defines H as ``definable'' if it is the set of values \emph{x} which satisfy some formula $\psi(x)$. 
Let us assume that this is to be understood as \emph{in V}.

This is not what one might naturally expect, and I would have been inclined to use more specific terminology to describe this condition.
Where the phrase ``makes the properties of certain sets invariant under forcing'' is used in conjecture 1, I would be inclined to say ``makes the formulae true in certain membership structures invariant under forcing''. 
[this isn't right]

\section{Peter Koellner}

