% $Id: p029.tex $
% bibref{rbjp029} pdfname{p029}

\documentclass[14pt,titlepage]{extarticle}
\usepackage{makeidx}
\usepackage{graphicx}
\usepackage[unicode]{hyperref}
\pagestyle{plain}
\usepackage[paperwidth=8.3in,paperheight=11.7in,hmargin={0.3in,0.3in},vmargin={0.5in,0.5in},includehead,includefoot]{geometry}
\hypersetup{pdfauthor={Roger Bishop Jones}}
\hypersetup{pdftitle={From Synthetic Biology to Synthetic Epistemology}}
\hypersetup{colorlinks=true, urlcolor=red, citecolor=blue, filecolor=blue, linkcolor=blue}
%\usepackage{html}
\usepackage{paralist}
\usepackage{relsize}
\usepackage{verbatim}
\usepackage{enumerate}
\makeindex
\newcommand{\ignore}[1]{}

\title{From Synthetic Biology \\to Synthetic Epistemology}
\author{Roger~Bishop~Jones}
\date{\ }


\begin{document}
%\frontmatter

%\begin{abstract}
% New conceptions in epistemology inspired in part by synthetic biology.
%\end{abstract}
                               
\begin{titlepage}
\maketitle

%\vfill

%\begin{centering}

%{\footnotesize
%copyright\ Roger~Bishop~Jones;
%}%footnotesize

%\end{centering}

\end{titlepage}

\ \

\ignore{
\begin{centering}
{\LARGE \bf From \\Synthetic Biology \\to \\Synthetic Epistemology\\}
\end{centering}
}%ignore

\setcounter{tocdepth}{1}
{\parskip-0pt\tableofcontents}

%\listoffigures

%\mainmatter

%\pagebreak

\section{Introduction}

The short story here goes via synthetic cognitive systems:
\begin{itemize}
\item Synthetic biologists will create new cognitive systems.
\item New cognitive systems means new things that know, new conceptions of knowledge and new ways of knowing.
\item The design of these is what is meant here by {\it synthetic epistemology}
\end{itemize}

We don't actually need synthetic biology to do synthetic cognitive systems.
Its just artificial intelligence, computer science will get us there.

But AI is predominantly anthropomorphic, most of these guys are trying to replicate or outdo human intelligence.
Synthetic biology throws artificial life into the mix, even new conceptions of life (new ways of coding genes and worse).
So, for these kinds of cognitive system, you can take a new sheet, decide what kind of knowledge will work best and lay out what it is (or they are) and how they work.

By calling it synthetic {\it epistemology}, I am presenting it as {\it philosophy}.
Why is it philosophy rather than just information (knowledge, cognitive) systems architecture?
The longer story which follows will I hope (inter alia) fill that in.

The story comes in three parts.
\begin{enumerate}
\item epistemological milestones

  a bit of history emphasising the evolution of normative epistemology
  
\item space synthetic biology

  a bit of futurism, setting the context demanding synthetic epistemology
  
\item synthetic cosmic epistemology

  the bare bones of a response to that demand

\end{enumerate}

\section{Epistemological Milestones}

\subsection{The Short Version}

\begin{description}
\item Pre-Socratics
\item Plato and Aristotle
\item Skepticism
\item Leibniz
\item Frege
\item Hume
\item Carnap
\end{description}

\subsection{The Other}

Epistemology is {\it the theory of knowledge}: what is it, how do we get it (if at all), how do we use it...

Its useful here to distinguish three kinds of epistemology, which I will call:
\begin{description}
\item [natural epistemology]:
  taking {\it usage} as determining meaning, study and analyse the workings of the word ``know''.
\item [normative epistemology]:
  never mind what people think they know, or how they use cognitive language, \emph{normative} epistemology tells them what knowledge \emph{really} is.
\item [synthetic epistemology]:
  for the purpose of engineering \emph{cognitive systems}, decide what ``knowledge'' really needs to be, how it is obtained, represented and applied.
\end{description}

The epistemological milestones in the history of western philosophy which seem to me significant as a prelude tosynthetic epistemology appear to be mostly \emph{normative}, and principally concerned with some kind of scientific knowledge, for which they may be seen as promulgating \emph{ideal} standards.
It is a small step from promulgating an ideal intended to influence scientists, to devising an abstract concept of knowledge to be build into an engineered cognitive system, once you have the capability to do that kind of engineering.

\begin{description}
\item{\bf pre-socratics}
  
    The civilisation of ancient Greece is credited with an innovation which we may think of as epistemological, in attitude towards knowledge.
    This was the shift from superstition, religion and tradition as the source of knowledge to a belief in \emph{reason}.
    This shift is associated with one from anthropomorphic explanations of earthly phenomena as resulting from the the will of superhuman divinities, to explanation obtained by reasoning from speculations about the structure of the natural world (e.g. from what elements substances are composed).

    In mathematics, reason was fruitful and reliable, the Greeks establishing mathematics for the first time as a theoretical discipline underpinned by rational, deductive, \emph{proof} of theoretical principles.
    The crowning achievement of this development was the compilation of mathematical knowledge in Euclid's \emph{Elements}, which formed the basis of mathematical education for over 2000 years.

    In other areas (metaphysics, cosmology), reason was less decisive and less fruitful.
    One philosopher would put forward a theory, and the next a completely different one.
    Reason was unable to settle which, if either, was correct.
    Worse, philosophers found it possible to prove almost anything via the derivation of contradictions.
    This technique (reductio ad absurdum) was used extensively by Zeno to prove that change was impossible, a doctrine of his teacher Parmenides directly opposed by that of Heraclitus, that the cosmos is in a state of perpetual flux.

    The new scientific epistemology of the pre-socratic Greeks was in trouble, and it fell to the two great system builders Plato and Aristotle to resolve these contradictions and put forward their own philosophical systems rooted in distinctive epistmological foundations. 
    
  \item{\bf Plato}

    Plato reconciled the philosophies of Parmenides and Heraclitus by recognising two distinct world, a platonic realm of ideal forms and a shadowy world of sensory appearances.
    Only of the former can we have true knowledge, rationally obtained by reasoning from innate knowledge of the forms remembered by the soul from a previous bodily existence.
    Of the world of appearances, constantly shifting, we can have only opinion.
    This philosophy is the progenitor of the \emph{rationalist} tradition which emphasises rationality as the source of knowledge, as opposed to \emph{empiricism} which considered knowledge as obtainable only, or primarily, through the senses.
    
  \item{\bf Aristotle}

    Plato's reconciliation of the contradictory ideas about knowledge which preceded him was soon contradicted by his pupil Aristotle, whose epistemology emphasised empirical knowledge, but also the role of reason deriving new knowledge from the empirically rooted fundamental principles of each science.
    A central achievement of Aristotle's philosophy was his normative conception of scientific method expressed through his account of \emph{demonstrative science} in the collection of works known as ``The Organon''.
    
  \item{\bf Bacon/Newton/Locke}

    Our next milestone is more than a thousand years later as modern science emerges in the European \emph{renaissance}.
    This was a lengthy process rather than a single event, Bacon, Newton and Locke among the scientists who featured in this process as \emph{empiricists} or \emph{expermiental scientists}.
    
    
  \item{\bf Leibniz}
  \item{\bf Hume}
  \item{\bf Frege}
  \item{\bf Carnap}
  \end{description}

\section{Space Synthetic Biology}

\subsection{Progression of Synthetic Biology}




\begin{description}
\item Genetic design of viruses, monocellular organisms 
\item Medical treatment by genetically engineered interventions
\item Gene line therapies (replacement of faulty genes)
\item Gene line enhancement (choice of optimal gene sets)
\item Evolution by Design
\item Ecosystem design for human space travel.
\item Cognitive ecosystem design for interstellar proliferation.
\end{description}

``Space Sythetic Biology'' is one of the more futuristic parts of the new biology which began with the discovery of the structure of DNA and progressed through genetic engineering (tinkering with DNA to get ``GMO''s) and synthetic biology (designing and creating completely new ``synthetic'' organisms and other more radical tranformations to the nature of life itself, including changing the genetic code).

Space synthetic biology (an embryonic discipline) is the engineering of whole ecosystems for transporting human explorers across the solar system and for supporting off-earth colonies.
When we come to consider exploration beyond the solar system, sending human beings becomes ever more difficult, and the engineering of self-proliferating intelligent systems will likely also fall to space synthetic biology.
 as a multidisciplinary enterprise in which \emph{cognitive} and \emph{computer} sciences have their place, possibly supported by a new kind of constructive philosopher, one of whose contributions is synthetic epistemology.


\section{Synthetic Cosmic Epistemology}





%\addcontentsline{toc}{section}{Bibliography}
%\bibliographystyle{alpha}
%\bibliography{rbj2}

%\addcontentsline{toc}{section}{Index}\label{index}
%{\twocolumn[]
%{\small\printindex}}

%\vfill

%\tiny{
%Started 2017-10-09

%Last Change 2017-10-09

%\href{http://www.rbjones.com/rbjpub/www/papers/p028.pdf}{http://www.rbjones.com/rbjpub/www/papers/p028.pdf}

%}%tiny

\end{document}

% LocalWords:
