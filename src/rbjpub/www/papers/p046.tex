% $Id: p046.tex $fi
% bibref{rbjp046} pdfname{p046}
\documentclass[10pt,titlepage]{article}
\usepackage{makeidx}
\newcommand{\ignore}[1]{}
\usepackage{graphicx}
\usepackage[unicode]{hyperref}
\pagestyle{plain}
\usepackage[paperwidth=5.25in,paperheight=8in,hmargin={0.75in,0.5in},vmargin={0.5in,0.5in},includehead,includefoot]{geometry}
\hypersetup{pdfauthor={Roger Bishop Jones}}
\hypersetup{pdftitle={Parents and Education}}
\hypersetup{colorlinks=true, urlcolor=red, citecolor=blue, filecolor=blue, linkcolor=blue}
%\usepackage{html}
\usepackage{paralist}
\usepackage{relsize}
\usepackage{verbatim}
\usepackage{enumerate}
\usepackage{longtable}
\usepackage{url}
\newcommand{\hreg}[2]{\href{#1}{#2}\footnote{\url{#1}}}
\makeindex

\title{\LARGE\bf Parents and Education}
\author{Roger~Bishop~Jones}
\date{\small 2022/01/19}


\begin{document}

%\begin{abstract}
% Who should know about and/or have influence over, what is taught to children.
%\end{abstract}
                               
\begin{titlepage}
\maketitle

%\vfill

%\begin{centering}

%{\footnotesize
%copyright\ Roger~Bishop~Jones;
%}%footnotesize

%\end{centering}

\end{titlepage}

\ \

\ignore{
\begin{centering}
{}
\end{centering}
}%ignore

\setcounter{tocdepth}{2}
{\parskip-0pt\tableofcontents}

%\listoffigures

\pagebreak


\section{Introduction}

The topic is ``\emph{Parents and Education}'', which is expanded in the first instance as:

\begin{enumerate}
\item Should parents
  \begin{itemize}
  \item[(a)] be informed about

    and/or
  \item[(b)] have influence on
    \end{itemize}
what is being taught to their children?

\item Is the answer to (1):
  \begin{itemize}
  \item[(a)] a moral imperative

    and/or
    \item[(b)] a democratic choice?
  \end{itemize}
\item To what extent is UK primary education consistent with the answers,
and if not, what would it take to bring it in line?
\end{enumerate}

These notes will not be tightly focussed on those specific questions.
To understand why this has come to be a matter of controversy some contemporary context is needed.

In brief, a natural response to that question is, ``why?''.
Why would parents feel the need to see in detail what was being taught to their children or to have any unfluence upon it?
I think for most of my life it would not have occurred to me that it should be necessary.
We have professionals who are better qualified to look after these things, public servants who in the end are answerable to democratically elected governments.

Another relevant ``why?'' here is ``why would anyone object to parents knowing what is being taught to their children?'', and in one possible answer to that question we see a reason why parents might seek influence: ``because they are intent on teaching things which parents would find objectionable''.


\section{Controversy on Curriculum and Teaching Methods}

Here are some possible areas of controversy which might lead parents to feel that they need to know or want to influemce what is being taught, or on which we might expect the state to exercise diligence and exercise control.

\begin{itemize}

\item[\bf religion:]

  \begin{itemize}
    
  \item Should fundamentalist religous or radical political views be taught (``as fact'', rather than discussed)?
  \item Should parents be able to influence the content of sex education?
    \item Should religious beliefs which conflict with received science be taught alongside or instead?
  \end{itemize}
  
\item[\bf curriculum balance:]

  STEM v humanities v sports and arts/music

  \item teaching ethos (progressive v. traditional, by discovery v. by instruction)

  \item[\bf political:]
    Radical political groups may seek ideologically based instruction.
    Knowing parents are likely to disapprove, they may conceal curricula detail, hiding it behind inoccuous gloss, and may resist parental or state influence (ignoring parent lobby groups and even the law).

    \item Save Our Schools v. Safe School Alliance
    
\end{itemize}

\section{Three Levels of Engagement}

The three are:

\begin{itemize}
\item Pragmatic/Methodological

  e.g. learn by discovery v. more traditional teaching

\item Progressive/Political

DEI inspired curricula, identity politics
  
\item Revolutionary/Ideological

  advancement of revolutionary agendas by teaching contrary to cultural norms, and radical divisive ideologies intended to create social conflict
\end{itemize}

\section{The Political Progressive Dimension}

This is where those belong who, however radical their views are on how education and society at large should be transformed, are intent on progressive reform rather revolutionary dismantlement.

The naive expectation in the first instance (it is perhaps a testimony to the era in which some of us grew up that such a naive conception might be possible) is that the main issues are not about the purpose of education, but rather about which methods are most effective in realising that purpose and about what kind of content should be prioritised, about the balance across the kinds of things that could be taught.

In that naive state, the idea that the there might be substantive disagreements not about what areas to include in the curriculum, but also about what the truths are in any area.
Beyond the possibility of dispute over the ``facts'', there is the question to what extent education should go beyond facts into values and morals.

In my life, it has been possible to believe that there is a broad consensus about facts, underpinned by various institutions whose business it is to look into the facts, and that we have a shared culture and a large measure of agreement about values (though this retrospective opinion may be unduly influenced by a sense of ongoing deteriorarion).
Educational instutions were primarily charged with teaching facts and skills, while upholding the moral compass of the age, which would primarily be transferred by osmosis and example.

No doubt that's a gross oversimplification or even badly at odds with your perceotions, but perhaps its a useful contrast with the present situation, in which consensus on the facts is disintegrating, trust in any kind of authority is greatly diminished, the educational curriculum has become a target for activists seeking to prevent the transfer of culture from parents to children and teach their own doctrines as not only factual but facts with moral force behind them.

\section{Revolution}

Under this heading are those who regard the existing system as irredemable and in need of complete destruction before anything better can be put in its place.

\cite{pluckrose-cynical,lindsay-racemarx,friere-poled,gottesman-criturn}

\appendix

\section{Critical Pedagogy}


\cite{friere-poled,gottesman-criturn}


\phantomsection
\addcontentsline{toc}{section}{Bibliography}
\bibliographystyle{rbjfmu}
\bibliography{rbj}

%\addcontentsline{toc}{section}{Index}\label{index}
%{\twocolumn[]
%{\small\printindex}}

%\vfill

\tiny{
Started 2022/01/15


\href{http://www.rbjones.com/rbjpub/www/papers/p046.pdf}{http://www.rbjones.com/rbjpub/www/papers/p046.pdf}

}%tiny

\end{document}

% LocalWords:
