% $Id: p021.tex,v 1.1 2014/11/08 19:43:29 rbj Exp $
% bibref{rbjp021} pdfname{p021}

\documentclass[10pt,titlepage]{article}
\usepackage{makeidx}
\usepackage{graphicx}
\usepackage[unicode,pdftex]{hyperref}
\pagestyle{plain}
\usepackage[paperwidth=5.25in,paperheight=8in,hmargin={0.75in,0.5in},vmargin={0.5in,0.5in},includehead,includefoot]{geometry}
\hypersetup{pdfauthor={Roger Bishop Jones}}
\hypersetup{colorlinks=true, urlcolor=red, citecolor=blue, filecolor=blue, linkcolor=blue}
\usepackage{html}
\usepackage{paralist}
\usepackage{relsize}
\usepackage{verbatim}
\makeindex
\newcommand{\ignore}[1]{}

\title{Abstract for SOTFOM II}
\author{Roger~Bishop~Jones}
\date{\ }

\begin{document}
%\frontmatter
                               
\begin{titlepage}
\maketitle

\begin{abstract}
More than one attempt at an abstract for submission to SOTFOM II, an upcoming symposium on the foundations of mathematics.
\end{abstract}

%\vfill

%\begin{centering}

%{\footnotesize
%copyright\ Roger~Bishop~Jones;
%}%footnotesize

%\end{centering}

\end{titlepage}

\setcounter{tocdepth}{2}
{\parskip-0pt\tableofcontents}

%\listoffigures

%\mainmatter

\section{Introduction}

This is a document for drafting abstracts for possible submission to SOTFOM II, and possibly drafting additional material not to be included in the abstract.
Maybe also for material arising from FOM discussions on related topics.

In this section I include for my convenience the description of the symposium and the call for papers.
In the next, several draft abstracts.
If the subsection containing a draft has a title, then that is the proposed title for the talk.
In subsequent sections, further materials.

The description of the symposium is:

\begin{quotation}
The focus of this conference is on different approaches to the foundations of mathematics. The interaction between set-theoretic and category-theoretic foundations has had significant philosophical impact, and represents a shift in attitudes towards the philosophy of mathematics. This conference will bring together leading scholars in these areas to showcase contemporary philosophical research on different approaches to the foundations of mathematics. To accomplish this, the conference has the following general aims and objectives. First, to bring to a wider philosophical audience the different approaches that one can take to the foundations of mathematics. Second, to elucidate the pressing issues of meaning and truth that turn on these different approaches. And third, to address philosophical questions concerning the need for a foundation of mathematics, and whether or not either of these approaches can provide the necessary foundation.
\end{quotation}

The call for papers says:
\begin{quotation}
We welcome submissions from scholars (in particular, young scholars, i.e. early career researchers or post-graduate students) on any area of the foundations of mathematics (broadly construed). While we welcome submissions from all areas concerned with foundations, particularly desired are submissions that address the role of and compare different foundational approaches. Applicants should prepare an extended abstract (maximum 1,500 words) for blind review, and send it to sotfom [at] gmail [dot] com, with subject `SOTFOM II Submission’.
\end{quotation}

\section{Draft Abstracts}
\pagebreak
\subsection{Foundational Pluralism and Semantic Embedding}

In discussing the development of the philosophy of Rudolf Carnap it seems sometimes to be suggested that his ``Logical Syntax of Language'' represents an overturning of a previous ``universalist'' position on logical foundations, replacing this with the pluralism embodied in his ``principle of tolerance''.
We can see. however, from Carnap's ``intellectual autobiography'', that the attitude which was later enunciated in \emph{the principle of tolerance} dates back to his student days, and was not thought by him to be in conflict with his work on logical foundations predating \emph{logical syntax}.
If we understand the progression here as an elaboration rather than a repudiation of his earlier position, then we may see that Carnap's early foundationalism in respect of Mathematics, becomes, in the light of his principle of tolerance, a ``foundational pluralism'' both for mathematics and for the empirical sciences.

The principle of tolerance asserts that the legitimacy of formal logical systems is independent of metaphysical considerations, particularly of ontological prejudice, and that thus liberated we are free to chose which logical systems to use on a pragmatic basis.
Carnap then concerns himself with metatheory and method, using particular formal systems to illustrate the methods whereby the syntax and semantics of logical systems can be rigorously defined.
The passage of time since has seen a great proliferation of logical systems, many dicussions of their relative merits, but few objective yardsticks wth which to make comparisons and come to choices.

It is a natural extension of Carnap's programme, falling naturally into a domain to which Carnap gave the name ``language planning'', to consider what kinds of metric might be applied to the evaluation of logical systems.

\pagebreak
\subsection{Perspectives on Foundational Evaluation}

The foundations of mathematics are naturally of greatest interest to Mathematicians and Philosophers.
The philosopher Rudolf Carnap thought of his role as facilitating new, formal, methods for science, to which the notion of mathematical foundation system had relevance, and contemporary Computer Science develops tools and methods for use by software and hardware engineers which may be built around logical foundation systems.
Carnap's \emph{principle of tolerance}, and the logical and foundational pluralism which flows from it, is accompanied by an emphasis on pragmatic evaluation of logical systems.
A similar emphasis might be expected from Computer Scientists and Computer Systems Engineers in their evaluation of foundation systems.
The comparisons between foundations which seem most relevant to such groups may diverge substantially from those which come easily to the minds of mathematicians and philosophers.
Part of the purpose of this essay is to seek philosophical insight by comparing these different pespectives on the evaluation of foundation systems.

I am therefore concerned here, not simply with comparing foundation systems, but with understanding the different kinds of comparison which arise from distinct conceptions of the role of foundation systems arising in varied academic and industrial contexts.
A subtheme, highlighted by the title of the essay, is to note connections between the philosophy of Carnap, oriented toward the support of formal science, and the methods and tools under continuous development by computer scientists for the support of formal methods in software, electronic hardware and other kinds of engineering.

I begin with the question, ``what is a foundation for mathematics, and what is its purpose and role?'' considering briefly some distinct views on such matters before offering a single conception which delimits the methods of evaluation and comparison to be considered in the sequel.
This conception of foundation system flows primarily and directly from the ideas of Frege and Carnap, and its articulation naturally leads to certain relatively definite partial orderings on foundation systems.
Two such orderings are suggested directly by Frege's expectation of a foundation system that it support the definition of the concepts of mathematics (on the one hand) and the derivation (from those definitions) of its theorems (on the other).
The later discoveries that such systems are of necessity incomplete, both semantically and syntactically, tells us that neither of these desiderata can be fully realised, and invites comparison between foundation systems based on \emph{how much} each can define, or on what they can prove.

\pagebreak
\subsection{Mathematical Foundations for Engineers}

It is often instructive to see a familiar subject matter from a novel perspective.
The foundations of mathematics are perhaps of greatest interest to mathematicians, mathematical logicians and philosophers, but the major part of this interest in metatheoretic.
Few make practical use of mathematical foundation systems, in the sense of using them by formally deriving mathematical theorems.

When it comes to comparing foundation systems, pragmatic considerations are likely to take second place to theoretical or philosophical considerations.

These foundation systems are nevertheless used in anger by some groups.
One diverse such group is the computer scientists who develop tools for formal verification, and the engineers who apply those tools in the development and verification of software, of digital hardware, or in a variety of other ways.

One such group is the community which has grown over the last 3 or 4 decades around the Cambridge HOL system and other proof tools supporting the same logical system.
These systems may be though of, and are often talked of, as implementing Alonzo Church's Simple Theory of Types.

\section{FOM Stuff}

This section is for material thought of as responses to issues raised on the FOM mailing list which relate well to the SOTFOM agenda.
Of course. this is LaTeX and FOM isn't, so if I did decide to post any of this then it would have to be re-formatted in plain text.

\subsection{Maximality}

I would like here to respond to some of the recent writings of Harvey Friedman under the heading:

\begin{quote}
Foundational Methodology -/Maximality
\end{quote}
.

Specifically Harvey addresses the following three questions:

\begin{enumerate}
\item Going deeply into what "maximality in set theory" really means, or can mean.
\item Whether there is a fundamental idea of "maximality in set theory"
that generates AxC. Same for other axioms of ZFC.
\item Whether there is a nontrivial fundamental idea of "maximality in
set theory" at all that meets certain fundamental criteria.
\end{enumerate}

Harvey says that his modus operandum is:

\begin{quote}
Use philosophy and philosophical considerations to generate foundational programs.
\end{quote}
.
My perspective on these issues is distinct from Harvey's.
I am not an academic, but have a lifetime interest in logic and the foundations
of mathematics and their application outside academia.
I have been professionally involved in the development of proof technology and in its
applications in industry.
Despite that background, my thinking is generally philosophical in character.

\subsection{The Foundational Role of Set Theory}

Set theory is both a ``foundation system'' providing a formal context in which most kinds of mathematics can be undertaken, and also a branch of mathematics in its own right.
It is entirely possible that what is good for set theory \emph{as a foundation system} and what might be conducive to its further development as a theory in its own right are quite different.
I am here concerned exclusively with its foundational role.

\subsection{What Maximality in Set Theory Might Mean}

I won't attempt to say what it really means, but here are some ideas on what could mean.

Here is a simplistic way of cutting mathematics into two pieces.
Sometimes mathematicians study particular structures, such as number systems.
Sometimes they study a whole class of structures typically determined by some axioms.

In the first case it there will often be no complete formalisation of the relevant theory, and to get a more complete theory one will seek axioms which are true of the structure under consideration but independent of the existing axioms.
Often this will be done in the context of set theory, and the structure of interest is \emph{defined} (in set theory) rather than axiomatised.
In this case the theory will still be incomplete, but to get a more complete theory its not the definition which is upgraded, but rather the axioms of set theory which facilitate reasoning from the definitions.

In the second case, because the chosen axioms are definitive of the class of structures under study, their consequences are likely to be all the theorems which are true of all the models of the axioms, and axiomatic extensions serve only to determine a narrower group of structures, rather than to give more complete deductive account of the chosen subject matter.
In these cases the choice of additional axioms will not be made by looking for unprovable truths, but by consideration of the merits of tne resulting theory.

In the former case, when an axiom is justified by arguments as to its truth, we call the justification `intrinsic'. in the latter, `extrinsic'.

I will here be concerned only with intrinsic justification, but some further remarks about the extrinsic cases.

Let us consider group theory as an example of the second case, in which axiomatic extensions are likely to have extrinsic justifications.
The study of group theory does not purely consist in deriving consequences of the axioms in a first order theory of groups.
The subject matter of group theory is not some generic group, but rather the category of groups (even if category theory is not itself deployed).
This investigation has traditionally been conceived of as taking place in set theory, either using the group axioms as defining characteristic of the sets and morphisms of interest, or reasoning metatheoretically about the models of the first order theory of groups.
The incompleteness of set theory may in that case present motives and opportunities for intrinsically justified axiomatic extensions. alongside extrinsically justified elaboration of the axiomatic refinement of the structures being studied. 





%\backmatter

%\appendix

%\addcontentsline{toc}{section}{Bibliography}
%\bibliographystyle{alpha}
%\bibliography{rbj}

%\addcontentsline{toc}{section}{Index}\label{index}
%{\twocolumn[]
%{\small\printindex}}

%\vfill

%\tiny{
%Started 2012-10-19

%Last Change $ $Date: 2014/11/08 19:43:29 $ $

%\href{http://www.rbjones.com/rbjpub/www/papers/p019.pdf}{http://www.rbjones.com/rbjpub/www/papers/p019.pdf}

%Draft $ $Id: p021.tex,v 1.1 2014/11/08 19:43:29 rbj Exp $ $
%}%tiny

\end{document}

% LocalWords:
