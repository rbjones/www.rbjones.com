% $Id: p026.tex$
% bibref{rbjp026} pdfname{p026} 

\documentclass[10pt,titlepage]{book}
\usepackage{makeidx}
\usepackage{graphicx}
\usepackage[unicode]{hyperref}
\pagestyle{headings}
\usepackage[twoside,paperwidth=5.25in,paperheight=8in,hmargin={0.75in,0.5in},vmargin={0.5in,0.5in},includehead,includefoot]{geometry}
\hypersetup{pdfauthor={Roger Bishop Jones}}
\hypersetup{colorlinks=true, urlcolor=red, citecolor=blue, filecolor=blue, linkcolor=blue}
%\usepackage{html}
\usepackage{paralist}
\usepackage{relsize}
\usepackage{verbatim}
%\bodytext{BGCOLOR="#eeeeff"}
\makeindex
\newcommand{\indexentry}[2]{\item #1 #2}
\newcommand{\glossentry}[2]{\item #1 {\index #1 #2}}
\newcommand{\ignore}[1]{}
\def\Product{ProofPower}
\def\ouml{\"o}
\def\auml{\"a}

\title{A Philosophical Kernel for Formal Deductive Analysis}
\date{\ }

%\begin{abstract}
%Positivism, latterly logical positivism and particularly the philosophy of Rudolf Carnap, was a scientifically oriented philosophy, in which the aim of philosophy was to support science, and philosophy was expected to strive for scientific standards of objectivity and rigour.
%A particular impetus for logical positivism and the philosophy of Carnap was the advances made in logic by Frege, and the conception of scientific philosophy advanced by Russell, in which philosophy was to be purely logical.
%However, beyond the advances which permitted deductive reason to be entirely formalised, the philosophies of Russell and Carnap were influenced by empiricist agendas logically distinct from an advocacy of logical rigour, which also inspired a rejection of most of the philosophical tradition and made a broad acceptance, even of the pure logical core, unlikely.
%In this essay I attempt to isolate a core philosophical framework suitable to underpin the widest adoption of formal methods to ensure logical rigour in all areas where deductive reasoning may have a role to play.
%In practice, at present, the principal users of formal derivations and of the software to facilitate their application are not logicians, mathematicians, scientists or philosophers, but engineers, who seek clarity in presentation of their designs and assurance that they realise their intended purpose.
%\end{abstract}

\begin{document}
\frontmatter
                               
\begin{titlepage}
\maketitle

\hspace{2in}

\vfill

\begin{centering}

Published by Roger Bishop Jones\\
www.rbjones.com\\
\vspace{0.2in}

ISBN-13: \\
ISBN-10: 

\vspace{0.2in}

{\footnotesize

First edition. Revision: 1 Date: 2016/06/05

\vspace{0.2in}

\copyright\ Roger~Bishop~Jones;

}%footnotesize

\end{centering}


\thispagestyle{empty}
\end{titlepage}

\setcounter{tocdepth}{2}
{\parskip-0pt\tableofcontents}
\listoffigures

\mainmatter

\addcontentsline{toc}{section}{Preface}

\section*{Preface}

In 1986 I had the opportunity to work on the development of secure computer hardware and software using the Camnbridge HOL system, a variant of Higher Order Logic with good extensible software support for developing formal specifications and proofs.
In this way I became familiar with formal methods which had their origin in philosophy but whose significance and value, I came to believe, were underestimated by philosophers.

In 1986, the company I worked for had moved away from this line of research, and my philosophical inclinations lead me to think about the broader significance of the methods we had become immersed in.
While seeking to give a concise account of the philosophical underpinnings for these methods I found that the story I wanted to give was simply not acceptable to much of the philosophical community.

One philosophical concept seemed particularly problematic, and exemplifies the difficulties.
The concept of {\it analyticity}.
A particular difficulty faced me in relation to the rejection of this concept by many philosophers.
It is clear that this rejection arose primarily from a series of criticisms by Quine of the ideas of Carnap, culminating in the influential essay ``Two Dogmas of Empiricism'' in which Quine advanced from rasing doubts to outright repudiation.

\chapter{Introduction}

Throughout the history of western philosophy there have been philosophers who have been sceptical of the rigour of philosophical arguments, have envied the rigour and fecundity of deduction in mathematics, and have sought ways in which philosophy might realise those standards of rigour.
Some of the philosophers seeking deductive rigour in philosophy have also aspired to facilitate rigorous deduction in the whole of science.
Leibniz by similar means, sought the automation of reason.

To the extent that philosophers have sought to address these aspirations through formalisation (which is our present concern), they have been hampered through almost the whole of that history by inadaquate logical systems for the formalisation of those domains.
It is my purpose here to argue that those days are now well past, and that some simple practical systems (logical systems with software support) suffice (at least in the logic they support) for the formalisation of all sound deductive argument.

Logic was first studied by Aristotle, who made the first steps toward formalisation in his work on the syllogism.
This formed a part of Aristotle's conception of ``demonstrative science'' in which the whole of scientific knowledge was seen as deductively derived from the fundamental principles of each domain of knowledge.

A broadly similar conception of science appears in modern times, sometimes known as the nomological-deductive conception of science.
In this conception it is the role of scientists to formulate ``laws'', preferably quantitative laws constituting a mathematical model of the relevant phenomena, from which both a general theory and the details of particular application are derived deductively.

However, neither in Aristotle's time nor in the beginnings of the modern times was the formal logic then known sufficient for the necessary deductions, which were undertaken in an informal way.
In the case of mathematics and its applications to science and elsewhere the clarity of the concepts involved was sufficient that these deductive derivations were sufficiently reliable for the establishment of a viable body of theory and it application.
Philosophy fared less well.

\section{The Genesis of HOL}

The HOL logic as implemented in the Cambridge HOL system (hol4) and some other tools (including ProofPower) is descended from Russell's theory of types, and the earlier philosophy and logic of Gottlob Frege.
In presenting an account of its merits it may be helpful to sketch the stages in this developmemt.

We could go all the way back to Aristotle. and there are connections going back that far which are relevant to our present concerns, but for the present purpose a good starting point is with the work of Frege, whose logical innovations were just the kind of break with Aristotelian logic needed for the great formalisation project to succeed.

To understand Frege it is good to begin with the philosophical position on the status of mathematics which came later to be called {\it logicism}.

The idea that mathematics is logical, not necessarily in those words, is present in many of Frege's philosophical predecessors, for example, very explicitly in Leibniz, and more obliquely in Hume's account of his ``fork'' (the distinction between {\it matters of fact} and {\it relations between ideas}).
It is the rejection of this idea in Kant which provoked Frege's desire to show that mathematics is logical in a manner suggested by the following dictum:

{\large
\begin{centering}
  mathematics = logic + definitions
\end{centering}
}

This tells us that all we need to do mathematics is to have the definitions of mathematical concept and derive the desired theorems from those definitions by deduction.
Kant had argued that mathematical propositions had content which went beyond the definition of the concepts, and that logic alone would not suffice to derive the theorems of mathematis from the definitions of its concepts.
Frege sought to show him wrong.

Two principal difficulties were found in Frege's plan (apart from the problem of execution which resulted in Frege's system being inconsistent).
The first was the status if the logical systems necessary for the formulation of the necessary definitions.
The kind of definition required for mathematical entities such as numbers is one in which numbers are identified with entities already known to exist,
A definition could not be allowed to introduce new entities, and so a logical system which 






\chapter{}

\backmatter

%\chapter*{Glossary}\label{glossary}
%\addcontentsline{toc}{chapter}{Glossary}
%
%\begin{description}
%\item[]
%\end{description}

\addcontentsline{toc}{chapter}{Bibliography}
\bibliographystyle{alpha}
\bibliography{rbj}

\addcontentsline{toc}{chapter}{Index}\label{index}
{\twocolumn[]
{\small\printindex}}

\vfill

\tiny{
Started 2010-01

Last Change $ $Date: 2014/11/08 19:43:29 $ $

\href{http://www.rbjones.com/rbjpub/www/papers/p008.pdf}{http://www.rbjones.com/rbjpub/www/papers/p008.pdf}

Draft $ $Id: p008.tex,v 1.37 2014/11/08 19:43:29 rbj Exp $ $
}%tiny

\end{document}

% LocalWords:  Arist
