% $Id: p026.tex$
% bibref{rbjp026} pdfname{p026} 

\documentclass[10pt,titlepage]{book}
\usepackage{makeidx}
\usepackage{graphicx}
\usepackage[unicode,pdftex]{hyperref}
\pagestyle{headings}
\usepackage[twoside,paperwidth=5.25in,paperheight=8in,hmargin={0.75in,0.5in},vmargin={0.5in,0.5in},includehead,includefoot]{geometry}
\hypersetup{pdfauthor={Roger Bishop Jones}}
\hypersetup{colorlinks=true, urlcolor=red, citecolor=blue, filecolor=blue, linkcolor=blue}
%\usepackage{html}
\usepackage{paralist}
\usepackage{relsize}
\usepackage{verbatim}
%\bodytext{BGCOLOR="#eeeeff"}
\makeindex
\newcommand{\indexentry}[2]{\item #1 #2}
\newcommand{\glossentry}[2]{\item #1 {\index #1 #2}}
\newcommand{\ignore}[1]{}
\def\Product{ProofPower}
\def\ouml{\"o}
\def\auml{\"a}

\title{A Philosophical Kernel for Formal Deductive Analysis}
\date{\ }

%\begin{abstract}
%Positivism, latterly logical positivism and particularly the philosophy of Rudolf Carnap, was scientifically oriented philosophy, in which the aim of philosophy was to support science, and philosophy was expected to strive for scientific standards of objectivity and rigour.
%A particular impetus for logical positivism and the philosophy of Carnap was the advances made in logic by Frege, and the conception of scientific philosophy advanced by Russell, in which philosophy was to be purely logical.
%However beyond the advances which permitted deductive reason to be entirely formalised, the philosophy of Russell and Carnap was influenced by empiricist agendas logically distinct from an advocacy of logical rigour through formalisation, which also inspired a rejection of most of the philosophical tradition and made a broad acceptabce even of the pure logical core by philosophers unlikely.
%In this essay I attempt to isolate a core philosiphical framework suitable to underpin the widest adoption of formal methods to ensure logical rigour in all areas where deductive reasoning may have a role to play.
%In practice, at present, the principal users of formal derivations and of the software to facilitate their application are not logicians, mathematicians, or philosophers, but engineers, who seek clarity in presentation of their designs and assurance that they realise their intended purpose.
%\end{abstract}

\begin{document}
\frontmatter
                               
\begin{titlepage}
\maketitle

\hspace{2in}

\vfill

\begin{centering}

Published by Roger Bishop Jones\\
www.rbjones.com\\
\vspace{0.2in}

ISBN-13: \\
ISBN-10: 

\vspace{0.2in}

{\footnotesize

First edition. Revision: 1 Date: 2016/06/05

\vspace{0.2in}

\copyright\ Roger~Bishop~Jones;

}%footnotesize

\end{centering}


\thispagestyle{empty}
\end{titlepage}

\setcounter{tocdepth}{2}
{\parskip-0pt\tableofcontents}
\listoffigures

\mainmatter

\addcontentsline{toc}{section}{Preface}

\section*{Preface}

In 1986 I had the opportunity to work with the Camnbridge HOL system, a variant of Higher Order Logic with good extensible software support for developing formal specifications and proofs, on the development of secure computer hardware and software.
In this way I became familiar with formal methods which had their origin in philosophy but whose significance and value, I came to believe, were greatly underestimated by philosophers.

In 1986, the company I worked for had moved away from this line of research, and my philosophical inclinations lead me to think about the broader significance of the methods we had become immersed in.
While seeking to give a concise account of the philosophical underpinnings for these methods I found that the story I wanted to give was simply not acceptable to much of the philosophical community.

One philosophical concept seemed particularly problematic, and exemplifies the difficulties.
The concept of {\it analyticity}


\chapter{Introduction}

Throughout the history of western philosophy there have been philosophers who have been sceptical of the rigour of philosophical arguments, have envied the rigour and fecundity of deduction in mathematics, and have sought ways in which philosophy might effectively emulate the standards of mathematics.
Some of these philosophers seeking deductive rigour in philosophy have also aspired to facilitate rigorous deduction in the whole of science.
Leibniz by similar means, sought the automation of reason.

To the extent that philosophers have sought to address these aspirations through formalisation (which is our present concern), they have been hampered through almost the whole of that history by inadaquate logical systems for the formalisation of those domains.
It is my purpose here to argue that those days are now well past, and that some simple practical systems (logical systems with software support) suffice (at least in the logic they support) for the formalisation of all sound deductive argument.

\section{The Genesis of HOL}

The HOL logic as implemented in the Cambridge HOL system (hol4) is descended from Russell's theory of types, and the earlier philosophy and logic of Gottlob Frege.
In presening an account of its merits it may be helpful to sketch the stages in this developmemt.

Of course, we could go all the way back to Aritotle. and there are connections going back that far which are relevant to our present concerns, but for the present purpose a good starting point is with the work of Frege, whose logical innovations were just the kind of break with Aristotelian logic needed for the great formalisation project to succeed.

To understand Frege it is good to begin with the philosophical position on the status of mathematics which came later to be called {\it logicism}.

The idea that mathematics is logical, not nevessarily in those words, is present in many of Frege's philosophical predecessors, for example, very explicitly in Leibniz, and more obliquely in Hume's account of his ``fork''.
It is the rejection of this idea in Kant which provoked Frege's desire to show that mathematics is logical in a manner suggested by the following dictum:

{\large
\begin{centering}
  mathematics = logic + definitions
\end{centering}
}

This tells us that all we need to do mathematics is to have the definitions of mathematical concept and derive the desired theorems from those definitions by deduction.
Kant had argued that mathematical propositions had content which went beyond the definition of the concepts.



\chapter{}

\backmatter

%\chapter*{Glossary}\label{glossary}
%\addcontentsline{toc}{chapter}{Glossary}
%
%\begin{description}
%\item[]
%\end{description}

\addcontentsline{toc}{chapter}{Bibliography}
\bibliographystyle{alpha}
\bibliography{rbj}

\addcontentsline{toc}{chapter}{Index}\label{index}
{\twocolumn[]
{\small\printindex}}

\vfill

\tiny{
Started 2010-01

Last Change $ $Date: 2014/11/08 19:43:29 $ $

\href{http://www.rbjones.com/rbjpub/www/papers/p008.pdf}{http://www.rbjones.com/rbjpub/www/papers/p008.pdf}

Draft $ $Id: p008.tex,v 1.37 2014/11/08 19:43:29 rbj Exp $ $
}%tiny

\end{document}

% LocalWords:  Arist
