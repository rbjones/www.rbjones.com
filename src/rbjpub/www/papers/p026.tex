% $Id: p026.tex$
% bibref{rbjp026} pdfname{p026} 

\documentclass[10pt,titlepage]{book}
\usepackage{makeidx}
\usepackage{graphicx}
\usepackage[unicode,pdftex]{hyperref}
\pagestyle{headings}
\usepackage[twoside,paperwidth=5.25in,paperheight=8in,hmargin={0.75in,0.5in},vmargin={0.5in,0.5in},includehead,includefoot]{geometry}
\hypersetup{pdfauthor={Roger Bishop Jones}}
\hypersetup{colorlinks=true, urlcolor=red, citecolor=blue, filecolor=blue, linkcolor=blue}
%\usepackage{html}
\usepackage{paralist}
\usepackage{relsize}
\usepackage{verbatim}
%\bodytext{BGCOLOR="#eeeeff"}
\makeindex
\newcommand{\indexentry}[2]{\item #1 #2}
\newcommand{\glossentry}[2]{\item #1 {\index #1 #2}}
\newcommand{\ignore}[1]{}
\def\Product{ProofPower}
\def\ouml{\"o}
\def\auml{\"a}

\title{The Formalisation of Knoweldge in HOL}
\date{\ }

%\begin{abstract}
%From Aristotle to Carnap there have been philosophers who have sought the improvement of scientific and philosophical rigour through the use of formal logic.
%Until very recently they have lacked adequate logical systems for the realisation of their ambitions.
%The situation is now changed, the inadequacies of logic are no longer an impediment to the main thrust of this tendency.
%In this document some reasons for considering the Cambridge HOL system an adequate logical basis are presented.
%\end{abstract}

\begin{document}
\frontmatter
                               
\begin{titlepage}
\maketitle

\hspace{2in}

\vfill

\begin{centering}

Published by Roger Bishop Jones\\
www.rbjones.com\\
\vspace{0.2in}

ISBN-13: \\
ISBN-10: 

\vspace{0.2in}

{\footnotesize

First edition. Revision: 1 Date: 2016/06/05

\vspace{0.2in}

\copyright\ Roger~Bishop~Jones;

}%footnotesize

\end{centering}


\thispagestyle{empty}
\end{titlepage}

\setcounter{tocdepth}{2}
{\parskip-0pt\tableofcontents}
\listoffigures

\mainmatter

\addcontentsline{toc}{section}{Preface}

\section*{Preface}

\chapter{Introduction}

In this introductory chapter I propose the lightest possible sketch of the whole, identifying the problem which I am here concerned with, and the present solution which I perceive.

I will tell first a little story about how I came upon the problem and perceived what I now suggest is a solution, and then speak of some of the history of ideas, much of it then unknown to me, which lay behind them.

I was born in 1948, so that my acquaintance with electronic computers was almost zero prior to University.
These things were not to be seen in home or school, though one school trip to Bradford University to feed papers program's into their computer on 5-track paper tape sewed the seed of an enthusiasm soon to shape my life.
In the sixth form I studied only mathematics and physics (to any non-trivial extent), but level of concentration on mathematics did not suit me.
I sought something more practical, applying to study Engineering at University, and secured a place at Churchill college. Cambridge by taking the entrance exam.



\chapter{}

\backmatter

%\chapter*{Glossary}\label{glossary}
%\addcontentsline{toc}{chapter}{Glossary}
%
%\begin{description}
%\item[]
%\end{description}

\addcontentsline{toc}{chapter}{Bibliography}
\bibliographystyle{alpha}
\bibliography{rbj}

\addcontentsline{toc}{chapter}{Index}\label{index}
{\twocolumn[]
{\small\printindex}}

\vfill

\tiny{
Started 2010-01

Last Change $ $Date: 2014/11/08 19:43:29 $ $

\href{http://www.rbjones.com/rbjpub/www/papers/p008.pdf}{http://www.rbjones.com/rbjpub/www/papers/p008.pdf}

Draft $ $Id: p008.tex,v 1.37 2014/11/08 19:43:29 rbj Exp $ $
}%tiny

\end{document}

% LocalWords:  Arist
