% $Id: p025.tex $
% bibref{rbjp024} pdfname{p024}

\documentclass[12pt,titlepage]{article}
\usepackage{makeidx}
\usepackage{graphicx}
\usepackage[unicode,pdftex]{hyperref}
\pagestyle{plain}
\usepackage[paperwidth=8.3in,paperheight=11.7in,hmargin={0.3in,0.3in},vmargin={0.5in,0.5in},includehead,includefoot]{geometry}
\hypersetup{pdfauthor={Roger Bishop Jones}}
\hypersetup{colorlinks=true, urlcolor=red, citecolor=blue, filecolor=blue, linkcolor=blue}
\usepackage{html}
\usepackage{paralist}
\usepackage{relsize}
\usepackage{verbatim}
\makeindex
\newcommand{\ignore}[1]{}

\title{Rationality}
\author{Roger~Bishop~Jones}
\date{\ }

\begin{document}
%\frontmatter
                               
\begin{titlepage}
\maketitle

\begin{abstract}
Notes for a philosophical discussion on rationality
\end{abstract}

%\vfill

%\begin{centering}

%{\footnotesize
%copyright\ Roger~Bishop~Jones;
%}%footnotesize

%\end{centering}

\end{titlepage}

\setcounter{tocdepth}{2}
{\parskip-0pt\tableofcontents}

%\listoffigures

%\mainmatter

\pagebreak

\begin{centering}
{\LARGE \bf Rationality}
\end{centering}

\section{Introduction}

Some questions to consider.

\begin{itemize}
\item What is rationality?
\item Is there a connection with "reason" and if so what?
\item Is it always good to be rational, or might it sometimes be better to be
irrational?
\item Is there a conflict between rationality and emotional sensitivity or
empathy?
(rationality is the hallmark of the enlightenment, the revolt against
which came as romanticism).
\item I'm thinking that we might consider the rationality or otherwise of various
possible policies concerning migration, terrorism and membership of the
European Union or of the campaigns for and against Brexit.
\end{itemize}

\section{What is Rationality?}



\section{Further}

We begin with some reasons why {\it rationality} is an interesting and important
concept to discuss philosophically.

First a few words about the importance of {\it reason}, for these two concepts are related,.
Arguably, it is our ability to reason about our environment which is the principle cause
of success as a species, insofar as science and technology result from observing closely
and reasoning about the world around us.

Beyond its relevance to science and technology, reason provides a civilised alternative
to physical violence in resolving conflicts of interest between individuals and groups.
When such reasonable methods of conflict resolution are systemised into large scale
social institutions, they facilitate a society which is fair and just in which individuals
can flourish.
The antithesis of reason, the resort to violence, is a prominent feature of terrorist
organisations intent on imposing extreme fundamentalist relious ideologies.
It seems important to keep a firm grip on these values in responding to the treats
we face, and in doing so, we should not be blind to the imperfections of our own
society.








%\backmatter

%\appendix

%\addcontentsline{toc}{section}{Bibliography}
%\bibliographystyle{alpha}
%\bibliography{rbj}

%\addcontentsline{toc}{section}{Index}\label{index}
%{\twocolumn[]
%{\small\printindex}}

%\vfill

%\tiny{
%Started 2012-10-19

%Last Change $ $Date: 2014/11/08 19:43:30 $ $

%\href{http://www.rbjones.com/rbjpub/www/papers/p019.pdf}{http://www.rbjones.com/rbjpub/www/papers/p019.pdf}

%Draft $ $Id: p022.tex,v 1.1 2014/11/08 19:43:30 rbj Exp $ $
%}%tiny

\end{document}

% LocalWords:
