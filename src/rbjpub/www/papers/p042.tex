% $Id: p042.tex $fi
% bibref{rbjp042} pdfname{p042}
\documentclass[10pt,titlepage]{book}
\usepackage{makeidx}
\newcommand{\ignore}[1]{}
\usepackage{graphicx}
\usepackage[unicode]{hyperref}
\pagestyle{headings}
\usepackage[paperwidth=5.25in,paperheight=8in,hmargin={0.75in,0.5in},vmargin={0.5in,0.5in},includehead,includefoot]{geometry}
\hypersetup{pdfauthor={Roger Bishop Jones}}
\hypersetup{pdftitle={A Small Assay at A Big Picture}}
\hypersetup{colorlinks=true, urlcolor=red, citecolor=blue, filecolor=blue, linkcolor=blue}
%\usepackage{html}
\usepackage{paralist}
\usepackage{relsize}
\usepackage{verbatim}
\usepackage{enumerate}
\usepackage{longtable}
\usepackage{url}
\newcommand{\hreg}[2]{\href{#1}{#2}\footnote{\url{#1}}}
\makeindex

\title{\LARGE\bf A Small Assay at A Big Picture}
\author{Roger~Bishop~Jones}
\date{\small 2021/06/05}


\begin{document}
\frontmatter

%\begin{abstract}
% Philosophy should be concerned, among other things, with the big picture.
% I am, but rarely allow myself to indulge that interest.
% Of late I have been trying to understand a number of philosophical problems through the lense of evolution, and this has encouraged me to think about where it is all going.
% This essay tries to articulate some of the ideas.
%\end{abstract}
                               
\begin{titlepage}
\maketitle

%\vfill

\begin{centering}

{\footnotesize
\copyright\ Roger~Bishop~Jones;
}%footnotesize

\end{centering}

\end{titlepage}

\setcounter{tocdepth}{2}

{\parskip-0pt\tableofcontents}

\mainmatter

%\listoffigures

\pagebreak

\section*{Preface}

\addcontentsline{toc}{section}{Preface}

\footnote{There may be ``hyperlinks'' in the PDF version of this monograph which either link to another point in the document  (if coloured blue) or to an internet resource  (if coloured red) giving direct access to the materials referred to (e.g. a Youtube video) if the document is read using some internet connected device.
Important links also appear explicitly in the bibiography.}

\part{Purposes}

\chapter{Introduction}

What is philosophy?
I ask this question of myself often, without pressing hard for a definitive or detailed response.
My laxity may be explained in part by the feeling that I have no right (or inclination) to prescribe for others how they use that word.

One thing that falls within the scope of philosophy, as I conceive it, is: ``the Big Picture'', whatever that may be.
And for me, there is a weakness in any philosophy which is not conducted in the context of such a story.
It is possible to work in the context of some overarching structure without feeling the need to articulate that structure, or even without being aware of the presumptions and predelictions of which it may be composed.
But in that case, dialogue with anyone who does not share that context may be compromised.

\paragraph{purpose}

For me, such a big picture involves purpose.
The big picture is an artefact, the construction of which cannot begin without some sense of purpose.
That purpose is something I chose for myself.
Not in complete disregard for what I imagine others might accept as an overarching purpose, but nevertheless, taking into account my social context, a personal choice.

The simplest example of this kind of purpose with which I am familiar is the utilitarian principle, that we should maximise the sum total of human happiness, so I shall articulate my own principle by contrast with the utilitarian ethic.

\section{Creativity?}

My thinking about evolution and philosophy began by extension to my historical explorations of the history of key concepts related to rationality.
The dismissal by Quine in the mid $20^{th}$ Century (and the acceptance of that dismissal by received philosophical opinion) of the analytic/synthetic distinction, struck me as a particularly stark example of the irrationality of a discipline (``analytic philosophy'') which conceived of itself as the high priestood of rational thought.

\paragraph{origins of philosophical rationality}~\\

This historical trail leads quickly to ancient Greece, to the axiomatic method which proved so effective in Greek mathematics and the attempts to apply similar methods beyond the bounds of mathematics which proved more mixed.
Though this provides the earliest examples of deductive inference and its systematic application in mathematics and philosophy, it is not immediately clear where it came from.
Was it just a bright idea of Thales who pulled the methods out ot thin air (or by some secular analogue of the divine revelation to which some spiritual and moral beliefs are credited)?

What little we know of the origins of mathematics as a theoretical deductive method comes to us because it coincided more or less with the availability of writing materials.
This made it possible for the results to be compiled into a succession of works, often called `elements' of geometry, culminating in the Elements of Euclid which has survived to this day.
Much of the story of these early intellectual endeavours is found in the writings of Aristotle.

\paragraph{origins of deductive reasoning}~\\

In reflecting on the nature of deduction and whether it too could have originated at this time, my inclinations proved sceptical.
Elementary deductions are often straightforward consequences of the meaning of words.
If a farmer has both goats and sheep, then everything he says about his livestock applies to both.
If he tells his son to bring the livestock in the barn then he expects that both the sheep and the goats will be brought into that shelter, and the realisation of that expectation is closely connected with the fact that the livestock being in the barn entails the sheep being in the barn and the goats being in the barn.
That his son does the right thing suggests that he can make that inference.

The evolution of language was a big (and protracted) evolutionary event, involving significant development to the human brain.
Our languages are not simple, the typical vocabulary is some 40,000 words.
The later stages of this development probably included the 600 thousand years leading to the appearance of anatomically modern homo sapiens, a period of rapid growth in the size of the brain after which brain size stabilise.
The complexity of a language is arguably orders of magnitude greater than the complexity of elementary logical deduction, the validity of which depends primarily on a knowledge of the semantic relationships between concepts, and may therefore be suspected of arising as part and prcel of the evolution of language, which in turn woud not itself have been an isolated development, but rather an aspect of the furtherance of those activities for which language provided evolutionary advantage.
Without that semantic knowledge, one cannot be said to have learned the language, or to be capable of its proper use.
With it, many of the semantic relationships which underpin correct deductive inference are transparent.

By means of this kind of informal reasoning, I have been persuaded that the practice of deductive inference (rather than any metatheory which may later have been developed) dates back to the origins of language (which in coincide with the evolution of modern humans, and to the beginning of verbal culture and its evolution.

This speculation for its credibility on greater detail in those final stages of evolution of the human brain and its capacity for language, the relevant period being the 800 thousand years before the appearance of anatomically modern homo sapiens, about 200 thousand years ago.

\paragraph{origins of irrationality}~\\

My looking back into the origins of rationality was provoked to a large extent by my perception that it isn't quite as widely exercised as may often be believed.
Its not just those subtle cases where I think there is a lapse that we need rest upon.
It is a commonplace that people, especially \emph{en masse}, are sometimes far from rational, and the phrase ``Madness of Crowds'' used by in his recent book \cite{}

%\part{Where From}

%\part{Where are We}

%\part{Where To}

\backmatter

%\cite{murray2019evolutionary}

\phantomsection
\addcontentsline{toc}{section}{Bibliography}
\bibliographystyle{rbjfmu}
\bibliography{rbj}

%\addcontentsline{toc}{section}{Index}\label{index}
%{\twocolumn[]
%{\small\printindex}}

%\vfill

%\tiny{
%Started 2021/05/21


%\href{http://www.rbjones.com/rbjpub/www/papers/p032.pdf}{http://www.rbjones.com/rbjpub/www/papers/p042.pdf}

%}%tiny

\end{document}

% LocalWords:
