% $Id: p042.tex $fi
% bibref{rbjp042} pdfname{p042}
\documentclass[10pt,titlepage]{book}
\usepackage{makeidx}
\newcommand{\ignore}[1]{}
\usepackage{graphicx}
\usepackage[unicode]{hyperref}
\pagestyle{headings}
\usepackage[paperwidth=5.25in,paperheight=8in,hmargin={0.75in,0.5in},vmargin={0.5in,0.5in},includehead,includefoot]{geometry}
\hypersetup{pdfauthor={Roger Bishop Jones}}
\hypersetup{pdftitle={Epistemic Futures}}
\hypersetup{colorlinks=true, urlcolor=red, citecolor=blue, filecolor=blue, linkcolor=blue}
%\usepackage{html}
\usepackage{paralist}
\usepackage{relsize}
\usepackage{verbatim}
\usepackage{enumerate}
\usepackage{longtable}
\usepackage{url}
\newcommand{\hreg}[2]{\href{#1}{#2}\footnote{\url{#1}}}
\makeindex

\title{\LARGE\bf Epistemic Futures}
\author{Roger~Bishop~Jones}
\date{\small 2021/06/05}

\begin{document}
\frontmatter

%\begin{abstract}
% Philosophy should be concerned, among other things, with the big picture.
% Of late I have been trying to understand everything, philosophically, through the lense of evolution.
% This essay spans almost the whole of evolution (from a particular vantage point in space time), from the beginnings of life on earth to its proliferation across the galaxy beyond its likely local demise.
%\end{abstract}
                               
\begin{titlepage}
\maketitle

\ignore{
\vfill

\begin{centering}

{\footnotesize
\copyright\ Roger~Bishop~Jones;
}%footnotesize

\end{centering}
}%ignore

\end{titlepage}

\setcounter{tocdepth}{2}

{\parskip-0pt\tableofcontents}

\mainmatter

%\listoffigures

\pagebreak

\section*{Preface}

\addcontentsline{toc}{section}{Preface}

This work is not intended to be autobiographical.
It may nevertheless be helpful to the reader to know something about my own person and life while considering whether to read the thoughts herein, if only in saving him from the attempt.
The preface is dedicated to that purpose, the body of the work excluded from it.

I first note my lack of credentials.
In formal education I progressed only so far as a BA in Mathematics and Philosophy, from a University not pre-eminent in either of those subjects.
Professionally I have been engaged primarily in software engineering, with minor excursions toward digital hardware development.

Though later I came to think of philosophy as an unrealisable dystopian vocation, philosophy barely entered into my life until my early twenties, the BA completed only by the age of 28.
Before that, my only recollection of philosophical thinking was of my reflections on sermons endured during my first year at grammar school.
Those reflections made me an atheist at the age of 12, with no further interest in debating the existence of god.

In school my poor memory ensured mediocre performance until I was able to specialise in the sixth form almost exclusively on mathematics and physics.
Accumulating facts I disdained.
Any temptation to attribute poor memory to disinterest or indolence would later be dispelled by acquaintance with effortless recall in others beyond my wildest aspiration.

That shortfall of memory may be the most important of the reasons why I could never have been a scholar.
Others include my antipathy to continuing in the study of an author in whose thought I perceived fatal errors, and the ease and frequency with which I arrive at that diagnosis.
I also read very slowly, and absorb only those parts which can be integrated into my world view (though not necessarily as true, and not necessarily without transformation).
The slowness of my read arises in part from the mental digressions, in default of which I consider a work uninteresting.
That critical dismemberment demands less talent than creative synthesis is a truism amply illustrated by academic philosophy.

\footnote{There may be ``hyperlinks'' in the PDF version of this monograph which either link to another point in the document  (if coloured blue) or to an internet resource  (if coloured red) giving direct access to the materials referred to (e.g. a Youtube video) if the document is read using some internet connected device.
  Important links also appear explicitly in the bibiography.}


\section{My Scepticism}

I think myself a sceptic.
Not a radical sceptic holding that no knowledge is possible, but one sceptical about much that is commonly considered authoritative.
Nevertheless, an advocate for rationality, taking scepticism to be an integral part of rationality.
There is no simple story about how my scepticism compares with, say, a typical academic philosopher.
Sure enough I will doubt much that is held by others, but also accept a great deal which is rejected.

I have been shocked by the irrationality of institutions which might have been expected to exemplify rationality, most notably by philosophy in what was for some time called the ``analytic'' tradition.
This has caused me puzzlement.
Certain other philosophical time-lines, notably Marxism reaching ultimately into activist tropes on social media and our streets through the Critical Theory of the Frankfurt school, augmented (or empovershed) by Post-modern scepticism, seem to have abandoned and abjured rationality, and stand testimonial to the power of contrary tendencies.
The apparent attempt to put ideology beyond reach of reasonable discussion is also puzzling (thugh not of course without ample precedent), and concerning.

My response to the puzzlement and concern has been to try and understand, to dig and probe in search of comprehension, in the hope that a contribution to understanding might be possible and helpful.
Why is it that rationality, to which surely is due credit for the mastery of man over his environment and the resulting material prosperity, seems so scarce, and becomes a target for extirpation?

Since my work here depends upon my scepticism, it may be helpful to leverage those doubts in making intelligible my approach and methods, and thence their cautious conclusions.

Though my scepticism has been curated over a lifetime, there have been three particular episodes which have seemed to me of importance and which have influenced the direction of my enquiiry.

These three episodes are, in turn, my deliberations about the incoherence of God, about the Quinean refutation of Logical Positivism and its influence, and my belated acquaintance with the trajectory of Marxist and post-Marxist thought and its 21st Century conscription of the concept of Justice via a complete shredding of any recognisable standard of rationality.

I propose to begin here with a brief account of the principle stages in my disillutionment, and the directions in which they have influenced my enquiries.

The development of my own sceptical thought is more complex and more continuous than the story I will tell here, but there have been four particular episodes which are significant to the development of the main ideas presented in this work, which have given direction to the ``research'' invloved.

The first of these concerns God, whose existence beame problematic for me in my first year at Grammar School.
I went as a boarder, and was obliged to attend a church service every Sunday, at which I therefore was subjected to sermons.
Naturally I tried to understand them, and, not distracted by lesser concenns I tried to understand what God could be.
Evidently not content to be told that he was mysterious and beyond our comprehension.
The difficulty was not so much that of fully comprehending God, as in understanding how anything whatever might have the attributes ascribed to him.
I don;t recall how far I got through the school year before I gave up on this enterprise, but I do remember the end-game.

At the end, the last puzzlement, and perhaps the greatest one, which engaged me before concluding that God did not exist, was that of comprehending how so many important, respected and distinguished men (and women) could believe in the existence of God, if, as I then strongly suspected, no such being existed.
The first memorable stage in the development of my scepticism was the realisation that I could not believe everything I was told even by those who might have been supposed most authoritative.

From that point on, religion, and most especially denate about the existence of God, had no interest for me.
Though I had then no knowledge of logic, my belief then was as close as it could have been to my present view that the concept of God is incoherent, and that no argument for God would be worth considering unless is rested on a definition which fell short of that conception of deity upon which the Christian church is founded.

There followed from this brief episode of spontaneous scepticism little of present interest for many years.
I was an indifferent pupil in most subjects, mostly due dysfunctional memory and minimal effort.
Only toward the end of my fourth year (after a change of teacher) did I rise near the top of the maths class, eventually spending my two years in the sixth form doing double-maths physics to the exclusion of any trace of the humanities.
Before I came to my next sceptical episode, which concerns academic philosophy, I must first have become acquainted with philosophy, of which I was as I left school quite innocent.
My curtailed spell at University undertaking the first year of the Mechanical Sciences tripod, did nothing to improve that situation, unless you count some deliberation on artificial intelligence engendered by my unrestricted access to an IBM 1130 during that year.
The next five years began my career in Software Engineering, primarily involving the implementation of software support for high level programming languages, and it was in my own time that I then broadened my horizons and read a little more widely.

The principle influences on my approach to the Western tradition in philosophy were Bertrand Russell and Logical Positivism, the latter refined later to focus on Rudolf Carnap.
I read Russell's autobiography and his ``History of Western Philosophy'', among the very few books I have ever read more than once.
For a youngster well soaked in mathematics, fascinated by electronic computers and slowly learning about formal logic, the achievement of ``Principia Mathematica'' cast a spell, as it had for many of Russell's most gifted contemporaries and many others since.
Before I got near undergraduate philosophy ``Principia Mathematica to *56'' \cite{russell1970}, the paperback edition of the first parts of the three volume megalith, had secured its place on my shelf, conferring upon me only the barest comprehension of what lay between its covers.
Beyond the formal mathematics, into the philosophy, A.J.Ayer's ``Language Truth and Logic'' \cite{ayer1936} was for me an intelligible account of a palatable attitude to philosophical obscurity which played a large part in shaping the perspective from which I observed my undergraduate instruction in philosophy.

Though I was then barely aware of the source of my own perspective, the fundamental (if disputed) distinctions between truths of logic,  empirical facts and value judgements were, and have remained for me, a bedrock on which all else rests.
Thus was prepared the ground for my scepticism of academic philosophy.
One famous paper was the seed from which my utter disillusion with even analytic philosophy as a rational pursuit ultimately sprang.
That paper was ``Two Dogmas of Empiricism'' by W.V.O. Quine \cite{quine51}, a paper published mid 20th century with the aim and effect of defeating ``Logical Positivism'' as school, and sidelining its most distinguished progenitor Rudolf Carnap.

That I perceived the paper as without merit (for reasons I will pass over as yet) on first acquaintance was not the source of my dismay, for I was not then aware of the impact that the paper had exerted, not only carrying the day against its targets but largely inhibiting further consideration of its central thesis for perhaps half century.
Whether I was correct in my assessment of the paper, my perception of its effects, which supposedly rendered meaningless the most fundamental distinction in the foundations of rational thought.
That distinction, between logical and empirical truth, had evolved through a convoluted 2000 years of Western philosophy and had recently been brought almost to perfection by substantial fundamental advances in the foundations of logic.

In considering a response to this major breach odf rationality among those of whom it might have least been expected (analytic philosophers), it seemed to me clear that a rational response, a logical dissection of the defects of the arguments, could not succeed, and I instead became interested in underpinning the decried distinctions by studying their history over the past 2000 years and perhaps showing how the understanding which could be articulated on the basis of the most modern advances in logic and its philosophy was a worthy culmination of those thousand years of technical and philosophical travail.

That thread of interest persisted for many years without significant issue, and was accompanied by a disdain for contemporary analytic philosophy which was just as complete as my failure to engage with that other kind of philosophy which was once called ``continental philosophy''.
It was only very recently I became aware of the enormous practical effects that some aspects of ``continental philosophy'' were effecting.
The scepticism injected into the Marxist derived ``Critical Theory'' by postmodern philosophy undermined any propspect of rational discourse on matters of great public interest, and seemed to call for a more fundamental defence than could be afforded by a study of the history of philosophy.


\chapter{Introduction}

This is a story about \emph{knowledge}, how it has evolved, how it is, and how it may develop.

In telling that story I connect a broad conception of knowledge and its variety with a number of other related concepts.
The first of these concepts is that of \emph{evolution}, understood as a progressive aggregation of knowledge -  chemical, biological, cultural and technnological.
A second is the notion of \emph{rationality}, closely related both to knowledge and to evolution.
Evolution in all its forms is viewed as aggregating knowledge, and in that way exhibiting both instrumental and epistemic rationality.
As evolution evolves so does the rationality which it exemplifies.

The aggregation of knowledge has predominantly been a social activity, and as such depends upon means of communication, which have themselves evolved continuously.
The evolution of language leads to new forms of knowledge and to oral culture.
The development of written forms better preserves knowledge across generations and spreads it far and wide.
The effectiveness of social groups depends upon an appropriate level of cooperation (while also benefiting from competition) and mechanisms of social cohesion may lead to the apparently irrational behaviours sometimes deprecated as tribal or cultish.

\section{The structure}

This work is divided into three parts correponding to historical stages in the evolution of homo-sapiens.
The first considers the period up the apperance of anatomically modern homo sapiens, encompassing a period of pre-biotic chemical evolution, a period of biological evolution effected by undirected variation and `natural' selection, and a period of evolution in which sexual selection became an important factor in determining the direction of evolution.

Anatomically modern homo sapiens appears around 200,000 years ago, and at roughly this time human development due to biological evolution, though continuing, is overtaken by and transformed by the more rapid changes arising from cultural evolution.
This period of cultural development and culturally influenced biological evolution is the subject of the second part, which brings the narrative up to the present day.

The science and technology which eventally flourished as a part of that cultural evolution has now brought us to a point at which we may expect to see the largest ever transformation to the process of evolution, as synthetic biology and informaton technology together permit us to take into our hands the design of future generations of the ecosphere, including the evolution of homo and the engineering of `intelligent' machinery and artificial life forms.
The core imperative of evolution is proliferation, and the third part of this essay concerns what and how we may now begin to proliferate across the cosmos.

The whole is presented as an evolutionary story in which there feature many different kinds of evolution.
As well as distinguishing between the various kinds of evolution involved, I offer an epistemic characterisation of evolution as a whole, in which the progress which we may hope to see from evolution lies in the progressive accumulation of knowledge.

\section{Rationality as Focal Point}

I'd like to sketch out here a contemporary issue around which this whole work revolves.
This concerns the contrast between human rationality and its apparent converse exhibited in some radical ideologies.

The seminal notion of rationality here is that which has been called \emph{instrumental} rationality, characterised by the adoption of means to ends which are most likely to realise those ends.
\emph{Epistemic} rationality, by contrast, concerns the relationship between evidence and belief, and is exhibited by those whose beliefs are those consistent with the evidence on which they are based.
Epistemic rationalty may be seen as derivable from ir entailed by instrumental rationality since true belief is generally instrumental and false belief counterproductive.

It is tempting to associate the social activism which has become increasingly forceful in these earty decades of the 21st Century with \emph{irrationality} for a number of reasons.
The first is the explicit rejection of rationality, as a weapon of opression by Western colonialists, supported by the scepticism and relativism which was brought into Critical Theory under the influence of `Postmodern' thinkers.
There are many other aspects of applied critical theory however, which seem to fly in the face of previously accepted rational standards.

Against this we may note the difficulty in making judgements about irrationality where the motives of the agents are not clearly understood, perhaps not even by them.
We must surely ask and hope to understand \emph{why} anyone should reject rationality.
If the motive is understood, the rejection of rationality (possibly only in transition) might then appear rational.

Steven Pinker has argued that arguments against rationality are self-defeating, for the presentation of arguments, he supposes, concedes the relevance of arguments and thus of rationality.
He neglects the rational, indeed deductive, method of \emph{reductio absurdum}, in which one begins a proof by assuming the negation of the proposition to be proven, and also the fallacy of \emph{petitio principii} which is involved in any rational attempt to establish a principle of rationality.

Our context lacks the prerequisites for demonstrative reasoning, we must work with less formality, and less assurance.
A defence of our tradition, even qualified, against fundamental ideological challenges, must not only reinforce our reasons for adopting the methods we use, and our most fundamental system of beliefs, but also must seek to understand and counter-challenge the motives which seeded the conflict and which facilitated its proliferation.

\section{An Evolutionary Preview}

Life on earth has evolved.

There have been many changes to the mechanisms involved, but three major transitions which deserve special attention, and one which is now in progress.

Those transitions were:

\begin{description}
\item[life:] The transition from chemistry to biology (4Ba).
\item[sexual:] The transition to sexual reproduction/selection (2Ba).
\item[cultural:] The beginning of culture and its evolution (200ka).
\item[technological:] Will synthetic biology kill natural selection?
\end{description}

In this preliminary discussion I will talk about why these transitions are of importance for the evolutionary process, and why an understanding of the transitions and of the different kinds of evolution they introduce may be worthwhile.

I look for the development of rationality and of those mechanisms of social behaviour which have the power to suspend rationality and secure behaviours which may have no survival or reproductive advantage beyond mere conformance with a social norm which itself has no merit.
These latter I will talk of under the term ``social behaviours'', taking in the first instance a very broad view of social behaviours as pre-cursors to the present day phenomena of interest.

In considering rationality and the relevant social behaviours I will take them as occurring at many levels.

Insofar as the rationality of concern in the first interest is \emph{instrumental}, and taking the purpose at hand to be determined in detail by context but in general as the aim to construct organisms which are successful in self-replication (in a particular ecological and social context), I suggest that evolution itself is \emph{rational}, it realises that purpose.
This is a bit like thinking of evolution as a ``blind watchmaker'', as achieving needed effects which would otherwise require intelligent design.
There are many different kinds of evolution which we will consider, and they do differ in the credibility of such alleged rationality.

Secondly, we may consider that the \emph{results} of the evolutionary process are predominantly rational, i.e. that they are effective in enabling organisms to replicate in some suitable niche.
This attribution is applicable in the first instance and in the most primitive organisms to capabilities and behaviours which are rigidly programmed by the genes of the organism, and thus does not involve anything which we might regard as rational deliberation.

A first step toward such deliberation is the evolution of behaviour which is more flexible, and allows the organism to succeed by adopting varying ways of realising some important end according to circumstances, or which allow the organism to proliferate in a wider range of environments.

\subsection{The Genesis of Life}

Direct evidence of life on earth dates back about 3.5 billion years.
Life appeared, it is generally supposed, as a result of a period of ``chemical evolution'' the nature and course of which is not well understood.



Before life evolved evolution was chemical, resulting in the construction of ever more complex molecules and chemical environments gradually more suitable for the support of biological organisms and the evolution of species.

All life on earth shares the characteristic that it consists of organisms which under certain circumstances, in a certain kind of environmental niche, are capable of proliferation, of self-reproduction.
It also shares more specific characteristics which may not be essential to that, such as the use of DNA to mediate in the reproductive process and to practically codify the structures and processes necessary to the life and proliferation of the organism.

The encoding of the structure of the organism in DNA has a profound effect on the process of evolution, and represents the transition of particular interest here, though it may not precisely align with that between inanimate and organic structures, which will depend on a precise definition of `life' which we have not and will not venture here.

It is at this point that evolution can be thought of as evolution of species by `natural selection' as described by Darwin, in which natural conditions select those organisms which survive to reproduce, and thereby gradually evolve the genetic and phenotypical characteristics of population.

\subsection{The Merits of Sexual Reproduction}

Important milestones in evolution are often the combined effect of multiple advancements.
Sexual reproduction may be seen in that way.

The earliest known life forms on earth were \emph{prokaryotes}.
Prokaryotes are single cells which do not have a cell nucleus, and reproduce asexually by cell division, a process which usually forns two cells genetically identical to the original.
Without other genetic innovation, the variation on which evolution depends would occur during the copying of the original genome, in default of which the progeny would be identical to the parent.

The evolution of features which required multiple genetic changes would only be possible if all those changes occurred on one line of descent, and they would only become

\subsection{Cultural Evolution and Selection}

\subsection{Synthetic Evolution}
  
\section{Concepts and Vocabulary}

Evolution is the unifying concept under which I discuss the development of certain phenomena of interest over extended periods of time.
Tracing back through the history may help us to understand the phenomena as they appear today and thence anticipate and accomodate their development in the future.

Some of these phenomena, for example language, culture and rationality may be thought exclusively human.
It may nevertheless be helpful to consider from what prior capabilities those human facilities evolved, and how that could have happened.
In doing this, terminology is desirable which reflects the connection with the fully fledged facility as seen in man while maintaining the distinction.

Often qualification will be a good way to do this.
Thus we may speak of the kind of culture whose inception occurs at about the same time as oral language as ``oral culture'' and speak also of the tool making skills passed from one generation of homo erectus to the next as part of a ``pre-lingual culture'', thus facilitating clarity by terminological fiat while side-stepping debates about the precise boundaries of established concepts.

Sometimes the important predecessors are not within reach by that method.
The concept of ``language'' exemplifies the problem.
There is, before any special terminology is attempted, a variation in usage, between professional academics and others and between academics in different disciplines.
Linguists may insist that languages have recursive generative grammers allowing infinitely many sentences of unbounded length and complexity, but others will use the word more liberally.
Nevertheless, when we trace back 

In viewing the whole evolutionary process as a progressive accumulation of knowledge, a great variety of kinds of knowledge are encompassed.
We may begin with the idea that an autocatalytic set embodies knowledge of how certain chemicals can be synthesised, and followed by the encoding in various forms (RNA, DNA) of knowledge about how to build particular proteins and their relevance to the organisms in which the codings are found.

Rationality, thought by Aristotle to be peculiarly human, may similarly be traced back to the origins of life and beyond by analogies based first on the instrumental effectiveness of evolution in realising organisms well adapted to proliferation in their own ecological niche.


In considering these pre-human characteristics it is somtimes natural to extend the scope of the existing concept.
Though some linguists will insist that only humans have language, it is not uncommon to hear the term extended more broadly.
The important distinctions which remain between language in humans and the ``language'' of birds or dolphins, can be preserved and made precise by appropriate qualifications.
If an exclusive feature of human languages is their recursive generative grammars, then perhaps we could call that type of language a recursive language?

Consider the concept of \emph{rationality}.
In its normal use this is thought to be an exclusively human characteristic.

This work flows from a concern about rationality and its contrary, their contributions to our present predicament and prospects.
Intimately bound up in this concern is that for knowledge, the theory of which, epistemology, yields the title of the work.

The method is historical, an exploration of the history leading to the present and the future which may lie beyond.
In looking back for an understanding of the past which will help us understand the present and shape the future, precursors may be instructive.
To understand rationality, it is useful to look back to those related phenomenon which precede rationality.
In distinguishing rationality from its precursors we are limited by the imprecision of our language, in which usage is diverse and boundaries indeterminate.
The approach adopted here is to qualify the concepts, giving us a label for the differning manifestations at each stage in the development.
Which of these stages is to be considered a precursor, and which a fully fledged variant, may then be academic.

Some illustrations may be helpful.
Let us consider language.
Some linguists insist that a bona fide \emph{language} must have a recursive grammar.
A language is a way of transmitting information.
To understand the origin of languages we may consider the function of communication as fundamental to language and consider what means of communication preceded languages.
Another expectation of languages is that they are symbolisms, 


\subsection{Knowledge}


\subsection{Evolution}

Evolution is a change, not necessarily without interruption.







\part{The Evolution of Homo Sapiens}

\chapter{Introduction}



\part{The Evolution of Culture}

\chapter{The Beginning of Culture}

When culture begins depends upon what we consider culture to be, how we \emph{define} the term.
The broader the definition, the more it encompasses and the sooner we may consider it begun.
In this conmtext culture should be understood as knowledge transmitted from one generation to the next \emph{non-genetically}.
Other candidate criteria include:

\begin{enumerate}
  \item transmission of adaptations
  \item and also shared within a generation, not just passed from parent to child.
  \item communicated by language
\end{enumerate}



\chapter{Where are We?}

Before considering where evolution will take us in the future, a few words about where we now are may provide some basis for projection into the future.

\section{The State of Evolution}

Evolution itself is at a point of inflection.
Biological evolution, the evolution of the human genome and the ecosphere we share, may be about to experience its most profound transformation.

Arguably, the dominant form of human evolution is now by \emph{cultural} selection.
To an increasing extent the evolution of the rest of the exosphere is influenced, for better or worse, but human culture, the technology we have developed and its intended and accidental effects.

To an unprecedented extent technology and the welfare state have marginalised the significance of genetically determined phenotypic factors which might limit human proliferation.
Though it is at this moment of marginal significance, the scope of cultural selection can now be extended by the techniques of synthetic biology.
These will allow cultural selection to take effect by specific genetic interventions rather than purely by selection of mating partners.
These interventions at present can be undertaken by sequencing of embryos and selection among them on the basis of detailed knowledge of the genomes, and will likely eventually to embrace editing of the genome, or even synthesis of a genome, techniques colloquially said to yield ``designer babies''.

These methods, to the extent to which they may be permitted,  belong to the future.
The novelties which characterise the present state of evolution are ones which arise from cultural evolution, and from the technical advances which have accelerated that process.



\part{Synthetic Evolution}

\chapter{Beyond the Paradox}

In ``The Open Society and its Enemies'' Karl Popper mentions in a footnote a ``Paradox of Tolerance'' to the effect that a completely tolerant society would be vulnerable to subversion by intolerant ideas, and society should therefore limit its tolerance of ideas to those which are not inconsistent with the continuation of a tolerant society.

That whole work of Popper was devoted to exposing the work of those philosophers whose ideas Popper considred the greatest threats to an ``open society', viz. Plato, Hegel and Marx.

As we have seen, the ideas which Popper implicates are alive and well, having evolved into forms which are much more persuasive and prolific than the originals.
Popper prescribed, however briefly, limits to tolerance, for the purpose of preventing the subversion of liberal democracies by totalitarian regimes, in the context of experience indicating that attempt to implement utopian ideas have resulted in totalitarian dystopias.
Other writers in the mid twentieth century were also inspired by similar motives to expose the workings of such regimes.
Perhaps the best known of these was George Orwell, who approached the expose through fiction in his ``Anmial Farm'' and ``1984'' \cite{orwell-af,orwell-1984,orwell-fd}.
A more scholarly approach was taken by Isaiah Berlin, who might perhaps have been an academic philosopher but for the negative effects of analytic positivism on the standing of political philosophy in mid 20th Century Oxford.
Instead he donned the mantle of historian of ideas and in that way contributed to our understanding.
 
\backmatter

%\cite{murray2019evolutionary}

\phantomsection
\addcontentsline{toc}{section}{Bibliography}
\bibliographystyle{rbjfmu}
\bibliography{rbj}

%\addcontentsline{toc}{section}{Index}\label{index}
%{\twocolumn[]
%{\small\printindex}}

%\vfill

%\tiny{
%Started 2021/05/21


%\href{http://www.rbjones.com/rbjpub/www/papers/p032.pdf}{http://www.rbjones.com/rbjpub/www/papers/p042.pdf}

%}%tiny

\end{document}

% LocalWords:
