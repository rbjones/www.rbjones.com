% $Id: p038.tex $
% bibref{rbjp038} pdfname{p038}
\documentclass[10pt,titlepage]{article}
\usepackage{makeidx}
\newcommand{\ignore}[1]{}
\usepackage{graphicx}
\usepackage[unicode]{hyperref}
\pagestyle{plain}
\usepackage[paperwidth=5.25in,paperheight=8in,hmargin={0.75in,0.5in},vmargin={0.5in,0.5in},includehead,includefoot]{geometry}
\hypersetup{pdfauthor={Roger Bishop Jones}}
\hypersetup{pdftitle={Oracular AI v2}}
\hypersetup{colorlinks=true, urlcolor=red, citecolor=blue, filecolor=blue, linkcolor=blue}
%\usepackage{html}
\usepackage{paralist}
\usepackage{relsize}
\usepackage{verbatim}
\usepackage{enumerate}
\usepackage{longtable}
\usepackage{url}
\newcommand{\hreg}[2]{\href{#1}{#2}\footnote{\url{#1}}}
\makeindex

\title{\bf\LARGE Oracular AI v2}
\author{Roger~Bishop~Jones}
\date{\small 20212-08-19}


\begin{document}
%\frontmatter

%\begin{abstract}
% A second start on this topic.
%
%\end{abstract}
                               
\begin{titlepage}
\maketitle

%\vfill

%\begin{centering}

%{\footnotesize
%copyright\ Roger~Bishop~Jones;
%}%footnotesize

%\end{centering}

\end{titlepage}

\ \

\ignore{
\begin{centering}
{}
\end{centering}
}%ignore

\setcounter{tocdepth}{2}
{\parskip-0pt\tableofcontents}

%\listoffigures

%\mainmatter

\section*{Preface}

It was in 1986 that I first had hands on an ``Interactive Theorem Prover'', the Cambridge HOL system,
engineered to support formal reasoning about digital electronics.
I had just began working on the applications of formal methods to the development of security and safety critical information systems.

One of my early reactions on acquaintance with this system was to perceive the broadness of its potential applications.
The theory hierarchy which HOL supported, enabling the structured development of abstract models and of  the underlying mathematics required to build such modesl was in effect a general purpose \emph{knowledge base} of a kind relevant to GOFAI \footnote{Good Old Fashioned AI} with some extra features not often mentioned in AI.

The ability of the system to ensure logical consistency, surely essential for large scale deductive reasoning to have any value, together with a strong cultural preference for staying within those bounds, and the general perception that applications in engineering require nothing more, provided a seed in my mind which I have nurtured (perhaps intermittently) ever since, without as yet succeeding in articulating its relevance.

Some years later, when my own practical involvement in the application of interactive theorem provers was drawing to a close, that seed found a new place in my more philosophical aspirations.
I have struggled ever since to articulate the ideas which grew from that seed, and in this essay I try once more, focusing as tightly as I can on the core insights.

These swim against the dominant tendency which is evident in AI of emulating the capabilities of human beings, hoping that once reached they will be readily surpassed.
They do so under the inspiration of ideas about how reasoning and the knowledge which can be derived from it can be made both more reliable and more broadly applicable.
The grounds for the belief that this is possible do not come from observation of the ways in which human beings commonly reason, but rather from advances in mathematics, theoretical computer science and philosophy which have transformed our understanding of deductive reasoning and its limits within the last couple of centuries (after a couple of millenia in which progress faltered).

Though these new logical methods were theoretically and philosophically fruitful, and hinged around the invention of potent strictly formal languages and deductive systems, the application of these formal systems (rather than their use in the development of new theoretical disciplines) was severely limited by the complexity of the detailed proofs which they required.
The use of these systems in real world applications, or even in the development of mathematical theories, was impractical until the advent of the digital stored program computer.
Even with this assistance, in the form of brute computational capability, the support remained short of what was needed to realise the full potential.

This became apparent as GOFAI came up against the problem of `combinatorial explosion' and came to understand that intelligent heuristics were essential to success in finding deductive proofs of non-trivial propositions.

\section{Introduction}

When stored program digital computers were first invented their applications primarily concerned doing large amounts of information processing or computation with almost perfect reliability and at superhuman speeds.
They were accurate and reliable.

As their computational power grew their applications were extended progressively, and this sometimes involved attempts to achieve ends which were much less clearly defined, involving more complex instructions which could less certainly be relied upon to achieve the intended purpose.

The kinds of brute computational power exhibited by these early computers might at first have been thought signs of intelligence, since human skill in computation had certainly been presumed a sign of intelligence.
But brute computational power soon came to be distinguished from intelligence.

As I write, generative AI and Large Language Models have momentarily set a new standard for the unreliability of Artificial Intelligence.
Not designed or trained to be reliable repositories if knowledge, or to be capable of any but the most eleentary reasoning, their exposure to vast quantities of human knowledge enables them to perform in many main-line subject matters in an apparently authoritative way, while morphing in more esoteric areas into fantasy, and failing under the most modest interrogation to demonstrate any but the most shallow comprehension in subjects whose generally accepted facts they can replay.

It may not be so hard to improve on this.
LLMs have proven capable of using tools effectively, and tools such as more reliable ways of saving accurately and reliably knowledge acquired, or reasoning about that knowledge may not be difficult to supply.
The discussion in this essay may be thought of as concerning the use by such an AI with a tool which is capable both of storing knowledge and of deductive reasoning in the context of that knowledge.
The effect alleged would be to enable Artificial Intelligence which is \emph{oracular} in relation to logical truth.

Oracles may be thought of as having great wisdom, possibly derived from divine connection, but here I use the term more narrowly.
For the purposes of this essay the term ``oracle'' is used for something which is always truthful in answering questions, but doesn't always answer.

The oracle of interest here can be asked whether a sentence in a formal language is a \emph{logical truth} a concept which I will try to characterise, but which ultimately cannot be made absolutely definite.

The term ``Logical Truth'' is philosophically controversial.
In my usage of that term I stand on a limb, for my use is very similar to that of Rudolf Carnap,and is synonymous with the term \emph{analytic}, a concept central to Quine's repudiation of the philosophy of Carnap in the mid 20th Century.

Its not my purpose here to argue about the terminology.
Some might insist that my conception of logical truth should more properly be spoken of as set theoretic truth, and I do not intend to argue against that opinion.
Carnap, who until 1952 used the terms ``logical truth'' and ``analytic truth'' synonymously %
\footnote{as is explicit in section 2 of ``Meaning and Necessity'' \cite{carnap56}}%
, eventually accepted defeat and began to use the term ``logical truth'' for a narrower concept%
\footnote{W.V. Quine's noted a defect in Carnap's definition of analyticity in \cite{carnap56}, which followed closely a defect first seen in Wittgenstein's ``Tractatus Logico-philosophicus''\cite{Wittgenstein1921}.
Carnap's response appeared in the paper ``Meaning Postulates''\cite{carnap52} in which for the first time he separates the concept of logical truth from that of analyticity.}.

The term ``Oracular AI'' as used here, refers to what AI might in principle be able to achieve if furnished with an oracle for logical truth.

One of the purposes of this essay is to discuss how thus notion of logical truth can be made precise, to consider the difficulties in implementing such a decision procedure and to talk about the value of approximations which fall short of logical omniscience.

\section{Some Historical Background}

The stark difference between the reliability of deduction in mathematics and ways of discovering truth in other domains has been plain since the ancient Greeks began the transformation of mathematics into a theoretical science.
Axiomatic geometry.
The results thus obtained, particularly in the axiomatic development of geometry, were reliable and were to be gradually accumulated and were ultimately gathered together as the Elements of Euclid.

By contrast, those same ancient Greeks, when attempting to reason about nature and the cosmos were unable to establish durable findings, and would find many ways in which deductive reasoning, via reductio, could establish absurd and contradictory conclusions.

The differential success of deductive reason in these distinct domains was to be reflected in the two worlds of Plato's philosophy, of which only that of Platonic ideals was susceptible to truw knowledge, to be reached by reason alone.
Aristotle sought to rescue the applicability of deduction to what we would now call empirical science through his conception of \emph{demonstrative science}, which relied for itscoherence on the special truth assuring characteristics of the fundamental principles of each science.

Millenia followed in which this situation remained largely stable.
The modern conception of science originating in the scientific revolution toward the end of the renaissance resulted in philosophy being split into two camps associated with a primary emphasis on reason and observation respectively as the source of knowledge.
Empiricism retained the idea of reasoning from scientific principles, but insisted on observation and empirical experiment for the discovery and verification of the principles.

\appendix

\section{Terminological Notes}

\subsection{On the notion of Logical Truth}

The concept of logical truth has a long history in which modern controversy plays a role.
It is not my aim here to argue a case for the particular usage of the concept which I have adopted for this essay, but rather to make some observations about how that usage relates to some milestones in the history of logic which seem important in the present context.
Most discussions about the concept proceed as though there is an objective truth about the meaning of the concept, and argue for a particular explanation of what that meaning is.
That is not what is going on here.
I do not claim to know what that concept really means, I aim only to explain the usage which I have adopted and mention some ways in which this usage connects with the ideas of some other philosophers.

Let me first mention the four philosophers who seem to me to have come closest to articulating the same concept, mostly in quite different terms: Plato, Hume, Frege and Carnap.


\subsubsection{Plato}

Plato lived at a time when systematic deduction had first shown its value in the development of mathematics, and had also been shown capable of proving any nonsense you like in metaphysics or cosmology.
This contrast was exhibited in the conflict between the philosophies of Parmenides (who believed that nothing changes) and Heraclitus (who saw a perpetual flux).

These two philosophies were reconciled through Plato's two worlds, that of platonic ideals, and the world of appearances.
Plato thus made the distinction between logical and empirical truth which is the basis for the conception of logical truth addressed in this essay.

\subsubsection{Empiricism}

Plato's pupil Aristotle was to make enormous contributions to logic, but possibly not material advancement of this particular distinction.
He was concerned by the difficulty of understanding in the context of Plato's philosophy how it was possible to reason about the concrete world, dismissed by Plato as the shadowy world of appearances of which true knowledge was not possible.
The characterisation of reason as effective only in the realm of ideas and forms unfortunately excluded it from relevance to that shadowy realm of impressions which could yield no true knowledge, but which vitally concerns us all.

In articulating the concept of \emph{demonstrative science} Aristotle gave a good account of how one can reason about the concrete world, not just the etherial world of ideas, but in doing so the line he drew was between what was scientifically necessary as determined by the logical consequences of fundamental scientific principles and the accidents of how things happen to be.
The line he thus drew between necessary and contingent truths, was that between \emph{physical} (rather than \emph{logical}) necessity and his concept of contingency was confined to the accidental rather than embracing scientific laws.

Aristotle's conception of logic, and his conception of necessity was to be dominant for thousands of years, and perhaps held back further refinement of Plato's distinction.

The debates which ultimately lead to its further refinement may be thought to have begun with the division of early modern philosophers into \emph{rationaists} (Descartes, Spinoza and Leibniz) and \emph{empiricists} (Bacon, Gallileo, Locke, Berkeley and Hume).

\phantomsection
\addcontentsline{toc}{section}{Bibliography}
\bibliographystyle{rbjfmu}
\bibliography{rbj2}

\addcontentsline{toc}{section}{Index}\label{index}
{\twocolumn[]
{\small\printindex}}

%\vfill

%\tiny{
%Started 2023-08-19


%\href{http://www.rbjones.com/rbjpub/www/papers/p038.pdf}{http://www.rbjones.com/rbjpub/www/papers/p038.pdf}

%}%tiny

\end{document}

% LocalWords:
