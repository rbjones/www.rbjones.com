% $Id: p038.tex $
% bibref{rbjp038} pdfname{p038}
\documentclass[10pt,titlepage]{article}
\usepackage{makeidx}
\newcommand{\ignore}[1]{}
\usepackage{graphicx}
\usepackage[unicode]{hyperref}
\pagestyle{plain}
\usepackage[paperwidth=5.25in,paperheight=8in,hmargin={0.75in,0.5in},vmargin={0.5in,0.5in},includehead,includefoot]{geometry}
\hypersetup{pdfauthor={Roger Bishop Jones}}
\hypersetup{pdftitle={Oracular AI v2}}
\hypersetup{colorlinks=true, urlcolor=red, citecolor=blue, filecolor=blue, linkcolor=blue}
%\usepackage{html}
\usepackage{paralist}
\usepackage{relsize}
\usepackage{verbatim}
\usepackage{enumerate}
\usepackage{longtable}
\usepackage{url}
\newcommand{\hreg}[2]{\href{#1}{#2}\footnote{\url{#1}}}
\makeindex

\title{\bf\LARGE Oracular AI v2}
\author{Roger~Bishop~Jones}
\date{\small 20212-08-19}


\begin{document}
%\frontmatter

%\begin{abstract}
% A second start on this topic.
%
%\end{abstract}
                               
\begin{titlepage}
\maketitle

%\vfill

%\begin{centering}

%{\footnotesize
%copyright\ Roger~Bishop~Jones;
%}%footnotesize

%\end{centering}

\end{titlepage}

\ \

\ignore{
\begin{centering}
{}
\end{centering}
}%ignore

\setcounter{tocdepth}{2}
{\parskip-0pt\tableofcontents}

%\listoffigures

%\mainmatter

\section*{Preface}

It was in 1986 that I first had hands on an ``Interactive Theorem Prover'', the Cambridge HOL system,
engineered to support formal reasoning about digital electronics.
I had just began working on the applications of formal methods to the development of security and safety critical information systems.

One of my early reactions on acquaintance with this system was to perceive the broadness of its potential applications.
The theory hierarchy which HOL supported, enabling the structured development of abstract models and of  the underlying mathematics required to build such modesl was in effect a general purpose \emph{knowledge base} of a kind relevant to GOFAI \footnote{Good Old Fashioned AI} with some extra features not often mentioned in AI.

The ability of the system to ensure logical consistency, surely essential for large scale deductive reasoning to have any value, together with a strong cultural preference for staying within those bounds, and the general perception that applications in engineering require nothing more, provided a seed in my mind which I have nurtured (perhaps intermittently) ever since, without as yet succeeding in articulating its relevance.

Some years later, when my own practical involvement in the application of interactive theorem provers was drawing to a close, that seed found a new place in my more philosophical aspirations.
I have struggled ever since to articulate the ideas which grew from that seed, and in this essay I try once more, focusing as tightly as I can on the core insights.

These swim against the dominant tendency which is evident in AI of emulating the capabilities of human beings, hoping that once reached they will be readily surpassed.
They do so under the inspiration of ideas about how reasoning and the knowledge which can be derived from it can be made both more reliable and more broadly applicable.
The grounds for the belief that this is possible do not come from observation of the ways in which human beings commonly reason, but rather from advances in mathematics, theoretical computer science and philosophy which have transformed out understanding of deductive reasoning and its limits within the last couple of centuries (after a couple of millenia in which progress faltered).

Though these new logical methods were theoretically and philosophically fruitful, and hinged around the invention of potent strictly formal languages and deductive systems, the application of these formal systems (rather than their use in the development of new theoretical disciplines) was severely limited by the complexity of the detailed proofs which they required.
The use of these systems in real world applications, or even in the development of mathematical theories, was impractical until the advent of the digital stored program computer.
Even with this assistance, in the form of brute computational capability, the support remained short of what was needed to realise the full potential.

This became apparent as GOFAI came up against the problem of `combinatorial explosion' and came to understand that intelligent heuristics were essential to success in finding deductive proofs of non-trivial propositions.

\section{Introduction}

When stored program digital computers were first invented their applications primarily concerned doing large amounts of information processing or computation with almost perfect reliability and at superhuman speeds.
They were accurate and reliable.

As their computational power grew their applications were extended progressively, and this sometimes involved attempts to achieve ends which were much less clearly defined, and involved much larger and more complex instructions which could less certainly be relied upon to achieve the intended purpose.

The kinds of brute computational power exhibited by these early computers might at first have been thought signs of intelligence, since skill in computation did previouslu demand intelligence.
But brute computational power soon came to be distinguished from intelligence.

Often oracles may be thought of as having great wisdom, possibly derived from divine connection, for present purposes the concept of oracle is used in a similar manner to its use in mathematical logic.

Such an oracle is capable of supplying the answers to a particular decision problem concerning the membership of some collection of interest.
In this essay we are primarily concerned with just one such set, the possibilities which would arise from the use of an oracle for that set, the difficulties arsing in implementing such an oracle or an approximation to it.

The set of interest here is the set of \emph{Logical Truths}, a term with a great deal of controversy behind it.
In my usage of that term I stand on a limb, for my use is very similar to that of Rudolf Carnap, and that usage considers it synonymous with the term \emph{analytic} as that term was used by Carnap, despite the term having apparently been discredited by W.V.O.Quine many years ago.

The term ``Oracular AI'' as used here, refers to what AI might in principle be able to achieve if furnished with an oracle for logical truth.

One of the purposes of this essay is to discuss how thus notion of logical truth can be made precise, to consider the difficulties in implementing such a decision procedure and to talk about the value of approximations which fall short of logical omniscience.

\section{Some Historical Background}

The start difference between the reliability of deduction in mathematics and ways of discovering truth in other domains has been plain since the ancient Greeks began the transofrmation of mathe atics into a theoretical science in the shape of axiomatic geometry.
THe results thus obtained were reliable and were to be gradually accumulated and ultimately gathered together as the ``Elements'' of Euclid.
By contrast, those same ancient Greeks, when attempting to reason about nature and the cosmos were unable to establish durable findings, and would find many ways in which deductive reasoning, via reductio, could establish absurd and contradictory conclusions.

The differential success of deductive reason in these distinct domains was to be reflected in the two worlds of Plato's philosophy, of which only that of Platonic ideals was susceptible to truw knowledge, to be reached by reason alone.
Aristotle sought to rescue the applicability of deduction to what we would now call empirical science through his conception of \emph{demonstrative science}, which relied for itscoherence on the special truth assuring characteristics of the fundamental principles of each science.

Millenia followed in which this situation remained largely stable.
The modern conception of science originating in the scientific revolution toward the end of the renaissance resulted in philosophy being split into two camps associated with a primary emphasis on reason and observation respectively as the source of knowledge.
Empiricism retained the idea of reasoning from scientific principles, but insisted on observation and empirical experiment for the discovery and verification of the principles.

\section{Logical Truth}

There has been philosophical controversy over the concept which I am here speaking of with the words \emph{logical truth}.
This terminology was at first used by Rudolf Carnap, alongside the concept of analyticity, but as Carnap's conception of analyticity progressed from the syntactic notion found in his \emph{Logical Syntax of  Language} 

\phantomsection
\addcontentsline{toc}{section}{Bibliography}
\bibliographystyle{rbjfmu}
\bibliography{rbj}

\addcontentsline{toc}{section}{Index}\label{index}
{\twocolumn[]
{\small\printindex}}

%\vfill

%\tiny{
%Started 2023-08-19


%\href{http://www.rbjones.com/rbjpub/www/papers/p038.pdf}{http://www.rbjones.com/rbjpub/www/papers/p038.pdf}

%}%tiny

\end{document}

% LocalWords:
