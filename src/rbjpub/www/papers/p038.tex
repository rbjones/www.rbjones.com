% $Id: p038.tex $
% bibref{rbjp038} pdfname{p038}
\documentclass[10pt,titlepage]{article}
\usepackage{makeidx}
\newcommand{\ignore}[1]{}
\usepackage{graphicx}
\usepackage[unicode]{hyperref}
\pagestyle{plain}
\usepackage[paperwidth=5.25in,paperheight=8in,hmargin={0.75in,0.5in},vmargin={0.5in,0.5in},includehead,includefoot]{geometry}
\hypersetup{pdfauthor={Roger Bishop Jones}}
\hypersetup{pdftitle={The Paradox of Repressive Tolerance}}
\hypersetup{colorlinks=true, urlcolor=red, citecolor=blue, filecolor=blue, linkcolor=blue}
%\usepackage{html}
\usepackage{paralist}
\usepackage{relsize}
\usepackage{verbatim}
\usepackage{enumerate}
\usepackage{longtable}
\usepackage{url}
\newcommand{\hreg}[2]{\href{#1}{#2}\footnote{\url{#1}}}
\makeindex

\title{\bf\LARGE The Paradox \\of \\Repressive Tolerance}
\author{Roger~Bishop~Jones}
\date{\small 2021-01-31}


\begin{document}

%\begin{abstract}
%
% Notes on the "Paradox of Repressive Tolerance".
%
% Discussion of Popper's "paradox of tolerance" and Marcuse's inversion of it as "Repressive Tolerance".
% 
%\end{abstract}
                               
\begin{titlepage}
\maketitle

%\vfill

%\begin{centering}

%{\footnotesize
%copyright\ Roger~Bishop~Jones;
%}%footnotesize

%\end{centering}

\end{titlepage}

\ \

\ignore{
\begin{centering}
{}
\end{centering}
}%ignore

\setcounter{tocdepth}{2}
{\parskip-0pt\tableofcontents}

%\listoffigures

\pagebreak

\ignore{
\section*{Preface}
\phantomsection

\addcontentsline{toc}{section}{Preface}
}%ignore

\section{Introduction}

It is my intention in this short note to discuss a problem which Popper raises in a footnote in Chapter 7 of \emph{The Open Society and its Enemies}\cite{popper-ose}.
In that footnote, Popper does little more than note the problem and assert the need and the right to do something about it.

I propose here to look in a little more detail at the nature of the problem, which might appear in quite diverse ways, and also to consider how liberal democracies might defend against subversion by these means.

I therefore consider two strategies for intolerant subversion exemplified here by the Islamic notion of \emph{Dawa}\cite{ali-dawa,sookhdeo-dawa} and the Marcusian \emph{Repressive Tolerance}\cite{marcuse-repressive}.

It is not crucial to the discussion, or to the purpose of this note, that either of the sketches I give here of these two kinds of ideology is completely accurate, since they stand here as exemplars of certain \emph{kind} of ideology.
The question of interest is how liberal democracy could possibly be protected against idologies of those kinds, which remains of interest even if these kinds had not in fact been instantiated.
To that question I do not offer an answer, though there are some possibilities which are discussed.
My aim is to exhibit the difficulties they pose, and perhaps to add some sense of how great those difficulties are.

\section{Poppers Concern}

``The paradox of tolerance'' is a phrase coined by Popper concerning the defence of a tolerant society from the risk of subversion by an intolerant ideology.

Poppers oft-quoted description of that ``paradox''  is:

\begin{quote}

Less well known is the paradox of tolerance: Unlimited tolerance must lead to the disappearance of tolerance. If we extend unlimited tolerance even to those who are intolerant, if we are not prepared to defend a tolerant society against the onslaught of the intolerant, then the tolerant will be destroyed, and tolerance with them.—In this formulation, I do not imply, for instance, that we should always suppress the utterance of intolerant philosophies; as long as we can counter them by rational argument and keep them in check by public opinion, suppression would certainly be most unwise. But we should claim the right to suppress them if necessary even by force; for it may easily turn out that they are not prepared to meet us on the level of rational argument, but begin by denouncing all argument; they may forbid their followers to listen to rational argument, because it is deceptive, and teach them to answer arguments by the use of their fists or pistols. We should therefore claim, in the name of tolerance, the right not to tolerate the intolerant. We should claim that any movement preaching intolerance places itself outside the law and we should consider incitement to intolerance and persecution as criminal, in the same way as we should consider incitement to murder, or to kidnapping, or to the revival of the slave trade, as criminal.

\end{quote}

\section{Religious Fundamentalism}

The most plausuble instance of religious fundamentalism posing a contemporary threat to liberal democracy is political ISLAM and its clearly documented strategy of DAWA.

Distinctive features of this ``threat'' (if indeed it is) are its \emph{tranparency} and \emph{honesty}.
The desired end-point, a global islamic caliphate under sharia law, is well documented, and the strategy for realising that end is well developed and clearly documented.
To be sure, it is not specifically targeted at the subversion of liberal democracy, which was unknown at its inception, but liberal democracies are nevetheless particularly vulnerable to the same strategy.

The writings which form the Quran fall into two parts, of which the first was written in Mecca and the second in Medina.
It is in the second that Da'wa, as a strategy for the promulgation of ISLAM, was presented, with the ultimate aim of establishing a global Islamic caliphate governed by sharia law.

This exhibits a number of characteristics.

First note the following characteristics of the idiology (in this case a religion) which Dawa is intended to proliferate:

\begin{itemize}
\item It concerns governance of society.
\item It advocates a fixed and unchangeable body of law.
\item It is not democratic.
\item It is underpinned by a rigid authority held to be infallible.
\item It is highly intolerant of challeges to that authority.
\item It advocates lethal violence not only against direct challenges but also against peaceful co-existence.
\end{itemize}

The strategy for proliferation probably is more subtle than I will here present it, but for our discussion it may be considered to have the following characterstics when it is applied to the subversion of a liberal democracy:

\begin{itemize}
\item It seeks to subvert via the ballot box.
\item To secure electoral majorities it adopts the following three methods:
  \begin{itemize}
  \item Immigration
  \item Procreation
  \item Persuasion
\end{itemize}
\item Though Dawa allows for more forceful methods when these are necessary and feasible, up to Jihad, these are not likely to be essential for subversion of a liberal democracy.
\item Once electoral majorities are established, democracy can be dismantled and sharia law established.
\end{itemize}

\section{Neo-Marxism}

Possibly I will here be using the term ``Neo-Marxism'' in a broader way than is usually.
That's not important.

In some ways this subversive strategy is diametrically opposite to the strategy exemplified by Da'wa as described in the previous section.
It is neither clear, transparent, nor honest or coherent.

Rather than progressing some utopian vision of how society should be, it obsesses with a dystopian pseudo-reality and has evolved in the petri-dish of academic freedom within a highly politicised academy to misrepresent how things are and locate all possible grounds for social division to tear down existing social structures and make room for the utopia which they are unable to coherently articulate, and which their own doctrines deny can ever be realised.

A focal point here is the notion of \emph{repressive tolerance} in which Herbert Marcuse adapts Popper's Paradox of Tolerance to exactly that purpose which it was intended to warn us against\cite{marcuse-repressive}.
This is not only a key element in itself, but is a model for systematic inversion.


\subsection{Repressive Tolerance}



\section{Safeguarding Democracy}





\phantomsection
\addcontentsline{toc}{section}{Bibliography}
\bibliographystyle{rbjfmu}
\bibliography{rbj}

\addcontentsline{toc}{section}{Index}\label{index}
{\twocolumn[]
{\small\printindex}}

%\vfill

%\tiny{
%Started 2021-01-31


%\href{http://www.rbjones.com/rbjpub/www/papers/p038.pdf}{http://www.rbjones.com/rbjpub/www/papers/p038.pdf}

%}%tiny

\end{document}

% LocalWords:
