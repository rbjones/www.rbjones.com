% $Id: p005.tex,v 1.1 2005/01/26 20:53:38 rbj Exp $
\documentclass{rbjk}

\begin{document}                                                                                   
\begin{article}
\begin{opening}  
\title{Metaphysical Positivism}
\runningtitle{Metaphysical Positivism}
\author{Roger Bishop \surname{Jones}}
\runningauthor{Roger Bishop Jones}
%\runningtitle{}

\begin{abstract}
An outline of Metaphysical Positivism.
\end{abstract}
\end{opening}

\tableofcontents

\section{Introduction}

A distinctive feature of philosophy is that it may be conducted in an armchair.
It does not require experimentation, and is thus distinguishable from experimental (but not theoretical) science.
Metaphysical positivism combines a liberal conception of the scope of philosophy, one can think philosophically on most matters, with caution about the conditions under which reason can hope to yield knowledge.
Caution about the reliability of reason is connected with skepticism about semantics, with a tendency to doubt that a problem has been formulated with sufficient clarity that it could have a definite answer.
This latter, particularly in relation to any question which is framed in ordinary rather than special or formal language.
One can rarely rely in the formulation of philosophical problems upon concepts as they come to us, advances in knowledge, and particularly in philosophical knowledge depend 

By the juxtaposition of {\it metaphysics} and {\it positivism} I suggeset a kind of philosophy which seeks to address the most difficult problems while recognising the risk of falling into pseudo-problems or going beyond the bounds of reason.

Methodological rigour depends upon clearly formulated problems addressed by sound methods.
Rigour is however a matter of degree, some well-formulated problems may not be susceptible of resolution by the most rigorous methods, and some important problems may be very difficult to clearly formulate.
Problems which can be clearly formulated and rigorously solved will very often belong to some special science.
That a problem is difficult to formulate clearly, or may not yield to the most rigourous methods is an indication that may best be considered a philosophical problem.

Metaphysical positivism is concerned with the entire spectrum of knowledge.
It is concerned with the methods which can be adopted to achieve the highest clarity and rigour in those domains in which this is possible, even though the problems themselved may belong to other disciplines.
It is concerned also with methods through which one might regress as the problems get tougher, particularly s they get harder to make clear or when they fall beyond the scope of the most rigourous methods.

Positivism usually involves a critique.
You will find that here, only between the lines.

The plan is first to articulate idealised rigorous methods, to explore their limits, to consider problems which lie beyond those limits and to seek the best extensions to the rigourous methods for addressing these difficult problems.
Often these more difficult problems will be metaphysical, though they will rarely be quite the traditioal problems of metaphysics.
For example, metaphysics is concerned with the ultimate nature of reality, and of course, what exists.
Metaphysical positivism embraces that concern, but starts from the position that the question ``what exists? has no answer, it is not a well formulated problem.
On the other hand, the question ``what might there be?'' can be made into a well-formulated problem to which partial answers can be given.
It is also an ``open'' question in the sense that our answers will always be incomplete, but that by progressing through ever more tenuous methods we can gradually fill out more of the answer.
Research into large cardinal axioms for set theory is such an enterprise.

\section{Rigorous Methods}

Mathematics is generally regarded as the most rigourous science because its results are established by deductive proof.

There are degrees of rigour, mathematics has not always been as rigorously conducted as it has been in the last century.

One reason why rigour is valued is that it reduces the chances of error.
The existence of a deductive proof in a sound deductive system is a guarantor of truth.
However we cannot ever be absolutely certain that we have such a proof.
There may be an error in a putative proof which has been overlooked.
The deductive system in which a proof has been constructed may not after all be sound.

Though rigour may be valued because it is thought to reduce the chance of error, differences it is desirable to have a notion of rigour which discriminates more finely than would a measure based of probability of error.
For example a deductive system either is or is not sound with respect to any definite semantics, but we may reasonably have different degrees of confidence in the soundness of a deductive system.
It is probable that all first order axiomatisations of arithmetic not stronger than Peano arithmetic are sound and hence that they all yield no untrue theorems.
But our grounds for belief in the soundness of these systems will vary.
In particular they will be partially ordered by inclusion on the set of axioms, the fewer axioms the greater confidence we can have of soundness.

%{\raggedright
%\bibliographystyle{klunamed}
%\bibliography{rbj,fmu}
%} %\raggedright

\end{article}
\end{document}
