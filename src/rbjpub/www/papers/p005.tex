% $Id: p005.tex,v 1.4 2006/10/22 13:38:40 rbj01 Exp $
% bibref{rbjp005} pdfname{p005} 
\documentclass{rbjk}
\usepackage{hyperref}
\begin{document}                                                                                   
\begin{article}
\begin{opening}  

\title{Notes on the Philosophy of Leibniz}
%\runningtitle{The Philosophy of Leibniz}
\author{Roger Bishop \surname{Jones}}
\runningauthor{Roger Bishop Jones}
%\runningtitle{

\begin{abstract}
Mainly about his {\it lingua characteristica} and {\it calculus ratiocinator}.
\end{abstract}
\end{opening}

\tableofcontents

\section{Introduction}

The two sections which follow contain:

\begin{enumerate}
\item some notes on what the idea was (of which I speak) and how Leibniz set about realising the idea, and any other related parts of Leibniz's thought,
\item some criticisms, of the feasibility of the idea or of Leibniz's ideas about how to go about its realisation, and some analysis of the impact of the criticisms.
\end{enumerate}

\section{Leibniz's Ideas}

The ideas of principle concern here are the two connected ideas of a {\it lingua characteristica} and a {\it calculus ratiocinator}.

The {\it lingua characteristica} was to be a universal formal language for which a {\it calculus ratiocinator} could be devised.
A {\it calculus ratiocinator} is, in modern parlance, a decision procedure for the truth of sentences in the {\it lingua characteristica}.

Since the language was to cover knowledge, and Leibniz sought the development of an encyclopedia as a cooperative venture which would embody all this knowledge.

\section{Criticisms of the Idea}
Its not hard to come up with serious criticisms of the essential idea, mostly based on things we now know but which were discovered after Leibniz.

\begin{itemize}
\item philosophical objections against rationalism
\item logical objections to the possibility of a decision procedure
\item the impossibility of completing science
\item complexity considerations
\end{itemize}

\subsection{Against Rationalism}

Leibniz thought that all knowledge could in principle be derived {\it a priori} and it might be thought that his conception of a {\it calculus ratiocinator} depends upon such a rationalist premise.

These may have been interlinked in Leibniz's thought, but the existence of a formal decision procedure for truth does not depend upon whether the knowledge in question is a priori.

If one conceives of the discovery of this decision procedure as an alternative to the progression of science towards complete knowledge of the empirical world, then anti rationalist arguments may have more relevance.
It seems that Leibniz did envisage a completion of the work, and this would depend either upon completion of science in one of two ways neither of which seems very likely.
Completion in the sense of having discovered and formalised all that there is to know seems an unlikely prospect.
Completion in that more limited sense in which we may in mathematics know a formal system in which all the truths are derivable, yielding a formal decision procedure, even though the answers to many particular problems may be unknown.

\subsection{Logical Objections}

We know that arithmetic, and second order logic, are incomplete.
This entails that there is no decision procedure for either of these two theories.
There are lots of other theories with similar characteristics.

Clearly the existence of Leibniz's {\it calculus ratiocinator} would entail a decision procedure for arithmetic, and hence is unrealisable in principle.

The best we could hope for would be an approximation.

Is a usable approximation plausible?

I think so.
For the purposes of mathematics, ZFC is a good formal approximation to set theoretic truth and provides answers to the vast majority of questions in mathematics.
It is also susceptible of strengthening by the use of large cardinal axioms and others if it were found for some application to be insufficient.
This goes far beyond the kind of mathematics required for science, and so if we had good (mathematical) scientific models of the universe then first order set theory would provide a very good approximation to a decision procedure of the kind sought by Leibniz.

This argument assumes that the calculus is expected to be an effective decision procedure in the modern sense (unknown to Leibniz), i.e. Turing computable.
There are some who now seem to believe that machines can be built which exceed the powers of Turing machines, and if this were the case then the objections here discussed would no longer be applicable.

\subsection{The Impossibility of Completing Science}

It is possible, many would think probable, that science will never be completed in the way required to make an oracle possible.

There are several ways in which this might happen:

\begin{description}
\item [irreducible non-determinism]

It may be that the universe is irreducibly non-deterministic so that no matter how complete our knowledge of the laws of physics and of the present state of the universe we could not completely predict its future course.
This is indeed what modern physics tells us.

\item[no final theory]

It is possible, as perhaps experience should now strongly suggest, that the laws of fundamental physics are and will always be approximate models of a reality which is to complex ever to completely formalise.
Whatever the laws appear to be within the limits of certain experimental facilities, it will appear more complex if the experiments are enhanced (e.g. as the energy of the interactions is increased).

\item[unformalisable boundary conditions]

To forecast the state of the universe in the future you need not only a complete formal account of the laws of the universe, you need to know the brute facts about how the universe has been configured at some point in its history.
It is possible that this will never be realisable.
Heisenberg's uncertainty principle suggests that such knowledge is unattainable, and even if we doubt that principle (which some do) one can still reasonably doubt that such complete knowledge is attainable.

It may also be the case that the information in the universe is not finite, and that there can be no formal description of the state of the universe, and hence no possibility of a decision procedure even if the laws of the universe were formalisable.

\end{description}

\subsection{Complexity Considerations}

Complexity considerations represent a major problem for the practical realisation of a decision procedure even for modest parts of mathematics.
Extremely close approximations to decision procedure for first order arithmetic can easily be obtained from accepted formal theories such as PA or ZFC.
However, though in principle such decision procedures will yield answers to almost all the questions we are likely to put to them, in practice even with today's technology many quite simple questions would not be answered within an acceptable timeframe.

\section{Remarks on the Significance of the Criticisms}

It may be argued that the purpose of the calculus is to mechanise reasoning, and that we should therefore distinguish two kinds of objection to it:

\begin{itemize}
\item those concerning the possibility of automating reason
\item those concerning the limits of reason 
\end{itemize}

Thus it may be argued that, even if the scope of deductive reason proves narrower than Leibniz imagined, its automation is a good idea, and that his project insofar as it applies to logic and mathematics is worth progressing even if there are severe limits on our prospects of complete mathematical models of the contingent world.
This would not entail limiting the scope of our ambitions strictly to mathematics.
Existing mathematical software is economically supported not merely by its applicatiosn in mathematics, or even in science, but by its usefulness to engineers.
The economic benefit of the automation of mathematical reasoning is likely to be greatest in its application to engineering design.
For this purpose a project inspired by Leibniz would be best conceived as supporting engineering applications of mathematically formulated scientific theories.

\section{Metaphysics}

\subsection{Russell on the Subject Predicate form}

Russell criticises, first as false and secondly as leading to many difficulties in Leibniz's metaphysics, the doctrine that every proposition has subject predicate form.

Insofar as the second matter is concerned, I accept that this doctrine may well have lead to difficulties, but I doubt that these difficulties were inevitable because it seems to me that the doctrine is sustainable and harmless.

It may be that I believe this only because I do not understand the doctrine adequately, and it certainly is the case that I do not understand it.
It seems to me that predication here embraces not only what one might think predication after an education in mathematical logic, but also the different notion of inclusion.
Possibly a coherent logical system could be (or has been) devised in which a single concept embraces both of these ideas, but I have no idea of why this would be a good idea.
Be that as it may, for each of these separate ideas there are logical systems which are unlimited in expressiveness and strength and which conform to the relevant notion of subject/predicate form.

The one properly describably in today's terminology as having that form is Church's Simple Theory of Types.
This is a reformulation of the unramified version of Russell's Theory of Types, based on the typed lambda-calculus.
In this propositions are terms of a certain type (the type of propositions) and must either be function applications (in which case the function must be a propositional function, or predicate, and the second is some ``subject'' of which the propositional function is asserted) or simply a boolean constant  (of which there are just the two, {\it true} and {\it false} which are inessential and can be dispensed with in favour of the predication of self identity to some object or its denial).

A second example of a logical system which is based on a kind of predication relates to the kind of predication in an assertion such as that ``men are mortal'', which we might more naturally regard as predicating inclusion between two concepts rather than predicating mortality of men.

Before mentioning this second system I note that this kind of predication is also represented in the first system in subject predicate form.
There is more than one way of presenting this proposition in that system, all of them having this particular kind of subject predicate form.
One method is to use universal quantification, and express the sentence as `for all x, if x is a man then x is mortal'.
In Church's system this proposition is represented as the predication of universality to the property expressed by the body of the quantified sentence.

Predication of this kind (inclusion) is more directly represented in the illative combinatory logic with the primitive illative combinator known as `restricted generality', so called because it is the equivalent in combinatory logic to quantification over a restricted domain.
In this combinatory logic all propositions are formed using this combinator, and therefore can be said in this sense to be predications.
For this particular case supposing that the properties of being mortal and being a man were represented respectively by the combinators O and A, then the predication in question would be written `$(\Xi O) A$`, and all propositions expressible in the system (which is as expressive as Russell's Theory of Types, have the form `$({\Xi} e1) e2$' where e1 and e2 are expressions in the language, i.e. all propositions are predicates of the kind at present under consideration.

Russell's concern, and the difficulties arising in Leibniz's philosophy, was how relations could be accomodated in such a system.
Though Liebniz's account may well have been as problematic as Russell found it, we have now these examples of expressive systems which can cope with relations without any richer propositional structure than one which can reasonably be described as being invariably of subject-predicate form.

So how are relations accomodated in these systems?
Well they are both functional systems in which functional abstraction is primitive, and in which higher order functions are permitted.
Functions are themselves a special kind of relation, and can be used to represent arbitrary relations.
When a higher order function is used to model an arbitrary (not necessarily a functional or many-one relationship) then a function of type $a \rightarrow (b \rightarrow BOOL)$ is used.
Such a function takes yields as its result a second function, and a proposition is realised only when this function is applied to the second relatum.
The subject of the sentence is the second relatum, the predicate is an expression consisting of the function applied to the first relatum.

\subsection{On Analysis}

Here are a few preliminary ideas, on reading Russell on Leibniz.

I'm not too clear what Leibniz meant by analytic, more reading required here.
He held to a kind of atomic theory of predication, somewhat suggestive of Russell's {\it Logical Atomism}, so presumably Russell's theory is an attempt to repair Leibniz's.
He thought that predicates were either complex or simple and that complex predicates were built from simple predicates, possibly just by conjunction.
In his thinking about the calculus ratiocinator the idea was that simple predicates would be given a prime number and that complex predicates could then be represented by the product of the primes of their simple constituents.
True sentences would then be predications in which the predicate is contained in the subject, i.e. in which the simple factors of the predicate are all simple factors of the subject, and hence ``contains in'' can be read ``divides''.

Now Russell talks as if an analytic truth was {\it by definition} one which could be shown to be true by an analysis along these lines.
He takes it as read that the incompatibility of two simple predicates {\it must} be synthetic.

The question arises for me then, at this point:
is Leibniz's idea about how analytic truth can established to be taken as a definition of a narrow conception of analysiticity, or as an incorrect theory about how one can establish compliance with a broader conception of analyticity.

I've said this as a question about analyticity, because that's how the passage I have been reading in Russell is talking.
However, Leibniz's idea of the `calculus ratiocinator' is that all truths are analysable in this way, not just analytic truths.

I had not previously been aware of the connection between logical atomism and Leibniz's philosophy.
This certainly makes Leibniz's philosophy of greater interest for me.

\section{From Leibniz to X-Logic}

This is an outline of a large writing project which is about those aspects of the thought of Leibniz which are relevant to the automation of reasoning, the history of those ideas down to the present day, and a modern story on the same theme.
The account throughout will focus on explanation of the ideas and their evolution and will cover both technical and philosophical aspects.
However, to whatever extent is possible, matters of a purely technical nature will be presented independently of any philosophical speculations, so that materials of this kind might conceivably be published to audiences not sympathetic to philosophical discussion.
It should be noted that there is a major subdivision which bears upon the technical/philosopical division.
This is the subdivision between the automation of reasoning about mathematics and its applications, and the automation of reasoning (to whatever extent that may be possible) in other problem domains. 

Now some very prelinary ideas about what topics would be covered.

\subsection{Leibniz's Ideas}

This is the starting point.

The following aspects of Leibniz's thought should be covered.
\begin{itemize}
\item the lingua characteristica and calculus ratiocinator - not just the bare ideas, but a certain amount of detail about how he sought to realise these ideas.
Its not clear to me how far this would take us into Leibniz's philosophy, by Russell's account it all hangs together rather tightly.
\item the calculus - this is relevant not only because the calculus is arguably the most applicable part of mathematics, but also because it was the perceived lack of rigour in the calculus which catalysed the developnments in the nineteenth century leading to the revolution in logic essential to success in the automation of reasoning.
The story of this kind of analysis can run right through the timeline, starting with Leibniz's contribution and perhaps Berkeley's criticism through the rigourisation of analysis, its arithmetisation and logicisation to its automation.
\item the scope of deductive reason, the theory of knowledge.
\end{itemize}

\subsection{Mathematics and Logic}

My inclination here is to do a history of analysis and the route from its rigourisation throught the logical foundations of mathematics into the new logic and the new discipline of mathematical logic.


\subsection{Computation}

Not sure how much interest, beyond a mere mention, Leibniz's work on calculators would yeild.
Not sure how much to make of the whole computational thread, it might be quite modest.

Here are some ideas about what might appear in here.

\begin{itemize}
\item Leibniz's work on calculators
\item his ideas on the calculus ratiocinator
\item something about babbage?
\item some relevant aspects of the theory of computation
\item the invention of the digital computer
\item relevant aspects of theoretical computing?
\end{itemize}

\subsection{Philosophical Aspects}

Not to clear what philosophical topics to address.
Primarily of course, those which bear upon the central concern!

\subsection{X-Logic}

This is the terminus.

\subsection{Organisation}

One possible organisation would be to do it in parts corresponding to the above subsections.
A second would be to do it more historically, weaving the threads together through the history.
I'm inclined to that approach

The stuff on Leibniz first, that on  X-logic last, and the other subsections inbetween, threaded together but possibly in different chapters so that it might be possible to read a thread by selecting a subset of the chapters.

%{\raggedright
%\bibliographystyle{klunamed}
%\bibliography{rbjk,fmu}
%} %\raggedright

\end{article}
\end{document}
