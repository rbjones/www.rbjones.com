% $Id: p009.tex,v 1.1 2006/03/25 22:50:37 rbj01 Exp $
\documentclass{rbjk}

\begin{document}                                                                                   
\begin{article}
\begin{opening}  
\title{Metaphysical Problems}
\runningtitle{Metaphysical Problems}
\author{Roger Bishop \surname{Jones}}
\runningauthor{Roger Bishop Jones}

\begin{abstract}
\end{abstract}
\end{opening}

%\def\tableofcontents{{\parskip=0pt\@starttoc{toc}}}
\setcounter{tocdepth}{4}
{\parskip-0pt\tableofcontents}

\section{Preface}

This monograph is presented in three parts.

The first sets the philosophical context.
It is technically superfluous but may help the reader to understand the motivation for the considerations which follow.
The second describes the language to be used in formulating the problems.
It is logically essential to an understanding of what follows.
The third part is the description of certain philosophical problems.

\part{First Philosophy}

This section is intended to put this work in context, primarily to assist in understanding the problems formulated later.

\section{What is Philosophy?}

I'm going to sketch here a conception of philosophy and of the philosopher.

When I began first to study philosophy at University I recall that some effort was made to disabuse students of common misconceptions about the nature of philosophy.
This was not so very long after the heyday of ``linguistic philosophy'', philosophy was still considered to be primarily concerned with conceptual analysis, and it was important for students to understand that {\it analytic} philosophy did not tell us how to live our lives, what political instutitions we should adopt or in any other domain what was right or good.

In diverse philosophy of the twentieth century there have been a number of very narrow conceptions of philosophy.
By contrast we may find closer to the origins of western philosophy, much broader conceptions of the nature of philosophy in classical Greece.

What I would like to say is that the philosophy which interests us here is of the more classical variety, but when we follow through the logic of some of these classical conceptions we come perilously close to slipping into one of the more narrow modern conceptions.

I propose therefore to begin with some ideas drawn from Aristotle and Plato, contrast these with some modern conceptions of philosophy and to pull some kind of compromise from this melee.
 
\subsection{First Philosophy in Aristotle}

Aristotle begins his {\it Metaphysics} with a discussion of what he variously and revealing calls {\it first philosophy}, {\it philosophy} simpliciter, and {\it wisdom}, and which we might today regard as a classicall conception of {\it metaphysics}, very different to modern conceptions.

He does this partly by mentioning the concerns of this kind of philosophy, viz. first causes or principles, but also by several attempts to describe linear spectra at one mysterious end of which this kind of philosophy can be found.

First philosophy is concerned with particularly exalted kinds of knowledge which Aristotle calls {\it wisdom}.
We get a grasp of what kind of thing this is by considering the differences between various more mundane kinds of knowledge and extrapolating.
The most primitive kind of knowledge is that which we have from our senses of what is here present to us.
The next is the broader knowledge which admits not only what is now present to our senses but also what has been so present in the past and has been committed to memory.
Next we have {\it experience}, which consists of an aggregation of this memorised sensory knowledge in some domain followed by {\it art} or {\it science} (which seem to be used interchangeably) which is arrived at by generalisation from the particulars of experience and is distinctive in involving universal judgements.

Now that we have reached science the spectrum continues by progression from lower to higher sciences, the lower being those practical arts which are most necessary for survival through from the practical to the theoretical then to those which are not necessary but recreational and then to those which are studied neither from necessity nor for pleasure but simply for their own sake, for the sake of knowledge.


\subsection{notes}

Modern philosophy seems to have left behind this desire for a deep understanding of the true nature of reality.
The study of nature has become the province of science, and the deepest questions about the nature of reality are now the province of fundamental physics.
Philosophy, having become in the 20th Century, with the emergence of analytic philosophy, painfully aware of how philosophers have underestimated the significance of language for their work, has become engrossed in language, and unwilling to trespass on the domain of any other science.

\subsection{Notes on Aristotle's Metaphysics}

Spectrum of knowledge:
\begin{itemize}
\item sensory
\item memory
\item experience
\item art = universal judgement i.e. generalisation
\item inventors of arts are wiser and superior to others
\item inventors of recreational arts are wiser than those of arts concerned with necessities
\item when necessary and recreational arts have been discovered we come to arts which are concerned neither with utility nor pleasure.
\end{itemize}

\begin{itemize}
\item masterworkers more honourable than manual workers because they know why
\item masterworkers are wiser in virtue of having the theory and knowing the causes
\item art more truly knowledge than experience, for artists can teach, men of mere experience cannot
\end{itemize}

``the point of our present discussion is this, that all men suppose what is called Wisdom to deal with the first causes and the principles of things; so that, as has been said before, the man of experience is thought to be wiser than the possessors of any sense-perception whatever, the artist wiser than the men of experience, the masterworker than the mechanic, and the theoretical kinds of knowledge to be more of the nature of Wisdom than the productive. Clearly then Wisdom is knowledge about certain principles and causes.''

\begin{itemize}
\item the wise man knows all things, as far as possible, although he has not knowledge of each of them in detail
\item he who can learn things that are difficult, and not easy for man to know, is wise
\item he who is more exact and more capable of teaching the causes is wiser
\item that which is desirable on its own account and for the sake of knowing it is more of the nature of Wisdom than that which is desirable on account of its results
\item the superior science is more of the nature of Wisdom than the ancillary
\item the wise man must not be ordered but must order, and he must not obey another, but the less wise must obey him
\end{itemize}

And yet another approach:

\begin{itemize}
\item Now of these characteristics that of knowing all things must belong to him who has in the highest degree universal knowledge; for he knows in a sense all the instances that fall under the universal
\item the most exact of the sciences are those which deal most with first principles; for those which involve fewer principles are more exact than those which involve additional principles, e.g. arithmetic than geometry
\item the science which investigates causes is also instructive, in a higher degree
\item understanding and knowledge pursued for their own sake are found most in the knowledge of that which is most knowable (for he who chooses to know for the sake of knowing will choose most readily that which is most truly knowledge, and such is the knowledge of that which is most knowable)
\item the first principles and the causes are most knowable
\item the science which knows to what end each thing must be done is the most authoritative of the sciences
\item and this end is the good of that thing
\end{itemize}

``Judged by all the tests we have mentioned, then, the name in question falls to the same science; this must be a science that investigates the first principles and causes; for the good, i.e. the end, is one of the causes.''

\section{Speculative Scepticism}

I think of myself as an extreme sceptic.

In the sense intended a sceptic is one who seeks knowledge, fails to find any, but continues to seek.
He therefore retains an open mind and is to be contrasted with someone who believes that he has found knowledge and no longer has an open mind.

Would-be skeptics often fall of the rails and slip into a negative dogmatism.
A {\it would-be} sceptic has in fact already fallen off the rails, if he thinks scepticism a desirable condition.

I will mention more specifically some of the pitfalls.

The most obvious is to affirm dogmatically that no knowledge is possible, and this is of course not extreme scepticism, but negative dogmatism.

Pyrrhonean scepticism is associate with the idea that a sceptic affirms no more than that ``appearances appear''.
A related concept is that of equipollence.
One can doubt not only the knowledge of a proposition but even that we can know a proposition to be more likely than not.
A proposition is ``equipollent'' if it and its negation are equally plausible.
This translates in some sceptics into a skeptical purpose: that not merely of refuting claims to knowledge but of demonstrating ``equipollence''.
This is of course, a kind of dogmatism.

\section{Constructive Rationalism}

\section{Positivism}

\section{Logic and Language}
\section{Descriptive Language and Semantics}

Philosophers have discovered in the 20th Century that language takes many forms and works in diverse ways.
Wittgenstein, having in his {\it Tractatus Logico-Philosophicus} presented language as essentially involved in saying something about the world, replaced this narrow view of language with the idea that languages are in general more like games, and may follow rules of a different character.

For present purposes, i.e. for the enunciation of the problems in metaphysics of present interest, the kind of language we need is primarily descriptive.
For such languages the rules which govern correct usage can be captured by a ``truth conditional'' semantics.

\subsection{Abstract Entities}

I'm going make some observations here about how the ``game'' of talk about abstract entities works. 

So far as mathematical entities are concerned, there seem to me to be two kinds of talk.
There is the {/it mathematical} talk and the {\it philosophical}.

For the mathematics, to talk about a class of entities it is necessary to that this class be axiomatically cahracterised or alternatively constructed from other mathematical entities (usually in the context of set theoretic ontology.

\part{Philosophical Language}

In this section is defined the language which I will use to state the problems.
This is not a formal language, and not a language sharply differentiated from ordinary English.
It is rather a particular usage of English which is convenient for my present purposes.

\part{Metaphysics}

\section{Abstract Ontology}

\section{Concrete Ontology}

%{\raggedright
%\bibliographystyle{klunamed}
%\bibliography{rbj,fmu}
%} %\raggedright

\end{article}
\end{document}
