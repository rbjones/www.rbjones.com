% $Id: p010.tex,v 1.5 2009/05/06 10:02:41 rbj Exp $ bibref{rbjp010} pdfname{p010} 
\documentclass[numreferences]{rbjk}
\usepackage{makeidx}
\newcommand{\ignore}[1]{}
\usepackage[unicode,pdftex]{hyperref}
\hypersetup{pdfauthor={Roger Bishop Jones}}
\hypersetup{colorlinks=true, urlcolor=black, citecolor=black, filecolor=black, linkcolor=black}

%\newtheorem{def}{Definition}
%\newtheorem{conj}{Conjecture}

\makeindex
\newcommand{\indexentry}[2]{\item #1 #2}
\newcommand{\indexname}[0]{Index}
\begin{document}                                                                                   
\begin{article}
\begin{opening}  
\title{Designing a Deductive Foundation System}
\runningtitle{Deductive Foundation Systems}
\author{Roger Bishop \surname{Jones}}
\date{$ $Date: 2009/05/06 10:02:41 $ $}
\runningauthor{Roger Bishop Jones}

\begin{abstract}
A discussion of issues in the design of formal logical foundation systems suitable for use in machine supported formal derivations of analytic models.
The outlines of a proposed system with a roadmap for turning the outline into a specification for implementation. 
\end{abstract}
\end{opening}

\vfill

\begin{centering}
\footnotesize{
Created 2006/10/22

Last Change $ $Date: 2009/05/06 10:02:41 $ $

\href{http://www.rbjones.com/rbjpub/www/papers/p010.pdf}{http://www.rbjones.com/rbjpub/www/papers/p010.pdf}

$ $Id: p010.tex,v 1.5 2009/05/06 10:02:41 rbj Exp $ $\\

}%footnotesize
\end{centering}

\newpage
%\def\tableofcontents{{\parskip=0pt\@starttoc{toc}}}
\setcounter{tocdepth}{4}
{\parskip-0pt\tableofcontents}

\section{Preface}

This essay is concerned with certain aspects of the design of a ``logical foundation system''.

I propose to begin with some discussion of how that expression is to be understood for present purposes and then to describe the gross structure of the following description.

First of all, the notion of ``foundation system'' is to be understood as similar to the kinds of system which have been described as foundations for mathematics.
However, the purpose is broader, and the intended scope encompasses all analytic truths, understanding that phrase to include the truths of mathematics.
It is not my purpose here to further discuss the notion of analyticity.

To clarify the kind of ``foundation system'' at stake I will mention briefly three different but related purpose which foundation systems might have.
There are:

\begin{description}
\item[semantic]\ 

A semantic foundation system is a language in which the semantics of other languages can be rendered.

\item[proof theoretic]\ 

A proof theoretic foundation system is a logic (a language together with an inference system) to which other logics may be reducible.

\item[mechanisable formal]\ 

A mechanisable formal foundation system is a language with a semantics and a deductive system which is suitable for mechanisation in the form of computer software which supports various aspects of the development of models and theories.
\end{description}

In all three cases is it a requirement of something being called a \emph{foundation system} in the present sense, that the application of the system requires no axiomatic extension, additional vocabulary as required for applications, being provided by \emph{definitions} or more liberally some notion of \emph{conservative extension} which must be so determined that such extensions do not risk the coherence of the system.

A foundation system would be universal if \emph{all} \ semantics or deduction (as appropriate) was reducible to it.
There is reason to doubt that there are universal foundations, but reason to hope that certain closely related families of foundation system may be universal (supposing all these concepts to have been made more precise).
The strongest contender for universality is the theory of well-founded sets, both as a foundation for \emph{abstract} semantics and as proof theoretic foundation.
The restriction here to abstract is important to the dual role, since, as I have argued elsewhere \cite{rbjp001} abstract semantics suffices for analyticity and deduction.

The role of the first two kinds of foundation system is primarily theoretical.
When logicians demonstrate that some result is provable in ZFC, they do not usually exhibit a formal proof in first order logic.
They present a proof to the normal standards for mathematics, which is rigorous but not formal.

It is only the third kind of foundation which is intended for use in formal derivations, which because of the very large number of minute steps involved, and practicable for extensive use only with appropriate suppoert from IT systems.
Once formal systems are applied in this kind of way rather than being confined to metatheoretic investigations, a host of practical issues arise which introduce desiderata alien to the metatheory and which conflict with the simplicity which is conducive to transparent metatheory.

The foundations of mathematics are of interest in several different academic disciplines, notably mathematics, mathematical logic, philosophy and computer science.
There is a wide variety of formal systems which have been advocated in such a role, and diverse views of what a foundation system is and of attributes are desirable in a foundation system.
It is therefore of concern here to present as clearly as possible the rationale for the choices which are presented.

The essay which follows is an offshoot from certain theoretical studies which are intended to lead to the specification of a new foundation system for implementation in software and application in specifying and reasoning about formal abstract models in any application domain.
This encompasses the whole of mathematics and many other application domains.
The purpose of writing the essay is to help in the clarification of the objectives and methods of this exercise.

The essay begins with some autobiographical material, whose purpose is to help elucidate the origin and motivation of various aspects of the foundation system sought and of the methods being used in its development,
It is intended that this section be entirely dispensible, and the reader is recommended to begin his reading after this section and to refer back to it only if he feels the need for a better understanding of the reasons for the choices which have been made in the sequel.

The next section presents the key choices which will determine the character of the foundation system choice.
It is a statement of requirement, followed by some broad indications of how the requirements will be met.
Then we come to a presentation of the method being progressed to yield a detailed formal specification of a foundation system meeting the stated requirements.

\section{Desiderata for Foundation Systems}

\subsection{Beyond Well-Foundedness}

\subsection{Structuring the Namespace}

\section{Methods}

The method is \emph{semantics first}, \emph{syntax last}.

We begin with the standard, reasonably tall, ontology of a pure well-founded set theory.
\emph{Standard} here means that the power sets are complete, which entails (for typical axiomatisations) that the ontology really is well-founded, though the well-foundedness is probably sufficient for our purposes.
\emph{reasonably tall} here means that every set is a member of a set with strong closure properties (closed under replacement).

We then transform this ontology in stages.

Three general kinds of transformation are employed which I call:

\begin{itemize}
\item[Inductive Transformations]
\item[Co-inductive Transformations]
\end{itemize}

Both of these involve obtaining a subset of the previous domain by some transfinite process which yields a domain of similar size to the original but in which the elements share various chosen structural features, and then defining new relations and/or operations over the domains which exploit this additional structure.

To describe this in a little more detail we give an account first of the well-founded sets, and then of the operation of the inductive and co-inductive definitions.

\subsection{Pure Well-Founded Sets}

Our ontological starting point is the pure well-founded sets.

Abstract ontology provides us with the building blocks for abstract models which ultimately can be used for reasoning about the real world, and form the subject matters of mathematical theories.

If we consider the ways in which complex structures might be built, these may all be thought of as consisting in putting together various components in specified ways to realise a larger structure of which the components are a part.
Such a structure might be described using a parts list and a method of construction.

Set theory confines itself to the very simplest way of combining parts into wholes, simple aggregation.
To know what how a set is built you need only know what are its members (which for present purposes are its parts).
The members of set are not arranged in any particular way, there can be no two distinct sets with the same members.
This is the principle of extensionality, which characterises the notion of set.

It turns out that this very simple method of combination cannot be surpasses, in the sense that the range of abstract structures obtainable by more complex methods of combination does not surpass what can be done with this most simple method of combination.
For this reason set theory is a prime candidate for an ontological foundation for abstract modelling.

What we have discussed so far is what kind of thing a set is.
To make our foundation we need to have a rich collection of sets, and it turns out that simple ideas for what that collection might be (such as ``all'' sets) do not yield satisfactory results.
Some further decisions have to be made to determine a suitable collection of sets.
This may be said to be the insight which first arose from the work of Frege on the foundations of mathematics when confronted with ``Russell's paradox, and the seemingly arbitrary choices about ontology which seem necessary in choosing a foundation system are one of the reasons why the instincts of Frege and Russell in thinking mathematics reducible to logic gave way to the view that set theory and any reasonable foundation for mathematics go beyond logic into some other domain.

To progress from the concept of set to a more or less determinate ontology of sets it is helpful to refine the concept a little.

The concept of a \emph{pure, well-founded collection} (for which we will use here the term ``pwf-set'') can be defined informally by transfinite recursion as follows:

\index{set}\index{set!pwf-set}

\begin{centering}
a pwf-set is any definite collection of pwf-sets (and nothing else is)
\end{centering}

From this definition we can conclude that the empty collection is a pwf-set.
Very explicitly the definition enables us to infer of any definite collection all of whose members are known to be sets, that the collection is also a set.
It also follows, not quite so trivially, that pwf-sets are indeed well-founded.

However, it appears that the concept of a pwf-set is cannot be definite, for if it were the pwf-sets would constitute a definite collection of pwf-sets, and a contradiction ensues.
In the context of some prior understanding about ontology the definition serves to distinguish the pure well-founded sets from any others there might be, but in a foundational context the extension of the concept it defines only becomes definite if some choice is made to limit the process implicit in the definition.

The definition corresponds closely to the more usual presentation known as the iterative or cumulative hierarchy of sets.
In this the formation of a domain of sets is described as taking place in stages, at each stage forming all new sets which can be formed from elements obtained in previous stages.
The formation of ``all'' sets at each stage is crowned by the supposition that the universe is then formed as this process runs through all possible stages.

The argument above contradicts this last possibility.
The misfortune of being unable to complete this process has a useful side effect which we will exploit.
However large the domain of pwf-sets we take our foundational ontology to be, there will always be more to be had, the first of which will be the domain itself.
This means that we can take our staring ontology and perform further constructions upon it.

\subsection{Inductive Transformations}

\subsection{New Relations for Old - Leaving Well-Foundedness}

\subsection{Co-Inductive Transformation}

\section{Specifics}

\subsection{Poly-Sets}

Poly-Sets are now being addressed in several other documents \cite{rbjp011,rbjt020,rbjMembership}.

\subsection{Poly-Categories and Functors}

\subsection{First Order Axiomatisation}

\subsection{Structured Type Theory}

{\raggedright
\bibliographystyle{klunum}
\bibliography{rbjk}
} %\raggedright

\twocolumn[\section{Index}\label{Index}]
{\small\printindex}

\end{article}
\end{document}
