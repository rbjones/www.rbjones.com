% $Id: p047.tex $
% bibref{rbjp047} pdfname{p047}
\documentclass[10pt,titlepage]{article}
\usepackage{makeidx}
\newcommand{\ignore}[1]{}
\usepackage{graphicx}
\usepackage[unicode]{hyperref}
\pagestyle{plain}
\usepackage[paperwidth=5.25in,paperheight=8in,hmargin={0.75in,0.5in},vmargin={0.5in,0.5in},includehead,includefoot]{geometry}
\hypersetup{pdfauthor={Roger Bishop Jones}}
\hypersetup{pdftitle={An Architecture for Propositional Knowledge}}
\hypersetup{colorlinks=true, urlcolor=red, citecolor=blue, filecolor=blue, linkcolor=blue}
%\usepackage{html}
\usepackage{paralist}
\usepackage{relsize}
\usepackage{verbatim}
\usepackage{enumerate}
\usepackage{longtable}
\usepackage{url}
\newcommand{\hreg}[2]{\href{#1}{#2}\footnote{\url{#1}}}
\makeindex

\title{\LARGE\bf An Architecture for Propositional Knowledge}
\author{Roger~Bishop~Jones}
\date{\small 2022/11/12}

\begin{document}

%\begin{abstract}
% An exercise in synthetic epistmology, constituting architectural principles for a hypothesized future Galactic cognitive system.
%\end{abstract}
                               
\begin{titlepage}
\maketitle

%\vfill

%\begin{centering}

%{\footnotesize
%copyright\ Roger~Bishop~Jones;
%}%footnotesize

%\end{centering}

\end{titlepage}

\ \

\ignore{
\begin{centering}
{}
\end{centering}
}%ignore

\setcounter{tocdepth}{2}
{\parskip-0pt\tableofcontents}

%\listoffigures

\pagebreak

\section*{Preface}


\addcontentsline{toc}{section}{Preface}

\footnote{There may be ``hyperlinks'' in the PDF version of this document which either link to another point in the document  (if coloured blue) or to an internet resource  (if coloured red) giving direct access to the materials referred to (e.g. a Youtube video) if the document is read using some internet connected device.
Important links also appear explicitly in the bibiography.}

\section{Introduction}

This essay is a thought experiment in synthetic epistemology, the construction of an episteme suitable for an intragalactic cognitive system.
The episteme discussed here is concerned only with propositional knowledge, which may be characterised as knowledge \emph{that} (knowing the truth of propositions) rather than knowing \emph{how} (having some skill, or nohow).

The distinction between knowledge and opinion is addressed obliquely here; logical coherence and deductive soundness are fundamental concerns.
The latter concern is made more critical because I presume that the deductive capability of the system is very high, and that any infelicity in the logical system will quickly lead to false and incoherent conclusions.

I begin by lightly sketching the scenario for which the episteme is intended.
The construction then takes place in stages corresponding to levels in an abstract model of the representation of propositional knowledge by the system.

In explaining and justifying some of the choices made, appeal will sometimes be made to the history of the development of philosophy, mathematics and science.

\section{The Scenario}

I imagine that \emph{homo sapiens} has suceeded in undertaking interstellar travel, and that self-proliferating intelligent systems have spread over a substantial portion of the Milky Way galaxy, perhaps a hundred thousand light years from planet earth at some points.

I imagine that there has been some kind of earthly sponsored space race, and that the bulk of this region of our galaxy is mainly occupied by the kinds of self-proliferating intelligent system which are most effective in proliferating themselves across the galaxy.
This means that they are:
\begin{itemize}
\item not human
\item technologically very advanced
\item dedicated to advancing the knowledge necessary for rapid proliferation across the galaxy
\end{itemize}

The life cycle of these systems runs something like this.
Let us begin the cycle with one of the largest (goegraphical) leaps of which the system is capable, i.e. a basic system has been sent over a large expanse of space into some region devoid of self-proliferating intelligent systems.
There may already have been intelligent probes which have contributed to the decision about where this `seed' should be directed, but on arrival in the inteded area there is some scope for choosing the best place to start development.

The system comes with enough knowledge to make a start, but will need to import knowledge on a large scale from its nearest neighbours in order to develop to the point at which it is capable of sending out a simliar seeds to some yet further destination.
The first task therefore is to begin extracting from its environment the raw products necessary to build out information and communications infrastructue, so that it becomes a functional node in the intragalactic cognitive network.
The information technology being deployed at this stage will be quite unlike any that we find here on earth, since sending or building a \$10B semiconductor plant would not likely be the best way to go.
Much later in the development of the system, that might be feasible, but not yet.

\section{Logical Truth and Truth Conditional Semantics}

The architecture of knowledge which I propose is layered.
The lowest layer is one of logical truths, on which all else is built.

In talking of logical truth we are concerned with \emph{propositional}, or \emph{declaative} language, which may be used to communicate about various subject matters.
The subject matters which may be addressed are called the domain of discourse of the language.
A propositional language will provide expressions which may be used to refer to things in its domain of discourse, and formulae or sentences which express propositions about those things.
A proposition is something which has a truth value depending on exactly how things are in the domain of discourse.
The conditions under which a sentence is true are called its \emph{truth conditions}, which are (perhaps implicitly) known to anyone who understands the language.
A sentence can therefore be used to convey information about the domain of discourse between people who have a common understanding of the truth conditions of the sentences.
The assertion by one party of a sentence communicates to other parties that the domain of discourse satisfies the conditions for truth of the sentence.

We use the term \emph{semantics} to refer to this aspect of the meaning of a language.
For the purposes of logic, it is the truth conditions only which concern us, but these attach only to sentences in the language.
Other expressions in the language which refer, for example to entities in the domain of discourse, have their own characterisation in the semantics which determines their role in setlling the truth of the sentences in which they occur.

Logical truth is a semantic notion, it is a property which attaches to certain sentences in languages with well defined truth conditions.
Truth conditions are a part of the meaning of sentences in a propositional language, sentences which express propositions.

\subsection{defining logical truth}

By \emph{logical}\index{logical} truths I mean those truths which follow deductively from the definitions of the concepts which appear in them.
Reasoning is deductive if the inferences involved are justifiable by reference only to certain aspects of the meaning of the sentences involved in the inferences.

The concept of logical truth and the above characterisation of it are controversial.
Much of the controversy is verbal, i.e. it is about the choice of language and the meaning of the terms.
Most of the technical claims that will be made below are not controversial, though once again the choice of language in which they are expressed may be.

An epicenter of opposition to the position which I here adopt relates to the thesis of \emph{logicism}, a position held by Gottlob Frege, Bertrand Russell and Rudolf Carnap, and arguably attributable retrospectively to David Hume (though not identically expressed by each), the essence of which is that the truths of mathematics are logical truths.
Despite the eminence of all these philosophers, the thesis has been regarded by the majority of philosophers since the middle of the 20th Century as discredited.

The simplest explanation for this is that the concept of ``logic'' became narrower durubg the first half of the 20th Century as the discipline of Mathematical Logic became established.
Early in this period problems arose in the status of abstract ontology, aspects of which at first had appeared uncontroversial and logical in character, but turned out to be perilous, and to some degree arbitrary.
This was first exposed, in the context of the establishment of logicism, by Russell's observation that Frege's formalisation of arithmetic fell foul of what came to be called ``Russell's paradox'' (though it, or very closely related problems,  was already known to Cantor) and could not therefore be relied upon to prove only true statements.

The resolution of Russell's paradox depended upon constraining the principles governing what abstract entities exist, and there seemed to be no wholly convincing argument to determine which restrictions should be adopted.
Seemingly arbitrary choices were needed to establish a coherent ontology.
It was therefore natural to doubt that the ontological principles required for the definition of number and the derivation of the truths of arithmetic could have the character of necessity expected of logical truths.
Against this is may be observed that any language in which arithmetic can be expressed makes use of arbitrarily chosen symbols for the logical and arithmetic operations whic appear in the sentences of arithmetic.
Langauge is conventional, and this is no bar to its ability to express logical truths.

Abstract ontology, by contrast with concrete ontology, corresponds to no observable feature of the world about us.
Some philosophers take the view that abstract entities are mere fictions, and that would provide an adequate basis for the matters discussed here.
However, fictions are usually the kind of thing which one would normally be able to detect empirically, but in fact do not exist, whereas abstract entities are not of that kind, the claim that they exist is not empirically testable.
Questions involving abstract entities only become meaningfull in a context which settles the abstract ontology, amd their resolution depends on reference to that context.

This attitude towards the status abstract entities is similar to that of Rudolf Carnap, who regarded ontological questions as being of two kinds.
The first kind are those which are taken in the course of defining some language, and subject to requirements of consistency are as freely chosen as any other feature od the language (though not without consequences for the utility of the resulting language),
The second kind are those which are posed in the language thus defined, and are to be resolved by reference to the choices made in designing the language, ideally aided by a reasonably complete formal deductive system which supports reasoning within the chosen domains of discourse.

\subsection{basic layer continued}

This most basic layer is also the layer which determines the \emph{abstract semantics} of the vocabulary for this and all other layers in the structure.
Abstract semantics is a technical term related to the intended meaning of the linguistic structures it relates to, and contains sufficient information to determine whether

Before giving more detail about the structure of this first layer, its necessary to explain the concept of logical truth as it is intended in this architecture.
For this purpose a bit of history may be helpful.

\subsection{Some History}

The first known systematic use of deductive reasoning was its use by the philosophers of ancient Greece for the development of mathematics and most successfully in Euclidean Geometry.
At the same time philosophers sought to understand nature and the cosmos using reason and found it to be unreliable and incapable of securing concensus in these matters.
The great philosophical systems of Plato and Aristotle attempted to resolve this descrpancy and promoting and underpinning  the use of reason more broadly than had hitherto succeeded.
In doing so they made the first steps toward identifying the kind of ``logical truth'' whicxh concerns us here.

The first approach to this came in Plato's distinction between his world of abstract forms and that of mere appearances.
The former domain was one which could successfully be comprehended by reason, in whicha domain reason could yield conclusive knowledge.
The world of appearances, of which we learn through our senses, cannot yield certain knowledge, but only mere opinion.
This is a division by subject matter, abstract versus concrete, and by method of enquiry, by reason and by observation, which also distinguishes the degree of trust which can be assigned to the resulting conclusion (certain knowledge versus subjective opinion.
The former domain corresponds to the notion of logical truth as it is intended here, but fails to give a sufficiently precise characterisation of the domain for our purposem.

Aristotle's philosophy sought to resolve some of the difficulties in Plato's ideas, and included six volumes on logic in which an important aim was to establish the notion of \emph{demonstrative science} and the deductive methods appropriate to it.
Many important logical concepts which are closely related to that of logical truth, such as the distinction between essential and accidental predication, that between necessary and contingent propositions, and the notion of demonstrative propositions.
Aristotle's theory of categories provides a way of distinguishing the kind of conceptual inclusions which arise in a taxonomical heirarchy, and hence may be thought essential or to do with meaning, from those between concepts related only by accident.
When we come to Hume, this distinction shifts more decisively away from doubtful metaphysical interpretation to one more easily understood to be concerned with meanings.

Much much later it is the articulation by David Hume of what has been called ``Hume's fork'' that provides the first unproblematic (if still lacking in precision) description of the distinction which is here important.
In a central place in the condensation of his philosophical position as his ``An Enquiry Concerning Human Understanding'' \cite{hume48} Hume speaks of the two kinds of proposition in the following terms:

\begin{quote}
``ALL the objects of human reason or enquiry may naturally be divided
  into two kinds, to wit, Relations of Ideas, and Matters of Fact.'' 
\end{quote}

The first kind, which we call here the truths of logic, he further describes thus:

\begin{quote}
``Of the first kind are the sciences of Geometry, Algebra, and
Arithmetic; and in short, every affirmation which is either
intuitively or demonstratively certain.
That the square of the hypotenuse is equal to the square of the two
sides, is a proposition which expresses a relation between these
figures.
That three times five is equal to the half of thirty, expresses a
relation between these numbers.
Propositions of this kind are discoverable by the mere operation of
thought, without dependence on what is anywhere existent in the
universe.
Though there never were a circle or triangle in nature, the truths
demonstrated by Euclid would for ever retain their certainty and
evidence.''
\end{quote}

I consider the first accurate characterisation of the domain of logical truth, of which are much more detailed technical definition will follow.
You may come to you own opinion once that precise delimitation is in place, though the strength of the correspondence is not essential to the resulting episteme.

Emmanual Kant, who credited Hume for waking him up from a dogmatic slumber disagreed with Hume's characterisation of this split, and specifically disagreed that mathematics belongs in Hume's first classm, introducing a new use for the technical term `analytic'.
The term `analytic' sounds rather like Hume's relation bewteen ideas, but Kant denied that the truths of arithmetic are analyic.
This denial was to prove a spur to the demonstration of the analyticty of arithmetic, the results of which provide the basis for the bottom layer of this architecture.

To give precision to the concept of logical truth (later to be identified with analyticity)  we must look primarily to the mathematicians of the century which followed Hume.

\subsection{Mathematical Logic}

Though mathematics had progressed as a deductive science since the ancient Greeks, the gold standard of rigour as exemplified by Euclidean geometry had not been maintained.
The shortfall had been particularly marked since the invention of the differential and integral calculus by Newton and Leibniz, which had precipitated a huge expansion of mathematics applicable to science and engineering despite a lack of clarity about some of the key concepts involved, notably that of infinitesimal numbers.

In the nineteenth century mathematicians began a process of rigourisation of analysis to put that discipline built on the ideas of Newton and Leibniz on a solid footing.
The first stage in this was to eliminate the use of infinitesimals, using the notion of limit to achieve the same ends.
This required a system of numbers in which limits exist for every convergent series, these were to be called `real' numbers.
Also needed was a domain of mathematical functions which included all the functions likely to arise in this expanding new branch of mathematics.
These issues were resolved by developing \emph{set theory}.
With the aid of set theory the rigourisation of analysis was achieved by arithmetisation, its reduction to the theory of whole numbers.

For the purposes of mathematics that probably would have sufficed, but Gottlob Frege had a bee in his bonnet about Immanual Kant's take on the status of Mathematics, which he claimed did not belong to logic, or in his special terminology, was not \emph{analytic}.
Frege set about demonstrating that ``Arithmetic is Analytic'', one formulation of a thesis which later came to be called ``logicism'', and was often rendered ``Mathematics is (or is reducible to) logic''.
To demonstrate this Frege devised a new formal logical notation which he called \emph{Begriffsschrift}\index{Begriffsschrift} (concept script).

Though intended for the rigorous formal derivation of the truths of arithmetic, Frege did not (as his title suggests) think it confined to that purpose, considering it generally appicable to rigorous formal reasoning.
His description of this notation was published in 1879 \cite{frege1879}, and represented the first major advamce in formal logic since Aristotle formulated his syllogistic.
The limitations of Aristotle's syllogistic were entirely overcome by this new notation, which was as Frege hoped suitable generally for the formalisation of deductive reasoning.

\subsection{Deductive Closure and Consistency}

It is reasonable to expect that the very large scale deductive inference which would be feasible in galactic scale distributed synthetic intelligence will give effective access to the najor part of the deductive closure of the knowledge in its possession.
Consequently it is highly desirable that those principles be logically coherent, otherwise that deductive closure will contain all propositions, true or false.

For this reason the principal method of augmenting the logical layer of our knowledge base will be ny conservative extension, and careful tracke will be kept of non-conservative elements and the conclusions drawn from them.

This remains the case as we ascent to higher layers in the architecture and consider the representation of empirical theories.
It is therefore intended that empirical theories will be couched as defined mathematical theories from which various applicable instances may be deduced.

\subsection{Logical Truth as General Database}

All ways of storing information require some way of referring to that information.
Typically that consists of names or positions, either of which can be interpreted in a logical system constraining heirarchical names.
We do not speak here of how the data is actually stored, just of how it is interprted logically for the purposes of inference.
The computations which one might undertake on these data structures yield logical truths expressing the result of the computation.


\section{Empirical Truth}

The second level of our architecture is concerned with the representation of empirical truth.
For the purpose


\subsection{Factorisation of Semantics}

Sentences expressing empirical propositions have as their subject matter the concrete world rather than abstract entities, though they may nevertheless refer to abstract entities (such as numbers) in the course of expressing an empirical claim.

Physical theories are commonly expressed in mathematical form, and by virtue of their form may be considered to offer an abstract (mathematical) model of a concrete phenomenon.
Reasoning about concrete reality may then be accomplished by logical reasoning about the abstract entities of the mathematical model, considering these entities to have a concrete as well as an abstract interpretation.

If we seek a truth conditional semantics for a language which is intended to speak of the concrete world, it is therefore possible to make use of an abstract model of the concrete phenomemon which would be appropriate for a mathematical model of the astract phenomenon.
The correspondence between the concrete entities concerned and their abstract representatives provides a factorisation of the truth conditional semantics, effectively defining the truth conditions of empirical sentences as those of its abstract representative.

This makes it possible and intelligible to use a purely logical substrate to represent empirical knowledge, provided that it is supplemented by an adequate description of the correspondence between the abstract and concrete entities.
In an axiomatic description of a physical system the physical laws would be presented as axioms.






\ignore{
\begin{quote}
``Matters of fact, which are the second objects of human reason, are not ascertained in the same manner; nor is our evidence of their truth, however great, of a like nature with the foregoing. The contrary of every matter of fact is still possible; because it can never imply a contradiction, and is conceived by the mind with the same facility and distinctness, as if ever so conformable to reality. That the sun will not rise to-morrow is no less intelligible a proposition, and implies no more contradiction than the affirmation, that it will rise. We should in vain, therefore, attempt to demonstrate its falsehood. Were it demonstratively false, it would imply a contradiction, and could never be distinctly conceived by the mind.''
\end{quote}
}%ignore

\appendix
\pagebreak
\phantomsection
\addcontentsline{toc}{section}{Bibliography}
\bibliographystyle{rbjfmu}
\bibliography{rbj2}

\addcontentsline{toc}{section}{Index}\label{index}
{\twocolumn[]
{\small\printindex}}

\vfill

\tiny{
Started 2022/11/12

\href{http://www.rbjones.com/rbjpub/www/papers/p047.pdf}{http://www.rbjones.com/rbjpub/www/papers/p047.pdf}

}%tiny

\end{document}

% LocalWords:
