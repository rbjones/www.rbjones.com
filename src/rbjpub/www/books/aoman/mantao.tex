% $Id: mantao.tex,v 1.1 2008/07/11 18:16:09 rbj Exp $

\section{Tao}\label{Tao}

\paragraph{version 2}

I don't think I knew anything about {\it the Tao} when I came up with my {\it personal anarchism}, though I might well have come across Zed Buddhism, which I now understand \footnote{from Raymond Smullyan \cite{smullyan77}} to have roots in both Buddhism and Tao.
When much later I came across the Tao, there seemed much in it which connected with my own ideas, as well as a quite a bit that didn't.

The connection was with some of the ideas about how to live.
The Tao builds around some nice ideas about life, a whole religion, speaking as if the Tao were some thing (or perhaps everything) and making the achievement of some mystical union with the Tao into an ultimate aim of life.
These appearances may be deceptive.

When you start learning to play the piano, you are likely to be given some simple rules to observe.
e.g. don't play the black notes with your thumb.
When you get a bit more advanced, for example if you look at Chopin's \'{e}tudes you find that the rules have to be broken.
The rules are in fact useful.
When you begin the music you play will be easier to play if you stick to the rules.
Later on you learn to break them when necessary.

Writings on the Tao are similar.
You are told that the Tao cannot be described, in a work which seems to have no other purpose.
You are told of {\it yin yang} the unity of opposites, which seems awfully like the doctrine that whenever you think you have an important truth you will find that its converse is also true.

\paragraph{version 1}

I'm going to refer occasionally to ideas which come from the ancient Chinese philosophy called {\it Taoism}.
I am not myself a Taoist, but my own philosophy which will be presented here contains some elements which seem to me similar to Taoist ideas, and reference to the Tao may help me to explain it.
I do not have a scholarly knowledge of Taoism, so it is certain that what I say about it will be not quite right, but I hope it will nevertheless be helpful in explaining my own philosophy.

Three ideas from the Tao will be helpful here in describing the character of this work.
They are {\it wu wei, yin-yang} and {\it te}.

Sometimes one can try too hard, and this interferes with your performance and prevents you from realising your goal.
Sometimes it would make sense to advise someone to try less hard {\it wu-wei} may be thought of as taking this to an extreme.
It is the idea of achieving things without effort, or better perhaps, spontaneously.
It is the antithesis of command and coercion.

The idea of {\it yin-yang}  