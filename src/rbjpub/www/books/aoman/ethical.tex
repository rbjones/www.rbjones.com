% $Id: ethical.tex,v 1.2 2008/06/26 19:00:59 rbj Exp $
\section{Ethical Anarchism}\label{Ethical}

Ethical systems present a challenge for anarchism because of their potential to introduce intolerance and coercion.

However, these are not essential characteristic of ethical systems.

It might appear that adoption of {\it personal anarchy} as a {\it philosophy for life} would make one a moral nihilist.
But this is not the case.
Not necessarily.

If you have an ethical code and you behave badly, and you feel that what you did was wrong, then you are in Christian terms a repentant sinner.
You will intend to behave better next time.
Maybe you will, maybe not.

What I am going to propose here, and what I am calling {\it ethical anarchism} is like a {\it meta}-ethic.
This is the top level in a multi-layered value system.
The meta-ethic consists in the idea that one should adopt an ethical system, using that term in a very loose sense, and try to behave according to that system.

So I'm saying that I think people should do what they think is right and not do what they think is wrong, and further that they should give a reasonable amount of thought to these things.

I'm not saying what ethical system people should adopt, nor am I saying how they should chose which system to adopt, except that they should not be too casual about it.

I'll expand a bit on how we get our moral ideas shortly, but first I want to say something about the significance of this layering of value systems.
Its natural to think of this as the most fundamental requirement, and as taking precedence over less fundamental principles.
But, though I do say you should adopt a system and act in accordance with it, I am not saying that I will approve of what you do.
You will get credit from me for acting according to you principles, but that won't stop me from condemning you if I disagree with your principles.

This is a bit like the way in which ``personal anarchism'' is a layering.
As a personal anarchist I say to myself, after due deliberation, do whatever feels right.
But that doesn't mean that when I make a mistake I can't recognise that I have made a mistake, do my best to correct it and hope that the next time my judgement is better.

Now a few more words about what kind of thing ethical anarchism will count as a moral system and how we can come by such a thing.
Firstly it doesn't have to be a codified set of moral principles.
In fact better not, for I am inclined to say that one should exercise at all times ones moral sense, and test principles in the context of application to see whether the circumstances are special enough that an exception to the principle is necessary.
So I believe that we most of us come with some sort of built in moral compass which does not consist in some concise rule book, and that there is a process we have to go through which is not purely rational to decide whether something is right or wrong.
Very often people do not like the responsibility which is implicit in this, they want to be able to refer to the rule book so that their moral pronouncements are not personal judgements.

This desire to depersonalise ethics I reject in two ways.
Firstly by insisting that the individual must chose the ethical system.
He may just chose to follow some established system, that prescribed by some religion of which he is a member for example.
Nevertheless, he has chosen to follow that religion, with its moral code, and even though he did not devise the moral code, he chooses to follow it.
The second respect is that I believe that the continual reference to something which is often called {\it conscience} is necessary, that following mechanical prescriptions is a kind of abdication of moral responsibility.

How do I imagine people will acquire ethical systems?
Well I'm not too sure.
My belief is that we start with some genetic material, and then absorb a more or less of the cultural context in which we are raised, and then as we mature we may come to question some or all of this, and our ethical values evolve as a result of thinking things through, observing and experiencing life, and discussing these things with others.
And of course, for many people spiritual guidance will be important.

So there is really, so far very little constraint here on what I am inclined to accept as being an ethical system.
