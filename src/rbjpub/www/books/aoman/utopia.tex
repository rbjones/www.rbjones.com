% $Id: utopia.tex,v 1.1 2008/06/26 19:00:58 rbj Exp $
\section{Anarchic Utopia}\label{utopia}

It is now widely recognised that to proceed to a desirable end, by means which are inconsistent with that end, is unlikely to be successful.

Here I sketch aspects of a society which is in transition toward an ideal anarchic state.
This is not a transition which we can expect ever to be completed.
It will fail to be ideal partly because our ideas of what is idea will advance (and recede into the future) as we approach them.
Partly also because the idea that the ideal society is anarchic is one which will not be shared by all, and which can only be realised by common consent.
We therefore seek advances which have merit in their own right which can be appreciated independently of any general ideological commitment to anarchy.

The following are the primary elements in the transformation:

\begin{itemize}
\item elimination of taxation
\item elimination of criminal law
\item elimination of civil law
\item replacement of hierarchies by networks
\end{itemize}

The general plan is that every individual and group in society should have a view about how they would ideally like society to be, about how the present situation falls short of that ideal, and should maximise the impact of their actions in promoting the desired changes.
An important part of this action will be promotional, the first step is to make clear where you stand.

Various changes are in progress which make it easier for individuals and groups to be proactive in this kind of way, which will make society more responsive to popularly perceived need for change.

Several general trends seem to me desirable.

The most obvious trend is for the state to do less, and for that to happen we need to have alternative ways of realising many of the things which are now accomplished by the state or using tax revenues.
Anarcho-capitalists may contemplate with equanimity the abandonment of all the functions currently funded by the state, trusting in the free-market to provide these services for those who can afford to pay.
Social anarchists may seek to make these provisions by voluntary cooperation in a context in which goods are distributed by means other than capitalist free markets.

In our case the free market in goods, labour and capital is to be retained and advanced, and though some services provided at present by the state will simply be privatised, and some discontinued or curtailed, a major expansion is expected of the non-profit sector, of funding by gift or grant or of provision of goods and services for the public benefit.
In this system, rather than eliminating income differentials, we hope that most people will have income surplus to their requirements, and that compulsory taxation will be gradually replaced by charitable donation.

