% $Id: personal.tex,v 1.4 2008/07/11 18:16:10 rbj Exp $
\subsection{Personal Anarchism}\label{Personal}

Personal anarchism is a lightweight {\it philosophy of life} which I adopted as a young man.

Round about my 22nd birthday I was working in an unfamiliar place with few friends, no woman, plenty of hormones.
From time to time I would get a bit depressed.
Depressions are often not very informative.
You know something is wrong, but you are left guessing about exactly what the problem is.
What I tend to do is think a lot, focus on anything that seems wrong with my life and see if I can find a way to fix it.

At this time I must have been doing a lot of this, because I was going through a lot of self-devised ``-isms'' schemes for putting my life in order, of which I am going to mention just two apparently contradictory schemes, ``rationalism'' and ``anarchism''.

I don't suppose anyone will be surprised to hear that I often thought through my problems, devised a scheme for fixing them, and then failed to implement the scheme.
``rationalism'' was the meta-scheme that one ought not to do that.
It seemed that there was no hope in life if, when one considered matters very carefully, thought them through and decided on the best course of action, if one then failed to act on the conclusions.
So I thought to fix this by attaching great significance to the general principle that one should carry through, regardless of how trifling the particular issue at stake might be.

I should perhaps say here, in mitigation, that there was no presumption that the deliberations involved should be particularly logical or rational.
Any amount of emotion, perhaps even the toss of a coin, might have played its part in determining the outcome of my deliberations, the only point on which my ``rationalism'' insisted was that once my mind was made up, I should get on and implement my decision.
It was just the supreme irrationality of making a decision and then not acting on it which I sought to expunge.

``anarchism'' when it came to me, was just the converse of ``rationalism''.
The denial that we should expect conscious deliberations and the conclusions which they reach to rule the roost.

Presumably I had been reading something about anarchism that I should have chosen that name, probably a collection of readings edited by Irving L. Horowitz which I acquired about that time.

There is no rocket science here, its all pretty elementary.
My ``rationalism'' is self-discipline.
The idea is that conscious mental processes should be ``in charge'' of what we do, but that there are some other aspects of our mental world which are a bit unruly and don't always cooperate.
What you have to do is use your will-power to force compliance.
``rationalism'' required that I routinely impose my decisions on the rest of my mind.

The contrary idea is that the mind is a community of cooperating mental processes and no-one is in charge.
Conscious intelligence is an important element in the mix, but is liable to be ignorant about what the game of life is really about, and for that reason may come up with ideas about what should be done which will not fly.
What we end up doing is something which emerges from these diverse mental processes and dispositions in ways that we don't understand.
It doesn't help for one part of the mind to try to coerce the whole into implementing its agenda.

So this ``personal anarchism'' endorses what actually happens.
It doesn't deny the importance of thinking things out.
This will often make a big difference to what actually happens.
But the idea is you think things out, and then you do whatever feels right.

There is almost nothing here.
I say to myself, do what you will.
What else could I have done?

Oddly enough, this almost wholly vacuous personal philosophy was a terminus for me.
No further -isms were forthcoming, no further searching for a rule for life.
For the next forty years I thought only occasionally on such matter, usually when things were going wrong.
In such circumstances, in moments of doubt, I simply re-affirm to myself the idea that I cannot do better than my best, and that the best I can do is to carry right on doing what I do.

Though that could easily be misleading.
The idea is not that when you make a mistake you pay no account, don't learn from it, and just go right on making the same mistake over again (though sometimes that seem to be just what I am doing).
There is no intention to curtail the post mortem, no intention to prevent that post mortem from transforming the way you approach similar situations in the future.
All it does is mitigate the despair.

There are different layers going on here, this relates to how we are told to deliver feedback to others.
Criticise the behaviour not the person.
In this case the person is yourself, and the idea is partly to prevent the perception of failure from becoming fundamental self-doubt, and to avoid fear of failure from inhibiting you from doing whatever it is that you are about to do.
Also, to look for teamwork, not to expect intellectual considerations to take precedence over feelings.
After a thorough analysis one may think that there appear to be conclusive reasons for some course, but nevertheless feel that it is the wrong thing to do, or that it is the wrong thing for {\it you} to do.
It is the feelings which count at the end, if the arguments have enough weight then they will change the way you feel, if not then they should not carry the day.

This makes it sound a bit more static than it was, for through these years from time to time the world would find new ways of testing my resolve, of engendering self-doubt, and each time this empty doctrine of personal anarchism would somehow richen and make a more solid base for me.







