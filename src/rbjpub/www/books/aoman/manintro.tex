% $Id: manintro.tex,v 1.1 2008/06/10 20:17:56 rbj Exp $
\section{Introduction}\label{Introduction}

Philosophy in its most general sense may be thought of as falling into two parts, theoretical and practical.

Theoretical philosophy is concerned with knowledge and understanding, of man, of the universe in which he finds himself, and of his place.
In ancient times this would encompass much of what we now consider to be science rather than philosophy, and still today we may refer to science as ``natural philosophy''.

Practical philosophy is concerned with action.
What to do?
Insofar as we may arrange things to suit ourselves, how should that be?

Anarchism is a doctrine about how certain things should be done, and as such, belongs to practical philosophy.
It is, according to my dictionary, ``the doctrine that all government should be abolished''.
Here however, it is used for a more general abhorrence, not specifically of government or the state but of compulsion or coercion in general, particularly of physical compulsion or of violence.

The slogan which gives our title originates with Kropotkin.
It comes from the insight that coercion employed by the state is a source of rather than a cure for disorder, and that the abolition of the state, rather than resulting in a collapse into disorder, will permit man's natural cooperative spirit to flourish and result in a affairs being better ordered.
The slogan is adopted here, to express my belief that compulsion and coercion are not only intrinsically objectionable but also inefficient.

This document is a manifesto, it provides a proposal for the transformation of that corner of the universe which is in our sway.
It is also a prospectus, for a book to follow, which will be my next contribution to that transformation.

The manifesto and the book in preparation are both presented as the outlines of a philosophical weltanschauung.
I would like to say a few words here to explain what in this case that means.

Most philosophy today is written by academics, who aim to deliver work meeting ``academic standards'', which are intended to ensure that the work is authoritative.
This work has an entirely different character.

I am not myself an academic, and so am not best qualified even to talk about academic standards, let alone deliver them, but I will sketch first what I imagine academic standards to involve and then describe in contrast the nature of the present enterprise.

Them aim of academic standards is to ensure that a piece of academic work is undertaken with the greatest possible rigour so that the results of the work can be taken as authoritative.
The idea is that what an academic publishes should not merely be an expression of his opinions on various topics, but should be an objective statement of a certain body knowledge which he has reliably established in his research.

The work we are presenting here addresses practical and theoretical problems on the grandest possible scale, and of the greatest difficulty.
There is no likelihood of a definitive resolution.
There are some problems which however difficult they may be, and however uncertain our conclusions will be, must nevertheless be addressed.
What to do, to secure the best possible future for our children, is one of them.
This involves not only guessing about the likely outcomes of our present choices, but also making decisions about what outcomes are and are not desirable.
It is my belief that the method implicit in academic standards is not adequate for addressing such problems.














