% $Id: manintro.tex,v 1.3 2008/07/11 18:16:09 rbj Exp $

\section{Introduction}\label{Introduction}

This essay is intended to be in some respects like a political manifesto.
Its principal aim is to change the world by describing how the world might be and putting forward some ideas on how we might make it so.
It is however, anarchistic, and so you should think of it as {\it anti-political} in intent.

To expand on this a little, a political manifesto would typically put forward a proposal for action to be taken by some political party subsequent to securing political power, and perhaps would say something about how such power would be achieved, which might be by contesting a democratic election or by violent overthrow of an autocratic regime.
In this case however, there is no party, and the envisaged transformation of society results primarily from changes to the attitudes and actions of ordinary people, rather than those of political representatives.

There are some similarities between the anti-political doctrine of anarchism and the ancient Chinese philosophy known as {\it Taoism}.
In some respects this document may be more like {\it Tao} than a western political manifesto, but still some way off.
I will neither be providing an account of the {\it Tao} (I could not), nor an example of it, but sometimes I will find it useful to refer to elements of {\it Tao}, as I understand it, in presenting my perspective.
To those knowledgeable about Tao this may seem like abuse, and to them I apologise.

The main thrust philosophically (and the main thrust is philosophical) remains western in character. 

\paragraph{}

It is the thesis of anarchism that a just and equable society might be organised in which individuals are not subject to coercion or compulsion, and that such a society would be preferable to those we have at present.

It is the purpose of this manifesto to sketch one way in which such a society might be arranged.
It is called a manifesto because it provides a plan of action, but it is also a prospectus for a longer work to follow on the same topic.

My contribution to the topic is intended to be philosophical in character, but the philosophy is pragmatic and is intended to provide an underpinning for the particular social ideas presented.
This is not to say that my conception of philosophical truth is relativistic, but rather to recognise that we may exercise choice not only in the form of our political institutions, but also in the language and conceptual framework through which we seek to understand and organise for our purposes the world we live in.

Anarchist thought has predominantly been a libertarian socialism, and has been antagonistic to capitalism.
Latterly however there has emerged anarcho-capitalism in which property rights are highly valued and the redistribution of wealth is not encompassed (though not actually proscribed).

This manifesto falls between these two traditions, in wishing to retain a free market economy while seeking welfare and fulfilment for all.
It therefore envisages a special greatly expanded role for non-profit entities, and an economy in which a high proportion of monetary transfers are not {\it for profit}.

A primary challenge which faces social anarchists is whether their social ideals can be realised without compulsion.
It is a principle objective of this manifesto to provide an answer to this challenge.

Philosophical discussions about political institutions usually depend upon some conception of human nature and/or of primitive society.
This may be used either to justify authoritarian institutions or to argue against such institutions.

\subsection{}

Philosophy in its most general sense may be thought of as falling into two parts, theoretical and practical.

Theoretical philosophy is concerned with knowledge and understanding, of man, of the universe in which he finds himself, and of his place.
In ancient times this would encompass much of what we now consider to be science rather than philosophy, and still today we may refer to science as ``natural philosophy''.

Practical philosophy is concerned with action.
What to do?
Insofar as we may arrange things to suit ourselves, how should that be?

Anarchism is a doctrine about how certain things should be done, and as such, belongs to practical philosophy.
It is, according to my dictionary, ``the doctrine that all government should be abolished''.
Here however, it is used for a more general abhorrence, not specifically of government or the state but of compulsion or coercion in general, particularly of physical compulsion or of violence.

The slogan which gives our title originates with Kropotkin.
It comes from the insight that coercion employed by the state is a source of rather than a cure for disorder, and that the abolition of the state, rather than resulting in a collapse into disorder, will permit man's natural cooperative spirit to flourish and result in a affairs being better ordered.
The slogan is adopted here, to express my belief that compulsion and coercion are not only intrinsically objectionable but also inefficient.

This document is a manifesto, it provides a proposal for the transformation of that corner of the universe which is in our sway.
It is also a prospectus, for a book to follow, which will be my next contribution to that transformation.

\subsection{Methods and Standards}

The manifesto and the book in preparation are both presented as the outlines of a philosophical weltanschauung.
I would like to say a few words here to explain what in this case that means.

Most philosophy today is written by academics, who aim to deliver work meeting ``academic standards'', which are intended to ensure that the work is authoritative.
This work has an entirely different character.

I am not myself an academic, and so am not best qualified even to talk about academic standards, let alone deliver them, but I will sketch first what I imagine academic standards to involve and then describe in contrast the nature of the present enterprise.

Them aim of academic standards is to ensure that a piece of academic work is undertaken with the greatest possible rigour so that the results of the work can be taken as authoritative.
The idea is that what an academic publishes should not merely be an expression of his opinions on various topics, but should be an objective statement of a certain body knowledge which he has reliably established in his research.

The work we are presenting here addresses practical and theoretical problems on the grandest possible scale, and of the greatest difficulty.
There is no likelihood of a definitive resolution.
There are some problems which however difficult they may be, and however uncertain our conclusions will be, must nevertheless be addressed.
What to do, to secure the best possible future for our children, is one of them.
This involves not only guessing about the likely outcomes of our present choices, but also making decisions about what outcomes are and are not desirable.
It is my belief that the method implicit in academic standards is not adequate for addressing such problems.














