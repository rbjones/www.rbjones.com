% $Id: mangod.tex,v 1.4 2008/07/11 18:16:11 rbj Exp $

\subsection{On The Existence of God}

\paragraph{version 2}

The first and least practical problem was that of the existence of God.
I came to think about this problem, not out of any intrinsic interest it might have had for me, but because when I began to attend secondary school at the age of 11 I was obliged to attend church services, was subjected to regular lengthy (so they seemed) sermons, and could not really begin to understand these sermons without having some conception of what God is.

This tells you a little about my character.
I am not the kind of person who naturally takes things on trust.

Anyway, my attempt to understand God was doomed, I failed, and from this failure I concluded that God does not exist.
Was I distressed by this?
Not at all.
I did have a bit of trouble understanding how so many distinguished, important, intelligent people could falsely believe in God, so that caused my to struggle for an intelligible conception of God for longer than I might otherwise have done.

I spent less than a year thinking about God before deciding he did not exist, and since then I have never really given the matter any further consideration.

So I'm not the kind of guy who needs to have a God or a religion.

\paragraph{version 1}

At the age of 11 I was sent to grammar school as a boarder.
On Sundays, attendance at a church service was compulsory, and this usually involved sitting through a sermon.
Naturally I tried to understand what the Vicar was telling me, and since God was often mentioned, this made me wonder about who or what God might be.

I had great difficulty figuring this out, the things we were told about God really didn't make a lot of sense to me.
I tried out a few ideas, about what kind of thing God might be, but none of them worked.
None of them accounted plausibly for the attributes which God was supposed to have.

I don't recall any of the detail, or how long it took me to conclude my cogitations, but by the beginning of my second year, when called upon to make a decision about whether to be ``confirmed'' in my faith, I then quite definitely did not believe in the existence of God, and therefore declined.

In this process, I have no recollection of consulting any book or person, I don't think it occurred to me to seek advice.
I wasn't hard to figure out what they would have said.
The problem was whether I could make sense of the idea of God, and if not, then the question of his existence did not arise, he was a nonsense.
I didn't take seriously the idea that He might be too mysterious for me to understand (but still very real).
It would have been OK for him to have complicated corners that I couldn't get my head around, so long as the basic idea made some kind of sense.

The greatest stumbling block in my giving up the search for meaning and settling for disbelief was the huge number of people, very respectable and important people, who professed belief in God.
Was it possible that they were all wrong?
Well I decided they were.

So this eleven or twelve year old boy, after a bit of a tussle, preferred his own untutored judgement to that of about half a world of his betters.
This is symptomatic of a character trait, independence of mind, and a disrespect for authority, which makes me by nature rather than by conviction, a kind of anarchist.

This isn't the whole of it, and I will touch upon other aspects in time, but I just want to use this to make a couple of points here.
Firstly, I don't imagine that many other people are the same, so my ideas about how I would like the world to be are unlikely to be widely shared.
Secondly, this isn't the kind of difference you can get over with an enlightening conversation.
It's a fundamental character trait.
It's conceivable (if improbable) that I might persuade someone that the idea of God makes no sense, and hence that he cannot exist.
But to persuade someone who looks to his betters for answers in these matters that he should instead just think it through and come to his own conclusions?
Not much hope.

Now this was a bit of {\it naive} philosophising.
There was no scholarship involved.
I did not study the literature, examine the arguments, sift out the good from the bad and determine the true answer.
Nor have I ever since been inclined to pay much attention to arguments about whether God exists or not.
The idea of God didn't make sense to me, you can't even begin to argue about whether he exists until you can make sense of the idea.

% ...