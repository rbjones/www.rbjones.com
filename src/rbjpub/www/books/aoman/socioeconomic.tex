% $Id: socioeconomic.tex,v 1.2 2008/06/26 19:00:59 rbj Exp $
\section{Socio-Economic Anarchism}\label{Socio-economic}

I propose in this section to consider how a typical western ``mixed economy'' (i.e. one in which many social needs are met through a capitalistic free-market economy, but some are provided in part or in whole through taxation and public expenditure) might evolve in the future.

Before we can do this we need to have some idea of what we are trying to achieve.
I'm going to put forward some desiderata which I hope will make sense not just anarchists but to most people, and then I'm going to put forward some ideas about how to progress these desiderata which just happen to fit in with my conception of anarchism.

\subsection{Desiderata}

Its simplest here to start with {\it utilitarianism} and say what I don't like about that and suggest a fix.

Utilitarianism is the idea that we should seek to maximise ``happiness''.
You imagine that happiness can be quantified, so that we could in principle add up the happiness of everyone in the universe giving an overall rating, and we take the maximisation of that as our goal.

Personally, its not my aim in life to be happy.
I do expect to be happy if I get what I want, but hedonism and utilitarianism have it the wrong way round for me.
Happiness is a possible consequence of achieving my goals, but it isn't actually the goal.
 
I prefer the word fulfilment, which for me has two connotations.
It connotes achieving whatever you were aiming to achieve in life, and it also suggests realisation of ones full potential, whatever than might be.

So I think we should be looking for a society in which as many people as possible are fulfilled, and succeed in living the kind of life which they desire for themselves.

This is all very woolly and I think that's what we should expect.
I don't think this is quantifiable, I don't think fulfilment can be aggregated or that we could know whether we were succeeding in maximising it, but still I think its worth saying that that's what we are after,

And note a few things about this.

Firstly everyone counts, and everyone counts equally, at least in the abstract. 
The idea is that we should arrange the world so that as many people as possible realise their full potential.

Secondly, it does matter how this fulfilment is distributed.
Its not good if one person gets a whole lot at the expense of several others getting none at all.
On the other hand, I don't think we should be aiming for a completely even distribution.
Once everyone has a decent amount its OK for some people to get much more than average.

The third very important item is that each individual gets to chose what counts as fulfilment for him.

% connection with anarchism

\subsection{How it's supposed to work}

Now I'm going to consider in the most simplistic terms how a modern democratic capitalist free market is supposed to deliver the desiderata (which I think to a large extent is supposed to be the case).

The needs of individuals or groups are met in some cases by private enterprise, in some cases by government.

The free market in goods, labour and capital is supposed to optimise production in the following ways:

Price competition ensures that:
 \begin{itemize}
   \item goods are produced by the most efficient producer for the lowest possible price
   \item people work for the employer who can make best use of their skills and obtain the best income for their work
   \item capital is employed on the projects which will deliver best economic return
 \end{itemize}

The behaviour of government is controlled through the ballot box.

\subsection{How it Fails to Work}

Head on price competition rarely occurs.
Normally different producers will produce different products, or will pretend that they are doing so, and then the competition turns into a marketing competition.
Most of the time consumers are more interested in status than in getting the best price, so they won't buy the cheapest product unless there is no doubt that it is not inferior, and competitive marketeers make it their business to cast doubt.

Where there is no apparent difference in the product, elaborate charging structures may be introduced to make price comparisons difficult (e.g. in the UK energy market).
Of course, they might just collude to fix prices, but this is generally illegal so they will just use more devious methods to interfere with price competition.

Prince competition is likely to be ineffective where large capital outlays are necessary to develop a product, e.g. in pharmaceutical products.
In these circumstances there will be patent protection and competition will be only with different drugs.
The vendor will charge as much as the market will bear irrespective of the development and production costs.
Drugs may therefore be cheap to produce but expensive to buy.
Software, and other information products are extreme cases, where there is no manufacturing cost and potentially very low distribution costs.
In such a market the seller will want to sell at the highest price at which the sale can be completed, but a use might like to have it cheaper.

\subsection{How to Make it Work}

The basic idea here is that doing things by cooperation and collaboration in peer to peer networks or marketplaces is generally more effective and efficient than by the use of compulsion and coercion in hierarchic command structures, especially when this is done on a large scale.
Decisions about what an intelligent agent should do are best taken by the agent.
Information and other kinds of support in making these decisions should be delivered to the agent to facilitate the decision process rather than trying to transfer the relevant information from the agent to some other party.
If the decisions are local then re-planning to take account of changed circumstance is more readily accomplished.

Many of the factors important for our quality of life can not easily be purchased in the marketplace or delivered by government.
Environmental factors, for example.
You cannot buy a reduction in global warming at the supermarket.
Nor can you chose to vote in a government which will fix the problem.

In free market economies the behaviour of suppliers in the market is determined by the spending decisions of purchasers, which is in turn influenced primarily by the perceived attributes of the products.
The purchase will have many side effects, for example in contributing to environmental problems, which are not taken into account in the purchase decision.

There is therefore scope for improvement to our quality of life by improving our knowledge of how purchase decisions impact upon it.

Let's not underestimate the potential scope of this approach to change.
It is a method for evolving our environment in all its aspects in whatever direction we care to move.
To use this tool we need to have a sense of the direction in which we seek change, and information about how our choices contribute to or detract from that desired change.