% $Id: intro.tex,v 1.4 2010/02/10 11:17:12 rbj Exp $

\def\rbjidintro{$$Id: intro.tex,v 1.4 2010/02/10 11:17:12 rbj Exp $$}

\chapter{Introduction}\label{Introduction}

In this part I'm going to present a problem, and my search for a resolution, in an informal and intuitive manner.
Then


The two principle elements of this resolution will be presented and further discussed in the following two parts of the book.

I'm going to do this {\it analytically}, and so the whole thing is going to be rather philosophical, but I have a deliberately broad idea about what counts as analysis, so it may not be how you expect philosophy to be.

Later on there will be much further exploration of what analysis might be, but for now I will give the lightest possible sketch of the kind of analysis I have in mind.

\section{Sketching a Method}

So, the idea is, that you start with some kind of problem, or maybe even an ostensibly unproblematic domain or topic that you want to examine.
The first thing you do is think about it, and try to understand the problem, or the subject matter.
In this, you may, if you think it might help, read whatever you think might help you to get a grip on it, either to give you relevant factual knowledge, or for ideas and methods which will help you to discover the solutions to your problem.

You may be fortunate enough to find something which looks like a solution, or perhaps not.
Either way, the next stage is to articulate your problem, write it down.
If you have a solution explain that as well, if not, you will have to settle for describing how you have gone about your search what you have learned from it so far.

Thus far, this could all be entirely intuitive (though it doesn't have to be).

Next you look back on all this, and see whether it hangs together, whether you can get a clearer grasp of the problem, its proposed solution, and get a firmer handle on whether the solution is likely to solve the problem.

The intuitive part yields primarily impressions and intuitive conclusions.
Not facts, subjective rather than objective.

Once you have this part on paper you can try to make it more precise, more objective.
You can test your proposed solution and look for objective grounds for the belief that it will work, and you can consider the alternatives and compare them with the preferred solution.

There is a class of analytic methods which may be considered preferable to others.
These I call nomologico-deductive methods, and they consist in describing the problem and its solution with a precision sufficient to enable deductive reasoning about it, possibly yielding a deductive proof of sufficiency.
Nudging the problem description in the direction of these preferred methods might be advantageous even if you don't get there.

\section{The Application}

The problem I intend to examine is one which I hit during an exercise in information systems design.
The information system under consideration can be seen as a kind of cognitive agent, something which holds and operates upon knowledge, or supposed knowledge.

This was an activity which falls under the compass of analysis as described above, for I had a simple idea relating to the architecture of a certain class of cognitive artifacts, and my aim was to describe the architectural ideas, and give the rationale which lay behind them.
For this kind of system, you won't be surprised to hear that the rationale had important philosophical elements in it.

I then discovered to my surprise, that a philosophical paper, written nearly half a century before had such a profound and lasting effect on philosophical opinion that the kind of rationale which I had intended to offer would receive no serious consideration from philosophers.

I was not unacquainted with this paper, I had myself read the paper more than twenty years before.
I had not then found any of its arguments to be convincing, and in the intervening years, despite much further relevant enlightenment, I had come no closer to accepting them.
So the problem which faced me was not that the philosophical rationale I intended to offer had been shown to be fatally flawed.
In that case the solution would be to find better rationale or abandon the proposal.

I remained in no greater doubt about the merits of the proposal or of the proposed philosophical underpinnings.
Why then should I not explain the defects in the paper which purported to undermine that kind of rationale?

I would not have found it difficult to offer grounds for rejecting the conclusions of the paper.
But adequate grounds had already been provided, by distinguished philosophers.
The consensus had moved against them nonetheless.

This was no mere consensus.
Most philosophical problems are debated over thousands of years without resolution, and the issues at stake in this problem had evolved over millennia.
Suddenly these were regarded as settled, to the extent that debate was virtually extinguished.

Furthermore, it seemed to me, that the acceptance of this paper could not be regarded as rational.
I had difficulty even in crediting that the author could have believed his own arguments.
This was, for me, stark evidence that, despite the evident intellectual credentials of many of its practitioners, analytic philosophy could not be regarded as a rational pursuit.

I believed, rightly or not, that the writing of a rebuttal could not prevail, and that conventional ``wisdom'' among professional philosophers represented a substantial obstacle to a project which was in its aim, practical rather than philosophical.


