% $Id: intro.tex,v 1.1 2010/01/10 21:30:10 rbj Exp $
\def\rbjidintro{$$Id: intro.tex,v 1.1 2010/01/10 21:30:10 rbj Exp $$}
\chapter{Introduction}\label{Introduction}

Why ``HOT'' philosophy?

In this work I aspire to systematic philosophy, and have tended in the past to give some appropriate name to the system which I have in mind.
Latterly it has been {\it Metaphysical Positivism} for my conception of analytic philosophy, and {\it Positive Philosophy} for a more broadly scoped philosophy of which Metaphysical Positivism is a proper part.
I have sought, however, a way of writing philosophy which comes closer to reflecting the current of my philosophical thought than could conceivably be presented as an account of a philosophical system.
My conception of philosophy moves forward at too rapid a pace for a stable account to be rendered without first inhibiting the flow and risking that the whole will cool and stultify.

What I attempt in this work is not to present and apply some stable conception of philosophy and its methods, but rather to achieve a unification of the process of writing prose with that of philosophical thinking.
The word ``HOT'' in my title, is intended to be suggestive of ``HOT off the press'', but closer perhaps to ``HOT off the mind''.
It is also an acronym, standing for ``Honest, Open and Transparent''.
My emphasis on these attributes of philosophy (which would perhaps normally be presumed and not thought to warrant mention) is provoked by my having been conscious continually over the last fifteen years that my opinions of contemporary analytic philosophy are such as would normally be passed over in silence.
I have tried to do this, but these opinions have so shaped the course of my philosophical deliberations that it is now doubtful whether an account could be given of any important part of my thinking without laying bare matters upon which philosophers would not normally touch.

Thus it must be admitted, that I do not believe that analytic philosophy, as it has been practised over the past half century, is a rational pursuit, and it is the central aim of this project to understand this supposed phenomenon, to find a way to do philosophy despite it, to plot a way forward which might mitigate the defect.

That most or all philosophy is unsound is not a novel doctrine.
With more or less force and scope it is held by a good proportion of the most influential philosophers throughout history.
Many philosophers have seemed to design their philosophy with the principal aim of addressing this problem.
In the twentieth century alone Russell and Moore had such a view of the idealism they forsook.
Wittgenstein had in both his early and late philosophies views highly dismissive of philosophy as a whole.
Carnap followed Russell in attempting to establish a more rigorous kind of philosophy think a key to this task could be found in the new logic, but his philosophy as a whole was swept away by the a new generation, lead by Quine and Kripke, who argued that Carnap's philosophy and methods were fatally flawed.

It is not my aim therefore to invest much space in a critique.

My course has been along the following lines.
Firstly, when first it seemed to me that I had a real problem, I wondered for a while how I had come to a philosophical weltanschauung so greatly at odds with received opinion, and gradually began to explore the origins of my own position, and then to trace the ideas back through history.
The conviction that analytic philosophy is not a rational activity was puzzling for me.
I needed to understand why.
My predicament was similar to that in which David Hume found himself.
He became convinced that our beliefs about many aspects of the external world were not rationally justified, and sought an explanation for why we should hold such beliefs in the study of ``Human Nature''.
