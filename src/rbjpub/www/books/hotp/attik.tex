\mainmatter

\newtheorem{iprin}{I}

\part{Preliminaries}\label{part0}

\chapter{Prelude 1}

One aim in this volume of HOT philosophy is to narrow the gaps between idea, expression and publication, and so to produce a volume which has more of the dynamic and excitement of vigorous philosophical debate.
The most important element of this is that the work is primarily concerned with new ideas, which are not yet set in stone but will emerge in the process of writing.
Secondary elements arise from the use of multiple publishing routes, some closely connected with the creative process.
The closest to the thinking edge will be the blogs and archived email discussion lists.
The writing of the book itself will be visible through regularly updated drafts published on the RBJones.com website.
The publication of the book will be by Print on Demand through amazon.com permitting the book to be made available within a very short space of time of completion of the writing.

``HOT'' has another significance: ``Honest, Open, Transparent''.

The book is an exploration in philosophical analysis, though not the application of a settled analytic method.
The purpose is to progress the evolution of a pluralistic conception of analytic method, as well as to achieve some worthwhile analysis in the selected problem domains.

The methodological pluralism is partly offered as an element of an approach to the ameliorisation of the first problem which we consider.
This is the perception that philosophy as it has been conducted in the twentieth century, is not a rational pursuit.
The perception is mine, and the first problem is to analyse this perception and see what basis there is for it.
The aim of the analysis here is to understand what transpired, by focussing on a small number of key episodes in the context of a picture in the large of the principle structures in 20th century analytic philosophy.
This analysis forms Part \ref{partI} of the book.

In Part \ref{partII} we consider some ideas from the theory of evolution which might help us to understand why people who are capable of behaving rationally and with high degrees of intelligence, often appear to behave irrationally.
The analysis here serves multiple purposes.
It continues the exploration of analytic method, and is particularly concerned with how we can analyse 

In Part \ref{partIII} we analyse some ideas which are prophylactic in intent.
The analysis is of ways of representing tentative knowledge which may be regarded as a reasonable response to an appropriate scepticism.
These constitute a systematic epistemic retreat in which instead of dogmatic assertions we record strength of the evidence supporting various alternative models of the problem domain.

\chapter{Prelude 2}

There are many kinds of philosophical analysis;
this work results in part from my own attempts to determine and articulate yet another analytic method.
In the attempt I have slipped into a kind of analytic pluralism, a recognition of the legitimacy and desirability of working with more than one method of analysis.

My methodological deliberations have been a part of a broader uncertaintly about how to do philosophy.
This was provoked (in part) by a road block inhibiting the completion of a philosophical exposition which I had expected to be modest and uncontroversial.
I wandered in a philosophical wilderness for more than a decade, seeking an understanding of the obstacle and a way to do philosophy despite it.

The main purpose of this book is to present analysis in progress applied in two problem domains, neither strictly philosophical.
Before engaging with these problems I set the stage with a description of my original difficulty as I perceived it, of aspects of the search which followed, and of the conception of philosophy and of analytic method which resulted.

Part \ref{partI} of the book provides the background for the approach to analysis explored in the next two parts.

The first problem on which I attempt an analysis (in Part \ref{partII}) concerns aspects of the theory of evolution, and their impact upon the nature of rational enquiry.

In Part \ref{partIII} I present and analyse ideas which connect epistemological concerns with aspects of the architecture of cognitive artifacts.

\chapter{Introduction}

I aspire to systematic philosophy, to a comprehensive and coherent philosophical \emph{weltanschauung}.
I have tended in the past to give some appropriate name to the system which I have in mind.
Latterly it has been {\it Metaphysical Positivism} for my conception of analytic philosophy, and {\it Positive Philosophy} for a more broadly scoped philosophy of which Metaphysical Positivism is a proper part.
I have sought, nevertheless, a way of writing philosophy which comes closer to reflecting the current of my philosophical thought than could plausibly be presented as an account of a philosophical system.

What I attempt in this work is not to present and apply some stable conception of philosophy and its methods, but rather to achieve a unification of the process of writing prose with that of philosophical thinking.
This is intended to give a snapshot of philosophical progress in the making, progress both in the application of philosophical method but also in its conception.

The methods are background for the purposes of the analyses in parts II and III.

\section{Why ``HOT'' philosophy?}

The word ``HOT'' in my title, is intended to be suggestive of ``HOT off the press'', but closer perhaps to ``HOT off the mind''.
It is also an acronym, standing for ``Honest, Open and Transparent''.
My emphasis on these attributes of philosophy (which would perhaps normally be presumed and not thought to warrant mention) is provoked by my having been conscious continually over the last fifteen years that my opinions of contemporary analytic philosophy are such as would normally be passed over in silence.
I have tried to do this, but these opinions have so shaped the course of my philosophical deliberations that it is now doubtful whether an account could be given of any important part of my thinking without laying bare matters upon which philosophers would not normally touch.

Thus it must be admitted, that I do not believe that analytic philosophy, as it has been practised over the past half century, is a rational pursuit, and it is the central aim of this project to understand this supposed phenomenon, to find a way to do philosophy despite it, to plot a way forward which might mitigate the defect.

\section{An Impediment to Philosophical Exposition}

That most or all philosophy is unsound is not a novel doctrine.
With more or less force and scope it is held by a good proportion of the most influential philosophers throughout history.
Many philosophers have seemed to design their philosophy with the principal aim of addressing this problem.
In the twentieth century alone Russell and Moore had such a view of the idealism they forsook.
Wittgenstein had in both his early and late philosophies views highly dismissive of philosophy as a whole.
Carnap followed Russell in attempting to establish a more rigorous kind of philosophy think a key to this task could be found in the new logic, but his philosophy as a whole was swept away by the a new generation, lead by Quine and Kripke, who argued that Carnap's philosophy and methods were fatally flawed.

It is not my aim therefore to invest much space in a critique.

My course has been along the following lines.
Firstly, when first it seemed to me that I had a real problem, I wondered for a while how I had come to a philosophical weltanschauung so greatly at odds with received opinion, and gradually began to explore the origins of my own position, and then to trace the ideas back through history.
The conviction that analytic philosophy is not a rational activity was puzzling for me.
I needed to understand why.
My predicament was similar to that in which David Hume found himself.
He became convinced that our beliefs about many aspects of the external world were not rationally justified, and sought an explanation for why we should hold such beliefs in the study of ``Human Nature''.

\chapter{Introduction 2}

I should like to be able to present my philosophical ideas in their own right, without troubling the reader with extraneous details of my own personal circumstances, and motivations, or the accidents by which life has lead me down this path.
However, I have come to the opinion that an attempt at such a presentation would make an understanding much harder for the reader to achieve.
I have elected instead to try to confine material of an autobiographical character in this prelude, which will serve either to set the readers mind on a better track for understanding the sequel or to free him of any inclination read further.

There is a single problem which dominates all others and I will first try to describe how I came to attach such importance to that problem.
In the synthesis which I have woven in my attempts to address this problem there are elements which are dictated not purely by the character of that problem, but also by various other problems which have interested me, and by my own character.
In this prelude I describe first the origins of my principle problem, and then a minimum of material on those other problems which fit into the synthesis which is my response to the problem.

\section{My Big Problem}

Let me first describe a small problem, my first approach to philosophical thinking.

At the age of 11 I was sent as a boarder to an English grammar school.
In the absence of any special arrangement, pupils were required to attend the local Anglican\footnote{
Church of England} church every Sunday.
This usually included a substantial sermon, and inevitably I found myself thinking about what was being said.
Since the whole was predicated on the existence of God, to make sense of these sermons I had first to make sense of the idea of God, and this I attempted, at length, during this regular period of reflection imposed upon me by my betters.
I though through a variety of ideas about what God might be, trying to find one which squared with the kinds of thing which were being said in the sermons, and which {\it made sense} to me, in which I could believe.
Eventually I abandoned the search.
I could make no sense of what I was being told, and concluded ultimately that it made no sense.

At the beginning of my second year pupils were given the choice whether to begin lessons preparatory to confirmation in the Anglican Church.
I was by then quite definite in my disbelief and declined, so my deliberations were entirely settled in that first year.
I have not since been tempted even to consider arguments for or against the existence of God, for me the concept lacked intelligible meaning.

The greatest difficulty in this process, so far as I can now recollect, was the consideration that so many important and distinguished people around me and in the world at large professed belief in God.
How could so many be mistaken or confused?
This consideration however, never stood a chance of prevailing against my own judgement.

Though I did not consider the matter at the time, in my opinion this early decisiveness in important philosophical matters is not evidence of intellectual precocity, but rather of an aspect of character which we may call {\it independence of mind}, a singular disinclination to take matters on trust.

My purpose in presenting this first problem and my resolution of it is primarily to illustrate by contrast the difficulty which I found myself in many years later and which is the single greatest influence shaping the philosophical deliberations which have lead to the composition of this book, and to the approach to philosophy which it exemplifies.
In 1951 W.V.Quine published his essay entitled ``Two Dogmas of Empiricism'' \cite{quine51}.
My own first acquaintance with this essay was probable in about 1975, though I have now no recollection of what I then thought of it.
I don't believe it had any impact on my philosophical views.

I did have difficulties as an undergraduate studying mathematics and philosophy.
One of my principle desires in doing philosophy was to address substantive rather than verbal problems.
I sought by the formulation of definitions to work with chosen concepts relevant to some problem domain rather than to use the concepts already in place and address problems couched in such terms which would then be liable to degenerate into investigation of their meanings.
The demise of ``ordinary language philosophy'', had not yet reached provincial English Universities, but in any case never inspired the kind of detachment from ordinary language which I sought in my philosophy.

Perhaps 20 years later I belatedly became aware of extent of the influence which had been exerted by Quine's essay.
I was then edging my way back into philosophical thinking, and sought to articulate philosophical background for some ideas about architecture for Artificial Intelligence.
I suddenly became aware that these philosophical underpinnings simply could not be articulated, because Quine had undermined concepts fundamental to the account.

This was now for me a problem.
I perceived that the point of view I wished to articulate would not be understood by philosophers, who, it seemed would reject it simply because it made use of distinctions which Quine was supposed to have shown to be untenable.

The problem was worse than that.
Not only did Quine's arguments leave me cold, he seemed not even to have advanced an intelligible thesis.
I found great difficulty in believing that Quine could have himself believed what he wrote, but worse yet, it seemed to me that the wholesale abandonment of the analytic/synthetic distinction by professional philosophers could not have been a rational response to the arguments put forward by Quine.

The concepts essential to my philosophy had been rejected by professional philosophers, but, it seemed to me, this had not been done on any rational basis.
There could be no point in a detailed rebuttal, the rejection of these fundamental concepts could not have been rational, and a rational r

\chapter{Introduction 3}\label{Introduction}

In this part I'm going to present a problem, and my search for a resolution, in an informal and intuitive manner.
Then


The two principle elements of this resolution will be presented and further discussed in the following two parts of the book.

I'm going to do this {\it analytically}, and so the whole thing is going to be rather philosophical, but I have a deliberately broad idea about what counts as analysis, so it may not be how you expect philosophy to be.

Later on there will be much further exploration of what analysis might be, but for now I will give the lightest possible sketch of the kind of analysis I have in mind.

\section{Sketching a Method}

So, the idea is, that you start with some kind of problem, or maybe even an ostensibly unproblematic domain or topic that you want to examine.
The first thing you do is think about it, and try to understand the problem, or the subject matter.
In this, you may, if you think it might help, read whatever you think might help you to get a grip on it, either to give you relevant factual knowledge, or for ideas and methods which will help you to discover the solutions to your problem.

You may be fortunate enough to find something which looks like a solution, or perhaps not.
Either way, the next stage is to articulate your problem, write it down.
If you have a solution explain that as well, if not, you will have to settle for describing how you have gone about your search what you have learned from it so far.

Thus far, this could all be entirely intuitive (though it doesn't have to be).

Next you look back on all this, and see whether it hangs together, whether you can get a clearer grasp of the problem, its proposed solution, and get a firmer handle on whether the solution is likely to solve the problem.

The intuitive part yields primarily impressions and intuitive conclusions.
Not facts, subjective rather than objective.

Once you have this part on paper you can try to make it more precise, more objective.
You can test your proposed solution and look for objective grounds for the belief that it will work, and you can consider the alternatives and compare them with the preferred solution.

There is a class of analytic methods which may be considered preferable to others.
These I call nomologico-deductive methods, and they consist in describing the problem and its solution with a precision sufficient to enable deductive reasoning about it, possibly yielding a deductive proof of sufficiency.
Nudging the problem description in the direction of these preferred methods might be advantageous even if you don't get there.

\section{The Application}

The problem I intend to examine is one which I hit during an exercise in information systems design.
The information system under consideration can be seen as a kind of cognitive agent, something which holds and operates upon knowledge, or supposed knowledge.

This was an activity which falls under the compass of analysis as described above, for I had a simple idea relating to the architecture of a certain class of cognitive artifacts, and my aim was to describe the architectural ideas, and give the rationale which lay behind them.
For this kind of system, you won't be surprised to hear that the rationale had important philosophical elements in it.

I then discovered to my surprise, that a philosophical paper, written nearly half a century before had such a profound and lasting effect on philosophical opinion that the kind of rationale which I had intended to offer would receive no serious consideration from philosophers.

I was not unacquainted with this paper, I had myself read the paper more than twenty years before.
I had not then found any of its arguments to be convincing, and in the intervening years, despite much further relevant enlightenment, I had come no closer to accepting them.
So the problem which faced me was not that the philosophical rationale I intended to offer had been shown to be fatally flawed.
In that case the solution would be to find better rationale or abandon the proposal.

I remained in no greater doubt about the merits of the proposal or of the proposed philosophical underpinnings.
Why then should I not explain the defects in the paper which purported to undermine that kind of rationale?

I would not have found it difficult to offer grounds for rejecting the conclusions of the paper.
But adequate grounds had already been provided, by distinguished philosophers.
The consensus had moved against them nonetheless.

This was no mere consensus.
Most philosophical problems are debated over thousands of years without resolution, and the issues at stake in this problem had evolved over millennia.
Suddenly these were regarded as settled, to the extent that debate was virtually extinguished.

Furthermore, it seemed to me, that the acceptance of this paper could not be regarded as rational.
I had difficulty even in crediting that the author could have believed his own arguments.
This was, for me, stark evidence that, despite the evident intellectual credentials of many of its practitioners, analytic philosophy could not be regarded as a rational pursuit.

I believed, rightly or not, that the writing of a rebuttal could not prevail, and that conventional ``wisdom'' among professional philosophers represented a substantial obstacle to a project which was in its aim, practical rather than philosophical.


\chapter{Rationality and Deduction}\label{RationalityDeduction}

``Rational'' is not a precisely defined term.
We may apply it to people, to their opinions, their conclusions, or to their actions, and in doing so we will usually be expressing approval.
We will consider an opinion rational if we suppose the evidence on which is it based to provide solid grounds for the belief, or an action rational if it is likely under the circumstances to secure the end for which it is construed.

These terms, however, with which we describe rationality, are not themselves precise.
It is not clear that a judgement of rationality based on this guidance will have more substance than a mere concurrence of opinion.

By contrast, the term {\it deductive}, closely related to rationality, can be made precise, and provides objective evaluation of reason which falls within its narrow remit.
The reputation of deduction has been built primarily from its successful application in the development of mathematics over the last two and a half millenia.
Throughout its history that reputation has been exploited by men who have sought to give force to their own opinions, and who therefore present as deductive reasoning which cannot properly be so described.
Many of these have been philosophers.

The notion of rationality has similarly been abused.
Something may be rational even if not deductive, and hence rationality transcends the limits of applicability of deduction, to the extent that it has no bounds.
Any conclusion or action may be deemed rational if it seems appropriate in the circumstances, however tenuous our grounds for this belief.
Secular authorities, be they individual, institutional or cultural, may under the mantle of rationality behave as if their conclusions approached the conclusiveness of sound deduction.

Deductive and rational as discussed are attributes of conscious deliberations.
Rational has another yet broader use, not for distinguishing good reasoning from bad, but for distinguishing conclusions and conduct based on conscious deliberations from intuitive or instinctive beliefs or behaviours.
In this sense we may (and will) contrast the rationalistic philosophical tradition which begins in ancient Greece with other traditions, such as the Chinese philosophy/religion of Dao, which are explicitly opposed to rationalism as a practical philosophy.

It will be my aim to explore the scope and limits of these notions, both in relation to theoretical matters (their effects on what we can count as knowledge) and practical (how they affect what we do).

\section{Descriptive Language}

Communication between living oraganisms is virtually as old as life itself, it is an essential element of sociality, and an invaluable survival tool.
At its most primitive simply sensing the proximity of another is an elementary factor in maintaining group cohesion.
At another level of complexity observing and mimicking may play a vital role before there emerges anything we might recognise as a language.
The use of noises or gestures for communication, perhaps warning of threats or as a part of mating rituals, may be though of as first steps toward the use of language.
Talk of ``language games'' in analytic philosophy emphasises the diversity of purposes which may be realised through language and was probably introduced to broaden philosophical discussion of language from its early twentieth century on a very special kind of language which is our present concern.

Descriptive language is language specialised to the communication of objective information, rather than for the diverse other purposes which it might serve.
A cry of alarm conveys information of the moment, if it were remembered and related at a later date it would no longer serve the same purpose.
There accumulation of communicable knowledge about the world requires a kind of language whose meaning is less sensitive to the context in which it is first expressed, which can be passed on indefinitely for the benefit of others.

The possibility of deductive inference flows from the adoption of descriptive language, and is so bound up in descriptive language that we cannot be said to understand descriptive languages unless we are able to perform elementary deductions.
A sentence formed in English by conjoining two other sentences using the logical operator ``and'' {\it entails} each of the constituent sentences, and from it we may deduce the constituents.
Someone who is not able to perform such deductions cannot be said to understand English sentences formed in this way.
Similarly a sentence formed in English from two other sentences using the connective ``or'' is entailed by each of its constituents, and may be deduced from either of them.
The ability to undertake this kind of deduction is an essential to a full understanding of the meaning of the connective.
In this way we can show that a knowledge the kinds of deduction with which we are familiar in modern formal logics is implicit in an understanding of ordinary language, and we may reasonably suppose that the routing use of such elementary deductions long predated even the concept of logic, deduction or proof.

Even without the concept of deduction, our language is also capable of talking about sound derivations.


\chapter{Introduction 4}

\section{Purpose}

This book is concerned with rationality.

There have been, during the last 150 years, major advances in our understanding of deductive reasoning and in the technologies available to support the application of deductive methods.
A thorough assimilation and effective application of these continuing advances would constitute a revolution in the role of deduction in rational processes, and has yet to be accomplished.

A method for the systematic application of deduction will be referred to here as a \emph{nomologico-deductive} method, this term emphasising that deduction can only properly be conducted in a well-defined and coherent logical context, which gives meaning to the results obtained.
We sometimes refer to such a context as a model, and consider the application of deductive methods to proceed by the construction of an abstract model of the intended subject matter, deductive knowledge about which may then inform our understanding of the problem at hand.

It will be the central concern of this book to explore the origins, nature and scope of such methods in the context of a less precisely delineated broader conception of \emph{rationality}.
The exploration is intended to constitute a \emph{philosophical analysis} and at the same time to contribute to the evolution of our conception of what it takes for analytic philosophy to be rational.

The whole is presented as a historical narrative which spans in a very selective way the entire history of life on this planet.
This first part provides a minimum of background before the narrative which is split into three parts as follows:

\begin{itemize}
\item[\ref{partI}] which concerns the evolution of our rational capabilities up to the point where they can be said to involve deduction
\item[\ref{partII}] which follows the development of our understanding of deduction and various connected philosophical logical and matehmatical themes up to the beginning of the twentieth century
\item[\ref{partIII}] covering the development of logic, semantics, and nomologico-deductive method, in philosophy, mathematics and computer science
\end{itemize}

Part \ref{partIV} provides ideas for the future.

\section{Rationality and Deduction}

The term \emph{rational} is here used as a term of approval.
It should not be understood as encompassing all that a \emph{rationalist} might endorse.
A rationalist is someone with a broad, possible a much too broad, conception of the scope of applicability of deduction.
However, in the \emph{reasonable} conception of rationality which is used here, it is not only possible but even common for someone to have roo broad an idea of what can be done by, and/or too broad a conception of what constitutes, deductive reason.
Thus it is possible for someone to be \emph{too rationalistic}, but not possible in the sense in which the term is used here for him to be \emph{too rational}.

It is then, in the sense we adopt, rational to have a good understand of the limits of what can be established by deductive reasoning, and a sharp eye for what actually constitutes deduction, and to reject arguments which go beyond the bounds or masquerade fraudulently as deductive.
Though deduction is a very powerful technique whose application can profitably be extended, its scope is narrow, and outside that scope decusions, both theoretical and practical, must be made, and can properly be judged rational or not.

It will be helpful to give some consideration to the nature of rationality as it transcends the limits of deductive reason, partly because this is the kind of rationality which we hope to exhibit in the larger part of this work.

For this the empiricist idea of the mind as an associative engine is helpful.
From this point of view the mind, once it has evolved beyond mere reflexive responses to environmental stimulae, is engaged in the organised storage of experience on the basis of which patterns are recognised which permit anticipation of future events as a result of the recognition that contemporary experience seems similar to some past experience of which the outcome has been memorised.
The recognition of such patterns provides a basis for a more discerning response to environment made more appropriate by knowledge of the past.
It also enables more elaborate responses to complex situations by permitting the construction of mental plans which have been validated by the intuitions of our associative memory about the likely outcome of each stage in the plan.


\section{Philosophical Analysis}

It will be the business of Part \ref{partIV} of this book to consider in some detail what philosophical analysis might be, in the light cast by the historical explorations which have preceded it.
However, the whole book is presented as an exemplar of certain kinds of philosophical analysis and a few words to explain how it might thus be construed may be in order.

It is the business of philosophy, as I conceive it, largely following Aristotle, to address problems which are deep and difficult, and which may therefore prove awkward candidates for hard core formal or rigorous nomologico-deductive methods.
Though we may hope to progress them towards that kind of analysis, if we are to begin at all some less demanding conception of analysis will be needed.

It is for that reason that I offer as the first stage in philosophical analysis the articulation, clearly but informally, of an intuitive understanding of the problem domain.
One may then hope to advance the account to the point at which is constitutes a model definite enough (if perhaps still informally presented) to constitute an adequate basis for deductive analysis.

The most suspect application of this minimalistic conception of analysis is to be found in Part \ref{partI}, where I address a subject matter in which I have no claim to expertise, the evolution of life on earth.
In this I exemplify my further belief that there are no limits to the scope of philosophical enquiry, which may go wherever it must for an understanding of the issues which concern it.
The philosopher does this at his own risk.
He will presumably keep mute unless confident of having something positive to offer, but may nevertheless encounter hostility from those in whose domain he transgresses.

\section{The Rationality of Analytic Philosophy}

An important aim is to consider the question whether analytic philosophy is a rational activity (on which you may conclude that I have doubts).
I will declare my prejudice here rather than leave you in suspense.

The answer of course does depend upon what we mean by rational, and also I suggest, upon the point of view from which the rationality of the activities are considered.
Some refinement of detail will be necessary, but here is a sketch of my present impressions.

First, I am not inclined to doubt that the behaviour of professional philosophers is rational, from their own point of view, meaning by this that they probably are acting on the whole, rationally in the furtherance of their own interests, in the context in which they find themselves.

However, in relation to a plausible account of the purpose of the institutions of which they are a part, I find reason to doubt that either individual philosophers or the relevant social groups or professional institutions act rationally in the pursuit of those purposes.

At a higher level matters may be seen once again in a more benign light.
If we ask whether sense can be made in evolutionary terms of what goes on in analytic philosophy, then my answer is a tentative yes, and an inclination to believe that sense could be made even if my own understanding is incorrect.
Can this be regarded as an instance of some kind of rationality?
Does this kind of explanation justify the claim that, from an evolutionary point of view the apparent pathologies turn out to be rational?

Into this picture we may then add the dynamic provided by this intervention (the writing of this book).
For I will say, that even if the present situation can be understood in evolutionary terms, it is rational to seek improvements, and it is reasonable to hope for some measure of success.
This intervention is of course a part of the evolutionary process, and it is potentially an evolution of that process rather than purely one accomplished by it.

\section{The Scope and Nature of Analytic Philosophy}

It is not my purpose here to provide any precise delineation of philosophy, but rather to give an informal indication of a broad conception of analytic philosophy.

My starting point is to couple the concept of \emph{Analytic Philosophy} with that of \emph{rationality}.
This may be thought of as a softening of the radical identification with logic proposed by Russell and largely followed by Rudolf Carnap.
In my present terminology, the narrow conception corresponds to the adoption only of nomologico-deductive methods, and the broader comes from admitting any method, or lack of method, which can be considered \emph{rational}.

To this characterisation we can add some ideas from Aristotle's \emph{Metaphysics} \cite{aristotleMETAP}.
For Aristotle the distinguishing characteristic of a philosopher \emph{wisdom}, and so his deliberations about ``first philosophy'' begin as a discussion of the nature of wisdom.

Aristotle, in describing ``first philosophy'' gives an analysis of kinds of knowledge in an ascending order

\begin{itemize}
\item sensation
\item memory (sensation + memory => experience)
\item artistic, having some "mechanical" competence
\item master-work, knowledge of theory
\item mastery of recreational rather then practical arts
\end{itemize}

saying of these:

\begin{quote}
"... the man of experience is thought to be wiser than the possessors of any sense-perception whatever, the artist wiser than the men of experience, the master-worker than the mechanic, and the theoretical kinds of knowledge to be more of the nature of Wisdom than the productive.
Clearly then Wisdom is knowledge about certain principles and causes." 
\end{quote}

I want to draw on some aspects of this discussion and discard others.
But first I note that we may think this progression of Aristotle's as not too far away from a naive conception of how intelligence might have evolved.
This is a topic I will look at in greater detail in Part \ref{partI}.
Its present relevance is the idea that philosophy is the most advanced form of intellectual enterprise.
Quite apart from the evolutionary connection, Aristotle is explicit in considering the later abilities in the list as superior to the earlier ones, at least in point of wisdom.

This arrogation of superiority in relation to other disciplines is probably one of the features of ``first philosophy'' which turned many philosophers against it during the second half of the twentieth century.

\label{partI}\label{partII}\label{partIII}\label{partIV}\label{partV}
