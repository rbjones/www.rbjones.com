\appendix

\chapter{Introduction}

I aspire to systematic philosophy, to a comprehensive and coherent philosophical \emph{weltanschauung}.
I have tended in the past to give some appropriate name to the system which I have in mind.
Latterly it has been {\it Metaphysical Positivism} for my conception of analytic philosophy, and {\it Positive Philosophy} for a more broadly scoped philosophy of which Metaphysical Positivism is a proper part.
I have sought, nevertheless, a way of writing philosophy which comes closer to reflecting the current of my philosophical thought than could plausibly be presented as an account of a philosophical system.

What I attempt in this work is not to present and apply some stable conception of philosophy and its methods, but rather to achieve a unification of the process of writing prose with that of philosophical thinking.
This is intended to give a snapshot of philosophical progress in the making, progress both in the application of philosophical method but also in its conception.

The methods are background for the purposes of the analyses in parts II and III.

\section{Why ``HOT'' philosophy?}

The word ``HOT'' in my title, is intended to be suggestive of ``HOT off the press'', but closer perhaps to ``HOT off the mind''.
It is also an acronym, standing for ``Honest, Open and Transparent''.
My emphasis on these attributes of philosophy (which would perhaps normally be presumed and not thought to warrant mention) is provoked by my having been conscious continually over the last fifteen years that my opinions of contemporary analytic philosophy are such as would normally be passed over in silence.
I have tried to do this, but these opinions have so shaped the course of my philosophical deliberations that it is now doubtful whether an account could be given of any important part of my thinking without laying bare matters upon which philosophers would not normally touch.

Thus it must be admitted, that I do not believe that analytic philosophy, as it has been practised over the past half century, is a rational pursuit, and it is the central aim of this project to understand this supposed phenomenon, to find a way to do philosophy despite it, to plot a way forward which might mitigate the defect.

\section{An Impediment to Philosophical Exposition}

That most or all philosophy is unsound is not a novel doctrine.
With more or less force and scope it is held by a good proportion of the most influential philosophers throughout history.
Many philosophers have seemed to design their philosophy with the principal aim of addressing this problem.
In the twentieth century alone Russell and Moore had such a view of the idealism they forsook.
Wittgenstein had in both his early and late philosophies views highly dismissive of philosophy as a whole.
Carnap followed Russell in attempting to establish a more rigorous kind of philosophy think a key to this task could be found in the new logic, but his philosophy as a whole was swept away by the a new generation, lead by Quine and Kripke, who argued that Carnap's philosophy and methods were fatally flawed.

It is not my aim therefore to invest much space in a critique.

My course has been along the following lines.
Firstly, when first it seemed to me that I had a real problem, I wondered for a while how I had come to a philosophical weltanschauung so greatly at odds with received opinion, and gradually began to explore the origins of my own position, and then to trace the ideas back through history.
The conviction that analytic philosophy is not a rational activity was puzzling for me.
I needed to understand why.
My predicament was similar to that in which David Hume found himself.
He became convinced that our beliefs about many aspects of the external world were not rationally justified, and sought an explanation for why we should hold such beliefs in the study of ``Human Nature''.

\section{Introduction (II)}

I should like to be able to present my philosophical ideas in their own right, without troubling the reader with extraneous details of my own personal circumstances, and motivations, or the accidents by which life has lead me down this path.
However, I have come to the opinion that an attempt at such a presentation would make an understanding much harder for the reader to achieve.
I have elected instead to try to confine material of an autobiographical character in this prelude, which will serve either to set the readers mind on a better track for understanding the sequel or to free him of any inclination read further.

There is a single problem which dominates all others and I will first try to describe how I came to attach such importance to that problem.
In the synthesis which I have woven in my attempts to address this problem there are elements which are dictated not purely by the character of that problem, but also by various other problems which have interested me, and by my own character.
In this prelude I describe first the origins of my principle problem, and then a minimum of material on those other problems which fit into the synthesis which is my response to the problem.

\section{My Big Problem}

Let me first describe a small problem, my first approach to philosophical thinking.

At the age of 11 I was sent as a boarder to an English grammar school.
In the absence of any special arrangement, pupils were required to attend the local Anglican\footnote{
Church of England} church every Sunday.
This usually included a substantial sermon, and inevitably I found myself thinking about what was being said.
Since the whole was predicated on the existence of God, to make sense of these sermons I had first to make sense of the idea of God, and this I attempted, at length, during this regular period of reflection imposed upon me by my betters.
I though through a variety of ideas about what God might be, trying to find one which squared with the kinds of thing which were being said in the sermons, and which {\it made sense} to me, in which I could believe.
Eventually I abandoned the search.
I could make no sense of what I was being told, and concluded ultimately that it made no sense.

At the beginning of my second year pupils were given the choice whether to begin lessons preparatory to confirmation in the Anglican Church.
I was by then quite definite in my disbelief and declined, so my deliberations were entirely settled in that first year.
I have not since been tempted even to consider arguments for or against the existence of God, for me the concept lacked intelligible meaning.

The greatest difficulty in this process, so far as I can now recollect, was the consideration that so many important and distinguished people around me and in the world at large professed belief in God.
How could so many be mistaken or confused?
This consideration however, never stood a chance of prevailing against my own judgement.

Though I did not consider the matter at the time, in my opinion this early decisiveness in important philosophical matters is not evidence of intellectual precocity, but rather of an aspect of character which we may call {\it independence of mind}, a singular disinclination to take matters on trust.

My purpose in presenting this first problem and my resolution of it is primarily to illustrate by contrast the difficulty which I found myself in many years later and which is the single greatest influence shaping the philosophical deliberations which have lead to the composition of this book, and to the approach to philosophy which it exemplifies.
In 1951 W.V.Quine published his essay entitled ``Two Dogmas of Empiricism'' \cite{quine51}.
My own first acquaintance with this essay was probable in about 1975, though I have now no recollection of what I then thought of it.
I don't believe it had any impact on my philosophical views.

I did have difficulties as an undergraduate studying mathematics and philosophy.
One of my principle desires in doing philosophy was to address substantive rather than verbal problems.
I sought by the formulation of definitions to work with chosen concepts relevant to some problem domain rather than to use the concepts already in place and address problems couched in such terms which would then be liable to degenerate into investigation of their meanings.
The demise of ``ordinary language philosophy'', had not yet reached provincial English Universities, but in any case never inspired the kind of detachment from ordinary language which I sought in my philosophy.

Perhaps 20 years later I belatedly became aware of extent of the influence which had been exerted by Quine's essay.
I was then edging my way back into philosophical thinking, and sought to articulate philosophical background for some ideas about architecture for Artificial Intelligence.
I suddenly became aware that these philosophical underpinnings simply could not be articulated, because Quine had undermined concepts fundamental to the account.

This was now for me a problem.
I perceived that the point of view I wished to articulate would not be understood by philosophers, who, it seemed would reject it simply because it made use of distinctions which Quine was supposed to have shown to be untenable.

The problem was worse than that.
Not only did Quine's arguments leave me cold, he seemed not even to have advanced an intelligible thesis.
I found great difficulty in believing that Quine could have himself believed what he wrote, but worse yet, it seemed to me that the wholesale abandonment of the analytic/synthetic distinction by professional philosophers could not have been a rational response to the arguments put forward by Quine.

The concepts essential to my philosophy had been rejected by professional philosophers, but, it seemed to me, this had not been done on any rational basis.
There could be no point in a detailed rebuttal, the rejection of these fundamental concepts could not have been rational, and a rational r

\chapter{Rationality and Deduction}\label{RationalityDeduction}

``Rational'' is not a precisely defined term.
We may apply it to people, to their opinions, their conclusions, or to their actions, and in doing so we will usually be expressing approval.
We will consider an opinion rational if we suppose the evidence on which is it based to provide solid grounds for the belief, or an action rational if it is likely under the circumstances to secure the end for which it is construed.

These terms, however, with which we describe rationality, are not themselves precise.
It is not clear that a judgement of rationality based on this guidance will have more substance than a mere concurrence of opinion.

By contrast, the term {\it deductive}, closely related to rationality, can be made precise, and provides objective evaluation of reason which falls within its narrow remit.
The reputation of deduction has been built primarily from its successful application in the development of mathematics over the last two and a half millenia.
Throughout its history that reputation has been exploited by men who have sought to give force to their own opinions, and who therefore present as deductive reasoning which cannot properly be so described.
Many of these have been philosophers.

The notion of rationality has similarly been abused.
Something may be rational even if not deductive, and hence rationality transcends the limits of applicability of deduction, to the extent that it has no bounds.
Any conclusion or action may be deemed rational if it seems appropriate in the circumstances, however tenuous our grounds for this belief.
Secular authorities, be they individual, institutional or cultural, may under the mantle of rationality behave as if their conclusions approached the conclusiveness of sound deduction.

Deductive and rational as discussed are attributes of conscious deliberations.
Rational has another yet broader use, not for distinguishing good reasoning from bad, but for distinguishing conclusions and conduct based on conscious deliberations from intuitive or instinctive beliefs or behaviours.
In this sense we may (and will) contrast the rationalistic philosophical tradition which begins in ancient Greece with other traditions, such as the Chinese philosophy/religion of Dao, which are explicitly opposed to rationalism as a practical philosophy.

It will be my aim to explore the scope and limits of these notions, both in relation to theoretical matters (their effects on what we can count as knowledge) and practical (how they affect what we do).

\section{Descriptive Language}

Communication between living oraganisms is virtually as old as life itself, it is an essential element of sociality, and an invaluable survival tool.
At its most primitive simply sensing the proximity of another is an elementary factor in maintaining group cohesion.
At another level of complexity observing and mimicking may play a vital role before there emerges anything we might recognise as a language.
The use of noises or gestures for communication, perhaps warning of threats or as a part of mating rituals, may be though of as first steps toward the use of language.
Talk of ``language games'' in analytic philosophy emphasises the diversity of purposes which may be realised through language and was probably introduced to broaden philosophical discussion of language from its early twentieth century on a very special kind of language which is our present concern.

Descriptive language is language specialised to the communication of objective information, rather than for the diverse other purposes which it might serve.
A cry of alarm conveys information of the moment, if it were remembered and related at a later date it would no longer serve the same purpose.
There accumulation of communicable knowledge about the world requires a kind of language whose meaning is less sensitive to the context in which it is first expressed, which can be passed on indefinitely for the benefit of others.

The possibility of deductive inference flows from the adoption of descriptive language, and is so bound up in descriptive language that we cannot be said to understand descriptive languages unless we are able to perform elementary deductions.
A sentence formed in English by conjoining two other sentences using the logical operator ``and'' {\it entails} each of the constituent sentences, and from it we may deduce the constituents.
Someone who is not able to perform such deductions cannot be said to understand English sentences formed in this way.
Similarly a sentence formed in English from two other sentences using the connective ``or'' is entailed by each of its constituents, and may be deduced from either of them.
The ability to undertake this kind of deduction is an essential to a full understanding of the meaning of the connective.
In this way we can show that a knowledge the kinds of deduction with which we are familiar in modern formal logics is implicit in an understanding of ordinary language, and we may reasonably suppose that the routing use of such elementary deductions long predated even the concept of logic, deduction or proof.

Even without the concept of deduction, our language is also capable of talking about sound derivations.
