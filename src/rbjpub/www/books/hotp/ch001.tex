% $Id: ch001.tex,v 1.1 2010/01/10 21:30:10 rbj Exp $
\def\rbjidch001{$$Id: ch001.tex,v 1.1 2010/01/10 21:30:10 rbj Exp $$}
\chapter{Rationality and Deduction}\label{RationalityDeduction}

``Rational'' is not a precisely defined term.
We may apply it to people, to their opinions, their conclusions, or to their actions, and in doing so we will usually be expressing approval.
We will consider an opinion rational if we suppose the evidence on which is it based to provide solid grounds for the belief, or an action rational if it is likely under the circumstances to secure the end for which it is construed.

These terms, however, with which we describe rationality, are not themselves precise.
It is not clear that a judgement of rationality based on this guidance will have more substance than a mere concurrence of opinion.

By contrast, the term {\it deductive}, closely related to rationality, can be made precise, and provides objective evaluation of reason which falls within its narrow remit.
The reputation of deduction has been built primarily from its successful application in the development of mathematics over the last two and a half millenia.
Throughout its history that reputation has been exploited by men who have sought to give force to their own opinions, and who therefore present as deductive reasoning which cannot properly be so described.
Many of these have been philosophers.

The notion of rationality has similarly been abused.

\section{Descriptive Language}

Communication between living beings is virtually as old as life itself.
It is an essential element of sociality, and throughout the evolution of life there has always been advantage to be derived from cooperation, influencing its direction alongside competition for resource.
