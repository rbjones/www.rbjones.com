% $Id: ch001.tex,v 1.2 2010/01/25 13:11:02 rbj Exp $
\def\rbjidchaab{$$Id: ch001.tex,v 1.2 2010/01/25 13:11:02 rbj Exp $$}
\chapter{Rationality and Deduction}\label{RationalityDeduction}

``Rational'' is not a precisely defined term.
We may apply it to people, to their opinions, their conclusions, or to their actions, and in doing so we will usually be expressing approval.
We will consider an opinion rational if we suppose the evidence on which is it based to provide solid grounds for the belief, or an action rational if it is likely under the circumstances to secure the end for which it is construed.

These terms, however, with which we describe rationality, are not themselves precise.
It is not clear that a judgement of rationality based on this guidance will have more substance than a mere concurrence of opinion.

By contrast, the term {\it deductive}, closely related to rationality, can be made precise, and provides objective evaluation of reason which falls within its narrow remit.
The reputation of deduction has been built primarily from its successful application in the development of mathematics over the last two and a half millenia.
Throughout its history that reputation has been exploited by men who have sought to give force to their own opinions, and who therefore present as deductive reasoning which cannot properly be so described.
Many of these have been philosophers.

The notion of rationality has similarly been abused.
Something may be rational even if not deductive, and hence rationality transcends the limits of applicability of deduction, to the extent that it has no bounds, any conclusion or action may be deemed rational if it seems appropriate in the circumstances, however tenuous our grounds for this belief.
However, secular authorities be they individual, institutional or cultural, may under the mantle of rationality behave as if their conclusions approached the conclusiveness of sound deduction.

Deductive and rational as discussed are attributes of conscious deliberations.
Rational has another yet broader use, not for distinguishing good reasoning from bad, but for distinguishing conclusions and conduct based on conscious deliberations from intuitive or instinctive beliefs or behaviours.
In this sense we may (and will) contrast the rationalistic philosophical tradition which begins in ancient Greece with other traditions, such as the Chinese philosophy/religion of Tao, which are explicitly opposed to rationalism as a practical philosophy.

It will be my aim to explore the scope and limits of these notions, both in relation to theoretical matters (their effects on what we can count as knowledge) and practical (how they affect what we do).

\section{Descriptive Language}

Communication between living oraganisms is virtually as old as life itself, it is an essential element of sociality, and an invaluable survival tool.
At its most primitive simply sensing the proximity of another is an elementary factor in maintaining group cohesion.
At another level of complexity observing and mimicking may play a vital role before there emerges anything we might recognise as a language.
The use of noises or gestures for communication, perhaps warning of threats or as a part of mating rituals, may be though of as first steps toward the use of language.
Talk of ``language games'' in analytic philosophy emphasises the diversity of purposes which may be realised through language and was probably introduced to broaden philosophical discussion of language from its early twentieth century on a very special kind of language which is our present concern.

Descriptive language is language specialised to the communication of objective information, rather than for the diverse other purposes which it might serve.
A cry of alarm conveys information of the moment, if it were remembered and related at a later date it would no longer serve the same purpose.
There accumulation of communicable knowledge about the world requires a kind of language whose meaning is less sensitive to the context in which it is first expressed, which can be passed on indefinitely for the benefit of others.

The possibility of deductive inference flows from the adoption of descriptive language, and is so bound up in descriptive language that we cannot be said to understand descriptive languages unless we are able to perform elementary deductions.
A sentence formed in English by conjoining two other sentences using the logical operator ``and'' {\it entails} each of the constituent sentences, and from it we may deduce the constituents.
Someone who is not able to perform such deductions cannot be said to understand English sentences formed in this way.
Similarly a sentence formed in English from two other sentences using the connective ``or'' is entailed by each of its constituents, and may be deduced from either of them.
The ability to undertake this kind of deduction is an essential to a full understanding of the meaning of the connective.
In this way we can show that a knowledge the kinds of deduction with which we are familiar in modern formal logics is implicit in an understanding of ordinary language, and we may reasonably suppose that the routing use of such elementary deductions long predated even the concept of logic, deduction or proof.

Even without the concept of deduction, our language is also capable of talking about sound derivations.
 


