% $Id: partII.tex,v 1.2 2010/03/12 13:33:13 rbj Exp $

\part{The Evolution of Irrationality}\label{partII}

My aim in this part of the book is to put forward a putative intuitive explanation of the evolution of the kind of ``irrational'' behaviour which discussed in Part \label{partI}, and then to subject this to analysis.
The explanation is based on ideas drawn from Howard Bloom's \emph{Global Brain} \cite{bloom}.

The central thesis of this book does not concern us here.
This is Bloom's answer to modern speculation about the emergence of intelligence in the elecronic networks which span the globe with ever increasing complexity.
Bloom's response is to argue that intelligent global networks in the biosphere predated electronic artefacts by billions of years, and have been present throughout the evolution of life on earth.

A major subtheme is the pervasiveness, again throughout the evolution of life on the planet, of social behaviour, and it is the evidence Bloom offers for this, together with its role in evolution which Bloom paints in vivid colours.

At a time when I was personally baffled by the course of analytic philosophy (as I saw it), the reading of Bloom's book came as an enlightenment.
After absorbing this account of evolution I (rightly or wrongly) began to find the problems intelligible.
Furthermore, to an extent much greater than that engendered by any other book on evolution, I began to read the world around me in a different way.

Such a sense of enlightenment carries in itself very little assurance.
We have all seen others who have nursed an entirely illusory conviction of enlightenment on some topic.
How can we tell whether such feelings are truthful or illusory?

Well one way would be to translate these intuitions into a formal model, to deductively derive particular predictions using this model, and to observer whether these predictions come to pass.
This is an empirical application of a general method or class of methods which I call here nomologico-deductive methods.
If we can do this we are on solid grounds.



\ignore{

This part will be devoted to the analysis of ideas from the following sources:

\begin{itemize}

\item Richard Dawkins \emph{The Selfish Gene} \cite{dawkinsSG} and \emph{The Extended Phenotype} \cite{dawkinsSG,dawkinsEP}

\item Howard Bloom \emph{Global Brain} \cite{bloom}

\item Some of my own.

\end{itemize}

The analytic method begins with carte blanche intuitive assessment, and then seeks to underpin or undermine these intuitive conclusions with some more objective grounds.

These two authors are aligned on opposite sides of one of the big debates in evolutionary theory, about whether selection occurs exclusively at the genetic level or whether there is such a phenomenon as ``group selection'', and it is part of the aim here to undertake some analysis of this debate as it is exposed in these works.
On this matter my present sympathies lie with Bloom (a member of the ``group selection squad'') against Dawkins (the most vociferous popular exponent against group selection).

Furthermore, Dawkins is an outspoken advocate of the authority of scientific institutions, and Bloom a sceptic on these matters.

Notwithstanding these points of conflict, the main target of the analysis will be an ``explanation'' of the irrationality of 20th century analytic philosophy using concepts drawn from Bloom and based primarily from evidence cited there.
This is a story which I have subjectively found to be illuminating, it has given me the impression of understanding the phenomenon.
The purpose of the analysis will be primarily to see whether this subjective sense of illumination can be shown to have any substance. 

The targets of the analysis are therefore:
\begin{itemize}
\item To comprehend the principle methods employed by Dawkins and by Bloom and to discover objective grounds for or against their validity.
\item To analyse the debate on group selection.
\item To consider whether the supposed explanation of the evolution of rationality has any better than purely intuitive support.
\end{itemize}

}
