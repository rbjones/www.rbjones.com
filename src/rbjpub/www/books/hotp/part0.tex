% $Id: part0.tex,v 1.1 2010/04/05 16:05:16 rbj Exp $

\mainmatter

\part{Prologue}\label{part0}


The book is divided into six parts, of which the first and last anticipate and recapitulate, leaving the substance divided among four.

It is a kind of analytic utopian philosophy, in which therefore, the primary concern is the future, and how it might possibly be made better than the present.
The basis for the ideas presented about the future is a certain perception of the past (most of it distant) and of the present (broadly construed as extending over a century or so).
The past is presented as the evolution, first of human intelligence (part \ref{partI}), and then of knowledge or more generally, culture (part \ref{partII}).
Part \ref{partIII} on the present (or recent past) is then a taking of stock, in which some opportunities for improvement are noted.
At last (part \ref{partIV}) we consider the future.

In considering future ``improvements'' I am concerned not only with how what we do now might be done better, but also with the challenges and opportunities presented by the developments which are now in progress and to which they may lead.
Notable among these are, on the one hand, cognitive machines sewn together with meat minds by global networks, which will transform the evolution of knowledge and culture and the other genetic engineering which will likewise transform the evolution of life on earth, including the human genome.

The challenge of anticipating and responding to such developments is great.
Calling this essay utopian invites the expectation of detailed blueprint and a totalitarian outcome.
This is not that kind of utopianism, it is a focussed consideration of how some of the things we do might be done better.
The contribution I seek to make is narrowly focussed, and concerns ``rationality''.
The manner of the contribution is analytic, it consists in analysis (broadly conceived) of rationality as it has been, how it now is, and how it might possibly be.

Though presented as a historical narrative, this is not historical scholarship.
The history is a way of making intelligible a perception of how matters now stand and of the dynamics of change.
To that end the historical perspective is contemporary, historical ideas are positioned in a development understood with hindsight.
To that end, this prologue provides a contemporary framework in the context of which the terms of the historical analysis should be understood.

This framework is a skeletal description of the kind of \emph{philosophical analysis} of which the work is intended to be an example.
The culmination of the book will be, in the epilogue (part \ref{epilogue}), a new conception of analysis.

\chapter{Philosophical Analysis}

I'm not inclined to offer prescriptions for the use of the word philosophy, or analysis of its very diverse applications.
Nor am I inclined to provide a basis for rejecting ideas simply on the grounds that they transgress some such demarkation.

My purpose in presenting a conception of philosophical analysis are, firstly, in this prologue, to make what follows more readily intelligible, and secondly, in the epilogue, to offer motivated ideas for the future of philosophical analysis.

I begin with some desiderata for philosophical analysis:

\begin{enumerate}
\item something about subject matters
\item that it be `rational'
\item that it seek to progress its subject matters from informal and intuitive beginnings in the direction of the formal and deductive
\item that the deductive skeleton of an analysis be clearly separated out from its other elements
\item that the context in which deduction takes place should be clearly understood and should be shown to be consistent
\end{enumerate}


