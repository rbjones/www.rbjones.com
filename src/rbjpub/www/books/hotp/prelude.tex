% $Id: prelude.tex,v 1.5 2010/03/12 13:33:13 rbj Exp $
\mainmatter
\def\rbjidprelude{$$Id: prelude.tex,v 1.5 2010/03/12 13:33:13 rbj Exp $$}

\chapter*{Prelude}\label{Prelude}
\addcontentsline{toc}{chapter}{Prelude}

One aim in this volume of HOT philosophy is to narrow the gaps between idea, expression and publication, and so to produce a volume which has more of the dynamic and excitement of vigorous philosophical debate.
The most important element of this is that the work is primarily concerned with new ideas, which are not yet set in stone but will emerge in the process of writing.
Secondary elements arise from the use of multiple publishing routes, some closely connected with the creative process.
The closest to the thinking edge will be the blogs and archived email discussion lists.
The writing of the book itself will be visible through regularly updated drafts published on the RBJones.com website.
The publication of the book will be by Print on Demand through amazon.com permitting the book to be made available within a very short space of time of completion of the writing.

``HOT'' has another significance: ``Honest, Open, Transparent''.

The book is an exploration in philosophical analysis, though not the application of a settled analytic method.
The purpose is to progress the evolution of a pluralistic conception of analytic method, as well as to achieve some worthwhile analysis in the selected problem domains.

The methodological pluralism is partly offered as an element of an approach to the ameliorisation of the first problem which we consider.
This is the perception that philosophy as it has been conducted in the twentieth century, is not a rational pursuit.
The perception is mine, and the first problem is to analyse this perception and see what basis there is for it.
The aim of the analysis here is to understand what transpired, by focussing on a small number of key episodes in the context of a picture in the large of the principle structures in 20th century analytic philosophy.
This analysis forms Part \ref{partI} of the book.

In Part \ref{partII} we consider some ideas from the theory of evolution which might help us to understand why people who are capable of behaving rationally and with high degrees of intelligence, often appear to behave irrationally.
The analysis here serves multiple purposes.
It continues the exploration of analytic method, and is particularly concerned with how we can analyse 

In Part \ref{partIII} we analyse some ideas which are prophylactic in intent.
The analysis is of ways of representing tentative knowledge which may be regarded as a reasonable response to an appropriate scepticism.
These constitute a systematic epistemic retreat in which instead of dogmatic assertions we record strength of the evidence supporting various alternative models of the problem domain.

\ignore{

There are many kinds of philosophical analysis;
this work results in part from my own attempts to determine and articulate yet another analytic method.
In the attempt I have slipped into a kind of analytic pluralism, a recognition of the legitimacy and desirability of working with more than one method of analysis.

My methodological deliberations have been a part of a broader uncertaintly about how to do philosophy.
This was provoked (in part) by a road block inhibiting the completion of a philosophical exposition which I had expected to be modest and uncontroversial.
I wandered in a philosophical wilderness for more than a decade, seeking an understanding of the obstacle and a way to do philosophy despite it.

The main purpose of this book is to present analysis in progress applied in two problem domains, neither strictly philosophical.
Before engaging with these problems I set the stage with a description of my original difficulty as I perceived it, of aspects of the search which followed, and of the conception of philosophy and of analytic method which resulted.

Part \ref{partI} of the book provides the background for the approach to analysis explored in the next two parts.

The first problem on which I attempt an analysis (in Part \ref{partII}) concerns aspects of the theory of evolution, and their impact upon the nature of rational enquiry.

In Part \ref{partIII} I present and analyse ideas which connect epistemological concerns with aspects of the architecture of cognitive artifacts.

}
