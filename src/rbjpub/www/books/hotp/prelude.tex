% $Id: prelude.tex,v 1.4 2010/02/06 16:19:48 rbj Exp $
\mainmatter
\def\rbjidprelude{$$Id: prelude.tex,v 1.4 2010/02/06 16:19:48 rbj Exp $$}

\chapter*{Prelude}\label{Prelude}
\addcontentsline{toc}{chapter}{Prelude}

This book is an exploration in philosophical analysis.

The subject matter of the analysis is an account of a perceived philosophical problem, of an extended search by the author for an understanding of and a resolution of that problem, of a sketchy and tentative diagnosis, and the barest outlines of an approach to its resolution.

This is presented in three parts, Part \ref{partI} describes and discusses the problem and the search for understanding and resolution.
Part \ref{partII} concerns diagnosis, Part \ref{partIII} resolution.

The resolution is a conception of philosophical analysis, and it is my aim to refine this conception in practice, exposing this to the reader, not only in the completed book, but also during the writing.
This is one element of the conception of this as ``HOT'' philosophy.
HOT off the press, coming to you as it is written.

``HOT'' has another significance: ``Honest, Open, Transparent''.
This relates primarily to the first part in which I aim to give the simple truth in describing the nature of the problem I perceived and the search which ensued.
Why is this worth mentioning?

``HOT'' in that same sense speaks also for diagnosis in which I offer in Part \ref{PartII}, and the solution in Part \ref{PartIII}.


\ignore{

There are many kinds of philosophical analysis;
this work results in part from my own attempts to determine and articulate yet another analytic method.
In the attempt I have slipped into a kind of analytic pluralism, a recognition of the legitimacy and desirability of working with more than one method of analysis.

My methodological deliberations have been a part of a broader uncertaintly about how to do philosophy.
This was provoked (in part) by a road block inhibiting the completion of a philosophical exposition which I had expected to be modest and uncontroversial.
I wandered in a philosophical wilderness for more than a decade, seeking an understanding of the obstacle and a way to do philosophy despite it.

The main purpose of this book is to present analysis in progress applied in two problem domains, neither strictly philosophical.
Before engaging with these problems I set the stage with a description of my original difficulty as I perceived it, of aspects of the search which followed, and of the conception of philosophy and of analytic method which resulted.

Part \ref{partI} of the book provides the background for the approach to analysis explored in the next two parts.

The first problem on which I attempt an analysis (in Part \ref{partII}) concerns aspects of the theory of evolution, and their impact upon the nature of rational enquiry.

In Part \ref{partIII} I present and analyse ideas which connect epistemological concerns with aspects of the architecture of cognitive artifacts.

}
