% $Id: prelude.tex,v 1.1 2010/01/10 21:30:10 rbj Exp $
\mainmatter
\def\rbjidprelude{$$Id: prelude.tex,v 1.1 2010/01/10 21:30:10 rbj Exp $$}
\chapter{Prelude}\label{Prelude}

The reader's understanding of the philosophical ideas presented in this book will be assisted, I believe, if I take the unconventional step of presenting here a certain amount of autobiographical material which will help to explain the nature and importance of the problems which the book addresses.

The introduction provides a first sketch of the material which is to follow, an outline of my approach to a solution of certain problems which have seemed to me important.
This prelude gives some clues about why it is that I came this conception of the important problems of philosophy.

There is a single problem which dominates all others and I will first try to describe how I came to attach such importance to that problem.
In the synthesis which I have woven in my attempts to address this problem there are elements which are dictated not purely by the character of that problem, but also by various other problems which have interested me.
In this prelude I describe first the origins of the principle problem, and then a minimum of material on those other problems which fit into the synthesis with which I address it.

The first was my first approach to philosophical thinking.

At the age of 11 I was sent as a boarder to an English grammar school.
In the absence of any special arrangement pupils were required to attend the local Anglican church every Sunday.
This usually included a substantial sermon, and inevitably I found myself thinking about what was being said.
Since the whole was predicated on the existence of God, to make sense of these sermons I had first to make sense of the idea of God, and this I attempted, at length, during this regular period of reflection imposed upon me by my betters.
I though through a variety of ideas about what God might be, trying to find one which squared with the kinds of thing which were being said in the sermons, and which {\it made sense} to me, in which I could believe.
Eventually I abandoned the search.
I could make no sense of what I was being told, and concluded ultimately that it made no sense.

At the beginning of my second year pupils were given the choice whether to begin lessons preparatory to confirmation in the Anglican Church.
I was by then quite definite in my disbelief and declined, so my deliberations were entirely settled in that first year.
I have never since been tempted even to consider arguments for or against the existence of God, for me the concept lacked meaning.

The greatest difficulty in this process, so far as I can now recollect, was the consideration that so many important and distinguished people around me and in the world at large professed belief in God.
How could so many be mistaken or confused?
This consideration however, never stood a chance of prevailing.

Though I did not consider the matter at the time, in my opinion this early decisiveness in important philosophical matters is not evidence of intellectual precocity, but rather of an aspect of character which we may call {\it independence of mind}, a singular disinclination to take matters on trust.

My purpose in presenting this problem and my resolution of it is primarily to illustrate by contrast the difficulty which I found myself in many years later and which is the single greatest influence shaping the philosophical deliberations which have lead to the composition of this book.
In 1951 W.V.Quine published an essay entitled ``Two Dogmas of Empiricism'' \cite{quine51}.
My own first acquaintance with this essay was probable in about 1975, at which time, though I found it disagreeable and unsound, it did not represent for me {\it a problem}.

About 35 years later than these early deliberations I belatedly became aware of extent of the influence which had been exerted by a philosophical essay published by W.V.Quine under the title ``Two Dogmas of Empiricism''.
The essay was not then new to me, my first acquaintance with it was probably twenty years before, and I then attached little importance to it.
The importance was attached to it by others, by the enormous and irrational influence which it appeared to exert, the testimonial thus rendered to the ultimately rhetorical character of analytic philosophy beneath its veneer of rational enquiry.



