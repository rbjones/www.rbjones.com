% $Id: prelude.tex,v 1.2 2010/01/25 13:11:02 rbj Exp $
\mainmatter
\def\rbjidprelude{$$Id: prelude.tex,v 1.2 2010/01/25 13:11:02 rbj Exp $$}
\chapter{Prelude}\label{Prelude}

I should like to be able to present my philosophical ideas in their own right, without troubling the reader with extraneous details of my own personal circumstances, and motivations, or the accidents by which life has lead me down this path.
However, I have come to the opinion that an attempt at such a presentation would make an understanding much harder for the reader to achieve.
I have elected instead to try to confine material of an autobiographical character in this prelude, which will serve either to set the readers mind on a better track for understanding the sequel or to free him of any inclination read further.

There is a single problem which dominates all others and I will first try to describe how I came to attach such importance to that problem.
In the synthesis which I have woven in my attempts to address this problem there are elements which are dictated not purely by the character of that problem, but also by various other problems which have interested me, and by my own character.
In this prelude I describe first the origins of my principle problem, and then a minimum of material on those other problems which fit into the synthesis which is my response to the problem.

\section{My Big Problem}

Let me first describe a small problem, my first approach to philosophical thinking.

At the age of 11 I was sent as a boarder to an English grammar school.
In the absence of any special arrangement pupils were required to attend the local Anglican church every Sunday.
This usually included a substantial sermon, and inevitably I found myself thinking about what was being said.
Since the whole was predicated on the existence of God, to make sense of these sermons I had first to make sense of the idea of God, and this I attempted, at length, during this regular period of reflection imposed upon me by my betters.
I though through a variety of ideas about what God might be, trying to find one which squared with the kinds of thing which were being said in the sermons, and which {\it made sense} to me, in which I could believe.
Eventually I abandoned the search.
I could make no sense of what I was being told, and concluded ultimately that it made no sense.

At the beginning of my second year pupils were given the choice whether to begin lessons preparatory to confirmation in the Anglican Church.
I was by then quite definite in my disbelief and declined, so my deliberations were entirely settled in that first year.
I have not since been tempted even to consider arguments for or against the existence of God, for me the concept lacked intelligible meaning.

The greatest difficulty in this process, so far as I can now recollect, was the consideration that so many important and distinguished people around me and in the world at large professed belief in God.
How could so many be mistaken or confused?
This consideration however, never stood a chance of prevailing against my own judgement.

Though I did not consider the matter at the time, in my opinion this early decisiveness in important philosophical matters is not evidence of intellectual precocity, but rather of an aspect of character which we may call {\it independence of mind}, a singular disinclination to take matters on trust.

My purpose in presenting this first problem and my resolution of it is primarily to illustrate by contrast the difficulty which I found myself in many years later and which is the single greatest influence shaping the philosophical deliberations which have lead to the composition of this book, and to the approach to philosophy which it exemplifies.
In 1951 W.V.Quine published his essay entitled ``Two Dogmas of Empiricism'' \cite{quine51}.
My own first acquaintance with this essay was probable in about 1975, though I have now no recollection of what I then thought of it.
I don't believe it had any impact on my philosophical views.

I did have difficulties as an undergraduate studying mathematics and philosophy.
One of my principle desires in doing philosophy was to address substantive rather than verbal problems.
I sought by the formulation of definitions to work with chosen concepts relevant to some problem domain rather than to use the concepts already in place and address problems couched in such terms which would then be liable to degenerate into investigation of their meanings.
The demise of ``ordinary language philosophy'', had not yet reached provincial English Universities, but in any case never inspired the kind of detachment from ordinary language which I sought in my philosophy.

Perhaps 20 years later I belatedly became aware of extent of the influence which had been exerted by Quine's essay.
I was then edging my way back into philosophical thinking, and sought to articulate philosophical background for some ideas about architecture for Artificial Intelligence.
I suddenly became aware that these philosophical underpinnings simply could not be articulated, because Quine had undermined concepts fundamental to the account.

This was now for me a problem.
I perceived that the point of view I wished to articulate would not be understood by philosophers, who, it seemed would reject it simply because it made use of distinctions which Quine was supposed to have shown to be untenable.

The problem was worse than that.
Not only did Quine's arguments leave me cold, he seemed not even to have advanced an intelligible thesis.
I found great difficulty in believing that Quine could have himself believed what he wrote, but worse yet, it seemed to me that the wholesale abandonment of the analytic/synthetic distinction by professional philosophers could not have been a rational response to the arguments put forward by Quine.

The concepts essential to my philosophy had rejected by professional philosophers, but, it seemed to me, this had not been done on any rational basis.
There could be no point in a detailed rebuttal, the rejection of these fundamental concepts could not have been rational, and a rational r

\section{Some Earlier Motivations}


\section{From There to Here}

