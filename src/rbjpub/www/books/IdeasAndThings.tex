% $Id: IdeasAndThings.tex,v 1.2 2010/02/10 11:17:12 rbj Exp $
\documentclass[10pt,titlepage]{book}
\usepackage{makeidx}
\usepackage[unicode,pdftex]{hyperref}
\pagestyle{headings}
\usepackage[twoside,paperwidth=5.25in,paperheight=8in,hmargin={0.75in,0.5in},vmargin={0.5in,0.5in},includehead,includefoot]{geometry}
\hypersetup{pdfauthor={Roger Bishop Jones}}
\hypersetup{colorlinks=true, urlcolor=black, citecolor=black, filecolor=black, linkcolor=black}
\usepackage{html}
\usepackage{paralist}
\usepackage{relsize}
\bodytext{BGCOLOR="#eeeeff"}
\makeindex
\newcommand{\indexentry}[2]{\item #1 #2}
\newcommand{\glossentry}[2]{\item #1 {\index #1 #2}}
\newcommand{\ignore}[1]{}
\def\ouml{\"o}

\title{Ideas and Themes, Logic and Philosophy}
\author{Roger Bishop Jones}
\date{\ }

\begin{document}
\frontmatter

\begin{titlepage}
\maketitle

\vfill

%\begin{abstract}
%A kind of intellectual autobiography, as a catalogue of ideas organised into themes and slotted into a chronology..
%\end{abstract}

\begin{centering}

{\parskip=0.3in
This book is dedicated to my father
\vfil
{\relsize{+2} Reginald Arthur Bishop Jones}
\vfil
and my to mother
\vfil
{\relsize{+2} Vina Bishop Jones}
}

\vfill

\footnotesize{
Started 2005-01-26

Last Change $ $Date: 2010/02/10 11:17:12 $ $

\href{http://www.rbjones.com/rbjpub/www/papers/p006.pdf}{http://www.rbjones.com/rbjpub/www/papers/p006.pdf}

Draft $ $Id: IdeasAndThings.tex,v 1.2 2010/02/10 11:17:12 rbj Exp $ $

\copyright\ Roger Bishop Jones;

}%footnotesize
\end{centering}

\thispagestyle{empty}
\end{titlepage}

{\parskip=0pt\tableofcontents}


\mainmatter
\chapter{Introduction}

This ``book'' contains notes on some of my ideas, and on the themes which tie them together.
Ideas which go places are usually collected together in substantial pieces of work and documented as such, but many of my ideas do no more than provide a fragment of the lens through which I peer upon the world.

I have adopted both a thematic and chronological presentation in the first two parts of the book, with various supplementary materials provided in a third part.

As it stands the material includes what now seems to me extraneous autobiographical material, which I expect to remove.

I offer no warranty that ideas presented here as if they were my own, really are my own ideas.
I do not wilfully present as my own ideas which I know came to me from some other, but I do not pretend to remember the source of everything which I learn, or the sources which contribute to my own innovations.
An idea, when its time has come, will occur in many thinkers at once.

\part{Themes}

\chapter{Artificial Intelligence}

\section{Introduction}

My claim to have had any ideas of interest which properly belong to the field of Artificial Intelligence is moot.
Its inclusion here among the principle themes, rather than as a small collection of oddments, reflect the role which the automation of reason plays in the motivation of much which belongs properly to the other themes.

So I will here mention, not just the few ideas which belong to artificial intelligence, but also the ways in which these ideas have influenced my thought in foundational and philosophical matters.

The first think to say is that I tend to the logical side of AI.
I mostly think of AI as a way of engineering computers that are cleverer at solving crisp problems of a kind which one would be able to give a precise formal mathematical or logical expression.
There is also a part of my ``Utopian'' or practical philosophy which regards networks and computers as socially significant, but this line of though is very little developed.

\subsection{The Logical Side}

Because this line of thought is about how to engineer the kind of intelligence which can solve logical and mathematical problems, it frequently crosses paths with my thinking about the Foundations of Mathematics.

\subsection{}

I like to think about {\it the big picture}.
For me this involves thinking about the distant future and the relationship between natural and artificial intelligence in that context.
In the latter part of my life it has seemed natural to construe this in the context of pervasive networked computing, in which intelligence is a characteristic of the whole, as well as of many of its parts.
In which it becomes decreasingly possible to localise any particular capability.

This networked intelligence is sometimes called the {\it Global Brain}\index{Global Brain} a term coined in 1982 by Peter Russell \cite{russellp82}.

\section{The Representation of Knowledge}

There is a line of thinking here which runs alongside my foundational thinking.

The reason for the connection is that fairly early on 

\section{The Automation of Reasoning}

\section{The Evolution of Analytic Intelligence}

\section{The Holistic SuperBrain}

\chapter{Foundations}

\chapter{Philosophy}

\chapter{Other Matters}

\part{As it Went}

\appendix

%\addcontentsline{toc}{chapter}{Glossary}
\chapter{Glossary}\label{glossary}

\begin{description}
\item[CESG]{\index{CESG}} Communications and Electronic Security Group (a part of GCHQ).
\item[class]{\index{class}} A large collection.
\item[GCHQ]{\index{GCHQ}} Government Communications Headquarters, UK.
\item[HOL]{\index{HOL}} Higher Order Logic, a specification language, a formal deductive logic, an interactive proof tool.
\item[NuPrl]{\index{NuPrl}} A tool supporting proof development in a ``New Proof/Program Refinement Logic'', which is a constructive type theory.
\item[PRG]{\index{PRG}} The Programming Research Group, at the University of Oxford.
\item[set]{\index{set}} A small collection.
\end{description}

\backmatter

\addcontentsline{toc}{chapter}{Bibliography}
\bibliographystyle{alpha}
\bibliography{rbj}

\addcontentsline{toc}{chapter}{Index}\label{index}
\twocolumn[]
{\small\printindex}

\end{document}
