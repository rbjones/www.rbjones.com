% $Id: bf.tex,v 1.2 2010/04/05 16:05:16 rbj Exp $
\chapter{Prelude}\label{Prelude}

In this chapter I sketch some history of the ideas which concern us leading to the revolution in logic which begins with the work of Frege.
It is a very selective history of philosophy and science from its origins in ancient Greece through to nineteenth century Europe.

Philosophy in the broadest sense of that tradition which for a some time in the 20th century was known as ``Analytic Philosophy'' has, to a substantial extent been concerned with {\it reason}.
While sharing this characteristic with Mathematics, it has in some respect remained poles apart from Mathematics.

For the most part Mathematics has been a discipline characterised by the rigour of its methods and the certainty of its results.

\section{The First Revolution}

The first ``revolution'' in philosophy was its birth, which we locate in ancient Greece.

Philosophy in ancient Greece was not so narrowly cast as it is now, it included science and often placed great emphasis on the importance of mathematics.

The distinctive feature which I would like to emphasise concerns the discovery and justification of knowledge.
Philosophy was born from a new disrespect for authority.
Instead of taking knowledge in tablets of stone, or by word of mouth from those who preceded them, the philosophers of ancient Greece {\it observed} the world about them and {\it thought} about it.
They tried to make sense of the world for themselves, basing their claims to knowledge on observation and reason.

In this disrespect for authority begins the tendency of philosophy to revolution.
In the millennia which followed the branches of knowledge which could be systematically and reliably developed became separate sciences in their own right, leaving to philosophy those less certain domains in which stability proved elusive.


\section{Pythagoras: c 532 b.c}

\index{Pythagoras}

I note here that Pythagoras has an interesting metaphysic, considering the world to be made of numbers.
We will end up in a position separated from Pythagoras by a hairs breadth on this topic.

%\section{Paremenides: c 515-450 b.c.}

%\index{Parmenides}

%Possibly the first rationalist?

%\index{Parmenides|)}


%\section{Socrates: 470-399 b.c.}
%\index{Socrates}

%A mention of the philosophical methods of Socrates and of his brand of scepticism.

%\index{Socrates}

\section{Aristotle: 384-322 b.c.}
\index{Aristotle}

Aristotle is of particular interest for our story for his having codified logic, though the main significance of this codification is its inadequacy.

\subsection{Epistemological Dichotomies}

The distinction between {\it necessary} and {\it contingent} truths is found in Aristotle {\it On Interpretation}.

The terms {\it analytic} and {\it synthetic} were known to the Greeks (see \cite{euclidEL1}) as kinds of proof, which we might now describe as {\it backward} and {\it forward} proofs respectively.

This usage is orthogonal to the more modern use to denote kinds of proposition, which we will discuss later.

\index{Aristotle}

\section{Chrysippus: c.280-c.207 b.c}

A stoic with a claim to the invention of propositional logic.

\section{Phyrro: -275 b.c.}
\index{Phyrro}

Phyrro is most important for taking scepticism to its ultimate end.

It is valuable to hold phyrronean scepticism in mind when considering more modern sceptics, and also in formulating or evaluating varieties of methodological scepticism.

\index{Phyrro}

\section{Galileo: 1564-1642}
\index{Galileo}

Galileo is our representative of the rise of science which in modern Europe stimulated a new era in philosophy as it had in ancient Greece.
This was at once a revival of free scientific enquiry after an extended period of subservience to religious precept, and an advance on the methods prevalent in ancient Greece.

Religious authority endorsed academic philosophers who in turn were primarily concerned with interpreting the doctrines of classical Greece, particularly those of Aristotle.
These philosophers were considered authoritative not only in the domains which we now accept as philosophical, but in the whole of science.
Galileo's tendency to believe and affirm the evidence of his own eyes in preference to the writings of Aristotle was unsympathetically received.

The Greeks recognised more than one kind of explanation, but attached particular importance to teleological explanations.
The emphasis on finding {\it purposes} obscured more pragmatic measures of the merit of a theory.

In Galileo we see clearly emerging a {\it nomological-deductive} model of science.
A {\it nomological} proposition is a proposition which states some physical law.
Galileo observed nature carefully and sought to formulate quantitative scientific laws which captured the regularities which he observed.
Though these general laws might not {\it explain} the phenomena in any sense which would be satisfactory to Aristotle, they did allow something which was in practice more useful.
They allowed detailed conclusions to be drawn by mathematical deduction about many details of behaviour which had not yet been observed.
They permitted diverse theoretical conclusions to be derived from a small number of basic laws, and then applied in engineering projects.

In this model two key elements are combined.
The use of observation and of reason rather than authority and prejudice, and the adoption of a kind of explanation which has pragmatic rather than academic merits.

Our story from here is more concerned with philosophers than with scientists, but is very concerned with the respective roles of observation and reason in the discovery, justification and application of knowledge.
Philosophers often take more extreme views than scientists, and the reasonable balance we find in Galileo is often lacking.

Over the centuries which followed the work of Galileo an understanding of the respective roles of reason and observation evolved in part through conflict between two complementary kinds of philosopher known as rationalists and empiricists.
Rationalists are philosophers who emphasise the role of reason in the discovery and justification of knowledge, empiricists by contrast emphasising the role of the senses.

\index{Galileo}

\section{Hobbes: 1588-1679}
\index{Hobbes}

Hobbes is distinctive for having applied systematically something like the nomological-deductive method to human nature and political theory.

\section{Descartes: 1596-1650}
\index{Descartes}

Descartes is our primary representative of rationalism.

Though the essence of the nomological-deductive model, which I have ascribed to Galileo, is the formulation of general physical laws describing the behaviour of the world about us, Descartes looked upon the work of Galileo and found it to be too fragmentary.

Even Galileo's emphasis on observation was a far cry from the experimental science of today.
In these early days of science spectacular advances could be made simply by paying attention to and thinking carefully about the kind of phenomena which were there to be seen without elaborate contrived experimentation.
The scientific laws being discovered in this way, most notably Newton's laws of motion and gravity, could be simple and pervasive in their implications.

It is not unreasonable for a more abstract thinker, finding this growth of scientific knowledge insufficiently well organised, to think reason the antidote.

But more emphatically Descartes' concern was not the fragmentary nature of the knowledge discovered by exceptional scientists such as Newton and Galileo.
It was the general unreliability of all that passed as knowledge.
Speaking specifically of philosophy he says in his {\it Discourse on Method} \cite{descartesDOM}

\begin{quotation}
I shall not say anything about philosophy, but that, seeing that it has been cultivated for many centuries by the best minds that have ever lived, and that nevertheless no single thing is to be found in it which is not subject of dispute, and in consequence which is not dubious, I had not enough presumption to hope to fare better there than other men had done.
\end{quotation}

Descartes responded with a {\it new method}.
First of all he sceptically sweeps aside all that has hitherto passed for knowledge, seeking to rebuild on a more solid foundation.

This is our first modern model of philosophical revolution.

The most important thing he gives us as the fruits of his labour (for present purposes) is his method.
This I will take as a watershed in {\it epistemology} and in {\it scientific methodology}.

Briefly Descartes' method may be described as follows.
First one must doubt all that is not beyond doubt, and then one must derive all other knowledge by careful reasoning modeled on the best mathematical derivations from the few fundamental principles which are ``presented to my mind so clearly and distinctly that I could have no occasion to doubt it''.

\index{Descartes}

\section{Locke: 1632-1704}
\index{Locke}

Locke is our first representative of {\it empiricism}.
His philosophy is an early attempt to produce a philosophy which fitted well with the viewpoint of empirical science.

\subsection{Empiricism}

We have noted two sources of knowledge, observation and reason, which grew to contest religion and the authority of ancient Greece with the emergence of modern philosophy.
We have visited some representative rationalists who have emphasised, or perhaps overstated, the role of deductive reason in establishing the truth.
We have also discovered that there may be distinct kinds of proposition one of which, necessary propositions, seems specially suitable for deductive demonstration.

Now we come to the other side of the coin.
The British empiricists Locke, Berkeley and Hume, each in their own way, affirmed that knowledge of our world depends essentially upon the evidence of our senses.
They were apt perhaps to regard such knowledge as might be obtained in other ways as rather trivial and insubstantive, and to take a sceptical view of how much, and what kind of knowledge,could be obtained even with copious observation.

The tension between rationalism and empiricism provides two main ingredients for our story.
Firstly this tension is a motor in the refinement of the partition of knowledge into analytic and synthetic which has fallen into disfavour in recent times, but which is a vital part of the future we paint.
Secondly the empiricists, especially David Hume, provide new kinds of scepticism, a primary source of philosophical revolution, of which I wish to take note.

\index{Locke}

\section{Leibniz: 1646-1716}\label{Leibniz}
\index{Leibniz}

Though Descartes may represent the most decisive and revolutionary breakpoint between the ancient and the modern, for our present purposes it is our second representative of rationalism which has perhaps the more interesting ideas to contribute.
When we come in Chapter \ref{PerfectCadence} to our vision for the future the connections with the thought of Leibniz will resonate throughout.

\subsection{Rationalism}

\index{rationalism}
Let me begin by pointing out in the simplest terms the basic mistake which gives rise to rationalism, so that we can then note that Leibniz, though undoubtedly a rationalist, knows it all and demands a more subtle rebuttal.

We have introduced the twin pillars with which modern philosophy dismissed the yoke of authority and precedent.
They were {\it observation} and {\it reason}.
So far we have seen little account of how the role of these two pillars may be differentiated.
As our story unfolds there will emerge from the mists, alongside these two {\it sources} of knowledge two distinct {\it kinds} of knowledge.
Knowledge for which reason alone suffices, and which is therefore known as {\it a priori}, by contrast with knowledge dependent upon observation, {\it a posteriori}.
We will later discover from empiricists that the kinds of knowledge which can be obtained {\it a priori} are insubstantial, concerning exclusively verbal matters which have no bearing upon the material world, and that all true knowledge of the real world must be obtained through the senses.
The defect of rationalism is in failing to recognise this fundamental divide between the kinds of propositions which can be known, and in supposing that reason unaided will assure us of truths about the real world rather than tautologies rooted in language.

\index{rationalism}
 
\subsection{Dichotomies}

\index{dichotomies}
The place of Leibniz in this story is not just as a rationalist who provides our most comprehensive early view of how much might be achieved by means of logic and its mechanisation, but also as our first introduction to the dichotomies which help us to realise that potential.

In his {\it Monadology} \cite{leibnizMON} Leibniz talks of two kinds of truth:
\begin{itemize}
\item those of {\it reasoning}.
\item those of {\it fact}.
\end{itemize}
of which:
\begin{itemize}
\item truths of reasoning are {\it necessary}.
\item truths of fact are {\it contingent}.
\end{itemize}
and:
\begin{itemize}
\item the reason for necessary truths can be found by {\it analysis}.
\end{itemize}

Leibniz also connects (in \cite{leibnizNCT}) these two kinds of truth with {\it a priori} and {\it a posteriori} knowledge.
Truths of reason can be known {\it a priori}, truths of fact can be known to mortals only {\it a posteriori}, through experience.
To God the creator's greater wisdom all is known {\it a priori}.

We witness here the first place in our story that these two distinct kinds of truth are identified.
A distinctive method is identified for establishing necessary truths, viz. analysis.
This also represents a stage in the evolution of the word {\it analysis} the history of which provides one of the threads in our story.

In ancient Greece the word {\it analysis} is used for a {\it kind of proof}.
This corresponds to what computer scientists might call a ``backward'' proof, meaning, a proof which is constructed by starting with the conjecture, and then analysing this conjecture showing how it can be derived from simpler conjectures until a proof from acceptable premises is obtained.
Since there is no presumption about the character of the ``acceptable premises'' to which an analytic proof reduces its conclusion, analytic proofs in this sense are not confined to demonstrating necessary propositions.
A contingent proposition might be the subject of an analytic proof in which it was shown to follow from accepted laws of physics.
This is how the deductive-nomological method works.

With Leibniz for the first time in our story the term {\it analytic} is used for a kind of proof, {\it explicitly associated} with {\it a kind of proposition}, {\it necessary} propositions.
The class of propositions which Leibniz considers necessary and thus provable is similar to those we might today regard in the same light, of which the paradigm cases have usually been those of mathematics.

\index{dichotomies}

\subsection{Sufficient Reason}

\index{reason!sufficient}
So Leibniz recognises that only certain propositions are {\it necessary} and that only those propositions can be analytically demonstrated.
What makes him a rationalist?

Leibniz's rationalism appears in his {\it principle of sufficient reason}, which in effect states that everything which is true can be proven, you just might have to cast around a bit to find the necessary premises, and indeed you might find the proof beyond you (but it won't be beyond God).
It is the kind of premises involved which make the difference between a necessary and a contingent proposition.

Leibniz regards all propositions as having subject predicate form, in the final analysis, and in all true propositions the truth is owing to the containment of the subject in the predicate.
This is the formula later to be used by Kant, which is often thought a narrow account of analyticity by those who now doubt that all necessary propositions can be considered logically to be of subject-predicate form.
We will see later that this position is more tenable than might be thought, and we will find Leibniz's rationalism has surprising similarities with a modern logical empiricism.

\index{reason!sufficient}

\subsection{Universal Language}

\index{characteristica universalis}
\index{calculus ratiocinator}
It is to Leibniz that we owe the idea of a {\it characteristica universalis} (universal language), equipped with a {\it calculus ratiocinator} (calculus of reasoning).
Leibniz envisaged that any problem could be solved by translating the problem into the universal language and then calculating the solution using the calculus of reasoning, which he sought to automate as a calculating machine.

An important part of our aspirations is just this same goal.
The calculating machines are now with us.
Advances in mathematical logic have now furnished us with adequate languages.
A mathematical model of the Universe formalised in a mathematical foundation system implemented on a digital computer would provide a semi-decision procedure for a large subset of mathematics and science.

\index{Leibniz}

\section{Berkeley: 1685-1783}
\index{Berkeley}

Though generally considered an empiricist, Berkeley's philosophy is motivated by goals diametrically opposed to those of other empiricists.
He was in some respects a throwback to days past when philosophy served primarily the interests of religion, and his main purpose was to remedy the defect in Locke's philosophy that it made God an inessential hypothesis in the explanation of the universe.

Berkeley simplified Locke's philosophy by eliminating the distinction between ideas and the outside world.
This made the persistence of material objects dependent on their continually being perceived, and hence upon the all perceiving supreme being.
That Berkeley defended this thesis as being more conformant with common sense tells us only that there are few limits to the absurdity of doctrines which philosophers will seek to buttress in this way.

Berkeley is our first {\it idealist}.
An idealist is one who denies the reality of physical objects as distinct from the ideas we have of them.
Idealism, and that more specific form of idealism {\it phenomenalism} are important features of empiricist thought right down to the present day, to which I will take exception right away.

Let me say firstly that idealism is conceived to solve a particular problem which in fact calls for no solution and is not helped by this kind of solution.
In Berkeley's special case one might say that it is conceived of to fix the problem that God would otherwise be inessential, however there is a slighly less harsh interpretation which is more useful for our present purposes.
From the beginning of empiricist thought the emphasis on sensory data as a source of knowledge is tempered with the belief that this data is not a reliable indicator of what there may be in the material world of which the sense data are taken to be signs.
Because the connection between sense data and reality is contingent, it may be argued sceptically that the sense data fail to confer any true knowledge.

Idealism is the attempt to overcome this epistemic gap by redefining the real world.
If we re-interpret our claims about the real world as claims about what we know about the real world, then we can perhaps then be said to have true knowledge.

\index{Berkeley}

\section{Hume: 1711-1776}
\index{Hume}

Emphasises the trichotomy:
\begin{enumerate}
\item relations of ideas
\item matters of fact
\item nonsense
\end{enumerate}

which was later to be central to logical positivism.
Unlike the logical positivists however, having enunciated this trichotomy, he models his philosophy on empirical science rather than on mathematics, i.e. as concerned with synthetic truths, mainly about the nature of man.

Hume's scepticism, by emphasising what kinds of inference are non-demonstrative provides the basis for a more accurate positioning of the analytic/synthetic dichotomy than had previously been obtained (though this may not have been his intention).

\index{Hume}

\section{Kant}
\index{Kant}

The principal relevance of Kant is in his disputing the tidyness of our triple dichotomies.
In this he finds fault with Hume, who (though not in these words) considered mathematics to be analytic.

Kant talks for the first time of a {\it kind of propositions} called analytic, but denies that this encompassess all those truths which can be known {\it a priori}, regarding mathematics as {\it a priori} but {\it synthetic}.
He gives explicit definitions of the term {\it analytic} which correspond quite closely to what was said by Leibniz about truths of reason.
One definition is about their following from the law of contradiction, (a {\it proof theoretic} definition).
The other is about containment of subject in predicate.

While later philosophers (notably Frege) have found fault with these definitions, the real problem is in the conclusions which he draws from them, particularly in the narrow scope which he inferrs for analyticity.

\index{Kant}

\section{The Status of Mathematical Truths}

In setting the context for the logical revolution we must mention some philosophy, for it was a philosophical fallacy which Frege's work was intended to refute.

David Hume, in his sceptical writings, had strengthened the distinction between logical and contingent truths by conceding that mathematics concerned ``relations between ideas'' and that mathematical truths (and few others) are susceptible of demonstrative proof.
In particular he asserted that matters of fact could not be so demonstrated and devoted some energy to pointing out the many kinds of conclusions which are in principle not susceptible of demonstrative proof.
With Hume we have, in spirit if not in letter, an analytic/synthetic dichotomy of propositions associated with {\it a priori} and {\it a posteriori} justification respectively.

Kant, disturbed from his slumbers by reading Hume, attempted a definition of the terms analytic and synthetic and countered Hume with the claim that mathematics consisted of {\it synthetic} truths which could nonetheless be justified {\it a priori}. 

Frege then aimed to put the story straight, at least in relation to arithmetic if not geometry, and it is this purpose which motivated the step beyond reconstructing the mathematical foundations of analysis (reducing analysis to arithmetic) to the logical foundations of mathematics, in which arithmetic itself is shown to be founded in logic.









