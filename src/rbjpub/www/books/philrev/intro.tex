% $Id: intro.tex,v 1.3 2008/06/26 19:00:58 rbj Exp $
\chapter{Overture}\label{Overture}

[{\it

This book project I no longer consider viable, but I may be able to plunder some of the material for other projects, and I may continue to work on some aspects of it in the hope that it may later fit in elsewhere.

The primary reasons for the lack of viability are two:
\begin{itemize}
\item There is too much here that requires a more scholarly treatment than I could ever give it.

\item The principle readership for a book like this would have to be philosophers, and though I remain keen to write philosophically, I am now happy that I don't need to write for philosophers.

\end{itemize}

}]

\section{Revolution}

Possibly the most significant revolutions which may be associated with philosophers are social or political revolutions involving the violent overthrow of establish governments of nation states.
The role of Rousseau in providing an ideological basis for the French revolution, and that of Marx in the Russian revolution are conspicuous examples.

Important though these are, they are not my present concern, which is more with logic and science than with politics and morality.
Revolution is endemic to philosophy itself, not just an occasional consequence of some kinds of philosophy.

The purpose of this essay is to give a personal perspective on an abortive revolution in philosophy.
A revolution which was by some expected to flow from a new epoch in the history of logic, but which was overwhelmed by counter-revolt and muddle.
I aim to give an account of this abortive revolt, not for the sake of history, but as a prelude to a prospectus for what may yet come.

\section{History}

The material leading to this prospectus is primarily but selectively historical.
The history begins in ancient Greece and makes its way speedily to the nineteenth century, providing just enough context for an understanding of the significance of the revolution in logic which begins with the work of Gottlob Frege.
Here the pace of our narrative slows and its focus narrows around some of the developments in logic which have taken place over the past 150 years.
We next turn from technical advances in logic to their impact on philosophy, the hopes inspired and the achievements and disappointments which ensued.
The impact of logic on philosophy in the twentieth century is what I have referred to as an {\it abortive coup}.
My final chapter (the point of it all) is devoted to speculation about how it might eventually be when the revolution in logic is assimilated into the philosophy of the future.

It has been noted, for example by Peter Hylton in \cite{hyltonREAP}, that writings in the history of philosophy are often more philosophical than they are historical.
Hylton in that work brings new standards of rigour into the history of philosophy.
My purposes here are wholly philosophical.
I have slipped into a kind of history after coming to believe that the philosophical ideas which I want to explore and present cannot be made intelligible in any other way.
My standards of historical scholarship are very weak indeed, and my attempts at history are unlikely to have any merit unless they help to make intelligible ideas for the future of philosophy which are found to be of interest.

\section{Themes}

Interwoven in this history are a number of themes which lead up to and lend significance to the logical revolution in philosophy which is our focal point.

In this introductory chapter I hope to prime the reader in these themes, so that she may more easily see their point and trace them through the chapters which follow.

In an attempt to organise my introduction to these various themes I have more or less arbitrarily collected these themes into bundles, each of which occupies one of the remaining sections of this chapter.

They are:
\begin{description}
\item[Logical Truth]
The ``revolution in logic'' which I discuss in chapter \ref{RevolutionInLogic}
yields (I will argue) near complete formalisation of {\it Logical Truth}, and is the culmination of two thousand years of philosophical evolution of our understanding of this concept.

\item[Revolutions]
A concern for certain particular revolutions which did or did not happen, and one which may or may not happen in the future leads me into a more general consideration of how often very radical changes take place in philosophical thinking, only later to be reversed or superseded.
Is this a sign of vigour or a symptom of disease?

\item[Rationality]
The significance of logic lies in its contribution to that more broadly encompassing notion {\it rationality}.
The revolutionary character of philosophy casts doubt upon its rationality, and its resistance to reformation in the light of new logical methods strengthens those doubts.
Standards of rationality depend not just upon available technique, but on cultural practices and institutional norms.

\end{description}

\section{Logical Truth}

To speak of {\it logical truth} is implicitly to affirm that there are different {\it kinds} of truth.

\subsection{Kinds of Truth}

The epistemic importance of the methods of logic depends upon there being a discernible division in the kinds of truth which concern us.
This is the difference between necessary and contingent truths, which has traditionally been considered closely related to a division of propositions between analytic and synthetic, and of kinds of justification into {\it a priori} and {\it a posteriori}.

Because of the importance of these distinctions we trace their history through our story, and we relate this also to two kinds of philosophical attitude, rationalism and empiricism, whose conflict has fuelled the refinement of our understanding of the dichotomies.

Emphasis on different ways of reaching the truth is the key element of the partial classification of philosophers into {\it rationalists} and {\it empiricists} the debate between which may have contributed to, and perhaps been partially resolved by, the identification of two different kinds of truth.
Multiple dichotomies have been discussed and related.
The {\it analytic/synthetic}, the {\it a priori/a posteriori} and the {\it necessary/contingent} dichotomies.
The logical revolution which we discuss provides the latest stage in the development of our understanding of these dichotomies.

\subsection{Rationalism and Empiricism}

\index{rationalism}
\index{empiricism}

While we use ``rationality'' as a generic label for laudable epistemic standards, the term ``rationalism'' has acquired a distinct sense of its own.
Rationalism consists in emphasising the role of reason in the justification of knowledge, possibly (but not always) at the expense of that of observation or experience.
Empiricism is its converse in which the importance of the senses is emphasised, possibly to the neglect of reason.

The opposition of two evolving attitudes towards knowledge provides fuel for the revolutions, which themselves conspire to {\it evolve} our understanding of two {\it kinds} of knowledge, the {\it a priori} and the {\it a posteriori}.


\section{Revolutions}

Two revolutions are of special interest.
The first a revolution in {\it logic}, the second a revolution in {\it philosophy} which has yet to follow from it.
Part of our problem is to understand why the philosophical revolution has not yet been forthcoming, part to understand whether and how it might yet arise.
This is set in the context of philosophy viewed as chronically unstable.
Past revolutions are both a guide to the prospects for future revolutions, and a record of a pathology which a future revolution might repair.

\subsection{Scepticism and Speculation}

\index{scepticism}
\index{speculation}

{\it Scepticism}, discussion of which falls within epistemology, with its less conspicuous partner {\it speculation} (more likely to be associated with metaphysics)  are the twin engines of philosophical revolt.

In the cause of fresh starts, scepticism clears the ground of stale dogmas, making a space on which creative speculation can build new structures.

Destruction is the easier part.
True mettle is tested in the constructive sequel, which depends upon a measure of speculation and risks again the sceptical assaults which paved its way.

\section{Rationality}

The birth of philosophy was itself a revolution in which there began a shift which I will loosely characterise as a shift from {\it superstition} to {\it rationality}, the emergence of philosophy, then encompassing mathematics and science, as an alternative to myth and religion.

This was a shift in attitudes toward knowledge which remains to this day incomplete.
The full lessons of the recent advances in logic, when assimilated into philosophy, concern the nature of rationality, and move us closer to the completion of that shift with which philosophy was born.

The first shift towards rationality is found in attitudes towards knowledge, shifting both what kind of thing counts as knowledge and how knowledge is to be discovered and justified.
So far as what counts as knowledge is concerned the shift is from explanations in terms of supernatural beings to explanations in more materialistic terms.
So far as methods and justification are concerned the shift is from tradition, authority and revelation, sources of knowledge which discourage disputation and debate, towards reason and observation, which each of us can independently dispute or confirm.

This is the beginning of science, once a part of philosophy, which has come a long way since then.
The progress of philosophy in terms of standards of rationality has been more faltering.
When a subject domain becomes less speculative and more systematic and reliable it is likely to be considered no longer a part of philosophy, so that philosophers remain concerned with matters which are difficult of resolution and which resist solution by systematic and reliable methods.

The general concern with ``rationality'' once this becomes self-conscious, leads us into {\it epistemology}.

\subsection{Epistemology}

\index{epistemology}

Though {\it logic} is my focus, {\it epistemology} is what I take to be the central concern of philosophy.

Philosophy was itself a revolution in attitude towards knowledge.
A move away from revelation, authority, religion, superstition and myth, towards observation and reason.
In modern times observation and experiment have become the primary tools of science, now independent of philosophy.
The domain of philosophy has centred on reason, but through epistemology it participates in the meta-theory of empirical science.

All the sub-themes each in their own way are part of this central concern with epistemology, including the revolutionary sub-theme with its implicit critique of the reliability of philosophical claims to knowledge.

\subsection{Epistemic Standards}

Different branches of knowledge exhibit different degrees of success in the establishment of durable and useful knowledge.
In the best cases claims to knowledge are seen to be objective, can be independently verified, command consensus and stand the test of time.
Mathematics has generally been regarded as a paradigm in this respect, while philosophy may be thought to occupy an opposite pole, with physical sciences, life sciences, social sciences and other branches of knowledge ranged between.

While the dominant factor in this kind of success may lie in the tractability or otherwise of the domain, changes in methods or culture may make significant differences.
A shift in the standards of rigour of mathematics was deliberately engineered in the nineteenth century, culminating in the logical revolution which provided completely new kinds of foundation for mathematical knowledge.

The revolution we anticipate in our final chapter is in part a revolution in the epistemic standards of philosophy.

\subsection{Language and Logic}

\subsubsection{Pathologies of Natural Language}

A factor in the establishment of objective knowledge is clarity of language.
Insofar as truth may properly be said to be ``relative'' it is relative to the semantics of the language in which it is expressed.
It is the essence of the semantics of descriptive language that it determines the conditions under which sentences (in context) are true.
Of course language is determined by culture, but once language is fixed so is the relationship between the truth values of purely descriptive sentences and objective conditions prevailing in the world. 
To the extent that semantics is unclear or uncertain, then so truth may fail to be objective.
To the extent that people differ covertly on their reading of semantics, the are likely to differ overtly on questions of truth, and if their discussions fail to uncover their differences on semantics, disagreement about truth will not be resolved.

Natural languages are defective in relation to semantics, there being in general no reliable way to distinguish between a disagreement about meaning and a disagreement about fact.
This is to say that the meaning of ``meaning'' in natural languages is rather muddled and incomplete.
Very many of the useful purposes served in ordinary language by descriptive language are not dependent upon any very precise understanding of what meaning is.
However, when we come to science and seek high standards of clarity and objectivity, often because lives depend on there being no misunderstanding, semantics becomes more important.

So far as clarity of language goes philosophy spans a wide spectrum.
Its language varies from the obscure and mystical to the nicest formal precision.
However, formal precision is rare, and for the rest I think it may reasonably be claimed that a very substantial part of the chronic disagreement in Philosophy is attributable to lack of clarity in language.
This is not entirely for lack of awareness of the significance of the issue.
One simple response to this issue, to preface the response to every question by ``well it depends exactly what you mean by...'', does not solve the problem.
In default of a satisfactory framework, the attempt to clarify language a concept at a time fails.
Each offered definition, in terms equally unclear, gives only the appearance of precision.
We will offer another tack later, but for now I will confine myself to a light sketch of some of the history.

My sketch distinguishes four stages in philosophical attitudes toward language, of which all but the first are quite recent.

\subsubsection{Oracular Evolution}

In the first stage philosophers, needing some special language to express their theories have often simply adopted a more precise usage of pre-existing terms, writing as if they were using more correctly a language already given.
It may be noted that this is not an {\it abuse} of language peculiar to philosophers, but a normal practice in any speciality which finds itself in need of more careful language than it inherits.
The attitude to pre-existing non-specialist use after the covert innovation may vary.
In some cases no conflict is perceived, in others the new usage is pressed as more correct than the old.
In philosophy this may be the case even though there is some impracticality in adopting the new usage for everyday purposes.

\subsubsection{Scientific Language}

Though this has little impact on philosophy modern science involves continuous development of language, including the augmentation or replacement of qualitative language by quantitative language.
The quantification of empirical science stimulates the development of the {\it a priori} science of mathematics and of its even more precise (if by today's standards still informal) language and notations.
Both of these developments were aided an abetted by their connection with objectivity, in the first case through observation, and in the second by the discipline of deductive demonstration (albeit, at this stage, a liberal constraint).

There is no clear dividing line between these aspects of language.
As in all specialist areas language evolves to meet the needs, in scope, precision and concision.
Science and mathematics are at the forefront, but their influence on language is not wholly confined to specialists, it leaks to some degree into wider usage.

\subsubsection{Formalisation}

The second stage occurs when, for the purposes of mathematics, completely formal and very precise languages are devised.
We now have, not just conceptual but more wholesale linguistic innovation.
For a brief period philosophers, notable Bertrand Russell take such languages to be ``logically perfect'', and natural languages to be imperfect.
Russell has no qualms in using a logical interpretation of descriptions which is clearly not quite the same as their previous use in natural language.
Some adjustments are necessary to make the logic right.
The distinction between language as needed by philosophers and as used more generally is becoming conspicuous.
We have here now the seeds of a level of precision in language which might transform parts of philosophy into a science as reliable as mathematics.

However we remain, in a sense, in ``bad faith''.
Pronouncements are made about the semantics of natural languages which are based on considerations extraneous to those languages.
There is a failure to recognise that formalisation creates new languages, which may be in some ways superior to natural languages without their offering any precise counterpart to all the features of natural language.
The procedure is at bottom an attempt at the same oracular evolution of language, with much more sophisticated machinery, but with the same reluctance to admit innovation and creativity in language.

In Russell we see a philosopher on the edge of this practice.
He fails to recognise the large element of discretion in the design of language, but does recognise that there really is a gap between natural languages and the ``idealised'' languages which he has devised.

\subsubsection{The Tractarian Denial}

Wittgenstein, in his early philosophy, first took logical languages not to be idealised languages but as models of the logical structure of the content of anything which could be said in any language.
By first taking this uncompromising stance he made the gap between natural and formal languages more problematic, paving the way for the completely new {\it naturalistic} attitude towards natural language which characterised his later philosophy.

\subsubsection{Semantics and Usage}

Now the study of ``ordinary language'' or of ``ordinary usage'' of language begins, and linguistic innovation, especially for philosophical purposes and whether covert or overt, becomes unfashionable.
The pervasive and timeless adaptation of language for particular purposes is now regarded as a pathology peculiar to an outdated mode of philosophising.
The use of language by those other than philosophers is taken as primary and philosophical theories, interpreted in this alien creed, are found lacking.

\subsubsection{Formal Leakage}

While philosophy has turned its back on formality and embraced natural language interpreted through usage, mathematics takes up the study of formality but declines to adopt formality practice.
But crucial leakage takes place.
Though the formality which proved vital in identifying and then eliminating the incoherence of naive set theory is not adopted, the more precise and crucially coherent concept of set, and the foundational presumption that abstract objects consist entirely of sets, is fully absorbed into the informal practice.
Both in terms of the clarity of its semantics (answers to questions like ``what is a number'') and in terms of the rigour and reliability of its demonstrations mathematics in the twentieth century has been conducted on a new plain.

\subsubsection{Formality in Science}

With the exception of parts of computer science formality, in that strict sense of the term we draw from symbolic logic, has so far had minimal impact on the practice of science and mathematics.
One factor in this is the complexity of working with fully formal notations and proofs, which outweighs the benefits to a science which finds mathematics quite satisfactory without it.

Relentlessly a modest stream of research in Computer Science and the continuing increases in available computing power are eroding the disincentives to formality and enhancing its benefits.
We will attempt to assess the implications for science and engineering of a future in which this technology is far enough developed to transform the way science is applied in engineering design.

\subsubsection{More Leakage}

Our main concern however is with the impact on the future of philosophy.
We envisage two avenues of impact on the language of philosophy.

The first through leakage.
Even in default of any adoption of formal notations in philosophy, concepts formally refined in other disciplines will be used in philosophy, to some small benefit in precision and objectivity.

However, the main prospect for radical change is through deliberate use of formality.
The key ideas which with which I would seek to provoke the adoption of formal techniques are:
\begin{enumerate}
\item that formalisability should become a universal test for the soundness of deductive arguments.
\item that the deductive parts of arguments should {\it always} be clearly separated and justified, so that the basis (or lack of it) for the deductive parts can be made more clear.
\end{enumerate}

This connects with precision of language in the following way.
When the deductive parts of arguments are clearly separated, even in default of any precise definition of the concepts involved, a meaning for the relevant concepts can be reverse engineered.
It happens in this way.
The premises of the argument may be thought of as axioms.
If the axioms can be shown to be consistent (and for most arguments we would expect that to be possible within set theory), then the set of all models which satisfy the axioms will provide a ``loosest interpretation'' for the concepts.
This ``loosest interpretation'' provides a precise indication of the scope of validity of the arguments.
With this kind of analysis the question of whether disagreements between philosophers are rooted in differences in conceptual content can be investigated.
