% $Id: ac.tex,v 1.2 2010/04/05 16:05:16 rbj Exp $
\chapter{Abortive Coup}\label{AbortiveCoup}

The advances in mathematical logic, which culminated ultimately in the formalisation of the analytic, threatened early in the twentieth century a {\it coup de grace}, sweeping away they whole of philosophy as it had hitherto been.

\section{Analytic Philosophy}

\index{analytic!philosophy}
That something revolutionary was happening in philosophy has retrospectively been recognised by the adoption of a new term for the kind of philosophy concerned.
The term {\it analytic} has now acquired a further application, to a {\it kind of philosophy}, and though often applied to the entire philosophical tradition back to ancient Greece which has had similar concerns, {\it analytic philosophy} is generally reckoned to have started with Russell and Moore at the beginning of the twentieth century.

For the moment I will pass by Moore's contribution, which will prove more significant later.

Though the term {\it analytic philosophy} had not then been coined, it is clear from some of Russell's writings that he was very close to connecting philosophy with analyticity.
Though Russell's description of his work in mathematics had been as demonstrating that mathematics {\it is} logic (see, for example \cite{russellPRM} preface to second edition), Frege's words in much the same context were that he sought to demonstrate that arithmetic is {\it analytic}, and I am not aware that Russell took exception to this formula.

When his work on the foundations of mathematics had been completed and Russell considered philosophical methods more generally, the place of logic in his conception was no less conspicuous.
In \cite{russellSMP} he concludes:

\begin{quotation}
Philosophy, if what has been said is correct, becomes indistinguishable from logic as that word has now come to be used. 
\end{quotation}

to Russell then, philosophy should be analytic because it relinquishes the {\it a posteriori} to science, and all that remains (bar nonsense) is the {\it a priori} consisting of {\it analytic} propositions which belong to logic.

Wittgenstein's Tractatus \cite{wittgenstein1921} reiterated Hume's trichotomy and (despite placing philosophy in the nonsense partition) provided impetus to the Vienna Circle, which however, stuck with Russell in regarding philosophy as analytic.
The prime English enthusiast for logical positivism, A.~J.~Ayer, would later write \cite{ayer1936}:

\begin{quotation}
In other words, the propositions of philosophy are not factual, but
  linguistic in character - that is, they do not describe the behaviour of
  physical, or even mental, objects; they express definitions, or the formal
  consequences of definitions. Accordingly we may say that philosophy
  is a department of logic.
\end{quotation}

The formalisation of logic, it was hoped, made possible an end to previously interminable philosophical disputes, rendering philosophy as reliable as mathematics.

This is our prospective {\it coup}, which proved abortive.

\section{Plan}

I will present this failure in the following pieces.

\begin{description}
\item [First:] a core proposition.
\index{logicism!philosophical}
A narrow view of how logic can contribute to {\it a priori} thought.
This I call {\it philosophical logicism}
\item [Second:] reason for doubt.
Some observations on why the core proposition might fall on stony ground.
\item [Third:] what the protagonists did.
The philosophical developments closest to exploiting the advances in logic were the philosophy of logical atomism and that of logical positivism.
These philosophies each in its own ways, went well beyond our core proposition.
These extra elements provided motivation and direction without which the core proposition could not have prospered, but also contained the seeds of failure, in the form of doctrines which could not ultimately be defended.
\item [Fourth:] the main defence.
The emphasis on language which Frege found necessary to solve his problems in the foundations of mathematics provided new problem domains and and modes of philosophising for those who prefer words to symbols.
\end{description}

From the standpoint of standards of rationality I shall argue that the core proposition offered hope of better to come, that the extra elements (such as metaphysics or empiricism) though perhaps intended to add substance, muddied the clear waters and weakened the package.
The broader linguistic counter-revolt which ensued was a genuine and substantial setback in the evolution of rationality.

\section{Philosophical Logicism}

There are many ways in which logic may be thought to impinge on philosophy.
I want to draw attention here to what was probably the simplest view of the relationship between these two spheres, and the startling and widely unacceptable consequences of this view.
This view is present, though perhaps not prominent, in Russell's writings and in the those of some of the logical positivists, including the English branch of that school to the extent that it is represented by the work of Alfred Ayer.

This position flowed from the identification between the {\it a priori} and logical necessity, together with considering synthetic, {\it a posteriori} truths as the province of empirical science and thus distinct from philosophy.
Philosophers should be concerned exclusively with the {\it a priori}, which is the sphere of {\it logical} truths, and which sphere had been rendered scientific (though not empirical, i.e. systematic and reliable) by modern developments in logic.

Most especially the hope was that philosophy could now, in point of reliability of its claims, become the peer of mathematics rather than remain poles apart in volatile speculative uncertainty.
From this standpoint the contribution of modern logical methods is simply to provide a reliable arbiter of philosophical truth.
A pre-logical step is required, which is to filter out those propositions which either because they are synthetic or because they are meaningless, are unsuitable candidates for the process.

This very narrow programme of logical excision and reconstruction was never, by itself, seriously attempted.
Mainly because it was swallowed up in larger enterprises in which greater benefits were sought from bringing logic to bear on philosophical problems.

It is my objective here to argue that the specific purpose of objectifying and removing from controversy the status of specific alleged a priori truths (i.e. settling whether they are indeed truths) was and still is both desirable and realisable, but that many of the other ideas which were confused with this very limited objective were not.

This I regard as a problem, largely a technical problem, internal to the efforts to apply logic to philosophy.

The technical problems, serious though they might have been, were outclassed by the cultural.

The transformation proposed for philosophy would have transformed it from a literary to a technical enterprise, and there can have been few like Russell capable of moving at ease in both these idioms.
Even Russell's star pupil, Ludwig Wittgenstein, despite possessing a penetrating vision into the significance of the new methods, was probably disinclined and temperamentally unsuited for the kind of detailed technical work which had been required in Principia Mathematica and might now be needed for this new approach to philosophy.

\section{The Vacuum}

Philosophy is similar to mathematics insofar as it may be thought primarily concerned with {\it a priori} knowledge.
It is for this reason that success in applying new logical methods to mathematics might have been followed by similar success in philosophy.

There were however crucial differences.
Even before the focus on logical foundations for mathematics, mathematics was generally thought to have the most reliable {\it standards of truth}.
Even when standards of proof were at their nadir and a cause of considerable concern to some mathematicians, the reliability of the results obtained was still very high.

Before the logical foundations of mathematics were addressed the mathematical foundations of analysis had been put right by reducing analysis to arithmetic, aided by a bit of set theory.
When it came to providing logical foundations for mathematics, there was little question about what mathematics was.
The main body of mathematics was established beyond dispute, all that was needed was a little extra work in the basement.

Once logical foundations for mathematics were in place and the question is asked, ``what might this mean for philosophy?'', what do we find?
If empirical matters are relegated to science, philosophy must deal only with the {\it a priori} such as one would expect to be derivable in just the same logical foundations as had been constructed for mathematics.
The formula:
\begin{equation}
Mathematics = Logic + Definitions
\end{equation}
even if elaborated to:
\begin{equation}
Mathematics = Logic + Ontology + Definitions
\end{equation}
is still {\it topic neutral} in all but the definitions.
It may be argued that in the logical foundation there is nothing to be done.

On the other hand, when it comes to supplying the definitions and deriving the doctrines, philosophy was poles apart from mathematics.
There was no consensus about the truths of philosophy, and arguably insufficient basis to agree what definitions might be relevant let alone what they should be.

The logicisation of philosophy was an altogether different proposition, because there was no body of {\it a priori} philosophical knowledge awaiting a foundation.

When philosophers attempted to logicise philosophy they mainly engaged in propounding new philosophical theories (or modern variants on old ones) rather than formalising established ones.
The influence of the new logic was primarily on the content of the philosophical theories, not on their formulation and derivation.

\section{Logical Atomism}

\index{Logical Atomism}

When Bertrand Russell came in 1913 to the end of his work on Principia Mathematica, and turned away from mathematics to more general philosophical considerations, he had the benefit of the fresh ideas of his student Ludwig Wittgenstein.
Wittgenstein's ideas provided, according to Russell's own account, basis for the philosophy which he was to call {\it Logical Atomism}.

Wittgenstein left Cambridge in August 1914, and there was no further contact between them until Russell had already published lectures on Logical Atomism which he had given in 1918, and Wittgenstein has already written his doctoral dissertation, which was to be published as the {\it Tractatus Logico-Philosophicus} \cite{wittgenstein1921}.

Though Russell's lectures were first to be published, it is best to talk first of the Tractatus.

\subsection{Tractatus Logico-Philosophicus}


The Tractatus may (though is not usually) be said to begin with an answer to the metaphysical question:
\begin{quotation}
What must the world be like for it to be possible to speak of it?
\end{quotation}
followed by an account of the logical aspects of language which form the basis for the metaphysics.

The apparent gap between the kind of language described by Wittgenstein and the languages used in everyday life made Russell presume in his introduction to the Tractatus that Wittgenstein was speaking of logically ideal languages, but Wittgenstein repudiated this suggestion.
For any language to express propositions about the world it would have to express these same structures.

Wittgenstein's account of the world in terms of atomic facts, of the way in which atomic propositions {\it picture} facts and his account of propositions in general as being truth functional combinations of atomic propositions may be thought to presage the model theoretic account of the semantics of first order logic later developed by Tarski and now accepted as the standard account.

We might now prefer to interpret this, not as telling us about the structure of the world, but as giving us one of many alternative equivalent accounts of how one may construct abstract models of the world.

Wittgenstein's Tractatus is present more as a {\it fait accompli} in which philosophy is put to rights, except to the extent that Wittgenstein debunks his own philosophy as meaningless.
What it does not appear to be is a prospectus for any future school of philosophy (though it might be said to have been treated in this way by some members of the Vienna Circle).

\subsection{Russell's Atomism}

Russell's atomism shares the metaphysical standpoint of the Tractatus, but adds to it a more definite standpoint about the nature of atomic facts, which are considered to be sense data.

In logical atomism we do not find any attempt to {\it formalize} and this rigourise philosophical discourse.
We find new philosophical doctrines, influenced by the lessons learned in the reduction of mathematics to logic.
The metaphysical doctrines may be thought of as derived from an understanding of the logical structure of language.
The added dimension in Russell's atomism is the idea that the physical world should be regarded as logically constructed from sense data in the same way that mathematical objects might today be regarded as constructed from sets.

A kind of programme for analytic philosophy appears in Russell's atomism.
The kind of analysis involved is the analysis of language as expressing truth functional combinations of claims about sense data.
For a brief period some philosophers attempted to follow this through, but success was elusive.

A more vital philosophical movement carrying forward the torch of logic was to be found in Vienna.

\index{Logical Atomism}

\section{Logical Positivism}

\index{Logical Positivism}

Logical positivism is sometimes presented as a philosophical movement which spun off a moment in the thought of Ludwig Wittgenstein, only to be repudiated by its creative force as he reached his philosophical maturity.

There can be no doubt that Wittgenstein did exert considerable influence upon the members of the Vienna Circle, but both the positivism and the new emphasis on logic were already in place before either Wittgenstein or his Tractatus appeared in Vienna.

Though Rudolf Carnap himself proclaimed the impact of Wittgenstein on his thought, it is not so easy to spot.
Carnap was already well acquainted with the work of Frege and Russell before he came to Wittgenstein.
He was also inclined to regard certain philosophical questions as devoid of content, e.g. realism versus idealism. 
The only doctrine which Carnap claims to have come from Wittgenstein is the principle of verification, though his own modestly anti-metaphysical position were made more radical by Wittgenstein's influence.
Though criticism of the principle of verification presents this as the central principle of logical positivism, without which the whole philosophy collapses, this certainly does not seem to be the case for Carnap's positivism.

\index{Logical Positivism}

\section{Linguistic Reaction}

\subsection{Wittgenstein}


\subsection{Linguistic Philosophy}

\section{Post Positivism}


