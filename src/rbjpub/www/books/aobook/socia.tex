% $Id: socia.tex,v 1.1 2008/07/11 18:16:09 rbj Exp $
\chapter{Society}

\section{State of Nature}

It is usual in the presentation of political philosophy to base arguments on premises about human nature or about human society prior to the emergence of the state, state of nature theories.
Very often these premises are either implausible or incredible.
They sometimes seem to be mere rationalisations of views held by the philosopher for which rational grounds are lacking.
Sometimes the substance of the premises is little more than that man naturally is either vicious (and in need of government to hold him in check) or virtuous (and has been corrupted by some aspect of ``civilisation'').
Since Darwin's theory of evolution was published it has been used to underpin such claims.

The resulting philosophy may nevertheless be influential.
There may be many readers who have similar sympathies who will not be deterred by flaws in the rationale.
Political philosophy of this kind can be enormously influential and can impact (indeed, terminate) the lives of millions.

The political philosophy we present here is necessarily of the same ilk, at least to the extent that it consists in the presentation of ideas about how matters might better be arranged which are underpinned by theories about how things are.

There is no question that my own ideas about the future are influenced by a conception of human nature, particularly of what might be called his social nature, and that no convincing presentation of these ideas can be made without exposing this background.
I propose to lay bare their genesis, to reveal as conspicuously as I can the way in which I came to my views and the considerations which persuade me of their merits.

Sometimes political philosophy is presented as if it consisted of deductive arguments from appropriate premises.
This is not the case here.
