% $Id: anarc01.tex,v 1.1 2008/07/11 18:16:10 rbj Exp $
\section{What is Anarchism?}

The main purpose of this section is to locate the usage I make in this book of the term {\it anarchism} relative to its other principle uses in the history of ideas.

The word {\it anarchy} come from the Greek {\it anarkhia} meaning ``without a leader''.

It has been used in modern times for political theories, doctrines or movements which either advocate the abolition of government or of the state, or more generally of compulsion or coercion.

Anarchism is distinguished from libertarian doctrines, which see a minimal state as necessary for the protection of individual liberties, by its uncompromising nature.
The distinction is made stark by the use of the term ``minarchist'' for the most radical libertarians who nevertheless feel the need for some minimal state.

The main tradition of anarchist thought has also been closely aligned with socialism in seeking not only liberty but equality.
Though there are many strands to anarchist thought, there is a sufficient coherence between these strands for there to be a perceptible movement of movements, and for some anarchists to downplay the differences by talk of ``anarchy without adjectives''.

Radical libertarian capitalists, who wish to abolish government but retain the free-market capitalist economy, are called ``anarcho-capitalists'' but are not accepted as anarchists by many of the social anarchists.

The ``anarchism'' here espoused will likewise be rejected by many present anarchists.
It falls in some respects between the social anarchists and the anarcho-capitalists.

Its distinguished most conspicuously by being conceived in terms of the minimisation of coercion (by any party) rather than in terms of the abolition of state or government, allowing that the discontinuation of coercion by the state would suffice.
It shares with anarcho-capitalism the retention of the existing free-markets in labour and capital, but anticipates radical transformations in the way those markets work, and a greatly shifted balance between for-profit and non-profit entities.
