% $Id: intro.tex,v 1.1 2008/07/11 18:16:10 rbj Exp $

\def\wiki#1{#1\index{#1} \footnote{\href{http://wikipedia.org/wiki/#1}{http://wikpedia.org/wiki/#1}}}
\def\Wiki#1{\footnote{\href{http://wikipedia.org/wiki/#1}{http://wikpedia.org/wiki/#1}}}

\chapter{Introduction}

[{\it
Material such as this, in italics in square brackets, are notes on what I am doing which are not necessarily intended to be retained in the final version.
}]

[{\it
This volume consist of several themes interwoven and grouped into chapters.
Some of these themes are chronological, and start in the past and pass through the present into the future.
Since these have varying time-spans, they will not be interwoven synchronously but they will all pass through the present in the same chapter.

The themes as present conceived are:

\begin{enumerate}
\item autobiographical

This is the story of how I wrote the book, but it begins nearly fifty years before I actually started writing.

\item evolution

This is actually mainly about how evolution has evolved, and how it will continue to evolve.
Starting in the primordial soup, and ending with designer babies and cyborgs.

\item rationality

The history of reason.
Probably starting in ancient Greece, and going through to machine intelligence.
In here is included much philosophy.at least on the theoretical side.

\item society

This is a history of social institutions, politics, economics and the rest.
This is the locus for practical philosophy.

\end{enumerate}

As these threads pass through the present into the future they yield a programme for the future.

While the book is in its early stages the themes will not be interwoven, but will have one chapter each, at the head of which the intended content of the theme will be indicated, in this manner.
}]

\begin{quote}
You never change things by fighting the existing reality. To change something, build a new model that makes the existing model obsolete.

Buckminster Fuller

\end{quote}



In this volume I present a systematic philosophy.

It is not an academic exercise, but is intended to serve a purpose.
It is written on the premise that how well our future turns out depends in part upon how carefully we think out what we want from it and what we can do to conform the future to our wishes.

This is a kind of utopian engineering, using that term to embrace both revolutionary and evolutionary change.
The tendency of the ideas we explore here is captured in our title.
Insofar as it is philosophy, this is {\it practical} philosophy.
Within the scope of this overriding purpose, theoretical philosophy is encompassed as providing a part of and a foundation for knowledge which may be instrumental in securing our goals.

To prepare you for your read I'm going to have to confess to some aspects of the book which I can't help regretting.
I very much would like to make this book, clean and proper in various desirable ways.
I should like to avoid writing about things which I don't know much about, I would like to have a clear story to tell with a solid rationale, and I would prefer not to say anything about myself.
But I'm going to have to do all these things.

I am, I hope, in prospect of delivering a reasonable case for future anarchism, partly because I am myself pathologically anarchic.
It is this pathology which persuades me that my best chance of clearly delivering my ideas is in a form which is substantially autobiographical, and which contains much of the wild speculation which has contributed to the formation of my present weltanschauung.
This work is an autobiographical presentation of aspects of a philosophical weltanschauung.
The aspects which concern us are those which bear the future of us all.

Back in ancient Greece philosophers didn't always write.
There are significant Greek philosophers, say Pyrrho of Ellis, who are known to us only by repute.
Not only have none of their works survive, so far as we know they just didn't write much, or anything.
I suppose Socrates is a better example, we know more about him because he appears in Plato's dialogues.
ting about things which I don't know much about; I would like to have a clear story to tell with a solid rationale, and I would prefer not to say anything about myself.
But I'm going to have to do all these things.

Back in ancient Greece philosophers didn't always write.
There are significant Greek philosophers, say Pyrrho of Ellis, who are known to us only by repute.
Not only have none of their works survive, so far as we know they just didn't write much, or anything.
I suppose Socrates is a better example, we know more about him because he appears in Plato's dialogues.
