% $Id: philart.tex,v 1.1 2008/09/21 08:32:12 rbj Exp $

\def\wiki#1{#1\index{#1} \footnote{\href{http://wikipedia.org/wiki/#1}{http://wikpedia.org/wiki/#1}}}
\def\Wiki#1{\footnote{\href{http://wikipedia.org/wiki/#1}{http://wikpedia.org/wiki/#1}}}

\section{Philosophy as Art}

In {\it Intellectual Impostures} \cite{sokal1998} Sokal and Bricmont criticise some ``postmodern'' and ``relativist'' authors for their misuse of scientific concepts in their work.
I am here at risk of being held in similar contempt, for an incompetent attempt to introduce art into an inappropriate context.
However, the ideas I present here, whatever their merits, have helped determine the character of this book, which could not have been written without their help.
For better or worse they must be discussed.

The closest I have come myself to being an artist as that term is generally understood, is in my efforts as an undistinguised amateur musician.
In music there is often a conspicuous division of artistic labour, between composer and performer, often blurred, sometimes set aside.
In some arts the performance side is simply absent, the artistic composition can be appreciated directly.

Philosophy, when its meditations are exposed to the world, usually appears as a form of literature, and it may be natural to think of any artistic merit as deriving from the manner of its expression rather than the substance of its ideas.
Certainly there is much written philosophy of which one may be disinclined, whatever merit there may be in the ideas, to consider the resulting literature a work of art.
However, to consider art as exclusively concerned with the aesthetics of presentation is clearly mistaken.
In a novel, or a symphony, we cannot regard the art as merely a matter of presentation. 

So I conceive of my philosophy as art, what is it that I have in mind, and why do I think that worthwhile?

Well first 


