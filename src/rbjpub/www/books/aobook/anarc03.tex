% $Id: anarc03.tex,v 1.1 2008/07/11 18:16:10 rbj Exp $
\section{Anarchism and Equality}

The great challenge for social anarchists is how to realise a reasonable distribution of wealth without coercion.
The difficulty of this is made more apparent by the contrast in performance between the socialist command economies and the free-market economies.

Social anarchists do not want command economies and seek collaborative organisation of production.
But it is hard to see how this can work without the free market.
Once we have free markets we will get differential income and will need redistributive taxation to realise equality.

It seems inevitable that an efficient economic system will result in differential incomes, and doubtful that any wholly voluntary mechanism will yield equality, particularly not where total wealth is comfortably above subsistence levels.

Our anarchism does not seek to deliver economic equality, seeking a sufficiency for all but admitting wide individual variations in wealth.
A major part of the innovation required to realise our anarchistic utopia concerns the ways in which the important functions now funded by compulsory taxation could be funded by entirely voluntary means.
 