% $Id: mppos.tex,v 1.2 2012/01/23 21:40:02 rbj Exp $

\chapter{Metaphysical Positivism}

Metaphysical Positivism is a contemporary positivistic philosophical
system, a further step in a historical thread of which the most recent
substantial advance was made by the Vienna Circle, and in particular
Rudolf Carnap as \emph{Logical Positivism}.

It is therefore convenient in giving an account of metaphysical
positivism to compare it to logical positivism, or for the sake of
specificity to the mature philosophy of Rudolf Carnap.

The positivism of Rudolf Carnap was primarily concerned with logical
analysis and hence belongs to theoretical philosophy, but Carnap's
philosophy was aimed at facilitating a transformation in the way in
which science was undertaken.
Though the results of philosophy (insofar as they constitute
declarative knowledge) were in Carnap's view logical, the role of
philosophy was not confined to obtaining such results.
Philosophy was more concerned with establishing languages and methods
for the conduct of science in a logically rigorous manner, and the
philosopher was therefore to be considered as leading to proposals on
such matters for consideration by scientists.

Positive philosophy provides a broader conception of philosophy within
which may be found an analytic core (\emph{metaphysical positivism})
closely related to Carnap's ideas.
Though I do not conceive it as \emph{making proposals}, it does
involve constructing a framework within which the kind of science
envisaged by Carnap might be conducted.
Carnap's interpretation of his work as the making of proposals was a
way of distinguishing what I would call prescriptive and descriptive
analysis (the latter following Strawson's usage for metaphysics).
The distinction is between the analysis of some language \emph{as it
  is found} and the design of ideal or optimal languages for
particular (or more general) purposes.

Metaphysical Positivism is concerned with rigour in science (and with
``scientific'', i.e. rigorous philosophy). 
Logical positivism was scientistic, and inclined to assimilate all
{\it bona fide} knowledge into science, and this contributed to an
undervaluation of practical philosophy and a little interest in
language which is not \emph{descriptive}.
Metaphysical positivism, as the analytic part of positive philosophy
extends into the domain of practical philosophy in limited but
important ways.
It is concerned with all deductive reasoning, and hence with reasoning
about values, morals, and in political or economic reasoning insofar
as it can be made deductively rigorous, or to the extent that a
deductive element can be isolated.

\section{Semantics and Epistemology}

A central feature of metaphysical positivism is the relationship
between semantics and epistemology.

In ascertaining truth, we must first clarify meaning, with particular
regard in respect of descriptive language, to truth conditions.
The semantics then influences the epistemology, via the
analytic/synthetic dichotomy.
A sentence is analytic if it is invariably true, synthetic if possibly
false.
In the former case, a justification may be given \emph{a priori}, in
the latter a justification, or some other kind of evaluation, should
be \emph{a posteriori}.

This is so far much the same as the position of Logical Posivism, and
the refutation of logical positivism was accomplished by criticisms of
this stance most notable by Quine in his ``Two Dogmas of
Empiricism''\cite{quine53} and Kripke in ``Naming and Necessity''.
The first attacked the analytic/synthetic distinction and the
possibility of defining the semantics of a language.
The second the relationship between the semantic distinction and the
epistemological one.

Quine's arguments represent a skepticism about semantics which if
taken seriously is lethal not only to Logical Positivism but to
deductive reason in general.
I will not go here into my reasons for disregarding Quine's critique
of the analytic/synthetic distinction, which is as important to
metaphysical positivism as it was to logical positivism.
I will sketch the rationale for maintaining the relationship between
the semantic and epistemological dichotomies pace Kripke.





