% $Id: intro.tex,v 1.3 2012/04/04 20:22:09 rbj Exp $
\mainmatter
\part{Preliminaries}
\chapter{Introduction}

This book advances the thesis that democracy is rational.
It considers how we may, and why we should, arrange our lives and the workings of the institutions around us in a rational way, arguing that democratic governance has a place in such an arrangement.

``Rationality'' comes in two varieties (for now), \emph{epistemic} rationality\index{rationality!epistemic} and \emph{instrumental} rationality\index{rationality!instrumental}, the former concerning our beliefs, and the latter our actions.
An action displays instrumental rationality if it can reasonably be expected to realise the purpose for which it was undertaken.
The adoption (or retention) of democracy is an action rather than a  proposition, so to consider it rational we must have in mind some purpose.

I put forward as an ideal to which we might strive, ``universal fulfillment'', the ideal that each one of us leads a fulfilling life.
It is against that purpose that the rationality of our arrangements will be discussed, and it is for that purpose that alternative proposals are discussed.
Life rarely is so simple, not even in its aspirations, and as the work progresses I will undertake some clarification of that ideal, expecting that it will be elaborated and complemented.

Utopian ideals such as this were perceived by philosophers such as Carl Popper \cite{PopperPOH} and Isaiah Berlin as having totalitarian tendencies, and though it is difficult to see how this could be the case for any genuinely democratic aspiration, anticipating how a democratic society might work is a challenge because to that fludity of choice by others which seems likely to confound any more detailed expectations.

To be rational is not to place an unrealistic burden upon logical reasoning, or to decry the importance of emotion in human life.
Deduction is fundamental to the practice of rationality, but deduction alone cannot tell us what is the case, it cannot tell us what we should value or what we should do.
A good understanding of the broad scope but severe limitations of deductive reason is an essential part of a complete rationality.

A key to that understanding of scope is the distinction emphasised by David Hume between knowledge of \emph{relations between ideas} and \emph{matters of fact}.
Those two domains of knowledge are themselves set apart from \emph{values} and \emph{morals} by another of Hume's maxims, which tells us that we cannot derive an \emph{ought} from an \emph{is}.
Accepting these Humean distinctions yields a tripartite classification of indicative sentences or \emph{propositions}, into three domains which I will lable provisionally:

\begin{itemize}
\item[]logical (\emph{a priori})
\item[]empirical (\emph{a posteriori})
\item[]valued
\end{itemize}

We may call the first two \emph{descriptive} and the last \emph{evaluative}, hoping that descriptive claims are \emph{objective}, having the same truth value for all, but noting that we may differ from others on the truth values of evaluative propositions, notwithstanding any belief we might have in their objectivity.

These distinctions are shifting and controversial, but I will argue that they are also fundamental and instrumental.
They correspond to epistemological criteria which determine how we should go about establishing their truth, \emph{a priori} propositions requiring no testimony of the senses, but \emph{a posteriori} propositions essentially depending on sensory testimony.
Values on the other hand, are more difficult to establish conclusively, partly because they may be subjective, pureley personal values requiring no justification (though possibly some explanation).

I will work backwards here in giving some further detail on how the principal concerns of these parts.

The distinctions play so great a role in this book that they form the basis for the division of the book into four parts.
In the following I expand a little on theses areas in reverse order.

\section{Practical Philosophy}


It is not uncommon for political philosophy to involve observation about the state of society before organised governance was established.
These historical empirical claims are sometimes implausible and lacking in evidential support.
Nevertheless, the rational conduct of human affairs depends upon an understanding of human nature, and it is only rational to accept ideas about how our objectives can be realised if we understand and accept the empirical basis for the supposition that these ideas will be instrumental.


\section{Empirical Knowledge}

The second and third parts together correspond broadly in scope and intention to Aristotle's \emph{Organon}, his works on logic which together described his conception of \emph{demonstrative science},  the first systematic account of how science should be conducted (and the beginning of the study of logic).
The Organon was so-called because Aristotle did not regard these works as being a part of science, but rather as methods and tools for use in the conduct of science.
Another work of Aristotle has some similarity is that now known as \emph{The Metaphysics} but by Aristotle described as concerned with \emph{First Philosophy}.
This is concerned primarily with \emph{being} and \emph{substance}, and may now be thought to correspond to some matters which belong to philosophy, and which we call \emph{ontology} and \emph{metaphysics}, and some which belong to physics.

It is not my intention here to turn back the clock, quite the reverse, both logic and scientific method have been completely transformed since Aristotle, and though scientists do now complain of the lack of respect for their disciplines among the general public, methodological regression is not a plausible remedy,
Much of that perceived problem might reasonably be described as irrationality on the part of of the public.
Nevertheless, general scepticism about science is fully warranted, and even though I do here present some ideas about how scientific instutions might move forward, a rational scepticism is part of the solution rather than the problem.


\section{Deductive Knowledge}

The first part convers a number of preliminary topics and provides some of the background knowledge which has helped to shape my own philosophical outlook and which will be helpful to me in explaining the rationale for my positions.
The second is concerned with \emph{a priori} knowledge, the third with knowledge \emph{a posteriori}, and the fourth with ``practical'' philosophy: values, ethics, economics and politics.

\section{Preliminaries}

The divisions here adopted concern what may be considered the core content of the respective domains, and are epistemological in character.
They do not characterise what I have to say in speaking of them.
Considering the domain of \emph{a priori} truth, of which the most developed part is mathematics...



