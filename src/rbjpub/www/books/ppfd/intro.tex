% $Id: intro.tex,v 1.3 2012/04/04 20:22:09 rbj Exp $
\mainmatter
\chapter{Introduction}

This is the second of two volumes in which a certain ideas which I
call \emph{Positive Philosophy} are presented.
The first volume \cite{rbjb004} is concerned with the
\emph{theoretical} aspects of Positive Philosophy, this volume is
concerned with the \emph{practical} side. 
The distinction between the two corresponds roughly to the distinction
between \emph{matters of fact}, and \emph{judgements of value} (after Hume) or that
between natural and moral philosophy in Aristotle.

The first volume therefore addresses questions in epistemology,
metaphysics and logic, in the philosophy of mathematics, science and
engineering.
This volume is concerned with what might be called personal or
existential matters, with ethics and morality, and with political,
economic, and other social matters.

In both volumes I am concerned to connect the philosophy with
what seem to me to be important contemporary issues.
In both cases the possible future impacts of information technologies is
a concern, particularly the positive opportunities arising.

In the first the theoretical issues are therefore intimately connected
with more practical matters concerning the automation of reason,
to the extent that the epistemology and logic are presented as an
abstract architecture for cognitive cooperation between man and
machine.

In this volume the ``practical'' philosophy is connected with some of
opportunities and challenges which information technology brings to
the conduct of our lives and the workings of social institutions.
Of these the ways in which individuals are now and will in the future
be able to influence the world around us, both through political and
economic institutions are of particular interest.

The two volumes are intended to be largely independent.
For that reason I begin this volume with an account of those aspects of the
theoretical philosophy (which I sometimes call ``metaphysical
positivism'') which are most important in this context.
Principally this consists in a conception of analytic method, which I
seek to apply here, together with some of the philosophical
underpinnings necessary in explaining and motivating the use of the
method. 

Beyond a general discussion of practical philosophy, two areas in
which information technology might enable or impell changes are
explored.

These concern the nature of democracy and the operation of `free-market'
economies.

The automation of reason is used in the first volume as a way of
focussing an account of a theoretical philosophy.
In this volume an exploration of possible changes in democratic
politics and free market economies serves a similar purpose, providing
a motivation for the preceding account of practical philosophy.
