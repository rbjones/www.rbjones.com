% $Id: intro.tex,v 1.3 2012/04/04 20:22:09 rbj Exp $
\mainmatter
\chapter{Introduction}

This book advances the thesis that democracy is rational.
It considers how we may, and why we should, arrange our lives and the workings of the institutions around us in a rational way, arguing that democratic governance has a place in such an arrangement.

``Rationality'' comes in two principal varieties, \emph{epistemic}\index{rationality!epistemic} and \emph{instrumental} rationality\index{rationality!instrumental}, the former concerning our beliefs, and the latter our actions.
An action displays \emph{instrumental} rationality if it can reasonably be expected to realise the purpose for which it was undertaken.
A claim to knowledge of, or a belief in some proposition is \emph{epistemically} rational if it is based on good grounds for supposing the proposition to be true.

The adoption (or retention) of democracy is an action rather than a  proposition, so to consider it rational we must have in mind some purpose.
My thesis may therefore be elaborated as, that democracy is rational for the purpose of realising \emph{universal fulfillment}.
I hope that, even for those not consumed by that particular purpose, the work may nevertheless be of some interest, since many of the considerations of rationality which follow may be relevant to other ultimate ends through their support for generic capabilities, including but not limited to economic efficiency.

Considerations of rationality may be applied not only to the actions and opinions of individuals, but also to the structure and operation of various collective entities or social institutions.
This applies, for example, to academic or industrial bodies engaged in research intended to extend our knowledge.
By similar extension we may consider the organs of government and come to a view as to whether or not they are fit for purpose, not only as a whole, but also in each component or aspect.
If we conclude that democratic governance, in some form, is our best bet for realising universal fulfillment, then for that purpose we may say that democracy is rational.

I have given first indications of the meaning of the concept of rationality, not definitions.
The concept is context sensitive in application, and may have normative content (as it often will have here).
In order to be rational we have in many cases to use our judgement, rather than follow clear cut rules.
When considering the rationality of others, we likewise use our judgement, which may differ from that of our subjects.
In considering their rationality we should not judge them irrational simply because their considered opinion differs from ours, the grounds should be stronger than that (though a complaint of poor judgement might be in order).
Typically particular kinds of defect give rise to a complaint of irrationality, particularly defects of reason.

My interest here is not simply or primarily with that summary judgement, rational or not, but with the detail behind it, for the thesis is not that everything which might pass as democratic is also a rational arrangement for the purpose in hand, but rather that some are.

I will advance my thesis using two historical threads intertwined in alternating chapters,
the first concerned broadly with the historical development of rationality, the second with that of democracy.
We live in a changing world, for the book to be worthwhile it must speak not to the past, nor the fleeting present, but to how things might be, so these two threads project from the past into the future.

Some further clarification of the terminology might still be thought appropriate, so a few words on why not a great deal will be forthcoming.
The term ``rationality'' has very diverse usage, partly because what is rational is context sensitive, partly because there is liable to be disagreement about what is rational in any particular context.
In some contexts it may be rational to adopt a sceptical stance, and to demand that key facts are established with near absolute certainty, in others it may be so essential to act that it is rational to make a best guess about some important but unknown factor and take a gamble.
We are primarily concerned here with instrumental rationality, and, in practice, use of the term rational will have in it a normative element, we are simply expressing the opinion that some course of action, some institutional arrangement, or some method is a good, or the best, way to achieve the desired end.
Thus, our preoccupation here is not with refining the concept of rationality, but deciding from among the alternatives open to us, which are the most effective ways of realising our objectives.

Democracy is likewise variously used, there are many different kinds of democracy, and inevitable disagreement about what systems properly fall under the concept.
The thesis that ``democracy'' is rational is not intended to assert that \emph{any} democracy is rational, rather that \emph{some} form of democracy can be argued to be rational.
It will be necessary to say something about \emph{what kinds} of democracy will do the job.

\section{The Narrative}

In considering the future, we must first understand the past, the historical dynamic, our present predicament and then project forward.
In painting a picture intended to cast light on and possibly shape our future, we seek to capture a process rather than a product.

This is presented in this work as a multilayered evolution, taking that term in its broadest sense, the character of the evolution varying between the layers and as each layer progresses (through time).
The intention is to present a {\it grand narrative}.
It is, I believe, primarily philosophical, and hence this narrative strives also to be its own metanarrative.
Its dedication to democracy makes it essential to leave favour abstract principles over concrete details, the latter to be settled by appropriate social structures and the ideals which underpin them rather than by philosophical dogma.

Here I now sketch some of the most significant layers in the evolutionary landscape that will follow.

We can talk (and some do) about the evolution of the universe.
Very little in this work will depend upon this perspective, but it does provide a context for our more intense focus on the evolution of homo sapiens, and may be significant in motivating the development of intergalactic promulgation within the lifetime of our sun, and for the incentives which that might provide for evolution of homo sapiens into life forms more suited to intergalactic travel or for the development of non-living intelligence more readily propagated throughout the cosmos.

Within that context of cosmic evolution, the establishment of life on earth creates a particular kind of evolution, of life-forms whose structure is encoded in DNA.
It is this kind of evolution which accounts for the genetic elements in those human characteristics which are relevant to our choice of social institutions including forms of governance.

Among social animals with certain kinds of capabilities associated with central nervous systems, genetic evolution may be overlain with cultural evolution, and is then likely to interact with genetic evolution by influencing choice of breeding partners, factors such as health and strength giving ground to wealth and (socially determined) status.


