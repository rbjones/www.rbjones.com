\mainmatter
\chapter{Introduction}

Darwin's theory of evolution can be stated quite simply.
Darwins work was no so much in articulating the theory, but in extensive research which provided evidence for the thesis.

The ideas behind this volume are also quite simple at their core, and I will waste no time in their exposition, which will be presented in three main stages.
First I will give a very concise statement of my central theses.
Then I will try to make them intelligible.
Next I present in greater detail my reasons for believing them to be true.
Finally some elaboration on their consequences will be attempted.

But before that let me show the gross structure of the story:

\begin{enumerate}
\item Evolution
\item Teleological Language
\begin{itemize}
\item role
\item purpose
\end{itemize}
\item Rationality
\begin{itemize}
\item instrumental
\item epistemic
\end{itemize}
\item Epistemology
\begin{itemize}
\item epistemes
\item epistemic evolution
\end{itemize}
\end{enumerate}

I begin with the definition of a conception of evolution broader than that of Darwin, broad enough to encompass the evolution of non-biological systems and hence the pre-biotic evolution of life, and the evolution of potentially non-biotic self-proliferating synthetic intelligent systems.

The exiting feature of evolution is that it realises structures which are well adapted to fulfill some specialised role in the proliferation of those kinds of individuals or subsystems to which they belong.
For this reason it is often thought appropriate to use teleological language to describe these structures.
We regard the role which they fulfill in proliferation as being their purpose, even though it is not a purpose conceived by some intelligent designer.

Supposing that we are thus content to talk of the purpose of these structures, we are then in a position to observe that evolution is \emph{rational} om that instrumental sense in which rationality consists in adopting a means to some end which is likely to realise the end.
Good design is a hallmark of instrumental rationality.

In a variety of ways we may then go on to attribute to evolution \emph{epistemic} rayionality.
The difference between instrumental and epistemic rationality is similar to that between knowing \emph{how} and knowing \emph{that}.

\ 

@@@@@@@@@@@@@@@@@@@@@@@@@@@@

\

Since Darwin published his work on the evolution of species, similar ideas have been applied in very many other areas.

In seeking to understand what is happening around us and what the future might hold, I too have looked at the past from an evolutionary perspective and considered what that might entail for our futures.
Though seeking understanding through evolution, the understanding I seek is philosophical, and in that understanding epistemological considerations have become preeminent.

In this book I aim to present the ideas, and in this introduction to provide a preview of the book.

In considering the evolution of life, of homo sapiens and of the culture which exercises so great an influence on our lives, I have noted many changes in \emph{the process} of evolution.
It seems likely that recent technological advamces will, as they mature, facilitate a transformation in how the evolution of homo sapiens and its progeny takes place.

Both before and after the biological evolution of homo sapiens evolutionary processes which do not wholly comply with the Darwinian conception of evolution, or the modern synthesis which followed it, have played essential roles.
I have therefore found it necessary to make definite a conception of evolution which, while encompassing Darwinian evolution, allows kinds of evolution which he did not address.

The idea of an evolutionary imperative comes from that broader conception of evolution which I characterise thus:

\begin{quote}
Evolution is \emph{progressive change} contingent upon \emph{differential proliferation}.
\end{quote}

The imperative which flows from that conception is ``Proliferate!''.

I will return later to a fuller explanation of these terms and that definition, but for the time being it may suffice to note that ``differential proliferation'' plays a role in this generalised idea of evolution similar to that of ``survival of the fitness'', or more accurately ``reproductive fitness'' in biological evolution.
This makes the definition strictly broader by dint of makng no assumption that proliferation occurs by reproduction.
That may not be the case (there may be no reproduction), for example, in pre-biotic chemical or molecular evolution, and may not be the case in the evolution of future self-proliferating synthetic intelligent systems.

In such evolutionary processes ``progress'' consists in the proliferation of those kinds of entities which are best adapted to proliferate in their own environmental niche.

\section{Evolutionary Teleology}

Though modern science differs from that of Aristotle in eschewing teleological explanations, some biologists have felt that teleological language is too valuable to discard.
The adaptive perspective sees in biological anatomy many structures which fulfil important practical roles in the survival and reproduction of the individual.
It is convenient to talk of that role as the purpose of the relevant structures, which convenience is not diminished by the possibility that some structures are not adaptive and do not have a role, or hence a purpose.

It is my aim to connect evolutionary progress with rationality and to understand evolution from an epistemological perspective, and this is facilitated by the adoption of telelogical language.

What we understand of rationality and epistemology derives from our experience of these in human activity.
Intelligent design is an important aspect of it, realised through human intelligence.
The evolution of intelligent species is the latest stage in a process of design effected by the evolutionary process, and in talking about that process it is convenient to use some of the language which we use in talking about human rationality and creativity, even though in the case of evolution there is no identificable entity to which one might attribute the intelligence and necessity.

\ 


@@@@@@@@@@@@@@@@@@@@@@@@@@@@@@@@@

\


For the purposes of this book I will proceed with a conception of evolution which is constrained  by neither of Darwin's two central planks, the idea of random variation and that of natural selection.

Evolutionary theory in Darwin's time was inevitably pit against the religious idea that the world and its inhabitants were created by an intelligent divinity.
It was essential to Darwin's theory that it be understood to be independent of divine intervention.

On the other hand, there is a great deal of support for the possibility that more or less arbitrary incremental change could be effected by selective breeding.
The essence of the theory is that those effects which might have been realised by selective breeding managed by divine intelligence can also be obtained without divine selection of breeding stock, by \emph{natural selection}.

Both of the constraints in this elementary account of evolution were necessary to formulate a theory in which divinity played no part.

In the age of genetic engineering and synthetic biology, the evolution of life on earth has now departed (perhaps as yet on a modest scale) from those to aspects of Darwinian evolution.
The variations which occur may now be elaborately designed, and the selection of which variants are propagated is far from natural.

The Modern Synthesis or Neo-Darwininan Theory, generally billed as a synthesis between Darwinian Evolution and Mendelian Genetics, by incorporating into evolutionary theory more of the specifics of the evolution of life on earth necessarily steps further away from a conception of evolution which can provide either an account of the origin of life on earth or of the kinds of evolution to which Mendelian genetics is not applicable.

For these reasons the conception of evolution which underlies the discussion in this book is broader than Darwinian evolution, and rests purely on the idea that evolution may occur in any context in which classes of entity (not necessarily living) proliferate, and that the effect of the evolutionary process is that those characteristics which best favour proliferation will come to predominate in successive generations.

\section{Some Words}\label{SomeWOrds}

Most words have diverse uses.
Some messages are can best be delivered by particular usage.
In this chapter I mention some of the most important words whose usage in this book is at risk of being misunderstood.
Its not my intention here to do other than clarify my intended usage, but in doing so I may nevertheless expose some of the opinions around which the book is constructed.

\subsection{Proliferation}

\index{proliferate}
\begin{quote}
To \emph{proliferate} is to become more numerous or more extensive.
\end{quote}

\subsection{Evolution}

Evolution is, as my title hints, enormously important to what I say, not only because an evolutionary perspective has shaped my understanding of the theses presented, but also because some of those central theses are \emph{about} evolution.

The word ``evolution''\index{evolution@EVO} has very diverse uses, but in scientific circles has come to be strongly associated with the evolutionary theory of Charles Darwin or more recent syntheses including Mendelian genetics or more.

Darwin's theory has the following principal elements:
\begin{enumerate}
\item It concerns the evolution of species
\item It supposes random variation
\item It operates by ``natural selection''
\end{enumerate}

What Darwin meant be ``natural selection'' is to be understood in contrast with what happens when domesticated animals are selectively bred.


\index{evolution!biological}
\index{evolution!biological!Darwinian}

A general conception of evolution with scope approprate for this volume is the following:

\begin{quote}
  A process of gradual change in a given system, subject, product etc., especially from simpler to more complex forms. 
\end{quote}

in relation to which I note that it does not necessarily refer to \emph{biological} systems, or even self-replicating systems, it does not require random variation in any reproductive or proliferating process, and it dpes not cpmstrain the types of selection which might be inolved.

