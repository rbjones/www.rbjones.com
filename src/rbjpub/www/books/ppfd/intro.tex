\mainmatter
\chapter{Introduction}\label{chap:Introduction}

This book is an exercise in \emph{synthetic epistemology}\index{epistemology!synthetic}, which is an abstruse kind of practical philosophy.
It is a thesis of this work that synthetic epistemology is engendered by ``\emph{The Evolutionary Imperative}'' to which imperative this book submits.

That imperative is ``Proliferate!'', and those who most successfully comply will dominate (at least, numerically) the cosmos.
This is offered as a truism which captured the broad conception of evolution as it should be understood in this volume, in which the notion of ``proliferation'' allows for variation, and the cumulative effects of differental proliferation yield systems ever more effective in proliferating in their niche or beyond.

.
Evolution, itself securing the effects of intelligent design, has `progressed' to instilling that capability into evolved species, so that homo sapiens himself is capable of intelligent design.
Key to that achievement is that homo sapiens has evolved as an epistemological engine, a cognitive system.

As earthly evolution becomes interplanetary, interstellar and then perhaps intergalactic, all those accidents of human evolution which reflect its origin on earth will gradually become less and less relevant to further proliferation and the evolution which enables it.
The further acceleration of proliferation will then be a cognitive process, and epistemological considerations will dictate the rate of progress.
Just as biological evolution would by then have been transformed by synthetic biology, the advancement of this expanding sphere of self proliferating cognitive intelligence will be underpinned by synthetic epistemology.

The knowledge required to succeed in each new phase of this evolution will be knowledge of how to survive and proliferate in yet more varied environments, how to reach and evolve in more distant or inaccessible places.
Success in discovering the science and in developing and applying the technologies thus enabled will be critical, and the architecture which best supports this cognitive enterprise may ultimately pervade the universe.

\ignore{
The architectural considerations discussed in this book are in part motivated by my perception of their relevance to this distant future scenario, but are offered independently as relevant to precisely this moment in history, as we approach the synthesis  of intelligent cognitive systems.
Our discussion addresses two main themes, that ``evolutionary imperative'' and those aspects of cognitive architecture which may be thought of as synthetic epistemology.
}

In the following I survey in turn the aspects of the history of evolution which provide context for the proposed epistemic synthesis, and the sketch the main features of the resulting synthesis.

\section{Lessons from Evolution}

\begin{enumerate}
\item The rationality of evolution
\item That evolution evolves
\item That cooperation is rational
\item Cultural evolution
\item Epistemic evolution
\item Evolution recast
\end{enumerate}

\subsection{The Rationality of Evolution}

I am not keem on the anthropomorphisation of nature, but I do here indulge in cautious stretching of language.
That extends to the use of teleological language, and thence to talkimg about the rationality of evolution.

We see that biological evolution yields organisms which are well adapted to survive and reproduce in some particular ecological niche.
In observing the features of these organisms, we fully understand the organism only when we comprehend the role each of its organs plays survival and reproduction, and it is not a great streatch to call that role its purpose, or to consider evolution to have created that organ for that purpose.

A course of action is \emph{instrumentally} rational if it is undertaken with some purpose in mind and may reasonably be expected to realise that purpose, and so in this tenuous analogy we may consider that evolution is a rational approach to the design of organs fit for purpose.

As the evolution of life on earth has progressed, the genetic diversity from which life built may be thought of as providing a body of knowledge about what proteins perform useful functions.
We may think of this as a precursor to \emph{epistemic} rationality, which concerns our grounds for belief in propositional knowledge (i.e. belief in the truth of propositions).
Evolution might be said to be a rational way of constructing knowleddge about what proteins are instrumental in the survival of individuals of the species in whose genomes the gene coding the protein appears, the geneome reading as a book naming and describing the necessary proteins.

We will discuss below how evolution pushes its capabilities down into the creatures which evolve, so that homo sapiens (and perhaps other species) exhibit both these kinds of rationality, later becoming culturally aware of their rationality and begin to self-consciously debate and construct epistemic standards.

\subsection{That Evolution Evolves}

That evolution evolves is clear, and may be seen at two important levels.
Considering only biological evolution, the machinery which supports reproduction, the design and fine tuning of which is crucial to the evolutionary process, has been refined over the course of the evolution of life on earth.

Some of the most important machinery is prerequisite for life as we know it, notably that machinery which allows the chemical basis for life to be encoded as genes and passed on to offspring during reproduction, encompassing not only the features which allow for the manufacture of very complex organic molecules, but also those which permit non-lethal variation.

Changes to this machinery might not be thought to constitute a change to evolution itself, but sometimes changes take place which will test and progress our conception of evolution itslef.

Darwin wrote ``on the origin of species by natural selection'' \cite{darwin-oos}, making both a parallel and contrast with selective breeding.
Key elements of the usual account of evolution include that variation occurs randomly during reproduction, and that selection of which parents suceed in reproducing is ``natural''.
Leaving aside the question what counts as ``natural'' here, it is appropriate to aknowledge that as soon as we come to sexually reproducing species, selection can also be described as ``sexual'', because not only is it the case that in most species mating is far from random, following elaborate (and evolved) mating rituals, but also that evolved sexual preferences are significant enough to determine anatomical effects which are counter-adaptive, except for their role in securing a sexual partner.
Darwin was of course well aware of this, and wrote about it in his ``Descent of Man and Selection in Relation to Sex'' \cite{darwin1890descent}.

Later, once evolution had begun the migration of rationality into intelligent species, sexual selection came under the influence of culture, and we may be justified in talking of ``cultural selection'', cultural norms exerting significant pressure on choice of mate, perhaps by giving an upper hand in choice of partner to close family, partially negating any evolved genetic dispositions, in favour of culturally evolved norms.

These various changes in the mechanics and nature of evolution are but a taste of the radical changes which are now afoot, in which both selection and variation are transfored by the development of technology into processes substantially controlled by conscious human deliberation and design.
Alongside this transformation in the character and machinary of evolution is the growing skill of homo sapiens in designing and constructing non-living artefacts which are physically capable and mentally intelligent, and which in due course may become or be parts of self-proliferating intelligent systems far better suited to interstellar proliferation than homo sapiens or its biological progeny.

Returning to synthetic epistemology, these more radical transformations to the evolutionary process leave us facing the possibility of interstellar intelligent self-proliferating systems whose ``life-cycle'' involves a partial or complete redesign before initiation of the next generation of such systems.
Part of what would enable rapid proliferation in such systems would be the growth in both theoretical and practical knowledge of the science and technologies involved, and the design of systems with that kind of R\&D capabiity would necessarily involve epistemological synthesis.

\subsection{That Cooperation is Rational}

Richard Dawkins in his popularisation of a gene-centric account of biological evolution \cite{dawkins-tsg} promoted a scepticism about altruism which seemed at odds with everyday experience.
The main ``exceptions'' to that scepticism related to contexts in which the beneficiaries of apparent altruism were close genetic relatives of the donor, so that from the gene centric viewpoint, no genuine altruism was involved.

It is notable that in the earliest stages of life on earth all life reproduced asexually, and in a bacterial colony (for example) the genetic identity of parnt and child would mean that self-less behaviour would be remain genetically selfish.
It may seem odd to talk about selfishness in bacteria, but even in life forms so primitive as bacteria social behaviour is pervasive, and includes the adoption in scarcity of behaviours unlikely to benefit the agent but potentially beneficial to the group, such as leaving the group in search of new sources of nutrition.

When later we come to sexually reproducing and perhaps socially selecting, the possibility that altruism, or more cautious forms of cooperative behaviour wll prove advantageous in evolutionary terms seems significant.
Intraspecies competition is most conspicuously for mating partners, and is generally undertaken in ways which limit the potential for harm to the participants.

The evolution of culture has seen the development of ethical norms which have increasing;y mitigate the worst excesses of even economic competition.

As we move forward into a future of evolution by design both pragmatic and ethical constraints are likely to lead to intelligent systems which bend over backwards to avoid conflict.

\subsection{Cultural Evolution}

Cultural evolution is important for this story in many ways.

It is difficult to overstate the very great changes which have taken place in the life of homo sapiens over the last few millenia, a blink of the eye in evolutionary timescales.

What concerns us most is the evolution of that part of culture which may be thought of as contributing to episteme, the accepted ways in which knowledge is gathered, validated and applied.
THe evolution of epistemes will be addressed in the next section, in this section the more general considerations about the development of culture are considered.



\ignore{

It is with the appearance of language and culture, probably coeval with homo sapiens, that we first see knowledge gathering to which epistemology may be directly relevant, by contrast with the  analogous phenomena observed earlier.

Prior to the advent of culture, the mature individual is shaped by the genetic material inherited by the foetus and the environment in which the foetus and immature individual came to maturity.
For example, the hormones present in the amniotic foetus is thought to influence the structure of the developing brain, the predominant visual environment will shape the perceptive capabilities of the adult, the language spoken to the infant (or lack of it) will shape the linguistic skills of the adult and many other skills, physical or mental, will never be as good as they might have been unless their acqisition is undertaken from the earliest.

One aspect of the developmental environment is what we call culture.
This has the distinctive feature that it may develop progressively and is passed as it does so from generation to generation.
Culture is not merely the passage of skills or knowledge from parents to their offspring.
It is a term we use for a body of knowledge and custom which is shared by a group.
Its effectiveness depemds upon its being shared.


The development of diverse cultures across the globe is a readily observable historical and contemporary phenomenon.
We can call that ``cultural evolution'', but without attaching a little more structure to the developments that attribution carries no benefit.
To talk of this as an evolutionary phenomenon is our broadened conception of evolution, I must argue that cultures proliferate, that the features we see in contemporary cultures derive from that capabiltiy.
We can then speculate about future cultural developments on that basis.

Cultural evolution is not well understood, particularly as to its mechanisms, but there can be little doubt that the development of culture has been critical in the advancement of humanity from a subsistence living as hunter gatherers to the prosperous society in which most of us now live.

Often this is suppposed to have been underpinned by mechanisms superficially similar to those of genetic evolution, resting on an analogy between genes and meme's.
There is so much uncertainty around the concept of meme, and so much dissimilarity between almost every aspect of the role of genes and the nature and role of memes that I have found no enlightenment in this concept.

Cultural development interacts with genetic biological evolution

}%ignore

\chapter{Sidelined}
\section{Evolution and its Imperative}

If we look into the future, on the timescales in which interstellar proliferation might take place, we can confidently expect that the nature of evolution will have been transformed beyond the confines considered by Charles Darwin, or even the subsequent `modern syntheses'.
The development of the science of genetics and the technologies of synthetic biology enable variation by design rather than chance and permit intelligent rather than natural selection.
Future generations may have genomes synthesised by recombination of selected sources, or created by intelligent design.
For this and other reasons a broadening of the evolutionary hypothesis is required to think in evolutionary terms beyond the 21st Century.

Future self-replicating systems need not reproduce sexually, need not be biological in any sense we now recognise, and need not consist of self-reproducing individuals.
Future information systems need not be built from devices constructed in huge foundaries, but may use technologies which are easy to replicate from readily available resources widely available across the cosmos.

Even if this were not the case, the proliferation of humanity across the solar system and into the cosmos would place homo sapiens in environments bearing no resemblance to those in which we evolved and would naturally lead to rapid evolution of the species.

To embrace this broader conception of evolution, we address those systems in which it can be said that change is engendered by proliferation of parts of the system, such that the rate of proliferation varies across the varies proliferating parts.
It is then the expectation that the features of those systems which proliferate more rapidly will come to predominate in the system, and that an overall progression will result from the growth in prevalence at each stage of those characteristics which are conducive to proliferation.

We may then say that these parts become gradually better optimised for poliferation in the evolving context provided by the system, and may be said to have been designed for that purpose.

What we see in the history of evolution on earth is that the environmental instability caused by external impacts on the system will place a premium on adaptability.
Those systems best able to cope with change will survive better than those to rigidly adapted to any particular state of the system.

\subsection{Evolution and Episteme}

The ability of human beings to \emph{know} is a genetically coded innate capability, but much of the detail of how it operates is culturally determined.
That part of a culture or subculture which contributes to epistemic standards, e.g. to how we decide whether a proposition is true, we call an \emph{episteme}\index{episteme}.
``\emph{synthetic epistemology}'', as that term is here used, is philosophy intended to construct or to contribute to the content or structure of epistemes.
Clearly, relativism is presupposed, that we have a choice, but this is not a radical relativism.
The choice of episteme is not without consequences and epistemes are subject to evolutionary pressures, even when constructed mindfully or purposefully.

Epistemes are a recent product of evolution, emerging as an aspect of language and culture, during the half a million years before homo sapiens appeared (maybe around three hundred thousand years ago).
Explicit talk about epistemes and the self-conscious manipulation of epistemes is a contemporary phenomenon.
The perception of epistemes as political power plays, leading to epistemic innovation as a strategem in revolutionary activism, was one aspect of the post-modern philosophy of  Michel Foucault.
The conception of \emph{synthetic epistemology} at play here is philosophical rather than political, but may nevertheless be construed as a kind of moderate activism.
The emphasis in this volume is on understanding the \emph{non-ideological} factors which are likely to influence the kinds of episteme which may prevail over the long term future.
In thus anticipating the future, we may become better prepared for it.

This account of my purpose may leave the reader in doubt about my choice of title, which suggests another subject matter altogether.
That choice of title does justice to the importance of evolutionary perspectives in my thinking about this epistemological problem.
Epistemology, and evolution are intimately connected, I suggest, because evolution is, and epistemes should be, instrumental.
Evolutionary progress is likely, in the long run, to favour those epistemes which effect the evolutionary imperative.

I will argue primarily about features of epistemes which flow from the evolutionary imperative, that they are not only worthy of consideration for our support, but also the likely product of future evolution, almost inevitable.

As a first step towards realising that ambition, I now offer a conception of evolution broad enough to encompass all the many kinds of evolution which will be considered.

\begin{quote}
Evolution is \emph{progressive change} contingent upon \emph{differential proliferation}.
\end{quote}

The \emph{evolutionary imperative} which flows from it is: ``Proliferate!'', or be overwhelmed by those who do.

The connection between evolution and epistemology can be illuminated through the concept of \emph{rationality}\index{rationality}.

\section{Rationality}

Rationality can be considered to come in two varieties, \emph{instrumental} and  \emph{epistemic}:

\begin{itemize}
\item[]\emph{instrumental rationality}\index{rationality!instrumental} is exhibited when those actions taken to realise some objective can reasonably be expected to succeed.
\item[]\emph{epistemic rationality}\index{rationality!epistemic} is exhibited when belief in a proposition is based on adequate evidence.
\end{itemize}

These are not good definitions, the supposed criteria are very imprecisely stated.
This defect is shared by every other definition I have ever seen.
There is in law a similar dependence on an imprecisely stated criteria, that of ``reasonable doubt'', and the ``reasonable person''.
These criteria for what is rational are determined by context, and the name we give to that sort of context is \emph{episteme}.

\section{Teleology}

To conceive of evolution as instrumentally rational, is a stretch, because that kind of rationality involves achievement of objectives, and hence would seem to require that there is some puposeful agent involved.

Evolution may reasonably be spoken of using teleological language.
The elaborate organisms which evolve do so because they are well adapted to proliferation in their environmental niche.
When we examine how the organism achieves proliferation, perhaps by surviving as it matures to the point at which it can reproduce, we can see that the various organs from which the organism is composed will mostly each fulfill some important role in that achievement, and that it is therefore natural to talk of that role as the pupose of the organ, and to consider that evolution has been instrumental in arriving at an organ which fulfills that purpose.

This kind of language should be entered into with a degree of caution, since it suggests that there is some mysterious entity, perhaps called ``nature'', which plays a material role in evolution.

In progressing this theme I will first of all talk about the future of evolution, by examining its past, the evolution of evolution, and by reflecting on where we now stand.
Bearing in mind that evolutionary expectation, I will then consider what kind of epistemes are likely to thrive in it, considering the possibilities as natural continuations of the epistemic evolution which has brought us to this point.

\ignore{


An important factor in these considerations is the perception that evolution is \emph{instrumental} and hence might be considerd \emph{rational}, and that it will therefore tend to favour epistemes which have similar characteristics, since they are conducive to proliferation of the subcultures which share those characteristics.
The idea here is that instrumentalism, far from being a tool of oppression, is an enemy of extreme ideology, conducive to prosperity and proliferation, and \emph{ethically neutral}.


There is a lot of talk about evolution in this book, but my perspective is more philosophical than scientific.

To write philosophically about evolution demands terminological clarity, especially when the boundaries of the concept are to be streatched.

According to dictionaries, just one meaning of evolution encompasses any process of gradual change.

Our usage is broader than Darwin's idea of the evolution of species by natural selection or its more modern elaborations, but more specific than the most general dictionary definitions.

\begin{quote}
Evolution is \emph{progressive change} contingent upon \emph{differential proliferation}.
\End{quote}

To proliferate here means simply to become more numerous or more extended, without a presumption of perfect reproduction or indeed that the proliferation takes place by reproduction at all (rather than by repetition of some other method of construction or formation as may be the case in pre-biotic molecular evolution).
THe word ``differential'' indicates that some kinds of entities proliferate more rapidly than others.
The idea of ``survival of the fittest'' refined perhaps by the concept of reproductive fitness suitably defined, is realised in this context by a concept if proliferative fitness, and becomes ``proliferation of the proliferatively fit''.

Thus conceived covers the pre-biotic evolution of life, and the likely and lengthy sequel to the very recent evolution of homo sapiens.

The evolutionary imperative is, I suggest, ``Proliferate!''.
We may think of this as an imperative issued by a mysterious enitity called ``, but in truth it is a necessary and sufficient natural condition pre-requisite to evolution.
i.e. evolution only takes place where we have things which proliferate.

Note that in keeping with the scoping of evolution beyond biology, ``proliferate'' need not connote ``reproduce'', but only requires increasing in numbers or extent.

If we continue along the lines of attributing to evolution matters whose natural occurrence is part of the phenomenon, we may note that evolution is \emph{instrumental} in the creation of entities which are able to proliferate in the environments within its scope.

£££££££££££££££££££££££


Darwin's theory of evolution can be stated quite simply.
Darwin's work was not so much in articulating the theory, but in the extensive research which provided evidence for it.

The ideas behind this volume are also quite simple at their core, and I will waste no time in their exposition, which will be presented in the following stages.
First a concise statement of my central theses.
Then I will try to make them intelligible.
After that I will attend to their plausibility, and finally elaborate on their consequences.

But before that let me show the gross structure of the story:

\begin{enumerate}
\item Evolution
\item Teleological Language
\begin{itemize}
\item role
\item purpose
\end{itemize}
\item Rationality
\begin{itemize}
\item instrumental
\item epistemic
\end{itemize}
\item Epistemology
\begin{itemize}
\item epistemes
\item epistemic evolution
\end{itemize}
\end{enumerate}

\section{First Sketch}
I begin with the definition of a conception of evolution broader than that of Darwin, broad enough to encompass the evolution of non-biological systems and hence the pre-biotic evolution of life, and the evolution of potentially non-biotic self-proliferating synthetic intelligent systems.

The exiting feature of evolution is that it realises structures which are well adapted to fulfill some specialised role in the proliferation of those kinds of individuals or subsystems to which they belong.
For this reason it is often thought appropriate to use teleological language to describe these structures.
We regard the role which they fulfill in proliferation as being their purpose, even though it is not a purpose conceived by some intelligent designer.

Supposing that we are thus content to talk of the purpose of these structures, we are then in a position to observe that evolution is \emph{rational} om that instrumental sense in which rationality consists in adopting a means to some end which is likely to realise the end.
Good design is a hallmark of instrumental rationality.

In a variety of ways we may then go on to attribute to evolution \emph{epistemic} rayionality.
The difference between instrumental and epistemic rationality is similar to that between knowing \emph{how} and knowing \emph{that}.



\ 

@@@@@@@@@@@@@@@@@@@@@@@@@@@@

\

Since Darwin published his work on the evolution of species, similar ideas have been applied in very many other areas.

In seeking to understand what is happening around us and what the future might hold, I too have looked at the past from an evolutionary perspective and considered what that might entail for our futures.
Though seeking understanding through evolution, the understanding I seek is philosophical, and in that understanding epistemological considerations have become preeminent.

In this book I aim to present the ideas, and in this introduction to provide a preview of the book.

In considering the evolution of life, of homo sapiens and of the culture which exercises so great an influence on our lives, I have noted many changes in \emph{the process} of evolution.
It seems likely that recent technological advamces will, as they mature, facilitate a transformation in how the evolution of homo sapiens and its progeny takes place.

Both before and after the biological evolution of homo sapiens evolutionary processes which do not wholly comply with the Darwinian conception of evolution, or the modern synthesis which followed it, have played essential roles.
I have therefore found it necessary to make definite a conception of evolution which, while encompassing Darwinian evolution, allows kinds of evolution which he did not address.

The idea of an evolutionary imperative comes from that broader conception of evolution which I characterise thus:

\begin{quote}
Evolution is \emph{progressive change} contingent upon \emph{differential proliferation}.
\end{quote}

The imperative which flows from that conception is ``Proliferate!''.

I will return later to a fuller explanation of these terms and that definition, but for the time being it may suffice to note that ``differential proliferation'' plays a role in this generalised idea of evolution similar to that of ``survival of the fitness'', or more accurately ``reproductive fitness'' in biological evolution.
This makes the definition strictly broader by dint of makng no assumption that proliferation occurs by reproduction.
That may not be the case (there may be no reproduction), for example, in pre-biotic chemical or molecular evolution, and may not be the case in the evolution of future self-proliferating synthetic intelligent systems.

In such evolutionary processes ``progress'' consists in the proliferation of those kinds of entities which are best adapted to proliferate in their own environmental niche.

\section{Evolutionary Teleology}

Though modern science differs from that of Aristotle in eschewing teleological explanations, some biologists have felt that teleological language is too valuable to discard.
The adaptive perspective sees in biological anatomy many structures which fulfil important practical roles in the survival and reproduction of the individual.
It is convenient to talk of that role as the purpose of the relevant structures, which convenience is not diminished by the possibility that some structures are not adaptive and do not have a role, or hence a purpose.

It is my aim to connect evolutionary progress with rationality and to understand evolution from an epistemological perspective, and this is facilitated by the adoption of telelogical language.

What we understand of rationality and epistemology derives from our experience of these in human activity.
Intelligent design is an important aspect of it, realised through human intelligence.
The evolution of intelligent species is the latest stage in a process of design effected by the evolutionary process, and in talking about that process it is convenient to use some of the language which we use in talking about human rationality and creativity, even though in the case of evolution there is no identificable entity to which one might attribute the intelligence and necessity.

\ 

@@@@@@@@@@@@@@@@@@@@@@@@@@@@@@@@@

\


For the purposes of this book I will proceed with a conception of evolution which is constrained  by neither of Darwin's two central planks, the idea of random variation and that of natural selection.

Evolutionary theory in Darwin's time was inevitably pit against the religious idea that the world and its inhabitants were created by an intelligent divinity.
It was essential to Darwin's theory that it be understood to be independent of divine intervention.

On the other hand, there is a great deal of support for the possibility that more or less arbitrary incremental change could be effected by selective breeding.
The essence of the theory is that those effects which might have been realised by selective breeding managed by divine intelligence can also be obtained without divine selection of breeding stock, by \emph{natural selection}.

Both of the constraints in this elementary account of evolution were necessary to formulate a theory in which divinity played no part.

In the age of genetic engineering and synthetic biology, the evolution of life on earth has now departed (perhaps as yet on a modest scale) from those to aspects of Darwinian evolution.
The variations which occur may now be elaborately designed, and the selection of which variants are propagated is far from natural.

The Modern Synthesis or Neo-Darwininan Theory, generally billed as a synthesis between Darwinian Evolution and Mendelian Genetics, by incorporating into evolutionary theory more of the specifics of the evolution of life on earth necessarily steps further away from a conception of evolution which can provide either an account of the origin of life on earth or of the kinds of evolution to which Mendelian genetics is not applicable.

For these reasons the conception of evolution which underlies the discussion in this book is broader than Darwinian evolution, and rests purely on the idea that evolution may occur in any context in which classes of entity (not necessarily living) proliferate, and that the effect of the evolutionary process is that those characteristics which best favour proliferation will come to predominate in successive generations.

\section{Some Words}\label{SomeWOrds}

Most words have diverse uses.
Some messages are can best be delivered by particular usage.
In this chapter I mention some of the most important words whose usage in this book is at risk of being misunderstood.
Its not my intention here to do other than clarify my intended usage, but in doing so I may nevertheless expose some of the opinions around which the book is constructed.

\subsection{Proliferation}

\index{proliferate}
\begin{quote}
To \emph{proliferate} is to become more numerous or more extensive.
\end{quote}

\subsection{Evolution}

Evolution is, as my title hints, enormously important to what I say, not only because an evolutionary perspective has shaped my understanding of the theses presented, but also because some of those central theses are \emph{about} evolution.

The word ``evolution''\index{evolution@EVO} has very diverse uses, but in scientific circles has come to be strongly associated with the evolutionary theory of Charles Darwin or more recent syntheses including Mendelian genetics or more.

Darwin's theory has the following principal elements:
\begin{enumerate}
\item It concerns the evolution of species
\item It supposes random variation
\item It operates by ``natural selection''
\end{enumerate}

What Darwin meant be ``natural selection'' is to be understood in contrast with what happens when domesticated animals are selectively bred.


\index{evolution!biological}
\index{evolution!biological!Darwinian}

A general conception of evolution with scope approprate for this volume is the following:

\begin{quote}
  A process of gradual change in a given system, subject, product etc., especially from simpler to more complex forms. 
\end{quote}

in relation to which I note that it does not necessarily refer to \emph{biological} systems, or even self-replicating systems, it does not require random variation in any reproductive or proliferating process, and it dpes not cpmstrain the types of selection which might be inolved.

}%ignore
