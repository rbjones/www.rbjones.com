% $Id: intro.tex,v 1.1 2011/12/04 19:50:45 rbj Exp $

\chapter{Introduction}

This is the second of two volumes in which a certain ideas which I call \emph{Positive Philosophy} are presented.
The first volume \cite{rbjb004} is concerned with the \emph{theoretical} aspects of Positive Philosophy, this volume is concerned with the \emph{practical} side.

The distinction between the two corresponds roughly to the distinction between matters of fact, and judgements of value.

The first volume therefore addresses questions in epistemology, metaphysics and logic, in the philosophy of mathematics science and engineering and generally with any domain in which deductive reasoning may be possible.
This volume is concerned with what might be called personal or existential matters, with ethics and morality, and with political, economic, and other social matters.

In both of these volumes I am concerned to connect the philosophy with what seem to me to be important contemporary issues an understanding of which may be important for our futures.
In both cases the future impact of information technologies is significant.
In the first the theoretical issues are connected with more practical matters connected with the automation of reason.
In this volume the ``practical'' philosophy is commected with some of opportunities and challenges which information technology brings to the conduct of our lives and the workings of social institutions.
Of these the ways in which individuals are now and will in the future be able to influence the world around us, both through political and economic institutions are of particular interest.

The two volumes are intended to be largely independent, and for that reason I begin this volume with an account of those aspects of the theoretical philosophy (which I sometimes call ``metaphysical positivism'') which are most important for an understanding of the practical philosophy to be considered.
I aim in this to give an account which is more compact and focussed, and which is oriented toward the subsequent topics, and which extends beyond the first volume into certain theoretical aspects of practical philosophy and the relationship between theoretical and practical philosophy.

Beyond a general discussion of practical philosophy, two areas in which information technology might enable or impell changes are explored.
These concern the nature of democracy, and the operation of market economies, and the ways in which these might enable significant changes to our conception of democracy and to the freedom and opportunities for the individual.

The automation of reason is used in the first volume as a way of focussing an account of a theoretical philosophy.
In this volume an exploration of possible changes in democratic politics and free market economies serves a similar purpose, providing a motivation for the preceding account of practical philosophy.

