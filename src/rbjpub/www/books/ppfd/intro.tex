% $Id: intro.tex $
\mainmatter
\chapter{Introduction }

The capability of \emph{homo sapiens} for intelligent rational thought, belief and action has been instrumental in our species becoming the dominant species of our planet, and in the rapid rise in population, wellbeing and prosperity which we have seen, particularly in modern times.

However, that capability for rationality is complemented, offset, or perhaps negated by strong social bonds and cultural influence over all aspects of life which may seem to evince irrational belief and behaviour.
I have struggled to understand this apparent conflict or to form opinions about how we might best to respond to it.

A part of the apparent irrationality in contemporary discourse is claim that rationality is a feature of certain cultures and is created by them for the purpose of oppressubg others.
This has motivated me in seeking a better understanding of rationality and its origins, and of those ideas which oppose or repress it.

From such beginnings this book has arisen.

In this introduction, I propose to elaborate on this seed to a sketch of and a rationale for the structure of the book.
First some clarification of the notion of rationality as it will be considered here.

\section{What is Rationality?}

Two particular kinds of rationality are important to this enterprise, of which the first is fundamental:

\begin{quote}
A course of action is \emph{instrumentally} rational if it is undertaken with some purpose in mind, and it is likely to realise that purpose.
\end{quote}

A second conception of rationality is arguably a special case concerned with circumstances in which the immediate purpose at hand is to know or to understand.
Establishing the knowledge might itself be instrumentally rational in realising some other purpose.

\begin{quote}
A belief is \emph{epistemically} rational if the grounds on which it is held make it likely that it is true. 
\end{quote}

These are not precise definitions.
They each connect rationality with reason.
In the instrumental case, reason to expect effectiveness in the epistemic case, reason to expect truth, but leave open the question what consitutes good reason.
The details I suggest are determined by context, by a culture representing a more or less sharp collection of ideas about what is reasonable which evolves and refines over time.
Though I am disclined to think that rationality in itself is a peculiarity of a single culture,  much of the detail in what we consider rational is culturally sensitive.
This applies also to the varieties of subculture which criss-cross our spcial landscaoe and us exemplified by Thomas Kuhn's theory of scientific paradignms \cite{kuhn2012structure}.
