% $Id: intro.tex $
\mainmatter
\chapter{Introduction }

The capability of \emph{homo sapiens} for intelligent rational thought, belief and action has been instrumental in our species becoming the dominant species of our planet, and in the rapid rise in population, wellbeing and prosperity which we have seen, particularly in modern times.
That capability for rationality is complemented, offset, or perhaps negated by strong social bonds and by cultural influence over all aspects of life, which may seem to evince irrational belief and behaviour.
I have struggled to understand this apparent conflict or to form opinions about how to address issues which arise from it.

A part of the apparent irrationality in contemporary discourse is the claim that rationality is a feature of certain cultures and is created by them for the purpose of oppressing others.
It seems more natural to believe that the capability for rationality is common to all humanity, and has value beyond any contribution it makes to manipulating others.
In response to these developments some have advocated a return to ``enlightenment values'' \footnote{Notably, Steven Pinker in his book \emph{Enlightenment Now}\cite{pinker-en}.}, others in despair of contemporary ``liberalism'' have professed to be ``classical liberals''.
I doubt that the tide of progressive activism can be reversed by turning back the clock, and therefore seek a forward looking philosophy which combines the best of our past principles with innovations drawn from contemporary critiques and advances, and from an understanding of rationality and culture as it is revealed by our history of biologica and cultural evolution.

From such beginnings this book has arisen.

In this introduction, I propose to give a preview of the book, covering all the most important points advanced, leaving only fine detail and peripheral accoutrements for those who have appetite to continue.

First some clarification of the notion of rationality as it will be considered here.

\section{What is Rationality?}

Two particular kinds of rationality are important to this enterprise, of which the first is fundamental:

\begin{quote}
A course of action is \emph{instrumentally} rational if it is undertaken with some purpose in mind, and it is likely to realise that purpose.
\end{quote}

A second conception of rationality is arguably a special case concerned with circumstances in which the immediate purpose at hand is to know or to understand.
Establishing the knowledge might itself be instrumentally rational in realising some other purpose.

\begin{quote}
A belief is \emph{epistemically} rational if the grounds on which it is held make it likely that it is true. 
\end{quote}

These are not precise definitions.
They each connect rationality with reason,
in the instrumental case reason to expect effectiveness, in the epistemic case reason to expect truth, but they leave open the question what consitutes good reason.
The details I suggest are determined by context, by a culture representing a more or less sharp collection of ideas about what is reasonable which evolves and refines over time.
Though I am disclined to think that rationality in itself is a peculiarity of a single culture,  much of the detail in what we consider rational is culturally sensitive.
This applies also to the varieties of subculture which criss-cross our social landscaoe and us exemplified by Thomas Kuhn's theory of scientific paradignms \cite{kuhn2012structure}.

\section{Evolution and Rationality}

The understanding of rationality which I seek here derives from a study of its origins and development.
I will argue that human rationality is co-eval with our species, but preceded by an evolutionary process which is itself rational, and is subsequently refined by the rather different process of cultural evolution.
We live at a moment of potential radical transformation in the nature of evolution, at which for the first time biological evolution may fall under the sway of cultural evolution, and the evolutionary progeny of homo sapiens become diverse and break the bounds of life as we know it.

To describe evolution as rational depends upon some conception of purpose which it effectively fulfills.
I don't claim that evolution actually \emph{has} a purpose, just that it nevertheless fulfills one, so if not strictly rational, its effects are those one would expect from a rational process with that purpose.
The purpose it fulfills is the design of things which are able to proliferate in available environemnts.

What we find is that as the available environments are populated by the species which have evolved to proliferate in them, the character of these environments us transformed and new, often more complex designs are necessary to proliferate in them.
As well as the environmental changes due to evolution, sometimes sudden and violent changes occur independently of the evolutionary process.
Earthquakes, volcanic activity and meteor impacts can have profound effects on the environment.
Environmental instability places a premium on adaptability. which ultimately leads to the evolution of dynamic rational capabilities in the objects of evolution rather than the process, for the sake of the very much more rapid response to environmental change that they enable.


\begin{itemize}
\item Evolution
\begin{itemize}
\item In General
\item Molecular
\item Biological
\item Sexual
\item Cultural
\item Synthetic
\end{itemize}
\item Rationality
\begin{itemize}
\item In General
\item Evolutionary
\item Neural
\item Cerebral
\item Nueral
\item Epistemic
\item Deductive
\item Humean
\end{itemize}
\item Elightenment
\begin{itemize}
\item Ancient Greece
\item THe Age of Reason
\end{itemize}
\end{itemize}
