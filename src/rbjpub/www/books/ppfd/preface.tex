\chapter*{Preface}
\addcontentsline{toc}{chapter}{Preface}

Utopianism has a bad name.
Those who envisage an ideal society may be willing to endure some discomfort along the way, and may be willing for others to incur even greater privations.
More seriously, human biology, and culture has evolved over extended timescales.
Noone knows how to design a working social system, and whatever merits are envisaged in an ideal society, they are unlikely to accrue.
It is not hard to elaborate on the reasons why ``utopian engineering'' is almost certain to fail.

Nevertheless, it is in human nature to seek to improve our lot, for which one has to be capable of comprehending how things might be better than they are, and how desirable improvements might be undertaken.
More urgently, we have a duty to ourselves and those around us to be mindful of the risks which face us, not merely of some deterioration in our situation, but even of catastrophy.

Karl Popper's advocacy against utopian thinking came with a preference for `piecemeal social engineering'.
What he did not discuss was the possibility that one might have a ``uptopian'' vision constrained by a recognition that genuine progress can only be realised by consensual incremental, evolutionary change.
Against the risk that the ideals embodied in this essay are considered utopian, I offer the consolaition that they are compatible only with that kind of progress, and that they are particularly concerned to protect our institutions against those totalitarian tendencies which have been so often associated with utopianism.
