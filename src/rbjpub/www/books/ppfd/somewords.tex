\chapter{Some Words}\label{SomeWOrds}

Most words have diverse uses.
Some messages are can best be delivered by particular usage.
In this chapter I mention some of the most important words whose usage in this book is at risk of being misunderstood.
Its not my intention here to do other than clarify my intended usage, but in doing so I may nevertheless expose some of the opinions around which the book is constructed.

\section{Proliferation}

\index{proliferate}
\begin{quote}
To \emph{proliferate} is to become more numerous or more extensive.
\end{quote}

\section{Evolution}

Evolution is, as my title hints, enormously important to what I say, not only because an evolutionary perspective has shaped my understanding of the theses presented, but also because some of those central theses are \emph{about} evolution.

The word ``evolution''\index{evolution@EVO} has very diverse uses, but in scientific circles has come to be strongly associated with the evolutionary theory of Charles Darwin or more recent syntheses including Mendelian genetics or more.

Darwin's theory has the following principal elements:
\begin{enumerate}
\item It concerns the evolution of species
\item It supposes random variation
  \item It operates by ``natural selection''
\end{enumerate}

What Darwin meant be ``natural selection'' is to be understood in contrast with what happens when domesticated animals are selectively bred.


\index{evolution!biological}
\index{evolution!biological!Darwinian}

A general conception of evolution with scope approprate for this volume is the following:

\begin{quote}
  A process of gradual change in a given system, subject, product etc., especially from simpler to more complex forms. 
\end{quote}

in relation to which I note that it does not necessarily refer to \emph{biological} systems, or even self-replicating systems, it does not require random variation in any reproductive or proliferating process, and it dpes not cpmstrain the types of selection which might be inolved.

