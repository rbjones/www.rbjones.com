\chapter{The Evolution of Evolution}\label{EvolutionOfEvolution}

That evolution evolves is important to the story I hope to tell, not least because the story concerns the future, and we are at a point in history which will likely prove to have been a particularly striking transition in the most significant kinds of evolution.

This chapter sketches some of the ways in which evolution here on earth has been transformed over the last three billion years, as background for later consideration of how it might change in the next three billion.

There is room for debate about what really consitutes a change to evolution itself.
I don't intend to engage with that question, but it may be helpful to say something about the different kinds of change which are discussed in this chapter.

I will touch upon the history of the theory, as it is told, first by Darwin and then by numerous professional biologists, but my main focus is on the history of evolution and the changes which have taken place in how evolution works rather than how we describe or theorise it.

Some of those changes are so substantial that they would be very likely to impact on how one might define the evolutionary process.
The modern synthese are closely enough wed to the mechanics of life on earth that pre-biotic evolution, cultural evolution, and the kind of evolution which we may see as the technologies of synthetic biology, artificial intelligence and artificial life are are developed and applied are surely beyond their scope. 
Some of those changes are important to the effectiveness and speed of the process, by transforming or fine tuning what kinds of variation are possible during replication and with what frequencies. 

The main phases in the evolution of life on earth, which are punctuated by notable shifts in what kind of evolution is taking place or how that evolution works are:

\begin{enumerate}
\item  Pre-biotic chemical evolution
\item  Evolution in prokaryotic life
\item  Eukaryotes and sexual reproduction
\item  Intelligence and Cultural Evolution
\end{enumerate}

Before addressing those major phases in the evolution of evolution, a few words about Darwin's theory.

\section{The Evolution of the Theory}

Though the idea of evolution has a longer history, it is only with Darwin that we first have a well articulated and copiously researched scientific theory, which he presented in his book ``The Origin of Species'' \cite{darwin-oos}. 

Darwin spoke of the evolution of species by natural selection.
Modern presentations often talk of the need also for variation and for that variation to be random.

The principle aim was to provide convincing evidence that life was not created in its present state by a divine creator, but evolved in a natural analogue of the process of selective breeding.
Talk of ``natural selection'' tells us that the selection which takes place was not accompished by selective breeding either by human or by divine agents.
It risks being read as replacing one kind of metaphysical entity with another, and that the success of evolution might be credited to the prescience of `mother nature'.
What is intended is that there is no selection taking place at all, but that not all are equally capable of or lucky in surviving and procreating in the circumstances they find themselves in.

The phrase ``survival of the fittest'' was also in use, which also has its disadvantages, the principle problem being that it is procreation rather than survival that counts.
`reproductive fitness' is one of the several improvements which were later promoted.

Darwin 

\section{Pre-Biotic Evolution}

Before the evolution of life, there was the `primordial soup', i.e. the ocean's of water on earth with at first quite simple other chemical constituents.
The progressive changes which took place in the composition of that primordial soup would not qualify as ``evolution'' either in Darwin's conception or in any of the later more sophisticated and eloborate syntheses.

In first instance the composition of this soup would alter simply because the rates of formation of more complex molecules from their constituents as they came together, by accident, with appropriate energy (perhaps supplied by lightening or volcanic action), differed from the rates at which these molecules were torn asunder.
These differences would have caused gradual change in the composition of the soup, with more complex molecules becoming more common, and thus beginning or accelerating the formation of even larger molecules.

A variety of mechanisms cause significant divergence in the rates of formation of complex molecules.
One is the existence of catalysts.
Catalysts are molecules which facilitate chemical processes while themselves remaining unchanged.
The formation of catalysts may therefore change the rate at which other complex molecules are formed, and hence accelerate evolution of the primordial soup.

Autocatalytic sets are collections of molecules including catalysts which not only jinly facilitate the formation of some chemical product, but also create all the constituents of the set, hence prefiguring the self-reproductive capability of cellular life.

These natural augmentations of the kinds of simple chemistry first found on earth fall well short of supporting the kinds of complexity found in even the most simple life forms, but the gradually changing differential rates of molecular formation will eventually lead to life and the special kinds of evolution particular to biological ecosystems.
Two important further ingredients are needed.
Firstly, for autocatalysis to be most effective the products of the chemical reactions need to be contained.
If they freely disperse in open oceanic waters, then their effectiveness is muted.
It is possible that small pools might provide environmens in which evolution progressed faster because of the limitations on dissipation which they offer, but ultimately it is the cell boundary which solves this problem.
The second problem is the enormous complexity which is found in the molecules which life forms depend upon, which because of that complexity will appear too infrequently to support life.

The miracle of life depends upon the use of copying mechanisms to reproduce these very complex molecules, but the variety of chemical effects which these molecules serve to effect makes any general copying mechanism infeasible.
Instead, a way of coding up their structure in molecules of regular structure which are readily copied and can be used to guide the construction of the intended chemical provided a solution to the creation of those complex proteins upon which life depends.

Exactly how that machinery evolved remains a mystery, but the enclosure of this machinery together with the \emph{genetic} codes for the proteins for running a cell made self replicating cellular life possible, and facilitated a new kind of evolution.

\section{Evolution in Prokaryotic life}

The simplest and earliest living cells were called prokayotes.
They reproduced by cell division, in that process replicating the genetic codes for the proteins which facilitated all the required chemical reactions for the continuation of the life of the cell.
The correct copying of this genetic information is what makes the cells resulting from the division copies of the original cell, but occasional errors will introduce errors and hence differences between cells, and differences in their ability to replicate.
Those modifications which facilitate replication will yield large polulations of descendents and in this way the forms of life will evolve.

\section{Eukaryotes and Sexual Reproduction}


\section{Intelligence and Cultural Evolution}
