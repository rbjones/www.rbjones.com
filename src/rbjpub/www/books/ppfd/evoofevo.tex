\chapter{The Evolution of Evolution}\label{EvolutionOfEvolution}

In this chapter I aim to give credence to the view that evolution is itself evolving.
I will mention some of the diverse kinds of evolution which we have already happened, in preparation for consideration of how human evolution may be transformed in the future.

The thesis that evolution evolves may be considered at two levels.
One concerns the machinery of evolution, the nuts and bolts.
The other consists in a progression, more important in this context, in the the broad characteristics of the evolutionary process, a progression through various kinds of evolution.
These are not unrelated, it will typically be a long progression of changes at the nuts and bolts level which eventually effects a transition at the upper level, the boundaries of which may not be sharp.

In order to contrast these kinds, some discussion of Darwinian evolution in the context of the broader conception of evolution I have proposed will be helpful.
Darwin's conception of evolution is more definite in the following ways.

First Darwin specifically addressed the evolution of (biological) species by natural selection, a conception which was intended to cover the entire evolution of life on earth, from the first unicellular prokaryotes (if they were indeed the first living organisms) through to homo sapiens.
Proliferation tool place by self-reproduction of the members of the species.
We would now understand this reproduction to be effected by the reproduction of genetic information which controls the development of the new organism from conception through to adulthood.
The first scientific account of this kind of genetic transmission of characteristics was due to Mendel, but Darwin was not aware of Mendel's work. 

I will illustrate this by comparing each of five kinds of evolution which have contributed to the present condition of homo sapiens.

For brevity these are referred to as pre-biotic, asexual, sexual, cultural, and cultural selection.

Dawinian evolution is characterised as the evolution of self-reproducing organisms by radom variation and natural selection.
Of the five kinds described above, only one fully fits this conception of abortion.
The greatest divergence from that is in pre-biotic and cultural, in which proliferation is not realised by self-reproducing organisms (and indeed not by organisms at all).
Within the domain of biological or organic evolution, variation from Darwin's model can be had by changes to the ways in which variety is introduced, or by changes in the kinds of selection at work.
Even in the period when the ecosphere consisted primarily of single celled prokaryotes which do not reproduce sexually, the sources of variety were far from random.
Even if the errors in duplicating genetic material in cell division were random, there would still be variety introduced by multiple mechanisms for `horizontal' gene transfer, which evolved and could therefore be expected to have been moulded by evolution to deliver kinds of variety conducive to proliferation of the participant.
It is these diverse generators of variety which would ultimately evolve into the elaborate recombinatory processes which characterise sexual reproduction, and which introduce a form of selection into the evolutionary process.
Sexual selection introduced a new dimension to evolution.

  
\ignore{

To do this I will work from the very abstract definition of evolution which I gave in Chapter \ref{chap:Introduction}, refining that definition to yield narrower conceptions of evolution applicable to the varieties discussed.

Biologists work with quite different definitions of evolution which serve ends distinct from those which concern is here.
But I will begin with a comparison between Darwin's conception of evolution and my most general characterisation, so as to further explain the latter.
Following Darwin's work Biological science greatly advanced building on his `dangerous idea', and the science and scientific methods were to form a `modern synthesis', an idea popularised by Julian Huxley \cite{huxley-tms}.


 In the expectation that the character of evolution will be transformed by technical recent and prospective technical development I have tabled a very broad conception of what might be considered evolutionary, intending to give credence to the idea that evolution might be transformed by talking about the transformations which have already occurred.

 The evolution which I perceive in evolution itself often consists in the gradual transformation of the mechanisms which are involved in its various crucial components, but sometimes amounts to more radical transformations which we might think of as paradigm shifts and might test the boundaries of our conception of what can reasonable be called evolution.

 My aim here is primarily to talk about those more substantial transformations in the nature of evolution with limited excursions in illustrative detail.
 To that end I introduce some terminology.

 First I expand a little on the broadened conception of evolution of within which all tge following are to be understood.

 There are a small number of kinds of evolution which I am mainly concerned:

\renewcommand{\theenumi}{\Alph{enumi}}
\begin{enumerate}
\item Pre-biotic chemical evolution
\item Asexual biological evolution
\item Sexual biological evolution
\item Cultural evolution
\item Culturally selected biologcial evolution
\item Synthentic biological evolution
\item Synthetic non-biological evolution
\end{enumerate}

 Those kinds of evolution are distinguished by the presence or absence of certain chanracteristics:

 \renewcommand{\theenumi}{\alph{enumi}}

\begin{enumerate}
\item Proliferation by self-reproduction

  This is of course a principle characteristic of biological evolution, and it is a stretch to consider as evolution a system which does not have this characteristic.
  The characteristic is not without its ambiguity.
  Inert entities which carry information might not be thought properly to be considered ``self-reproducing'' however easy it may be for some more active agent to copy them, but genes are surely in that category, requiring elaborate chemical machinery for reproduction.
  Against that point we observe that self-reproduction always does depend on a suitable environment, and that viruses and bacteriophages, clearly designed for proliferation, depend upon just that sort of reproductive environment for their self-replication.
  Nevertheless, in the pre-biotic evolution of the complex molecules essential to life most molecules come to be in ways which involve no connection with a similar molecule which could be said to have self-replicated, the involve only the repetition of a mode of synthesis which requires no model.
  
\item Germ-soma differentiation
\item `Natural' selection
\item Sexual selection
\item Cultural selection
\item Synthentic variation
\item Intelligent selection
\end{enumerate}

 I enclose ``natural'' in quotes above because I use the term to reflect the usage of Charles Darwin, even though its moot whether it should be considered any more natural than any of the other kinds of selection.
 I believe it was thought of in the day as consisting in survival of the fittest, though may now be supplanted by the idea of  ``reproductive fitness'', recognising that survival alone will not suffice, once must also reproduce.

 Consequently, sexual reproduction becomes a significant amendment to the machinery of reproduction, since it introduces an evolvable selection mechanism which may send evolution in directions which have nothing to do with survival but a great deal to do with more or less arbitrary constraints on reproduction.
 Its not my intentino here to cast sexual reproduction in a negative light on the evolutionary stage.
 When we look more closely at the whole package we see good reason for it to have become the dominant way in which life proliferates.

 Let us pause to attach the attributes we have listed to the nominated kinds of evolution.

 \renewcommand{\theenumi}{\Alph{enumi}}

\begin{enumerate}
\item Prebiotic evolution has non of the identified special characteristics
\item Asexual biolgical evolution has a, b, c
\item Sexual biological evolution has a, b, d
\item Cultural evolution may have e, f, g 
\item Culturally selected biological evolution has a, b, e and g
\end{enumerate}

\chapter{The Evolution of Evolution}\label{EvolutionOfEvolution}

Thgough the Theory of Evolution which has come to us from Charles Darwin is specific to the evolution of biological species there are other kinds of evolution which are part of a broader evolutionary development of which biological evolution is a part.
It is generally believed that before the first living organisms appeared, there was an a process of evolution taking place in ``the primordial soup'' which ultimately created the first living organisms.
After a long period of biological evolution intelligent life appeared, and with it culture became significant and itself developed progressively in an evolutionary process quite different from those two previous kinds of evolution.
Modern science now enables the construction of intelligent artefacts and has enabled synthetic biology, these promise profound changes in the way in which both biological evolution takes place.

Arguably these devlopments to the nature of evolution are progressive, and it may therefore be reasonable to speak of this phenomeon as the evolution of evolution.
At any rate, I believe it is a useful perspective and will facilitate the projections into the future which we will discuss in later chapters.

To progress this idea, we will consider the principle features of Darwin's theory of evolution, consider how these various other kinds of evolution differ from Darwin's, and consider under what broader conception of evolution, the evolution of evolution might fit.

Darwin wrote of the evolution of species by natural selection.

Let me summarise the key elements of Darwin's theory as:

\begin{enumerate}[label=\Alph*]
\item Living organisms naturally proliferate by self-reproduction 
\item Many traits of such organisms are heritable, i.e. if present in the parent they will nornally be present also in the child.
\item Even hereditary traits are not always faithfully transmitted to progeny, but are subject to a degree of variation, which to a first approximation may be considered random.
\item Not all variants are equally capable of surviving to adulthood and procreating, this effect is attribued to ``natural selection'' which is contrasted with the selective breeding by which the characteristics of domesticated animals are shaped.
\item By this means the characteristics of a species are gradually transfored to provide a better fit with its environment or environmental niche.
  \item Speciation occurs when the characteristics of groups which occupy different niches or are geographically separated, and hence are not in a position to interbreed, diverge, and may eventually diverge so far as to prevent interbreeding (in sexually reproducing species).
\end{enumerate}

Which may then be glossed as `the evolution of biological species through random variation and natural selection'.

Darwin's knowledge of the mechanics of heredity was limited, and he was not acquainted with Mendel's work on genetics.
The incorporation of Mendelian genetics into the evolutionary thesis was undertaken
in the first half of the twemtieth century and the result was called the `Modern Synthesis'. \footnote{So named by Julian Huxley \cite{huxley-tms}.}
During the 20th Century the science of inheritance advanced rapidly, the discovery of the role and structure of DNA transformed the field.
At the same time a new perspective on evolution became prominent, the gene centered perspective originating with G.C. Williams.\footnote{Popularised by Richard Dawkins in ``The Selfish Gene'' \cite{dawkinsSG}.}

The gene centred evolutionary perspective, as presented by Dawkins, incorporated severe contraints on the possibility that altruistic traits might be evolved, and the gene focus allowed selection only of traits tending to the proliferation of the genes which facilitate them, rather that the development of characteristics beneficial to organisms, tribal groups or species.
Some professional biologists were not happy with such gene reductionism, and continued to press the case for ``Group Selection'' of various kinds.

Continued advances in evolutionary science and the ongoing debate about group selection have stimulated attempts at ``extended evolutionary syntheses'' which incorporate admit multi-level selection.

\chapter{The Evolution of Evolution}\label{EvolutionOfEvolution}


\Cite{dawkins-eoe}

That evolution evolves is important to the story I hope to tell, not least because the story concerns the future, and we are at a point in history which will likely prove to have been a major transition in the most significant kinds of evolution.

This chapter sketches some of the ways in which evolution here on earth has been transformed over the last three billion years, as background for later consideration of how it might change in the next three billion.

There is room for debate about what really constitutes a change to evolution itself.
I don't intend to engage with that question, but it may be helpful to say something about the different kinds of change which are discussed in this chapter.
Though the simplest account here of the evolution of evolution presents as a linear progression, by contrast with the divergent and multiplying threads in the evolution of life on earth, it is not our presumption that there is only one kind of evolution of importance at any time.

I will touch upon the history of the theory, which is to say, how biologists have talked about evolution, which has a rich history going back into antiquity, but because biologists are naturally focussed on \emph{biological} evolution and in syntheses which add contemporary scientific understanding of evolutionary biology into comprehensive syntheses, my main impetus is to abstract and move away from their complexity and specificity.

My main interest here is to talk, not about the theory, but about the evolution of evolution itself, first to emphasise how much evolution there has been, and secondly to pave the way for speculation about the even more radical changes to evolution which can now be expected.

Some of those changes are so substantial that they would be very likely to impact on how one might define the evolutionary process.
The modern syntheses are closely enough wed to the mechanics of life on earth that pre-biotic evolution, cultural evolution, and the kind of evolution which we may see as the technologies of synthetic biology, artificial intelligence and artificial life are are developed and applied are surely beyond their scope. 
Some of those changes are important to the effectiveness and speed of the process, by transforming or fine tuning what kinds of variation are possible during replication and with what frequencies. 

The main phases in the evolution of life on earth, which are punctuated by notable shifts in what kind of evolution is taking place or how that evolution works are:

\begin{enumerate}
\item  Pre-biotic chemical evolution
\item  Evolution in prokaryotic life
\item  Eukaryotes and sexual reproduction
\item  Intelligence and cultural evolution
\item  Synthetic evolution
\end{enumerate}

Before addressing those major phases in the evolution of evolution, a few words about Darwin's theory.

\section{The Evolution of the Theory}

Though the idea of evolution has a longer history, it is only with Darwin that we first have a well articulated and copiously researched scientific theory, which he presented in his book ``The Origin of Species'' \cite{darwin-oos}. 

Darwin wrote on the evolution of species by natural selection.

The principle aim was to provide convincing evidence that life was not created in its present state by a divine creator, but evolved in a natural analogue of the process of selective breeding.
Talk of ``natural selection'' tells us that the selection which takes place was not accompished by selective breeding either by human or by divine agents.
It risks being read as replacing one kind of metaphysical entity with another, and that the success of evolution might be credited to the prescience of `mother nature'.
What is intended is that there is no selection taking place at all, but that not all are equally capable of or lucky in surviving and procreating in the circumstances they find themselves in.

The phrase ``survival of the fittest'' was also in use, which also has its disadvantages, the principle problem being that it is procreation rather than survival that counts.
`reproductive fitness' is one of the several improvements which were later promoted.

\section{Pre-Biotic Evolution}

Before the evolution of life, there was the `primordial soup', i.e. the ocean's of water on earth with at first quite simple other chemical constituents.
The progressive changes which took place in the composition of that primordial soup would not qualify as ``evolution'' either in Darwin's conception or in any of the later more sophisticated and eloborate syntheses.

In first instance the composition of this soup would alter simply because the rates of formation of more complex molecules from their constituents as they came together, by accident, with appropriate energy (perhaps supplied by lightening or volcanic action), differed from the rates at which these molecules were torn asunder.
These differences would have caused gradual change in the composition of the soup, with more complex molecules becoming more common, and thus beginning or accelerating the formation of even larger molecules.

A variety of mechanisms cause significant divergence in the rates of formation of complex molecules.
One is the existence of catalysts.
Catalysts are molecules which facilitate chemical processes while themselves remaining unchanged.
The formation of catalysts may therefore change the rate at which other complex molecules are formed, and hence accelerate evolution of the primordial soup.

Autocatalytic sets are collections of molecules including catalysts which not only jinly facilitate the formation of some chemical product, but also create all the constituents of the set, hence prefiguring the self-reproductive capability of cellular life.

These natural augmentations of the kinds of simple chemistry first found on earth fall well short of supporting the kinds of complexity found in even the most simple life forms, but the gradually changing differential rates of molecular formation will eventually lead to life and the special kinds of evolution particular to biological ecosystems.
Two important further ingredients are needed.
Firstly, for autocatalysis to be most effective the products of the chemical reactions need to be contained.
If they freely disperse in open oceanic waters, then their effectiveness is muted.
It is possible that small pools might provide environmens in which evolution progressed faster because of the limitations on dissipation which they offer, but ultimately it is the cell boundary which solves this problem.
The second problem is the enormous complexity which is found in the molecules which life forms depend upon, which because of that complexity will appear too infrequently to support life.

The miracle of life depends upon the use of copying mechanisms to reproduce these very complex molecules, but the variety of chemical effects which these molecules serve to effect makes any general copying mechanism infeasible.
Instead, a way of coding up their structure in molecules of regular structure which are readily copied and can be used to guide the construction of the intended chemical provided a solution to the creation of those complex proteins upon which life depends.

Exactly how that machinery evolved remains a mystery, but the enclosure of this machinery together with the \emph{genetic} codes for the proteins for running a cell made self replicating cellular life possible, and facilitated a new kind of evolution.

\section{Evolution in Prokaryotic life}

The simplest and earliest living cells were called prokayotes.
They reproduced by cell division, in that process replicating the genetic codes for the proteins which facilitated all the required chemical reactions for the continuation of the life of the cell.
The correct copying of this genetic information is what makes the cells resulting from the division copies of the original cell, but occasional errors will introduce errors and hence differences between cells, and differences in their ability to replicate.
Those modifications which facilitate replication will yield large polulations of descendents and in this way the forms of life will evolve.

\section{Eukaryotes and Sexual Reproduction}

The evolution of prokaryotic life begins with single celled organisms which are relatively simple.
The most simple such organisms are nevertheless dependent upom an array of complex protients to survive and replicate, and the contruction of those proteins is realised by reference to a coding in DNA of the sequence of amino acids from which they are formed.
When such a cell replicates by division, it is essential that the entire book of codes, the genes which characterise the chemistry of the organism, are faithfully copied into each of the resulting cells, and for this reason the genetic codes are usually all recorded on a single circular DNA molecule.
Errors in that copying prpcess are the principle source of varation which yields the differentials in proliferation on which evolutionary progress depends.

The differential will usually arise through copying errors which substantially reduce the ability of the cell to proliferate, and that genetic variation will appear only briefly in the population.
Rarely an error will prove beneficial and will be retained in growing progeny which may ultimately out-compete all those lacking the innovation.

One consequence of this evolutionary process is that life becomes gradually more complex, and as the complexity mounts, the likelyhood of a mutation proving beneficial decreases.
This is exacerbated by the need for multiple genetic changes to accomplish complex adaptations.

\section{Intelligence and Cultural Evolution}
}%ignore
