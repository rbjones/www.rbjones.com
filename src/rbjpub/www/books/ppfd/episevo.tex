\chapter{Epistemic Evolution}\label{EpistemicEvolution}

The forward looking thinking about epistemes which is the purpose of this book is built upon three main planks.
The first is an appreciation of how epistemes have evolved over the history of planet earth.
The second is the very recent explicit weaponisation of epistemic choice by activists empowered by the philosophy of Michel Foucault.
The third is a speculation about how contemporary and anticipated scientific and technological advances will shape the evolution of epistemes in the near and distant future.

Epistemology is the theory of knowledge and is closely coupled with the concept of rationality, not least through that of epistemic rationality.
In considering what constitutes rationality whether practical or epistemic, we are engaged in epistemology, for even when we are not considering the kind of propositional knowledge which is concerned in epistemic rationality, there is a kind of operational practcal knowledge which is relevant to practical rationality.

It does that from a particular perspective which is motivated in part by my perception of what the dominant features of our future evolution will be, an epistemic perspective.

