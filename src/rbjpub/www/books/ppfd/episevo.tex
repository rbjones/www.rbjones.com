\chapter{Epistemic Evolution}\label{EpistemicEvolution}

Evolution may be considered as an epistemic phenomenon, one which is substantially concerned with discovering and accumulating knowledge.
The knowledge thus accumulated concerns the kinds of structures which proliferate in a variety of environments, and the ways in which they succeed in proliferation.

A tangible sign that evolution is now recognised as accumulating knowledge is the establishment of seed, gene or DNA banks, though these banks capture only some of the ecological diversity which is the treasure trove of evolution.

If evolution discovers knowledge, the way in which it works may be thought of as a precursor for the modern conception of an \emph{episteme}\index{episteme}, and as the workings of evolution are transformed, so we may suppose, the implicit episteme develops with it.

One aspect of the evolution of life on earth has been the progressive incorporation of epistemic methods initially exhibited implicitly by the process of evolution acting in Dawkin's colourful language as ``the blink watchmaker'' into capabilities exhibited by the products of evolution; by creatures which adapt by seeking solutions to everyday problems and executing them without need of evolutionary adaptation.
Ultimately evolution has delivered cognitive awareness, the capability to reflect not only on the problems, but also on the ways in which solutions can be discovered.


\ignore{
The forward looking thinking about epistemes which is the purpose of this book is built upon three main planks.
The first is an appreciation of how epistemes have evolved over the history of planet earth.
The second is the very recent explicit weaponisation of epistemic choice by activists facilitated by the philosophy of Michel Foucault.
The third is a speculation about how contemporary and anticipated scientific and technological advances will shape the evolution of epistemes in the near and distant future.

Epistemology is the theory of knowledge and is closely coupled with the concept of rationality, not least through that of epistemic rationality.
In considering what constitutes rationality whether practical or epistemic, we are engaged in epistemology, for even when we are not considering the kind of propositional knowledge which is concerned in epistemic rationality, there is a kind of operational practical knowledge which is relevant to practical rationality.
}%ignore
