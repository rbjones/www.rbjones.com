\section*{PREFACE}

In 1929 four philosophers wrote a document entitled ``The Scientific Conception of the World: The Vienna Circle'', which subsequently became known as ``The Manifesto'' of the Vienna Circle.
The manifesto promotes a conception of philosophy and science within the historical tradition of \emph{positivism}, but distinguished by the adoption of a new method called \emph{Logical Analysis} exploiting then recent revolutionary advances in logic.

This philosophical point of view, which became known as \emph{logical positivism}, did have a degree of success, and was further progressed during the first half of the twentieth century.
But the kind of logical analysis which it envisaged never became established in mathematics, science or in philosophy.
The reasons for its failure (in its own terms), are complex, and a matter of controversy.
It is the opinion of the present author that there is substantial merit in important parts of this system, that the impedements to their more widespread adoption, though still substantial, are not as great as they were, and that the construction of a positive philosophy for the 21st century might be of interest and value.
This is a sketch of my own ideas on how such a system might look, curtailed in the first instance to the same size as the manifesto of the Vienna Circle.
I follow broadly the structure of the original.

\section{Some History}

The term \emph{positivism} was coined by Auguste Comte, who also used the adjective \emph{positive} to indicate particular disciplines construed on positivistic lines (particularly \emph{positive science}).
Comte considered himself to be advocating and developing ideas which had been introduced previously by the pioneers of modern (as opposed to Aristotelian) science.
Men such as Gallileo and Bacon.
Subsequent commentators have therefore taken positivism as a historical tendency which begins before Comte, and I follow Kolokowski in taking positivism as a philosophical tendency to have begun with David Hume.
The most recent variant of positivism was that of the Vienna Circle, \emph{Logical Positivism}.
The present proposal differs in important respects from logical positivism.
It is a further step in the evolution of positivist ideas.
Rather than invent a new adjective to distinguish this variant of positivism from its predecessors, I shall refer to the philosophical aspects as \emph{positive philosophy}.

That the ideas here touched upon constitute a form of positivism is not beyond dispute.
To see the connection with this historical tendency it will be useful to discuss what might be the essential characteristics of positivism.
First however, note that there are two conceptions of the breadth of positivism, sometimes the narrow notion of positivism is considered (e.g. by J.|S.Mill) \emph{good} positivism, and the rest (which was always a part of Comte's conception of positivism but which was only addressed thoroughly later in his life).
Similarly, the present conception of positive philosophy is of a philosophical system which embraces both theoretical and practical philosophy, but for the time being I will address only the theoretical side.
For this purpose Kolakowski's characterisation is a good starting point.
Kolakowski first describes positivism as:
\begin{quotation}
"a collection of rules and evaluative criteria referring to human knowledge" which tells us what kinds of proposition might count as knowledge of the world and gives norms for what questions are meaningful
\end{quotation}
and then goes on to detail four rules which make those ideas more concrete:

\begin{enumerate}
\item \emph{phenomenalism}

our knowledge of the world is confined to phenomena, and science should confine itself to systematising these phenomena
\item \emph{nominalism}

a radical ontological parsimony
\item \emph{value subjectivism}

value claims do not constitute objective knowledge
\item \emph{the unity of science}

the sciences are united by a common method
\end{enumerate}

We can see however, already in logical positivism, at least insofar as it is exemplified by the philosophy of one of the leading figures in the Vienna Circle, Rudolf Carnap, that positivism while true to Kolakowski's epistemological characterisation, has moved beyond strict compliance with his four rules.
The postivism proposed by this manifesto steps on from that of Rudolf Carnap, and to understand how it nevertheless connects with the positivist tradition it will be helpful to go right back to some of the forerunners of positivism in ancient Greece.

Following Kolakowski I take David Hume as the first positivist, and characterise his positivism as a kind of moderated scepticism.

The scepticisms of Plato's academic and of the Pyrrhoneans may be caricatured as the demand for absolute certainty.
Sceptics thought of themselves as seeking but failing to find knowledge, and the reasons for this failure can for present purposes conveniently be compressed into two.

The first is scepticism about the senses.
The senses cannot be trusted, because our sensory impressions are not logically connected with the external world of which they are our only evidence.
They do not entail anything beyond themselves, the world of which they speak may be entirely illusory.
The sceptic therefore tells us that ``appearances appear'' and nothing else is known of what may have caused them.

If the sceptic appeals here to the logical independence of appearances from anything more concrete, we might thing that he has implicitly aknowledge logical inference as a source of knowledge.
If something was entailed by our sensory impressions then surely we might know it by inference from them.

But the sceptic also has arguments against logical inference, against deductive proof.
I need not go into detail here, for my purpose here is simply to exhibit scepticism as occuring in two stages, first scepticism about the senses, then senses about reason.
Hume's positivism we will see, begins with this distinction.
