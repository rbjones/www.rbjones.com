% $ $Id: theproject.tex$

\chapter{The Project}\label{TheProject}

We have touched upon the grand project of Leibniz in which his universal characteristic
and calculus of reasoning appear.
In this chapter I begin by describing that project in more detail as a preliminary
to putting forward a contemporary descendent of that project.
The path from the former to the latter takes us through some of the history of
intervening idea so as to make intelligible the new as development from the old. 

\section{Leibniz}

Before we look at the particular parts of Leibniz's work of relevance here it will
be of value to have a more general impression of the character of Leibniz's thought,
for which some words of his will be helpful.

\begin{quotation}
first, I was thoroughly self-taught; as soon as I entered into the study of any science,
I immediately sought out something new, frequently before I even completely understood its known, farniliar contents.
Thus I gained in two ways: I did not fill my head with empty assertions (resting on learned authority rather than on actual evidence) which are forgotten sooner or later; furthermore, I did not rest until I had penetrated to the root and fiber of each and every theory and reached the principles themselves from which I might with my own power find out everything I could that was relevant.
\end{quotation}



\section{The Characteristic and Calculus}

In this closer look at Leibniz's project a simple reconstrucion is used to present
it in a form which anticipates the later presentation of a contemporary successor.

We are interested to know:

\begin{itemize}
\item what Liebniz sought to achieve
\item the means by which Leibniz sought his objective
\item the reasons which Leibniz had for believing
\end{itemize}

and to offer an analysis of the whole thus presented.

These materials are not all readily found in Leibniz's writings, so to a significant
extent what I present here is a speculative and simplified reconstruction of certain
aspects of his thinking.

\subsection{What he sought to achieve}

Leibniz tells us that he first conceived of his ideas in these matters as a young man
of 18 years, and that two years later he wrote his first book \emph{Dissertatio de Arte Combinatoria}\cite{leibniz66}
bearing upon the topic.
His principal aim seems to have been to bring to science and philosophy that certainty and
reliability which he found in mathematics.
He felt that such an advancement in scientific and philosophical method would be very greatly
beneficial to humanity.

Of course, there is here in his thinking, no clear separation of means and ends,
he conceived of an exiting possibility as a young man, and tried to realise that
possibility for the rest of his life.

However a clean separation suits my purposes here, providing a simple prototype
for a more complex contemporary project.
It is for these purposes reasonable to characterise Leibniz's project as seeking
to devise or develop:

\begin{enumerate}
\item a notation suitable for expressing precisely the whole of mathematics, science,
and philosophy in their broadest sense, in particular that which Aristotle called
\emph{demonstrative science}, in such a way that
\item the sentences of this language can be given a \emph{decidable} arithmetic encoding,
i.e. an assignment of numbers to terms of the language such that the truth of the
propositions of science can be decided by numerical computation.
\end{enumerate}

We can see here that 

\subsection{How he intended to proceed}

Leibniz had some quite definite ideas about how this could be done, which can be presented
in three parts.

\begin{enumerate}
\item conceptual atomism and the encoding of complex concepts
\item truth as conceptual inclusion
\item the decision procedure for truth
\end{enumerate}

\subsection{Some practical problems}

\subsection{Some philosophical matters}

\section{The Development of Ideas}

\subsection{Hume's Forks}

\subsection{Kants Awakening}

\subsection{Early Logicism}

\subsection{Logical Atoms}

\subsection{Carnap}

\subsection{Turing and AI}

\section{A Contemporary Re-Conception}

