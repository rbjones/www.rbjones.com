% $ $Id: theproject.tex$

\chapter{Following Leibniz}\label{Following Leibniz}

The project around which this book revolves originates with Gottfried Wilhelm Leibniz


\chapter{The Project}\label{TheProject}

It is my aim here devise a philosophical framework suitable to
underpin a particular enterprise, \emph{the project}. 
This might sound like a narrowly scoped philosophy, but ``the
project'' is broad enough to encompass the whole of mathematics,
science and engineering when undertaken by deductively sound methods. 
The project is, in brief, the automation of deductive reason and its
applications, undertaken in a particular manner which I hope to make
clear. 
The relationship between that project and the philosophical ideas
which I offer as a supporting context is complex. 
A full and detailed account of the project depends upon the
philosophy, but the project plays an important role in the
articulation of the philosophy, they are hand-in-glove. 

In this section I make a first attempt at describing the project
trying to minimize reference to the underpinning philosophy. 
In subsequent chapters the philosophical issues are entered into in
earnest, and in this way it is hoped that aspects of the project are
made more clear. 

The scope of the phrase ``automation of deductive reason'' is intended
to include \emph{intelligent} reasoning, but not to encompass much of
the aims or methods of that field known as artificial or machine
intelligence. 

Here and in subsequent sections explanations I use try to make the
ideas clearer partly by talking about their historical evolution. 
In this section the connection with some of the pioneers and
visionaries who have contributed to the formalization and automation
of reason in ways particularly relevant to out project are drawn out. 
The principle figures in this are Gottfried Wilhelm Leibniz and Rudolf
Carnap, as exemplars of the philosophers with similar projects, and
Alan Turing is discussed for a contrast with the kind of automation of
reason I have in mind here. 

\section{Leibniz, Characteristic and Calculus}

Leibniz was a \emph{rationalist}, a philosopher who was particularly
concerned with what can be established by reason rather than
observation. 
The distinction between the two he recognized in the dictum:

\begin{quote}
``There are two kinds of truths, truths of reason and truths of fact.''
\end{quote}

Despite recognizing this distinction, he perceived that all these
truths might be codified formally, so that the distinction between
truths and falsehoods could be determined by computation. 

His vision was \emph{universalistic} in several respects.
He envisaged a \emph{universal
  characteristic}\index{universal characteristic}, which was to be a
formal logical language in which all knowledge might be codified, and
a \emph{calculus ratiocinator}\index{calculus~ratiocinator}, a method
of calculating the truth value of any sentence in his universal
characteristic. 

These two ideas provide the first prototype for this work.

Leibniz did not imagine that the truth of scientific hypotheses could
be established in this way, rather that the corpus of scientific
knowledge could be encoded in the universal characteristic, the
calculus then settling questions in the context of that knowledge. 
In consequence, Leibniz also promoted the collaborative development of
an encyclopedia, of scientific academies and journals. 
Leibniz perhaps also guessed that the computations involved might be
non-trivial, and contributed also to the development of the
information technology of his age, calculating machines. 

Leibniz in these ideas was at least three centuries ahead of his time.
The logic he knew was essentially that of Aristotle, and fell short of
what is needed for the formalization of science by a wide margin. 
The necessary innovations in logic did not appear for another two
hundred years. 
Beyond the adequacy of modern formal languages to express the entire
corpus of scientific knowledge, there lie known limitations in the
extent to which formal deduction can capture truth in these systems. 
Mechanical decision procedures are known not to exist, so for these
reasons we know that no calculus could be relied upon to settle all
well formulated problems in the universal characteristic. 

These limitations are perhaps of only marginal significance for
applications in science and engineering, but even within these
limitations there are serious problems of computational
intractability. 
For many or most problems whose answer is in principle computable, it
will not be accomplished in a reasonable timescale or through the
deployment of available resources. 
Brute computation will therefore not suffice in any but the most
simple cases, and something closer to the workings of human
intelligence would be needed, to come close to an effective
implementation of Leibniz's calculus. 

The technology of calculating machines was even further from sufficiency.
It remains moot whether today's information technology is up to the task.
Beyond the adequacy of the hardware, there are problems with the
software, potentially even less tractable. 

Leibniz's ideas in these matters had little influence in his own time
or for centuries following, not just because they were so far removed
from what was then realizable, but also because they were not even
published. 
His largest impact pressed matters in the opposite direction, and came
from his independent invention of the differential and integral
calculi, and particular for the notation which he devised for this
epoch making mathematical innovation. 
The epoch thus spawned was two centuries of rapid development of
mathematics applicable in science and mathematics. 
These enormous advantages took place despite the \emph{reduction}
consequent on the introduction of infinitesimal quantities. 
Right at the beginning of this period the philosopher Berkeley
criticized these new developments, but it was two hundred years before
mathematicians began to take the problems of rigour seriously and the
rigorization of analysis began, ultimately leading to the new
developments in logic which made it sufficient for Leibniz's project. 

The rigorization of analysis was accomplished first by the
displacement of infinitesimals in a precise notion of limit and by
construction of real numbers from natural numbers, effectively
reducing analysis to arithmetic. 
This done, the next challenge taken up by Kettle's Frege was to show
that arithmetic was simply a matter of logic, requiring no special
metaphysical insights. 
In rising to this challenge Frege, inspired in part by the example of
Leibniz, re-invented logic in the form of his \emph{Begriffsschrift}
or concept notation, which provided the basis not merely for the
logicisation of arithmetic, but also potentially for Leibniz's
universal characteristic. 

Frege's project was more narrowly scoped, but even the formalization
of arithmetic was an arduous undertaking, into which Frege had
invested decades of industry when Bertrand Russell pointed out to him
the inconsistency of his basic principles. 
At about the time when Russell and Whitehead were completing their own
assault on the logicisation of mathematics (a substantial hurdle in
which was ensuring against the kind of inconsistency found in Frege's
system) the young Rudolf Carnap was attending as an undergraduate
Frege's lectures on his new logic and absorbing much of Frege's
attitude to rigour in deductive reasoning and its general
applicability. 

\section{Carnap's Programme}

Carnap was first apprised of modern logic as an undergraduate by
Frege's lectures on his `Begriffsschrift''
\cite{frege1879,heijenoort67}. 
At this stage in his development Carnap clearly showed an interest
both in philosophy and in mathematics and science. 
He was also strongly inclined towards the precision of language which
he found in Frege's logic which he contrasts with the teaching of
logic in philosophy. 
Within the sciences, he preferred physics because of its greater
conceptual clarity. 
He also perceived the distinction between ethical, evaluative, emotive
and metaphysical language and scientific doctrine. 

From this position as an undergraduate just before the great war he
moves forward in a period of about seven years to the point at which
we can see the formulation of the central ideas which motivated his
work through the rest of his life. 
The most significant new influence in forming these ideas came from
Bertrand Russell. 
Carnap became acquainted with \emph{Principia
  Mathematica}\cite{russell10} and began to prefer its notation to
that of Frege. 
He also begins to feel that a concept is not clearly understood until
he can see how to express it in symbolic language. 

At the end of 1921 Carnap read Russell's \emph{Our Knowledge of the
  External World}\cite{russell1921} and was inspired by Russell's
characterization of ``logico-analytic method'' in philosophy. 
It is at this point that Carnap self-consciously devotes himself to
this philosophical method and begins intensive reading of Russell's
writings on the theory of knowledge and the methodology of science. 

Russell, in his work on \emph{Principia Mathematica} had undertaken
with Whitehead a task similar in character to Frege's project. 
However, Frege's focus had been on the logicisation of mathematics.
Russell had a broader conception of the applicability of the methods,
and advocated a kind of philosophy in which such methods were used
exclusively. 
Carnap took up this challenge enthusiastically.

Carnap's philosophical programme therefore involved first the idea
that philosophy in general should be \emph{analytic} in the specific
sense that its methods and results are logical, and should be obtained
by the new logical methods pioneered by Frege and Russell. 
Secondly Carnap conceived of the role of the philosopher as being
primarily concerned with facilitating the progress of empirical
science. 
The kind of facilitation he had in mind was analogous to the
innovations in mathematics undertaken by Frege and Russell, the
establishment of new languages in which scientific laws could be
precisely enunciated and the surrounding theoretical and
methodological framework for the formalization of science. 

The work of Frege and Russell had been \emph{universalistic} in the
sense that they sought a single new language in which the whole of
mathematics could be developed. 
At this stage Carnap's thinking was along similar lines, but since he
was trying to carry forward the ideas into empirical science, it did
not seem feasible to stick with the same purely logical systems. 
Carnap's aims at this point may therefore be thought similar to
Leibniz's desire for a \emph{universal} characteristic in which all of
science could be formalized. 
Later Carnap became more pluralistic, but the idea of using formal
logic to encode scientific knowledge and of formalizing deductive
reasoning about science places his enterprise within the scope of
Leibniz's. 

Like Leibniz, Carnap's ideas were ahead of their time.
He was working in the context of modern symbolic logic, which is (I
shall argue), no longer shackled by the limitations of Aristotelian
syllogistic. 
But strict formality in language and proof is arduous, and good
mechanical support a prerequisite to widespread adoption of such
methods. 
\emph{Principia Mathematica} formalized a significant part of
mathematics, and was very influential during the formative years of
the new academic discipline of \emph{mathematical logic}. 
These new developments did have a significant impact on Mathematics,
which was made more rigorous through the systematic use of axiomatic
set theory. 
But mathematicians did not follow the example of Russell and Whitehead
by taking up strictly formal notations or formal proofs. 
Not even in mathematical logic did this happen, for mathematical logic
became a meta-theoretic discipline in which formal systems were
studied rather than applied. 
Proofs in mathematical logic may often be concerned with formal
systems, but they are not themselves formal. 

Not only was formalization rejected by mathematics and science, it was
soon to be rejected, together with Carnap's entire philosophical
outlook, by philosophy. 
The first major assault on Carnap's position was by Quine, who had
been sniping at some of the core doctrines, notably the concept of
analyticity, ever since his first exposure to Carnap's philosophy as a
Harvard postgraduate visiting Carnap in Vienna. 
Quine's critique modulated into outright repudiation of the central
tenets of Carnap's philosophy in the his ``Two Dogmas of
Empiricism''. 
It was not long before an attack came on another flank from Saul
Kripke, who dismantled the identification of analyticity and necessity
which were inter-definable in Carnap's conceptual scheme and thereby
invented a new kind of metaphysics. 

The idea of Russell and Carnap, that philosophy should consist of
logical analysis and of Carnap that it should also be conducted
formally both disappeared. 
Symbolic or mathematical logic does play a substantial role in
analytic philosophy, but it is the meta-theoretic techniques of
mathematical logic which are used in philosophy. 
There is no general adoption of formality as a way of achieving rigour
in philosophy, no general recognition that there is a deficit in
rigour might requires such a remedy. 

Another impediment to the realization of Leibniz's project was, during
Carnap's lifetime, beginning to be dissolved. 
The digital electronic computer was invented and the first steps
toward using this technology for the automation of deduction were in
progress. 
The torch was about to pass from philosophy, not to mathematical
logic, but to Computer Science. 

\section{Information Technology}

These two philosophers strove to realize ambitions which were soon to
be made more feasible by the invention of the stored program digital
computer, and with it another academic discipline, Computer Science,
which over the second half of the 20th century would conduct an
enormous amount of research relevant to both of their projects.
The first home of logic, Philosophy, and the new discipline of
mathematical logic, treated logic in a meta-theoretically.
They took formal logic primarily as something to be studied rather
than directly used (by contrast with the earlier work by Frege and
Russell and Whitehead in which large parts of mathematics were
formally derived).
Computer Science, as well as conducting theoretical and
meta-theoretical investigations relevant to computing, found reason to
undertake proofs formally, with the assistance of their computing
machinery.

For these reasons, the primary academic locus of research relevant to the
ambitions of Leibniz and Carnap moved from Philosophy to Computer
Science.
However, the idea of using computers for science was more
predominantly pursued by purely computational rather than formally
logical methods.
I hope to make this distinction clearer, to make the case that
logical methods deserve to be given special consideration, to give a
philosophical context in which it make sense to do that on a
substantial scale, and to consider some aspects of how this might be progressed.

\subsection{Alan Turing}

Alan Turing was a mathematical logician, computational engineer and a
visionary thinker whose ideas on artificial intelligence have been
influential on research in the automation of reasoning.

For our present purposes it is primarily his writing on artificial
intelligence which is of interest, as encompassing objectives even
broader than our own, but Turing was an important figure in a
milestone in our understanding of computation which was reached during
the 1930s.

One of the tenets of the philosopher and mathematician David Hilbert
in the early part of the 20th century was that all definite
mathematical problems are susceptible of solution.
This thesis was held to be equivalent to the effective decidability of
validity in first order predicate logic, which was called the \emph{entscheidungsproblem}.
To make the problem precise enough to be given a definitive
mathematical answer it was necessary to make precise the notion of effective calculability.
This resulted in several different logicians putting forward different
notations in which arbitrary algorithms (methods of solution) could be
expressed.
Church came up with concise notation for functional abstraction known
as ``the lambda calculus'', Kleene with a system for defining
numerical functions by recursion, Emil Post came up with a system
based on transforming strings using a set of ``productions'' which
defined transformations on the strings.
All these systems provided ways of defining effectively computable
functions over the natural numbers, but the one which looked most like
a description of a computing machine was that invented by Alan Turing,
since called the Turing machine \cite{turingOCN}.

These different ways of describing effective computational processes
all turned out to be similarly expressive, and this remarkable
discovery underpinned the idea that they did indeed all capture the
notion of effective calculability.
The \emph{entscheidungsproblem} having by these means been made more
definite, Church and Turing independently solved the problem by
exhibiting problems which were provably unsolvable (by any effective procedure).

This result limits what could possibly be achieved by Leibniz's
\emph{calculus ratiocinator}, as did G\"odel's previous result on the
incompleteness of arithmetic.
It seems from Leibniz's description that he probably did imagine that
any definite question within the scope of the scientific knowledge
codified in the \emph{lingua characteristica} would be answerable
using his calculus and a sufficiently rapid calculator.
The G\"odel and Church-Turing results show that this cannot be the case.
I will look a little closer later at these and other limiting factors
and consider their significance for ``the project''.

Turing also wrote on Artificial Intelligence, and his conception of
Artificial Intelligence provides an alternative ideal for the
automation of reason which is not predicated on machines achieving the
impossible.
His seminal paper in this area was \emph{Computing Machinery and Intelligence}\cite{turingCMI}.
Turing's essential idea was that human intelligence, though very
differently implemented, is not fundamentally distinct from or more
capable than the kind of information processing which could be
undertaken by a sufficiently complex and powerful Turing machine.
In addressing the question whether a machine can think, Turing
described an ``imitation game'' to make this question more precise,
this later became known as the Turing test.
The Turing test involves human beings interacting with a person and a
machine through similar interfaces which conceals which of the two was
involved, i.e. using a keyboard to converse with someone or something
in a distant or private place.
If the observer cannot tell the difference between the man and the
machine when he interacts with it in this way, then perhaps we should
be ready to acknowledge that the machine thinks, and hence exhibits
some important aspects of human intelligence.

\subsection{The Science of Computing}

During the first half of the 20th century the technology of
computation had  moved rapidly.
The Turing machine provided a very simple abstract prototype for an
important transition, from special purpose to general purpose
computational engines.
It anticipated the general purpose stored program digital computer,
and these relatively complex calculating machines were just becoming
technically feasible.
They first appeared in the 1940s using technologies such as
electromagnetic relays, and the electronic vacuum tube.
Soon the transistor emerged, and the technology of computation began a
long period of progressive miniaturization yielding ever smaller and
faster building blocks for digital computers.

With the invention of the digital computer came the new academic
discipline of Computer Science, and enthusiasm for formal systems and
the automation of reason passed from Philosophy and Mathematical Logic
(itself just an infant) to Computer Science.
Computer scientists were interested in formal languages because they
allowed algorithms to be described precisely for execution by a computer.
The complexity of these algorithms rapidly increased and correctness
became an issue.
How could one be sure that the instructions for achieving some
computational objective would realize their intended aim?
Since formally defined algorithms could be construed as mathematical
entities, it seems that certainty might best be realized by a
mathematical (i.e. a deductive) proof.
Unfortunately these proofs turned out themselves to be even more
complex than the programs whose correctness they were intended to
show, and the questions then arose how one could facilitate the
construction of these proofs and how one could be sure that the proofs
were correct?
The computer itself provided one answer to these questions.
The computer could assist in the construction of the proofs, and since
it is in the nature of formal proofs that their correctness can be
checked mechanically, they could also check the proofs.
In this way, that part of Computer Science which was particularly
interested in ensuring that computer programs are correct was drawn
into the the use of formal notations, formal deductive systems and the
automation of reason.

Meanwhile another branch of Computer Science developed which sought to
program computers so that instead of merely doing large scale
drudgery, they exhibited some kind of intelligence.
This was the field of Artificial Intelligence, and later the related
interdisciplinary enterprise of Cognitive Science.
Within these disciplines a variety of approaches to the problem were
pursued, but despite this variety a single important cleavage was
captured by the contrast between ``neats'' and ``scruffs''.
The project around which this book is constructed may be thought of
leading towards an extremely ``neat'' AI, so this distinction can be
helpful in initial sketches of the project.

The distinction between neats and scruffs may be seen through the
contrast between ``connectionist'' AI and methods based on theorem proving.
The connectionist approach to AI approximates the human brain by
simulation of a neural network using highly parallel processing.
The idea is to devise a network which is capable of learning and then
to teach it the required skills.
If such a system achieved a competence in a complex deductive science
it would do so indirectly.
Such competence would be an application of its general intelligence,
not a source or means of attaining that intelligence.

``neats'', on the other hand, think more like Leibniz.
For them the intelligent capabilities are build on a foundation which
includes a formal language for representing propositional knowledge
and a deductive inferential capability.
This does not necessarily mean that they are aimed at different
problem domains to the ``scruffs''.
Neats may seek systems capable of ordinary language discourse and
common sense reasoning, but still approach this on logical foundations
which would be unfamiliar to most people.

The neat/scruffy distinction is concerned with means rather than ends.
There is a related distinction which concerns ends rather than means.
This is the distinction between taking the aim of a research programme
to be the replication of some aspect (or the whole of) human
intelligence, of whether some other characterization, independent of
the methods or competence of human beings, is give of the objective of
the research.

\subsection{Automation and Intelligence}\label{AutomationAndIntelligence}

We arrive then at my first characterization of the project which this
philosophical dissertation seeks to underpin.

The aim of the project is a comprehensive formalization of the
deductive aspects of philosophy, mathematics, empirical science and
engineering design.

\subsection{Some Distinctions}

Developments since Leibniz and Carnap give us better insights into
the extent to which their projects are realizable.
The diversity of related research which has been undertaken enables us
to discriminate many different similar kinds of project and to
identify more precisely the kind of projects which Leibniz and Carnap
might have entertained if they were working today.


\section{The Philosophy}\label{ThePhilosophy}

Here are the bare bones of a philosophical framework to underpin ``the project''.

The philosophy is positivistic, insofar as it delineates a particular
approach to empirical science and its application, which it is
suggested may in some ways be preferable to extant methods.
In this I am not being prescriptive, I do not assert that science
\emph{should} be conducted in this way.

The framework is, in the first instance, conceptual.
This consists to a large extent in the choice of particular meanings
for terms which have a long history of shifting usage.
Some of the most fundamental of these come in pairs as dichotomies,
i.e. disjoint concepts which together exhaust some significant larger
domain.
The most fundamental of these is the distinction between logical and
empirical truths.
Simple though it might appear to be, it is a source of endless
philosophical controversy, and it will therefore be necessary to
consider in detail the evolution of this distinction over its entire
history and come to as precise an understanding of this fundamental
cleavage as possible.

Such fundamental concepts are not mere accidents of language, and are
one element of the philosophical framework which we may think
constitute a kind of \emph{metaphysics}.
\footnote{On this my disagreement with Carnap lies only in the scope
  of applicability of the term \emph{metaphysics}.
It does not constitute metaphysics under either of the two principle
categories which constitute metaphysics for Carnap.
I take great care to make the meaning of these concepts definite and
clear, so meaninglessness, I hope, is avoided.
Nor does this kind of metaphysics constitute, in our framework,
\emph{a priori} justifications about \emph{synthetic} claims.
}%footnote
