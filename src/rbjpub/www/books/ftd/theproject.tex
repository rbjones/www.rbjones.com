% $ $Id: theproject.tex$

\chapter{The Project}\label{TheProject}
%\chapter{Leibniz, Characteristic and Calculus}\label{Leibniz}

We have touched upon the grand project of Leibniz in which his universal characteristic
and calculus of reasoning appear.
In this chapter I begin by describing that project in more detail as a preliminary
to putting forward a contemporary descendent of that project.
The path from the former to the latter takes us through some of the history of
intervening idea so as to make intelligible the new as development from the old.

To gain an impression of the scale and character of Leibniz's ideas it is useful to connect them with
the great philosophical system of Aristotle on which they build and to which they respond.

\section{The Philosophy of Aristotle}

In the time of Aristotle ``philosophy'', meaning ``love of knowledge'' encompassed science as
it then was, which constituted the main body of Aristotle's work said to number around 200 volumes
of which but 31 have survived.

Aristotle divided the sciences into three kinds.
\emph{theoretical} science being knowledge pursued for 
its own sake, \emph{practical} science being oriented to action and including ethics and politics, and
the productive sciences, which were arts (including peformance arts) and crafts.
Not included in this classification were the six volumes which later became known as ``the Organon'', Aristotle's
works on Logic the central purpose and achievement was to examine reason and its place in a conception of
science which Arisotle dubbed ``demonstrative science''.
The volumes on logic were regarded by Aristotle as not belonging to science, but as providing
methods for the conduct of science.

This monumental achievement was a dominating force force for nearly two thousand years
subject only to emendation and elaboration by the medieval scholastics, until the renaissance
stimulated the development of modern science, superceding that of Aristotle, in the richness
of its results, but also importantly replacing Aristotle's method with new ideas more heavily
emphasising observation and experiment advocated and practiced by men such as Copernicus, Bacon,
and Gallileo.

In that renaissance culture open to innovation and change, Leibniz absorbed Aristotle, eager to
innovate in fundamental and revolutionary ways.

\section{Leibniz}

Before we look at the particular parts of Leibniz's work of relevance here it will
be of value to have a more general impression of the character of Leibniz's thought,
for which some words of his will be helpful.

\begin{quotation}
``first, I was thoroughly self-taught; as soon as I entered into the study of any science,
I immediately sought out something new, frequently before I even completely understood its known, farniliar contents.
Thus I gained in two ways: I did not fill my head with empty assertions (resting on learned authority rather than on actual evidence) which are forgotten sooner or later; furthermore, I did not rest until I had penetrated to the root and fiber of each and every theory and reached the principles themselves from which I might with my own power find out everything I could that was relevant.''
\end{quotation}

His regard for Aristotle's logic was high:
\begin{quotation}
I hold the invention of the syllogistic form to be one of the most beautiful
inventions of the human mind, and indeed one of the most notable. It is a
sort of universal mathematics, the importance of which is too little
known. One can say that a type of infallibility is contained in it, provided
that one knows how to use it well and has the opportunity to do so,
something that is not always the case.
\end{quotation}

Leibniz tells us here that he studied with the constant purpose of moving knowledge forward, the desire
improve and complete, not only in modest ways but in the most radical ways he can conceive.
Leibniz studies scholastic presentation of the logic of Aristotle from the age of 12 years.
At 14 he is writing critiques and formulating projects for reform of logic, and at that
time has ideas which are the seeds of a revolutionary project which occupied him for the rest of his life.

\begin{quotation}
``While still a boy, possessing only the
rudiments of common logic and ignorant of mathematics, the plan arose in me -- by what inspiration I do
not know -- that an analysis of concepts could be devised from which truths could be extracted through
certain combinations and evaluated like numbers. It is pleasant even now to recall by what arguments,
however youthful, I came to the idea of so great a thing.''
\end{quotation}

He has these ideas before he becomes well acquainted with the mathematics of his day,
which familiarity comes only during his period in Paris (from .
Later he talks of the system as improving and bringing to wider application \emph{the method of mathematics},
and uses the term \emph{universal mathematics} for his conception of formal logic.

\section{The Characteristic and Calculus}

In this closer look at Leibniz's project a simple reconstrucion is used to present
it in a form which anticipates the later presentation of a contemporary successor.

We are interested to know:

\begin{itemize}
\item what Liebniz sought to achieve
\item the means by which Leibniz sought his objective
\item the reasons which Leibniz had for believing that the means would realise his ends.
\end{itemize}

and to offer an analysis of the whole thus presented.

These materials are not all readily found in Leibniz's writings, so to a significant
extent what I present here is a speculative and simplified reconstruction of certain
aspects of his thinking.

\subsection{What he sought to achieve}

Leibniz tells us that he first conceived of his ideas in these matters as a young man
of 18 years, and that two years later he wrote his first book \emph{Dissertatio de Arte Combinatoria}\cite{leibniz66}
bearing upon the topic.
His principal aim seems to have been to bring to science and philosophy that certainty and
reliability which he found in mathematics.
He felt that such an advancement in scientific and philosophical method would be very greatly
beneficial to humanity.

Of course, there is here in his thinking, no clear separation of means and ends,
he conceived of an existing possibility as a young man, and tried to realise that
possibility for the rest of his life.

However a clean separation suits my purposes here, providing a simple prototype
for a more complex contemporary project.
It is for these purposes reasonable to characterise Leibniz's project as seeking
to devise or develop:

\begin{enumerate}
\item a universal language suitable for international scientific communications and for
compiling an encyclopaedia encompassing the whole of mathematical,
metaphysical and scientific knowledge.
\item a comprehensive encyclopaedia compiled in the universal language
\item a notation suitable for expressing precisely the whole of mathematics, science,
and philosophy in their broadest sense, in particular that which Aristotle called
\emph{demonstrative science}, in such a way that
\item the sentences of this language can be given a \emph{decidable} arithmetic encoding,
i.e. an assignment of numbers to terms of the language such that the truth of the
propositions of science can be decided by numerical computation.
\end{enumerate}

\subsection{How he intended to proceed}

At the core of Leibniz's project was his belief that the truth of categorical
propositions (whether necessary or contingent) could be decided arithmetically.
But he never settled finally on a scheme for arithmetic encoding and a corresponding
decision algorithm, he continued to develop these ideas throughout his life.

I will present here what seems to be his closest approach to a method meeting his
desiderata, which involves the following elements.

\begin{enumerate}
\item conceptual atomism and the encoding of complex concepts
\item truth as conceptual inclusion
\item the decision procedure for truth
\end{enumerate}

His \emph{conceptual atomism} was the view that all concepts are definable in terms
of a modest number of primitive concepts by limited means.
In the most sophisticated, a complex predicate is defined as a conjunction of primitive
concepts and negations of primitive concepts.
We are thus considering scientific language as if all scientific concepts were like a taxonomy.

Concepts defined in this way can be represented arithmetically by using distinct prime numbers for
various primitive concepts, and representing the conjunction of primitive concepts by multiplication.
The complements of primitive concepts are accomodated in this scheme by coding a complex predicate
using two numbers, the first of which represents uncomplemented primitive concepts as the product of
their representing prime numbers, and the second of which represents the primitive concepts whose
complements appear in a similar manner, as the product of the corresponding (uncomplemented) primitive
concepts, negating the whole.

Thus, if we have three primitive concepts A , B and C, represented respectively by
the first three primes, then the following codings exemplify  the coding method.

\begin{itemize}
\item[A and B] is represented by 2*3 = 6, -0
\item[A and not B] is represented by 2, -3
\item[A and not B and C] by 2*5= 10, -3
\end{itemize}

Once represented in this way the truth value of categorical
propositions can be calculated in the following way.

For the proposition ``A is predicated of all B'' to be true every primitive concept
positively occuring in A, must also occur positively in B, and similarly for negative
occurrences.
This will be the case if the two numbers representing A each divide exactly into the two
representing B.

\section{Critique}

\section{Subsequent Developments}

Of these the two most important are the emergence of modern symbolic logic and the discipline of
mathematical logic, and the development of electronic digital information storage and processing
systems.

The impact ot the latter is significant for the \emph{representation} of propositions, which
in Leibniz's idea was required to effectively encode fully expanded definitions of all the
concepts involved, so that a calculator, without access to an distinct body of propositions
might decide the truth of the single proposition submitted to it.


Russell, in his analysis of the philosophy of Leibniz observes that ``all sound philosophy should begin with an analysis of propositions''.
Without necessarily agreeing with Russell on this point, that is where I propose to begin.
It is probably more correct to call what is offered here a theory of propositions than of knowledge,
insofar as we are concerned with the representation of propositions, the organisation of large bodies of
propositions, and the things which one might expect to do with such a body of propositions, notably
its extension by definitional or inferential means.
This is a contemporary alternative to the Leibniz's universal language, characteristic and calculus,
which can (and should be) very different to Leibniz's conceptions because of the various important
developments which have taken place since his time.

\chapter{A Theory of Knowledge}\label{TheoryOfKnowledge}

This chapter provides an outline of a presents a contemporary reworking of some of leibniz's central
ideas, particularly the formalisation of knowledge and the automation of deductive reasoning.

This reworking is presented as a pluralistic \emph{framework} for knowledge representation and its
deductive elaboration.

The context required for an undestanding of Leibniz's ideas in this are is his
understanding of logic, particularly of language and how it could be made formal
to facilitate the autmation of deduction, both for the purposes of establishing
the truth, and to facilitate innovation.
Intimately involved with this is his metaphysic.

The presentation here is of a way of thinking about language, particularly \emph{propositional} knowledge,
and of its  

\section{The Evolution of Language}

The ideas presented here may be thought of in many ways.
One of these is as a further evolution of the concept of \emph{language}, just as
Leibniz, with his characteristic and calculus was proposing not just a new language
(as his universal language would have been), but, in his characteristic, and entirely
new {\it kind} of language.
It will be helpful in understanding the kind of innovation proposed, to first review
some stages in the development and evolution of language, exhibiting a trend which
we now propose to extend.
Along the way I will be establishing some terminology


Social behaviour pervades the animal kingdom from its very beginnings, language is
a part of that behaviour.

