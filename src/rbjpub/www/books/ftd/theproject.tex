\def\rbjidtheproject{$$Id: theproject.tex,v 1.1 2011/10/25 09:10:45 rbj Exp $$}

\section{The Project}\label{TheProject}

It is my aim here devise a philosophical framework (or \emph{weltanschaung}) suitable to underpin a particular enterprise, \emph{the project}.
To give a sense of what \emph{kind} of project I have in mind I first mention here two distinguished previous philosophers who have placed similar projects at the center of their enterprise.
I will then connect these ideas with more recent developments which belong to Computer Science


The two are Gottfried Wilhem Leibniz and Rudolf Carnap, and in this chapter I say just enough about their work to give a first understanding of how my own enterprise relates to it.

\subsection{Leibniz, Characteristic and Calculus}

Leibniz was a \emph{rationalist}, a philosopher who was particularly concerned with what can be established by reason rather than observation.
The distinction between the two he recognised in the dictum:

\begin{quote}
``There are two kinds of truths, truths of reason and truths of fact.''
\end{quote}

Despite recognising this distinction, he perceived that all these truths might be codified formally, so that the distinction between truths and falsehoods could be determined by computation.

His vision was \emph{universalistic} in several respects.
He envisaged a \emph{universal characteristic}\index{universal!characteristic}, which was to be a formal logical language in which all knowledge might be codified, and a \emph{calculus ratiocinator}\index{calculus!ratiocinator}, a method of calculating the truth value of any sentence in his universal characteristic.

These two ideas provide the first prototype for this work.

Leibniz did not imagine that the truth of scientific hypotheses could be established in this way, rather that the corpus of scientific knowledge could be encoded in the universal characteristic, the calculus then settling questions in the context of that knowledge.
In consequence, Leibniz also promoted the collaborative development of an encyclopaedia, of scientific academies and journals.
Leibniz perhaps also guessed that the computations involved might be non-trivial, and contributed also to the development of the information technology of his age, calculating machines.

Leibniz in these ideas was at least three centuries ahead of his time.
The logic he knew was essentially that of Aristotle, and fell short of what is needed for the formalisation of science by a wide margin.
The necessary innovations in logic did not appear for another two hundred years.
Beyond the adequacy of modern formal languages to express the entire corpus of scientific knowledge, there lie known limitations in the extent to which formal deduction can capture truth in these systems.
Mechanical decision procedures are known not to exist, so for these reasons we know that no calculus could be relied upon to settle all well formulated problems in the universal characteristic.

These limitations are perhaps of only marginal significance for applications in science and engineering, but even within these limitations there are serious problems of computational intractability.
For many or most problems whose answer is in principle computable, it will not be accomplished in a reasonable timescale or through the deployment of available resources.
Brute computation will therefore not suffice in any but the most simple cases, and something closer to the workings of human intelligence would be needed, to come close to an effective implementation of Leibniz's calculus.

The technology of calculating machines was even further from sufficiency.
It remains moot whether today's information technology is up to the task.
Beyond the adequacy of the hardware, there are problems with the software.

\subsection{Carnap's Programme}

