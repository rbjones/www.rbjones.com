% $ $Id: theproject.tex$

\chapter{The Project}\label{TheProject}
%\chapter{Leibniz, Characteristic and Calculus}\label{Leibniz}

We have touched upon the grand project of Leibniz in which his universal characteristic
and calculus of reasoning appear.
In this chapter I begin by describing that project in more detail as a preliminary
to putting forward a contemporary descendent of that project.
The path from the former to the latter takes us through some of the history of
intervening ideas so as to make intelligible the new as developed from the old.

To gain an impression of the scale and character of Leibniz's ideas it is useful to connect them with
the great philosophical system of Aristotle on which they build and to which they respond.

\section{The Philosophy of Aristotle}

In the time of Aristotle ``philosophy'', meaning ``love of knowledge'' encompassed science as it then was, which constituted the main body of Aristotle's work, said to number around 200 volumes of which but 31 have survived.

Aristotle divided the sciences into three kinds.
\emph{theoretical} science being knowledge pursued for 
its own sake, \emph{practical} science being oriented to action and including ethics and politics, and
the productive sciences, which were arts (including peformance arts) and crafts.
Not included in this classification were the six volumes which later became known as ``the Organon'', Aristotle's
works on Logic.
The central purpose and achievement of The Organon was to examine reason and its place in a conception of science which came to be known as ``demonstrative science''.
The volumes on logic were regarded by Aristotle as not belonging to science, but as providing methods for the conduct of science, hence the name ``Organon'' which means tool or organ.

As has often happened since, particularly in relation to advances in logical rigour, the adoption by Aristotle of these ideas in his own scientific writing was not conspicuous, for in detail they were not up to the job.
The advances in logic which would have been necessary to capture the kind of deductive reasoning found in mathematics and science would not be possible until considerable further advances in mathematics.

Aristotle's achievement in the Organon was nevertheless a dominating force for nearly two thousand years, subject only to qualification and elaboration by medieval scholastics.
The renaissance saw the beginnings of modern science, superceding that of Aristotle in the richness of its results, but also importantly, replacing Aristotle's method with new ideas more heavily emphasising observation and experiment, advocated and practiced by men such as Copernicus, Bacon, and Gallileo.
On the formal logic at the core of demonstrative science, the work of Aristotle remained definitive until ``mathematical logic'' was developed in the nineteenth century.
Though dominant, Aristotle's logic was not readily applicable, and its dominance was academic.

In that renaissance culture open to innovation and change, Leibniz absorbed Aristotle, eager to innovate in fundamental and revolutionary ways.

\section{Leibniz}

Before we look at the particular parts of Leibniz's work of relevance here it will be of value to have a more general impression of the character of Leibniz's thought,
for which some words of his will be helpful:

\begin{quotation}
{\it first, I was thoroughly self-taught; as soon as I entered into the study of any science,
I immediately sought out something new, frequently before I even completely understood its known, farniliar contents.
Thus I gained in two ways: I did not fill my head with empty assertions (resting on learned authority rather than on actual evidence) which are forgotten sooner or later; furthermore, I did not rest until I had penetrated to the root and fiber of each and every theory and reached the principles themselves from which I might with my own power find out everything I could that was relevant.}
\end{quotation}

His regard for Aristotle's logic was high:
\begin{quotation}
{\it I hold the invention of the syllogistic form to be one of the most beautiful
inventions of the human mind, and indeed one of the most notable. It is a
sort of universal mathematics, the importance of which is too little
known. One can say that a type of infallibility is contained in it, provided
that one knows how to use it well and has the opportunity to do so,
something that is not always the case.}
\end{quotation}

Leibniz tells us here that he studied with the constant purpose of moving knowledge forward, the desire to improve and complete, not only in modest ways but in the most radical ways he can conceive.
Leibniz studied scholastic presentations of the logic of Aristotle from the age of 12 years.
At 14 he is writing critiques and formulating projects for reform of logic, and at that time has ideas which are the seeds of a revolutionary project which occupied him for the rest of his life.

\begin{quotation}
  {\it
    While still a boy, possessing only the
  rudiments of common logic and ignorant of mathematics, the plan arose in me -- by what inspiration I do not know -- that an analysis of concepts could be devised from which truths could be extracted through certain combinations and evaluated like numbers.
  It is pleasant even now to recall by what arguments, however youthful, I came to the idea of so great a thing.
  }%it
\end{quotation}

He had these ideas before he became well acquainted with the mathematics of his day, which familiarity comes only during his period in Paris (1672-76).
Later he talks of the system as improving and bringing to wider application \emph{the method of mathematics}, and uses the term \emph{universal mathematics} for his conception of formal logic.

\subsection{The Characteristic and Calculus}

In this closer look at Leibniz's project a simple reconstrucion is used to present it in a form which anticipates the later presentation of a contemporary successor.

We are interested to know:

\begin{itemize}
\item what Liebniz sought to achieve
\item the means by which Leibniz sought his objective
\item the reasons which Leibniz had for believing that the means would realise his ends.
\end{itemize}

and to offer an analysis of the whole thus presented.

These materials are not all readily found in Leibniz's writings, so to a significant extent what I present here is a speculative and simplified reconstruction of certain aspects of his thinking.

\subsection{What he sought to achieve}

Leibniz tells us that he first conceived of his ideas in these matters as a young man of 18 years, and that two years later he wrote his first book \emph{Dissertatio de Arte Combinatoria}\cite{leibniz66}
bearing upon the topic.
His principal aim seems to have been to bring to science and philosophy that certainty and reliability which he found in mathematics.
He felt that such an advancement in scientific and philosophical method would be very greatly beneficial to humanity.

There is here, in his thinking, no clear separation of means and ends, he conceived of an existing possibility as a young man, and tried to realise that possibility for the rest of his life.

A clean separation suits my purposes here, providing a simple prototype for a more complex contemporary project.
It is for these purposes reasonable to characterise Leibniz's project as seeking to devise or develop:

\begin{enumerate}
\item a universal language suitable for international scientific communications and for compiling an encyclopaedia encompassing the whole of mathematical, metaphysical and scientific knowledge.
\item a comprehensive encyclopaedia compiled in the universal language
\item a notation suitable for expressing precisely the whole of mathematics, science, and philosophy in their broadest sense, in particular that which Aristotle called
\emph{demonstrative science}.
This he called \emph{lingua characteristica} or the {\it universal characteristic}.
\item a method of mechanically calculating the the truth value (its truth or falsity) of any proposition properly expressed in the \emph{lingua characteristica}.
  This he called the \emph{calculus ratiocinator}
  \item calculating machines capable of effecting the \emph{calculus ratiocinator}
\end{enumerate}

There are two broad areas of activity here which can be described as rigorous science and formal science.
Leibniz's interest centres around new languages, in the first case suitable for use directly by scientists and suitable for the publication and dissemination of the results of scientific research, in the second the `language' is his \emph{lingua characteristica}, an arithmetic encoding suitable for mechanised computation of the consequences of the established scientific theories.

Our present concern is with the latter, in a context in which computational capabilities are better established, but the limits of what can be achieved by mechanisation remain uncertain.

Leibniz's ideas about the formalisation and mechanisation of demonstrative science were well ahead of his time. He made many contributions to the science of his day, facilitating the establishment of scientific institutions and journals. He also developed the differential and integral calculus, independently of Newton but with notational innovations that made the calculus easier to use. However, his ideas about formalising and mechanising science could not be realised until much later, when fundamental advances in logic and computation made them more approachable. Despite very great advances they have not yet produced the fruits that Leibniz envisioned.

In a later chapter we will look more closely at the details of Leibniz's proposal to give a more finely grained account of the history.
Here I will first sketch the main problems which prevented his vision from being practicable in his own time, and then trace the subsequent developments which have removed or mitigated those problems.
Then I offer a contemporary recasting of his project which the rest of the book will seek to progress.

\subsection{Critique}

One might suppose that as a rationalist Leibniz founded his project on in the belief that all knowledge could be known by reason and that it would therefore inevitably fail when confronted with deciding factual matters without benefit of observational data.

Leibniz's rationalism was however not nearly so extreme, and the project is predicated on scientists compiling an encyclopaedia of scientific knowledge which would then be encoded into his \emph{lingua characterisatica} before the decision procedure for truth could be mechanised by a machine executing the \emph{calculus ratiocinator}.

Leibniz's calculus was not a way of discovering scientific laws, but rather, a way of elaborating their consequences.
Aristotle's conception of demonstrative science involved the establishment of the principles of each science by means other than the syllogistic, the demonstrative claims would then be logically derivable from them, and the notion of \emph{demonstrative truth} was applicable only to the conclusions drawn from the first principles.
Leibniz's calculus accomplished the latter task, but did not contribute to the former.
His `lingua chracteristica' would need to encode the first principles of all the sciences.

His fundamental insight was that Aristotelian Syllogistic logic was mechanisable, and that the consequences of received science thus mechanically verifiable.

Two difficulties in this are significant for our project.
The first lies in the limitations of Aristotle's conception of `categorical proposition'.\index{proposition!categorical}
Categorical propositions were the kinds of proposition involved in syllogistic inference, and the forms of these propositions were too limited for expressing the principles or the conclusions involved in mathematics and science.

This narrow understanding of the richness of language permitted Leibniz to discover a decision procedure which, though effective, was applicable only to language too impoverished for mathematics or science.

Leibniz correctly noted that these propositions asserted inclusion between two concepts, for example, `all men are mortal' asserts that the concept of mortal is included in the concept of man, the class of men is a subclass of the class of mortals.

We will look a little closer at this issue later, but for now, we simply note, that the kind of \emph{lingua characteristica}\index{lingua characteristica} which Leibniz had in mind (but never realised), could not have been adequate to the task, and the \emph{calculus ratiocinator}\index{calculus ratiocinator} which might well have sufficed for such a language, would not suffice for the more complex languages which might adequately serve mathematics and science.

\section{Subsequent Developments}

Of these the two most important are the emergence of modern symbolic logic and the discipline of mathematical logic, and the development of electronic digital information storage and processing systems.

\subsection{Frege and Russell}

\index{Frege}\index{Hume}

A hundred years after Leibniz, after a cogent account of the epistemological status of mathematics by David Hume, an alternative view was presented by Immanuel Kant.

David Hume believed that the statements of mathematics belonged to a category of propositions which asserted 'relations between ideas', rather than among those which asserted `matters of fact'.
For Hume mathematics was similar in epistemic status to logic, and could be known by the same method, logical proof.

Kant disagreed.\index{Kant}
For Kant the truths of mathematics could only be derived logically if certain non-logical principles were used.
He also held that those principles could be known \emph{a priori}, without recourse to observation of the material world.
Because mathematics depends on those non-logical principles, in the terminology which Kant himself introduced, mathematics was \emph{synthetic}.\footnote{The word was not new (having previously been used for a kind of proof), but this use of the word (to demarcate a kind of proposition) was Kant's innovation.}
In Kant's new terminology, Hume's view was that mathematics is \emph{analytic}.

During the nineteenth century mathematicians began to pay some attention to logic, a subject matter previously studied only in philosophy.
Gottlob Frege, a mathematician and philosopher, took issue with Kant's position on the status of mathematics, and set about demonstrating that Mathematics is \emph{analytic}.
Frege's maxim was that mathematical truths were the logical consequences of the definitions of the mathematical concepts which they employ.
This can be encapsulated in the slogan:

\begin{center}
  {\it Mathematics = Logic + Definitions}
\end{center}

  
Inspired in part by the previous ambitions of Leibniz, Frege devised a new logical notation suitable for the expression of mathematics.
This he called `\emph{begriffsschrift}'\index{begriffsschrift}, concept script.
Though devised for mathematics, Frege considered it suitable for language more generally.
He furnished this new formal notation with explicit and detailed rules of inference, something which other mathematicians had not attempted.
His concept script. thus presented with a formal deductive system, greatly surpassed the expressive and deductive power of Aristotle's categorical propositions and Syllogistic.
It was Frege's intention that it should be sufficient to derive the theorems of mathematics from definitions of the mathematical concepts they contain.

This was a development of enormous importance, sweeping aside the Aristotelian syllogistic which had dominated logic for more than two thousand years and replacing it with a much richer and more powerful system adequate to the purposes conceived by Aristotle and Leibniz\footnote{with qualifications...}, but previously beyond reach.

Frege then set about using his new logical system for the formal derivation of the theorems of arithmetic, aiming thereby to demonstrate the analyticity of mathematics and refute an important feature of the philosophy of Kant.

Meanwhile, Bertrand Russell,\index{Russell, Bertrand} was also approaching the same objective.
He had studied and published a book \cite{russellPL} on the philosophy of Leibniz early in his career and was already well advanced in writing ``\emph{The Principles of Mathematics}'' \cite{russellPRM} when he became acquainted with the work of Frege.
He discovered on examination of the first volume of Frege's "\emph{Grundgesetze der Arithmetik}"\footnote{Principles of Arithmetic} \cite{frege1893} that a contradiction could be derived in Frege's logical system.
This he communicated to Frege at a time when the second volume of the \emph{grundgesetze}, ostensibly completing
Frege's demonstration of the analyticity of mathematics (the thesis which became known as `logicism') was about to be published.

Frege's project never recovered from this blow, but it had been Russell's intention to proceed to the same end after completing his own informal account of the principles of mathematics. \cite{russellPRM}

The incomsistency of Frege's system was a huge problem for Russell in his own attempts at the same objective.
To establish the logicist thesis it would be necessary to deliver a formal deductive system, capable of deriving the theorems of arithmetic, which did not suffer from the defect in Frege's system, or any other such logical defect, and to do that in a way which was properly motivated by philosophically convincing foundational principles.
This took Russell the best part of a decade and, of course, not all philosophers were convinced.

The system Russell devised he called the ``theory of types''; it was first published in 1908 \cite{russell08}.
Formal derivations involve very many small inference steps, and the resulting proofs would ultimately occupy three large volumes\footnote{``Principia Mathematica''\cite{russell1913}}\index{Principia Mathematica} on which Russell collaborated with Alfred North Whitehead.\index{Whitehead, Alfred North}

Cumbersome though it was, the possibility of rigorous formalisation of mathematics had been established, and some of the major obstacles to Leibniz's project were overcome, though his own conception of how to accomplish that project had been left in the dust with Aristotle's logic (though not Aristotle's conception of ``Demonstrative Science'').
The philosophical purpose of the efforts of Frege and Russell was nevertheless in vain.
The thesis that \emph{mathematics is analytic} would have only a fleeting period of popularity.
Whatever its merits that thesis was not essential to the ideas of Aristotle and Leibniz which concern us here.
The idea that science could be rendered as a deductive theory after the manner of Euclidean geometry but with the scientific laws in the place of geometric axioms remained intact, and at this point no serious doubt about the possibility of mechanisation had ben established.

The twentieth century was to see many developments which improved understanding of what is in principle possible in relation to these aspirations.

\subsection{Rudolf Carnap}

Rudolf Carnap's early acquaintance with these ideas came from the sparsely attended lectures of Frege on his \emph{begriffsschrift}.
It was during this period that Carnap came to feel that propositions are never truly clear until they have been expressed in formal notation.

Carnap declared the principle influences on his philosophy to have been the work of Frege and Russell.
Frege in relation to logic and semantics, and Russell a broader philosophical influence.
Though Russell's own formal work addressed only the derivation of mathematics, his later philosophical stance, particularly as revealed in \cite{russell21}, concerned both the adoption of logic as the method for philosophical research, also advocating similar methods in the sciences.




