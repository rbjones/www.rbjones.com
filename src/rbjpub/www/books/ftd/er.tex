% $Id: er.tex,v 1.1 2012/01/23 21:40:02 rbj Exp $

\chapter{Epistemic Retreat}\label{EpistemicRetreat}

This chapter is philosophically epistemological, providing a kind of
constructive epistemology in the form of some architectural principles
in relation to the representation of knowledge in networked storage
systems and its processing by distributed processing systems involving
both natural and artificial processing elements. 

The chapter is therefore a conflation of aspects of philosophical
epistemology with abstract architectural design for knowledge
processing systems. 
This necessitates (or flows from) some novelty in my conception of
both these enterprises, so I will begin with some discussion of the
innovations involved. 

Epistemology may most briefly be characterised as \emph{the theory of
  knowledge}. 
It has typically been anthropomorphic, i.e. concerned specifically
with \emph{human} knowers, and sometimes linguistic, concerned with
the specifics of the meaning of the concept of knowledge. 

Here I am interested in knowledge in information systems, not in human
brains, and seek to avoid giving any deep scrutiny to the meaning of
the word \emph{know}.
The conduct of epistemology without attachment to the concept of
knowledge is analogous to the preference of science to avoid vague and
relative concepts such as ``hot'' and ``cold'', in favour of objective
and precise properties such as ``temperature in degrees Kelvin''.
Instead of claiming knowledge, we will be aiming to provide more
objective descriptions of grounds for the truth of
propositions, or of evidence showing the reliability and fidelity of
abstract models of the real world.
I do not prescrible particular measures, but provide instead a context
in which a plurality of comparative evaluations can be accomodated.

The first comparision is the one provided by Hume's fork, and this has
a major impact on the epistemology and the knowledge architecture.
Because of the great precision with which ``relations between ideas''
can be expressed, such propositions we regard as assertable, and in
connection with such assertions ``epistemic retreat'' involves a form
of assertion in which the grounds for asserting truth are made
explicit.
On the other side of Hume's fork, in relation to ``matters of fact'',
epistemic retreat in the first instance reflects the imprecision in
our knowledge of Plato's fleeting world of appearances.
Scientific knowledge is regarded as embodied in abstract models of the
real world, which are applied by deduction (this is our version of a
\emph{nomoligical-deductive} scientific method).

A principle effect of the status of Hume's fork in this epistemology
is that our formal knowledge base asserts only \emph{logical} or
\emph{amalytic} truths.
In it, scientific theories are represented through abstract models,
which will in general have concrete interpretations, but which have
the logical status of definitions rather than assertions.

Epistemology has been concerned with the refutation of scepticism.
It here embraces an open scepticism (but not a dogmatic negative
scepticism), and is therefore oriented not with the refutation of
scepticism but with the establishment of a viable constructively
sceptical system.

The first step in the moderation of a pyrrhonean scepticism is the
recognition that, though no proposition is absolutely certainly true,
some appear to be more certain than others, and among the more
doubtful there are also degrees of doubt.
Can we know with absolute certainty that some proposition is more
certain than another?
No, but we can form opinions about relative certainty which are
themselves more solid than our opinions about truth (again, abstaining
from absolutes).
Among these comparative assurance judgements are the relations between
a proposition and the evidence we have for it.
It will generally be the case that our confidence in the evidence on
which we base an opinion will by stronger than that in the proposition
we infer from the evidence.
Similarly, it will be the case that a statement effectively describing
the evidential support for a theory will can be more confidently
asserted than the theory itself.

This is the principal way in which Metaphysical Positivism interprets
the positivistic principle that science should not go beyond
presentation of observational data.
It is not taken to impede the formulation of general scientific
theories which go beyond the content of any possible body of
experimental evidence.
It is taken instead to impede the \emph{assertion} of such empirical
generalisations as \emph{truths}.
The positive scientist instead compiles various bodies of evidence
which provide a basis for decisions about when the scientific
generalisation might be applied.

The experimental data obtained will provide information about the
fidelity and accuracy of the theory as modelling the real world in a
variety of circumstances, so that someone contemplating use of the
theory will be in a position to form an opinion about whether the
theory will provide a sufficiently reliable and accurate model for his
purposes.

The mitigated scepticism of positivistic philosophy, following David
Hume, accepts a priori truths of reason as certain, but expects of
positive science that it does not go beyond what is entailed by the
experimental or observational evidence.
Since most scientific theories involve empirical generalisations which
go beyond any finite collection of particular observations, positive
science would seem by this doctrine to be eviscerated.

Metaphysical positivism recognises two principles which constitute a
kind of foundationalism.
The first is connected with the sceptical doctrine that all we know is
that ``appearances appear''.
However, what we count as an ``appearance'' is not confined to sensory
impressions.
Any impression which we may form on the truth of some proposition,
whatever its source, is counted as an appearance.
Such appearances we accept as what they are, and the body of science
is considered to constituted just an organised presentation of a great
deal of such material.
The enterprise is not solipsistic, it is collaborative, and we
therefore recognise as significant the source of the impression, the
identity of the person or entity who was the subject of the
impression.
Because of the broadening of the notion of appearance, propositions
expressing such appearances are called ``opinions'' and are to be
tagged or digitally signed by the entity whose opinion they are.
The kind of entity which has opinions is called an ``authority''.

Because of the collaborative nature of the enterprise, an opinion will
normally be formed by an authority on the basis of (as entailed or
otherwise supported by) some collection of opinions of other authorities.
It is therefore normal for an opinion to be expressed, on the
assumption that various authorities can be trusted, or more
specifically, on the assumption that all their previous opinions are
true.
The collection of authorities on the basis of whose opinions a further
opinion is formed, together with the authority forms the new opinion,
give a measure of the risk associated with the opinion which which I
call a degree of assurance.
These degrees of assurance are partially ordered.
The more authorities whose opinions are involved, the lower the level
of trust.
The partial ordering becomes a lattice when we allow that an opinion
may be endorsed by several authorities on the basis of distinct
collections of other opinions.
Adding more independent opinions increases the degree of assurance.

