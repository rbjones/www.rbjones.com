% $Id: restate.tex,v 1.14 2015/04/23 09:58:08 rbj Exp $

\chapter{Metaphysical Positivism}\label{MetaphysicalPositivism}

The theoretical side of positive philosophy is here referred to as \emph{Metaphysical Positivism}\index{Metaphysical Positivism}.
In light of the discussions so far I now present a concise statement of that position.

Metphysical Positivism is presented in a manner similar to the philosophy of Rudolf Carnap not as a body of doctrine but as a proposal for consideration.
This proposal is linguistic, methodological, and epistemological.
It concerns how we might use language to represent, reason about and apply knowledge.
It is meta-theoretical or meta-philosophical at multiple levels.

Metaphysical positivism provides certain background materials in the
context of which positive philosophy must be understood.
Positive philosophy itself is simply philosophy conducted in the
context of that background.
The boundary between the two is not sharp but approximates to the
traditional division between theoretical philosophy and practical
philosophy.

In theoretical philosophy we consider those aspects of philosophy
which have greatest impact on philosophical method, viz. metaphysics,
logic, epistemology, and certain aspects of scientific method.

Metaphysical positivism is so called firstly because of its close
connection with \emph{logical positivism}, particularly with the
philosophy of Rudolf Carnap, and because of the single most striking
difference between it and logical positivism, which is in its use of
the term \emph{metaphysics}.
It may therefore be helpful to begin with some remarks about the
similarities and differences between metaphysical positivism and its
predecessor.

\section{First Base}

{\it Metaphysical Positivism} is principally concerned with {\it the
  foundations of knowledge}\index{Foundations!of knowledge}. 
It therefore places {\it epistemology}, the theory of knowledge, to a
central place in theoretical philosophy. 

If we seek to build an enduring structure, it is best to build on a
solid foundation. 
Critics of foundationalisms have taken foundationalism as demanding
that such a foundation be immune to doubt, that it be, as a foundation
for knowledge, absolutely solid.
The foundationalism of Metaphysical positivism is not predicated on
the existence of such foundations.
It is rather the more pragmatic aim, given that we must start
somewhere, to find the best place to start from.
A foundation is therefore not to be absolutely solid, but just solid
enough; fit for purpose.

When I speak here of a foundation as a place to start, this should not
be taken too strictly.
When we build a house, we begin with the foundations.
The construction of the foundations may take perhaps one third of the
time required for completing the house.
In this case, the foundation is not something which we simply identify
and use as a starting point.
It is a stage in the construction, after which the character of the
enterprise changes.

The foundation provides something on which a house can be built.
The reason why the house stands securely is because it is built on a
solid foundation.
This is not the reason why a foundation is solid.
A foundation is not evaluated in the same way as the house.
Sometimes a foundation is solid because it consists of concrete laid
on solid rock.
Sometimes a foundation is solid because it is a concrete raft laid on
something a much less rigid, perhaps clay.
Sometimes a foundation is solid because it is made by driving piles
into ground which is rather soft.

Ultimately, in these cases, the criteria are pragmatic and based on
experience.
The best foundation for a building is chosen taking into account the
nature of the land on which it is to be build, the kind of building it
is to support, and a great deal of experience and scientific knowledge
of how different kinds of foundation will behave in these
circumstances.

The foundationalism of Metaphysical Positivism is similarly pragmatic.
It consists of ideas about what ways of establishing, evaluating and
applying different kinds of knowledge have proven effective, and on
what methods we may expect to be effective as information technology
and other factors transform the way we work with knowledge in the
future.

The foundationalism of Metaphysical Positivism is thus
self-consciously futuristic, it is oriented towards ways of working
which will make the most of future advances in information
engineering. 
The pragmatic aspect brings with it an epistemological pluralism.
The foundationalisms I here espouse are not offered to the exclusion
of other approaches.

There are three major stages in the foundations, closely connected
with Hume's two forks.
The stages are addressed in sequence, each building on the earlier
foundations.

An important part of the foundations proposed is simply conceptual.
It is in the adoption of certain concepts with relatively definite
meanings.
The first concepts to consider are those which distinguish the three
kinds of foundation.

The most fundamental of these comes from Hume's fork, and the
principal concept which we associate with this dichotomy which Hume
identified is that of {\it analyticity}.
Our first foundations are therefore foundations for analytic truth,
and this is the main focus of the discussion here.

In relation to analytic truth, we do not advocate that any proposition
be regarded in as true in a completely unqualified way.
The suggestion is that our system involves only the expression of
opinions on analyticity, and that such opinions are in general
expressed as based on certain other opinions.
In very many cases these will be very solid opinions.
Often the opinion will be the ``opinion'' of a an interactive proof
tool which has constructed and checked a formal proof of the
proposition and made use of no other opinion.

The second kind of foundationalism concerns synthetic propositions,
the other side of Hume's first fork.
Let us think of scientific laws as typical of this kind of knowledge.
In respect of such laws I do not advocate that these should be
considered in terms of truth and falsity.
Experience tells us that scientific law are generally no more than
approximations to ``the truth''.
Whether or not this is always the case, there are sufficiently many
useful scientific laws which are known not to be strictly true that an
epistemology which recognizes the pragmatic value of these
(and that they do constitute a part of our {\it knowledge} of the world)
is desirable.

Scientific laws are therefore considered as models of aspects of
reality which are not regarded as either true or false, but as more or
less accurate and reliable models of various aspects of the real world.
The construction of such models is a purely logical matter, and the
theoretical aspects of science which consist in the analysis of such
logical models are covered under the foundational proposals for
analytic truth.
The new epistemological problems which arise concern the relationship
between these abstract theories and those aspects of the world which
they model.

\section{By Comparison with Logical Positivism}

Here and throughout, whenever I speak of logical positivism this should
be understood to refer specifically to the philosophy of Rudolf Carnap
whenever it concerns a matter on which the logical positivists may not
have been unanimous.

The headline contrast with logical positivism is in relation to the
word metaphysics, which I use in a manner quite distinct from the way
in which it is used by Carnap.
The best known feature of Carnap's philosophy is his repudiation of
metaphysics, around which, it is easy to suppose, his entire philosophy
revolves.
Metaphysical positivism embraces metaphysics, but the kinds of
metaphysics which are accepted are not the kinds which were rejected
by Carnap.

Metaphysics for Carnap is construed in very specific ways, and rather
more narrowly than is usual in the positivistic tradition.
Positivism is usually associated with nominalism, and involves the
denial that abstract entities exist.
Carnap on the other hand, was an ontological pragmatist, it sufficed
for him, that reasoning about or using abstract entities is convenient for science, to
justify their use.
His paper \emph{Empiricism, Semantics and Ontology} is an exposition
of his liberal attitudes in these matters.

The metaphysics which Carnap did reject fell primarily under two
headings, but we see also from his autobiographical writings that the
antipathy to metaphysics is rooted in the same attitudes which lead to
his principle of tolerance in relation to forms of language.

Its worth mentioning this attitude towards language, before looking at
the two specific conceptions of metaphysics.
As a student Carnap enjoyed discussion with fellow students with a
variety of metaphysical views.
Such netaphysical views were apparent in the choice of language with
which they talked about the world.
Thus, students of a positivistic bent would speak of the world in
phenomenalistic language, whereas others would speak directly of
physical entities rather than their phenomenal manifestations.

Carnap found these schools of thought to be intolerant of each other,
each thought the language of the other illegitimate or incorrect in
some way (typically in its underlying {\it ontology}).
Carnap himself was quite happy to talk to each of these groups in
their own idiom, to discuss science and philosophy with them in their
own terms, but they felt he was by this being inconsistent.

Carnap's attitude was pragmatic.
This pragmatism is made explicit later in his philosophy as his
``principle of tolerance''.
Carnap saw the underlying issues which were considered decisive in
determining the legitimacy of these languages, essentially the
question of what kinds of entity really exist, to be meaningless.
This is his first rejection of metaphysics, the feeling that these
questions of ultimate ontology lacked objective content and should be dealt
with pragmatically and non-exclusively.
If more than one conception of such ontological fundamentals could
each be of value in understanding the world, then their use could not
properly be proscribed by metaphysical arguments.

From these origins we come to two specific conceptions of metaphysics.
The first is \emph{the synthetic a priori}, the second heading covers
claims which have no definite meaning.

In Carnap's conception of metaphysics the first of these categories is
void.
Likewise in metaphysical positivism.
It is so partly because of the definitions (which are adopted in
metaphysical positivism) of the concepts, and partly as an adopted
epistemological criterion.
Of these, more later.

So far as those which fail to be synthetic because they are
meaningless, the position of metaphysical positivism is softer.
It is in the nature of philosophy that it uses or investigates
concepts whose meaning may be uncertain, or difficult to articulate.
Dogmatic scepticism about meaning in various degrees is common among
academic logicians and philosophers, and is not a feature of
metaphysical positivism.
In this context by \emph{dogmatic} scepticism we mean the movement
from incomprehension to rejection.
Our position in relation to doubt about meaning is to reserve
judgement.
However, if the proposition in doubt were offered as synthetic a
  priori, in the sense in which these terms are understood in
  metaphysical positivism, then a firmer rejection would be called for. 

Since Kripke the rejection of the synthetic a priori has generally
been supposed to have been refuted.
However, it can be seen that insofar as the relevant arguments are
sound, then they must relate to concepts distinct from those adopted
in metaphysical positivism (and distinct from these concepts as used
by Carnap).
The first step in showing this is to note that Carnap \emph{defines}
necessity as in terms of analyticity.

................


Having seen the historical development of philosophical positivism,
having reconsidered positivism in the light of the principal
criticisms which were levelled at its most recent manifestation in
\emph{Logical Positivism}, and taking account of certain ideas
about how information technology may transform the nature of
knowledge, it is now time to draw these themes together in a concisely
stated positivistic synthesis.

Metaphysical Positivism is a graduated, constructive scepticism.
In describing it as sceptical the emphasis is placed upon an open
minded suspension of judgement.

This suspension is graduated, and does not deny apparent and
sometimes quite radical differences in our confidence of working
hypotheses.
The most fundamental of such differences are associated with that
between logical and empirical knowledge associated with the
analytic/synthetic distinction, and this leads to quite different ways
of evaluating and affirming analytic and synthetic hypotheses.
The constructive side of this scepticism leads us into an epistemology
which is coupled with architectural principles for the the future
expansion of our knowledge in the context of a globally shared
information infrastructure.

\subsection{Metaphysics}



\section{Principal Features}

Metaphysical positivism is primarily concerned with analytic method,
and with the conceptual framework in which such methods can be
articulated, evaluated and applied.

It is both linguistically and methodologically pluralistic.
As in Carnap's pluralism language is adopted on the basis of pragmatic
considerations.
We are however aware, as Carnap was, that choice of vocabulary is
important, and there is no suggestion that these choices are
arbitrary.

 
