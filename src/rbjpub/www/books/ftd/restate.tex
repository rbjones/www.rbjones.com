% $Id: restate.tex,v 1.8 2012/01/23 21:40:02 rbj Exp $

\chapter{Metaphysical Positivism}

Having seen the historical development of philosophical positivism,
having reconsidered positivism in the light of the principal
criticisms which were levelled at its most recent manifestation in
\emph{Logical Positivism}, and taking account of certain ideas on
about how information technology may transform the nature of
knowledge, it is now time to draw these themes together in a concisely
stated positivistic synthesis.

Metaphysical Positivism is a graduated, constructive scepticism.
In describing it as sceptical the emphasis is placed upon an open
minded suspension of judgement.

This suspension is graduated, and does not deny apparent and
sometimes quite radical differences in our confidence of working
hypotheses.
The most fundamental of such differences are associated with that
between logical and empirical knowledge associated with the
analytic/synthetic distinction, and this leads to quite different ways
of evaluating and affirming analytic and synthetic hypotheses.
The constructive side of this scepticism leads us into an epistemology
which is coupled with architectural principles for the the future
expansion of our knowledge in the context of a globally shared
information infrasture.

