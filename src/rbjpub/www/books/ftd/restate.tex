% $Id: restate.tex,v 1.9 2012/02/14 20:42:21 rbj Exp $

\chapter{Metaphysical Positivism}\label{MetaphysicalPositivism}

\emph{Metaphysical Positivism}\index{Metaphysical Positivism@\textbf{Metaphysical Positivism}} is
the name I have given to the theoretical aspects of Positive
Philosophy, which is itself of broader scope.

The reason for naming a philosophical manner in this case is
primarily for convenience of reference.
Metaphysical positivism provides certain background materials in the
context of which positive philosophy must be understood.
Positive philosophy itself is simply philosophy conducted in the
context of that background.
The boundary between the two is not sharp but approximates to the
traditional division between theoretical philosophy and practical
philosophy.

In theoretical philosophy we consider those aspects of philosophy
which have greatest impact on philosophical method, viz. metaphysics,
logic, epistemology, and certain aspects of scientific method.

Metaphysical positivism is so called firstly because of its close
connection with \emph{logical positivism}, particularly with the
philosophy of Rudolf Carnap, and because of the single most striking
difference between it and logical positivism, which is in its use of
the term \emph{metaphysics}.
It may therefore be helpful to begin with some remarks about the
similarities and differences between metaphysical positivism and its
predecessor.

\section{By Comparison with Logical Positivism}

Here and throughout, whenever I speak of logical positivism this should
be understood to refer specifically to the philosophy of Rudolf Carnap
whenever it concerns a matter on which the logical positivists may not
have been unanimous.

The headline contrast with logical positivism is in relation to the
word metaphysics, which I use in a manner quite distinct from the way
in which it is used by Carnap.
The best known feature of Carnap's philosophy is his repudiation of
metaphysics, around which, it is easy to suppose, his entire philosophy
revolves.
Metaphysical positivism embraces metaphysics, but the kinds of
metaphysics which are accepted are not the kinds which were rejected
by Carnap.

Metaphysics for Carnap is construed in very specific ways, and rather
more narrowly than is usual in the positivistic tradition.
Positivism is usually associated with nominalism, and involves the
denial that abstract entities exist.
Carnap on the other hand, was an ontological pragmatist, it sufficed
for him that reasoning abstract entities was convenient for science to
justify their use.
His paper \emph{Empiricism, Semantics and Ontology} is an exposition
of his liberal attitudes in these matters.
The metaphysics which Carnap did reject fell primarily under two
headings.
The first is \emph{the synthetic a priori}, the second heading covers
claims which have no definite meaning.

The first of these categories in Carnap's conception of metaphysics is
void likewise in Metaphysical Positivism.
It is so partly because of the definitions (which are adopted in
metaphysical positivism) of the concepts, and partly as an adopted
epistemological criterion.
Of these, more later.

So far as those which fail to be synthetic because they are
meaningless, the position of metaphysical positivism is softer.
It is in the nature of philosophy that it uses or investigates
concepts whose meaning may be uncertain, or difficult to articulate.
Dogmatic scepticism about meaning in various degrees is common among
academic logicians and philosophers, and is not a feature of
metaphysical positivism.
In this context by \emph{dogmatic} scepticism we mean the movement
from incomprehension to rejection.
Our position in relation to doubt about meaning is to reserve
judgement.
However, if the proposition in doubt were offered as synthetic a
  priori, in the sense in which these terms are understood in
  metaphysical positivism, then a firmer rejection would be called for. 

Since Kripke the rejection of the synthetic a priori has generally
been supposed to have been refuted.
However, it can be seen that insofar as the relevant arguments are
sound, then they must relate to concepts distinct from those adopted
in metaphysical positivism (and distinct from these concepts as used
by Carnap).
The first step in showing this is to note that Carnap \emph{defines}
necessity as in terms of analyticity.

................


Having seen the historical development of philosophical positivism,
having reconsidered positivism in the light of the principal
criticisms which were levelled at its most recent manifestation in
\emph{Logical Positivism}, and taking account of certain ideas on
about how information technology may transform the nature of
knowledge, it is now time to draw these themes together in a concisely
stated positivistic synthesis.

Metaphysical Positivism is a graduated, constructive scepticism.
In describing it as sceptical the emphasis is placed upon an open
minded suspension of judgement.

This suspension is graduated, and does not deny apparent and
sometimes quite radical differences in our confidence of working
hypotheses.
The most fundamental of such differences are associated with that
between logical and empirical knowledge associated with the
analytic/synthetic distinction, and this leads to quite different ways
of evaluating and affirming analytic and synthetic hypotheses.
The constructive side of this scepticism leads us into an epistemology
which is coupled with architectural principles for the the future
expansion of our knowledge in the context of a globally shared
information infrastructure.

..............

\section{Principal Features}

Metaphysical positivism is primarily concerned with analytic method,
and with the conceptual framework in which such methods can be
articulated and evaluated and applied.

It is both linguistically and methodologically pluralistic, as in
Carnap's pluralism language is adopted on the basis of pragmatic
considerations.
We are however aware, as Carnap was, that choice of vocabulary is
important, and there is no suggestion that these choices are
arbitrary.

 
