% $Id: restate.tex,v 1.1 2009/03/01 15:40:00 rbj Exp $
\section{Restatement}\label{Restatement}

\subsection{Descriptive Language}

The definitions given here are intended to limit consideration to languages which ``have'' a semantics of a certain form.
This term ``have'' is not intended in any concrete sense, but in a sense similar to notions of existence in mathematics, i.e. it is required that such a semantics exists, not that we know it, understand it, can write it down or even that it could be written down (as in ``there exists a real valued function'').

The term {\it descriptive language} is used here to refer to languages in which one may make statements about the ``real'' world.
For present purposes I intend this to exclude languages whose subject matter is entirely abstract, such as set theory.

In such languages {\it indicative} sentences are those which (given certain contextual information) say something about the world, rather than, for example, conveying a command.

A {\it statement} is an indicative sentence together with sufficient context to disambiguate the sentence.
Disambiguation here means determining the proposition expressed by the sentence in this context.

A {\it proposition} is the meaning of an indicative sentence which encompasses the truth conditions of the sentence.

\subsection{The Dichotomies}

The term ``possible world'' here refers generally to the kind of circumstance relative to which the truth conditions of a proposition are given (and may be language specific).

A proposition is {\it necessarily t} if it has truth value ``t'' in every possible world (I do not insist on there being just two truth values, the number of truth values is determined by the semantics of the language).

A proposition is {\it contingent} if it does not have the same truth value in every possible world.

A sentence is {\it analytically t} for some truth value ``t''	if the proposition it expresses (its meaning) is ``necessarily t''

A sentence is {\it synthetic} if it is not ``analytically t'' for any truth value ``t''.

A justification for some claimed proposition is {\it a priori} if it makes no reference to any empirical observation or any contingent proposition.
Note however, that the constraint does not apply to any aspect of the justication of a statment	which is concerned exclusively with establishing its meaning (i.e. establishing which proposition is expressed by the statement)

I propose that we should accept only ``a priori'' justifications for necessary propositions, and only ``a posteriori'' justifications for contingent propositions.

Its intended that there are no substantive claims in the above, these
are all proposals for usage, except the last, which is some other
kind of proposal.

\subsection{Some Consequences}

If we confine consideration to languages with two truth values (i.e.
in which a proposition is either true or false in every possible world)
then there are three essentially coincident dichotomies.

The identity of the distinctions is more conspicuous if the definitions
of necessary and contingent, are extended to statements in the obvious
way,
