\def\rbjidrestate{$$Id: restate.tex,v 1.7 2011/11/29 16:45:35 rbj Exp $$}

\chapter{Chapter}

\section{Tolerance, Pluralism, Metaphysics}

These three topics are intimately interwoven in Carnap's philosophy.
I'm going to attempt an illucidation elements of Carnap's philosophy by discussing some interpretations of Carnap on pluralism which seem to me to be, in various degrees, mistaken.

I discuss the views on Carnap's pluralism of three fictitious philosophers, R, A and K.
Their positions are suggested to me by the writings of three actual philosophers, but it is not necessary for my present purposes to resolve the question of what those real philosophers actually meant, it suffices to discuss positions they might possibly have meant.

First a brief preliminary statement of some key aspects of Carnap's notion of pluralism and his principle of tolerance.
According to Carnap's intellectual autobiography \cite{carnap63,carnap63a} his tolerance was anticipated in his student days by a willingness to discuss issues with friends in a variety of ``philosophical languages'', which corresponded to distinct and incompatible metaphysical stances.
The principle of tolerance did not appear until much later, in ``logical syntax'' \cite{carnap34,carnap37}.
It is understandable that people will come away from this book with diverse and incompatible ideas of what Carnap's pluralism is.
Carnap does not actually use the word pluralism in the book.
He does enunciate his ``principle of tolerance'' in \section 17:

\begin{quote}
It is not our business to set up prohibitions, but to arrive at conventions.
\end{quote}

Which is further explained in that section as:
\begin{quote}
Everyone is at liberty to build up his own logic. i.e. his own form of language, as he wishes.
\end{quote}

I begin with R.
R claimed that he was more pluralistic than Carnap.
I reacted against that, since I didn't see that Carnap's pluralism actually excluded anything.
The feature of his pluralism which he thought beyond Carnap's was that he would accept that the same sentence could have different meanings and different truth values.
But this is also the case for Carnap.
For Carnap this would be possible by having that same sentence in two different ``language frameworks'' (which we may perhaps think of as being both the grammar of the language and the semantics in some form).
I mention this here because it contrasts with the next one.

K took a more substantial interest in Carnap's philosophy and wrote quite a bit about it.
He recognises two interpretations of Carnap's pluralism, on the one hand as a substantial thesis, and on the other as a proposal.
However, his critique is concerned with the former.
In this case he takes Carnap to be taking a radical view entailing the indeterminacy of truth in languages like arithmetic and set theory, so that Carnap is to be understood as saying that these truths are not completely determinate but can be chosen arbitrarily.
