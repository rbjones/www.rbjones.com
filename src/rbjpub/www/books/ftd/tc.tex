
\chapter{Analyticity and Analysis}\label{AnalyticityAnalysis}

{\it[
Having established the distinction between logical and empirical
truths we now turn to a closer consideration of logical truth. 

There are two main considerations.
One is the scope and relevance of logical truth.
This should include a discussion of its place in analytic philosophy,
in mathematics, empirical science, and engineering. 
A principle aim in this discussion is both to re-emphasize, as Hume
did, the severe limits to what can be established purely by deduction,
and to show that it nevertheless is of the greatest practical
significance. 
The limitations and potential are to be made real by illustrations of
a different character to those of Hume but which are connected with
the various concerns of my own positive philosophical thinking,
theoretical and practical. 
Specifically in relation to philosophy, some discussion of the nature
of philosophical analysis making the distinction between a claim being
analytic and a claim being ``about language'', i.e. between subject
matter and epistemic status. 

I would like to introduce here the notion of an analytic oracle, which
can then be refined in the next chapter to the FAn oracle.

]}%it

We have seen that the \emph{words} ``analytic''\index{analytic} and
``synthetic''\index{synthetic} acquired in the philosophy of Kant a
sense distinct both from their use outside academic philosophy and
from previous philosophical and mathematical usage.

In non-philosophical use the related terms
\emph{analysis}\index{analysis} and \emph{synthesis}\index{synthesis}
have diverse application, generally concerned with \emph{taking apart}
or \emph{putting together} the parts of some complex whole. 
In the special domain of logical proof, the usage in classical Greece
was similar, connoting two \emph{methods of proof}.
An analytic proof proceeded by analysis of the proposition to be
proven, ultimately reducing it to principles which can be known
without proof.
A synthetic proof begins instead with axiomatic principles reasoning
forward until the desired theorem is eventually reached.
In contemporary automation of reason, these different methods, which
we will consider further in a later chapter, are sometimes known as
backwards and forwards proof respectively. 

Kant\index{Kant} introduced a new usage in which analytic and
synthetic are applied to propositions (in the Aristotelian sense),
classifying them along the lines of Hume's fork. 
Kant gave two criteria for this classification, one proof theoretic
(concerning the manner in which such a proposition might be
established), and the second \emph{semantic} concerned with meanings. 
However characterized, analyticity in this sense is closely connected
with deductive reason, and it is our purpose in this chapter to relate
this precise technical concept with the very general notion of
analysis, both in its academic and its more worldly applications. 

\section{Truth Conditions and Necessity}\label{TruthConditionsAndNecessity}

The ideas central to this book cannot clearly be expressed without the use of a number of technical terms.
One of the most fundanental is the idea of `truth conditions', which is not a complex idea, but deserves careful explanation.

Though Ludwig Wittgensein has made much in his philosophy of the diverse ways in which language works, the narrower conception of language from which he saw his ideas diverging is particularly important in this work.
To treat seriously of this kind of language does not require us to deny that language nay sometimes have a different character, or to think those other kinds or uses of language unimportant.

The language which is of particular importance to us here can reasonably be called \emph{propositional language}, because in it we are able to express \emph{propositions}, which we normally do using \emph{indicative sentences}.
A proposition is used to convey explicit information on some topic.
It does so by having a definite meaning, which is understood by both the person who writes (or speaks) the sentence and by the person reading what he has written.
Knowing the meaning of a sentence, or understanding the proposition it expresses, consists in understanding under what circumstances the proposition (and the sentence expressing it) holds true, and under what circumstances it is false.
Asserting the proposition then communicates that those conditions under which it is false do not obtain, and that the circumstances at hand are among those for which the proosition is held to be true according to ts accepted meaning.

The meaning of the word `meaning' is not straightforward to completely or precisely elaborate, and for natural languages is likely to have many aspects which are not relevant to its use for purely propositional reasoning.
For our purposes, only one aspect of what might be included in the meaning of propositional language is significant, and that is the truth conditions of the proposition expressed (taking into account any context necessary to disambiguate any particular assertion).
A systematic account of the truth conditions of the propositions expressible in some language as its `truth conditional semantics'.

To give a full account of the truth conditions of some language there are some further complications which warrant mention in this context.
The first is context sensitivity.
This is particularly important for natural languages, in which many word are inhrently relational, and context is used to understant relative to what standard they are to be understood.
`Large' is a typical example.
To assert that something is large, has a meaning which is sensitive to what it is that is alleged to be large, and may be sensitive to other things as well.
A large spider is likely to be much smaller than a large, or even a small elephant, and a large rocket ignited at home on Bonfire night will be very much smaller than even the smallest rocket for extraterrestrial transport.

We now come to the term `necessary' when this is used as a technical term in philosophy.

A proposition is necessary if its truth conditions are trivial in the sense that in every conceivable circumstance it is true, in no circumstance is it false,
Alternatively we might say, that a proposition is necessary if under no circumstance is it false, in other words, if it is not possible for it to be false,

In these words we have connected necessity and possibility.
The two are interdefinable, since we could just as well have defined `possible' by saying that a sentence is possibly true if its negation is not necessarily true.

We may therefore define necessity in terms of possibility, or vice-versa, but not both,
We will stick to the former.
What we then say about possibility will then determine the notion of necessity.

There are multiple kinds of necessity.
We will consider three:

\begin{enumerate}
\item logical necessity
\item set theoretic necessity
\item metaphysical necessity
\item physical necessity
\end{enumerate}

  
