% $Id: ak.tex,v 1.2 2015/04/23 09:58:07 rbj Exp $

\chapter{The Architecture of Knowledge}\label{ArchitectureKnowledge}

It is a thesis of this work that the advancement of information
technology renders choice of \emph{analytic method}, not only in
philosophy but wherever analysis might prove useful, dependent upon
the software available to support the method, and that the
architectural design of such software should take place in the context
of an explicit (if generic or pluralistic) conception of analytic method.

This is an interdependency which may usefully be considered at the
very earliest stages and at the highest and most abstract levels in
the development of method and of information or knowledge architecture.
The interdependency is such that we may be tempted to identify a
certain kind of architectural design with a certain kind of fundamental
philosophy or meta-philosophy.

In this chapter I undertake an analysis of knowledge architectures based
the ideas about knowledge which have been so far presented.
It is not desirable that an architectural discussion in a book of
philosophy enter into much specific detail, so the aim here will be
the analysis of certain ideas about the structure of knowledge as
represented in information systems and the interaction between such
conceptions of structure with the kinds of functionality which the
information systems might then support, the methods which they
facilitate, and the directions of future development to which they are
sympathetic.

Rather than attempting wholly to effect the integration of architectural design
and constructive philosophical analysis,
I will present these as two different perspectives upon a single
enterprise, in this chapter the architectural design, in the next the
philosophical perspective.

In this chapter the discussion will fall into two parts.
An architecture is an abstract high level description.
The analysis or evaluation of an architecture, must be undertaken
against some prior conception of the aims which the architecture is
intended to realise.
In engineering terms these are high-level requirements.
In philosophical terms, these requirements correspond to a delineation
of the problem domain.

The first level at which \emph{positive philosophy} departs from being
purely analytic is in the choice of subject matter.
A substantive statement about some practical matter, perhaps in
politics or economics, may be clothed in a pure analysis, implicit in
the choice of system to be studied.
To promulgate ideas about how society might be organized, it would
suffice to proceed by analysis, considering a class of realisations of
the ideas and examining their relative merits.

This is the manner in which I proceed here.
My interest is in certain approaches to the development of knowledge
as a collaborative enterprise (which it usually is) making effective use
of our developing information processing and networking capabilities.
I begin the architectural discussion by setting out the domain of
enquiry as a statement of requirements, and then proceed to consider
and compare some of the ways in which those requirements might be met.

These two stages, statement of requirements, response to requirements,
will not be monolithic.
The requirements will be stated little by little.
To each stage architectural responses are considered, and the
requirements may then be augmented in the light of the analysis.

\section{Requirements from Leibniz and Carnap}

I have already identified the projects of Leibniz and Carnap as points
of departure, so I begin with some key features of those projects.

The principal elements of Leibniz's project were:

\begin{itemize}
\item A Universal Language.
\item An Encyclopaedia encompassing all scientific knowledge.
\item A Calculus Ratiocinator for deciding truth.
\end{itemize}

Carnap's project was more narrowly scoped, but shared the first two
items recast in pluralistic terms.

Here we adopt all three, adjusting the statement in the direction of
pluralism, and thinking of information technology.

Thus we are interested here in:

\begin{itemize}
\item[RA1] A system for the representation of propositional knowledge.
\item[RA2] A substantial online knowledge base represented in that system.
\item[RA3] Software supporting the further development and application of the knowledge.
\end{itemize}

Carnap attached considerable importance to the distinction between
analytic and synthetic propositions.
Though he did not acknowledge the influence of Hume, he agrees with
Hume in characterizing essentially the same dichotomy, Hume's fork, in three
distinct ways.
Carnap sought to adapt methods similar to those of Frege and Russell
in the formal derivation of analytic propositions to languages in
which synthetic propositions could be expressed and used in formal
derivations.

Carnap's approach to the meta-theory of such empirical languages is not
from our point of view entirely satisfactory.
The approach envisaged here to the connection of our languages with
the empirical world is entirely different.

\subsection{Support for Rigorous Deduction}

The central idea of interest concerns the role of deduction, in the
acquisition and application of knowledge, and the possible
contribution of information technology in support of deduction.

Deduction is relevant in two general ways.
The separation of deductive from other kinds of reason permits the
deductions themselves to be checked more rigorously, and exposes the
premises (which might otherwise have been obscured) on which the
deductions depend.
We are here concerned with a machine supported injection of formality
into reasoning, the intended effect of which is to improve precision,
rigour and reliability.

Formality and deduction also potentially enable new kinds of
functionality to be realised.
This is because the search processes involved in finding proofs of
logical conjectures can serve to discover witnesses for existential
claims, and hence solutions to design problems.
The ability to demonstrate compliance of such a solution underpins
and sanitizes the application of exotic and possibly unreliable
methods during the search for a solution, hence allowing solutions to
be discovered which might never be found by less exotic algorithms.

\section{Epistemic Retreat}

The positivistic idea of epistemic retreat influences the requirement
in various ways.

Firstly we distinguish between analytic and synthetic judgements, and
between formal and informal claims.
The system is primarily concerned with the formal side.

As in Hume, we accept analytic propositions as exhausting those which
can be known with certainty.
In fact we go one further, taking empirical claims to be at best
approximations, best thought of and formally represented as
\emph{models} of aspects of reality.
As such they should be assessed or affirmed not as true or false but
in more complex and informative terms.
Thus, what we affirm of an empirical theory is, not its truth, but its
applicability to certain aspects of reality, and the accuracy and
reliability with which it models those aspects under various
circumstances. 

Formally, we do not assert an empirical claim in connection with such
a model, we instead formulate the theory as the definition of an abstract model
of the empirical system under consideration.
On the basis of this definition we can then undertake theoretical
elaboration of the theory, draw consequences in relation to its
application in hypothetical situations, and formally evaluate the
theory against experimental data presented in the terms of the
abstract model.


\begin{itemize}
\item{RE1} Only analytic propositions are formally asserted.
\item{RE2} Empirical theories are represented as abstract models.
\item{RE3} Judgements are qualified by a measure of assurance. 
\end{itemize}







