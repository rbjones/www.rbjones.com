% $Id: language.tex,v 1.1 2012/01/23 21:40:02 rbj Exp $

\chapter{Language Planning}\label{LanguagePlanning}

\section{Universalism and Pluralism}

Frege and Russell, pioneers of the formalisation of mathematics, each
conceived of a single language in which the whole of mathematics might
be expressed and proven.

Frege's formula has been paraphrased:

\begin{quote}
Mathematics = Logic + Definitions
\end{quote}

Which may be read as asserting that given a suitable logic, the whole
of mathematics can be developed simply by writing down formal
definitions of the concepts of mathematics, and then proving the
required mathematical theorems involving those concepts.

There is no reason in this scheme even to contemplate the possibility
of making use of more than one language, but if we do, problems
immediately arise.
The simplest is that the idea of a formal deductive system is such
that deductions take place exclusively in one language.
The deduction rules which govern the structure of proofs in such a
system only allow premises within a single language to be used in a
proof.
The effect of this is to ensure that if more than one language is
used, the results obtained in one language cannot be used in the proof
of results in another language.
This kind of awkwardness is not a feature of informal mathematics and
would represent a serious and avoidable flaw in a formal approach to
mathematics. 

Rudolf Carnap, inspired by Bertrand Russell to adopt a ``scientific''
approach to philosophy, and hoping to do for science what Russell had
done for mathematics, i.e. to establish methods suitable for the
formalisation of science.
This was to be done by adapting the new logical methods to the domain
of empirical science.

There was in this an immediate difficulty, arising from the necessity
of making empirical claims.
Concepts applicable to the material world cannot in general be defined
in terms of purely logical contexts.
It does not seem that one can proceed by analogy with Frege and
plausibly argue that:

\begin{quote}
Empirical Science = Logic + Definitions
\end{quote}

Something more seems to be needed, and it is natural to suppose that
this is a language which goes beyond logic by including concepts which
are empirically significant.

So it appears that Carnap's enterprise compelled him to consider
new languages.
Even before this, Carnap already was, by his own autobiographical
account, a pluralist with respect to language (though he might not at
that time have described himself in those words).
His pluralism at that time was primarily in relation to distinct ways
of talking about the world which were associated with different
metaphysical postures.
The most solid example of this is the distinction between the
materialistic language of science, in which material objects are the
subject matter of the language, and the phenomenalistic language of
empiricist philosophers going back to Hume, who confine themselves to
talking not about material objects but about the sensory evidence
which we have of that world.

It is characteristic of Carnap's quite original attitude to
metaphysics that he was happy to talk in either of these languages
even though their proponents took them to be representative of
incompatible views on metaphysics.
Carnap regarded the underlying metaphysical issues as meaningless, and
took a pragmatic attitude to the use of the languages, eschewing any
supposed metaphysical commitment that might be supposed to entail.
His pluralism, to be made explicit later in his ``principle of
tolerance'', was to the effect that any language can be adopted,
subject to pragmatic considerations, rejecting the metaphysical
dogmatists who asserted the legitimacy of only that language which
embodied their preferred metaphysics.

At the same time as Carnap enrolls himself into what he thinks of as
Russell's programme, in the 1920's, the high noon of the Vienna
circle, Hilbert at the centre of a group of mathematicians in Berlin
is moving forward on a different conception of how to formalise
mathematics, and Carnap, eager to absorb all the new developments in
logic which might be relevant to his project, is paying attention.

Hilbert's approach to formal mathematics, though fully embracing the
new logical methods and pressing them forward, is distinct from the
universalistic conceptions of Frege and Russell and harks back to the
axiomatic method first enunciated in Euclid's elements.
Hilbert had already applied the new methods to the axiomatic theory of
geometry, achieving standards of rigour which for the first time
surpassed those found in Euclid's geometry.
This method envisages a special language for each branch of
mathematics in which the primitives were ``implicitly'' defined by
systems of axioms.

This is a substantial break with the ideas of Frege, who was very
fussy about definitions.
It is essential in the development of mathematics to pay close
attention to definitions, since making use of a definition which is
incoherent invites paradox and invalidates all proofs which depend on
those definitions.
Risks are also associated with undertaking definitions on a peicemeal
basis (defining a function first over one domain, and then over some
other domain), because of the possibility that these partial
definitions might conflict, or that the partiality itself might result
in unsound reasoning (Frege had no methods for reasoning about partial functions).


