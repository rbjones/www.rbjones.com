% $Id: anan.tex,v 1.1 2012/01/23 21:40:02 rbj Exp $

\chapter{Analyticity and Analysis}\label{AnalyticityAnalysis}

\emph{[
]}

We have now considered whether the idea of analyticity or broad
logical truth is a tenable concept, and we have seen the evolution of
more precise characterisations of these concepts.

Other points of controversy important to this project remain,
concerned with the scope, applicability and importance of analytic
truth.
These are important to us here for two distinct kinds of reason.

The theoretical core of Positive Philosophy, \emph{Metaphysical
  Positivism}, is primarily an \emph{analytic} philosophy, and it is
essential in presenting a conception of analytic philosophy to address
some of the reasons why a purely analytic philosophy might be thought
to be of narrow scope and limited value.

The project we are considering is of broader scope, just as were the
projects of Leibniz and Carnap, encompassing not merely an approach to
philosophical analysis, but also important and substantial parts of
mathematics, empirical science, engineering and other activities in
which deductive reasoning might play a significant role.

There is a circle to be squared here.
There is one perspective from which the entire enterprise is without
content, and represents a preoccupation with academic trivia, and
another diametrically opposed perspective in which it is of the
greatest practical significance and may be thought to warrant
substantial and energetic prosecution.

\section{The Scope of Analytic Truth}

I have presented a history of the evolution of the concept of
analyticity and related concepts, and have adopted in Metaphysical
Positivism a conception which is similar to that described by Hume
``relations between ideas''.
A good recent account of this concept may be found in the writings of
Rudolf Carnap, but more importantly analyticity is a characteristic of
the theorems of a large class of modern tools for constructing and
reasoning in formal logical theories.
Further sharpening remains of interest from a philosophical point of
view, and will be discussed later, but for practical purposes, even
its relevance to applications demanding the very highest standards of
rigour, the concept and our technologies for checking when it applies
are sufficient.

It remains a matter of controversy how significant analyticity,
analytic truths, and methods in which they figure prominently are or
might be.

Analytic truths may be thought of as falling into two principal
groups.

The first group consists of true claims in languages whose subject
matter is entirely ``abstract''.
For this we understand abstract entities as mere ideas, about which we
reason by offering a ``definition'' of the relevant domain of entities
and then reasoning logically from the definition to conclusioms about
that domain which will be true \emph{by definition}.
This group encompasses the whole of mathematics.
It also encompasses reasoning about abstract models of any domain
whatever, whether or not one considers the model to be
``mathematical'', so long as the model is well-defined and the method
is deductive.

This is a \emph{Platonic} conception of the scope of deductive
knowledge, and can be broadened very broadly in the direction which
Aristotle took, to yield a body of analytic truths which might be
thought to be \emph{about} the material world rather than purely
concerned with abstract entities.
There are several ways in which this can be done in the context of
modern logic without embracing the complexities of Aristotelian
metaphysics.

Carnap's approach is to adopt formal languages whose subject matter is
the material world, to define the semantics of the languages by giving
the truth conditions, and to define analyticity in terms or such a
truth conditional semantics.
The analytic truths are then those sentences in such languages whose
truth conditions are invariably satisfied.

\section{Analytic Philosophy}

The term \emph{analytic philosophy} was first applied in the $20^th$
century to a kind of philosophy which began at the turn of the century
as Bertrand Russell \index{Bertrand Russell} and G.E.Moore
\index{G.E.Moore}.
These two men, though sharing the idea that philosophy should be in
some sense analytic, had quite different conceptions of the kind of
analysis involved, and their differences remained significant as
methods of analysis evolved throughout the $20^{th}$ century.

Russell conception of analysis was shaped by the new methods in logic
in the development of which he played an important part.
He believed that failures of rigour in philosophical reasoning
resulted in some cases from imperfections in ordinary language, and
could be avoided if a special ideal language were adopted along the
lines of the Theory of Types which he devised with A.N. Whitehead
\index{A.N. Whitehead} for the formalisation of mathematics in
Principia Mathematica.
He did not advocate or practice the adoption of such formality in
philosophical rather than mathematical reasoning, but did advocate the
adoption of similar logical methods.
An example the kind of method he envisaged is the use of logical
constructions.
By such means Russell advocated a ``scientific'' philosophy in which
logical methods transformed philosophy into a rigorous deductive
discipline progressively advancing in a manner similar to that of
mathematics and science, rather a perpertual sequence of conflicting
theories and a lack of solid progress.
We may recall Plato's prescription here, in which the subject matter
of philosophy and the only place where true knowledge might be
attained is in the world of ideal forms.
Even though Russell was an empiricist, his conception of philosophy is
as a deductive discipline.

On the other hand Moore's conception of analysis concerned natural
languages and consisted in the clarification of
concepts in those languages, yielding illumination consistent with
common sense.
There is a connection here with Socractic method, but neither Socrates
nor Plato conferred such authority upon ordinary language and common
sense.
Socrates sought the true nature of concepts such as justice and
virtue, and though he believed that ordinary men were in some sense
posessed of these true concepts, they nevertheless might not correctly
grasp them untill they are induced by a Socratic dialogue to properly
recall this knowledge.

Is it the case that analytic philosophy in the $20^{th}$ century was
narrower in its scope than philosophical tradition from which it grew,
and if so is this a necessary consequence of the conception of
philosophy as analytic, or of some particular ideas of what kind of
analysis was at stake.

Two mid century conceptions of the nature of analytic philosophy
illustrate the issue, those of Rudolf Carnap, exhibited in his
Philosophy of Logical Syntax and the subsequent developments of his
ideas, and those of the ``linguistic philosophy'', exemplified by
J.L.Austin, which prevailed briefly in post war Oxford.

A central thesis of Carnap's philosophy of logical syntax was that
philosophy consists of logical analysis and yields results insofar as
they are established truths, which are themselves analytic.
This apparent narrowing of scope is confirmed by explicit exclusion of
fields such as ethics, not only from the domain of analytic
philosophy, but from scientific discourse altogether (bearing in mind
that this kind of philosophy is itself conceived of as scientific in
character, though not empirical).
The severity of the apparent narrowing here is mitigated by the
predominance in the actual philosophy of Rudolf Carnap of work which
is methodological, including work which ensues in some proposal for
the use of language.
Though philosophy has relinquished any claim to offer factual
enlightenement about the world, it may nevertheless make material
contributions to the advancement of such knowledge by contributions to
the methods of science.

Linguistic philosophy in its most extreme form conceives of philosophy
as the analysis of natural language.







\section{Analyticity in Science}
