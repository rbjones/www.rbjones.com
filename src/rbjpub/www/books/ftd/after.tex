\def\rbjidafter{$$Id: after.tex,v 1.5 2012/01/23 21:40:02 rbj Exp $$}

\section{After Hume}\label{After}

Hume's fork abolishes a certain conception of metaphysics, making
a difficulty in establishing any other.

What it leaves is something that looks rather like science, falling
into two parts.
One part contains only matters devoid of empirical content, albeit
including the whole of mathematics.
The other contains opinions of the empirical world, obtained by
guesswork based on sensory impressions.

Not many philosophers, or scientists find this a very attractive
picture.
Much philosophical reaction to this seeks to refute the general
scepticism in Hume's position, but the refutation of his empirical
scepticism is more of benefit to science than to philosophy.
The significance to philosophy of Hume's fork is more acute in
relation to those special kinds of knowledge, beloved particularly to
Plato, which are the truest subject matter of philosophy, and were
later to be known as metaphysics.
From this point of view \emph{positivism}, first amply exemplified in Hume,
is an \emph{anti-philosophical} philosophy, in which philosophy gives up its
own true ground yielding knowledge to science.

It is understandable that many philosophers will react specifically
against this central feature of positivism,
and in this section we will consider some of these reactions.

These we consider in three aspects.

\begin{enumerate}
\item
Firstly there arises in the ongoing dialectic, further refinement of
our knowledge of exactly where the fundamental line is drawn.
\item
Secondly there are challenges to and reaffirmation of the idea that a
single dichotomy is involved.
\item
Finally, coupled with the previous two there are various ways of
reviving and of dismissing the possibility of some kind of
metaphysics.
\end{enumerate}

Alongside these philosophical themes there are two technical problems
which are gradually addressed.
In order for the dichotomies to be definite it is necessary for
languages to have definite meaning.
One way to achieve this is to define new languages, and for them, give
a definite semantics and deductive system.
The ability to do this appears on the scene very late, little more than
a century ago.
The history both before and after that point of transition of ideas
about semantics and proof, is part of our present concern.

The clarity which I find in Hume's description of his fork, is like a moment of lucidity in life of confusion.
For the next


\subsection{Kant}

The first challenge we consider came from Kant, who was awoken from
his dogmatic slumbers by Hume, and was moved to reject the unity of
the triple dichotomy.

Kant was the first philosopher to make prominent use of the notion of
analyticity.
He has a dual conception, in both following Leibniz in ideas going
back to Aristotle.
The first is apparently semantic, that an analytic sentence is one in which the
subject contains the predicate.
This is in Leibniz, but Leibniz's conception of conceptual containment
is extensional, and therefore corresponds to truth not analyticity.
The other is ``proof theoretic'' (insofar as one can talk of proof
theory at this time), a sentence is analytic if it follows
from the law of non contradiction.
This latter corresponds to Hume's ``intuitively or demonstratively
certain'' which in turn is from Aristotle (the ``intuitive'' part
corresponds to first principles which must be essential truths in
which again the predicate is contained in the subject).

What is distinctive about Kant's philosophy is not his definition of
analyticity, which adds nothing to what is found in Hume and Leibniz,
but his view on what things fall under the definition.
He has reverted to something closer to the earlier empiricist Locke,
for whom the corresponding notion is confined to trivia.

Notably Kant excludes two groups of \emph{a priori} truths from
analyticity, those of arithmetic and of geometry.

\subsection{Bolzano}

With Bolzano we see a first approach to the definition of concepts
relevant to Hume's fork which advance significantly beyond Aristotle.

The techniques used by Bolzano are a considerable advance, and the
conception of logical truth he promulgated is quite close to the idea
of logical truth which has dominated mathematical logic ever since.
However, this is a narrower concept than the ones we have considered
so far, and depends upon the identification of concepts into logical
and non-logical.

Though the technical apparatus which Bolzano deploys is a great
advance, the concepts he defines with this aparatus take us further
away from a notion of logical or analytic truth which is properly
complementary to the notion of empirical or synthetic truth.

\subsection{Frege}

Frege's logical and philosophical work was in part aimed at
overturning Kant's critique of Hume, in particular Kant's refusal to
accept that mathematics is analytic.
To achieve this aim Frege invented a new kind of formal logical system
of which the two most important features were the abandonment of the
Aristotelian emphasis on the subject/predicate form of propositions.
Instead of predicates Frege worked more generally with functions, among
which predicates are functions whose results are always truth
values.
Crucially, logical sentences now have arbitrary complexity, and the
universal quantifier is introduced to operate over an arbitrarily
complex propositional function.

For Frege the notion of logical and analytic truths are separated.
Logical truth similar to Bolzano's in being defined through a
separation of concepts into logical and non-logical, but analytic
truth in Frege represents the most precise formulation of Hume's
truths of reason to that date.

In the analysis of Frege's contribution the notion of free logic becomes significant, (though this is not a term he uses).
This is so because in his \emph{Begriffschrift} or \emph{concept notation} those aspects of deductive logic which do not involve ontology, or questions of what exits, are satisfactorily addressed.
However, when he later comes to apply essentially the same logical system to the development of arithmetic it is necessary to incorporate axioms which suffice to establish an ontology sufficient for mathematics, and at this stage Frege introduces principles which are not logically consistent.

\subsection{Russell}

Russell began his work on mathematical logic rather later than Frege, having been born in 1872, not long before the publication in 1879 of Frege's Begriffschrift.
Furthermore, because Frege's work was ignored, Russell did not become aware of it until he was already well progressed with his own approach to the logicisation of mathematics.
Prior to this work Russell had already studied and written his own account of the work of Leibniz.
Russell perceived Leibniz as having been handicapped by to closely following the logic or Aristotle, most particularly his retaining the idea that all propositions have subject/predicate form.
His logical ideas build on the work of Piece and Schroder on the relational calculus, which provided a more general form for propositions in which a sentence might in effect have multiple subjects of which jointly some predicate is asserted (predicates involving multiple subjects are called relations).

\subsection{Wittgenstein}

\subsection{Tarski}

\subsection{Carnap}

\subsection{Quine}

Having studied Russell's \emph{Principia Mathematica} even as an undergraduate, 

\subsection{Kripke}
The following table gives a chronology.

\begin{table}[h]
\begin{center}
\begin{tabular}{|l l p{4.5cm}|}
\hline
585 BC & Thales & Mathematics\\
428-348 BC & Plato & The Theory of Forms\\
384-322 BC & Aristotle & Essential v. Accidental \\
& & Necessary v. Contingent\\
& & Demonstrative v. Dialectical\\
1632-1704 & Locke & on trifing propositions\\
1646-1716 & Leibniz & \\
1711-1776 & Hume & relations between ideas v. matters of fact \\
1724-1804 & Kant & the synthetic, \emph{a priori} \\
1781-1848 & Bolzano & the method of variations \\
1845-1918 & Cantor & set theory \\
1848-1825 & Frege & Begriffschrift, sinn and bedeutung \\
1848-1825 & Russell & the theory of types \\
1871-1953 & Zermelo & the cumulative hierarchy \\
1889-1951 & Wittgenstein & logical truths as tautologies \\
1901-1983 & Tarski & definition of truth \\
1891-1970 & Carnap & logical syntax, the method of intensions and extensions\\
1908-2000 & Quine & against the analytic/synthetic distinction and modal logics, holism\\
1940- & Kripke & separating analyticity, necessity and a priority via rigid designators\\
\hline
\end{tabular}
\caption{Development of the Analytic/Synthetic distinction}
\label{tab:AnalyticSynthetic}
\end{center}
\end{table}

