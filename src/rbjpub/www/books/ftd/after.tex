\def\rbjidafter{$$Id: after.tex,v 1.7 2015/04/23 09:58:07 rbj Exp $$}

\section{After Hume}\label{After}

Hume's fork abolishes a certain conception of metaphysics, making
a difficulty in establishing any other.

What it leaves is something that looks rather like science, falling
into two parts.
One part contains only matters devoid of empirical content, albeit
including the whole of mathematics.
The other contains opinions of the empirical world, obtained by
guesswork based on sensory impressions.

Not many philosophers, or scientists find this a very attractive
picture.
Much philosophical reaction to this seeks to refute the general
scepticism in Hume's position, but the refutation of his empirical
scepticism is more of benefit to science than to philosophy.
The significance to philosophy of Hume's fork is more acute in
relation to those special kinds of knowledge, beloved particularly to
Plato, which are the truest subject matter of philosophy, and were
later to be known as metaphysics.
From this point of view \emph{positivism}, first amply exemplified in Hume,
is an \emph{anti-philosophical} philosophy, in which philosophy gives up its
own true ground yielding knowledge to science.

It is understandable that many philosophers will react specifically
against this central feature of positivism,
and in this section we will consider some of these reactions.

These we consider in three aspects.

\begin{enumerate}
\item
Firstly there arises in the ongoing dialectic, further refinement of
our knowledge of exactly where the fundamental line is drawn.
\item
Secondly there are challenges to and reaffirmation of the idea that a
single dichotomy is involved.
\item
Finally, coupled with the previous two there are various ways of
reviving and of dismissing the possibility of some kind of
metaphysics.
\end{enumerate}

Alongside these philosophical themes there are two technical problems
which are gradually addressed.
In order for the dichotomies to be definite it is necessary for
languages to have definite meaning.
One way to achieve this is to define new languages, and for them, give
a definite semantics and deductive system.
The ability to do this appears on the scene very late, little more than
a century ago.
The history both before and after that point of transition of ideas
about semantics and proof, is part of our present concern.

The clarity which I find in Hume's description of his fork is like a
moment of lucidity in life of confusion. 
Not until Rudolf Carnap do we find a return to and a positive
refinement of that distinction.

\subsection{A Broad Sketch of the Development}

It would be easy in a sketch of the subsequent history of Hume's fork
to loose the central issues in the detail of the very considerable
developments in logic and philosophy since then.

To help make draw these out the headlines are sketched here before
a little more detail is supplied.

Kant supplied the first challenge to Hume's conception of a single
dichotomy.
Kant rescued a limited conception of metaphysics by separating
the \emph{a priori} from analytic truth (which is contrasted with
synthetic truth), and declaring that our knowledge of arithmetic, of
space and time were \emph{synthetic} \emph{a priori}\index{a priori}.
It is not clear here whether Kant's conception of analyticity differed
materially from Hume's relations between ideas, or whether the concept
was the same but Kant disagreed about its extension.

\subsection{Kant}

The first challenge we consider came from Kant, who was awoken from
his dogmatic slumbers by Hume, and was moved to reject the unity of
the triple dichotomy.

Kant was the first philosopher to make prominent use of the notion of
analyticity.
He has a dual conception, in both following Leibniz in ideas going
back to Aristotle.
The first is apparently semantic, that an analytic sentence is one in which the
subject contains the predicate.
This is in Leibniz, but Leibniz's conception of conceptual containment
is extensional, and therefore corresponds to truth not analyticity.
The other is ``proof theoretic'' (insofar as one can talk of proof
theory at this time), a sentence is analytic if it follows
from the law of non contradiction.
This latter corresponds to Hume's ``intuitively or demonstratively
certain'' which in turn is from Aristotle (the ``intuitive'' part
corresponds to first principles which must be essential truths in
which again the predicate is contained in the subject).

What is distinctive about Kant's philosophy is not his definition of
analyticity, which adds nothing to what is found in Hume and Leibniz,
but his view on what things fall under the definition.
He has reverted to something closer to the earlier empiricist Locke,
for whom the corresponding notion is confined to trivia.

Notably Kant excludes two groups of \emph{a priori} truths from
analyticity, those of arithmetic and of geometry.

\subsection{Bolzano}

With Bolzano we see a first approach to the definition of concepts
relevant to Hume's fork which advance significantly beyond Aristotle.

The techniques used by Bolzano are a considerable advance, and the
conception of logical truth he promulgated is quite close to the idea
of logical truth which has dominated mathematical logic ever since.
However, this is a narrower concept than the ones we have considered
so far, and depends upon the classification of concepts into logical
and non-logical.

Though the technical apparatus which Bolzano deploys is a great
advance, the concepts he defines with this apparatus take us further
away from a notion of logical or analytic truth which is properly
complementary to the notion of empirical or synthetic truth.

\subsection{Frege}

Frege's logical and philosophical work was in part aimed at
overturning Kant's critique of Hume, in particular Kant's refusal to
accept that mathematics is analytic.
To achieve this aim Frege invented a new kind of formal logical system
of which the two most important features were the abandonment of the
Aristotelian emphasis on the subject/predicate form of propositions.
Instead of predicates Frege worked more generally with functions, among
which predicates are functions whose results are always truth
values.
Crucially, logical sentences now have arbitrary complexity, and the
universal quantifier is introduced to operate over an arbitrarily
complex propositional function.

For Frege the notion of logical and analytic truths are separated.
Logical truth similar to Bolzano's in being defined through a
separation of concepts into logical and non-logical, but analytic
truth in Frege represents the most precise formulation of Hume's
truths of reason to that date.

In the analysis of Frege's contribution the notion of free logic
becomes significant, (though this is not a term he uses). 
This is so because in his \emph{Begriffsschrift} or \emph{concept
notation} those aspects of deductive logic which do not involve
ontology, or questions of what exits, are satisfactorily addressed. 
However, when he later comes to apply essentially the same logical
system to the development of arithmetic it is necessary to incorporate
axioms which suffice to establish an ontology sufficient for
mathematics, and at this stage Frege introduces principles which are
not logically consistent.

\subsection{Russell}

Russell began his work on mathematical logic rather later than Frege,
having been born in 1872, not long before the publication in 1879 of
Frege's Begriffsschrift. 
Furthermore, because Frege's work was ignored, Russell did not become
aware of it until he was already well progressed with his own approach
to the logicisation of mathematics. 

Prior to this work Russell had already studied and written his own
account of the work of Leibniz, and Russell's philosophy is
significantly influenced by Leibniz.
Russell perceived Leibniz as having been handicapped by too closely
following the logic or Aristotle, most particularly his retaining the
idea that all propositions have subject/predicate form. 
Russell's logical ideas build on the work of Pierce and Schr\"oder on the
relational calculus.
The relational calculus provided a more general form for
propositions.
A sentence might in effect have multiple subjects of which jointly
some predicate is asserted (predicates involving multiple subjects are
called relations).

The divergence from Aristotle in this matter was independently
progressed by Frege and by Pierce and Schr\"oder.

\subsection{Wittgenstein}

While studying engineering at Manchester, Ludwig Wittgenstein became
interested in philosophy through an interest in the logical
foundations of mathematics.
On advice from Frege, Wittgenstein went to Cambridge to study under
Russell at a time when Russell was deeply involved in the completion
of \emph{Principia Mathematica}\index{Principia
Mathematica}\cite{russell10}.

\subsection{Tarski}

\subsection{Carnap}

\subsubsection{Carnap and Positive Philosophy}

In Hume we see a single dichotomy, described from several
perspectives.
These distinct perspectives were already characterised by appropriate
volabulary which was soon to be augmented by Kant, as he was inspired
to disagree with Hume.

With Carnap, once again these different perspectives came together,
and in Positive Philosophy this conception of three dichotomies
intimately coupled together in describing a single fundamental
cleavage returns.

Between Hume and Carnap great advances were made in logic as a result
of which Carnap's conception of the fork (though he did not use that
term) was much more precise than Hume's, but I think can be seen to be
essentially the same.
Carnap's more precise conception, very close to that in my ``positive
philosophY'' will be used as a yardstick which I will use as I sketch
the evolution of the concepts.

There are two areas of present interest in which Carnap's philosophy
can be compared with that of Hume.
The first is in the way in which the fork is characterised.
The second is in the conception of metaphysics.

\subsubsection{Carnap on the Analytic/Synthetic dichotomy}

While Hume implicitly connects the three dichotomies through a
description of a single dichotomy which appeals to features of each,
the dichotomies are separated out by Kant, who introduces the
use of the terms analytic and synthetic for this purpose.
Subsequent philosophers then follow Kant but it is only in Carnap that
these three dichotomies are once again so closely linked as to be
almost identical.

In Carnap's later philosophy it is the semantic account of the
dichotomy which is fundamental.

\ignore{

Propositional language is conceived of as a vehicle for distinguishing
the particular contingent world in which we live from all those other
worlds which might possibly have but did not in fact happen.
The proposition draws a line in this field of possibilities, on one
side of which lie those under which that proposition would be true,
and on the other those under which it would be false.
Two special cases may arise, in which all possibilities lie on one
side of this line, in one case the proposition being necessarily true and
in the other necessarily false.
I shall count either of these as an instance of a necessary
proposition (rather than using ``necessarily'' as short for
``necessarily true'').
In all other cases we call the proposition ``contingent''.
\index{necessary}\index{necessarily true}\index{necessarily false}\index{contingent}

This account of the modal nature of propositions is connected in the
philosophy of Carnap with semantics via the idea he takes from
Wittgenstein's Tracatatus that logical truths are tautologous.
When thought of as a semantic feature the terminology of Kant is used.
A sentence is called analytic if it can be known to be true
independently of how the world is, simply by reasoning from its
meaning.
This happens precisely when the truth conditions determined by the
semantics show the truth value to be fixed independently of the state of
the world.
A sentence is analytic if and only if the proposition it expresses is
necessary.

This connection between analyticity and necessity was seen as an
aspect of the rejection of metaphysics.
It is a moderate conventionalism telling us that necessary truth flows
not from fundamental features of reality, but from our choices about
semantics (or the accidents of their evolution).
However, in the formulation of a the kind of truth conditional
semantics on which this connection is based, it is necessary to
determine the range of the truth valuations.
The designer of a language (in the case of the formal languages with
which Carnap was principally concerned) must chose what are the
possibilities under which the truth values of sentences are to be
determined.
Choice of language is for Carnap pragmatic, but the metaphysician may
argue that a language will only be useful if the choice of possible
world embodied corresponds with reality, so that the analytic
sentences are just those which express, in his metaphysics, the
necessary propositons.

In this context it seems natural to adopt the following
epistemological principle, which concerns what kind of justification
we would wish to receive to convince us of the truth of a proposition.
If the proposition is thought to be contingent, then we should expect
some evidence from observation of how the world in fact is.
If the proposition is thought to be necessary, then we should expect
grounds for believing in its truth or falsity which are independent of
particular facts about the world we live in.

The former kind of justification is traditionally called \emph{a
  posteriori} and the latter \emph{a priori}, which terms mean ``from
behind'' and ``from before'', a justification after empirical or
before empirical observation respectively.

The reasoning is simple.
The truth value of a necessary proposition is independent of which
possible work is actual, whereas the truth value of a contingent
proposition depends essentially on the actual world.
Information about the actual world is relevant in one case but not in
the other.

Note that we are discussing here the justification of a claim (as to
the truth or falsity of some proposition), not the
question of how we or anyone else came to believe the claim, and we
are not supposing that such a justification must be conclusive.

The connection we are discussing is that between the epistemic status
of a proposition and its modal status, between the kinds of
justification we accept and whether the proposition is necessary or
contingent.
The suggestion is that we should consider necessary propositions to
require an \emph{a priori} justification, and a contingent proposition
to require a justification \emph{a posteriori}.
What justification we look for is a pragmatic choice.

}%ignore


\subsection{Quine}

Having studied Russell's \emph{Principia Mathematica} even as an undergraduate, 

\subsection{Kripke}
The following table gives a chronology.

\begin{table}[h]
\begin{center}
\begin{tabular}{|l l p{4.5cm}|}
\hline
585 BC & Thales & Mathematics\\
428-348 BC & Plato & The Theory of Forms\\
384-322 BC & Aristotle & Essential v. Accidental \\
& & Necessary v. Contingent\\
& & Demonstrative v. Dialectical\\
1632-1704 & Locke & on trifling propositions\\
1646-1716 & Leibniz & \\
1711-1776 & Hume & relations between ideas v. matters of fact \\
1724-1804 & Kant & the synthetic, \emph{a priori} \\
1781-1848 & Bolzano & the method of variations \\
1845-1918 & Cantor & set theory \\
1848-1825 & Frege & Begriffsschrift, sinn and bedeutung \\
1848-1825 & Russell & the theory of types \\
1871-1953 & Zermelo & the cumulative hierarchy \\
1889-1951 & Wittgenstein & logical truths as tautologies \\
1901-1983 & Tarski & definition of truth \\
1891-1970 & Carnap & logical syntax, the method of intensions and extensions\\
1908-2000 & Quine & against the analytic/synthetic distinction and modal logics, holism\\
1940- & Kripke & separating analyticity, necessity and a priority via rigid designators\\
\hline
\end{tabular}
\caption{Development of the Analytic/Synthetic distinction}
\label{tab:AnalyticSynthetic}
\end{center}
\end{table}

