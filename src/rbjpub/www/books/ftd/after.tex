% $Id: after.tex,v 1.3 2009/08/05 06:47:10 rbj Exp $
\section{After Hume}\label{After}

Hume's fork abolishes a certain conception of metaphysics, and leaves a difficulty in establishing any other.

What it leaves is something that looks rather like science, falling into two parts.
One part contains only matters devoid of real content, albeit including the whole of mathematics, and the other contains all opinions of the empirical world, which is obtained by guesswork based on sensory impressions.

Not many philosophers, or scientists find this a very attractive position.
Much philosophical reaction to this seeks to refute the general scepticism in Hume's position, but the refutation of his empirical scepticism is more of benefit to science than to philosophy.
The significance to philosophy of Hume's fork is more acute in relation to those special kinds of knowledge, beloved particularly to Plato, which are the truest subject matter of philosophy, and were later to be known as metaphysics.
Positivism in that conception which is first fully realised in Hume, is an anti-philosophical philosophy, in which philosophy gives up its own true ground yielding all true knowledge to science.

It is understanding that many philosophers will react specifically against this central feature of positivism,
and in this section we will consider some of these reactions.
These we consider in three aspects.
\begin{enumerate}
\item
Firstly there arises in the ongoing dialectic, further refinement of our knowledge of exactly where the fundamental line is drawn.
\item
Secondly there are challenges to and reaffirmation of the idea that a single dichotomy is involved.
\item
Finally, coupled with the previous two there are various ways of reviving and of dismissing the possibility of some kind of metaphysics.
\end{enumerate}

\subsection{Kant}

The first challenge we consider came from Kant, who was awoken from his dogmatic slumbers by Hume, and was moved to reject the unity of the triple dichotomy.
