% $Id: intro.tex,v 1.18 2015/04/23 09:58:08 rbj Exp $

\chapter{Introduction}\label{IntroductionA}

This book consists of {\it constructive} or {\it synthetic} philosophy.
By that I mean, a philosophical dissertation whose purpose is to articulate:

\begin{itemize}
\item an idea or objective, or complex of related ideas, to be realised
\item a conceptual framework in which that idea can be made more precise
\item a description of requirements to be satisified in realising the idea
\item an architectural description of a way in which those requirements might be realised
\item some reasoned grounds for believing that the architecture described will meet the specified requirements
\end{itemize}

As philosophy the book belongs to the {\it analytic} tradition, characterised by its use of reason to seek understanding.
In describing it as {\it synthetic} I highlight my intention to talk about, in a constructive manner, not what is, but what might be.
This I hope will give the book a more dynamic character than is typical of philosophical tomes, a sense of moving forward with a little urgency rather than a contemplative meandering in which minutiae are scrutinised with a rigorous diligence born of belief in the value of all knowledge.

To that end many matters which might otherwise have been discussed in an introductory chapter are here swept aside for an immediate first cut at stating a project goal and refining it to a collection of requirements.
To bring home how philosophy could possibly be constructive in the manner required, the project goal will come from a philosopher and the requirements will arise largely from considering how the views of subseqent philosophers have influenced our reading of that goal, only in the final stages of that history drawing up the ideas of mathematical logicians and computer scientists.

\section{Beginning with Leibniz}

Our project is (yet another) inspired by Leibniz, at least in part.

The particular part of Leibniz's work which we take up is his idea of a method of resolving or answering all future disagreements in philosophy and science by codifying all our knowledge in a universal formal language (whch he called the {\it lingua characteristica} and then {\it calculating} (aided by calculuting machines) the answer to questions, by an infaliible method which he called his {\i calculus ratiocinator}.

It seems likely that Leibniz was lead to his conception of a {\it calculus ratiocinator} by discovering in his youth a method (what we would now call am algorithm) for deciding the truth of ``categorical propositions'', a form of predication central to Aristotle's logic.
Believing the class of propositions which fell within the scope of his method to be universal, or at least, sufficient to express scientifig knowledge and any conclusions we might derive from it, Leibniz perceived the possibility of resolving both philosophical and scientific issues and problems in an efficient an reliable way by algorithmic, preferably mechanical, computation.

Leibniz was mistaken on important points as a result of which the principal aims of his project we now know to be unacheivable without refinement and qualification.
Nevertheless, subject to such refinements, it is Leibniz's project we seek here to revive.

More cautiously stated I therefore provide here a first statement of objectives clearly related to this early work of Leiniz.

The principal objective is therefore to provide a context in which:

\begin{itemize}
\item most logical, mathematical, scientific and factual knowledge can be formalised
\item reasoning from that body of knowledge can be substantially automated, providing reliable establishment of the results obtained
\end{itemize}

Logic is central to these ideas of Leibniz, it is Aristotle's logic on which Leibniz is building.
 


\section{Hume's Fork}

``Hume's fork'' classifies ``the objects of Human reason or enquiry'' as relations between ideas or matters of fact.
One of the characteristics which distinguish these two kinds of knowledge is {\it certainty}.
Reasoning about relations between ideas, notably in mathematics, yields certain truths, reasoning about matters of fact 


\begin{itemize}
\item The system will be conncerned only with analytic truths
\end {itemize}

\chapter{Introduction}\label{IntroductionB}

This book of philosophy is written in the context of changes in the human
condition, mediated by the onward march of information technology, which
might be expected to have a profound impact on the character of philosophy
and to which philosophy may be expected to contribute.
It aims to reflect upon the nature of these changes and to offer a contribution
to their realisation.

Since the invention of the digital computer, facilitated by continuous advances
in semiconductor technology, progressively wider ranges of intellectual skills
previously confined to human beings are now subject to automation.
Almost all published knowledge is now in electronic form, and the ways in which
computers are able to work with this knowledge are rapidly advancing.
The time will come when machines will be generally accepted as \emph{knowing}
and as \emph{reasoning} from that knowledge.

Such changes in the world test understanding of our languages, and language
evolves to deal with them.
If it is uncertain today whether inaminate machines might properly be said to
have knowledge and to be intelligent, it neverless seems likely that as our
interactions with machines and their contribution to our knowledge increasingly
match or surpass what we expect from humans, then the relevant parts of our
language will evolve to encompass them.

In this evolving intellectual landscape, there is much scope for philosophical
analysis of language, but of greater interest here is not a passive analysis but
an active engagement in the development of language, knowledge and reason, to take
best advantage of the opportunities presented.

My aims here are in some respects similar to those addressed by A.M.~Turing
in his paper ``Computing Machinery and Intelligence''.
In that paper, while discussing machine intelligence, Turing avoids questions
about meanings by inventing his ``imitation game''.
The imitation game, later dubbed the ``Turing test'', focusses the evaluation of
machine intelligence upon the possibility of machines successfully emulating humans.

Many workers in the field of ``Artificial Intelligence'' seek to emulate
human capabilities much as they are found in people.
But the history of computing shows that automation of human intellectual labour
usually involves a substantial transformation in methods and results.

What will the impact of these changes be on the nature of philosophy?
One answer to that question which I present here is that a new kind of
constructive epistemology will be needed which is at once a philosophical
theory and a part of the cognitive architecture of the future.

In order for this philosophical dissertation to be a plausible contribution
to the cognitive architecture of our future, I propose to render the philosophy
in a way sympathetic to that possibility.
The ways in which research and development are at present conducted in the
areas most significant for the capabilities we are considering will be examined,
particularly the earliest parts of the development cycles and the ways in which
they connect with and draw on pre-existing science and technology and the
broader history of ideas.

The ways in which a philosophical treatise may contribute are unlike those in
which some definite technical contribution would.
It is not expected that the philosophical ideas articulated here be \emph{adopted} by developers
in the way in which some more concrete technology would be, but rather that
the ideas might contribute to the development of pre-competitive standards.

I begin with some preliminary discussion of, on the one hand, some relevant
ideas in the philosophical literature, and on the other, some industrially
oriented standardisation activities (in the next two sections).
Then, the needs of industry in connection with cognitive architectures will
be considered, and their connection with academic disciplines of mathematical
logic and philosophy.
The contention is that this problem domain is more demanding of philosophical
context, or will benefit from more substantive philosophical context, than any
prior technology, because of the special difficulties arising when reasoning
about knowledge.

In the development of information technology, there is continual contention between
the adoption of open standards by large numbers of players, and a contest among
proprietary standards.
A dominant market player will often strengthen his position by retaining proprietary
standards while smaller forces will either try to interface with those standards
or combine with others to compete, which will often be done through open standards.
Subtle variants of this pattern are common.

The philosophy here will be aimed at supporting the development of open standards,
partly because there is a kind of agenda here.

\section{Philosophical Predecessors}

I think of the project here described as a sucessor to the project which
was central to the philosophy of Rudolf Carnap, and as a contemporary ancestor
of a project to which Leibniz dedicated much energy.
Some core features of the philosophy of David Hume provided keystones in
the architecture which Carnap devised.
This first sketch will connect with the ideas of these three philosophers.
Subsequent chapters will make a wider range of historical connections
as the various aspects of the project are examined more closely.

The parts of Leibniz's work which are of interest here are his
\emph{lingua characteristica} and \emph{calculus ratiocinator} and the
rationale surrounding them.
The \emph{lingua characteristica} was intended to be a universal formal language
in which the whole of human scientific knowledge could be expressed precisely,
and the \emph{calculus ratiocinator} was an effective method for calculating
the answer to questions raised in that language.

These are precedents for one approach to the development of artificial intelligence
in which some manner of representation for knowledge is chosen and algorithms
are then sought for reasoning intelligently with that knowledge. 
They correspond to more specifically \emph{logical} approaches to AI
in which the representation of knowledge is in a language which is thought
of as a formal logic and a core component of the intelligent capability is
deductive inference.

...

It is epistemology because it is concerned with knowledge, with the language
in which it is expressed and the manner in which we can reason about it.
It is constructive rather than descriptive because it does not offer an account
of what knowledge is but a suggestion for what it might be,
This suggestion is itself intended to be constructive in the special sense
of \emph{computationally realised} or realisable, but Leibniz's contribution here is
primarily architectural rather than in the detail (though he does go so far as to
offer a decision procedure for the truth of syllogisms, and makes contributions
to the design of mechanical calculators).

One of the principal difficulties which limited the sucess of Leibniz's project
was the weakness of the Aristotelian logic with which he was working.
A revolution in logic was needed.
That revolution began in the second half of the nineteenth century and the seminal
work in this area of Gottlob Frege was to inspire Rudolf Carnap to his own
variant of constructive epistemology.

Unlike Leibniz, Carnap's project was not computational, it had not quite the same breadth.
Carnap sought to transform philosophy and science by the use of these new logical
methods.
Bertrand Russell had shown, with A.N.~Whitehead in \emph{Principia Mathematica}\cite{russell13}
how the new logical methods could be used for the formalisation of mathematical propositions and
their proofs, and he had put forward a vision of a new kind of scientific philosophy
based on these methods.
This provided a second principal impetus to Carnao, who sought to transform philosophy
and science by facilitating the application of the new logical methods in those disciplines.
The role of the philosopher was to furnish formal languages suitable for expressing
empirical theories with deductive systems allowing the consequences of the theories
to be reliably established.
He also addressed the problem of empirical confirmation of scientific theories.
Throughout Carnap is engaged in the theory of knowledge, but this is not a descriptive
analysis, but a constructive or synthetic activity.
He devises languages, with their semantics and deductive systems, and he develops
methods.
The output of his epistemology is not certain theories about what knowledge is, it
is a set of proposals about what it might be.

\section{Technological Perspective}

Having touched upon the philosophical context I want now to sketch the technological context
in which the kind of philosophy which I engage in here may be seen as an early part of
the continuous process of technology development.

In the global information technology industry huge investments are made in the development
and advancement of new technologies ultimately delivering new products for the marketplace.
A new semiconductor fabrication plant might cost as much as \$10B to build and equip, but that
is the end of a decade or more of process development.
Just one of the chips then manufactured might contain billions of transistors interconnected
to realise computational capabilities which have been developed over many years.

Because of the huge costs of developing new semiconductor technologies even the larges companies
will enter into multi-corporation collaboratives in the early part of the development process.
One aspect of the work undertaken by such collaborations will often be standards for the new
technologies under development.
At this stage the technology development can be regarded as \emph{pre-competitive}
(though the technology may still be considered to compete with other approaches to delivering
similar benefits).
The purpose of the standards developed (in some cases) is to prepare for a later competitive phase
in which these technologies are brought to market by the companies who have contributed to the
pre-competitive research and development which delivers the standards.

An example is the Hybrid Memory Cube Consortium.
Hybrid memory is a semiconductor technology in which active logic and memory chips are stacked in a single package
achieving higher density and higher performance than intetconnected planar organisation.

\begin{quote}
The goal of the Hybrid Memory Cube Consortium is to facilitate HMC Integration into a wide variety of systems, platforms and applications by defining an adoptable industry-wide interface that enables developers, manufacturers and enablers to leverage this revolutionary technology.
\end{quote}

The consortium consists of nine ``developer members'' who actively participate in the development of a set of standards and have ``equal voice and voting power on the final specifications'', and roughly 150 ``adopter members''.
Typically the developers will be those with the greater financial stake in the outcome, who are planning involvement in the manufacture of the technology in question and are therefore motivated to invest respource or capital in the development of the technology, while adopters will be those who expect to design or manufacture other products which use components delivering the inferfaces and whose product design therefore depends on the specifications.
The aim of all who expect to be involved in this new kind of technology is to prevent a proliferation of competing standards which creates additional expense and risk.

There are many ways in which precompetitive research can be undertaken, and which work best will depend upon the technologies concerned.
Software technologies have pioneered the use of ``open source'' developments in which software is developed by a collaborative team
with the intention of making the software freely available without licence fees.
Often the people and corporations involve will be those who have a commercial interest in exploiting the resulting software, who will perhaps produce other products which depend upon.
This ``open source'' methods have now spread into computer hardware (e.g. the ``openstack'' initialtive for large server farms).

If we track earlier into the development of new technologies then we are likely to find ourselved looking at academic research or collaborations involving industrial and academic partners.
Often academic research which delivers commercially exploitable technologies will result in commercial spin-offs followed by continued collaborative between the original academic institutions and their commercial spin-offs.

Yet further back we can trace \emph{ideas} to fundamental research in academic departments seemingly remote from commercial application.
Research in information technology, for example, will be located academically in departments of Computer Science, but may make extensive use of theoretical knowledge from mathematics departments including mathematical logic to which fundamental contributions have been made by philosophers (Aristotle, Leibniz, Frege, Russell, Church, Martin-Lof).


\chapter{Old Introduction}\label{Old Introduction}

My aim in this volume is to weave together and project into the future two threads disentangled from the web of history.

The first of these concerns the precursors, origins, development and reinvigoration of positive philosophy,
a tendency in philosophy originating with David Hume and August Comte.

The second concerns the automation of reason, an enterprise now associated with cognitive science
and information technology, but with an important place in the history of ideas beginning for us
with the seminal ideas of Gottfried Wilhelm Leibniz.

\section{Positive Philosophy}

The terms \emph{positivism} and \emph{positive science} were introduced by August Comte for:
\begin{itemize}
\item a \emph{philosophy} rooted in the new science introduced by men such as Bacon, Gallileo and Newton,

and for
\item a rigorous \emph{empirical science} following Compte's understanding of their method.
\end{itemize}

Some of the principal elements of that philosophy and its reconciliation of scepticism with science
are seen in the philosophy of David Hume who is therefore often considered to be the first positivist.
Hume is known today principally for his scepticism.
He lived at a time when the sceptical writings of Sextus Empiricus had been republished
and continued to exert influence as philosophers sought to reconcile sceptical thought
with the sucesses of modern science.

Positivism is just such a reconciliation.
David Hume was himself an enthusiast for the scientific method seen in the work of Newton,
and sought to apply that method in a new and more rigorous kind of philosophy.
We may see his challenge as bringing to bear sceptical arguments against speculative
metaphysics while reconciling scepticism with newtonian science.
His response to the challenge represented by radical scepticism rested on the distinction between logical and empirical truths,
pushing into the background scepticism about deductive reasoning, while emphasising sceptical
arguments about what can conclusively be inferred from our sensory experience of the world.
His manner of reconciling modern science with those limitations was not wholly satisfactory
and this aspect of positivism continues to be problematic to tne present day and will be
a matter of concern in our reformulation of positive philosophy and science.

Positivism may therefore be seen to have grown through the mitigation of radical
Greek scepticism, particularly through Hume's reconciliation of elements of scepticism
with modern science.
The historical thread which reaches us and moves forward as positivism can
therefore be seen to have begun in the radical scepticism of Greek philosophers,
notably the academic and pyrrhonian sceptics.
These emerge from the history of Greek philosophy, which also, in the work
of Plato and Aristotle provides key elements of Hume's moderation of scepticism.

\section{The beginnigs of deductive reason}

\emph{[This doesn't belong here!]}

The history of mathematics as a theoretical discipline begins in Greece about two and half millenia ago.
Computational methods using number systems appeared in a number of prior civilisations,
but the Greeks were the first to develop mathematical theories populated by general principles
established by deductive proof.

The systematic development and application of deductive methods first appears in Greece at this time.
Reason applied in this way to the development of mathematics was prolific and reliable,
enabling the progressive development of a substantial body of mathematical theory.
Compilations of known mathematics were made as this body of knowledge grew,
of which the most comprehensive was the Elements of Euclid, the centrepiece of which
was his \emph{axiomatic method} and its application in the theory of Euclidean geometry,
taught in schools to this day.

The ancient Greeks did not confine their deductive skills to mathematics,
but their applications in other domains, for example to the metaphysics of the cosmos,
were less successful.
Outside mathematics it is more difficult to distinguish deductive reasoning from other
kinds of reasoning, partly because no body of generally accepted truths appears.
We may talk then of the failure of reason to create consensus without being sure to
what extent the methods would now be considered deductive.
It is nevertheless remarkable that the Greeks felt free to speculate about the
nature of the cosmos (for example) rather than confining their beliefs to a body
of religious or superstitious doctrine underpinned by some kind of authority.

The application of reason for theoretical purposes outside mathematics failed to
establish concensus, different schools of philosophy came up with quite distinct
systems of belief and no compiliation of, say, accepted knowledge of the cosmos,
was ever possible.
The unreliability of even deductive reason outside mathematics was exhibited
starkly by the paradoxed of Zeno, who showed the impossibility of motion by
his famous argument involving the tortoise and the hare. 
As we approach the time of Plato the failure of consensus is crystallised
in the mutually contradictory philosophies of Parmenides and Heraclitus,
the former holding that nothing changes, and the latter that nothing stays the same.

\section{The Automation of Reason}

Three centuries ago the philosopher, logician, mathematician,
scientist and engineer Gottfried Wilhelm Leibniz conceived, as a young
man, a grand project, so far ahead of its time that no-one has yet
come close to realizing it.
The principal elements his project were:
\begin{itemize}
\item A \emph{universal formal language} in which all scientific
  knowledge might be expressed.
\item An \emph{encyclopedia} of science formalized in that language.
\item A \emph{calculus} or algorithm for answering question expressed
  in that language.
\item Mechanical \emph{calculators} able to perform the necessary computations.
\end{itemize}

He devoted considerable energy to progressing these ideas.
But his ideas were `pie in the sky', there was no hope of success.

Since then many important developments have taken place in philosophy,
logic, mathematics and information technology, and most of the now known
impediments to the realization of something close to his dream have
now been surmounted.

Today it is natural to think that digital electronic computers solve the mechanical side
of the project, and that the rest of the project is just \emph{software} of
various kinds (not just code).
Though Leibniz did work on the hardware (in his day mechanical
calculators), his main interest was in what we would now call software,
the algorithms and the data upon which they operated, and his approach
to this was philosophical and logical.
 
This is at once an information technology project and philosophical research.
To engineer a ``cognitive agent'', to build something which has or can
acquire knowledge, and can reason from that knowledge to solve
problems, can only be done in the context of a suitable philosophical
framework (though this might not be entirely explicit).
At its most abstract, architectural design for cognitive artifacts is
philosophy.

The kind of philosophy required to underpin such a project is not
piecemeal philosophical analysis, it is a systematic philosophical
synthesis.
It is the aim of this book to undertake such a synthesis, in such a
way that its relationship with the engineering of a certain kind of
intelligent artifact is made clear.
The engineering enterprise I call here ``the project''.
The philosophy is intended to underpin that project and to constitute
its earliest most abstract stages.

It is therefore intended to present by stages together,
\begin{itemize}
\item a conception of an engineering project one of whose goals is the
  automation of engineering design (the aims of the project),
\item some relatively abstract ideas on how that project can be
  organized and implemented (some architectural design)
\item a philosophical framework in the context of which the project
  aims and the proposed architecture can be described and in which the
  reasons for believing that the architecture might achieve the aims
  can be articulated.
\end{itemize}

In this introduction I shall survey the structure of the exposition
and sketch the principal themes.

It is in the nature of this project that it combines materials from
and seeks to interest devotees of diverse disciplines.
The various themes discussed demand varied background, much of which I
cannot hope to supply.
Though the philosophical the logical and the technological aspects are
intimately intertwined, it is hoped that readers with a particular
competence and interest in one aspect will find most of the material
of interest to him intelligible without a complete mastery of the
other aspects of the presentation.

\section{Sub-Themes}

\subsection{The Organon}

The collection of those works of Aristotle which concerned logic is
known as \emph{the organon}.
This word comes from the Greek word for a \emph{tool}.

It has this name because logic was conceived by Aristotle and by later
scholars as providing a tool, rather than as a purely academic
pursuit.
The principal application of that tool was to be demonstrative
science, the derivation of necessary truths in the various sciences
from the first principles of those sciences.
This conception of the role of logic in science is similar to that
in the more recent \emph{nomologico-deductive} model of science,
differing not so much in the role of logic as in the origin and
status of the first principles from which deduction begins.
 
The emphasis on logic as a tool might well have been unimportant
through most of the history of logic, since until recently Aristotle's
organon has dominated the field.
However, though well intentioned, Aristotle's formal logic is
inadequate for any non-trivial scientific application, and the study
of logic has remained the province of Philosophers.

With the advent of modern logic, beginning in the second half of the
nineteenth century in the work of mathematicians and philosophers such as Boole,
Frege and Pierce, there was spawned a new discipline of mathematics,
\emph{mathematical logic}, which was primarily meta-theoretical in
character.
Academic interest in logic now spans multiple disciplines, of which
the most important are philosophy, mathematics and computer science.
There are various ways in which these disciplines make use of
logic as a tool, but its use as a tool in the manner envisaged by
Aristotle, as the means whereby conclusions are drawn from various
first principles, is rare.
The formal deductive systems around which mathematical logic is centred
are its subject matter rather than its method.
Insofar as mathematical logicians provide tools for use in other
disciplines it is their metatheoretical results which find 
appication, the practice of deduction remains largely untouched.

This book belongs to a line of philosophical works (notably,
Aristotle, Leibniz and Carnap) of which the aim is
to contribute to the means for the application of formal deductive
methods in the establishment and application of knowledge.
This work is subordinate to that purpose.

The realization of that objective depends upon a philosophical
context which cannot be taken for granted, or even taken from a shelf
and dusted off.
Over the last half-century philosophy has moved in other directions,
and in order to do so has undermined the philosophical basis for the
most recent manifestation of the project in the philosophy of Rudolf
Carnap.
For a resumption of the project, new philosophical foundations are
required.

\subsection{Historical Threads}

The philosophical innovation required in support of the project
consists substantially in the re-establishment of ideas which have
fallen into disrepute.
This includes specific ideas such as the \emph{analytic/synthetic}
dichotomy, and entire philosophical perspectives or systems such as
\emph{positivism}.
Their reestablishment, at least as viable alternatives to received
opinion, will not be realized by a detailed refutation of the
arguments which displaced them.


In logic over the last 150 years and in Computing over the last 50
years there have been very many new developments of a kind which one
might expect to give philosophers pause for thought.
But new ideas in philosophy itself are very rare, most of the twists
and turns in philosophical fashion are revivals of old ideas in new
clothes.
It is in the nature of philosophical progress that it often appears
through millennia of debate as ideas are proposed, developed, disputed,
rejected and perhaps ignored for a while before rising again
re-engineered for a new philosophical climate.

If we seek afresh to understand and to advance such ideas, tracing
their development through history may be helpful in getting or in
conveying an understanding of their contemporary manifestation.

Several historical threads serve I hope to illuminate aspects of the
present proposal.
The philosophers whose ideas are touched upon in these sketches have
spent a lifetime developing the ideas.
The purpose of the sketches is to make the ideas presented here
clearer by connecting them with their historical roots.

The first of these historical threads sketches the analytic/synthetic
dichotomy and a variety of related dichotomies and concepts.
This is closely connected with the development of ideas of logical
truth, and of truth conditional semantics.

Alongside such questions relating to the establishment of
\emph{meaning}, there is the question of \emph{truth} and how it may
be established or evaluated.
Doubts about our ability to establish truths are at their most severe
in \emph{scepticism} which part of the historical background to and
continuous with \emph{positivism}, which combines elements of
scepticism with a strict attitude towards rigour in science.
Our positivism departs from its predecessors substantially in ways
which can be illuminated by consideration of those sceptical roots,
and which make it natural to think of as a graduated positive
scepticism.
This is closely connected with the question of rigour, and with the
tension between rigour and progress, in science, mathematics and
philosophy.

Though there appears to be at any point in time a trade-off between
rigour and productivity (which is perhaps easiest to see in
mathematics), at times advances are made which allow both to advance
at once, and a new balance to be achieved.
This happened in the nineteenth century for mathematics, a period of
consolidation in which standards of rigour in mathematical analysis
were transformed.
The last stages in this transformation involved the invention of
modern methods in logic, and established the possibility of the formal
derivation of mathematics.
The greater precision in locating the most abstract subject matters of
mathematics, through the agency of axiomatic set theory, resulting in
achievements of high standards of rigour which were sustained through
a century of continuous mathematical innovation.
These higher standards depended only peripherally on the new
mathematical logic.
Axiomatic set theory provided sufficient additional clarity to the
definition of mathematical concepts, that standards of rigour were
able to advance without the adoption of formal derivation as the
standard for mathematical proof.
It sufficed for a mathematical to convince his peers that a formal
derivation would be possible.

In the second half of the century, the application of digital
computers to the support of formal notations and deductive systems has
created a new domain in which for the first time formal notations and
languages are extensively used.
Stored program computers demand and support the use of such formal
languages.
Communicating unambiguously with computers becomes a prime motivation
for the use of formal notations, which provide not only the motivation
but also the kinds of support which facilitate the use of formal languages.

\subsection{Foundationalism}

If we look for the most fundamental ideas in philosophy we find an
inextricably interrelated complex of ideas belonging to several
distinct areas of philosophy.

Philosophically this work is primarily \emph{epistemological}, as one
might expect, and the philosophical system presented gives central
place to epistemology. 

Epistemology is approached in what might perhaps be described as an
instrumental manner.
Two aspects of epistemology receive no attention.
The first is the meaning of the word ``know'', and hence for some
philosophers the question of what knowledge \emph{is}, is not
considered.
Second is any aspect of how knowledge is acquired or applied which is
peculiarly human.

Instead of considering what knowledge \emph{is} we are concerned with
how knowledge might be represented and applied.

The epistemology is \emph{foundational} both in relation to logical
and empirical knowledge.
The distinction between these two, the logical and empirical, is a
keystone of the project and the philosophy.
The project realizes an general analytic method a principal feature of
which is the systematic and thorough separation of these two kinds of
knowledge.
Logical knowledge (under a broad conception of logical truth as
analyticity) is represented formally and established by deductively
sound methods.
Empirical knowledge is represented formally using abstract logical
models.
The relationship between abstract formal theories and the systems
which they model is ultimately informal.

\section{By Chapter}

\paragraph{The Project}

Chapter \ref{TheProject} describes in greater detail the project,
its objectives and the architecture proposed to realised those objectives.
This is presented by starting with an account of Leibniz's project,
tracing the history of the ideas from the 17th through to the
21st century, and then offering a revised conception of such a
project for the 21st.

The philosophical side of this history ends in debacle.
The last major philosophical proponent of a significant fragment of
the Leibnizian enterprise was Rudolf Carnap in whose philosophy the
formalization of science had a central place.
In mid 20th century, fundamental philosophical ideas with a key
place in Carnap's philosophy were repudiated by W.V.Quine.
In particular, the analytic/synthetic distinction, subject to
continuous critique by Quine since his first exposure to Carnap's
philosophy, was given a full blooded and uncompromising repudiation in
Quine's influential ``Two Dogmas of Empiricism''.
Though not aligning himself with Quine on the analytic/synthetic
dichotomy Saul Kripke then teased apart the triumvirate of concepts
which had been identified by Carnap (analyticity, necessity and the
\emph{a priori}), allowing Kripke to inaugurate a new kind of
metaphysics, and set analytic philosophy on a new track fundamentally
at odds with the philosophy of Carnap.
Thus Kripke contributed to the subsequent widely held view that the
philosophy of Rudolf Carnap (sometimes known as \emph{Logical
  Positivism}) had been decisively refuted as a result of technical
advances by two of the most highly competent and respected
philosopher-logicians of the period.

\paragraph{Fundamental Dichotomies}

Chapter \ref{FundamentalDichotomies}.

The philosophical framework I offer here is closer to the philosophy
of Carnap than to that of any other philosopher, and it is therefore
necessary in Chapter \ref{FundamentalDichotomies} to repair some of
the damage done to this point of view.
Chapter \ref{FundamentalDichotomies} considers the status of these three dichotomies, the
distinctions between analytic and synthetic, between necessity and
contingency, and that between the \emph{a priori} and the \emph{a posteriori}.
Though not always with this vocabulary, similar distinctions have been
talked of throughout almost the entire history of western philosophy.
Through this history one can see both a gradual refinement of these
concepts and also reversals.
I therefore lightly trace this history showing how Carnap's
understanding of these dichotomies was reached, and then review and
respond to some of the modern criticisms which are still held by many
to be decisive against Carnap.
In this chapter I argue that the refutation of Carnap on these matters
is one more demonstration of the contrast between the rigour of
mathematics and that of philosophy.
An illustration, I might say, of the \emph{irrationality} of philosophy.

\paragraph{Analyticity and Analysis}

In Chapter \ref{AnalyticityAnalysis} I then take the concept of
analyticity as established, and the question of its significance is
examined in greater detail. 
In philosophy the twentieth century was called \emph{the age of
  analysis}, and the principal kind of philosophy progressed by
academics was called \emph{analytic philosophy}. 
In the philosophy of Rudolf Carnap and the logical positivists the
connection between the concept of analyticity and analytic philosophy
was simple. 
Insofar as philosophy was concerned with establishing the truth of
propositions (in the manner in which mathematicians establish results
by proving theorems, or science establishes physical laws by
observation and experiment) the results of the kind of philosophical
analysis envisaged by Carnap would be analytic, though in his hands
such results play a secondary role to the articulation of methods and
the definition of languages or concepts suitable for science. 
For none of the many other conceptions of philosophical analysis which
appear in the 20th century was there such a simple connection
between analyticity and analysis. 
In this chapter we look at the relevance of analytic truth to various
kinds of analysis, both philosophical and scientific. 
This is done using a kind of philosophical thought experiment.
Suppose that we had an oracle (man or machine) which could tell us of
any conjecture whether or not it was an analytic truth? 
What impact would that have on the various kinds of analysis under
consideration?

\paragraph{Computation and Deduction}

Chapter \ref{ComputationAndDeduction}.

With the concept of analyticity and hence of \emph{deductive
  soundness} in place it is time to further refine my characterization
of the project, as being concerned, firstly, with \emph{deductively
  sound computation} and ultimately with useful approximations to
\emph{the terminator}. 
These terms are explained in Chapter
\ref{ComputationAndDeduction}, and a variety of
contemporary research trends are compared with the research thus envisaged.
Since the beginning of mathematical logic many different conceptions of
have evolved of proof and its relation to computation.
The approach envisaged here is clarified in the context of a general
discussion of these different conceptions of proof.

\paragraph{Rigour, Skepticism And Positivism}

Chapter \ref{RigourSkepticismAndPositivism}.

Carnap's philosophy had been intended to provide a way forward for
philosophy to the achievement of standards of rigour comparable to
those of mathematics (an ideal which has been held by many
philosophers over the last 2500 years), but his program had been
defeated, illustrating  just those defects that he sought to remedy.
To reinstate this idea I sketch another historical thread in Chapter \ref{RigourSkepticismAndPositivism}.
This is a history of rigour in mathematics and in philosophy.
It is a history also of those philosophers who have perceived the
rational deficit in philosophy, or in the search for knowledge more
generally, of the \emph{sceptics} of ancient Greece, and of the more
modern tradition of \emph{positivism} consisting of a kind of
mitigated or constructive scepticism in which high standards of rigour
are articulated for both \emph{a priori} and \emph{a posteriori} sciences.

\paragraph{Epistemic Retreat}

Chapter \ref{EpistemicRetreat}.

The epistemologically conservative aspects of positivism give rise in
this philosophy and architecture to the notion of ``epistemic
retreat''.
This involves an admission of general doubt, but the acceptance of
degrees of doubt, and hence a partial ordering of conjectures
indicating in a relative way how well they have been established, and
what level of confidence they may be viewed.

\paragraph{Language Planning}

Chapter \ref{LanguagePlanning}.

Carnap's pluralism, a willingness to accept the use of any
well-defined language on a pragmatic basis, gives rise to the problem
of ``language planning'', addressing for example the problems arising
from the use of multiple different languages for different aspects of
the same problem.
In Chapter \ref{LanguagePlanning} the architecture is further
developed by addressing these matters.

Carnap built on the purely mathematical foundational ideas of Frege
and Russell, but sought to apply the new logical methods to the
empirical sciences.
He believed that this required innovation on his part to admit
languages suitable for talking about the material world rather than
purely about mathematics, and in making this transition he also moved
from a purely universalistic conception of logic (in which one language
sufficed) to a pluralistic conception of the language of science.
His contribution to this pluralistic world was by way of meta-theory,
he spoke about how languages might be defined in their syntax,
semantics and proof rules, taking this to be a proper philosophical
contribution to the methodological advancement of science.
The proliferation of languages thus envisaged would demand some kind
of activity which he called ``language planning'', but did little on.

The project I envisage embraces the pluralism of Carnap, and therefore
depends upon an architecture which admits multiple languages and
permits large scale applications involving more than one language.
This connects with other initiatives in computing, notably the idea of
an ``Extensible Markup Language'' (XML) and the many related ideas
which have built up around it, including the idea of a ``semantic web''.

So far as applications to empirical science are concerned the project
we outline does not adopt the approach of Carnap.
Instead we envisage that the project provides or empirical science
(and ultimately of various engineering enterprises which depend upon it)
by regarding these as working exclusively with abstract models of the
physical world, and take the connection between such models and the
concrete world to be beyond the scope of these formal methods.

Instead of asserting the truth of an empirical theory we instead
evaluate the contexts in which it provides a useful model of aspects
of the real world, and evaluate the model in different application
domains in terms of reliability and fidelity or precision.
Carnap's notion of language planning (on which he himself said
little), is one domain in which our project might best be compared
with the W3C Semantic Web initiative.

\paragraph{Foundationalisms}

Metaphysical Positivism does not answer all philosophical problems,
but it does influence what might be considered a worthwhile
philosophical problem for further investigation.
Chapter \ref{Foundationalisms} looks at some of these.

\paragraph{The Architecture of Knowledge}

In chapter \ref{ArchitectureKnowledge} we now table an architectural
  proposal, in the form of a set of key requirements and a set of
  architectural features intended to realize those requirements, and a
  rationale for the belief that they do indeed realize them.

\paragraph{Metaphysical Positivism}

With the architectural proposal in place we return in chapter
\ref{MetaphysicalPositivism} to the philosophy.

This first involves gathering together a coherent and rounded
philosophy sufficient to underpin the proposed architecture, an
important element of this is foundational.
The second part concerns the methods supported by the architecture and
their scope of applicability.



\section{How to Read this Book}

Far be it for me to say how you, reader should proceed.
However, here are some ideas, and some observations on how I have
tried to write it which might be helpful.

I rarely myself read a book linearly from cover to cover.
The book nominally addresses a very broad range of potential readers,
many of whom will be interested in only some aspects of its subject
matter.
In writing it I have therefore tried to make it possible for readers
to reach those parts which matter most to them without having to
struggle through too great a jungle of detail which might seem to them
peripheral.

To this end I have tried to begin and end each chapter with summary
material which for some readers might suffice, and to include 
references as specific as possible in the text to prior materials upon
which an understanding may depend.

I have felt it desirable, in order to make as clear as possible the
ideas which I present, to make use of stories about the history of
various aspects of the subject matters.
Often the work of philosophers who have spent a lifetime producing an
important body of original work will be spoken of in a few sentences
which cannot be a fair account of their work, even in some special
corner.

In order to avoid misrepresentation I have used wherever possible the
device of enunciating a position which, whether it was ever held by
any philosopher or not, is useful in making a point.
One or more philosophers may be named as having inspired this
position, without going into a detailed examination (which I am
rarely best equipped to undertake) of how closely it does correspond to
the positions they in fact held.

This is a method not unrelated to that of the philosopher {\it Saul
  Kripke}\index{Kripke, Saul} in his examination of certain ideas suggested to him by the
writings of {\it Ludwig Wittgenstein}\index{Wittgenstein, Ludwig}.
The method may be adopted of connecting a philosophical problem which
is thought to be of interest in its own right with the history of the
subject.
Alternatively, for those whose interest is primarily historical and
exegetical, the process of rational reconstruction may begin in this
way, with a definite model of some aspect of a philosophers work,
which may cast light by evaluation of its similarities and differences
with the textual sources, and which may be refined in the light of
such comparisons into progressively more complex and subtle models
less readily seen as diverging from the target of analysis.

In this work it is the former motivation which concerns us
exclusively.
The second, as a kind of analysis is of interest from a meta-theoretic
point of view, but is not here practiced in anger. 

Too rigorous an attempt to distill historical illustrations into
hypothetical positions not directly attributed would however be unduly
cumbersome.
This mode of presentation is reserved for the most important and
substantial points, and much background is presented as if historical
fact, but should nevertheless be thought of in a similar manner, as so
simple an account as could at best be true in spirit, only to be
dissipated on closer inspection.