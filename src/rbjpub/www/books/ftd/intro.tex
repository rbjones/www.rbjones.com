% $Id: intro.tex,v 1.1 2009/03/01 15:40:00 rbj Exp $
\section{Introduction}\label{Introduction}

The headline purpose of this monograph is to give a contemporary re-statement and refinement of a philosophical idea which has roots as old as philosophy itself, but which received its first clear articulation at the center of the philosophy of David Hume.

Hume's version of this is sometimes called ``Hume's fork'' and begins:

\begin{quote}
``ALL the objects of human reason or enquiry may naturally be divided into two kinds, to wit, Relations of Ideas, and Matters of Fact.''
\end{quote}

and is presented more fully in section \ref{HumesFork}.

The idea is not Hume's, it speaks of a division so fundamental that signs of it can be seen throughout the history of western philosophy, but in Hume we first see this distinction, clearly stated, playing a pivotal role in an important philosophical system.
In Hume's exposition are apparent three different characterisations of a single fundamental dichotomy, one based on subject matter (or meaning), one on metaphysical or modal status, and one epistemological.
These three characterisations of the one dichotomy give rise to the title of this monograph.
That they are all characterisation of the same fundamental dichotomy, or indeed that any of them is even meaningful, remains controversial after more than 2000 years of philosophical debate. 


Progress in philosophy is often felt to be elusive or illusory.
Hume's fork, its origins and its modern manifestations, provide a tantalising combination of evidence for and against the thesis that there is progress in philosophy.
In its history we find continuing interplay with some of the most important ideas at the core of Western philosophy.

My aim in this monograph is twofold.
The first is re-statement and refinement.
The second is to give credibility to the ideas thus presented by placing them in their historic context, and in the course of this to probe the very nature of this kind of philosophy (which has latterly been called ``analytic''), what it is and what it might be.

In presenting ``history'' as part of my narrative, I am particularly aware of my own weakness as a scholar.
My aim in doing so is to find common threads which illuminate fundamental concerns.
An author with such ambitions must have a strong sense of what is important and what may or should be disregarded, which will inevitably be personal, and must hope that his intuitions in these matters bear fruit in a story which others will find illuminating.