% $Id: intro.tex,v 1.18 2015/04/23 09:58:08 rbj Exp $


\chapter{Introduction}\label{Introduction}

My aim in this volume is to weave together and project into the future certain threads disentangled from the web of history.
The forward projection is intended to be, at the same time, a kind of constructive or positive {\it philosophy} and a kind of architectural {\it engineering}.
Some say that to understand the future one must build it, others that to understand intelligence one must re-construct it.
There are elements of these insights in this enterprise.

Philosophy provides for this venture a conceptual framework, language with which to talk about:
\begin{itemize}
\item \rbjdefined{language} itself,
\item {\it knowledge} as represented in language,\\
  and
\item {\it reasoning} from a body of such knowledge to new insights entailed by that knowledge.
\end{itemize}

In order to construct an architecture for cognition one must surely have some conception of what cognition is.
In the process of synthesising and articulating such an architecture, that conception of cognition will be progressively refined.
Philosophy and engineering are therefore at least closely interrelated in these matters, and perhaps we may say that the single enterprise is at once philosophy {\it and} engineering.

There is a long history in the field of Artificial Intelligence of combining {\it science} and engineering.
It began in Computer Science, which is itself as much an engineering descipline as an empirical or theoretical science, and then became a popular method of research in psychology, before the scientific aspects of artificial intelligence were combined under the umbrella of {\it cognitive science}, built around the idea that "thinking can best be understood in terms of representational structures in the mind and computational procedures that operate on those structures."\footnote{Paul Thagard, \it Stanford Encycopaedia of Philosophy}

The architecture presented here is not offered as a solution to the problem of artificial intelligence.
Insofar as it is concerned with {\it cognition} it is with cognition stripped of psychology, in a manner similar to the way in which modern logic was pioneered by Frege as distinct from if abstracted from ``the laws of thought''.
The questions whether machines can think, or could be conscious, are not ours.
We know that machines can compute, can infer, and we are here concerned with how to do that bigger and better.
We seek to facilitate a general paradigm shift from the transformation of data by computation to the growth and application of knowlege by deduction, completely subsuming the computational.
This does perhaps stretch the notion of deduction.

The architectural proposal is intended as a contribution towards the engineering of systems which organise knowledge and facilitate reasoning which is viable even if the only real intelligence is supplied by the humans using the system, as is the case for present day interactive theorem proving systems.
It is intended to be an architecture within which machine intelligence will fit well, and is offered in the expectation that it will eventually be augmented in that way, and its characteristcs and utility will be then transformed.

Nor is this monograph intended to belong to the philosophy of AI.
Artificial intelligence is not its subject matter, and the knowledge and reasoning we are here concerned with is abstracted away from the study of human cognition, in much the same way that the work of Gottl{\"o}b Frege took the psychologism out of logic.
We are not concerned with laws of thought which govern how cognition takes place in human brains.

In this introductory chapter the core of the conceptual framework is first sketched, and then I outline the way in which this is underpinned and elaborated in the rest of the volume.

\section{Propositions, Entailment and Deduction}

Wittgenstein made much of the diversity of language, likening languages to games.
My interest here is in an idealised kind of language, abstracted from some of the ways in which our natural languages are used.

This kind of language, which I call {\it propositional} language, permits communication via {\it sentences}, which have an objective {\it meaning} which includes {\it truth conditions}.
Truth conditions are conditions which may or may not be satisfied, according to how things are.
When such a sentence is asserted then the literal or explicit content of the assertion is the claim that the way things are satisfies those conditions.
Thus the sentence ``it is raining in Oasby'' has truth conditions which are satisfied if there is at present liquid water precipitating in Oasby, and its assertion communicates that those conditions do in fact obtain, i.e. that liquid water descends from the heavens, falling in Oasby.

{\it Proposition} is the word we use for the things expressed by such a sentence, and it is knowledge represented by such propositions which we can elaborate by deductive reasoning, which is that kind of reasoning whose automation concerns the book.




