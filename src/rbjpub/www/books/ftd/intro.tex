% $Id: intro.tex,v 1.5 2009/07/14 14:26:10 rbj Exp $
\section{Introduction}\label{Introduction}

{\it First philosophy}\index{philosophy!first} is the name given by Aristotle to that part of philosophy which is concerned with the most fundamental problems, the study of which demanded and delivered the highest degree of the {\it wisdom} posessed by the greatest philosophers.

Aritotle's characterisation of {\it first philosophy} is heavily laden with value judgements from which we glean that Aristotle regarded ability in philosophy, and those who posess it, as superior to all other kinds.
{\it First Philosophy} may therefore be seen as an arrogation of superiority by philosophers over other academic disciplines.
This may be part of the reason for the disdain with which {\it first philosophy} has often been regarded during the last century.

Aristotle's writings on {\it first philosophy} are found in a volume which was entitled by later editors {\it Metaphysics}\index{metaphysics}, and that name has become associated with certain aspects of {\it first philosophy} which philosophers can regard as their own inner sanctum without presuming upon other disciplines.
Metaphysics in its purest form is the philosophical study of what ultimately there is.
Though metaphysics may be thought too far removed from more practical disciplines to impinge directly upon them, its basis has been disputed throughout history, and this dispute has provided a continuous historical dialextic between those philosophers who believe that there is important knowledge of this kind to which philosophers have best access, and those who doubt the reality of knowledge beyond that which may be discovered by scientific methods, belonging to physics itself rather than metaphysics, or to some other scientific discipline.

In ancient Greece the most extreme opponents of metaphysicians were the sceptics, who doubted all.
In modern times, conspicuously with Hume, a synthesis, or principled compromise, was realised between scepticism and dogmatism admitting scientific knowledge but rejecting metaphysics.
The dialectic then continued, not now between large scale dogmatic systems and total scepticism, but between rationalists who sought to justify metaphysics and positivists who continued to deny it.

This work aims to progress this historical dialectic, attempting another synthesis, in which the positivist grounds for rejection of metaphysics are embraced and strengthened, but are seen to apply only against a certain conception of metaphysics, as necessary (or a priori) synthetic truths.
From an understanding of how metaphysics can be seen as intimately related to semantics, a new conception of metaphysics can be reached, and with this we may approach the identification of important metaphysical problems  which can be progressed by philosophical methods.

This is a work of {\it theoretical philosophy}, i.e. philosophy concerned with {\it knowledge}, rather than values or their implications (which topics belong to {\it practical philosophy}).

We approach this new metaphysics by first presenting the historical process restating the case against metaphysics and understanding why existing conceptions of metaphysics are defective.

\index{Hume, David!fork}
Hume's version of this is sometimes called ``Hume's fork''.
It begins:

\begin{quote}
``ALL the objects of human reason or enquiry may naturally be divided into two kinds, to wit, Relations of Ideas, and Matters of Fact.''
\end{quote}

and is presented more fully in section \ref{HumesFork}.

The idea is not Hume's, it speaks of a division so fundamental that signs of it can be seen throughout the history of western philosophy, but in Hume we first see this distinction, clearly stated, playing a pivotal role in an important philosophical system.
In Hume's exposition are apparent three different characterisations of a single fundamental dichotomy, one based on subject matter (or meaning), one on metaphysical or modal status, and one epistemological.
These three characterisations of the one dichotomy give rise to the title of this monograph.
That they are all characterisation of the same fundamental dichotomy, or indeed that any of them is even meaningful, remains controversial after more than 2000 years of philosophical debate. 

Progress in philosophy is often felt to be elusive or illusory.
Hume's fork, its origins and its modern manifestations, provide a tantalising combination of evidence for and against the thesis that there is progress in philosophy.
In its history we find continuing interplay with some of the most important ideas at the core of Western philosophy.
