% $Id: intro.tex,v 1.3 2009/04/25 10:01:44 rbj Exp $
\section{Introduction}\label{Introduction}

This is a work of {\it theoretical philosophy}, i.e. philosophy concerned with {\it knowledge}, rather than values or their implications (which topics belong to {\it practical philosophy}).
Its title comes from Aristotle who addressed this topic is a book which he called {\it metaphysics}, simply because it came after his work on physics.
We are therefore concerned in part with topics which have been called {\it metaphysics} since Aristotle.

Metaphysics is concerned with some of the most difficult problems in philosophy and has had a chequered history, often becoming and more often accused of being wild speculation or plain nonsense.
One way to view the history of philosophy is as progressing by the invention, the eradication and the revival of varieties of metaphysics.  

For a philosopher who seeks to understand the deepest mysteries there is some difficulty in avoiding nonsense and in locating those real and fundamental problems which may be called first philosophy.
A part of that enterprise consists in the elimination of pseudo-problems and unsound methods, and there is merit for the aspiring metaphysician in attending to the writing of those philosophers who have most effectively argued the case against metaphysics.

In this work, metaphysics is approached by taking some of its opponents seriously, and thereby coming to an understanding of what metaphysics cannot be.
A contemporary re-statement is provided refining a philosophical idea which has roots as old as philosophy itself, but which received its first clear articulation at the centre of the philosophy of David Hume.

\index{Hume, David!fork}
Hume's version of this is sometimes called ``Hume's fork''.
It begins:

\begin{quote}
``ALL the objects of human reason or enquiry may naturally be divided into two kinds, to wit, Relations of Ideas, and Matters of Fact.''
\end{quote}

and is presented more fully in section \ref{HumesFork}.

The idea is not Hume's, it speaks of a division so fundamental that signs of it can be seen throughout the history of western philosophy, but in Hume we first see this distinction, clearly stated, playing a pivotal role in an important philosophical system.
In Hume's exposition are apparent three different characterisations of a single fundamental dichotomy, one based on subject matter (or meaning), one on metaphysical or modal status, and one epistemological.
These three characterisations of the one dichotomy give rise to the title of this monograph.
That they are all characterisation of the same fundamental dichotomy, or indeed that any of them is even meaningful, remains controversial after more than 2000 years of philosophical debate. 


Progress in philosophy is often felt to be elusive or illusory.
Hume's fork, its origins and its modern manifestations, provide a tantalising combination of evidence for and against the thesis that there is progress in philosophy.
In its history we find continuing interplay with some of the most important ideas at the core of Western philosophy.

My aim in this monograph is twofold.
The first is re-statement and refinement.
The second is to give credibility to the ideas thus presented by placing them in their historical context, and in the course of this to probe the very nature of this kind of philosophy (which has latterly been called ``analytic''), what it is and what it might be.

In presenting ``history'' as part of my narrative, I am particularly aware of my own weakness as a scholar.
My aim in doing so is to find common threads which illuminate fundamental concerns.
An author with such ambitions must have a strong sense of what is important and what may or should be disregarded, which will inevitably be personal.
He must hope that his intuitions in these matters bear fruit in a story which others will find illuminating.
