% $Id: intro.tex,v 1.11 2012/02/14 20:42:22 rbj Exp $
\chapter{Introduction}\label{Introduction}

\section{The Project}

Three centuries ago the philosopher, logician, mathematician,
scientist and engineer Gottfried Wilhelm Leibniz conceived, as a young
man, a grand project, so far ahead of its time that no-one has yet
come close to realizing it.
The principal elements his project were:
\begin{itemize}
\item A \emph{universal formal language} in which all scientific
  knowledge might be expressed.
\item An \emph{encyclopedia} of science formalized in that language.
\item A \emph{calculus} or algorithm for answering question expressed
  in that language.
\item Mechanical \emph{calculators}.
\end{itemize}

He devoted considerable energy to progressing these ideas.
But his ideas were pie in the sky, there was no hope of success.

Since then many important developments have taken place in philosophy,
logic, and information technology, and most of the now known
impediments to the realization of something close to his dream have
now been surmounted.

Today it is natural to think that computers solve the mechanical side
of the project, and that the rest of the project is just software of
various kinds.
Though Leibniz did work on the hardware, in his day mechanical
calculators, his main interest was in what we would now call software,
the algorithms and the data upon which they operated, and his approach
to this was philosophical and logical.
 
This is at once an information technology project and philosophical research.
To engineer a ``cognitive agent'', to build something which has or can
acquire knowledge, and can reason from that knowledge to solve
problems, can only be done in the context of a suitable philosophical
framework.
At its most abstract, architectural design for cognitive artifacts is
philosophy.

The kind of philosophy required to underpin such a project is not
piecemeal philosophical analysis, it is a systematic philosophical
synthesis.
It is the aim of this book to undertake such a synthesis, in such a
way that its relationship with the engineering of a certain kind of
intelligent artifact is made clear.
The engineering enterprise I call here ``the project''.
The philosophy is intended to underpin that project and to constitute
its earliest most abstract stages.

It is therefore intended to present by stages together,
\begin{itemize}
\item a conception of an engineering project one of whose goals is the
  automation of engineering design (the aims of the project),
\item some relatively abstract ideas on how that project can be
  organized and implemented (some architectural design)
\item a philosophical framework in the context of which the project
  aims and the proposed architecture can be described and in which the
  reasons for believing that the architecture might achieve the aims
  can be articulated.
\end{itemize}

In this introduction I shall survey the structure of the exposition
and sketch the principal themes.

It is in the nature of this project that it combines materials from
and seeks to interest devotees of diverse disciplines.
The various themes discussed demand varied background, much of which I
cannot hope to supply.
Though the philosophical the logical and the technological aspects are
intimately intertwined, it is hoped that readers with a particular
competence and interest in one aspect will find most of the material
of interest to him intelligible without a complete mastery of the
other aspects of the presentation.

\section{Themes}

\subsection{The Organon}

The collection of those works of Aristotle which concerned logic is
known as \emph{the organon}.
This word comes from the Greek word for a \emph{tool}.

It has this name because logic was conceived by Aristotle and by later
scholars as providing a tool, rather than as a purely academic
pursuit.
The principal application of that tool was to be demonstrative
science, the derivation of necessary truths in the various sciences
from the first principles of those sciences. 
 
The emphasis on logic as a tool might well have been unimportant
through most of the history of logic, since until recently Aristotle's
organon has dominated the field.
However, though well intentioned, Aristotle's formal logic is
inadequate for any non-trivial scientific application, and the study
of logic has remained the province of Philosophers.

With the advent of modern logic, beginning in the second half of the
nineteenth century with mathematicians and philosophers such as Boole,
Frege and Pierce, there was spawned a new discipline of mathematics,
\emph{mathematical logic}, which was primarily meta-theoretical in
character.
Academic interest in logic now spans multiple disciplines, of which
the most important are philosophy, mathematics and computer science.
There are various ways in which these disciplines make use of
logic as a tool, but its use as a tool in the manner envisaged by
Aristotle, as the means whereby conclusions are drawn from various
first principles, is rare.

This book belongs to a line of philosophical works (notably,
Aristotle, Leibniz and Carnap) of which the aim is
to contribute to the means for the application of formal deductive
methods in the establishment and application of knowledge.
This work is subordinate to that purpose.

The realization of that objective depends upon a philosophical
context which cannot be taken for granted, or even taken from a shelf
and dusted off.
Over the last half-century philosophy has moved in other directions,
and in order to do so has undermined the philosophical basis for the
most recent manifestation of the project in the philosophy of Rudolf
Carnap.
For a resumption of the project, new philosophical foundations are
required.

\section{Historical Threads}

The philosophical innovation required in support of the project
consists substantially in the re-establishment of ideas which have
fallen into disrepute.
This includes specific ideas such as the \emph{analytic/synthetic}
dichotomy, and entire philosophical perspectives or systems such as
\emph{positivism}.
Their reestablishment, at least as viable alternatives to received
opinion, will not be realized by a detailed refutation of the
arguments which displaced them.


In logic over the last 150 years and in Computing over the last 50
years there have been very many new developments of a kind which one
might expect to give philosophers pause for thought.
But new ideas in philosophy itself are very rare, most of the twists
and turns in philosophical fashion are revivals of old ideas in new
clothes.
It is in the nature of philosophical progress that it often appears
through millennia of debate as ideas are proposed, developed, disputed,
rejected and perhaps ignored for a while before rising again
re-engineered for a new philosophical climate.

If we seek afresh to understand and to advance such ideas, tracing
their development through history may be helpful in getting or in
conveying an understanding of their contemporary manifestation.

Several historical threads serve I hope to illuminate aspects of the
present proposal.
The philosophers whose ideas are touched upon in these sketches have
spent a lifetime developing the ideas.
The purpose of the sketches is to make the ideas presented here
clearer by connecting them with their historical roots.

The first of these historical threads sketches the analytic/synthetic
dichotomy and a variety of related dichotomies and concepts.
This is closely connected with the development of ideas of logical
truth, and of truth conditional semantics.

Alongside such questions relating to the establishment of
\emph{meaning}, there is the question of \emph{truth} and how it may
be established or evaluated.
Doubts about our ability to establish truths are at their most severe
in \emph{scepticism} which part of the historical background to and
continuous with \emph{positivism}, which combines elements of
scepticism with a strict attitude towards rigour in science.
Our positivism departs from its predecessors substantially in ways
which can be illuminated by consideration of those sceptical roots,
and which make it natural to think of as a graduated positive
scepticism.
This is closely connected with the question of rigour, and with the
tension between rigour and progress, in science, mathematics and
philosophy.

Though there appears to be at any point in time a trade-off between
rigour and productivity (which is perhaps easiest to see in
mathematics), at times advances are made which allow both to advance
at once, and a new balance to be achieved.
This happened in the nineteenth century for mathematics, a period of
consolidation in which standards of rigour in mathematical analysis
were transformed.
The last stages in this transformation involved the invention of
modern methods in logic, and established the possibility of the formal
derivation of mathematics.
The greater precision in locating the most abstract subject matters of
mathematics, through the agency of axiomatic set theory, resulting in
achievements of high standards of rigour which were sustained through
a century of continuous mathematical innovation.
These higher standards depended only peripherally on the new
mathematical logic.
Axiomatic set theory provided sufficient additional clarity to the
definition of mathematical concepts, that standards of rigour were
able to advance without the adoption of formal derivation as the
standard for mathematical proof.
It sufficed for a mathematical to convince his peers that a formal
derivation would be possible.

In the second half of the century, the application of digital
computers to the support of formal notations and deductive systems has
created a new domain in which for the first time formal notations and
languages are extensively used.
Stored program computers demand and support the use of such formal
languages.
Communicating unambiguously with computers becomes a prime motivation
for the use of formal notations, which provide not only the motivation
but also the kinds of support which facilitate the use of formal languages.

\section{Foundationalism}

If we look for the most fundamental ideas in philosophy we find a
inextricably interrelated complex of ideas belonging to several
distinct areas of philosophy.





Philosophically this work is primarily \emph{epistemological}, as one
might expect, and the philosophical system presented gives central
place to epistemology. 

Epistemology is approached in what might perhaps be described as an
instrumental manner.
Two aspects of epistemology receive no attention.
The first is the meaning of the word ``know'', and hence for some
philosophers the question of what knowledge \emph{em}, is not
considered.
Second is any aspect of how knowledge is acquired or applied which is
peculiarly human.

Instead of considering what knowledge \emph{is} we are concerned with
how knowledge might be represented and applied.

The epistemology is \emph{foundational} both in relation to logical
and empirical knowledge.
The distinction between these two, the logical and empirical, is a
keystone of the project and the philosophy.
The project realizes an general analytic method a principal feature of
which is the systematic and thorough separation of these two kinds of
knowledge.
Logical knowledge (under a broad conception of logical truth as
analyticity) is represented formally and established by deductively
sound methods.
Empirical knowledge is represented formally using abstract logical
models.
The relationship between abstract formal theories and the systems
which they model is ultimately informal.




\section{By Chapter}

\paragraph{Chapter \ref{TheProject}}

The first stage of this, in Chapter \ref{TheProject}, is to give the
next stage of detail in describing the project, the aims and the
architecture.
This is presented by starting with an account of Leibniz's project,
tracing the history of the ideas from the $17^{th}$ through to the
$21^{st}$ century, and then offering a revised conception of such a
project for the $21^{st}$.

The philosophical side of this history ends in debacle.
The last major philosophical proponent of a significant fragment of
the Leibnizian enterprise was Rudolf Carnap in whose philosophy the
formalization of science had a central place.
In mid $20^{th}$ century, fundamental philosophical ideas with a key
place in Carnap's philosophy were repudiated by W.V.Quine.
In particular, the analytic/synthetic distinction, subject to
continuous critique by Quine since his first exposure to Carnap's
philosophy, was given a full blooded and uncompromising repudiation in
Quine's influential ``Two Dogmas of Empiricism''.
Though not aligning himself with Quine on the analytic/synthetic
dichotomy Saul Kripke then teased apart the triumvirate of concepts
which had been identified by Carnap (analyticity, necessity and the
\emph{a priori}), allowing Kripke to inaugurate a new kind of
metaphysics, and set analytic philosophy on a new track fundamentally
at odds with the philosophy of Carnap.
Thus Kripke contributed to the subsequent widely held view that the
philosophy of Rudolf Carnap (sometimes known as \emph{Logical
  Positivism}) had been decisively refuted as a result of technical
advances by two of the most highly competent and respected
philosopher-logicians of the period.

\paragraph{Chapter \ref{FundamentalDichotomies}}

The philosophical framework I offer here is closer to the philosophy
of Carnap than to that of any other philosopher, and it is therefore
necessary in Chapter \ref{FundamentalDichotomies} to repair some of
the damage done to this point of view.
Chapter \ref{FundamentalDichotomies} considers the status of these three dichotomies, the
distinctions between analytic and synthetic, between necessity and
contingency, and that between the \emph{a priori} and the \emph{a posteriori}.
Though not always with this vocabulary, similar distinctions have been
talked of throughout almost the entire history of western philosophy.
Through this history one can see both a gradual refinement of these
concepts and also reversals.
I therefore lightly trace this history showing how Carnap's
understanding of these dichotomies was reached, and then review and
respond to some of the modern criticisms which are still held by many
to be decisive against Carnap.
In this chapter I argue that the refutation of Carnap on these matters
is one more demonstration of the contrast between the rigour of
mathematics and that of philosophy.
An illustration, I might say, of the \emph{irrationality} of philosophy.

\paragraph{Chapter \ref{AnalyticityAnalysis}}

In Chapter \ref{AnalyticityAnalysis} I then take the concept of
analyticity as established, and the question of its significance is
examined in greater detail. 
In philosophy the twentieth century was called \emph{the age of
  analysis}, and the principal kind of philosophy progressed by
academics was called \emph{analytic philosophy}. 
In the philosophy of Rudolf Carnap and the logical positivists the
connection between the concept of analyticity and analytic philosophy
was simple. 
Insofar as philosophy was concerned with establishing the truth of
propositions (in the manner in which mathematicians establish results
by proving theorems, or science establishes physical laws by
observation and experiment) the results of the kind of philosophical
analysis envisaged by Carnap would be analytic, though in his hands
such results play a secondary role to the articulation of methods and
the definition of languages or concepts suitable for science. 
For none of the many other conceptions of philosophical analysis which
appear in the $20^{th}$ century was there such a simple connection
between analyticity and analysis. 
In this chapter we look at the relevance of analytic truth to various
kinds of analysis, both philosophical and scientific. 
This is done using a kind of philosophical thought experiment.
Suppose that we had an oracle (man or machine) which could tell us of
any conjecture whether or not it was an analytic truth? 
What impact would that have on the various kinds of analysis under
consideration?

\paragraph{Chapter \ref{ComputationAndDeduction}}

With the concept of analyticity and hence of \emph{deductive
  soundness} in place it is time to further refine my characterization
of the project, as being concerned, firstly, with \emph{deductively
  sound computation} and ultimately with useful approximations to
\emph{the terminator}. 
These terms are explained in Chapter
\ref{ComputationAndDeduction}, and a variety of
contemporary research trends are compared with the research thus envisaged.
Since the beginning of mathematical logic many different conceptions of
have evolved of proof and its relation to computation.
The approach envisaged here is clarified in the context of a general
discussion of these different conceptions of proof.

\paragraph{Chapter \ref{RigourSkepticismAndPositivism}}

Carnap's philosophy had been intended to provide a way forward for
philosophy to the achievement of standards of rigour comparable to
those of mathematics (an ideal which has been held by many
philosophers over the last 2500 years), but his program had been
defeated, illustrating  just those defects that he sought to remedy.
To reinstate this idea I sketch another historical thread in Chapter \ref{RigourSkepticismAndPositivism}.
This is a history of rigour in mathematics and in philosophy.
It is a history also of those philosophers who have perceived the
rational deficit in philosophy, or in the search for knowledge more
generally, of the \emph{sceptics} of ancient Greece, and of the more
modern tradition of \emph{positivism} consisting of a kind of
mitigated or constructive scepticism in which high standards of rigour
are articulated for both \emph{a priori} and \emph{a posteriori} sciences.

\paragraph{Chapter \ref{EpistemicRetreat}}

The epistemologically conservative aspects of positivism give rise in
this philosophy and architecture to the notion of ``epistemic
retreat''.
This involves an admission of general doubt, but the acceptance of
degrees of doubt, and hence a partial ordering of conjectures
indicating in a relative way how well they have been established, and
what level of confidence they may be viewed.

\paragraph{Chapter \ref{LanguagePlanning}}

Carnap's pluralism, a willingness to accept the use of any
well-defined language on a pragmatic basis, gives rise to the problem
of ``language planning'', addressing for example the problems arising
from the use of multiple different languages for different aspects of
the same problem.
In Chapter \ref{LanguagePlanning} the architecture is further
developed by addressing these matters.

Carnap built on the purely mathematical foundational ideas of Frege
and Russell, but sought to apply the new logical methods to the
empirical sciences.
He believed that this required innovation on his part to admit
languages suitable for talking about the material world rather than
purely about mathematics, and in making this transition he also moved
from a purely universalistic conception of logic (in which one language
sufficed) to a pluralistic conception of the language of science.
His contribution to this pluralistic world was by way of meta-theory,
he spoke about how languages might be defined in their syntax,
semantics and proof rules, taking this to be a proper philosophical
contribution to the methodological advancement of science.
The proliferation of languages thus envisaged would demand some kind
of activity which he called ``language planning'', but did little on.

The project I envisage embraces the pluralism of Carnap, and therefore
depends upon an architecture which admits multiple languages and
permits large scale applications involving more than one language.
This connects with other initiatives in computing, notably the idea of
an ``Extensible Markup Language'' (XML) and the many related ideas
which have built up around it, including the idea of a ``semantic web''.

So far as applications to empirical science are concerned the project
we outline does not adopt the approach of Carnap.
Instead we envisage that the project provides or empirical science
(and ultimately of various engineering enterprises which depend upon it)
by regarding these as working exclusively with abstract models of the
physical world, and take the connection between such models and the
concrete world to be beyond the scope of these formal methods.

Instead of asserting the truth of an empirical theory we instead
evaluate the contexts in which it provides a useful model of aspects
of the real world, and evaluate the model in different application
domains in terms of reliability and fidelity or precision.
Carnap's notion of language planning (on which he himself said
little), is one domain in which our project might best be compared
with the W3C Semantic Web initiative.

\paragraph{Chapter \ref{ArchitectureKnowledge}}

We now table an architectural proposal, in the form of a set of key
requirements and a set of architectural features intended to realize
those requirements, and a rationale for the belief that they do indeed
realize them.

\paragraph{Chapter \ref{MetaphysicalPositivism}}

With the architectural proposal in place we return to the philosophy.

This first involves gathering together a coherent and rounded
philosophy sufficient to underpin the proposed architecture, an
important element of this is foundational.
The second part concerns the methods supported by the architecture and
their scope of applicability.

\paragraph{Chapter \ref{DiggingDeeper}}

Metaphysical Positivism does not answer all philosophical problems,
but it does influence what might be considered a worthwhile
philosophical problem for further investigation.
This is like a prospectus for further philosophical research.

\paragraph{Chapter \ref{ClimbingHigher}}

Where does this architecture take us, why should it be implemented?
