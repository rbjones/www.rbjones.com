% $Id: intro.tex,v 1.10 2012/01/23 21:40:02 rbj Exp $
\chapter{Introduction}\label{Introduction}

Three centuries ago the philosopher, logician, mathematician,
scientist and engineer Gottfried Wilhelm Leibniz conceived, as a young
man, a grand project, so far ahead of its time that no-one has yet
come close to realising it.
The principal elements his project were:
\begin{itemize}
\item A \emph{universal formal language} in which all scientific
  knowledge might be expressed.
\item An \emph{encyclopedia} of science formalised in that language.
\item A \emph{calculus} or algorithm for answering question expressed
  in that language.
\item Mechanical \emph{calculators}.
\end{itemize}

He devoted considerable energy to progressing these ideas.
But his ideas were pie in the sky, there was no hope of success.

Since then many important developments have taken place in philosophy,
logic, and information technology, and most of the now known
impediments to the realisation of something close to his dream have
now been surmounted.

Today it is natural to think that computers solve the mechanical side
of the project, and that the rest of the project is just software of
various kinds.
Though Leibniz did work on the hardware, his main interest was in the
software, and his approach to this was philosophical and logical.
This is at once an information technology project and philosophical research.
To engineer a cognitive agent, to build something which has or can
acquire knowledge, and can reason from that knowledge to solve
problems, can only be done in the context of a suitable philosophical
framework.
If you want to start such a project from scratch, then you may have to
spend some time in philosophical groundwork.
At its most abstract, architectural design for cognitive artifacts is philosophy.

The kind of philosophy required to underpin such a project is not
piecemeal philosophical analysis, it is a systematic philosophical
synthesis, and it is the aim of this book to undertake such a
synthesis, in such a way that its relationship with the engineering of
a certain kind of intelligent artifact is made clear.
The engineering enterprise I call here ``the project''.
The philosophy is intended to underpin that project and to constitute
its earliest most abstract stages.

It is therefore intended to present by stages together,
\begin{itemize}
\item a conception of an engineering project one of whose goals is the
  automation of engineering design (the aims of the project),
\item some relatively abstract ideas on how that project can be
  organised and implemented (some architectural design)
\item a philosophical framework in the context of which the project
  aims and the proposed architecture can be described and in which the
  reasons for believing that the architecture will achieve the aims
  can be articulated.
\end{itemize}

The first stage of this, in Chapter \ref{TheProject}, is to give the
next stage of detail in describing the project, the aims and the
architecture.
This is presented by starting with an account of Leibniz's project,
tracing the history of the ideas from the $17^{th}$ through to the
$21^{st}$ century, and then offering a revised conception of such a
project for the $21^{st}$.

The philosophical side of this history ends in debacle.
The last major philosophical proponent of a significant fragment of
the Leibnizian enterprise was Rudolf Carnap in whose philosophy the
formalisation of science had a central place.
In mid $20^{th}$ century, fundamental philosophical ideas with a key
place in Carnap's philosophy were repudiated by W.V.Quine.
In particular, the analytic/synthetic distinction, subject to
continuous critique by Quine since his first exposure to Carnap's
philosophy, was given a full blooded and uncompromising repudiation in
Quine's influential ``Two Dogmas of Empiricism''.
Though not aligning himself with Quine on the analytic/synthetic
dichotomy Saul Kripke then teased apart the triumvirate of concepts
which had been identified by Carnap (analyticity, necessity and the
\emph{a priori}), allowing Kripke to inaugurate a new kind of
metaphysics, and set analytic philosophy on a new track fundamentally
at odds with the philosophy of Carnap.
Thus Kripke contributed to the subsequent widely held view that the
philosophy of Rudolf Carnap (sometimes known as \emph{Logical
  Positivism}) had been decisively refuted as a result of technical
advances by two of the most highly competent and respected
philosopher-logicians of the period.

The philosophical framework I offer here is closer to the philosophy
of Carnap than to that of any other philosopher, and it is therefore
necessary in Chapter \ref{FundamentalDichotomies} to repair some of
the damage done to this point of view.
Chapter 3 considers the status of these three dichotomies, the
distinctions between analytic and synthetic, between necessity and
contingency, and that between the \emph{a priori} and the \emph{a posteriori}.
Though not always with this vocabulary, similar distinctions have been
talked of throughout almost the entire history of western philosophy.
Through this history one can see both a gradual refinement of these
concepts and also reversals.
I therefore lightly trace this history showing how Carnap's
understanding of these dichotomies was reached, and then review and
respond to some of the modern criticisms which are still held by many
to be decisive against Carnap.
In this chapter I argue that the refutation of Carnap on these matters
is one more demonstration of the contrast between the rigour of
mathematics and that of philosophy.
An illustration, I might say, of the \emph{irrationality} of philosophy.

With the concept of analyticity and hence of \emph{deductive
  soundness} in place it is time to further refine my characterisation
of the project, as being concerned, firstly, with \emph{deductively
  sound computation} and ultimately with useful aproximations to \emph{the terminator}.
These terms are explained in Chapter
\ref{ComputationAndDeduction}, and a variety of
contemporary research trends are compared with the research thus envisaged.
Since the beginning of mathematica logic many different conceptions of
have evolved of proof and its relation to computation.
The approach envisaged here is clarified in the context of a general
discussion of these different conceptions of proof.

Carnap's philosophy had been intended to provide a way forward for
philosophy to the achievement of standards of rigour comparable to
those of mathematics (an ideal which has been held by many
philosophers over the last 2500 years), but his programme had been
defeated, illustrating  just those defects that he sought to remedy.
To reinstate this idea I sketch another historical thread in Chapter \ref{RigourSkepticismAndPositivism}.
This is a history of rigour in mathematics and in philosophy.
It is a history also of those philosophers who have perceived the
rational deficit in philosophy, or in the search for knowledge more
generally, of the \emph{sceptics} of ancient Greece, and of the more
modern tradition of \emph{positivism} consisting of a kind of
mitigated or constructive scepticism in which high standards of rigour
are articulated for both \emph{a priori} and \emph{a posteriori} sciences.

Having more fully restated and defended the relevant parts of Carnap's
philosophy, I add further to the characterisation of the project
features which correspond to his philosophy and the positivist tradition.
Carnap's pluralism, a willingness to accept the use of any
well-defined language on a pragmatic basis, gives rise to the problem
of ``language planning'', addressing for example the problems arising
from the use of multiple different languages for different aspects of
the same probelm.
In Chapter \ref{EpistemicRetreat} the architecture is further
developed by addressing these matters.
The epistemologically conservative aspects of positivism give rise in
this philosophy and architecture to the notion of ``epistemic
retreat''.
This involves an admission of general doubt, but the acceptance of
degrees of doubt, and hence a partial ordering of conjectures
indicating in a relative way how well they have been established, and
what level of confidence they may be viewed.

Carnap built on the purely mathematical foundational ideas of Frege
and Russell, but sought to apply the new logical methods to the
empirical sciences.
He believed that this required innovation on his part to admit
languages suitable for talking about the material world rather than
purely about mathematics, and in making this transition he also moved
from a purely univeralistic conception of logic (in which one language
sufficed) to a pluralistic conception of the language of science.
His contribution to this pluralistic world was by way of metatheory,
he spoke about how languages might be defined in their syntax,
semantics and proof rules, taking this to be a proper philosophical
contribution to the methodological advancement of science.
The proliferation of languages thus envisaged would demand some kind
of activity which he called ``language planning'', but did little on.

The project I envisage embraces the pluralism of Carnap, and therefore
depends upon an architecture which admits multiple languages and
permits large scale applications involving more than one language.
This connects with other intiatives in computing, notably the idea of
an ``Extensible Markup Language'' (XML) and the many related ideas
which have built up around it, including the idea of a ``semantic web''.

So far as applications to empirical science are concerned the project
we outline does not adopt the approach of Carnap.
Instead we envisage that the project provides or empirical science
(and ultimately of various engineering enterprises which depend upon it)
by regarding these as working exclusively with abstract models of the
physical world, and take the connection between such models and the
concrete world to be beyond the scope of these formal methods.

Instead of asserting the truth of an empirical theory we instead
evaluate the contexts in which it provides a usefull model of aspects
of the real world, and evaluate the model in different application
domains in terms of reliability and fidelity or precision.
Carnap's notion of language planning (on which he himself said
little), is one domain in which our project might best be compared
with the W3C Semantic Web initiative.

There is an intimate mutual dependency between the philosophical ideas
which are at stake and the automation of reason.
The philosophy provides a context in which an approach to the
automation of reason and its applications can be understood.
It also promotes methods, capable of making philosophy and empirical
science deductively rigorous, whose practical feasibility depends upon
advances in the automation of reason.

Having outlined the broader philosophical context, Chapter 5 then
looks in greater detail at the philosophies of Leibniz and Carnap,
principally on those aspects most significant for their formalisation projects.
The aim here is to make clear the relationship between the
philosophical ideas and these projects.
Chapter 6 considers related research and development which has taken
place in Computer Science and other disciplines since the formulation
of Carnap's ideas, and I then begin to put together a new
philosophical framework offered as progressing the ideas of Leibniz
and Carnap, together with a new conception of the project.

