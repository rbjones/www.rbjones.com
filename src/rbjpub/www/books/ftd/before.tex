%$Id: before.tex,v 1.9 2012/11/01 15:59:57 rbj Exp $

\section{Before Hume}\label{Before}

The distinctive place of Hume in the history of ``the fork'' which
bears his name is not a mark of his priority in making the
distinction, but rather of an important point in its development.
Hume provided a concise identification of the right distinction (I now suggest)
and gave this distinction a central place in his philosophy.
He gives evidence (in what is are now regarded as Hume's various
sceptical doctrines) of a sound intuitive grasp of the scope and
limitations of logical truth and deductive inference.

To underpin the significance of Hume's distinction, and to see that,
notwithstanding its apparent simplicity, it is by no means simple to
arrive at, I now look at some of the most important stages in the
earlier development of the ideas.

---------------------------


The philosophers I will consider here are Socrates, Plato, Aristotle,
Descartes, Locke and Leibniz.

Most of the elements which we find in Hume's fork can be found in the
earliest of these philosophers.

The relevant ideas which these philosophers discussed include Plato's
world of ideals and of appearances, the notions of essential and
accidental predication, and of necessary and contingent truth, and
Aristotle's logic and the idea of demonstrative truth.
Leibniz is important for his conception of the scope of logic and the
possibility of arithmetisation and mechanisation.
His intended method for these depends upon a conceptual atomism which
was to exert significant influence on the philosophy of Russell and of
the Early Wittgenstein, and thus indirectly upon Carnap's conception
of logical truth.


The distinction between the logical and the empirical influenced two
major tendencies in modern philosophy, namely \emph{rationalism} and
\emph{empiricism} of which Leibniz represented the first and Hume the
second.
The distinctive feature of these tendencies has been respectively the
overestimation and underestimation of the scope of deduction,
championed by metaphysically and scientifically inclined philosophers
(or philosophically inclined scientists).
The dialogue between these has therefore served to refine the
distinction between the logical and the empirical.
These two tendencies may be seen to have been anticipated by Plato and
Aristotle, of whom Plato seems like the rationalist, and Aristotle is
closer to empiricism.
Curiously the connections with Leibniz and Hume are crossed over, it
is Plato the rationalist who seems to provide the clearer anticipation
of Hume's fork, and it is Aristotle the less rationalistic of the two
who provides the logic on which Leibniz's ideas were built.

Hume's principle description of his fork is in terms of subject
matter, the distinction between knowledge of ideas and knowledge of
the world. 
In Plato we see something quite similar, he distinguishes between the
eternally stable world of Platonic ideals, of which we can have
certain knowledge obtained by reason, and a world of appearances, in
constant flux, and of whose fleeting nature we can at best have
tentative opion based on the unreliable testimony of our senses.
Not only does the distinction of subject matter match, but the key
characteristics of these domains, that in the one we have solid
precise, reliable knowledge, and in the other, nothing of the kind,
are agreed between them.

For an understanding of the
philosophy of Socrates and Plato it may be helpful to consider first
the pre-Socratic philosophers.

\subsection{The Pre-Socratics}

The history of the analytic/\-synthetic dichotomy is connected with that
of deductive inference, which is generally held to have begun with
Greek mathematics.

It might be argued that the ability to undertake elementary
deductions is an essential part of competence in a descriptive
language.
For to know the meaning of concepts, one must also know at least the
more obvious cases of conceptual inclusion.
One cannot be said to know the meaning of the word ``mammal'' if one
does not know that all mammals are animals, and hence that any general
characteristic of animals is possessed by mammals.
The ability to draw inferences purely based on an understanding of
language does not however entail the capability to discriminate
between such \emph{deductive} inferences and inferences which are
based not merely on meanings but also on supposed facts, so no clear
grasp of what inferences are deductive need be involved.
Furthermore, competence in drawing conclusions may by entirely
unwitting, there may be not awareness of the idea of inference.

It is in Ionia%
\footnote{A part of classical Greece now belonging to coastal Anatolia,
  itself a part of Turkey.}
 in the 6th century BC that mathematics was transformed into a
 theoretical discipline, a key feature of which is the practice of
 \emph{proving} general mathematical laws (rather than simply
 recording them for general use).
At this time \emph{philosophy}, the love of knowledge, encompassed
mathematics and science as well as what we might now call philosophy.
The use of reason in mathematics was very successful, and Greek
mathematics grew into a substantial body of rigorously derived,
knowledge of which much is known to us through the thirteen books of
the Elements of Euclid (c300 BC). 
In Euclid's geometry, developed deductively using an explicitly
documented deductive \emph{axiomatic method}\index{axiomatic method},
Greek mathematics achieved a standard of rigour which was not to be
surpassed for two thousand years.

The success of deduction in mathematics was not reflected in other
aspects of the work of the pre-Socratic philosophers.
It is distinctive of Greek philosophy, beginning with the Ionians,
that religion and myth was not a dominant influence, and that
philosophers sought knowledge by rational means, rather than deferring
to any kind of authority.
Ionian philosophers, often described as cosmologists, sought unifying
principles, as a way of understanding the diversity of the world
around them. 
Different philosophers adopted different hypothesis about the ultimate
constituents of the universe.
Thales held that the world originated from water, Anaximenes from air.
Empedocles held that all matter was formed from air, water, earth and
fire.
By contrast with mathematics, Greek philosophy became a great
diversity of irreconcilable doctrines.

The philosophical search for unity and stability underlying apparent
diversity and flux yielded no universally accepted result.
This is nowhere more conspicuous than in the philosophies of
Heraclitus\index{Heraclitus} and Paremenides\index{Parmenides}, the
first asserting an external flux in which nothing remained stable and
the second that nothing changes. 
The failure of reason to resolve differences in this domain of enquiry
was underlined by the arguments of Zeno of Elea\index{Zeno of Elea},
who devised many paradoxical arguments%
\footnote{Of which the best known that of Achilles and the Tortoise.}
to reduce to absurdity the possibility of change, in support of the
philosophy of Parmenides.
In this way it is made conspicuous that apparently the same method,
reason, works very well for mathematics but fails miserably in
metaphysics.

\subsection{Two Kinds of Stability}

We can understand the search of the pre-Socratic philosophers for the
unity and stability underlying the diversity of the world by analogy
either with modern science or with mathematics.

In the case of science we seek physical laws which govern the changes
which take place in the world.
The world changes, but the laws are immutable.
In modern science the preferred way of formulating scientific laws is
using mathematics.
The scientific hypothesis is that the relevant aspects of the world
correspond more or less exactly with the structure of some
mathematical model.

The unchanging fundamental truths may therefore be sought either in empirical
scientific laws or in mathematics.

These two alternatives connect with tendencies in modern philosophy,
empiricism and rationalism, and to aspects of the two great
philosophical systems of classical Greece, those of Plato and Aristotle.
The systems of Plato and Aristotle in different ways anticipate Hume's
fork, and in the rationalist and empiricist tendencies we can see a
dialectic on the scope of deduction through which the distinction
evolved towards Hume's formulation of the fork. 

\subsection{Plato's Theory of Ideals}

Plato sought to place philosophy on as rigorous a footing as
mathematics, and he did this with a synthesis of the ideas of
Heraclitus and Parmenides which recognized two distinct domains of
rational discourse.
Mathematics succeeded because it reasoned deductively about abstract ideas


In Plato Hume's fork is anticipated as a distinction between two \emph{worlds}, the world of \emph{ideal forms}\index{ideal forms} and the world of \emph{appearances}\index{appearances}.
Thought of in this way the distinction is about subject matter, 


\position{Distinction}{the \emph{world of forms} and \emph{world of appearances}}

\position{Distinction}{the \emph{a priori} (knowledge) and \emph{a posteriori} (opinion)}

\position{Distinction}{\emph{essential} and \emph{accidental} predication}

\position{Connection}{\emph{definition} and \emph{essence}}

\subsection{Aristotle's Logic and Metaphysics}


\subsection{Rationalist Philosophy}

\subsection{British Empiricism Before Hume}\index{Empiricism}

Hume was one of a line of British philosophers who emphasized the role of experience in the acquisition of knowledge.
Important figures in this tradition were Bacon\index{Bacon}, Hobbes\index{Hobbes} and Locke\index{Locke}.

\subsection{Leibniz}

Leibniz conceived of the project to which this book is devoted.
We are concerned here with just one aspect of his work, which is his contribution to the ideas leading to Hume's fork.

Through most of the intellectual history right up to the $20^{th}$ century Aristotle is a dominating influence on thinking about all aspects of Hume's fork.
Some of the influence of Aristotle is negative, his subject-predicate analysis of propositions and his syllogistic conception of logic were both to narrow and adherence to Aristotle's ideas may have inhibited the development of logic sufficient for our purposes.

Leibniz conceived of the idea that reason might be automated as a young man, and pursued this project for the rest of his life.
His insight was specifically about how the Aristotelian syllogism could be automated through arithmetisation, thus anticipating a method which was to become famous in logic when applied by G\"odel in his incompleteness results.

Within this context the logical and metaphysical ideas which underpinned and provided technical substance to his project are important.

Like Aristotle, Leibniz did have the concepts of Necessary and Contingent truth, he also had the concepts of \emph{a priori} and \emph{a posteriori} truth which were closely connected, as in Hume.
There is also a connection with semantics, similar to that in Aristotle, through their role in \emph{a priori} proof and hence in the establishment of necessary truths.
To this is added a precise description of a mechanizable process of analysis whereby necessary truths might be established, which differs greatly in character from any previous work in logic.

Though Leibniz's work in this area (by contrast with his work on the calculus) exerted little influence in his time, but proved an inspiration for some of the leading figures in the development of logic in the $19^{th}$ and $20^{th}$ centuries, including Gottlob Frege and Bertrand Russell.
His influence is conspicuous in Russell's\index{Russell} philosophy of logical atomism\cite{russellPLA} and Wittgenstein's\index{Wittgenstein} Tractatus Logico-Philosophicus \cite{wittgenstein1921}.

Leibniz was a \emph{rationalist} philosopher and his conception of the division which concerns us sometimes sounds radical.
He holds for example that, at least for God, all truths are necessary.
But he nevertheless does distinguish between necessary and contingent truths and, for us mere mortals the dividing line between these two falls more or less where Hume will put it.
Mathematics is necessary, science is contingent.
He also closely connects, as will Hume, necessity and \emph{a priority}.

In relation to the analytic/\-synthetic distinction, Leibniz still uses these concepts as they are used in classical Greece, to describe two different approaches to logical proof, contemporary usage of ``synthetic'' comes from Kant.
Leibniz does have something like a semantic and a proof theoretic characterization of necessary and \emph{a priori} truth.
On the proof theoretic side it is of interest not just that Leibniz considers necessary truths to be those susceptible of a priori \emph{proof}, thus implicitly distinguishing the method of verification from the manner of discovery. 
On the semantic side we have an explanation of necessity and of proof via the analysis of definitions.

Leibniz tells us that \emph{truth} is a matter of conceptual containment, the containment of the concept of the predicate in the concept of the subject.
This kind of containment we would now describe as \emph{extensional} and would not accept as a characterization of logical or necessary truth, but rather of truth in general, including contingent truths.
There is another kind of conceptual containment which applies only to necessary truths, and that may variously described as \emph{intensional}, \emph{essential} or risking circularity, as necessary or analytic.
This is the containment of \emph{meanings} and descriptions of the role of definitions in the process of proof are consistent with the view that logical truths correspond to containment of meanings rather than containment of extensions.

This distinction does not feature in Leibniz, his distinction between necessary and contingent propositions for us mortals is made in two other ways.
The first is by appeal to the omniscience of God and the limitations of man.
Men are not capable of the kind of complete comprehension of meaning which God exhibits.
Propositions which are contingent for us are those whose analysis is beyond the limits of our knowledge or analytic capability, but will nevertheless, if true, be knowable by analysis for God.
The second way of making the distinction is through the principle of ``sufficient reason''.
Logical necessities can be shown to be true by reduction to the law of identity of which self-predication is an instance.
The procedure is to analyze the predicate and subject by expanding their definitions and discarding components present in the subject but not in the predicate, until the subject and predicate turn out to be identical.
The remaining necessities arise from ``the principle of sufficient reason''\index{sufficient reason}, which tells us that nothing happens without sufficient reason, and more specifically that the world is the way it is because from the logically possible alternatives available to God, he has chosen the best.

We seem to have here two kinds of regression from the position occupied by Aristotle.
We see a reduction in the clarity of the concept of demonstrative truth, by introducing the idea that the definitions of concepts may be beyond our ken, and intelligible only to God.
This replaces the distinction between essence and accident, which is closer to the idea that in one case \emph{meaning} suffices, and in the other observation will prove necessary.

Leibniz's ideas on mechanisation of the syllogism flow from a conceptual atomism.
The word ``term'' is used in Aristotelian logic for the subject and predicate of a proposition in subject/predicate form, both of which may be thought of in more modern language as expressing concepts.
Leibniz's atomism is the idea that there are simple concepts from which all other concepts are formed in specific ways, together with the retention from Aristotle of the idea that all propositions have subject/predicate form.

In this is it important to distinguish the simple concepts from simple expressions which may be used to express them.
Leibniz atomism here is not about syntax, it is about those things which the syntax expresses, the underlying concepts.
A complex concept may nevertheless have a simple name, a complex concept is one which is defined in terms of other concepts, a simple concept is one which has no such definition.
The atomistic thesis is that there are simple concepts, and that all complex concepts have a definition directly or indirectly in terms of simple concepts.

More specifically, a complex concept can always be analyzed as a finite conjunction of literals (in todays terminology) where a literal is either a simple concept or the negation of a simple concept.

Leibniz saw that if every distinct simple concept were represented by a unique prime number, then a conjunction of concepts could be represented by the product of primes.
The fundamental theorem of arithmetic tells us that any such a product can be factorized to retrieve the primes corresponding to the simple constituents.
He proposed to represent arbitrary complex concepts by a pair of such products, i.e. a pair of numbers, of which the first number is the product of the primes representing the concepts which occur positively in the definition of the complex concept, and the second number is the product of the primes representing the concepts which occur negatively in the complex.
The same concept cannot appear both positively and negatively, or else the definition is inconsistent, so these two numbers will be relatively prime (having no common factors).
If Leibniz were correct then once the simple concepts have been identified and given unique prime numbers, all complex concepts are representable as pairs of co-prime numbers.

It is then possible to describe (quite simply) calculations which determine the truth of any proposition of the four forms which Aristotle uses in his syllogistic, (when these propositional schemes, which include variables instead of definite concepts in the subject and predicate places, are instantiated with specific concepts).







 

