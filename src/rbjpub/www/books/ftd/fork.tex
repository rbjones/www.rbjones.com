\def\rbjidfork{$$Id: fork.tex,v 1.5 2011/11/29 16:45:35 rbj Exp $$}

\chapter{Fundamental Dichotomies}

We saw in the last chapter how the most recent allotrope of Leibniz's project in the philosophy of Carnap was derailed by the rejection of some of the most fundamental tenets of Carnap's philosophy.

It is conceivable that the project could be revived on a different basis, for neither Quine nor Kripke, despite their devastating critiques actually abandoned the idea of deductive reason or the use of formal deductive systems.
My own view is that these criticisms were all fallacious and provide a startling example of the irrationality of philosophy, bolstering the case for the wholesale adoption of formality in philosophy.

In this chapter I therefore focus on the concept of analyticity and its relation to those of necessity and of the \emph{a priori}.
Before addressing specifically the criticisms of Carnap's position by Quine and Kripke, I propose to sketch the history of the evolution of these concepts over the past two and a half millenia.
I will present this as falling in two principle phases, which I characterise as concerned respectively with predecessors of this fundamental dichotomy, and refinement of our understanding of the concept.
The point of transition I suggest, is with Hume, though this is more of an expository device than a dogmatic claim.
The idea that there exists any such definite point of transition is tenuous, but the supposition helps give structure to the narrative.

It is with Hume I suggests that we find a first account of the dichotomy which is not easily seen to be in some respects defective.
This is perhaps as much to do with Hume not seeing himself as defining the distinction, but rather as drawing attention to it and placing it in a central place in his philosophy.

\section{Hume's Fork}\label{HumesFork}

Once again we see Hume's opening sentence\cite{humeECHU}:

\begin{quote}
``ALL the objects of human reason or enquiry may naturally be divided into two kinds, to wit, Relations of Ideas, and Matters of Fact.''
\end{quote}

In two paragraphs he then expands on the two categories he has introduced.
In the first instance I will talk separately about these two descriptions.

\subsection{Relations of Ideas}

\begin{quote}
``Of the first kind are the sciences of Geometry, Algebra, and Arithmetic; and in short, every affirmation which is either intuitively or demonstratively certain. That the square of the hypothenuse is equal to the square of the two sides, is a proposition which expresses a relation between these figures. That three times five is equal to the half of thirty, expresses a relation between these numbers. Propositions of this kind are discoverable by the mere operation of thought, without dependence on what is anywhere existent in the universe. Though there never were a circle or triangle in nature, the truths demonstrated by Euclid would for ever retain their certainty and evidence.''
\end{quote}

\subsection{Matters of Fact}

\begin{quote}
``Matters of fact, which are the second objects of human reason, are not ascertained in the same manner; nor is our evidence of their truth, however great, of a like nature with the foregoing. The contrary of every matter of fact is still possible; because it can never imply a contradiction, and is conceived by the mind with the same facility and distinctness, as if ever so conformable to reality. That the sun will not rise to-morrow is no less intelligible a proposition, and implies no more contradiction than the affirmation, that it will rise. We should in vain, therefore, attempt to demonstrate its falsehood. Were it demonstratively false, it would imply a contradiction, and could never be distinctly conceived by the mind.''
\end{quote}

