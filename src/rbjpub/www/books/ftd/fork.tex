% $Id: fork.tex,v 1.7 2012/02/14 20:42:22 rbj Exp $
\chapter{Fundamental Dichotomies}\label{FundamentalDichotomies}

\emph{[In this chapter, as in all chapters which are primarily
    historical, it is particularly important not to slip into too
    purely chronological an account. 
All the life is in the particular themes which the narrative is
intended to illuminate, and it is the development of these
themes which must be at all times in the foreground. 
I do not know how to achieve this.
Part of the difficulty in producing the story is that the detailed
content of these themes has to be well understood in order to find a
good way of presenting them, but one feels that the required level of
understanding can only be extracted from the history. 
]}

We saw in the last chapter how the most recent successor to
Leibniz's\index{Leibniz} project, in the philosophy of Rudolf
Carnap\index{Carnap, Rudolf}, was derailed by the rejection of some of
the most fundamental tenets of Carnap's philosophy. 

It is conceivable that the project could be revived on a different
basis, for neither Quine nor Kripke, despite their devastating
critiques, actually abandoned the idea of deductive reason or the use
of formal deductive systems.
If I was myself convinced of the soundness of those criticisms then I
might follow that course, but I am not.
It is best instead to answer them.

In this chapter I therefore focus on the concept of analyticity and
its relation to those of necessity and of the \emph{a priori}.
Before addressing specifically the criticisms of Carnap's position by
Quine and Kripke, I propose to sketch the history of the evolution of
these concepts over the past two and a half millennia.
I will present this as falling in two principal stages, first the
development of predecessors of these fundamental concepts and then the
subsequent refinement of our understanding of the concept. 
The point of transition I suggest, is with Hume, though this is more
an expository device than a dogmatic claim.
The idea that there exists any such definite point of transition is
tenuous, but the supposition helps to give structure to the narrative.

It is with Hume, I suggest, that we find a first account of the
dichotomy which is not easily seen to be in some respects defective. 
This is perhaps as much to do with Hume not seeing himself as defining
the distinction, but rather as drawing attention to it and placing it
in a central place in his philosophy. 

Before entering into the history I will sketch the technical content
of the narrative, to provide a framework in which the evolving detail
can be placed.
This is done using the ideas of Hume.

\section{Hume's Fork}\label{HumesFork}

David Hume was a philosopher of the Scottish Enlightenment.
The enlightenment was a period of ascendency in the place of reason in
the discussion of human affairs, when science had secured its
independence from the authority of church and state and had a new
confidence in its powers derived substantially from the successes of
Newtonian physics.

Hume looked upon the philosophical writings of his contemporaries and
found in them two principal kinds, an ``easy'' kind which appealed to
the sentiments of the reader, and a ``hard'' kind which trawled deeper
and appealed to reason.
This latter kind, ``commonly called'' metaphysics, was preferred by
Hume, but found nevertheless, by him, to be lacking, infested with religious 
fears and prejudices.
Hume's feelings about these aspects of philosophy were not vague
misgivings.
He had a specific epistemological criterion which he saw these
philosophical doctrines as violating.

Hume's project involves an enquiry into the nature of human reason for
the purpose of eliminating those parts of metaphysics which go beyond
the limits of knowledge, and establishing a new metaphysics on a
solid foundation limited to those matters which fall within the scope
of human understanding.

David Hume wrote his philosophical \emph{magnum opus}, \emph{A Treatise on
  Human Nature} \cite{humeTHN}\index{Hume!treatise} as a young man.
He was disappointed to find his work largely ignored and otherwise
misunderstood, and thought perhaps that his presentation had been at
fault.
To improve matters he wrote a much shorter work more tightly focussed
on the core messages which he thought of greatest importance, which he
called \emph{An Enquiry into Human Understanding}
\cite{humeECHU}\index{Hume!enquiry}.

In a central place both logically and physically in this more concise
account of his philosophy he says:

\begin{quote}
``ALL the objects of human reason or enquiry may naturally be divided
  into two kinds, to wit, Relations of Ideas, and Matters of Fact.'' 
\end{quote}

We shall see that Hume is here identifying a single dichotomy which
corresponds to all three of the distinctions which here concern us.
In his next two paragraphs he expands in turn on the kinds he has thus
introduced.

\subsection{Relations of Ideas}

\begin{quote}
``Of the first kind are the sciences of Geometry, Algebra, and
Arithmetic; and in short, every affirmation which is either
intuitively or demonstratively certain.
That the square of the hypotenuse is equal to the square of the two
sides, is a proposition which expresses a relation between these
figures.
That three times five is equal to the half of thirty, expresses a
relation between these numbers.
Propositions of this kind are discoverable by the mere operation of
thought, without dependence on what is anywhere existent in the
universe.
Though there never were a circle or triangle in nature, the truths
demonstrated by Euclid would for ever retain their certainty and
evidence.''
\end{quote}

Hume is distinctive here among empiricist philosophers in having a
broad conception of the \emph{a priori} (though he does not use that
term here), allowing notable the whole of mathematics.
In this he may be contrasted for example with Locke who allowed
only certain rather trivial logical truths to be knowable \emph{a
  priori}.
Nevertheless, Hume's conception of the \emph{a priori} remains narrow
by comparison with the rationalists, and in particular, as Hume will
later emphasize, excludes metaphysics.

\subsection{Matters of Fact}

\begin{quote}
``Matters of fact, which are the second objects of human reason, are not ascertained in the same manner; nor is our evidence of their truth, however great, of a like nature with the foregoing. The contrary of every matter of fact is still possible; because it can never imply a contradiction, and is conceived by the mind with the same facility and distinctness, as if ever so conformable to reality. That the sun will not rise to-morrow is no less intelligible a proposition, and implies no more contradiction than the affirmation, that it will rise. We should in vain, therefore, attempt to demonstrate its falsehood. Were it demonstratively false, it would imply a contradiction, and could never be distinctly conceived by the mind.''
\end{quote}

The evolution of following three dichotomies are the subject matter,
though we will find other related dichotomies which feature in the
history.

The terms which I will use to speak of them, in this chapter are:
\begin{itemize}
\item{analytic/synthetic}
\item{necessary/contingent}
\item{a priori/a posteriori}
\end{itemize}

As I shall use these terms these are divisions of different kinds of
entity, by different means.

The first is a division of sentences, understood in sufficient context
to have a definite meaning, and is a division dependent upon that
meaning.

The second is a division of \emph{propositions}, which may be
understood for present purposes as \emph{meanings} of sentences in
context.
The division is made according to whether the proposition
expressed must under all circumstances have the same truth value, or
whether its truth value varies according to circumstance.
In this we are concerned with two particular notions of necessity,
those of logical and of metaphysical necessity, the latter being
sometimes taken to be broader than the former.
A part of the role of Hume's fork in positivist philosophy is to
banish metaphysical necessity insofar as this goes beyond logical necessity.

The third is for our purposes also a division of propositions, on a
different basis.
It concerns the status of claims or of supposed knowledge of
propositions. 
It is expected that such a claim must in some way be
\emph{justified} if we are to accept it, and that the kind of
justification required depends upon the proposition to be justified.
The justification is a priori if it makes no reference to observations
about the state of the world.

\subsection{The Place of The Fork in Hume's Philosophy}

The mere statement of the fork (which we shall see, is not original in
Hume) is of lesser significance than the role which it plays in Hume's
philosophy, which serves to clarify the distinction at stake and draw
out its significance.

Hume's philosophy, like Descartes' comes in two parts of which the
first is sceptical in character, and the second constructive.
In both cases the sceptical part clears the ground for a new approach
to philosophy which is then adopted in the constructive phase.

For our present purposes we are concerned principally with the first
sceptical phase of Hume's philosophy, because of the delineation of the scope
of deductive reason, and hence of the analytic/synthetic dichotomy
which is found in Hume's sceptical arguments.
This delineation is baldly stated in Hume's first description of the
distinction between ``relations between ideas'' and ``matters of
fact'', for there Hume tells us that no matter of fact is
demonstrable.

This bald statement would by itself have little persuasive force if it
were not followed up with more detail, even though ultimately this
detail does not so much underpin the distinction as depend upon it.

Hume's further discussion begins with the consideration of what
matters of fact can be known `beyond the present testimony of 
our senses or the records of our memory'.
The inference beyond this immediate data is invariable causal, we
infer from the sensory impressions or memories to the supposed causes
of those impressions.
But these are not logical inferences, causal necessity is for Hume no
necessity at all (even less the inference from effect to cause).
Hume's central thesis that matters of fact are not demonstrable is
in this way reduced first to the logical independence of cause and
effect, and then to the distinction between deductive (and hence sound)
inference and inductive inference (whereby we infer causal
regularities and their consequences).

Given that Hume considers all inferences from senses to be based on
induction, and sees no validity in causal inference, it follows that
from information provided directly to us by the senses nothing further
can be deduced which is not simply a restatement, selection or summary
of the information itself.
Further enlightenment from this sceptical doctrine is primarily the
application of this principle to various kinds of knowledge.
In the process Hume does a certain amount of 



\section{A Contemporary Perspective}

In tracing the history of the analytic/synthetic and related
distinctions I hope to show how their various historical
manifestations relate, and to present a perspective from which the
various developments may be evaluated perhaps as plain progress,
perhaps as advances in certain respects, perhaps as involving or
constituting regress.

Such evaluations can only be made from a particular point of view, and
in this section that point of view is outlined, and related, in the
first instance to Hume's fork.





