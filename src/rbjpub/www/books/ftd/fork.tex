\def\rbjidfork{$$Id: fork.tex,v 1.4 2011/11/04 16:38:42 rbj Exp $$}

\chapter{Fundamental Dichotomies}



\section{Hume's Fork}\label{HumesFork}

Once again we see Hume's opening sentence\cite{humeECHU}:

\begin{quote}
``ALL the objects of human reason or enquiry may naturally be divided into two kinds, to wit, Relations of Ideas, and Matters of Fact.''
\end{quote}

In two paragraphs he then expands on the two categories he has introduced.
In the first instance I will talk separately about these two descriptions.

\subsection{Relations of Ideas}

\begin{quote}
``Of the first kind are the sciences of Geometry, Algebra, and Arithmetic; and in short, every affirmation which is either intuitively or demonstratively certain. That the square of the hypothenuse is equal to the square of the two sides, is a proposition which expresses a relation between these figures. That three times five is equal to the half of thirty, expresses a relation between these numbers. Propositions of this kind are discoverable by the mere operation of thought, without dependence on what is anywhere existent in the universe. Though there never were a circle or triangle in nature, the truths demonstrated by Euclid would for ever retain their certainty and evidence.''
\end{quote}

\subsection{Matters of Fact}

\begin{quote}
``Matters of fact, which are the second objects of human reason, are not ascertained in the same manner; nor is our evidence of their truth, however great, of a like nature with the foregoing. The contrary of every matter of fact is still possible; because it can never imply a contradiction, and is conceived by the mind with the same facility and distinctness, as if ever so conformable to reality. That the sun will not rise to-morrow is no less intelligible a proposition, and implies no more contradiction than the affirmation, that it will rise. We should in vain, therefore, attempt to demonstrate its falsehood. Were it demonstratively false, it would imply a contradiction, and could never be distinctly conceived by the mind.''
\end{quote}

