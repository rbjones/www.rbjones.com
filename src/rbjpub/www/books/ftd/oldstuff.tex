% $Id: oldstuff.tex,v 1.3 2012/02/14 20:42:22 rbj Exp $

\chapter{Introduction}

{\it
[
The introduction needs now to begin with some discussion of the significance of the automation of reason and of the relevance to it of philosophy.
It is then desirable to sketch the main lines of the relationship between them which will drive the subsequent detailed exposition.
The main lines are:
\begin{itemize}
\item First and foremost the ``fundamental triple dichotomy'' and its historical development.
\item Secondly, the nature of metaphysics, and certain conceptions of metaphysics which connect with the evolution of the triple dichotomy.
\begin{itemize}
\item as an approach to semantics
\item as the synthetic-\emph{a priori}
\item as meaningless
\item as absolute
\item in contrast with skepticism before and positivism after Hume
\end{itemize}
\item 
\item
\end{itemize}

However, my present conception of how the philosophical themes can be put together with the motivating purpose in a systematic and enlightening way is too sketchy for the introduction to be written.
]
}

It is part of the historical role of philosophy to provide a context for the systematic pursuit of knowledge.
Such a context, framework or system, is philosophy for philosopher and non-philosopher alike.
It needs to be compact and focussed.

\section{First Philosophy}

{\it First philosophy}\index{philosophy!first} is the name given by Aristotle to that part of philosophy which is concerned with the most fundamental problems, the study of which demanded and delivered the highest degree of the {\it wisdom} possessed by the greatest philosophers.

Aristotle's characterization of {\it first philosophy} is heavily laden with value judgements from which we glean that Aristotle regarded ability in philosophy, and those who posses it, as superior to all other kinds.
{\it First Philosophy} may therefore be seen as an arrogation of superiority by philosophers over other academic disciplines.
This may be part of the reason for the disdain with which {\it first philosophy} has often been regarded during the last century.

Aristotle's writings on {\it first philosophy} are found in a volume which was entitled by later editors {\it Metaphysics}\index{metaphysics}, and that name has become associated with certain aspects of {\it first philosophy} which philosophers can regard as their own inner sanctum without presuming upon other disciplines.
Metaphysics in its purest form is the philosophical study of what ultimately there is.
Though metaphysics may be thought too far removed from more practical disciplines to impinge directly upon them, its basis has been disputed throughout history, and this dispute has provided a continuous historical dialectic between those philosophers who believe that there is important knowledge of this kind to which philosophers have best access, and those who doubt the reality of knowledge beyond that which may be discovered by scientific methods, belonging to physics itself rather than metaphysics, or to some other scientific discipline.

In ancient Greece the most extreme opponents of meta-physicians were the sceptics, who doubted all.
In modern times, conspicuously with Hume, a synthesis, or principled compromise, was realized between scepticism and dogmatism admitting scientific knowledge but rejecting metaphysics.
The dialectic then continued, not now between large scale dogmatic systems and total scepticism, but between rationalists who sought to justify metaphysics and positivists who continued to deny it.

This work aims to progress this historical dialectic, attempting another synthesis, in which the positivist grounds for rejection of metaphysics are embraced and strengthened, but are seen to apply only against a certain conception of metaphysics, as necessary (or a priori) synthetic truths.
From an understanding of how metaphysics can be seen as intimately related to semantics, a new conception of metaphysics can be reached, and with this we may approach the identification of important metaphysical problems  which can be progressed by philosophical methods.

This is a work of {\it theoretical philosophy}, i.e. philosophy concerned with {\it knowledge}, rather than values or their implications (which topics belong to {\it practical philosophy}).

We approach this new metaphysics by first presenting the historical process restating the case against metaphysics and understanding why existing conceptions of metaphysics are defective.

\index{Hume, David!fork}
Hume's version of this is sometimes called ``Hume's fork''.
It begins:

\begin{quote}
``ALL the objects of human reason or enquiry may naturally be divided into two kinds, to wit, Relations of Ideas, and Matters of Fact.''
\end{quote}

and is presented more fully in section ?.

The idea is not Hume's, it speaks of a division so fundamental that signs of it can be seen throughout the history of western philosophy, but in Hume we first see this distinction, clearly stated, playing a pivotal role in an important philosophical system.
In Hume's exposition are apparent three different characterizations of a single fundamental dichotomy, one based on subject matter (or meaning), one on metaphysical or modal status, and one epistemological.
These three characterizations of the one dichotomy give rise to the title of this monograph.
That they are all characterization of the same fundamental dichotomy, or indeed that any of them is even meaningful, remains controversial after more than 2000 years of philosophical debate. 

Progress in philosophy is often felt to be elusive or illusory.
Hume's fork, its origins and its modern manifestations, provide a tantalizing combination of evidence for and against the thesis that there is progress in philosophy.
In its history we find continuing interplay with some of the most important ideas at the core of Western philosophy.

\chapter{Restatement}\label{Restatement}

\section{Descriptive Language}

The definitions given here are intended to limit consideration to languages which ``have'' a semantics of a certain form.
This term ``have'' is not intended in any concrete sense, but in a sense similar to notions of existence in mathematics, i.e. it is required that such a semantics exists, not that we know it, understand it, can write it down or even that it could be written down (as in ``there exists a real valued function'').

\index{language!descriptive}
The term {\it descriptive language} is used here to refer to languages in which one may make statements about the ``real'' world.
For present purposes I intend this to exclude languages whose subject matter is entirely abstract, such as set theory.

In such languages {\it indicative} sentences are those which (given certain contextual information) say something about the world, rather than, for example, conveying a command.

\index{statement}
A {\it statement} is an indicative sentence together with sufficient context to disambiguate the sentence.
Disambiguation here means determining the proposition expressed by the sentence in this context.

\index{proposition}
A {\it proposition} is the meaning of an indicative sentence which encompasses the truth conditions of the sentence.

\section{Justification}

\index{justification}
For some purposes the justification of propositions is considered, for example, in considering whether a proposition is known.
A justification is said to be {\it a priori} if it depends on no information about the actual state of the universe, and is otherwise {\it a posteriori}.

Individuals or institutions may take a view on what kind of justification is appropriate for any given proposition or class of propositions.
In that case a proposition may be said to be {\it a priori} if an {\it a priori} justification is thought the be appropriate for it.
Such classifications concern the {\it epistemic status} of propositions.

The epistemic status of a proposition in this sense is not dependent upon whether the proposition is know, or even whether it can be known, it depends only on the kind of justification which is admissible for that proposition or class of propositions.

\section{The Dichotomies}

\index{possible!world}\index{proposition}
The term ``possible world'' here refers generally to the kind of circumstance relative to which the truth conditions of a proposition are given (and may be language specific).

\index{necessary}
A proposition is {\it necessarily t} if it has truth value ``t'' in every possible world (I do not insist on there being just two truth values, the number of truth values is determined by the semantics of the language).

\index{contingent}
A proposition is {\it contingent} if it does not have the same truth value in every possible world.

\index{analytic}
A sentence is {\it analytically t} for some truth value ``t''	if the proposition it expresses (its meaning) is ``necessarily t''

\index{synthetic}
A sentence is {\it synthetic} if it is not ``analytically t'' for any truth value ``t''.

\index{a priori}
A justification for some claimed proposition is {\it a priori} if it makes no reference to any empirical observation or any contingent proposition.
Note however, that the constraint does not apply to any aspect of the justification of a statement	which is concerned exclusively with establishing its meaning (i.e. establishing which proposition is expressed by the statement)

I propose that we should accept only ``a priori'' justifications for necessary propositions, and only ``a posteriori'' justifications for contingent propositions.

Its intended that there are no substantive claims in the above, these
are all proposals for usage, except the last, which is some other
kind of proposal.

\section{Elaborations}

\subsection{Epistemic Status}

Two factors which do not influence the epistemic status of a proposition under this proposal should be understood.

Firstly, since we have defined epistemic status exclusively in terms of justification, the manner in which the truth of a proposition is apprehended or discovered is not in itself relevant to its epistemic status.
The epistemic status under this account depends exclusively upon the kind of justification which is deemed appropriate for the proposition.

Secondly, in considering the epistemic status of some proposition the manner in which the proposition is understood from some statement which we may suppose to express it is immaterial to the epistemic status of the proposition.

Under some circumstances the meaning of a statement may be difficult to ascertain, for example, if the meaning of a name is the thing named, then sentences involving that name may not be well understood by those little acquainted with the thing named.
Consequently, empirical investigation may be necessary to establish the meaning of a statement (or the proposition expressed by it), and these considerations may need to be mentioned in the context of a justification of the proposition.
Such facts are for present purposes not regarded as facts about the world appearing in the justification.
The relevant facts are those which participate in the justification of the proposition proper, not those which may participate in the identification of the proposition expressed by some statement.

\section{Some Consequences}

If we confine consideration to languages with two truth values (i.e.
in which a proposition is either true or false in every possible world)
then there are three essentially coincident dichotomies.

The identity of the distinctions is more conspicuous if the definitions
of necessary and contingent, are extended to statements in the obvious
way,


