% $Id: cd.tex,v 1.1 2012/01/23 21:40:02 rbj Exp $

\chapter{Computation and Deduction}\label{ComputationAndDeduction}

{\it[
Having established the distinction between logical and empirical
truths we now turn to a closer consideration of logical truth. 

There are two main considerations.
One is the scope and relevance of logical truth.
This should include a discussion of its place in analytic philosophy,
in mathematics, empirical science, and engineering. 
A principle aim in this discussion is both to re-emphasise, as Hume
did, the severe limits to what can be established purely by deduction,
and to show that it nevertheless is of the greatest practical
significance. 
The limitations and potential are to be made real by illustrations of
a different character to those of Hume but which are connected with
the various concerns of my own positive philosophical thinking,
theoretical and practical. 
Specifically in relation to philosophy, some discussion of the nature
of philosophical analysis making the distinction between a claim being
analytic and a claim being ``about language'', i.e. between subject
matter and epistemic status. 

The chapter will then trace the history of computation and logic,
showing the relationship between these two topics, showing the radical
chamges to them which have arisen from the advent of mathematical
logic and the digital computer, leading to the idea of sound
computation. 

Finally there will be some discussion of the step forward from ``sound
computation'' to ``crisp AI'', making use of the idea of ``the
terminator''.

There is a difficulty here about how this chapter relates to the
following one giving a history of scepticism and positivism.
I may have to reverse the order, or combine the two or chose a
different way of breaking down the material.

]}

\section{Before the Greeks}

It is generally said that the ancient Greeks invented mathematics and
deduction.

This is probably true of the self conscious use of deduction and the
development of mathematics as a theoretical discipline.

However, there are some points worth making about what happened before
that.
The man in the street, hearing the word ``mathematics'' thinks
primarily of arithmetic, and of practical skills of computation.
He may have little conception of mathematics as a theoretical
discipline.

Mathematics in that sense preceded the Greeks.
Notation for numbers, numerical computation and elementary
geometrical capabilities probably go back as far as 10,000
years BC.
Before classical Greece, the principal sources of expertise in these
practical matters were Bablyonia and Egypt.
There are at this stage methods for computation and for solving
certain kinds of algebraic problems, but these methods are presented
without any attempt at justification, there is neither a theoretical
discipline establishing results by deductive proof.

The ability to perform elementary steps of deductive reasoning is
probably coeval with descriptive language, since one cannot be said to
understand the language without grasping and being able to apply
elementary conceptual inclusions.
That is to say, for example, that if someone knows the meaning of
``azure'' then he knows that everything which is azure is also blue.
However, the first signs of deduction being used elaborately and
systematically, and the idea or deductive \emph{proof} are found at
the beginning of the classical period of ancient Greece at about 600BC.




The Greek civilisation is thought to date back to about 2,800 BC, but
mathematics as a theoretical discipline originates with the school of
the philosopher Thales in Ionia at around 600 BC.

