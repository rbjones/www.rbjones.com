\def\rbjidpositivism{$$Id: positivism.tex,v 1.1 2011/11/04 16:38:42 rbj Exp $$}

\chapter{Rigour, Scepticism and Positivism}\label{RogourSkepticismPositivsim}

In defending key elements of the philosophy with which I propose to underpin the automation of reason I have found it necessary to reject wholesale two of the most potent influences on analytic philosophy since the mid $20^{th}$ century.

If philosophy can go so badly astray, even after the modern revolution in logic transformed it, what hope is there that the ideas I champion can be better.
Is the descent into nihilism not inevitable?

This line of thought is not new, it recurs throughout the history of Western Philosophy.
In this chapter I trace some of this history.
I do this partly to make my negative conclusions seem less startling and more plausible, but principally because key features of \emph{Positive Philosophy} (the positivism!) have evolved from more extreme scepticisms over a long period of time, and are best understood in that context.

In each historical era different sources of dogma provide new principal targets for the sceptic.
In the first wholly sceptical philosophies, those of the Pyrrhonists and the sceptics of Plato's academy, a principal target was the dogmatic metaphysics of the pre-socratic philosophers, and the great syntheses of Plato and Aristotle.
When pyrrhonism first reappeared in Europe its target was the dogmas of Catholicism.
Today we have many sources of dogmatic disinformation, but the one of greatest concern here is academia in general and analytic philosophy in particular.

The thought of the radical Greek sceptics was passed into history principally in the comprehensive sceptical writings of Sextus Empiricus\cite{sextusempiricusOOP}, which began to influence modern European thought after being translated from Greek into Latin in the sixteenth century.

These revived sceptical arguments figured at first in the religious controversies of the reformation, and were deployed against Catholic dogma.
Influential among these new pyrrhoneans was Michel de Montaigne.

Once these sceptical arguments were rediscovered they were not so much adopted as moderated.
This was particularly important for science, and it was scientific philosophers such as Gassendi and Mersennes who
and the most important enduring tradition which emerged from this moderation was \emph{positivism}, which rejects pseudo-science and promotes higher standards for ``positive'' science.

In modern times the arch sceptic and the target of much anti-sceptical argument in epistemology has beed David Hume.
Greek scepticism, despite providing most of the ammunition for Hume, figures less prominently.
Hume's philosphy was a sequel to a variety of sceptical though through the sixteenth and seventeenth centuries, which begins with the religious controversies of the reformation.

In this chapter the history of scepticism and positivism is sketched out with the aim of clarifying the positivistic elements of \emph{positive philosophy} and their relevance to \emph{the project}.
