% $Id: positivism.tex,v 1.7 2015/04/23 09:58:08 rbj Exp $

\chapter{Rigour, Scepticism and Positivism}\label{RigourSkepticismAndPositivism}

{\it[
The main point of this chapter is to lead us into two features of
positive philosophy and of the architecture for crisp AI.

These are:
\begin{description}
\item[Epistemic Retreat]
At its simplest this is the idea that, instead of asserting a claim,
one retreats to describing the evidence in favor of it.
For science it is the idea that a scientific theory is to be presented
as an abstract model, and rather than asserting the ``truth'' of the
model (whatever that might mean) one makes statements about its
fidelity and utility in various circumstances.
Part of this is the idea that the assertion of a logical truth is to
be done formally, whereas empirical claims are asserted only
informally.
\item[Graduated Scepticism]
Not sure whether to call this scepticism, but the idea is that instead
of choosing between two theories and then calling one true and the
other false, one suspends judgement about truth and confines oneself
to comparative evaluation (of various kinds).
\item[Assurance and Authority]
The second principle idea, which is applicable primarily to logical
truth, is that these truths are asserted by \emph{authorities} in
which we may think of an authority as anything which may wish to
formally express an opinion about the truth of some conjecture.
Authorities may do so baldly, or may come to such an opinion in the
light of the opinions of other authorities.
In the latter case, the assertion will mention those other
authorities, and the combination of the set of authorities thus cited
and the authority expressing the opinion is regarded as a level of
assurance which fits into a lattice structure providing a partial
ordering of such assurance levels.
\end{description}

The chapter also needs to connect to contemporary reasons for
scepticism and to the reasons for doubt about philosophy arising from
the work of Quine and Tarski.

]}


In defending key elements of the philosophy with which I propose to
underpin the automation of reason I have found it necessary to reject
wholesale two of the most potent influences on analytic philosophy
since the mid 20th century.

If philosophy can go so badly astray, even after the modern revolution
in logic transformed it, what hope is there that the ideas I champion
can be better. 
Is the descent into nihilism not inevitable?



This line of thought is not new, it recurs throughout the history of
Western Philosophy. 
In this chapter I trace some of this history.
I do this partly to make my negative conclusions seem less startling
and more plausible, but principally because key features of
\emph{Positive Philosophy} (the positivism!) have evolved from more
extreme scepticisms over a long period of time, and are best
understood in that context. 

In each historical era different sources of dogma provide new
principal targets for the sceptic.  
In the first wholly sceptical philosophies, those of the Pyrrhonists
and the sceptics of Plato's academy, a principal target was the
dogmatic metaphysics of the pre-Socratic philosophers, and the great
syntheses of Plato and Aristotle. 
When pyrrhonism first reappeared in Europe its target was the dogmas
of Catholicism. 
Today we have many sources of dogmatic disinformation, but the one of
greatest concern here is academia in general and analytic philosophy
in particular. 

The thought of the radical Greek sceptics was passed into history
principally in the comprehensive sceptical writings of Sextus
Empiricus\cite{sextusempiricusOOP}, which began to influence modern
European thought after being translated from Greek into Latin in the
sixteenth century. 

These revived sceptical arguments figured at first in the religious
controversies of the reformation, and were deployed against Catholic
dogma. 
Influential among these new pyrrhoneans was Michel de Montaigne.

Once these sceptical arguments were rediscovered they were not so much
adopted as moderated. 
This was particularly important for science, and it was scientific
philosophers such as Gassendi and Mersennes who sought a constructive
scepticism compatible with the new science.
The most important enduring tradition which emerged from this
moderation was \emph{positivism}, which rejects pseudo-science and
promotes higher standards for ``positive'' science. 

In modern times the arch sceptic and the target of much anti-sceptical
argument in epistemology has been David Hume. 
Greek scepticism, despite providing most of the ammunition for Hume,
figures less prominently. 
Hume's philosophy was a sequel to a variety of sceptical thought through
the sixteenth and seventeenth centuries, which begins with the
religious controversies of the reformation.  

\section{Systematic Skepticism}

Systematic, wholesale skepticism, the doubt that there is any
true knowledge, was found in two important philosophical traditions
in post-classical Greece.
The best known of these was that associated with the name of Pyrrho of
Ellis (though it is not known how many of their ideas actually came from Pyrrho).
The other is the Academy of Plato, which, after the  death of Plato
underwent periods of radical scepticism.

The pyrrhoneans offered a comprehensive range of skeptical arguments which were later documented by Sextus Empiricus\index{Empiricus, Sextus}, which we need not examine in detail.
We need only consider a single kind of very general argument. which is argument by regress of justification.
In its simplest form it asserts that:

\begin{enumerate}
\item to count as knowledge a belief must be justified
\item a justification of a proposition consists of a number of known premises, together with an argument known to be conclusive which shows that the proposition must be true if the premises are.
\item a justification therefore depends on prior knowledge, either of the supposed evidential support or of the validity of some form of inference, and we therefore have an infinite regress and no knowledge is possible.
\end{enumerate}

Of course, this argument undermines itself, and so cannot itself yield the knowledge that knowledge is impossible.

Philosophies which are skeptical about knowledge risk incoherence, and are often accused of inconsistency.
The most naive incoherence would be to claim to know that nothing can be known.
This accusation has been levelled, but it is doubtful that any skeptic on record made that mistake.
The closest is the doctrine that we can know nothing but this single proposition, which is consistent if a little {\it ad hoc}.

There are two subtler kinds of inconsistency which are more significant for us.

The first is equivocation about the meaning of the verb ``to know'
We are concerned with ``knowing that'', knowledge of true propositions, rather than ``knowing how''.
This generally involves a true belief, and usually depends on the knower having adequate grounds for his belief, thus the formula: knowledge is justified true belief.
However, the requirements in terms of justification vary widely.
In some discourse this requirement of justification lapses, and someone may be said to ``know'' a fact simply because he has been apprised of it.
On the other extreme the requirement for justification can be exhaustive, in some contexts the required justification must be conclusive.
What counts as conclusive is also up for grabs,
In the context of a philosophical discussion a justification may need to be the possession of evidence which logically entails the candidate proposition.

Arguments in favor of radical scepticism will often implicitly assume the need for the highest standards of justification.
However, once established in this way, the resulting scepticism may be applied to all kinds of ``knowledge'' even in contexts where the required standards of justification is weak or nugatory.

One way of preventing inconsistency is to oppose dogmatism with doubt.
Thus, the skeptic responds to a claim to knowledge, not by contradicting that claim, but merely casting doubt upon it.
This leads to sceptics who seek to establish ``equipollence''\index{equipollence} a situation in which the evidence for and against a proposition are perfectly balanced.

Pyrrhonean scepticism advocates reserving judgement, keeping an open mind, rather than the dogmatic assertion of some proposition.
The pyrrhonean is sometimes portrayed as someone seeking knowledge who fails to find the certain knowledge that he seeks.
There are two elements of inconsistency or equivocation which can be observed in an account at this level.
The first is the contrast between the two objectives, on the one hand that of seeking knowledge (albeit unsuccessfully), and on the other of seeking equipollence in respect of any particular judgement.
Surely if equipollence is an objective, then this represents a prejudice against the possibility of knowledge, rather than a positive attitude of seeking knowledge?
This accusation can perhaps be deflected by presenting the search for equipollence as a way of testing a conjecture, and hence establishing the proposition by failing to show equipollence.

A second point of apparent equivocation is in the word dogmatic.
In ordinary parlance a dogmatist is someone who holds onto a fixed belief in the face of contrary evidence.
In Pyrrhonean usage one becomes a dogmatist in virtue of even the smallest departure from complete doubt.
The mere expression of a tentative belief counts as dogmatism.
The sceptical arguments however, very often depend on the demand for conclusive justification, and hence have force only against dogmatism in its more ordinary usage, rather than in the more liberal interpretation which embraces mere opinion.

When we come to {\it positivism}\index{positivism} we find a kind of mitigated scepticism in which the systematic doubt is a part of scientific method, is a way of testing hypotheses.
Elements of skepticism provide a basis for positivism and its notion of positive science, and will have an influence on our proposed ``architecture of knowledge'', so we seek here a way of removing inconsistencies and equivocations to obtain a defensible position on which to build.

In this connection we leave the Pyrrhonists to consider the graduated scepticism of the Academic skeptic {\it Carneades}.

---------------- Extract something from Skepticism ----------------


\section{positivism}

\begin{itemize}
\item Positivism as a moderation of skepticism
\item Positivism as scientific method
\item Kolakowski's characterization of positivism
\item Hume's fork
\end{itemize}

The term positivism was coined by \emph{Auguste Comte}\index{Comte, Auguste} and refers to the whole of his broad ranging philosophical thought.
Typically however the term has been used for a narrower position centering around his conception of \emph{positive science}, and has been construed as relating to a philosophical tradition many elements of which were first clear in the philosophy of David Hume.
This is the perspective on positivism which is found in the history by \emph{Kolakowski}\index{Kolakowski}\cite{kolakowskiPP}, and which we work with here.

Comte held that the human mind progresses through distinct phases, the \emph{theological} the \emph{metaphysical} and the \emph{positive}.
In the theological stage the hidden nature of things is connected with the belief in supernatural beings, in the metaphysical stage the hidden nature of things is sought without resort to the supernatural, and in the positive stage scientific understanding supersedes both theology and metaphysics through the formulation of universal laws about the phenomena which do not depend on hidden entities.
He offered his conception of positive science not as an innovation, but rather as an account of the scientific method already established by his predecessors, men such as Bacon and Galileo.




\section{Rudolf Carnap}
\subsection{Tolerance, Pluralism, Metaphysics}

These three topics are intimately interwoven in Carnap's philosophy.
I'm going to attempt an elucidation elements of Carnap's philosophy by discussing some interpretations of Carnap on pluralism which seem to me to be, in various degrees, mistaken.

I discuss the views on Carnap's pluralism of three fictitious philosophers, R, A and K.
Their positions are suggested to me by the writings of three actual philosophers, but it is not necessary for my present purposes to resolve the question of what those real philosophers actually meant, it suffices to discuss positions they might possibly have meant.

First a brief preliminary statement of some key aspects of Carnap's notion of pluralism and his principle of tolerance.
According to Carnap's intellectual autobiography \cite{carnap63,carnap63a} his tolerance was anticipated in his student days by a willingness to discuss issues with friends in a variety of ``philosophical languages'', which corresponded to distinct and incompatible metaphysical stances.
The principle of tolerance did not appear until much later, in ``logical syntax'' \cite{carnap34,carnap37}.
It is understandable that people will come away from this book with diverse and incompatible ideas of what Carnap's pluralism is.
Carnap does not actually use the word pluralism in the book.
He does enunciate his ``principle of tolerance'' in \section 17:

\begin{quote}
It is not our business to set up prohibitions, but to arrive at conventions.
\end{quote}

Which is further explained in that section as:
\begin{quote}
Everyone is at liberty to build up his own logic. i.e. his own form of language, as he wishes.
\end{quote}

I begin with R.
R claimed that he was more pluralistic than Carnap.
I reacted against that, since I didn't see that Carnap's pluralism actually excluded anything.
The feature of his pluralism which he thought beyond Carnap's was that he would accept that the same sentence could have different meanings and different truth values.
But this is also the case for Carnap.
For Carnap this would be possible by having that same sentence in two different ``language frameworks'' (which we may perhaps think of as being both the grammar of the language and the semantics in some form).
I mention this here because it contrasts with the next one.

K took a more substantial interest in Carnap's philosophy and wrote quite a bit about it.
He recognizes two interpretations of Carnap's pluralism, on the one hand as a substantial thesis, and on the other as a proposal.
However, his critique is concerned with the former.
In this case he takes Carnap to be taking a radical view entailing the indeterminacy of truth in languages like arithmetic and set theory, so that Carnap is to be understood as saying that these truths are not completely determinate but can be chosen arbitrarily.

\begin{table}[h]
\begin{center}
\begin{tabular}{|l l p{5cm}|}
\hline
585 BC & Thales & deductive mathematics \\
6C BC & Ionia & metaphysical cosmology \\
 & Zeno & paradoxes of motion \\
5C BC & sophists & man is the measure of all things \\
470-399 BC & Socrates & \\
384-322 BC & Aristotle & the organon and the metaphysics \\
428-348 BC & Plato & the theory of forms \\

c300 BC & Euclid & the axiomatic method \\
c365-270 BC & Pyrrho & scepticism \\
c180-110 BC & Carneades & graduated scepticism \\
c1288–1348 & William & \\
& of Occam & nominalism \\
 &  & \\
 &  & \\
 &  & \\
 &  & \\
 &  & \\
 &  & \\
 &  & \\
 &  & \\
\hline
\end{tabular}
\caption{Rigour, Scepticism, Positivism}
\label{tab:RigourScepticismPositivism}
\end{center}
\end{table}