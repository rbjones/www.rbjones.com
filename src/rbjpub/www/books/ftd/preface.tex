% $Id: preface.tex,v 1.7 2014/11/08 19:43:28 rbj Exp $

%\renewcommand{\cfttoctitlefont}{\Large}
%\renewcommand{\cftlottitlefont}{\Large}

{\parskip=0pt\tableofcontents}
%\addcontentsline{toc}{section}{Contents}
%{\parskip=0pt\listoftables}
%\addcontentsline{toc}{section}{List of Tables}

\newcommand{\listpositionname}{Positions}
\newlistof[chapter]{position}{pos}{\listpositionname}

\newcommand{\position}[2]{%
\refstepcounter{position}
{\paragraph{\textbf{#1 \theposition}} #2}
%\index{#2}
\addcontentsline{pos}{position}{\protect\numberline{\theposition} #2}\par}

%\newcommand{\answer}[1]{%
%\refstepcounter{answer}
%\par\noindent\textbf{Answer \theanswer. #1}
%\addcontentsline{ans}{answer}{\protect\numberline{\theanswer}#1}\par}

%\renewcommand{\cftpostitlefont}{\Large}

\chapter*{Preface}\label{Preface}
\addcontentsline{toc}{chapter}{Preface}

In this volume I present philosophical ideas belonging to the positivist tendency in philosophy.
Philosophical positivism is sometimes said to be an \emph{anti-philosophical tendency} because of its rejection of many philosophical problems and their proposed resolutions.

Positivism, as here presented, is a graduated constructive skepticism which aims to offer a philosophical framework for the conduct of science and engineeering, and for their application to the benefit of society.
The need for a contemporary restatement of positivist philosophy is underpinned by advances in technology which for the first time in human history bring machines which may know.

David Hume, an important figure in the history of positivism. emphasised two important distinctions that we might now describe as that between logical and factual truths, and that between objective truths and subjective judgements of value.
.
These distinctions are central to the philosophy outlined here, and help to demarcate the scope of the volume.
This volume is concerned primarily with the domain of objective truth, encompassing the truths of logic and mathematics, and those of empirical science.
Matters involving values, sometimes called ``practical philosophy'' will be addressed in a subsequent volume.
Even when concerned primarily with objective truth, the choice of a ``philosophical framework'' of the kind here envisaged is pragmatic.
The proposals here are motivated by the belief that they will be better in various respects than the alternatives.

\mainmatter
