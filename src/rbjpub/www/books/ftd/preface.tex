% $Id: preface.tex,v 1.8 2015/02/04 06:39:29 rbj Exp $

%\renewcommand{\cfttoctitlefont}{\Large}
%\renewcommand{\cftlottitlefont}{\Large}

{\parskip=0pt\tableofcontents}
%\addcontentsline{toc}{section}{Contents}
%{\parskip=0pt\listoftables}
%\addcontentsline{toc}{section}{List of Tables}

\newcommand{\listpositionname}{Positions}
\newlistof[chapter]{position}{pos}{\listpositionname}

\newcommand{\position}[2]{%
\refstepcounter{position}
{\paragraph{\textbf{#1 \theposition}} #2}
%\index{#2}
\addcontentsline{pos}{position}{\protect\numberline{\theposition} #2}\par}

%\newcommand{\answer}[1]{%
%\refstepcounter{answer}
%\par\noindent\textbf{Answer \theanswer. #1}
%\addcontentsline{ans}{answer}{\protect\numberline{\theanswer}#1}\par}

%\renewcommand{\cftpostitlefont}{\Large}

\chapter*{Preface}\label{Preface}
\addcontentsline{toc}{chapter}{Preface}

In this volume I present philosophical ideas belonging to the \emph{positivist} tendency in philosophy.

Philosophical positivism is sometimes said to be an \emph{anti-philosophical} tendency because of its rejection of many philosophical problems and methods.
Positivism, as here presented, is a graduated constructive skepticism which aims to offer a philosophical framework for the conduct of science and engineeering, and for their application to the benefit of society.

The need for a contemporary restatement of positivist philosophy is underpinned by advances in technology which for the first time in human history bring machines which may \emph{know}.
This is an opportunity to do things better, but the possibility of improvement depends upon the perception of shortfall.
Skepticism provides that perspective, but the most radical scepticism also undermines the possibility of improvement.

Progress therefore depends upon a moderate constructive scepticism in which defects are perceived which may not be beyond remedy.
Positivistic philosophy might be cast in this role, but has historically often been too radical a scepticism advocating an unrealsable remedy.
A central theme in this book is to revise the origins and history of positivism and to offer a moderation of the positivist critique some improvements to the remedies.

David Hume, an important figure in the history of positivism. emphasised two important distinctions that we might now describe as those between logical and factual truths, and between objective truths and value judgements.
These distinctions are central to the philosophy outlined here, and help to demarcate the scope of the volume.
This volume is concerned primarily with the domain of objective truth, encompassing the truths of logic and mathematics and those of empirical science.
Matters involving values, sometimes called ``practical philosophy'' will be addressed in a separate volume.
Even when concerned primarily with objective truth, the choice of a ``philosophical framework'' of the kind presented here is pragmatic.
Theoretical philosophy is here considered subservient to practical philosophy, and though the subject matter of this volume is theoretical, pragmatic considerations motivate and permeate the discussion.

\mainmatter
