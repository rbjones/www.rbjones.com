% $Id: preface.tex,v 1.8 2015/02/04 06:39:29 rbj Exp $

%\renewcommand{\cfttoctitlefont}{\Large}
%\renewcommand{\cftlottitlefont}{\Large}

{\parskip=0pt\tableofcontents}
%\addcontentsline{toc}{section}{Contents}
%{\parskip=0pt\listoftables}
%\addcontentsline{toc}{section}{List of Tables}

\newcommand{\listpositionname}{Positions}
\newlistof[chapter]{position}{pos}{\listpositionname}

\newcommand{\position}[2]{%
\refstepcounter{position}
{\paragraph{\textbf{#1 \theposition}} #2}
%\index{#2}
\addcontentsline{pos}{position}{\protect\numberline{\theposition} #2}\par}

%\newcommand{\answer}[1]{%
%\refstepcounter{answer}
%\par\noindent\textbf{Answer \theanswer. #1}
%\addcontentsline{ans}{answer}{\protect\numberline{\theanswer}#1}\par}

%\renewcommand{\cftpostitlefont}{\Large}

\chapter*{Preface\markboth{Preface}{}}\label{Preface}
\addcontentsline{toc}{chapter}{Preface}

I have explored for many years the ground between {\it philosophy} and {\it information systems architecture}, seeking a place for the ideas presented here.
That these two subject matters are adjacent may come to some as a surprise, mitigated perhaps by the evident aspirations of many technologists confident that an era of {\it intelligent machines}, machines which {\it know}, {\it think}, and {\it reason} rather than merely compute, is fast approaching.

The engineering of {\it artificial intelligence} is predominantly oriented toward replicating, and perhaps surpassing, various human intellectual capabilities.
It is not generally rooted in a critique of, or the desire to improve, the ways in which {\it homo sapiens} thinks and reasons.
But philosophers have traditionally studied {\it the theory of knowledge} and have sometimes elaborated and promoted ideas on how we might do better.
More specifically, philosophical interest in {\it scientific method} has a long history, and has often been normative in character, prescribing how science {\it should} be done rather than (or as well as) describing or analysing how it {\it is} done.
Even when not explicity prescriptive, a description of scientific method which distinguishes {\it bona-fide} from {\it pseudo-} science is implicitly normative.

A global distributed hybrid intelligence, in which man and machines realise a capability which neither could on its own, might possible emerge from the chaos of the free market without an overarching architectural vision having been cogently articulated.
But what we realise in that way may not be close to what we would {\it choose} if we could.
The kind of architectural thinking needed to exercise that kind of choice begins with questions which have traditionally belonged to philosophy, not least the question ``what is {\it knowledge}?''.

The way in which engineers answer such questions may be fundamentally different to that typical of philosophers.
Most philosophers will approach that question taking language as {\it given}, and may then wrestle with the semantic puzzles created by diversity and inconsistency in the usage which is his primary evidence of meaning.
Scientists and engineers may treat language in a more pragmatic way, adapting it to their purposes.
Too often our language is ill-equipped to express the features of our world which emerge from scientific research, and the scientist will need new vocabulary.
Sometimes giving a refined meaning in a special context to a word which already has a diversity of uses may be more congenial than inventing new terms.
Engineers devising new artefacts may also have difficulty in finding the words they need to describe their novel constructions, and will adapt and extend the language as necessary.

In information engineering, the scale of linguistic innovation has been gigantic.
Hundreds, if not thousands, of new languages have been devised for ``telling'' computers what to do, not only {\it programming} languages, but also languages intended for every stage in the elaborate process of developing large scale computer systems, and a myriad of languages for describing the different problem domains to which computers are now applied.

This more pragmatic attitude toward language also appears in some kinds of philosophy, notably among {\it positivists}, who eschew the absolute truths of metaphysics in favour of conventional truths rooted in pragmatic choices of language and terminology and empirical truths about the world we live in.
Aspects of this positivist tradition in philosophy contribute to the {\it positive philosophy} exemplified by the ideas which this monograph synthesises into an architecture for hybrid man-machine cognition.

\mainmatter
