% $Id: preface.tex,v 1.4 2012/03/23 15:41:40 rbj Exp $

\newcommand{\listpositionname}{Positions}
\newlistof[chapter]{position}{pos}{\listpositionname}

\newcommand{\position}[2]{%
\refstepcounter{position}
{\paragraph{\textbf{#1 \theposition}} #2}
%\index{#2}
\addcontentsline{pos}{position}{\protect\numberline{\theposition} #2}\par}

%\newcommand{\answer}[1]{%
%\refstepcounter{answer}
%\par\noindent\textbf{Answer \theanswer. #1}
%\addcontentsline{ans}{answer}{\protect\numberline{\theanswer}#1}\par}

%\renewcommand{\cftpostitlefont}{\Large}

\chapter*{Preface}\label{Preface}
\addcontentsline{toc}{chapter}{Preface}

The discovery of the structure of DNA may be the most significant
scientific discovery ever made.
It marks a stage in the evolution of life on earth at which the nature
of evolution itself steps forward.

This discovery was made possible by the previous most significant
product of the evolutionary process, the evolution of intelligence.

In evolutionary timescales the structure of DNA was uncovered at much
the same time as the invention of the stored program digital computer.
The digital computer provides potentially for intelligence a future
even more radical than the changes in life forms and the ecosphere
which we may expect to flow from genetic engineering.

The difficulties in realising machine intelligence are however very
great.
Over the past half century considerable effort has been devoted to
that enterprise with limited success.

\mainmatter
