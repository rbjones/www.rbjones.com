\def\rbjidpreface{$$Id: preface.tex,v 1.1 2011/11/04 16:38:42 rbj Exp $$}

\section*{Preface}\label{Preface}
\addcontentsline{toc}{section}{Preface}

It is a commonplace that the pace of change accelerates and that the twentieth century saw an unprecedented expansion in scientific knowledge.

We may each have our own ideas on which of the innovations of the twentieth century will have the greatest future impact on our quality of life.
To me the discovery of the structure of DNA, and the invention of the general purpose digital computer seem the most significant.

Understanding the structure of DNA is of course a mere prelude to a comprehensive understanding of its role, the decoding and analysis of the human genome.
The most profound consequence of this is its probable effect on the evolution of the human species and of all life on our planet, with a gradual shift from evolution by natural selection among variants generated by random mutation and sexual recombination to evolution by design.

The most distinctive feature of humanity is however intellectual and spiritual, and it is in information technology rather than genetics that we see the greater prospect for transcending human intelligence.
As we approach the point at which information technology begins to rival humanity in something like intelligence, the prospect that this new intelligence will be sociopathic provides a note of caution.
If machine intelligence were to emerge in the chaotic way that human intelligence evolved this would be a real danger, long recognised in fiction and addressed nominally by Asimov's three laws of robotics.
We have seen however, how an apparent champion of the individual can make a nonsense of the concept of freedom and become platform for totalitarianism.

The philosophy outlined in this monograph is partly motivated by the idea that automation of certain important kinds of reason can be approached by methods which guarantee against sociopathy.

A broader motivation is the skeptical perception that \emph{rationality} is a rarer commodity than it is commonly supposed to be, and that the world might be a better place if that were not the case.

\mainmatter
