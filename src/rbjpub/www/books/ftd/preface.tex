% $Id: preface.tex,v 1.5 2012/04/04 20:22:09 rbj Exp $

\newcommand{\listpositionname}{Positions}
\newlistof[chapter]{position}{pos}{\listpositionname}

\newcommand{\position}[2]{%
\refstepcounter{position}
{\paragraph{\textbf{#1 \theposition}} #2}
%\index{#2}
\addcontentsline{pos}{position}{\protect\numberline{\theposition} #2}\par}

%\newcommand{\answer}[1]{%
%\refstepcounter{answer}
%\par\noindent\textbf{Answer \theanswer. #1}
%\addcontentsline{ans}{answer}{\protect\numberline{\theanswer}#1}\par}

%\renewcommand{\cftpostitlefont}{\Large}

\chapter*{Preface}\label{Preface}
\addcontentsline{toc}{chapter}{Preface}

{\it
[I really don't have much clue what to do with this preface.
The plan for the preface is as follows.
Every now and then I will just throw it away and write another.
Hopefully, sometime, I will write one that I like, and then keep it.]
}%it


This is a book of applied theoretical philosophy, participating in the
spirit of Aristotle's `first philosophy', for some time now
unfashionable among contemporary philosophers. 

One aspect of that spirit is the idea that philosophy should be
concerned with the most fundamental problems which transcend the
boundaries of other academic disciplines. 
A second is that `first philosophy' should supply a context in which
other kinds of philosophy can fruitfully be conducted. 
We may think of this as a \emph{foundational} role, as the provision
of some solid basis for the rest of ``philosophy'' in its broadest
sense (literally, from the Greek \emph{the love of knowledge}). 

Aristotle thought first philosophy to be a superior pursuit, undertaken
by the most accomplished of men, and that such men should not only
command respect, but should have authority over those engaged in less
august undertakings.
It is perhaps this aspect of Aristotle's writings which has encouraged
some recent philosophers to speak as if \emph{first philosophy} is an
attempt  by philosophers to impose their intellectual authority on
other disciplines, and have preferred instead to study how those other
disciplines are conducted, rather than make pronouncements about how
they \emph{should} be conducted.

In this work I seek a third alternative, which is to put forward a
philosophical framework conducing to the search for knowledge and its
application as a contribution \emph{imter alia} to the considerations
of others about how they should proceed.

Western Philosophy has traditionally been divided into two parts,
theoretical and practical philosophy.
Theoretical philosophy is concerned primarily with knowledge, and how
(if at all) it can be established and applied.
Practical philosophy on the other hand, is concerned with matters
closer to the ways we live our lives, with the values and morals which
shape our conduct, and with the ways in which societies may be
organised, politically and economically.

In speaking here of the \emph{application} of theoretical philosophy
it is not practical philosophy that I have in mind.
It is to the development of technologies which support the primary
subject matters of theoretical philosophy, knowledge.
The prospects for such applications of theoretical philosophy were
transformed in the $20^{th}$ century by the new discipline of symbolic
or mathematical logic, and the invention of the stored program digital
computer.
These advances made more plausible the development of technologies
first envisaged hundreds of years previously by the philosopher,
mathematician, scientist and engineer, Gottfried Wilhelm Leibniz.

With these developments the nature of knowledge itself may be
transformed.
Firstly by the possibility that there may be for the first time
inanimate \emph{cognitive agents} (things which \emph{know}).
...

\mainmatter
