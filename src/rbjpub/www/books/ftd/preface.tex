% $Id: preface.tex,v 1.6 2012/06/03 21:32:25 rbj Exp $

\newcommand{\listpositionname}{Positions}
\newlistof[chapter]{position}{pos}{\listpositionname}

\newcommand{\position}[2]{%
\refstepcounter{position}
{\paragraph{\textbf{#1 \theposition}} #2}
%\index{#2}
\addcontentsline{pos}{position}{\protect\numberline{\theposition} #2}\par}

%\newcommand{\answer}[1]{%
%\refstepcounter{answer}
%\par\noindent\textbf{Answer \theanswer. #1}
%\addcontentsline{ans}{answer}{\protect\numberline{\theanswer}#1}\par}

%\renewcommand{\cftpostitlefont}{\Large}

\chapter*{Preface}\label{Preface}
\addcontentsline{toc}{chapter}{Preface}

{\it
[I really don't have much clue what to do with this preface.
The plan for the preface is as follows.
Every now and then I will just throw it away and write another.
Hopefully, sometime, I will write one that I like, and then keep it.]
}%it

This is the first of two volumes in which I hope to give a first
account of a philosophical point of view which I call \emph{Positive
  Philosophy}.
In this preface I describe the scope of the enterprise and the way in
which it is divided across the two volumes, by reference to the
divisions in the philosophy of Aristotle.

To a first approximation the first volume is concerned with
\emph{theoretical} philosophy, the second with \emph{practical}
philosophy.
Of these, theoretical philosophy is concerned with knowledge, with
logic, mathematics and empirical science, and practical philosophy
with action, with ethics, politics and economics.

These two categories, the theoretical and the practical, appear in
Aristotle, but are not exhaustive of his work.
Two other divisions are \emph{the organon}, which consists of
Aristotle's writings on or related to logic, conceived rather more
broadly than today, and \emph{productive science}, which involve arts
and crafts, or more generally, how to make or do things.
In this two-fold division of positive philosophy, the first volume,
theoretical philosophy, includes also logic.
In addition to this augmented theoretical philosophy, and the second
volume on practical philosophy there is an important element of what
Aristotle might have thought productive philosophy, concerned with a
kind of production which I think of primarily as \emph{engineering},
albeit primarily \emph{software} engineering.




\mainmatter
