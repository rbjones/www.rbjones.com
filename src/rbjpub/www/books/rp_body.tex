\usepackage[T1]{fontenc}
\usepackage{textcomp}
\renewcommand{\rmdefault}{ppl}
\linespread{1.04}

\makeatletter
\def\cleardoublepage{\clearpage\if@twoside \ifodd\c@page\else
\hbox{}
\vspace*{\fill}
\begin{center}
\end{center}
\vspace{\fill}
\thispagestyle{empty}
\newpage
\if@twocolumn\hbox{}\newpage\fi\fi\fi}
\makeatother

\usepackage{fancyhdr}
\pagestyle{fancyplain}

%\pagestyle{headings}
\usepackage[twoside,paperwidth=5in,paperheight=8in,hmargin={0.75in,0.5in},vmargin={0.5in,0.5in},includehead,includefoot]{geometry}
\usepackage{tocloft}
\usepackage{tocbibind}
%\bodytext{BGCOLOR="#eeeeff"}
\makeindex
\newcommand{\indexentry}[2]{\item #1 #2}
\newcommand{\ignore}[1]{}

\fancyhfoffset[EL,RO]{0pt}
\newcommand{\aref}{}
\newcommand{\bookname}{}
\renewcommand{\chaptermark}[1]{\markboth{#1}{}}
\renewcommand{\sectionmark}[1]{\markright{#1}}
\lhead[\fancyplain{}{\thepage}]         {\fancyplain{}{}}
\chead[\fancyplain{}{\slshape\leftmark}]                 {\fancyplain{}{\slshape\rightmark}}
\rhead[\fancyplain{}{}]       {\fancyplain{}{\thepage}}
\lfoot[\fancyplain{}{\aref}]            {\fancyplain{}{}}
\cfoot[\fancyplain{}{}]                 {\fancyplain{}{}}
\rfoot[\fancyplain{}{}]                 {\fancyplain{}{\aref}}

\renewcommand{\headrulewidth}{0pt}

\title{Rationality and Purpose \\ -- \\ a future}
\author{Roger Bishop Jones}
\date{\ }

\begin{document}
\frontmatter

\begin{titlepage}
\maketitle

\hspace{2in}

\vfill

\begin{centering}

{\small

Written and published by Roger Bishop Jones\\
www.rbjones.com\\

\vspace{0.2in}

ISBN-13: \\
ISBN-10: 

\vspace{0.2in}

}%small
{\scriptsize

First edition. \hfil Version 0.1 \hfil 2015-07-02

\vspace{0.2in}

\copyright\ Roger Bishop Jones;

}%scriptsize

\end{centering}

\thispagestyle{empty}

\end{titlepage}

{\parskip=0pt\tableofcontents}


%\chapter*{Preface}\label{Preface}
%\addcontentsline{toc}{chapter}{Preface}

\mainmatter

\chapter{Introduction}

This work consists of my thoughts about the future.
Not about {\it my} future, a fantasy with greater reach.

The title, not closely considered, but arrived at in the following way.

We consider the future of humanity, of which rationality has been considered
a distinctive feature.
The supposed rationality of {\it homo sapiens} has been made possible by the
evolution of his large brain.
There are some other special features which may account for the success of 
homo sapiens relative, say, to that of the blue whale (with a way larger brain),
but let's pass over those for now.

From where we are now, it looks like two important developments which bear upon
that rationality are under way.

The first is that we (humanity) are intent on building intelligent machines, and,
I think we {\it will} suceed, not sure when, but there's a very fuzzy line and
I don't think it matters exactly when we cross it (many people think that once
we are slightly over it we will rapidly be way over, but I think that attaches
a bit too much significance to exactly how intelligent we humans are right now). 

The second is that genetic engineering and synthetic biology are about to
spawn an acceleration in the pace of evolution of us humans (and the rest of
the ecosphere).

There are many other big changes which seem to be upon us, but those two are
for me the most significant, and that is therefore my focus here, because these
impact most directly upon that special tool which got us here (and for lovers
of ``the singularity'' play a key role in that singularity).

Its natural enough for anyone wanting to understand these kinds of development to ask ``what is intelligence?''.
The slant here is slightly different, I ask not about {\it intelligence} but more specifically
about that special thing which intelligence supports, {\it rationality}.

I consider therefore some key developments which will transform the character of rationality,
partly because they provide almost unlimited intelligence.
I sidestep the analysis of intelligence by supposing it to provide just one thing,
near complete {\it deductive closure}.

Rationality is concerned primarily with how we realised our ends.
The focus on {\it means} which is implicit in a study in rationality begs to be complemented
by some consideration of {\it ends} in a rounded consideration of the distant future.
However, just as means should be subordinate to ends, ends are subordinate to {\it purpose}
which we may therefore think of as a polar opposite to those effective means which constitute
rationality.

In considering purpose here I am interested in relationship of purpose in individuals and in
the whole.
The whole here, not only of humanity, but of all that grows from and around it, of our evolutionary
descendents ever more closely integrated with the technology they have spawned and with each
other through that technology.

\chapter{}

My concern here is with the {\it distant} future; as far distant as reason can reach.
I'm not thinking in specific terms, where possible I will be reasoning {\it in the limit}, about things which we might expect to become true {\it eventually} and to remain truth thenceforth, barring cataclysm.

It might seem improbable that the distant future would be so well ordered.
Some credibility might be lent to the idea that there might be such limits from consideration of the evolutionary theory of {\it punctuated equilibrium}.
The idea here is that the evolution of species on our planet occurs primarily in bursts of innovation punctuated by long periods of near stasis.
The bursts of activitity are precipitated by some development or event which causes a sufficiently large change in the selective environment that the various organisms in the ecosystem, previously perfectly attuned to the environmental niche which they have occupied, find themselves in changed circumstances to which they are not well adapted.
The species then evolve rapidly until in due course a new equilibrium is reached and the pace of evolutionary change reverts to its more usual geological snails pace.

Unfortunate[y we are now at a point in the history of our evolution at which extrapolation from our evolutionary past into the future is more perilous than it ever has been.
We are about seize control of the levers of evolution, both the genetic variation and the selective pressures which have hitherto been substantially ``natural'', will soon be effected by deliberate choice.
This will be an end to evolutionary equilibrium which well might turn out to be self perpetuating.

There might nevertheless be way to anticipate where this might take us.

Let us suppose that (for reasons we will go into more closely later) there is a long term tendency to greater efficiency,
to greater rationality in the way in which we attempt to realise our purposes.





\backmatter

\bibliographystyle{plain}
\bibliography{rbj2}

\clearpage
%\listoftables
%\addcontentsline{toc}{section}{List of Tables}

\clearpage
%\listofposition
%\addcontentsline{toc}{chapter}{List of Positions}

\twocolumn[
%\section*{Index}\label{index}
%\addcontentsline{toc}{section}{Index}
]
{\small\printindex}

\vfil

\end{document}
