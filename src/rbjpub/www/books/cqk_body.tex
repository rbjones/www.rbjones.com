\usepackage[T1]{fontenc}
\usepackage{textcomp}
\renewcommand{\rmdefault}{ppl}
\linespread{1.04}

\makeatletter
\def\cleardoublepage{\clearpage\if@twoside \ifodd\c@page\else
\hbox{}
\vspace*{\fill}
\begin{center}
\end{center}
\vspace{\fill}
\thispagestyle{empty}
\newpage
\if@twocolumn\hbox{}\newpage\fi\fi\fi}
\makeatother

\usepackage{fancyhdr}
\pagestyle{fancyplain}

%\pagestyle{headings}
\usepackage[twoside,paperwidth=5in,paperheight=8in,hmargin={0.5in,0.25in},vmargin={0.25in,0.25in},includehead,includefoot]{geometry}
\usepackage{tocloft}
\usepackage{tocbibind}
%\bodytext{BGCOLOR="#eeeeff"}
\makeindex
\newcommand{\indexentry}[2]{\item #1 #2}
\newcommand{\ignore}[1]{}

\fancyhfoffset[EL,RO]{0pt}
\newcommand{\aref}{}
\newcommand{\bookname}{}
\renewcommand{\chaptermark}[1]{\markboth{#1}{}}
\renewcommand{\sectionmark}[1]{\markright{#1}}
\lhead[\fancyplain{}{\thepage}]         {\fancyplain{}{}}
\chead[\fancyplain{}{\slshape\leftmark}]                 {\fancyplain{}{\slshape\rightmark}}
\rhead[\fancyplain{}{}]       {\fancyplain{}{\thepage}}
\lfoot[\fancyplain{}{\aref}]            {\fancyplain{}{}}
\cfoot[\fancyplain{}{}]                 {\fancyplain{}{}}
\rfoot[\fancyplain{}{}]                 {\fancyplain{}{\aref}}

\renewcommand{\headrulewidth}{0pt}

\title{Carnap, Quine and Kripke\\ \Large - \\ the rout of positivism}
\author{Roger Bishop Jones}
\date{\ }

\begin{document}
\frontmatter

\begin{titlepage}
\maketitle

\hspace{2in}

\vfill

\begin{centering}

{\small

Written and published by Roger Bishop Jones\\
www.rbjones.com\\

\vspace{0.2in}

ISBN-13: \\
ISBN-10: 

\vspace{0.2in}

}%small
{\scriptsize

First edition. \hfil Version 0.1 \hfil 2015-07-02

\vspace{0.2in}

\copyright\ Roger Bishop Jones;

}%scriptsize

\end{centering}

\thispagestyle{empty}

\end{titlepage}

{\parskip=0pt\tableofcontents}


\chapter*{Preface}\label{Preface}
\addcontentsline{toc}{chapter}{Preface}

\mainmatter

\chapter{Introduction}

\chapter{Quine on Carnap}

The central startling feature of Quine's relationship with
Carnap is his rejection the analytic/synthetic distinction.
This chapter is devoted to an analysis of the debate around
that issue.
The issue is sufficiently fundamental that it could not fail
to be connected with other important problems.
It was so fundamental a difficulty for Carnap that we can
see its effects upon his philosophy from the time that Quine
first entered professional philosophy to the end of Carnap's
life.

The following analysis will track the trajectory of Carnap's
philosophy under this influence, the major features of which
are sketched here.

Quine's first engagement with Carnap followed the publication
of Carnap's {\it The Logical Syntax of Language}\cite{Carnap37},
which he had the privilege of reading as it was typed up by
Ina Carnap.

When Quine returned to Harvard he was invited to give three
lectures on Carnap's philosophy.  
At this stage in Quine's career it is generally held that
he was an uncritical enthusiast for Carnap's philosophy,
Quine himself was later to describe these lectures as ``sequacious''.
However, the first of these lectures, in which Quine is explicitly
not giving an account of Carnap's work, but rather providing some
backgound on ``the analytic character of the a priori'', there
are many points at which Quine's presentation diverges significantly
from what we might have expected from Carnap.
This first lecture is quickly expanded by Quine into the paper
{\it Truth by Convention}\cite{Quine36}, suggesting that Quine
felt there to be some new ideas here, and indeed the paper is
billed by Quine as questioning received opinion.

The principal thrust of {\it Logical Syntax} is then abandoned by
Carnap as he embraces formal semantics, though this is not so
much a change of purpose as a revised approach to realising
that purpose, scientific philosophy.
The principal influence which Carnap later aknowledges in this
transition is the semantic work of Tarski, but
the move to semantics can also be seen as responding to a
misundertanding or mis-presentation (or misrepresentation)
of Carnap's position implicit in Quine's first lecture.

Though Carnap might have hoped that his embracing of semantics
would render his account of analyticity more transparent, there
ensues, principally in the year 1943 an extended correspondence
between Carnap and Quine which fails to secure a rapprochement.

At the same time, Quine is publishing doubts about modal logics.
For Carnap the concepts of analyticity and of logical necessity
are intimately connected, indeed mutually interdefinable, so
it is not surprising that Carnap's next major work addressed these
issues.


\section{Carnap's Logical Syntax}

\section{Quine on Syntax and Truth by Convention}

\section{Carnap on Semantics}

\section{Correspondence on Analyticity}

\section{Meaning and Necessity}

\section{Two Dogmas}

\section{Responses to Two Dogmas}

\chapter{Kripke's Metaphysics}

\chapter{Moving Forward}

\backmatter

\bibliographystyle{alpha}
\bibliography{carnap,quine}

\clearpage
%\listoftables
%\addcontentsline{toc}{section}{List of Tables}

\clearpage
%\listofposition
%\addcontentsline{toc}{chapter}{List of Positions}

\twocolumn[
%\section*{Index}\label{index}
%\addcontentsline{toc}{section}{Index}
]
{\small\printindex}

\vfil

\end{document}
