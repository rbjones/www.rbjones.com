\usepackage[T1]{fontenc}
\usepackage{textcomp}
\renewcommand{\rmdefault}{ppl}
\linespread{1.04}

\makeatletter
\def\cleardoublepage{\clearpage\if@twoside \ifodd\c@page\else
\hbox{}
\vspace*{\fill}
\begin{center}
\end{center}
\vspace{\fill}
\thispagestyle{empty}
\newpage
\if@twocolumn\hbox{}\newpage\fi\fi\fi}
\makeatother

\usepackage{fancyhdr}
\pagestyle{fancyplain}

%\pagestyle{headings}
\usepackage[twoside,paperwidth=5in,paperheight=8in,hmargin={0.75in,0.5in},vmargin={0.5in,0.5in},includehead,includefoot]{geometry}
\usepackage{tocloft}
\usepackage{tocbibind}
%\bodytext{BGCOLOR="#eeeeff"}
\makeindex
\newcommand{\indexentry}[2]{\item #1 #2}
\newcommand{\ignore}[1]{}

\fancyhfoffset[EL,RO]{0pt}
\newcommand{\aref}{}
\newcommand{\bookname}{}
\renewcommand{\chaptermark}[1]{\markboth{#1}{}}
\renewcommand{\sectionmark}[1]{\markright{#1}}
\lhead[\fancyplain{}{\thepage}]         {\fancyplain{}{}}
\chead[\fancyplain{}{\slshape\leftmark}]                 {\fancyplain{}{\slshape\rightmark}}
\rhead[\fancyplain{}{}]       {\fancyplain{}{\thepage}}
\lfoot[\fancyplain{}{\aref}]            {\fancyplain{}{}}
\cfoot[\fancyplain{}{}]                 {\fancyplain{}{}}
\rfoot[\fancyplain{}{}]                 {\fancyplain{}{\aref}}

\renewcommand{\headrulewidth}{0pt}

\title{Carnap, Quine and Kripke\\ \Large - \\ the rout of positivism}
\author{Roger Bishop Jones}
\date{\ }

\begin{document}
\frontmatter

\begin{titlepage}
\maketitle

\hspace{2in}

\vfill

\begin{centering}

{\small

Written and published by Roger Bishop Jones\\
www.rbjones.com\\

\vspace{0.2in}

ISBN-13: \\
ISBN-10: 

\vspace{0.2in}

}%small
{\scriptsize

First edition. \hfil Version 0.1 \hfil 2015-07-02

\vspace{0.2in}

\copyright\ Roger Bishop Jones;

}%scriptsize

\end{centering}

\thispagestyle{empty}

\end{titlepage}

{\parskip=0pt\tableofcontents}


\chapter*{Preface}\label{Preface}
\addcontentsline{toc}{chapter}{Preface}

\mainmatter

\chapter{Introduction}

\chapter{Carnap's Mission}

Rudolf Carnap was a man with a mission.

The second half of the nineteenth century saw increasing interest on the part of mathematicians
in \emph{logic}, a subject previously of interest primarily to philosophers
in which the logic of Aristotle had prevailed for over 2000 years.
The resulting advances, were rapid and substantial, completely transforming the character
of the subject and realising logical systems for the first time which were sufficiently powerful
for the derivation of the theorems of pure mathematics and other \emph{a priori} disciplines.  

In his book \emph{Our Knowledge of the External World as a Field For Scientific Method in Philosophy} \cite{russell21} Bertrand Russell describes the revolutionary impact which he hoped these developments might have on the character and conduct of philosophy.

\begin{quote}
``The study of logic becomes the central study in philosophy: it gives the method of research in philosophy, just as mathematics gives the method in physics....	
All this supposed knowledge in the traditional systems must be swept away, and a new beginning must be made. . . .''
\end{quote}

Carnap attended lectures by Gottlob Frege as an undergraduate, studied his work in greater depth as a postgraduate student
and was already an enthusiast for the adoption of the new logical methods in philosophy when he read Russell's
book.
He was inspired and later wrote in his \emph{Intellectual Autobiography}
\cite{carnap63}:

\begin{quotation}
``I felt as if this appeal had been directed to me personally.
To work in this spirit would be my task from now on! And indeed henceforth the application of the new logical instrument for the purposes of analyzing scientific concepts and of clarifying philosophical problems has been the essential aim of my philosophical activity.''
\end{quotation}

In this chapter I describe the context in which the dialogue between
Carnap and Quine began.

My account of the dialogue between Carnap and Quine will begin with Carnap's {\it Logical Syntax of Language} \cite{carnap34, carnap37}, so I will provide in this chapter the context needed for that account, which will fall into two parts.
The first of these will be an an account of \emph{positivism}, primarily as this first appears in the philosophy of David Hume, of the reaction against Hume's view of mathematics by Kant and of the purported refutation of that reaction in the thesis of \emph{logicism} by Frege and Russell.
The second will sketch the development of Carnap's philosophy prior to the turning point inspired by G\"odel which lead to {\it Logical Syntax}.

\section{Humean Positivism}

David Hume published his philosophical \emph{magnum opus}, \emph{A Treatise of Human Nature} \cite{hume39}
as a young man.
He had penetrating insight into the nature of philosophy, found traditional metphysicians to
be lacking, and conceived of a new way of doing philosophy modelled on the scientific method
whose exponent Isaac Newton had inspired through his scietific achievements.

In \emph{The Treatise} Hume applied the new method extensively, but his work
was received with indifference or hostility.

It is not infrequently the case that a philosopher may fail to recognise
his most important ideas.
These may be fundamental, and easily described in few words, but too simple
to provide a basis for a substantial work of philosophy.

In the face of indifference to his great accomplishment Hume decided that
an improved presentation of the most important elements of his philosophy
might be more sucessful in bringing the ideas to the attention of the
public, and to that end he wrote \emph{An Enquiry Concerning Human Understanding} \cite{hume48}.

Hume's ideas were a prototype for much that is found in Carnap's
\emph{Logical Positivism} and a brief account of them will help in explaining the distinctive
contribution that Carnap made to the positivist tradition.
Carnap himself, though adopting the term \emph{logical positivist} during his time as
a member of the Vienna Circle, was later uncertain about calling himself a positivist,
considering whether the term \emph{logical empiricist} might be more appropriate.

We will see good reason for such reticence, in important ways Carnap had diverged from
the positivist tradition.
I believe that Carnap should be regarded as having moved forward positivism onto new
more solid ground, and will present Carnap's innovations as fully within the
spirit of positivism.

Hume begin's his enquiry by depricating the {\it metaphysicians}





\section{Kant's Awakening}

\section{Logicism}

\section{Conventionalism}

\section{Carnap before `Syntax'}

\chapter{Quine on Carnap}

\nocite{carnap56,carnap63,copi67,quine53,quine61, quine61a,quine66,quine86}

The central startling feature of Quine's relationship with
Carnap is his rejection the analytic/synthetic distinction.
This chapter is devoted to an analysis of the debate around
that issue.
The issue is sufficiently fundamental that it could not fail
to be connected with other important problems.
It was so fundamental a difficulty for Carnap that we can
see its effects upon his philosophy from the time that Quine
first entered professional philosophy to the end of Carnap's
life.

The following analysis will track the trajectory of Carnap's
philosophy under this influence, the major features of which
are first sketched here, and later examined in greater detail.

Quine's first engagement with Carnap followed the publication
of Carnap's {\it The Logical Syntax of Language}\cite{carnap37},
which he had the privilege of reading as it was typed up by
Ina Carnap.

When Quine returned to Harvard he was invited to give three
lectures on Carnap's philosophy.  
At this stage in Quine's career it is generally held that
he was an uncritical enthusiast for Carnap's philosophy,
Quine himself was later to describe these lectures as ``sequacious''.
However, the first of these lectures, in which Quine is explicitly
not giving an account of Carnap's work, but rather providing some
backgound on ``the analytic character of the a priori'', there
are many points at which Quine's presentation diverges significantly
from what we might have expected from Carnap.
This first lecture is quickly expanded by Quine into the paper
{\it Truth by Convention}\cite{quine36}, suggesting that Quine
felt there to be some new ideas here, and indeed the paper is
billed by Quine as questioning received opinion.

The principal thrust of {\it Logical Syntax} is then abandoned by
Carnap as he embraces formal semantics, though this is not so
much a change of purpose or mission as a revised approach to realising
that purpose, scientific philosophy and positive science
(in Carnap's revised account of that Comtean conception).
The principal influence which Carnap later aknowledges in this
transition (from logical syntax to formal semantics) is the
semantic work of Tarski, but the move to semantics can also
be seen as responding to a misundertanding or mis-presentation
(or misrepresentation) of Carnap's position implicit in Quine's
first lecture and Quine's {\it Truth by Convention} \cite{quine36}
which followed hard on its heels.

Though Carnap might have hoped that his embracing of semantics
would render his account of analyticity more transparent and less
objectionable to Quine, there ensues, principally in the year 1943
an extended correspondence between Carnap and Quine \cite{carnap90}
which fails to secure a rapprochement.

At the same time, Quine is publishing doubts about modal logics.
For Carnap the concepts of analyticity and of logical necessity
are intimately connected, indeed mutually interdefinable, so
it is not surprising that Carnap's next major work addressed these
issues.

\section{The Structure of the Analysis}

\subsection{The Issues at Stake}

I begin by describing the issues at stake between Carnap and Quine,
first the central thrust of Carnap's philosophical mission contrasted
with such as can be said about that of Quine at this stage in his
development, then in a little more detail under the four headings:

\begin{itemize}
\item Logical truth
\item The analytic/synthetic dichotomy
\item Necessity and modal logic
\item Ontology and metaphysics
\end{itemize}

The intention is to sketch before entering in to the detail of the dialogue
the signficance of the topics for Carnap's philosophical agenda, and
to give a rationale for the areas in which the analysis of the dialogue
will be focussed.

\subsection{Seven Stages in the Dialogue}

The dialogue is then entered into in detail.
This is done in eight stages, mainly corresponding to books by Carnap
and papers or correspondence from Quine critical of elements of one
book and influencing the character of the next.

The books are:

\begin{itemize}
\item[I] The Logical Syntax of Language \cite{carnap34,carnap37} (see also \cite{carnap35})
\item[III] Introduction to Semantics \cite{carnap42}
\item[V] Meaning and Necessity \cite{carnap47}, Empiricism, Semantics and Ontology \cite{carnap50}
\item[VII] The Philosophy of Rudolf Carnap \cite{carnap63}
\end{itemize}

The critiques from Quine which lead us in our dialogue from one
book to the next include:

\begin{itemize}
\item[II] Truth by convention \cite{quine36}
\item[IV] The 1943 correspondence on analyticity \cite{carnap90} and reservations about modal logic \cite{quine43,quine47}
\item[VI] ``On What There is'' \cite{quine51} and ``Two Dogmas of Empiricism'' \cite{quine51b}
\end{itemize}

Of course, this presentation of the structure of the dialogue somewhat oversimplifies
the intricacies, as will be seen when we get into the detail, but it serves
as a structure for the exposition.

\subsection{Purposes and Methods, Large and Small}

Having examined the dialogue in detail, I then pull back to consider its characteristics,
particularly the strength of the arguments, and the force of the rhetoric.

Carnap sought to facilitate the transformation of philosophy into a ``scientific discipline'', intending by this an \emph{a priori} discipline.
As an \emph{a priori} science philosophy might hope to emulate the rigour traditionally associated with mathematics and achieve a similar progressive establishment of durable knowledge instead of an ever changing melee of conflicting opinion.

The criticism of Quine was not addressed to this overarching purpose, but systematically undermined the most fundamental principles upon which Carnap sought to progress that purpose.

In the analysis of the debate around these issues I will be considering these discussion not just at face value, but as exemplars of philosophic method.
At each stage in the analysis we will be considering what each author is seeking to achieve and by what methods he pursues his objectives, so that at the end we can come to an assessment of the character of the dialogue, including the extent to which this meets the standards which Carnap promoted.


\section{Act I: Carnap's Logical Syntax}

\section{Act II: Quine on Syntax and Truth by Convention}

\section{Act III: Carnap on Semantics}

\section{Act IV: Correspondence on Analyticity}

\section{Act V: Meaning and Necessity}

\section{Act VI: Ontology and Dogmas}

\section{Act VII: Epilogue}

\section{Presumptions, Claims, Arguments, Rhetoric}

\chapter{Kripke's Metaphysics}

Kripke's philosophy came too late for there to be significant dialogue
between him and Carnap, and Kripke did not explicitly address the
philosophy of Carnap.

However, an important early aspect of Kripke's philosophy is the rejection
of the intimate connection between analyticity, necessity and the \emph{a priori}
which was central to Carnap's philosophy (though the former two were
technical terms of Carnap gave formal definitions while ``a priori'' was
not, the terms, ``empirical'', ``verifiable'' or ``confirmable'' being preferred).
An upshot Kripke's more complex view of the relationship between these concepts
was the re-affirmation of the synthetic-\emph{a priori} and the re-appearance of
metaphysics.

His work may therefore be seen, even by philosophers not wholly convinced
by Quine's rejection of the analytic/synthetic dichotomy as a final nail
in the coffin of logical positivism and the philosophy of Rudolf Carnap.

A reader who has come this far into the book will not be surprised to find
me rejecting that point of view.
I will be going in some detail into this new kind of metaphysics, but there
are very simple reasons for doubting that Kripke's fundamental insights have
the supposed impact on Carnap's philosophy, which I will sketch lightly
before considering the matters in greater depth.

My discussion of Kripke will be one primarily of his lectures
\emph{Naming and Necessity} \cite{Kripke72}, given in 1970, and is focussed
on the relationship of this work with the philosophical views of Carnap, on
the impact that the ideas have for the tenability of Carnap's ideas
and on the status which Kripke's ideas and the kinds of metaphysics which
they encourage might be expected to have from Carnap's point of view.

The short story on why Carnap need not have been concerned is as follows.

First of all we should note that Kripke's concern is with the English
language, though we might expect much of his dicussion to be equally
applicable to other natural languages (and the discussion does involve
the views of philosophers who wrote in other languages), there is no
apparent intent to generalise across all possible languages.
Many of the difficulties which he discusses are features of natural
languages which are not replicated in most formal languages.

Second, Kripke, does not appear to be using these technical terms
with the same meanings which Carnap attaches to them, so the conclusions
he draws about the relationships between the concepts are not
conclusions about the concepts as used by Carnap.
Kripke does not consider it necessary to give a precise account
of the meanings of the terms.
Whatever the merits of this approach may be, it is clear that there are
crucial differences between Kripke's usage and that of Carnap, the most
crucial on being that the concepts of necessity and analyticity are for
Carnap interdefinable.
\footnote{In \emph{Meaning and Necessity} \cite{carnap47} 39-1 p 174, Carnap defines necessity in terms
of analyticity.}

This might be regarded as a sleight-of-hand, but this is a carefully
considered position on Carnap's part, the merits of which by comparison
with the alternatives which we might infer from Kripke's discussion are
not our present concern but will be discussed later.
For the present preamble I am concerned with whether Kripke can be held
to have shown that Carnap was mistaken in identifying the three dichotomies,
which he cannot be unless he actually addresses the dichotomies about
which Carnap spoke.

Greater disparities can be seen between the concept \emph{a priori} as used
by Kripke and as used by many other philosophers.
Several difficulties arise here.

At first Kripke raises difficulties with modal characterisations of the
\emph{a priori} (along the lines ``can be known without appeal to experience'')
and proposes to dispense with the locution ``a priori truth'' in favour
of examining particular instances of knowing or believing \emph{a priori}.
So he demurs from clarifying his usage of the term \emph{a priori} in this
way, but then promptly abandons his resolution and continues to use the
term not only for these particular instances, but also as a property of
sentences which he distinguishes from necessity.

What Kripke is doing here, with this technical term from philosophy,
is judging particular instances as if it was a term of ordinary language
in the use of which we can all be regarded as competent without being
furnished with a definition.

However, the particular examples he addresses make clear that his usage
diverges from what many philosophers would expect, in the following two
respects.

The first is that the \emph{a priori} status of a proposition is generally
held to be determined by the kind of justification which would be expected
of anyone purporting to establish the proposition, and is not determined
by the manner in which any particular individual might have come to know
the proposition.

The second is that in assessing whether empirical evidence is required
to justify some proposition, empirical evidence which is required to establish
that some sentence under consideration does indeed express that proposition
must be disregarded.
Given that all (or at least, some) of our knowledge of even our native tongue
is acquired through the senses, and that the association between sentences
of that language and their meaning is contingent, and justification of such
a sentence will depend, if only implicitly, empirical evidence in establishing
its meaning before its truth can be established.
Hence, if this were to be taken into account, all claims in natural languages
would be \emph{a priori}.
Alternatively we may say that \emph{a priori} is a property or attribute of
propositions not of sentences, and it is only after establishing, howsoever it might be done.
the proposition expressed by a sentence that its status can be addressed
and should be addressed independently of the way in which the proposition
was established.

A third consideration in relation to the \emph{a priori}/\emph{a posteriori}
distinction is that, since it concerns the manner of \emph{justification},
we may consider that we have some discretion in chosing what we are to accept
in justification of a proposition.
It seems natural to expect that a proposition which is contingent,
i.e. does not have the same truth value in every possible world (situation or circumstance)
can only be shown to be true by furnishing some evidence that we are in a world
which meets its truth conditions, and will by that token be considered \emph{a posteriori}.
Similarly, if we can establish a proposition \emph{a priori} the that justification
must hold good for every possible circumstance, and the proposition must be necessary.
These two are contrapositives, reinforcing one side of a possible connection between
the \emph{a priori} and necessity.
The other half is less clear.
There is no apparent reason why we might not accept an \emph{a posteriori} justification
for a necessary proposition, but such a justification would establish only its truth,
not its necessity, and would fall short of the standard of justification which one
might expect for a necessary proposition.

Similar arguments to these can probably be found all the way back to Aristotle,
so why should we expect these to prevail against the modern arguments devised in
full knowledge of this context which appear to overturn them.

The main point I make here is that standards of justification are discretionary,
and when an objective result is purportedly obtained which not only is not grounded
in a standard of justification for which a case has been made, then the discussion
needs to be recast, introducing criteria of justification (going beyond consideration
of \emph{the meaning} of the term \emph{a priori} into the pragmatics of such criteria.

We will look in vain for such considerations in Kripke, and may therefore remain
unmoved from any initial disposition to adopt that mosts straightforward criterion
which results in a pragmatic identification of the \emph{a priori} with necessity.

It is my aim in what follows to work through Kripke's three lectures testing his
conclusions against these conservative considerations, bearing in mind the specific
interpretations of these terms which we associate with Carnap and comparing the
way in which the examples would be spoken of by Carnap with the Kripke's conclusions.
In this way we may discover whether there is anything in Kripke's arguments which
crosses the divergence in terminology.
If the apparent divergence in doctrine comes entirely down to divergent terminology
we must then consider the relative merits of the two ways of speaking about these
fundamental concepts.

The breaking apart of these dichotomies is a liberating prelude to the development
of Kripke's metaphysics.
If Carnap's position remains unscathed, how then does the metaphysics shape up
when read from Carnap's perspective?

We call this metaphysics primarily because it concerns necessity \emph{de re},
i.e. necessary propositions which are synthetic.
In Carnap's language there is no such thing, and so if we take an
argument from Kripke which establishes a conclusion which he reads as metaphysics.
Carnap will not necessarily dispute the result, he may find fault in the argument,
but if the argument is sound he will then take a different view on its status.
I will work through selected materials to expose these different ways of talking
about the same phenomena.

\chapter{Following Carnap}

\chapter{Epilogue}

\backmatter

\bibliographystyle{plain}
\bibliography{rbj2}

\clearpage
%\listoftables
%\addcontentsline{toc}{section}{List of Tables}

\clearpage
%\listofposition
%\addcontentsline{toc}{chapter}{List of Positions}

\twocolumn[
%\section*{Index}\label{index}
%\addcontentsline{toc}{section}{Index}
]
{\small\printindex}

\vfil

\end{document}
