\usepackage[T1]{fontenc}
\usepackage{textcomp}
\renewcommand{\rmdefault}{ppl}
\linespread{1.04}

\makeatletter
\def\cleardoublepage{\clearpage\if@twoside \ifodd\c@page\else
\hbox{}
\vspace*{\fill}
\begin{center}
\end{center}
\vspace{\fill}
\thispagestyle{empty}
\newpage
\if@twocolumn\hbox{}\newpage\fi\fi\fi}
\makeatother

\usepackage{fancyhdr}
\pagestyle{fancyplain}

%\pagestyle{headings}
\usepackage[twoside,paperwidth=5in,paperheight=8in,hmargin={0.5in,0.25in},vmargin={0.25in,0.25in},includehead,includefoot]{geometry}
\usepackage{tocloft}
\usepackage{tocbibind}
%\bodytext{BGCOLOR="#eeeeff"}
\makeindex
\newcommand{\indexentry}[2]{\item #1 #2}
\newcommand{\ignore}[1]{}

\fancyhfoffset[EL,RO]{0pt}
\newcommand{\aref}{}
\newcommand{\bookname}{}
\renewcommand{\chaptermark}[1]{\markboth{#1}{}}
\renewcommand{\sectionmark}[1]{\markright{#1}}
\lhead[\fancyplain{}{\thepage}]         {\fancyplain{}{}}
\chead[\fancyplain{}{\slshape\leftmark}]                 {\fancyplain{}{\slshape\rightmark}}
\rhead[\fancyplain{}{}]       {\fancyplain{}{\thepage}}
\lfoot[\fancyplain{}{\aref}]            {\fancyplain{}{}}
\cfoot[\fancyplain{}{}]                 {\fancyplain{}{}}
\rfoot[\fancyplain{}{}]                 {\fancyplain{}{\aref}}

\renewcommand{\headrulewidth}{0pt}

\title{Carnap, Quine and Kripke\\ \Large - \\ the rout of positivism}
\author{Roger Bishop Jones}
\date{\ }

\begin{document}
\frontmatter

\begin{titlepage}
\maketitle

\hspace{2in}

\vfill

\begin{centering}

{\small

Written and published by Roger Bishop Jones\\
www.rbjones.com\\

\vspace{0.2in}

ISBN-13: \\
ISBN-10: 

\vspace{0.2in}

}%small
{\scriptsize

First edition. \hfil Version 0.1 \hfil 2015-07-02

\vspace{0.2in}

\copyright\ Roger Bishop Jones;

}%scriptsize

\end{centering}

\thispagestyle{empty}

\end{titlepage}

{\parskip=0pt\tableofcontents}


\chapter*{Preface}\label{Preface}
\addcontentsline{toc}{chapter}{Preface}

\mainmatter

\chapter{Introduction}

\chapter{Carnap's Mission}

In the latter half of the nineteenth century witnessed the application of mathematics
to the domain of \emph{logic}, a subject previously of interest primarily to philosophers
in which the logic of Aristotle had prevailed for over 2000 years.
The resulting advances, involving a number of mathematicians and philosophers, were rapid and substantial,
completely transforming the character of the subject and making it sufficiently powerful for the first
time for the derivation of the theorems of pure mathematics and other \emph{a priori} disciplines.  

In his book \emph{Our Knowledge of the External World as a Field For Scientific Method in Philosophy} \cite{russell21} Bertrand Russell describes the revolutionary impact which he hoped these developments might have on the character and conduct of philosophy.

\begin{quote}
``The study of logic becomes the central study in philosophy: it gives the method of research in philosophy, just as mathematics gives the method in physics....	
All this supposed knowledge in the traditional systems must be swept away, and a new beginning must be made. . . .''
\end{quote}

Carnap attended lectures by Frege as an undergraduate, studied his work in greater depth as a postgraduate student
and was already an enthusiast for the adoption of the new logical methods in philosophy when he read Russell's
book.
He was inspired and later wrote in his \emph{Intellectual Autobiography} xshell
\cite{carnap63}:

\begin{quotation}
``I felt as if this appeal had been directed to me personally.
To work in this spirit would be my task from now on! And indeed henceforth the application of the new logical instrument for the purposes of analyzing scientific concepts and of clarifying philosophical problems has been the essential aim of my philosophical activity.''
\end{quotation}

In this chapter I describe the context in which the dialogue between
Carnap and Quine began.
The principal


\chapter{Quine on Carnap}

\nocite{carnap56,carnap63,copi67,quine53,quine61, quine61a,quine66,quine86}

The central startling feature of Quine's relationship with
Carnap is his rejection the analytic/synthetic distinction.
This chapter is devoted to an analysis of the debate around
that issue.
The issue is sufficiently fundamental that it could not fail
to be connected with other important problems.
It was so fundamental a difficulty for Carnap that we can
see its effects upon his philosophy from the time that Quine
first entered professional philosophy to the end of Carnap's
life.

The following analysis will track the trajectory of Carnap's
philosophy under this influence, the major features of which
are first sketched here, and later examined in greater detail.

Quine's first engagement with Carnap followed the publication
of Carnap's {\it The Logical Syntax of Language}\cite{carnap37},
which he had the privilege of reading as it was typed up by
Ina Carnap.

When Quine returned to Harvard he was invited to give three
lectures on Carnap's philosophy.  
At this stage in Quine's career it is generally held that
he was an uncritical enthusiast for Carnap's philosophy,
Quine himself was later to describe these lectures as ``sequacious''.
However, the first of these lectures, in which Quine is explicitly
not giving an account of Carnap's work, but rather providing some
backgound on ``the analytic character of the a priori'', there
are many points at which Quine's presentation diverges significantly
from what we might have expected from Carnap.
This first lecture is quickly expanded by Quine into the paper
{\it Truth by Convention}\cite{quine36}, suggesting that Quine
felt there to be some new ideas here, and indeed the paper is
billed by Quine as questioning received opinion.

The principal thrust of {\it Logical Syntax} is then abandoned by
Carnap as he embraces formal semantics, though this is not so
much a change of purpose or mission as a revised approach to realising
that purpose, scientific philosophy and positive science
(in Carnap's revised account of that Comtean conception).
The principal influence which Carnap later aknowledges in this
transition (from logical syntax to formal semantics) is the
semantic work of Tarski, but the move to semantics can also
be seen as responding to a misundertanding or mis-presentation
(or misrepresentation) of Carnap's position implicit in Quine's
first lecture and Quine's {\it Truth by Convention} \cite{quine36}
which followed hard on its heels.

Though Carnap might have hoped that his embracing of semantics
would render his account of analyticity more transparent and less
objectionable to Quine, there ensues, principally in the year 1943
an extended correspondence between Carnap and Quine \cite{carnap90}
which fails to secure a rapprochement.

At the same time, Quine is publishing doubts about modal logics.
For Carnap the concepts of analyticity and of logical necessity
are intimately connected, indeed mutually interdefinable, so
it is not surprising that Carnap's next major work addressed these
issues.

\section{The Structure of the Analysis}

\subsection{The Issues at Stake}

I begin by describing the issues at stake between Carnap and Quine,
first the central thrust of Carnap's philosophical mission contrasted
with such as can be said about that of Quine at this stage in his
development, then in a little more detail under the four headings:

\begin{itemize}
\item Logical truth
\item The analytic/synthetic dichotomy
\item Necessity and modal logic
\item Ontology and metaphysics
\end{itemize}

The intention is to sketch before entering in to the detail of the dialogue
the signficance of the topics for Carnap's philosophical agenda, and
to give a rationale for the areas in which the analysis of the dialogue
will be focussed.

\subsection{Seven Stages in the Dialogue}

The dialogue is then entered into in detail.
This is done in eight stages, mainly corresponding to books by Carnap
and papers or correspondence from Quine critical of elements of one
book and influencing the character of the next.

The books are:

\begin{itemize}
\item[I] The Logical Syntax of Language \cite{carnap34,carnap37} (see also \cite{carnap35})
\item[III] Introduction to Semantics \cite{carnap42}
\item[V] Meaning and Necessity \cite{carnap47}, Empiricism, Semantics and Ontology \cite{carnap50}
\item[VII] The Philosophy of Rudolf Carnap \cite{carnap63}
\end{itemize}

The critiques from Quine which lead us in our dialogue from one
book to the next include:

\begin{itemize}
\item[II] Truth by convention \cite{quine36}
\item[IV] The 1943 correspondence on analyticity \cite{carnap90} and reservations about modal logic \cite{quine43,quine47}
\item[VI] ``On What There is'' \cite{quine51} and ``Two Dogmas of Empiricism'' \cite{quine51b}
\end{itemize}

Of course, this presentation of the structure of the dialogue somewhat oversimplifies
the intricacies, as will be seen when we get into the detail, but it serves
as a structure for the exposition.

\subsection{Purposes and Methods, Large and Small}

Having examined the dialogue in detail, I then pull back to consider its characteristics,
particularly the strength of the arguments, and the force of the rhetoric.

Carnap sought to facilitate the transformation of philosophy into a ``scientific discipline'', intending by this an \emph{a priori} discipline.
As an \emph{a priori} science philosophy might hope to emulate the rigour traditionally associated with mathematics and achieve a similar progressive establishment of durable knowledge instead of an ever changing melee of conflicting opinion.

The criticism of Quine was not addressed to this overarching purpose, but systematically undermined the most fundamental principles upon which Carnap sought to progress that purpose.

In the analysis of the debate around these issues I will be considering these discussion not just at face value, but as exemplars of philosophic method.
At each stage in the analysis we will be considering what each author is seeking to achieve and by what methods he pursues his objectives, so that at the end we can come to an assessment of the character of the dialogue, including the extent to which this meets the standards which Carnap promoted.


\section{Act I: Carnap's Logical Syntax}

\section{Act II: Quine on Syntax and Truth by Convention}

\section{Act III: Carnap on Semantics}

\section{Act IV: Correspondence on Analyticity}

\section{Act V: Meaning and Necessity}

\section{Act VI: Ontology and Dogmas}

\section{Act VII: Epilogue}

\section{Presumptions, Claims, Arguments, Rhetoric}

\chapter{Kripke's Metaphysics}

\chapter{Following Carnap}

\chapter{Epilogue}

\backmatter

\bibliographystyle{plain}
\bibliography{rbj2}

\clearpage
%\listoftables
%\addcontentsline{toc}{section}{List of Tables}

\clearpage
%\listofposition
%\addcontentsline{toc}{chapter}{List of Positions}

\twocolumn[
%\section*{Index}\label{index}
%\addcontentsline{toc}{section}{Index}
]
{\small\printindex}

\vfil

\end{document}
