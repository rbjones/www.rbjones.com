\chapter{Introduction}

In this volume I consider the question ``What should we do?'', in a manner which I think of as ``philosophical'', without attaching any definite meaning to the term ``philosophical''.
Some philosophers who I respect, Bertrand Russell and Rudolf Carnap have had quite definite ideas on what philosophy should be, but these seem to me to be ideas about particular kinds of philosophy which they thought specially important, varieties of philosophical analysis, which, by themselves seem to me inadequate even for the kinds of philosophical enterprise of interest to me, let alone the legitimate interests of philosophically minded men who have grown from very different cultural roots.

In this first chapter I want to give a preliminary sketch ranging over the whole topic, on which I will later elaborate.

The method is loosely analytic, but nevertheless, the primary thrust is a kind of synthesis.
A picture is being put together.
A part of that picture will be an analysis of analytic method, in which it will be made clearer in what ways we can be both analytic and synthetic at once.
Very often I will seek higher ground from which to examine the progress of the work.
Philosophy is refective and metatheoretic on many levels.

Many things may be subjected to analysis.
Language is one of those things.
Some say that all other things can be analysed and all wisdom may be found through the analysis of language, for language goes into everything.
By contrast, I find that the analysis of language can impede insight into its subject matters, and in any case is less interesting than some of the non-linguistic matters of which it may speak. 

Before beginning the analysis of \emph{the problem} which draws out this philosophical system, some clarificatory discussion may be helpful of \emph{the question} which provides our problem.
The sentence ``What should we do?'' is context sensitive, the context determining the group to whom the pronoun ``we'' refers. and perhaps also some shared purpose or ideal which it is hoped to realise.
My intention here is that the group in question is humanity as a whole.
However, it is not humanity should act as a group that is the primary concern, but how each human individual should act.
The conduct of global institutions enters into the problem indirectly, through the ways in which individuals can influence such institutions.

A second point of clarification concerns how the word ``should'' is here intended.
In asking what we should do, we might be enquiring about moral obligations, we might be asking about other kinds of obligation, or we might simply be discussing purely discretionary matters.
``What should we do today?'' might be the prelude to a discussion of personal preferences.
It is my intention here that the question should be taken in that latter most liberal sense.

This does not however exclude from consideration such obligations or moral imperatives as impose upon us.
These will naturally influence out choice of action.

With these points behind us let us now consider two ways in which an answer to our question might be approached.

On the one hand, we might respond as purposeful individuals, identify some desirable state of affairs, and consider what course of action most likely to realise that desirable end.
On the other we might respond as hedonists, considering possible courses of action on their own merits as enjoyable ends in themselves, and consider among the possibilities those from which we will get the greatest pleasure.
Of course, we do not have here a dichotomy, mixing and blending is not out of the question, and each of these two extreme cases may involve the other.
For example, even the most extreme hedonist is likely to be willing to undertake purposeful activities which may be essential preliminaries to the pleasures he seeks.
Generally pleasures cost money, and the hedonist will find it necessary to obtain funding by fair means or foul, for his pleasurable pursuits.
The purposeful individual who wants to realise a better society, will likely in his conception of that society provide for the happiness and well being of its citizens.
The difference between the two might simply be in how far sighted their deliberations and the actions based upon them prove to be.

We find therefore, that the discussion of what to do will depend both on determination of what ends are desirable, and of how various courses of action influence the likelihood of realising these ends.
At this point we may introduce the philosophical distinction between values and facts. and note that philosophy itself may be considered as divided into two parts, practical philosophy concerned with matters involving values, and theoretical philosophy concerned with objective non-evaluative knowledge.

