% $Id: rationality.tex,v 1.3 2008/07/11 18:16:09 rbj Exp $
\chapter{Rationality}\label{Rationality}

In this essay I want to consider the nature of rationality and its relevance to utopian philosophy.

Rationality is an attribute of beliefs, behaviours and plans or courses of action.

We can reduce the latter to the former by first observing that questions of rationality only arise for behaviours or action plans, if we have some goal in mind.
In that case a plan may be said to be rational if it is rational to believe that it is the plan most likely to achieve the goal.
Our motivations are rarely singular, it is rational to take account in striving for one goal how our actions may affect other things we care about.
Rational behaviour cannot therefore be judged simply by its relation to its ostensible purpose, it must be judged in a more holistic manner, in the light of the impact of the behaviour on our life as a whole.
Let us proceed however, for the time being, on the presumption that when a sufficiently broad understanding of our motives and purposes is at hand the rationality of our actions can be reduced to the rationality or our believing them efficient in securing our ends.

This manner of reducing rationality of actions to rationality of beliefs gives us a clue to the relevance of rationality to utopian philosophy.
It is because of the connection between rationality and efficacy.
If we wish to progress, then our actions should be rational.

Rationality then, at this first level of analysis, concerns us not with the utopian vision, but with how we seek to realise that vision.
We need only scratch the surface to see that this clean separation of concerns is illusory.
The what and the how are inextricably intertwined by the desire that our ideal be realisable, and by the expectation that progress is continuous, and that the fabric of the future is woven from the ways in which we progress toward the future.

Nevertheless, I want to hang on to this idea that rationality is coupled with means rather than ends, and that rational means are desirable because they are most likely to be effective.
This gives a whole dimension of utopian philosophy which is largely independent of what shape we imagine utopia to take, and on which we can therefore hope for a very broad range of consensus.

The plan for the rest of this essay is then firstly to consider in more detail what is rational, on the presumption that rationality is a good thing.

Secondly to consider in the light of that more detailed analysis of the nature of rationality, to what extent the world is rational.
Here we are concerned not only with the beliefs and behaviours of individuals, but also of social groups at all levels.
Wherever groups can be said collectively to endorse or promote beliefs or doctrine, or whenever groups undertake actions to secure ends, they can be judged as doing so rationality or otherwise.
This is a part of our understanding of the world we live in.
It is relevant to our conception of what changes we may desire and to our conception of what means will suffice to secure those changes.

Thirdly, I think it desirable to consider the consequences of such irrationality as we discover.
Is this a big problem which causes or contributes to features of our present world which we would dearly like to change, or is it a minor problem and concern with rationality a pedantic irrelevance.

Finally I propose to consider whether and how we might hope to change individuals and institutions so that they are more rational.
