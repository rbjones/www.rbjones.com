% $Id: rationality.tex,v 1.1 2007/01/24 15:32:53 rbj01 Exp $
\chapter{Rationality}\label{Rationality}

In this essay I want to consider the nature of rationality and its relevance to utopian philosophy.

Rationality is an attribute of beliefs, behaviours and plans or courses of action.

We can reduce the latter to the former by first observing that questions of rationality only arise for behaviours or action plans, if we have some goal in mind.
In that case a plan may be said to be rational if it is rational to believe that it is the plan most likely to secure goal.
Our motivations are rarely singular, it is rational to take account in striving for one goal how our actions may affect other things we care about.
Rational behaviour cannot therefore be judged simply by its relation to its ostensible purpose, it must be judged in a more holistic manner, in the light of the impact of the behaviour on our life as a whole.
Let us proceed however, for the time being, on the presumption that when a sufficiently broad understanding of our motives and purposes is at hand the rationality of our actions can be reduced to the rationality or our believing them efficient in securing our ends.

This manner of reducing rationality of actions to rationality of beliefs gives us a clue to the relevance of rationality to utopian philosophy.
It is because of the connection between rationality and efficacy.
If we wish to progress, then our actions should be rational.

Rationality then, at this first level of analysis, concerns us not with the utopian vision, but with how we seek to realise that vision.
We need only scratch the surface to see that this clean separation of concerns is illusory.
The what and the how are inextricably intertwined by the desire that our ideal be realisable, and by the expectation that progress is continuous, and that the fabric of the future is woven from the ways in which we progress toward the future.

Nevertheless, I want to hang on to this idea that rationality is coupled with means rather than ends, and that rational means desirable because they are the ones most likely to be effective.
This gives a whole dimension of utopian philosophy which is largely independent of what shape we imagine utopia to take, and on which we can therefore hope for a very broad range of consensus.

The plan for the rest of this essay is then firstly to consider in more detail what is rational, on the presumption that rationality is a good thing.

Secondly to consider in the light of that more detailed analysis of the nature of rationality, to what extent the world is rational.
Here we are concerned not only with the beliefs and behaviours of individuals, but also of social groups at all levels.
Wherever groups can be said collectively to endorse or promote beliefs or doctrine, or whenever groups undertake actions to secure ends, they can be judged as doing so rationality or otherwise.
This is a part of our understanding of the world we live in.
It is relevant to our conception of what changes we may desire and to our conception of what means will suffice to secure change.

Thirdly, I think it desirable to consider the consequences of such irrationality as we discover.
Is this a big problem which causes or contributes to features of our present world which we would dearly like to change, or is it a minor problem and concern with rationality a pedantic irrelevance.

Finally I propose to consider whether and how we might hope to change individuals and institutions so that they are more rational.

\section{The Nature of Rationality}

If the rationality of a course of action depends on there being reason to believe that it will be effective in securing its purpose, rationality hinges upon the whether our beliefs about the consequences of our actions are reasonable.

In order to have reasonable grounds for predicting the effects of our actions we must have, in some very general sense, a {\it predictive model} of the relevant aspects of the world.
Rationality therefore hinges upon certain kinds of knowledge from which we are able to draw reasonable conclusions about the future effects of our actions.

In some cases the knowledge on which action is based may be independent of language in any familiar sense of that word.
A fox may use knowledge of the layout of his local countryside to corner his prey without having any way of expressing that knowledge more explicitly than in the behaviour in which it is manifest.
A mother's knowledge of how to divine the cause of her child's distress, or the knowledge which underlies many other skills, may be just as ineffable.
In these cases however, belief and reason can be said to play little part, rational though the actions may be, and when things go wrong we are more likely to describe the failure as a lapse of judgement than as a case of irrationality.

At another extreme we may consider a game of chess.
Here the players act in an idealised world in which they have a perfect model for predicting the consequences of their actions.

To behave rationally we need to have knowledge of the world which permits us to chose between courses of action on the basis of a reasonably reliable prediction about their consequences.
Sometimes that knowledge may come from our own experience, but often it will have come from the experience of others.
The dissemination of knowledge is most effective through language.

Knowledge may be prescriptive and procedural, but when so expressed it is specific to some task.
When expressed descriptively knowledge though specific to some subject matter, may be to some degree independent of purpose, and hence more general than prescriptive knowledge.
For language to work the rules of language must be fixed and understood between those who speak and those who listen.
For descriptive language the rules are most abstract in a semantics for the language, and the kind of semantics which is most relevant is a truth conditional semantics.

A truth conditional semantics can usefully be factored into an abstract semantics and a correspondence between the abstract entities in terms of which that semantics is given, and the features of the world of which those entities can be said to be models.

Once abstract semantics is separated out, we can develop logic and mathematics, which provide us with advanced re-usable technology for building models.

\section{Are we Rational?}

Judgements about rationality are not straightforward.
To know whether an act is rational we must know the purposes of the agent, and we must know what he knows.
We may not be able to trust what he says about his purposes (it may be rational for him to lie about them).
There may be a case, when we look into the facts of nature for claiming that our purposes are not only other than we claim them to be, but even other than we suppose them to be.
It might be reasonable to claim that our purpose is to maximize the proliferation of our genes, or it might be reasonable to regard a persons actions as a better indication of his purposes than his words.

Despite these difficulties there remain reasonable grounds for doubt about prevalence of rationality.

Let me first rule out the cases where our purposes are alleged to be other than we suppose them to be.
The conception of rationality which is most relevant in this context is that in which we are consciously formulating goals and acting to realise those goals.
The question is then whether the actions we chose are the best ways we know to realise those conscious goals.
If our ``true'' purpose is different, that has no bearing on the rationality of our course of action, which is concerned only with its relation to the intended purpose (not to some purpose of which we are unconscious).

In these sanitized conditions, where we know what we are after and account is taken of our limited knowledge of the effects of the various courses of action we might take, the score is probably not too bad.

Two factors predominate in preventing  what an impartial observer might think a doubtful decision from being irrational.
The first is that very often our actions are purposeless, we have not consciously thought about what we hope to achieve by them.
Most people are mostly jockeying for position in their peer group, a substantial influence on their behaviour is what others are doing, at least those who rank high in that group.
The second is that when we do know what we want, the information on which our choice of action is based may be sparse or unreliable.
This particularly applies when it comes to spending money.

Knowledge about what you get for your money is polluted by a couple of huge factors.
The financial incentives which competitors for your funds have for you to believe anything which will persuade you to buy their product.
The second is the peer thing again.
Information is only made available if it makes a difference, when most people don't ask for technical specifications before they purchase (they just go on the look of things), the few who would like to check out the spec find them hard to come by.

\section{Why Irrationality is a Bad Thing}

\section{How to Make Things Better}
