% $Id: semantics.tex,v 1.2 2008/07/11 18:16:09 rbj Exp $
\chapter{Foundations}\label{Foundations}

Semantics is concerned with the meaning of language.

In this chapter I want to discuss some of the choices we have about language and semantics.

My topic in this essay is descriptive language.
My purpose is to put forward some ideas about how descriptive language {\it should} work.

This purpose is utopian, I assert that we have a choice, and propose to make a recommendation about how we should exercise that choice.

We have choices both about what language to use, and about how to use the chosen language.
There is no clear division between these two kinds of choice, because there is no definite point at which eccentric use of a natural language turns into the use of some other strange language.

In our choice about what language to use, we may chose either to use a natural language, in which case we will often have further choices to make about detailed usage, or or we may chose an artificial or formal language.
In the former case, while still speaking in English perhaps, we may import semantic precision by reference to formal systems.
I may speak informally but precisely in plain English about sets, having first made it clear that the sets of which I speak are the elements in the domain of a model of the first order axiomatization of set theory known as {\it ZFC}.
My usage of set theoretic vocabulary acquires precision from being coupled with the meaning of the formalized concepts. 

\section{Notes}

Abstract ontology and semantics.

Logical partitions of things and sentences.

Logical truths.

Factual ``truths'', models of reality, claims about models.


\section{Semantics and Models}

When we use language among ourselves to communicate information about the world, we do so in the context of some model of the world (or part of it).
This model might be built into language, or it might have been separately constructed, perhaps by a scientist, an architect, a schoolteacher.

The model is not always (perhaps not ever) wholly faithful to the aspect of the world which it represents, and by its own infidelity places constraints on how accurately, truthfully, we can speak using it.
When we talk of the world in this way truth is not black and white.
In some cases it might be, that from a divine perspective one might be able to separate out infidelities in the model from infidelities in its use.

This idea comes from Plato of course, it is akin to Platonic idealism, but differs in locating the models in the semantics of language, rather than in an ontological realm of platonic forms.
These models of aspects of our world in the semantics of our languages, reflect a necessity of knowing or presupposing something about our subject matter before we can begin to speak of it.