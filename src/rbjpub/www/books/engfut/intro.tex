% $Id: intro.tex,v 1.1 2007/01/24 15:32:53 rbj01 Exp $
\chapter{Introduction}\label{Introduction}

I introduce here a small collection of philosophical essays.
I intend them to form a philosophical work small in scale but broad in scope, in these respects similar to David Hume's {\it An Enquiry Concerning Human Understanding} \cite{humeECHU}, Bertrand Russell's {\it The Problems of Philosophy} \cite{russellPP} and Ludwig Wittgenstein's {\it Tractatus Logico-Philosophicus} \cite{wittgenstein1921}.

I have sought to unify the essays under a single purpose.
That purpose is utopian, but realistic.
My aim is to construct a systematic philosophical world view, starting from the enquiry:

\begin{quote}\label{Q1}
How ought the world to be?
\end{quote}

In attempting to address this question I will be concerned with values and ends, with how things are and how they might be, with ways and means of making choices about the future and of acting effectively upon them.

The project is {\it utopian} in its concern with the future, and its implicit interest in {\it the best possible} future.
But realistic rather than idealistic, concerned only with those futures which we might possibly realise.
I will be concerned with methods for developing and realizing conceptions of utopia.
From my most crude statement of method may be derived first connections with my other principle subject matters, {\it rationality} and {\it evolution}.

The method I propose is, first, to give consideration to what is desirable, consideration to what is possible, and then to select from what is possible that which is most desirable.
In considering what is possible I shall advocate that we be rational.
To investigate rationally the possibilities for our future, we must begin with some knowledge of our present situation, and particularly upon the dynamics of the reality we inhabit, how it changes.
This leads us to evolution, which, for present purposes, is to be considered a loose term covering the development of a wide variety of complex systems.

Rationality concerns some of the most fundamental and most general methodological resources for solving practical problems without prejudice.
Rational methods are value-neutral, supporting the realization of whatever ends we may adopt.
However, the effects of mere irrationality on a global scale are morally pernicious, and rationality therefore becomes possibly the single most important means to realizing a good society.

Rationality is not only value neutral, it is neutral with respect to the facts.
Whatever the facts, and whatever our values, rationality will help us make the best of things.
{\it Evolution} is the heading under which I discuss {\it how things are}.
The emphasis is placed on how things change, and on long term processes of change.
An understanding of how things change is, however, parasitic upon how things have been, and may tell us something about how they will be.
Evolutionary theories have been around for almost as long as history, and have often been used to show what we cannot be.
A modern example is found in the idea of {\it The Selfish Gene} \cite{dawkinsSG}.
To find the best possible future we must judge the credibility of constraints inferred from evolutionary theory, and I will devote some space to rejecting some of these skeptical theories.

Beyond these three major themes I want to introduce here another three threads.

The first thread takes us beyond rationality into {\it faith}, without which the way forward may seem a hopeless struggle.
I am not myself a religious person, my own faith is not in God.
The role of faith and its reconciliation with the imperatives of rationality is therefore for me an interesting and important question.
It is also a question, I believe, on which the credibility of my philosophy depends, for no conception of the future can be true which provides no basis for faith and commitment.

A second minor thread concerns choice and destiny.
Throughout this work I hope to draw out the choices which are open to us, not just in the future shape of our world and character of our society, but even in philosophical fundamentals such as the meanings of our concepts and languages.
Nevertheless, alongside faith and despite pervasive freedom of choice we may still nourish a sense of destiny (the utopian counterpart of dystopian fatalism).

My final thread is once again a problem of reconciliation.
The reconciliation of personal freedom with social involvement.

\section{Utopia}

Be clear that my opening question is loosely put, and that part of my problem is to refine the question. 
In particular, I have no definite intent that `ought' should be taken in any moral sense.

Jointly and separately we have choices to make (or evade) about our future, and about our present actions, with a greater or lesser awareness of how those choices may affect our future.
Each of us individually will have little effect on global issues, though they may have a profound effect on us.
Those of us who have greatest effect will most likely achieve that effect through our influence on the beliefs and behaviour of others.

These factors make Utopian writing as a statement of person preferences about the future much less promising an activity than as the search for a vision of the future which might command broad support.

Nevertheless, my conception of utopian philosophy is rooted in choice, and is perhaps distinguished by that feature.
It is perhaps surprising how rarely even libertarian political philosophy recognizes that the future is ours to chose.
Robert Nozick \cite{nozickASU}, though a champion of personal liberty, constructs arguments which purport to show that {\it only} a ``minimal state'' can be justified.
Among the liberties which libertarians seek to protect, the right to chose, even if by democratic process, the scope of activity by the state does not number.

There are difficulties for utopian thinkers in recognizing the rights of others to participate in shaping the future.
Karl Popper has argued \cite{popperPOH} that ``utopian engineering'' inevitably leads to totalitarian political systems, and should be eschewed in favour of ``piecemeal engineering''.
In this Popper creates a false dichotomy.
He presumes that one must either have grand all embracing ideas about the future, or else proceed by incremental change.
But the end and the means are not so coupled.
Utopian thinking does not prejudice the manner of change.
The grandest visions might possibly be realized by evolutionary change.

Popper underpins his attack on utopian engineering with arguments against ``historicism'', broadly speaking the view that the future of society can be predicted.
But we can accept severe limitations on our ability to forecast the future, and still find it worth our while to chose those actions today which seem most conducive to our preferred future.
Chaos theory tells us that our ability to forecast climate by detailed simulation of atmospheric change is negligible.
But despite the chaos, we can still predict with very high probability that summer will be warmer than winter, and with reasonable confidence that large scale emission of CO$_2$ will lead to global warming.
Whatever our level of confidence in scientific predictions about global warming, a choice has to be made.
Do we act to restrain emissions, or not?

Single issues more easily grab the headlines than utopian scenarios, and it is easier to imagine that we know the choices which face us in relation to these issues.
But our future well-being does not depend solely upon a small number of clear cut conspicuous problems.
It is rational to look for the big picture, and to invest in thinking about where we want to be and how we can get there.

Attempting to answer this question is utopian in the following moderate sense.
The question suggests that there is more than one possible future for this world, and that it is possible to distinguish some of these possibilities as preferable.
The way the world ought to be, surely is, the best of all the ways it might possibly be.
``possible'' here must not be taken logically, many logically possible worlds are not possible futures for this world.
My question is at once {\it utopian} and {\it realistic}.

Let me now begin to explain how I intend to approach the question.
Talking of choosing among possible worlds conjures up images of possible worlds as entities like billiard balls, which we can seize upon and compare individually.
Answering the question is then analogous to judging a beauty competition.

Possibilities are not like billiard balls, they are harder to get a grip on, and can rarely be grasped as individuals.
We deal with them through descriptions, which slice through the space of possibilities separating those which satisfy the description from those which do not.
The same consideration applies to the world as it now is.
To reason about it, and to discuss its future, we must work with descriptions.
These descriptions will not uniquely identify any possible world.
The odds are they will be not entirely accurate, and therefore, among the possible worlds which satisfy the description, this actual world will not number.
I therefore seek descriptions of the future which are consistent with the present and which distinguish those possible futures which seem to me preferable.

This consideration of the future seems to me to be a rational activity, it is desirable that it be done in a rational manner, and it is desirable that in certain important respects that future utopian society itself be rational.
For these reasons considerations relating to {\it rationality} will form a part of the discussion.

\begin{description}
\item[Utopian Engineering] methods
\item[Markets, democracy and evolution]
\item[Globalization]
\end{description}

\section{Rationality}

Rationality is concerned with {\it efficiency}.

A person behaves rationally if his actions are, so far as he knows, those best calculated to secure his ends.

A person believes rationally if his beliefs are consistent with the evidence available to him, and he has taken appropriate pains to ensure that he is well informed.
Rationality is connected with efficiency, for beliefs are more likely to be true if rational, and behaviour is more likely to secure its intended purpose if calculated in the light of true rather than false beliefs.

To judge whether behaviour is rational, we must know how it is motivated.
Much apparently irrational behaviour is motivated by the impression it is intended to create upon others.
If I buy a car more expensive than I need for transportation, my act may nevertheless be rational.
If I am motivated to appear successful and wealthy, an expensive car may be the rational choice.

A judgement about rationality can therefore take place only when values and purposes have been accounted for.
It is then a judgement about whether a course of action is likely to secure the desired end.

A belief is rational if it is supported by the evidence.
However, the rationality of some act does not depend solely upon the evidence for beliefs on which the effectiveness (in realizing the desired end) is predicated.
It depends also on what is at stake, on the consequences of failure.
I can rationally accept and act upon the word of my five year old child about where in the house I left my car keys, since the only adverse consequence is that I waste a few seconds following his lead before, perhaps, having to resume my search.
But a life and death decision would not rationally be based on such tenuous evidence if there were any prospect of more conclusive evidence.

\begin{description}
\item[Language and Logic]
\item[Rational Epistemology]
\item[Economics and Truth]
\item[Knowledge Engineering] 
\end{description}

\section{Evolution}

To decide on our future, we must know what futures are possible.
To discern the future we must know how the world is, how it is changing, how it might change.
To know how the world is changing we must know how it has been and how it did then change.
The statics and dynamics of our world, its history and evolution, and are inseparably interwoven.
Our path from past through present to future is a continuous fabric woven from these threads, warp and weft, evolution and history.

In a rational attempt to secure the kind of future we desire, a knowledge of the past will be of assistance.
Particularly helpful will be a knowledge of how, in the past, the world has changed.
A lot of that change has been ``evolutionary'', indeed, to the extent that this term covers the whole of biological evolution, which has been highly discontinuous in character (often as a consequence of environmental trauma) it is not certain that much change is excluded from consideration as evolutionary.

Biological evolution has given us intelligence, and intelligent society provides an ideal medium for social or cultural evolution.
Cultural evolution knows no bounds, and certainly embraces the kind of utopian social change which I consider here.

Cultural evolution is of course, not at all the same as biological evolution, which is itself not a single uniform phenomenon, but a diverse process itself continuously evolving.

In consideration of the relevant of evolutionary insights and theory to utopian thinking it is desirable first to examine skeptically the ideas which appear most severely to limit our possibilities.
Our prospects of realizing the best for ourselves are poor in the face of ill-founded skepticism.

\begin{description}
\item[Evolutionary Fallacies]
\item[Varieties of Evolution]
\item[Evolutionary Engineering] 
\end{description}

\section{Reason and Faith}

It is rational to withhold judgement where evidence is inconclusive.
Faith is belief in the absence of rational justification.
Those who consider rationality is progress, must surely leave faith behind.

I don't think so.
Whenever we aspire to great things, we reach out to achieve the improbable.
We struggle to achieve the unachievable.

\section{Choice and Destiny}

In utopian thinking, in fundamental philosophy, in the personal problems of day to day life, choice is pervasive.

In utopian thinking the choice about when to make a decision is important.
Many choices properly belong to the future not the present, but some choices must be made today if they are to be effective.

In philosophy the failure to recognize a choice when it presents itself can be a source of fruitless controversy.
Russell thought that the nature of number had not been discovered until the 19th century, and consequently that mathematics had until that time been ill-founded.
Russell's account of the nature natural numbers now appears to many philosophers as an arbitrary choice which need never have been taken.
In relation to real numbers there was a substantive issue to be resolved (whether or not to admit infinitesimals) and it was valuable to discover (though this happened well before Russell) that infinitesimals could be disposed of and thus to make more definite the concept of real number.

In my conception of rationality, which looms large in my utopian theories, many important features flow from pragmatic choices about how to use language.
The answers which I reach on fundamental issues in metaphysics, semantics, philosophical logic, epistemology and elsewhere depend upon the choice of particular conceptual schemes.

In the larger questions about society, for example the role of the state and the rights of the individual, I do not argue the necessity of any particular arrangement, but argue for a pluralistic society in which progress is driven by constructively diverse views on what should be done and how.

Nowhere is choice more important than in our own personal lives, but here it may be both boon and blight.
Choice is an agony which we may yearn to resolve.
There is a natural progression from a period of uncertainty leading to the making of a choice and on to the actions which implement that choice.
Of small choices there is in life an endless flow, but many of us yearn for a sense of purpose which spans a lifetime.
This yearning may spawn an angst which dissolves, if at all, into a sense of destiny, to which we surrender in relief.

\section{Man and Society}

It is a paradox of human nature that we desire personal freedom, but find our fulfillment only through others.
A convincing prospectus must weave both these threads into its fabric.