% $Id: philosophyn.tex,v 1.1 2007/01/24 15:32:53 rbj01 Exp $
\chapter{Utopian Philosophy}\label{UtopianPhilosophy}


\section{Choice}

First mention motivating philosophical fallacies:
\begin{enumerate}
\item asking questions which don't have answers
\item supposing that questions about language have definite answers
\end{enumerate}

Then mention some important kinds of choice:

\begin{itemize}
\item concepts and their meaning
\item languages and their semantics
\item models of reality
\item value systems
\end{itemize}

\section{Values}

To make a choice we need to know what are our options, and preferably, some basis for judging which is the best.

\section{Creative}

Writing books is always creative, it would be strange to attempt to write a book which had already been written.
Often however, the creativity may be thought to lie in the selection, organization and presentation of the subject matter of the book, which will consist of some body of knowledge itself discovered rather than created.
Truth is eternal.

In creative arts there may be a very real sense that an artist is doing something completely new, and that his work is primarily an act of creation rather than of discovery.

Mathematics is predominantly thought of as exploring an infinitely extended domain of timeless truths, but the discovery of its theorems is preceded by choice of concepts.
Such choices make the development of new branches of mathematics seem more like creativity than discovery.
However, these choices are a very tiny part of mathematics.
A tiny part even of the development of new branches of mathematics, which is dominated by the development of the theory once the fundamental concepts have been fixed.
Most students of mathematics, up to degree level, will learn about mathematics as it has been developed, and about how to apply that mathematics, but will never be involved in creating new mathematics.

In philosophy language arguably has a greater significance than it does in mathematics, though this has not been appreciated until recent times.
Even now that language has come center stage, there has been little recognition of discretion in the meaning of concepts.

\section{Old Stuff}

This book exemplifies an approach to philosophy which I proposed to call ``utopian philosophy''.
This chapter provides a first account of what utopian philosophy is intended to be.

There are four main concerns addressed by the proposed method:-

\begin{description}

\item[Purpose]\ 

The starting point of utopian philosophy is the desire to approach all philosophical problems in the context of a sense of purpose and value, and the denial that knowledge should be pursued for its own sake irrespective of utility.

\item[Choice]\ 

Many choices influence the direction of a philosophical enquiry.
These include the choice of purpose and values in the context of which the enquiry is directed.
They include choice of concepts and language and their precise meaning and usage.
These choices about language have profound effects upon philosophical conclusions, affecting logic, metaphysics science, ethics, political theory and utopian thought. 

\item[Values]\ 

My concern will be not only to consider what choices are open to us, but also to consider the respective merits of the available alternatives.

\item[Utopia]\ 

It is proposed to enter into philosophy (in its fullest generality) through the consideration of the grandest choices which we can contemplate, choices about the future of our world.
Most other significant question are then encountered along the way.

\item[Personalisation]\ 

By contrast with Plato's method, considering how society should be from the point of view of the ``philosopher king'', who has supreme wisdom and absolute power, I propose to consider the utopian problem from the point of view of ordinary (as well as extraordinary) people.
The problem is not just to consider the merits of possible global futures, but also to consider for each individual how he personally can best act to promote his preferred futures, giving global issues their proper significance for that individual.

\end{description}

\subsection{Purpose}

\subsection{Choice}

Before the twentieth century philosophers had relatively low levels of awareness of the significance of language to the problems they were investigating, and language formed a small part of what philosophers thought of as their subject matter.

Consequently, philosophers sometimes unwittingly integrated into their theories linguistic innovations in the light of which commonplace beliefs appeared to be incorrect.
Sometimes philosophers made use of special terminology of uncertain meaning.

\section{Older Stuff}
I'd like to characterize this volume as a whole as {\it utopian philosophy}.
What I have in mind is very close neither to the utopian writing nor to the those kinds of philosophy with which I am acquainted.
The mere juxtaposition of these terms helps a little to move each in the right direction, and the rest of this essay is devoted to giving the reader a fuller account of what I intend by the term utopian philosophy.

Let me begin with a few words about how these two words pull together for me.
The word utopian has two associations which I want to disavow for my present purposes.
The first is of unrealistic idealism.
The second is with detail and fantasy (again an element of unreality).

To talk of utopian {\it philosophy} helps a little to distance us from the expectation of a lot of concrete detail.
It is principles which interest us, preferably abstract or philosophical principles.

To talk of philosophy as utopian (rather than of philosophical utopianism) is intended to place philosophy in the context of some of the largest and most important social issues.
It is my hope that fundamental philosophical insights can be seen to be relevant to these larger issues, and that doing philosophy in that context can be fruitful.

\subsection{Knowledge for Its Own Sake}

Godfrey Harold Hardy (1877-1947), a Cambridge mathematician, wrote an unapologetic short book entitled ``A Mathematician's Apology''  \cite{hardyMA}, in which he defended the pursuit of pure mathematics without regard for utility.
This is probably the best documented, and perhaps the most extreme example, of an attitude towards knowledge which is pervasive in academe, that knowledge is a good thing in itself, regardless of any utility it might have.

Not everyone is like Hardy, who was possibly at one extreme of a spectrum on which I find myself at an opposite.
This might be thought of as reflecting a disagreement about a matter of fact, or about values, or simply as concerned with personal preference.
My own preferences in this matter help to explain why I attempt {\it utopian philosophy}.
I shall describe them for the sake of the light they may cast on just what I suppose {\it utopian philosophy} to be and why I think it of interest (and utility).

I share with many others a leaning toward mathematical (and other kinds of) abstraction, which leads me from time to time to consider problems in mathematics, mathematical logic, and philosophy. 
I am, however, wholly incapable of sustained application to any enterprise the utility of which I cannot comprehend.
I have an ample imagination, and find no difficulty in persuading myself of the importance of many problems which others might think academic.
But I cannot sustain an interest where my imagination fails in its search for motivating utility.

Not all arguments against evaluating academic research in term of utility promote the idea that all knowledge is good.
Hardy was not arguing that all mathematics should be regarded as equally good regardless of utility.
He argued the aesthetic merit of mathematics, the best mathematics is not more useful, it is more beautiful.

\subsection{Utopia}

The word {\it utopian} is used by contrast with {\it realistic}.
It is also sometimes associated with very detailed descriptions of some utopian society.
Karl Popper has connected utopian engineering with totalitarianism.

The utopianism I consider here is intended in none of these ways, so I would like to begin by clarifying in what sense and in what way my enterprise is utopian.

First of all, I certainly intend that this work be well connected with reality.
However, it is a kind of speculative philosophy which though well-intentioned undoubtedly takes risks.
I intend the work to be realistic, but cannot realistically expect that everyone will agree with me that it is.

My purpose is utopian, insofar as I seek the very best that may be had (for us all), but I don't pretend to know what that is.
Instead of aiming at some particular end point I believe we should consider carefully what is good, and what is bad, about the world we live in, promote the former and deplore the latter.
This sounds perhaps like what Popper has called ``piecemeal engineering'', in contrast to utopian engineering.
I am however keen to separate the purpose and the method.

An important (possibly the most important) aspect of these utopian investigations is a concern with the relationship between individuals and global issues.
Most of us feel powerless to influence very many of the important things in our life, which may appear altogether indifferent to human action (like the weather) or may be influenced only by people with whom we have no contact and over whom we can exert no influence.
Nevertheless, global phenomena do affect us as individuals, probably a great deal more than we comprehend.
And the influence that humanity can exert is mediated by the actions of individuals, though very often actions which the individual does not consider voluntary, or (even more often) of which he does not understand the consequences.

The relationship between the choices and actions of individuals and the consequent global effects, and that between global phenomena and individual lives, are the most distant in a wide variety of interrelationships between phenomena of various degrees of locality and social groups of different size and complexity.
We cannot hope to understand the extremes without some view of the intermediate relationships.
In the economic domain Adam Smith's ``invisible hand'' may be supposed to ensure that attention to detail suffices, the larger issues then solving themselves without our further consideration.
A similar thesis extended to social change in general might be that continual small improvements to those things we can understand and influence will suffice in the fullness of time to address the larger problems which we feel powerless to act directly upon.
Utopian philosophy is a product of the view that getting the big picture right depends upon addressing the big problems, they may not melt away when we ignore them.

\subsection{Popper's Critique of Utopian Engineering}

``Utopian Engineering'' is a phrase coined by Karl Popper \cite{popperOSE2}.
Popper grew up in a Europe ravaged by ``utopian'' enterprises turned totalitarian, and considered totalitarian regimes the inevitable result of utopian experiments.
Alongside utopian ideals Popper criticised ``historicism'' (another Popperian neologism \cite{popperPOH}).
By ``historicism'' Popper meant the belief in historical destiny.
He denied that there can be any prediction of the course of human history, by scientific or any other rational methods.
Popper claimed to have shown that ``for strictly logical reasons, it is impossible for us to predict the future course of history''.

Forecasting the course of history is certainly difficult.
More difficult the more detail is attempted, and the more distant the future predicted.
Nevertheless, many decisions depend upon a judgement about the effects of present actions on future well-being.
These decisions we must make as best we can.
Making them well may depend upon making reasonable and well informed judgements about the future course of history.