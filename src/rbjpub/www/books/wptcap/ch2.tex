% $Id: ch2.tex,v 1.1 2010/03/12 13:33:13 rbj Exp $
\chapter{The Achievement of the Tractatus}

\section{Unquestioned Legacy}

The following doctrines are accepted by Wittgenstein, and form that part of his legacy which he does not question in the Tractatus: 
\begin{enumerate}
\item metaphysical realism
\item Augustine's picture of language
\item denotational semantics
\item the demand for determinacy of sense
\item anti-psychologism
\item the topic neutrality of logical operators
\end{enumerate}


The main achievements of the Tractatus (viewed from afar) are:
\begin{itemize}
\item[(i)] its criticisms of Frege and Russell
\item[(ii)] its metaphysical picture of the relationship of thought, language and reality
\item[(iii)] its account of the nature of the propositions of logic
\item[(iv)] its critique of metaphysics and its conception of future philosophy as analysis
\end{itemize}

Each of these achievements was marred by error which Wittgenstein himself would later point out.


\section{Criticisms of Frege and Russell}

\begin{itemize}
\item[(i)] Russell's multiple relation theory of judgement
\item[(ii)] misconstrued relation between language and logic

He means here, between natural languages and formal logics.
Frege \& Russell regarded natural languages as deficient and formal logic as idealised languages in which these deficiencies are removed.
For Wittgenstein: ``all the propositions of everyday language, just as they stand, are in perfect logical order'' \cite{wittgenstein1921} 5.5563.

\item[(iii)] Russell and Frege held that expressions such as `object', `concept', `relation'... are names.

According to Wittgenstein: ``The propositions of logic contain only apparent variables, and, whatever may turn out to be the proper explanation of apparent variables, its consequence must be that there are no logical constants.''

\item[(iv)] Generalisations about forms are not propositions of logic ({\it contra} Russell)
\item[(v)] Russell and Frege regarded propositions as names (of facts and truth values respectively)

Wittgenstein disagreed on the ground that it makes no sense to negate a name.
\item[(vi)] Russell and Frege considered the logical connectives as the names of logical entities.

But it is a dire error to think that `p $\vee$ q' has the same form as `aRb'.
\item[(vii)] Russell's view of philosophy was `a retrogression from the method of physics'.

It is inconceivable that philosophy should share the methods of the natural sciences.
\end{itemize}

\section{The metaphysical picture of the relation of thought language and reality}




\section{The positive account of the propositions of logic}

\section{The critique of metaphysics and the conception of future philosophy as analysis}

Propositions are truth functional combinations of atomic propositions.
In natural languages is may not be clear as a result of confusing features of surface grammar what is the logical structure of the proposition expressed.

The purpose of philosophy is not to propound new knowledge in the form of true propositions, but to analyse apparently problematic sentence uncovering their true logical structure and thereby dispelling any confusions which may have arisen.

In six respects Wittgenstein inaugurated the ``linguistic turn'' in analytic philosophy:

\begin{description}
\item[first] in setting limits to thought by setting limits to language it put language at the centre of philosophical investigations
\item[secondly] by its commitment to logico-linguistic analysis of synthetic propositions
\item[thirdly] by advocating the exposure of metaphysics as nonsense
\item[fourthly] by his clarification of the essential nature of the propositional {\it sign}
\item[fifthly] investigation of phenomena is to be by logical analysis of linguistic descriptions of the phenomena
\item[sixthly] the `peculiar mark of logical propositions [is] that one can recognise that they are true from the symbol alone, and this fact contains in itself the whole philosophy of logic.'
\end{description}

