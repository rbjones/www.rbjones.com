% $Id: ch1.tex,v 1.1 2010/03/12 13:33:13 rbj Exp $
\chapter{The Background}
\section{The Origins of Analytic Philosophy}

Analysis is taken to mean the decomposition of something into its constituents.

\index{analysis!kinds of|(}

Kinds of philosophical analysis:

\begin{description}
\item[psychological]\ 

(as practiced by the classical British empiricists)
in which complex ideas are analysed into their constituents.
\item[logico-analytic]\ 

(Russelian) using the ``new logic'' as an instrument for the {\it logical} analysis of objective phenomena.
\item[conceptual]\ 

(Moorean) concerned with the analysis of (mind-independent) concepts rather than `ideas' or `impressions'.
\item[logical atomism]\ 

\item[Cambridge analysis]\ 

of the interwar years

\item[Logical Positivism]\

\item[Connective Analysis]\

\rbjnote{This is Hackers preferred term (originating with Strawson) for something like ``linguistic philosophy''.
Connective analysis, he says, is exemplified in various forms in post-2nd-war Oxford.

In this method the term {\it analysis} has lost much of its force, since it involves descriptions of usage and there are no analyses of complexes into simple constituents. 
}%rbjnote
\item[Therapeutic Analysis]

\end{description}

\index{Moore}
Talking of the unclear nature of Moorean analysis Hacker says:
\begin{quotation}
What is clear is that analysis was not conceived to be of language, but of something objective which is signified by expressions.
\end{quotation}

\rbjnote{
This puzzles me somewhat since:
\begin{enumerate}
\item It seems to me that Moore is quite clear in ``A defence of Common Sense'' that he defends the truth of commonsense propositions but regards philosophers as having something to contribute to the analysis of these propositions.
\item The analysis of ``something objective which is signified by expressions'' seems to me a reasonable kind of linguistic analysis.
\end{enumerate}

It appears from Hacker's discussion that it is the lack of discussion of {\bf usage} which makes him consider this not as linguistic analysis.
Or perhaps it is simply that Moore considers analysis as going beyond a mere understanding of {\bf meaning}, so that one can understand a sentence perfectly well and yet not know its proper analysis.
Moore not only defends common sense belief as true, but ordinary understanding as properly grasping meaning, but thinks that there is nevertheless some worthwhile discovery to be made by philosophical analysis.
To construe this still as having something to do with linguistic analysis one might consider how a linguist might attempt a formalisation of the semantics of a natural language, and count himself to be doing something original and revealing, without holding that ordinary users of the language who have no knowledge of his formal semantics have a deficient understanding of the language they are using.
Different kinds of semantic analysis may serve different purposes.
}%rbjnote

\index{analysis!kinds of|)}

\index{Russell}
\index{Bradley}
Russell comes to analysis initially from a rejection of absolute idealism, and most importantly his rejection of Bradley's doctrine of relations.
Bradley's doctrine of relations held that all relations are {\it internal}, i.e. as essential properties of their relata.
Russell saw this doctrine as responsible for five fallacies of absolute idealism, viz:
\begin{enumerate}
\item monism
\item the coherence theory of truth
\item the doctrine of concrete universals
\item the ideality of spirituality of the real
\item the internal relation between mind and the objects of knowledge
\end{enumerate}

Russell believed that this made mathematics impossible, thinking that the asymmetric relations essential to mathematics could not be reduced to properties of their relata.

\rbjnote{
This is in fact incorrect, as is shown by the development of mathematics using Church's Simple Theory of Types \cite{churchSTT}.
However, the technique (now known as ``currying'' after the logician H.B.Curry) for doing this does not appear to have been noted until Schonfinkel's paper on combinatory logic \cite{schonfinkelCL} was published in 1924 (though the necessary machinery, higher order functions, is already present in Frege's Begriffsschrift \cite{frege1879}).
}%rbjnote

Both Moore and Russell replaced absolute idealism with ``unbridled platonic realism''.


\section{The Problem Setting Context of the {\it Tractatus}}

\begin{quotation}
Frege's great achievement as a logician had been to give a complete formalisation of the first-order predicate calculus with identity.
\end{quotation}

\rbjnote{
Sorry, I have to object to this, which (alongside treating his later work as second order logic) credits Frege with a knowledge which was only obtained after the set theoretic paradoxes had come to prominence and quite a bit of work had been done to resolve them.
Frege's Begriffsschrift is closer to a type-free ``higher order'' logic (which today sounds like nonsense).
Frege is himself quite explicit in section 11 on {\bf Generality} that quantification is permitted over functions (and for Frege a predicate is a propositional function).
He makes use of this in the theory of sequences in part III of his monograph in which quantification over predicates is common (see section 26 \cite{frege1879}, for an example). 
}

There was little understanding by Frege and Russell of:

\begin{itemize}
\item[(i)] logical truths
\item[(ii)] logical necessity
\item[(iii)] the meaning or content of logical truths
\item[(iv)] the relations between propositions of logic
\item[(v)] the status and epistemic grounds of primitive axioms
\item[(vi)] the nature of logical connectives
\item[(vii)] the relationship between logical propositions and normative ``laws of thought''
\end{itemize}

These were the issues which Wittgenstein addressed in the second decade of the century.



