\chapter*{Preface}
\addcontentsline{toc}{chapter}{Preface}

These notes are prepared for my own purposes while studying this volume, which I have done in the hope that it will mitigate my lack of that historical scholarship which might normally be thought prerequisite for the writing which I am attempting on my own account.

I should perhaps mention that I am, within the spectrum of varieties of philosophical analysis, relative to Hacker, at an opposite pole.
I am largely in sympathy with Russell and his attitude to the later philosophy of Wittgenstein, and have even been somewhat sceptical of the merits of Wittgenstein's earlier work.

It is therefore gratifying to me that I have found Hacker's book to be readable, interesting, and very valuable for my enterprise.
I too often find myself quite incapable of reading a philosopher whose views seem to me fundamentally mistaken, and am then sometimes faced with the difficulty of how to account for my dismissal of a point of view with which I am barely acquainted.
This applies with greatest force to Wittgenstein, whose influence on analytic philosophy has been second to none, but whose writings I have found difficult to take seriously.

Fortunately for me, this volume is not primarily concerned with giving an account of Wittgenstein's philosophy, which Hacker has done elsewhere, but rather with the rest of analytic philosophy and how it has been influenced by Wittgenstein.
However low my opinion might be of the value of Wittgenstein's work, I do not doubt that it has been influential, or that some grasp of that influence is indispensable to even the most superficial historical understanding of analytic philosophy.

It is a remarkable feature of this book that it treats a major philosopher, already deceased for almost half a century at the time of writing, considers the criticisms which have been brought to bear upon his philosophy, and finds none of them to be of merit.
The sense of unreality which this engenders is enhanced by the casualties incurred by most of the other philosophers discussed, whose work is often found to have been properly criticised by an argument which, however, fails for some reason to apply to Wittgenstein's apparently similar position.
The achievement seems to me the more remarkable for its smooth seamless undetectability.
I find myself having to look hard to find the points at which I disagree with Hacker, I at no point feel any inclination to throw the book in the bin, and yet at the end of the book my radically opposed viewpoint remains intact.


