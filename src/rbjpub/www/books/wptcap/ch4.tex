% $Id: ch4.tex,v 1.1 2010/03/12 13:33:13 rbj Exp $
\chapter{The Inter-war Years}

\section{Cambridge between the wars}

Brief account of Broad's philosophy.
He identified with the style of analytic philosophy pioneered by Moore and Russell but was not influenced by Wittgenstein (except perhaps indirectly through Russell).

Russell's philosophy from 1912 to the mid 1920's was influenced by Wittgenstein in the following respects:

\subsubsection{pre-war}

conversations with Wittgenstein and the typescript `Notes on Logic' (1913) resulted in Russell abandoning his {\it Theory of Knowledge}, and then assimilating many of Wittgenstein's ideas into {\it The Philosophy of Logical Atomism} \cite{russellPLA}.

He accepted from Wittgenstein:
\begin{enumerate}
\item the distinction between names and propositions
\item the reducibility of the connectives to the Scheffer stroke
\item the eliminability of logical connectives into T/F notation
\item the denial that logical operators are the names of logical objects
\item the insight that the logical form of a belief is a function of the logical form of what is believed
\item that logical propositions (and hence those of mathematics) are tautologous
\item the isomorphism between a proposition and the fact it depicts
\item the thesis of extensionality
\item the logical independence of atomic facts
\end{enumerate}

He disagreed on the following:
\begin{enumerate}
\item on what can be shown but not said (suggesting the use of a metalanguage)
\item on equations and the status of mathematics
\item on scientific methods in philosophy
\end{enumerate}

\rbjnote{
The claim that logical truths are tautological is incorrectly taken to be incompatible with the claim that they are about abstract or logical objects.
This is connected with the criticism of Russell's advocacy of scientific methods, which is taken either to conflate philosophy with the natural sciences except perhaps to the extent that it is engaged in metaphysics.
However, it is possible to hold that logic is tautologous, and that it is about abstract objects, and that it is incremental in the way that all science is (including mathematics).
Russell was influenced by Wittgenstein to abandon positions which still seem to me very reasonable.
}%rbjnote

\subsubsection{post-war} 



\section{Wittgenstein in Cambridge}

\rbjnote{
Apparently, in the development of Wittgenstein's ideas between the Tractatus and his later philosophy, a key element was the abandonment of the logical independence of atomic propositions.
Logical dependencies between atomic propositions seems to have been given the tag ``determinate exclusion''.
From the observation that there are such (and this is necessary if the semantics of any non-logical concepts are permitted to result in logical truths, e.g. if distinct colour attributions are to be incompatible) it is inferred (how is not explained by Hacker, and probably not by Wittgenstein either) that ``one cannot conjoin `A is 2 foot long' with `A is three foot long'''.
There are of course other examples of Wittgenstein's regarding necessary falsehoods as senseless.
The result of this is to make it appear that the Tractarian truth functional account of the logical connectives, and hence the whole Tractarian account of propositions, is irretrievably damaged.
So far I am ignorant of any argument to support this which is stronger than dogmatic assertion.
}%rbjnote

The problems in the logical theory of the Tractatus lead Wittgenstein on to doubts about the metaphysical aspects.
\begin{enumerate}
\item the world does not consist of facts\ 

a description of the world consists of statements of facts (viz: true statements), not an enumeration of things
\item facts are not the kind of thing envisaged in the Tractatus\ 

pointing out a fact is simply making a true statement
\item simple and complex are relative notions not absolute
\end{enumerate}

With this the picture theory fell too, and the thesis of isomorphism between language and reality, and the idea of logical form, and from here, the ``Augustinian'' model of language.

\rbjnote{
Which seems to be something like the idea that language has a denotational semantics.
}%rbjnote

Now we have lost the basis for the kind of philosophical analysis envisaged in the Tractatus and the Cambridge style analysis to which it had contributed.
`New level analysis' (reductive)  was to be replaced by `same level analysis' (paraphrastic, connective).


``Philosophy, as Wittgenstein now conceived of it, consists in the dissolution of philosophical problems.
All philosophy can do is to destroy idols (and that includes not creating new ones, such as `the absence of an idol').
One of the greatest impediments for philosophy is the expectation of new, deep, hitherto unheard of information or explanation.
Philosophy produces no new knowledge, but only grammatical elucidations - reminders of how we use words - which unravel the knotted skein of our philosophical reflections.''

\section{Oxford between the wars}


Oxford philosophy was moribund until the mid 1930s, when, after a generation gap created by the great war, a new generation of oxford philosophers began to stir under the stimulus of ideas, primarily from Cambridge and Vienna.
Wittgenstein's impact was insignificant except upon Ryle and Ayer.

The link with Cambridge came first with Price who undertook some of his postgraduate studies in Cambridge and ``introduced the idea that young Oxford could and should learn from Cambridge''.
Price brought back to Oxford a concern with sense datum theories of perception, which became a central Oxford preoccupation until it ``crumbled'' under the assault of G.A.Paul, Ryle, Wittgenstein and Austin.

The principal Oxford figures were Ryle, Ayer and Austin.
Ryle had in his early papers an `Occamising zeal', in search of:
\begin{itemize}
\item[(a)] realism without additional entities to apprehend
\item[(b)] realism without fabricated apprehendings
\end{itemize}
and considered philosophical problems to be {\it of a special sort} rather than {\it about a special kind of thing}.
Later in the 1930's he began to write about category distinctions, which became a distinctive feature of his mature post second war philosophy.

Of his early `occamising' papers the most influential was `Systematically Misleading Expressions' \cite{ryleSME}.
In this paper he took the view that statements unproblematic in ordinary discourse become misleading in a philosophical reflection, because their grammatical or syntactic form is {\it improper} to the states of affairs which they record.
It is the task of philosophy to recast these more properly.
There are four classes of these:
\begin{itemize}
\item[(i)] Quasi-ontological statements, which incorrectly appear to assert existence
\item[(ii)] Quasi-platonic statements, which misleadingly suggest the existence of universals
\item[(iii)] Descriptive and quasi-descriptive phrases which misleadingly appear to be referential
\item[(iv)] Quasi-referential phrases (e.g. `the top of the tree') which misleadingly suggest the existence of entities
\end{itemize}
This appears to have some similarities with Russell's theory of descriptions, and to provide further examples of the kind of analysis which might be expected to flow from Logical Atomism.

As the war approached Ryle moved away from this position with his 1938 paper `Categories', in which he argued that philosophical antinomies and puzzlements stem from failure to apprehend category differences between expressions.
The concept of a category mistake was to be prominent in Ryle's post war philosophy.

