% $Id: ch3.tex,v 1.1 2010/03/12 13:33:13 rbj Exp $
\chapter{The Impact of the Tractatus: The Vienna Circle}

\section{The Vienna Circle}

The Tractatus was an important influence on the philosophical position developed by the Vienna circle, which was also influence by sessions held by some members of the circle with Wittgenstein.

Some Tractarian positions were not adopted, e.g.:

\begin{itemize}
\item the picture theory of the proposition
\item the doctrine of showing and saying
\item the metaphysics of logical atomism
\end{itemize}



What was embraced included:

\begin{itemize}
\item the account of the nature and limits of philosophy
\item the conception of logic and of logical necessity
\item the ideal of logical analysis of language
\end{itemize}

\vspace{2in}

\section{Philosophy, analysis, and the scientific world view}

\section{The demolition of metaphysics}

\section{Necessary propositions, conventionalism, and consistent empiricism}

\section{The principle of verification}

\section{The unity of science}

Logical positivism collapsed for the following reasons:

\begin{itemize}
\item[(i)] disagreement about the reductivist base, and failure to carry through a convincing reduction to any of the alternatives

\rbjnote{
Carnap, as indicated in his {\bf principle of tolerance} was {\bf pluralistic} about the reductive base.
}%rbjnote


\item[(ii)] reductivism entailed {\it either} methodological solipsism {\it or} radical behaviourism neither of which proved acceptable

\rbjnote{
Can't see why this should be the case for a physicalist reductionism.
This does abolish irreducibly mental language, but reduction to the physical is not at all the same as reduction to behaviour (i.e. a mind/brain identity thesis need not entail any kind of behaviourism),
}%rbjnote

\item[(iii)] the thesis of extensionality proved difficult to defend.
\item[(iv)] neither the principle of verification nor verifiability as a criterion of meaningfulness were capable of watertight formulation

\rbjnote{
Though there abandonment leaves may other aspects of logical positivism intact.
}%rbjnote
 
\item[(v)] conventionalism with respect to necessary truth was shown to be inadequate

\rbjnote{
A weak form of conventionalism is to state that necessary truths are true solely in virtue of the meaning of the language in which they are expressed.
If ``solely in virtue of'' is simply intended here to mean ``without need of any empirical evidence'' then I myself continue to defend this hypothesis.
However, if conventionalism is interpreted as entailing that the laws of logic can be explained as conventional, so that ``solely in virtue of'' is intended to mean ``without making use of any logic'' then there is a problem in the necessary machinery for applying the conventions.
If we consider formal languages then Frege's formula ``Mathematics = Logic + definitions'' is helpful, in that it suggests that analytic  statements are reducible to logical statements (provided one has a logic in which definitions are possible, i.e. which provides an adequate abstract ontology).
A conventionalist even of the weak variety would still want to say that the truths of the pure logic were also true as a result of the choices we made in formulating the language, but deny that any account of the conventions can provide an absolute terminus in the infinite regress which can be expected when we attempt the anything. 
}%rbjnote

\item[(vi)] problems in the use of classical logic for the analysis of ordinary language

\rbjnote{
But how much did this matter?
Were they actually trying to do this?
Were they not concerned with scientific language, and urging the use of formal notations in preference to natural languages?
}%rbjnote

\item[(vii)] the unity of science was problematic, both in terms of language and in terms of method
\item[(viii)] the conception of philosophy and of analysis was too narrow
\item[(ix)] the equation of `the language of science' with the whole domain of empirical knowledge was, at best, misleading
\item[(x)] analysis as translation or as reduction ``proves to be far to restrictive for purposes of philosophical clarification
\end{itemize}
