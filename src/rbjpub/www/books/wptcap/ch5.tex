% $Id: ch5.tex,v 1.1 2010/03/12 13:33:13 rbj Exp $
\chapter{The Achievement of the {\it Investigations}}

\section{The hedgehog and the fox}

\index{Isiah Berlin}
\index{Archilocus}
Isiah Berlin, in \cite{berlinHF}, quoted the Greek poet Archilocus:
\begin{quotation}
\noindent
The fox knows many things, but the hedgehog knows one big thing.
\end{quotation}
using this to distinguish different kinds of philosophical thought.

In these terms Wittgenstein began with his Tractatus as a hedgehog, and having exhausted this mode of thought was the better equipped to show its illusory nature.
In his later re-incarnation as a fox, Wittgenstein:
\begin{quotation}
\noindent
described the sources of error, the roots of philosophical confusion, the temptations of metaphysical illusion, the irresistible allure of apparently sublime logical insight, with a depth of understanding and richness of detail unparalleled in the history of the subject.
\end{quotation}
in the {\it Philosophical Investigations}.
Frank Ramsey observed \cite{ramseyU} that in long standing philosophical disputes:
\begin{quotation}
\noindent
it is a heuristic maxim that the truth lies not in one of the two disputed views but in some third possibility which has not yet been thought of, which we can only discover by rejecting something assumed as obvious by both the disputants.
\end{quotation}

\noindent
This was done by Wittgenstein many times over:
\begin{itemize}
\item[(a)] language - undercuts realist/idealist divide over what linguistic expressions denote by giving accounts of meaning exclusively in terms of usage
\item[(b)] logic - instead of investigating how logical propositions can be both true and vacuous Wittgenstein investigated the role of such propositions in our linguistic transactions
\item[(c)] metaphysics - true metaphysical statements are rules for the use of expressions in misleading guise, not descriptions of any kind of reality
\item[(d)] epistemology - in the normal use of statements about appearances {\it there is no such thing as not knowing} and hence no such thing as knowing, and hence they cannot constitute evidence {\it contra} foundationalism
\item[(e)] induction - the very demand for justification is senseless
\item[(f)] psychology - 1st person psychological propositions are avowals, manifestations or expressions of experiences not assertions, third person psychological propositions are descriptions for which there are public behavioural criteria. 

\end{itemize}


\section{The repudiation of analysis}

Wittgenstein repudiated reductive and constructive analysis in favour of connective and therapeutic analysis.

\section{The nature of philosophy}

\section{Metaphysics}

\section{Philosophy of language and the unity of the {\it Investigations}}

\section{Philosophical psychology}
